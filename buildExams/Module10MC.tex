\documentclass[14pt]{article}
%General Packages
\usepackage{multicol, enumerate, enumitem, hyperref, color, soul, setspace, parskip, fancyhdr}

%Math Packages
\usepackage{amssymb, amsthm, amsmath, bbm, latexsym, units, mathtools}

%All math in Display Style
\everymath{\displaystyle}

% Packages with additional options
%\usepackage[T1]{fontenc}
\usepackage[headsep=0.5cm,headheight=0cm, left=1 in,right= 1 in,top= 1 in,bottom= 1 in]{geometry}
\usepackage[usenames,dvipsnames]{xcolor}

% SageTeX
\usepackage{sagetex}

% Package to use the command below to create lines between items
\usepackage{dashrule}
\newcommand{\litem}[1]{\item#1\hspace*{-1cm}\rule{\textwidth}{0.4pt}}

\pagestyle{fancy}
\lhead{Module\,10M\,-\,Modeling\,with\,Power\,Functions}
\chead{}
\rhead{Progress Exam 0}
\lfoot{debug}
\cfoot{}
\rfoot{Version C}

\begin{document}
\pagestyle{fancy}

\begin{sagesilent}
load("../Code/generalPurposeMethods.sage")
load("../Code/keyGeneration.sage")
keyFileName = "Module10M"
version = "C"
\end{sagesilent}

\begin{enumerate}
\setcounter{enumi}{45}


\begin{sagesilent}
moduleNumber=10
problemNumber=46
load("../Code/modelingTrack/modelingPower/constructJointModel.sage")
\end{sagesilent}

\litem{ \sage{displayStem}

\sage{displayProblem}

	\begin{enumerate}[label=\Alph*.]
  \item \( \sage{choices[0]} \)
  \item \( \sage{choices[1]} \)
  \item \( \sage{choices[2]} \)
  \item \( \sage{choices[3]} \)
  \item \( \sage{choices[4]} \)
	\end{enumerate}
}

\begin{sagesilent}
moduleNumber=10
problemNumber=47
load("../Code/modelingTrack/modelingPower/constructDirectModel.sage")
\end{sagesilent}

\litem{ \sage{displayStem}

\sage{displayProblem}

	\begin{enumerate}[label=\Alph*.]
  \item \( \sage{choices[0]} \)
  \item \( \sage{choices[1]} \)
  \item \( \sage{choices[2]} \)
  \item \( \sage{choices[3]} \)
  \item \( \sage{choices[4]} \)
	\end{enumerate}
}

\begin{sagesilent}
moduleNumber=10
problemNumber=48
load("../Code/modelingTrack/modelingPower/identifyModelVariation.sage")
\end{sagesilent}

\litem{ \sage{displayStem}

 \sage{displayProblem}

	\begin{enumerate}[label=\Alph*.]
  \item \( \sage{choices[0]} \)
  \item \( \sage{choices[1]} \)
  \item \( \sage{choices[2]} \)
  \item \( \sage{choices[3]} \)
	\end{enumerate}
}

\begin{sagesilent}
moduleNumber=10
problemNumber=49
load("../Code/modelingTrack/modelingPower/constructIndirectModel.sage")
\end{sagesilent}

\litem{ \sage{displayStem}

 \sage{displayProblem}

	\begin{enumerate}[label=\Alph*.]
  \item \( \sage{choices[0]} \)
  \item \( \sage{choices[1]} \)
  \item \( \sage{choices[2]} \)
  \item \( \sage{choices[3]} \)
  \item \( \sage{choices[4]} \)
	\end{enumerate}
}

\begin{sagesilent}
moduleNumber=10
problemNumber=50
load("../Code/modelingTrack/modelingPower/identifyModelPopulation.sage")
\end{sagesilent}

\litem{ \sage{displayStem}

\begin{tabular}{c|c|c|c|c|c|c|c|c|c}
\textbf{Year} & 1 & 2 & 3 & 4 & 5 & 6 & 7 & 8 & 9 \tabularnewline
\hline
\textbf{Pop.} & \sage{populations[0]} & \sage{populations[1]} & \sage{populations[2]} & \sage{populations[3]} & \sage{populations[4]} & \sage{populations[5]} & \sage{populations[6]} & \sage{populations[7]} & \sage{populations[8]}
\end{tabular}

	\begin{enumerate}[label=\Alph*.]
  \item \( \sage{choices[0]} \)
  \item \( \sage{choices[1]} \)
  \item \( \sage{choices[2]} \)
  \item \( \sage{choices[3]} \)
	\end{enumerate}
}

\end{enumerate}

\end{document}

