\documentclass[14pt]{article}
%General Packages
\usepackage{multicol, enumerate, enumitem, hyperref, color, soul, setspace, parskip, fancyhdr}

%Math Packages
\usepackage{amssymb, amsthm, amsmath, bbm, latexsym, units, mathtools}

%All math in Display Style
\everymath{\displaystyle}

% Packages with additional options
%\usepackage[T1]{fontenc}
\usepackage[headsep=0.5cm,headheight=0cm, left=1 in,right= 1 in,top= 1 in,bottom= 1 in]{geometry}
\usepackage[usenames,dvipsnames]{xcolor}

% SageTeX
\usepackage{sagetex}

% Package to use the command below to create lines between items
\usepackage{dashrule}
\newcommand{\litem}[1]{\item#1\hspace*{-1cm}\rule{\textwidth}{0.4pt}}

\pagestyle{fancy}
\lhead{Module\,8\,-\,Logarithmic\,and\,Exponential\,Functions}
\chead{}
\rhead{Progress Exam 3}
\lfoot{Summer\,C\,2020}
\cfoot{}
\rfoot{Version C}

\begin{document}
\pagestyle{fancy}

\begin{sagesilent}
load("../Code/generalPurposeMethods.sage")
load("../Code/keyGeneration.sage")
keyFileName = "Module8"
version = "C"
\end{sagesilent}

\begin{enumerate}
\setcounter{enumi}{35}


\begin{sagesilent}
moduleNumber=8
problemNumber = 36
load("../Code/logExp/solveExpDifferentBases.sage")
\end{sagesilent}

\litem{
\sage{displayStem}

\[ \sage{displayProblem} \]

\begin{enumerate}[label=\Alph*.]
\item \( \sage{choices[0]} \)
\item \( \sage{choices[1]} \)
\item \( \sage{choices[2]} \)
\item \( \sage{choices[3]} \)
\item \( \sage{choices[4]} \)
\end{enumerate} }


\begin{sagesilent}
moduleNumber=8
problemNumber=37
load("../Code/logExp/solveByLogProperties.sage")
\end{sagesilent}

\litem{ \sage{displayStem}
	\[ \sage{displayProblem} \]
	\begin{enumerate}[label=\Alph*.]
  \item \( \sage{choices[0]} \)
  \item \( \sage{choices[1]} \)
  \item \( \sage{choices[2]} \)
  \item \( \sage{choices[3]} \)
  \item \( \sage{choices[4]} \)
	\end{enumerate}
}

\begin{sagesilent}
moduleNumber=8
problemNumber=38
load("../Code/logExp/domainRangeLog.sage")
\end{sagesilent}

% TYPE 1 - Describe the domain/range of Logarithmic functions.
\litem{ \sage{displayStem}
\[ \sage{displayProblem} \]
	\begin{enumerate}[label=\Alph*.]
		\item \( \sage{choices[0]} \)
		\item \( \sage{choices[1]} \)
		\item \( \sage{choices[2]} \)
		\item \( \sage{choices[3]} \)
		\item \( \sage{choices[4]} \)
	\end{enumerate}
}

\begin{sagesilent}
moduleNumber=8
problemNumber=39
load("../Code/logExp/solveByConverting.sage")
\end{sagesilent}
\litem{ \sage{displayStem}
\[ \sage{displayProblem} \]
	\begin{enumerate}[label=\Alph*.]
  \item \( \sage{choices[0]} \)
  \item \( \sage{choices[1]} \)
  \item \( \sage{choices[2]} \)
  \item \( \sage{choices[3]} \)
  \item \( \sage{choices[4]} \)
	\end{enumerate}
}

\begin{sagesilent}
moduleNumber=8
problemNumber=40
load("../Code/logExp/domainRangeExp.sage")
\end{sagesilent}

% TYPE 2 - Describe the domain/range of Exponential functions.
\litem{ \sage{displayStem}
\[ \sage{displayProblem} \]
	\begin{enumerate}[label=\Alph*.]
  \item \( \sage{choices[0]} \)
  \item \( \sage{choices[1]} \)
  \item \( \sage{choices[2]} \)
  \item \( \sage{choices[3]} \)
  \item \( \sage{choices[4]} \)
	\end{enumerate}
}

\end{enumerate}

\end{document}

