\documentclass{extbook}[14pt]
\usepackage{multicol, enumerate, enumitem, hyperref, color, soul, setspace, parskip, fancyhdr, amssymb, amsthm, amsmath, latexsym, units, mathtools}
\everymath{\displaystyle}
\usepackage[headsep=0.5cm,headheight=0cm, left=1 in,right= 1 in,top= 1 in,bottom= 1 in]{geometry}
\usepackage{dashrule}  % Package to use the command below to create lines between items
\newcommand{\litem}[1]{\item #1

\rule{\textwidth}{0.4pt}}
\pagestyle{fancy}
\lhead{}
\chead{Answer Key for test copies Version A}
\rhead{}
\lfoot{5370-9939}
\cfoot{}
\rfoot{test}
\begin{document}
\textbf{This key should allow you to understand why you choose the option you did (beyond just getting a question right or wrong). \href{https://xronos.clas.ufl.edu/mac1105spring2020/courseDescriptionAndMisc/Exams/LearningFromResults}{More instructions on how to use this key can be found here}.}

\textbf{If you have a suggestion to make the keys better, \href{https://forms.gle/CZkbZmPbC9XALEE88}{please fill out the short survey here}.}

\textit{Note: This key is auto-generated and may contain issues and/or errors. The keys are reviewed after each exam to ensure grading is done accurately. If there are issues (like duplicate options), they are noted in the offline gradebook. The keys are a work-in-progress to give students as many resources to improve as possible.}

\rule{\textwidth}{0.4pt}

\begin{enumerate}\litem{
Simplify the expression below into the form $a+bi$. Then, choose the intervals that $a$ and $b$ belong to.
\[ (-5 + 8 i)(-2 + 7 i) \]The solution is \( -46 - 51 i \), which is option B.\begin{enumerate}[label=\Alph*.]
\item \( a \in [58, 68] \text{ and } b \in [19, 21] \)

 $66 + 19 i$, which corresponds to adding a minus sign in the second term.
\item \( a \in [-49, -43] \text{ and } b \in [-58, -48] \)

* $-46 - 51 i$, which is the correct option.
\item \( a \in [58, 68] \text{ and } b \in [-23, -18] \)

 $66 - 19 i$, which corresponds to adding a minus sign in the first term.
\item \( a \in [-49, -43] \text{ and } b \in [46, 55] \)

 $-46 + 51 i$, which corresponds to adding a minus sign in both terms.
\item \( a \in [9, 11] \text{ and } b \in [55, 58] \)

 $10 + 56 i$, which corresponds to just multiplying the real terms to get the real part of the solution and the coefficients in the complex terms to get the complex part.
\end{enumerate}

\textbf{General Comment:} You can treat $i$ as a variable and distribute. Just remember that $i^2=-1$, so you can continue to reduce after you distribute.
}
\litem{
Simplify the expression below into the form $a+bi$. Then, choose the intervals that $a$ and $b$ belong to.
\[ \frac{-72 + 33 i}{5 + 4 i} \]The solution is \( -5.56  + 11.05 i \), which is option E.\begin{enumerate}[label=\Alph*.]
\item \( a \in [-228.5, -227] \text{ and } b \in [10, 12.5] \)

 $-228.00  + 11.05 i$, which corresponds to forgetting to multiply the conjugate by the numerator and using a plus instead of a minus in the denominator.
\item \( a \in [-12.5, -10.5] \text{ and } b \in [-3.5, -2.5] \)

 $-12.00  - 3.00 i$, which corresponds to forgetting to multiply the conjugate by the numerator and not computing the conjugate correctly.
\item \( a \in [-6, -5] \text{ and } b \in [452.5, 453.5] \)

 $-5.56  + 453.00 i$, which corresponds to forgetting to multiply the conjugate by the numerator.
\item \( a \in [-15.5, -13] \text{ and } b \in [8, 9.5] \)

 $-14.40  + 8.25 i$, which corresponds to just dividing the first term by the first term and the second by the second.
\item \( a \in [-6, -5] \text{ and } b \in [10, 12.5] \)

* $-5.56  + 11.05 i$, which is the correct option.
\end{enumerate}

\textbf{General Comment:} Multiply the numerator and denominator by the *conjugate* of the denominator, then simplify. For example, if we have $2+3i$, the conjugate is $2-3i$.
}
\litem{
Simplify the expression below into the form $a+bi$. Then, choose the intervals that $a$ and $b$ belong to.
\[ \frac{18 - 88 i}{-3 - i} \]The solution is \( 3.40  + 28.20 i \), which is option A.\begin{enumerate}[label=\Alph*.]
\item \( a \in [3, 4] \text{ and } b \in [27.5, 29.5] \)

* $3.40  + 28.20 i$, which is the correct option.
\item \( a \in [3, 4] \text{ and } b \in [281.5, 282.5] \)

 $3.40  + 282.00 i$, which corresponds to forgetting to multiply the conjugate by the numerator.
\item \( a \in [-6.5, -4.5] \text{ and } b \in [86.5, 88.5] \)

 $-6.00  + 88.00 i$, which corresponds to just dividing the first term by the first term and the second by the second.
\item \( a \in [-15.5, -14] \text{ and } b \in [24, 25] \)

 $-14.20  + 24.60 i$, which corresponds to forgetting to multiply the conjugate by the numerator and not computing the conjugate correctly.
\item \( a \in [33.5, 35.5] \text{ and } b \in [27.5, 29.5] \)

 $34.00  + 28.20 i$, which corresponds to forgetting to multiply the conjugate by the numerator and using a plus instead of a minus in the denominator.
\end{enumerate}

\textbf{General Comment:} Multiply the numerator and denominator by the *conjugate* of the denominator, then simplify. For example, if we have $2+3i$, the conjugate is $2-3i$.
}
\litem{
Simplify the expression below into the form $a+bi$. Then, choose the intervals that $a$ and $b$ belong to.
\[ (-10 + 9 i)(5 - 6 i) \]The solution is \( 4 + 105 i \), which is option E.\begin{enumerate}[label=\Alph*.]
\item \( a \in [-55, -47] \text{ and } b \in [-61, -52] \)

 $-50 - 54 i$, which corresponds to just multiplying the real terms to get the real part of the solution and the coefficients in the complex terms to get the complex part.
\item \( a \in [-107, -102] \text{ and } b \in [-16, -14] \)

 $-104 - 15 i$, which corresponds to adding a minus sign in the second term.
\item \( a \in [0, 7] \text{ and } b \in [-108, -102] \)

 $4 - 105 i$, which corresponds to adding a minus sign in both terms.
\item \( a \in [-107, -102] \text{ and } b \in [13, 18] \)

 $-104 + 15 i$, which corresponds to adding a minus sign in the first term.
\item \( a \in [0, 7] \text{ and } b \in [102, 112] \)

* $4 + 105 i$, which is the correct option.
\end{enumerate}

\textbf{General Comment:} You can treat $i$ as a variable and distribute. Just remember that $i^2=-1$, so you can continue to reduce after you distribute.
}
\end{enumerate}

\end{document}