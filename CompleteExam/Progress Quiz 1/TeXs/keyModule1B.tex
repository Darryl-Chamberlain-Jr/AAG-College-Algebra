\documentclass{extbook}[14pt]
\usepackage{multicol, enumerate, enumitem, hyperref, color, soul, setspace, parskip, fancyhdr, amssymb, amsthm, amsmath, bbm, latexsym, units, mathtools}
\everymath{\displaystyle}
\usepackage[headsep=0.5cm,headheight=0cm, left=1 in,right= 1 in,top= 1 in,bottom= 1 in]{geometry}
\usepackage{dashrule}  % Package to use the command below to create lines between items
\newcommand{\litem}[1]{\item #1

\rule{\textwidth}{0.4pt}}
\pagestyle{fancy}
\lhead{}
\chead{Answer Key for Progress Quiz 1 Version B}
\rhead{}
\lfoot{2654-6976}
\cfoot{}
\rfoot{Fall 2020}
\begin{document}
\textbf{This key should allow you to understand why you choose the option you did (beyond just getting a question right or wrong). \href{https://xronos.clas.ufl.edu/mac1105spring2020/courseDescriptionAndMisc/Exams/LearningFromResults}{More instructions on how to use this key can be found here}.}

\textbf{If you have a suggestion to make the keys better, \href{https://forms.gle/CZkbZmPbC9XALEE88}{please fill out the short survey here}.}

\textit{Note: This key is auto-generated and may contain issues and/or errors. The keys are reviewed after each exam to ensure grading is done accurately. If there are issues (like duplicate options), they are noted in the offline gradebook. The keys are a work-in-progress to give students as many resources to improve as possible.}

\rule{\textwidth}{0.4pt}

\begin{enumerate}\litem{
Simplify the expression below and choose the interval the simplification is contained within.
\[ 9 - 14 \div 15 * 19 - (16 * 11) \]
The solution is \( -184.733 \), which is option C.\begin{enumerate}[label=\Alph*.]
\item \( [-277.07, -265.07] \)

 -272.067, which corresponds to not distributing a negative correctly.
\item \( [181.95, 185.95] \)

 184.951, which corresponds to not distributing addition and subtraction correctly.
\item \( [-188.73, -180.73] \)

* -184.733, which is the correct option.
\item \( [-168.05, -156.05] \)

 -167.049, which corresponds to an Order of Operations error: not reading left-to-right for multiplication/division.
\item \( \text{None of the above} \)

 You may have gotten this by making an unanticipated error. If you got a value that is not any of the others, please let the coordinator know so they can help you figure out what happened.
\end{enumerate}

\textbf{General Comment:} While you may remember (or were taught) PEMDAS is done in order, it is actually done as P/E/MD/AS. When we are at MD or AS, we read left to right.
}
\litem{
Simplify the expression below into the form $a+bi$. Then, choose the intervals that $a$ and $b$ belong to.
\[ (-3 + 4 i)(-10 - 7 i) \]
The solution is \( 58 - 19 i \), which is option D.\begin{enumerate}[label=\Alph*.]
\item \( a \in [52, 62] \text{ and } b \in [14, 21] \)

 $58 + 19 i$, which corresponds to adding a minus sign in both terms.
\item \( a \in [2, 6] \text{ and } b \in [61, 69] \)

 $2 + 61 i$, which corresponds to adding a minus sign in the first term.
\item \( a \in [29, 33] \text{ and } b \in [-31, -24] \)

 $30 - 28 i$, which corresponds to just multiplying the real terms to get the real part of the solution and the coefficients in the complex terms to get the complex part.
\item \( a \in [52, 62] \text{ and } b \in [-21, -17] \)

* $58 - 19 i$, which is the correct option.
\item \( a \in [2, 6] \text{ and } b \in [-62, -51] \)

 $2 - 61 i$, which corresponds to adding a minus sign in the second term.
\end{enumerate}

\textbf{General Comment:} You can treat $i$ as a variable and distribute. Just remember that $i^2=-1$, so you can continue to reduce after you distribute.
}
\litem{
Choose the \textbf{smallest} set of Complex numbers that the number below belongs to.
\[ \sqrt{\frac{-1210}{11}}+\sqrt{85} \]
The solution is \( \text{Nonreal Complex} \), which is option C.\begin{enumerate}[label=\Alph*.]
\item \( \text{Pure Imaginary} \)

This is a Complex number $(a+bi)$ that \textbf{only} has an imaginary part like $2i$.
\item \( \text{Irrational} \)

These cannot be written as a fraction of Integers. Remember: $\pi$ is not an Integer!
\item \( \text{Nonreal Complex} \)

* This is the correct option!
\item \( \text{Rational} \)

These are numbers that can be written as fraction of Integers (e.g., -2/3 + 5)
\item \( \text{Not a Complex Number} \)

This is not a number. The only non-Complex number we know is dividing by 0 as this is not a number!
\end{enumerate}

\textbf{General Comment:} Be sure to simplify $i^2 = -1$. This may remove the imaginary portion for your number. If you are having trouble, you may want to look at the \textit{Subgroups of the Real Numbers} section.
}
\litem{
Choose the \textbf{smallest} set of Real numbers that the number below belongs to.
\[ \sqrt{\frac{12100}{484}} \]
The solution is \( \text{Whole} \), which is option E.\begin{enumerate}[label=\Alph*.]
\item \( \text{Not a Real number} \)

These are Nonreal Complex numbers \textbf{OR} things that are not numbers (e.g., dividing by 0).
\item \( \text{Rational} \)

These are numbers that can be written as fraction of Integers (e.g., -2/3)
\item \( \text{Integer} \)

These are the negative and positive counting numbers (..., -3, -2, -1, 0, 1, 2, 3, ...)
\item \( \text{Irrational} \)

These cannot be written as a fraction of Integers.
\item \( \text{Whole} \)

* This is the correct option!
\end{enumerate}

\textbf{General Comment:} First, you \textbf{NEED} to simplify the expression. This question simplifies to $110$. 
 
 Be sure you look at the simplified fraction and not just the decimal expansion. Numbers such as 13, 17, and 19 provide \textbf{long but repeating/terminating decimal expansions!} 
 
 The only ways to *not* be a Real number are: dividing by 0 or taking the square root of a negative number. 
 
 Irrational numbers are more than just square root of 3: adding or subtracting values from square root of 3 is also irrational.
}
\litem{
Simplify the expression below into the form $a+bi$. Then, choose the intervals that $a$ and $b$ belong to.
\[ \frac{63 - 44 i}{-5 - 8 i} \]
The solution is \( 0.42  + 8.13 i \), which is option E.\begin{enumerate}[label=\Alph*.]
\item \( a \in [36.5, 38.5] \text{ and } b \in [8, 9.5] \)

 $37.00  + 8.13 i$, which corresponds to forgetting to multiply the conjugate by the numerator and using a plus instead of a minus in the denominator.
\item \( a \in [-13.5, -12] \text{ and } b \in [4, 6.5] \)

 $-12.60  + 5.50 i$, which corresponds to just dividing the first term by the first term and the second by the second.
\item \( a \in [-8, -5.5] \text{ and } b \in [-4, -1.5] \)

 $-7.49  - 3.19 i$, which corresponds to forgetting to multiply the conjugate by the numerator and not computing the conjugate correctly.
\item \( a \in [-0.5, 1] \text{ and } b \in [722.5, 724.5] \)

 $0.42  + 724.00 i$, which corresponds to forgetting to multiply the conjugate by the numerator.
\item \( a \in [-0.5, 1] \text{ and } b \in [8, 9.5] \)

* $0.42  + 8.13 i$, which is the correct option.
\end{enumerate}

\textbf{General Comment:} Multiply the numerator and denominator by the *conjugate* of the denominator, then simplify. For example, if we have $2+3i$, the conjugate is $2-3i$.
}
\litem{
Simplify the expression below and choose the interval the simplification is contained within.
\[ 10 - 15^2 + 14 \div 4 * 8 \div 16 \]
The solution is \( -213.250 \), which is option A.\begin{enumerate}[label=\Alph*.]
\item \( [-213.35, -211.36] \)

* -213.250, this is the correct option
\item \( [234.26, 235.58] \)

 235.027, which corresponds to two Order of Operations errors.
\item \( [-217.03, -214.29] \)

 -214.973, which corresponds to an Order of Operations error: not reading left-to-right for multiplication/division.
\item \( [236.03, 238.16] \)

 236.750, which corresponds to an Order of Operations error: multiplying by negative before squaring. For example: $(-3)^2 \neq -3^2$
\item \( \text{None of the above} \)

 You may have gotten this by making an unanticipated error. If you got a value that is not any of the others, please let the coordinator know so they can help you figure out what happened.
\end{enumerate}

\textbf{General Comment:} While you may remember (or were taught) PEMDAS is done in order, it is actually done as P/E/MD/AS. When we are at MD or AS, we read left to right.
}
\litem{
Choose the \textbf{smallest} set of Complex numbers that the number below belongs to.
\[ \frac{0}{18 \pi}+\sqrt{5}i \]
The solution is \( \text{Pure Imaginary} \), which is option A.\begin{enumerate}[label=\Alph*.]
\item \( \text{Pure Imaginary} \)

* This is the correct option!
\item \( \text{Irrational} \)

These cannot be written as a fraction of Integers. Remember: $\pi$ is not an Integer!
\item \( \text{Rational} \)

These are numbers that can be written as fraction of Integers (e.g., -2/3 + 5)
\item \( \text{Not a Complex Number} \)

This is not a number. The only non-Complex number we know is dividing by 0 as this is not a number!
\item \( \text{Nonreal Complex} \)

This is a Complex number $(a+bi)$ that is not Real (has $i$ as part of the number).
\end{enumerate}

\textbf{General Comment:} Be sure to simplify $i^2 = -1$. This may remove the imaginary portion for your number. If you are having trouble, you may want to look at the \textit{Subgroups of the Real Numbers} section.
}
\litem{
Choose the \textbf{smallest} set of Real numbers that the number below belongs to.
\[ -\sqrt{\frac{2805}{11}} \]
The solution is \( \text{Irrational} \), which is option B.\begin{enumerate}[label=\Alph*.]
\item \( \text{Integer} \)

These are the negative and positive counting numbers (..., -3, -2, -1, 0, 1, 2, 3, ...)
\item \( \text{Irrational} \)

* This is the correct option!
\item \( \text{Rational} \)

These are numbers that can be written as fraction of Integers (e.g., -2/3)
\item \( \text{Not a Real number} \)

These are Nonreal Complex numbers \textbf{OR} things that are not numbers (e.g., dividing by 0).
\item \( \text{Whole} \)

These are the counting numbers with 0 (0, 1, 2, 3, ...)
\end{enumerate}

\textbf{General Comment:} First, you \textbf{NEED} to simplify the expression. This question simplifies to $-\sqrt{255}$. 
 
 Be sure you look at the simplified fraction and not just the decimal expansion. Numbers such as 13, 17, and 19 provide \textbf{long but repeating/terminating decimal expansions!} 
 
 The only ways to *not* be a Real number are: dividing by 0 or taking the square root of a negative number. 
 
 Irrational numbers are more than just square root of 3: adding or subtracting values from square root of 3 is also irrational.
}
\litem{
Simplify the expression below into the form $a+bi$. Then, choose the intervals that $a$ and $b$ belong to.
\[ (-10 - 2 i)(7 - 9 i) \]
The solution is \( -88 + 76 i \), which is option A.\begin{enumerate}[label=\Alph*.]
\item \( a \in [-88, -80] \text{ and } b \in [69, 82] \)

* $-88 + 76 i$, which is the correct option.
\item \( a \in [-53, -49] \text{ and } b \in [103, 105] \)

 $-52 + 104 i$, which corresponds to adding a minus sign in the first term.
\item \( a \in [-53, -49] \text{ and } b \in [-105, -100] \)

 $-52 - 104 i$, which corresponds to adding a minus sign in the second term.
\item \( a \in [-70, -66] \text{ and } b \in [15, 24] \)

 $-70 + 18 i$, which corresponds to just multiplying the real terms to get the real part of the solution and the coefficients in the complex terms to get the complex part.
\item \( a \in [-88, -80] \text{ and } b \in [-85, -71] \)

 $-88 - 76 i$, which corresponds to adding a minus sign in both terms.
\end{enumerate}

\textbf{General Comment:} You can treat $i$ as a variable and distribute. Just remember that $i^2=-1$, so you can continue to reduce after you distribute.
}
\litem{
Simplify the expression below into the form $a+bi$. Then, choose the intervals that $a$ and $b$ belong to.
\[ \frac{-18 - 77 i}{-1 + 4 i} \]
The solution is \( -17.06  + 8.76 i \), which is option A.\begin{enumerate}[label=\Alph*.]
\item \( a \in [-17.5, -16.5] \text{ and } b \in [7, 10] \)

* $-17.06  + 8.76 i$, which is the correct option.
\item \( a \in [-290.5, -289.5] \text{ and } b \in [7, 10] \)

 $-290.00  + 8.76 i$, which corresponds to forgetting to multiply the conjugate by the numerator and using a plus instead of a minus in the denominator.
\item \( a \in [-17.5, -16.5] \text{ and } b \in [148.5, 149.5] \)

 $-17.06  + 149.00 i$, which corresponds to forgetting to multiply the conjugate by the numerator.
\item \( a \in [17.5, 18.5] \text{ and } b \in [-20, -19] \)

 $18.00  - 19.25 i$, which corresponds to just dividing the first term by the first term and the second by the second.
\item \( a \in [19, 20] \text{ and } b \in [-0.5, 1] \)

 $19.18  + 0.29 i$, which corresponds to forgetting to multiply the conjugate by the numerator and not computing the conjugate correctly.
\end{enumerate}

\textbf{General Comment:} Multiply the numerator and denominator by the *conjugate* of the denominator, then simplify. For example, if we have $2+3i$, the conjugate is $2-3i$.
}
\end{enumerate}

\end{document}