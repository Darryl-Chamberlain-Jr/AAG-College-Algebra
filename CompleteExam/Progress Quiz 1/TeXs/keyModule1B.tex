\documentclass{extbook}[14pt]
\usepackage{multicol, enumerate, enumitem, hyperref, color, soul, setspace, parskip, fancyhdr, amssymb, amsthm, amsmath, bbm, latexsym, units, mathtools}
\everymath{\displaystyle}
\usepackage[headsep=0.5cm,headheight=0cm, left=1 in,right= 1 in,top= 1 in,bottom= 1 in]{geometry}
\usepackage{dashrule}  % Package to use the command below to create lines between items
\newcommand{\litem}[1]{\item #1

\rule{\textwidth}{0.4pt}}
\pagestyle{fancy}
\lhead{}
\chead{Answer Key for Progress Quiz 1 Version B}
\rhead{}
\lfoot{3735-1698}
\cfoot{}
\rfoot{Spring 2021}
\begin{document}
\textbf{This key should allow you to understand why you choose the option you did (beyond just getting a question right or wrong). \href{https://xronos.clas.ufl.edu/mac1105spring2020/courseDescriptionAndMisc/Exams/LearningFromResults}{More instructions on how to use this key can be found here}.}

\textbf{If you have a suggestion to make the keys better, \href{https://forms.gle/CZkbZmPbC9XALEE88}{please fill out the short survey here}.}

\textit{Note: This key is auto-generated and may contain issues and/or errors. The keys are reviewed after each exam to ensure grading is done accurately. If there are issues (like duplicate options), they are noted in the offline gradebook. The keys are a work-in-progress to give students as many resources to improve as possible.}

\rule{\textwidth}{0.4pt}

\begin{enumerate}\litem{
Simplify the expression below into the form $a+bi$. Then, choose the intervals that $a$ and $b$ belong to.
\[ \frac{-27 - 11 i}{6 + 8 i} \]

The solution is \( -2.50  + 1.50 i \), which is option D.\begin{enumerate}[label=\Alph*.]
\item \( a \in [-5, -4] \text{ and } b \in [-2.5, 0] \)

 $-4.50  - 1.38 i$, which corresponds to just dividing the first term by the first term and the second by the second.
\item \( a \in [-1, -0.5] \text{ and } b \in [-4, -2] \)

 $-0.74  - 2.82 i$, which corresponds to forgetting to multiply the conjugate by the numerator and not computing the conjugate correctly.
\item \( a \in [-3, -1.5] \text{ and } b \in [149.5, 151] \)

 $-2.50  + 150.00 i$, which corresponds to forgetting to multiply the conjugate by the numerator.
\item \( a \in [-3, -1.5] \text{ and } b \in [1, 2] \)

* $-2.50  + 1.50 i$, which is the correct option.
\item \( a \in [-252, -249] \text{ and } b \in [1, 2] \)

 $-250.00  + 1.50 i$, which corresponds to forgetting to multiply the conjugate by the numerator and using a plus instead of a minus in the denominator.
\end{enumerate}

\textbf{General Comment:} Multiply the numerator and denominator by the *conjugate* of the denominator, then simplify. For example, if we have $2+3i$, the conjugate is $2-3i$.
}
\litem{
Choose the \textbf{smallest} set of Real numbers that the number below belongs to.
\[ \sqrt{\frac{53361}{121}} \]

The solution is \( \text{Whole} \), which is option E.\begin{enumerate}[label=\Alph*.]
\item \( \text{Integer} \)

These are the negative and positive counting numbers (..., -3, -2, -1, 0, 1, 2, 3, ...)
\item \( \text{Rational} \)

These are numbers that can be written as fraction of Integers (e.g., -2/3)
\item \( \text{Not a Real number} \)

These are Nonreal Complex numbers \textbf{OR} things that are not numbers (e.g., dividing by 0).
\item \( \text{Irrational} \)

These cannot be written as a fraction of Integers.
\item \( \text{Whole} \)

* This is the correct option!
\end{enumerate}

\textbf{General Comment:} First, you \textbf{NEED} to simplify the expression. This question simplifies to $231$. 
 
 Be sure you look at the simplified fraction and not just the decimal expansion. Numbers such as 13, 17, and 19 provide \textbf{long but repeating/terminating decimal expansions!} 
 
 The only ways to *not* be a Real number are: dividing by 0 or taking the square root of a negative number. 
 
 Irrational numbers are more than just square root of 3: adding or subtracting values from square root of 3 is also irrational.
}
\litem{
Choose the \textbf{smallest} set of Complex numbers that the number below belongs to.
\[ \sqrt{\frac{-1092}{6}} i+\sqrt{238}i \]

The solution is \( \text{Nonreal Complex} \), which is option B.\begin{enumerate}[label=\Alph*.]
\item \( \text{Rational} \)

These are numbers that can be written as fraction of Integers (e.g., -2/3 + 5)
\item \( \text{Nonreal Complex} \)

* This is the correct option!
\item \( \text{Not a Complex Number} \)

This is not a number. The only non-Complex number we know is dividing by 0 as this is not a number!
\item \( \text{Irrational} \)

These cannot be written as a fraction of Integers. Remember: $\pi$ is not an Integer!
\item \( \text{Pure Imaginary} \)

This is a Complex number $(a+bi)$ that \textbf{only} has an imaginary part like $2i$.
\end{enumerate}

\textbf{General Comment:} Be sure to simplify $i^2 = -1$. This may remove the imaginary portion for your number. If you are having trouble, you may want to look at the \textit{Subgroups of the Real Numbers} section.
}
\litem{
Simplify the expression below into the form $a+bi$. Then, choose the intervals that $a$ and $b$ belong to.
\[ (-5 + 4 i)(9 + 6 i) \]

The solution is \( -69 + 6 i \), which is option A.\begin{enumerate}[label=\Alph*.]
\item \( a \in [-69, -68] \text{ and } b \in [4, 12] \)

* $-69 + 6 i$, which is the correct option.
\item \( a \in [-46, -38] \text{ and } b \in [22, 25] \)

 $-45 + 24 i$, which corresponds to just multiplying the real terms to get the real part of the solution and the coefficients in the complex terms to get the complex part.
\item \( a \in [-69, -68] \text{ and } b \in [-10, -2] \)

 $-69 - 6 i$, which corresponds to adding a minus sign in both terms.
\item \( a \in [-24, -16] \text{ and } b \in [-67, -65] \)

 $-21 - 66 i$, which corresponds to adding a minus sign in the first term.
\item \( a \in [-24, -16] \text{ and } b \in [61, 69] \)

 $-21 + 66 i$, which corresponds to adding a minus sign in the second term.
\end{enumerate}

\textbf{General Comment:} You can treat $i$ as a variable and distribute. Just remember that $i^2=-1$, so you can continue to reduce after you distribute.
}
\litem{
Simplify the expression below into the form $a+bi$. Then, choose the intervals that $a$ and $b$ belong to.
\[ (-2 + 6 i)(9 + 7 i) \]

The solution is \( -60 + 40 i \), which is option B.\begin{enumerate}[label=\Alph*.]
\item \( a \in [20, 26] \text{ and } b \in [-70.3, -65] \)

 $24 - 68 i$, which corresponds to adding a minus sign in the first term.
\item \( a \in [-60, -58] \text{ and } b \in [37.7, 40.5] \)

* $-60 + 40 i$, which is the correct option.
\item \( a \in [20, 26] \text{ and } b \in [66, 69.5] \)

 $24 + 68 i$, which corresponds to adding a minus sign in the second term.
\item \( a \in [-60, -58] \text{ and } b \in [-41.4, -39.2] \)

 $-60 - 40 i$, which corresponds to adding a minus sign in both terms.
\item \( a \in [-18, -16] \text{ and } b \in [40.7, 44.7] \)

 $-18 + 42 i$, which corresponds to just multiplying the real terms to get the real part of the solution and the coefficients in the complex terms to get the complex part.
\end{enumerate}

\textbf{General Comment:} You can treat $i$ as a variable and distribute. Just remember that $i^2=-1$, so you can continue to reduce after you distribute.
}
\litem{
Simplify the expression below and choose the interval the simplification is contained within.
\[ 11 - 20^2 + 12 \div 5 * 14 \div 4 \]

The solution is \( -380.600 \), which is option C.\begin{enumerate}[label=\Alph*.]
\item \( [411.04, 415.04] \)

 411.043, which corresponds to two Order of Operations errors.
\item \( [-393.96, -386.96] \)

 -388.957, which corresponds to an Order of Operations error: not reading left-to-right for multiplication/division.
\item \( [-381.6, -372.6] \)

* -380.600, this is the correct option
\item \( [417.4, 423.4] \)

 419.400, which corresponds to an Order of Operations error: multiplying by negative before squaring. For example: $(-3)^2 \neq -3^2$
\item \( \text{None of the above} \)

 You may have gotten this by making an unanticipated error. If you got a value that is not any of the others, please let the coordinator know so they can help you figure out what happened.
\end{enumerate}

\textbf{General Comment:} While you may remember (or were taught) PEMDAS is done in order, it is actually done as P/E/MD/AS. When we are at MD or AS, we read left to right.
}
\litem{
Choose the \textbf{smallest} set of Complex numbers that the number below belongs to.
\[ \sqrt{\frac{-595}{7}}+\sqrt{0}i \]

The solution is \( \text{Pure Imaginary} \), which is option E.\begin{enumerate}[label=\Alph*.]
\item \( \text{Irrational} \)

These cannot be written as a fraction of Integers. Remember: $\pi$ is not an Integer!
\item \( \text{Nonreal Complex} \)

This is a Complex number $(a+bi)$ that is not Real (has $i$ as part of the number).
\item \( \text{Rational} \)

These are numbers that can be written as fraction of Integers (e.g., -2/3 + 5)
\item \( \text{Not a Complex Number} \)

This is not a number. The only non-Complex number we know is dividing by 0 as this is not a number!
\item \( \text{Pure Imaginary} \)

* This is the correct option!
\end{enumerate}

\textbf{General Comment:} Be sure to simplify $i^2 = -1$. This may remove the imaginary portion for your number. If you are having trouble, you may want to look at the \textit{Subgroups of the Real Numbers} section.
}
\litem{
Simplify the expression below and choose the interval the simplification is contained within.
\[ 19 - 4 \div 1 * 9 - (5 * 8) \]

The solution is \( -57.000 \), which is option B.\begin{enumerate}[label=\Alph*.]
\item \( [-177, -169] \)

 -176.000, which corresponds to not distributing a negative correctly.
\item \( [-62, -56] \)

* -57.000, which is the correct option.
\item \( [-24.44, -18.44] \)

 -21.444, which corresponds to an Order of Operations error: not reading left-to-right for multiplication/division.
\item \( [52.56, 61.56] \)

 58.556, which corresponds to not distributing addition and subtraction correctly.
\item \( \text{None of the above} \)

 You may have gotten this by making an unanticipated error. If you got a value that is not any of the others, please let the coordinator know so they can help you figure out what happened.
\end{enumerate}

\textbf{General Comment:} While you may remember (or were taught) PEMDAS is done in order, it is actually done as P/E/MD/AS. When we are at MD or AS, we read left to right.
}
\litem{
Choose the \textbf{smallest} set of Real numbers that the number below belongs to.
\[ -\sqrt{\frac{119025}{529}} \]

The solution is \( \text{Integer} \), which is option C.\begin{enumerate}[label=\Alph*.]
\item \( \text{Rational} \)

These are numbers that can be written as fraction of Integers (e.g., -2/3)
\item \( \text{Not a Real number} \)

These are Nonreal Complex numbers \textbf{OR} things that are not numbers (e.g., dividing by 0).
\item \( \text{Integer} \)

* This is the correct option!
\item \( \text{Whole} \)

These are the counting numbers with 0 (0, 1, 2, 3, ...)
\item \( \text{Irrational} \)

These cannot be written as a fraction of Integers.
\end{enumerate}

\textbf{General Comment:} First, you \textbf{NEED} to simplify the expression. This question simplifies to $-345$. 
 
 Be sure you look at the simplified fraction and not just the decimal expansion. Numbers such as 13, 17, and 19 provide \textbf{long but repeating/terminating decimal expansions!} 
 
 The only ways to *not* be a Real number are: dividing by 0 or taking the square root of a negative number. 
 
 Irrational numbers are more than just square root of 3: adding or subtracting values from square root of 3 is also irrational.
}
\litem{
Simplify the expression below into the form $a+bi$. Then, choose the intervals that $a$ and $b$ belong to.
\[ \frac{36 - 22 i}{5 + 6 i} \]

The solution is \( 0.79  - 5.34 i \), which is option E.\begin{enumerate}[label=\Alph*.]
\item \( a \in [3.5, 6] \text{ and } b \in [1.5, 3.5] \)

 $5.11  + 1.74 i$, which corresponds to forgetting to multiply the conjugate by the numerator and not computing the conjugate correctly.
\item \( a \in [47, 48.5] \text{ and } b \in [-7, -4] \)

 $48.00  - 5.34 i$, which corresponds to forgetting to multiply the conjugate by the numerator and using a plus instead of a minus in the denominator.
\item \( a \in [6.5, 8] \text{ and } b \in [-4.5, -3] \)

 $7.20  - 3.67 i$, which corresponds to just dividing the first term by the first term and the second by the second.
\item \( a \in [0.5, 2.5] \text{ and } b \in [-328, -325.5] \)

 $0.79  - 326.00 i$, which corresponds to forgetting to multiply the conjugate by the numerator.
\item \( a \in [0.5, 2.5] \text{ and } b \in [-7, -4] \)

* $0.79  - 5.34 i$, which is the correct option.
\end{enumerate}

\textbf{General Comment:} Multiply the numerator and denominator by the *conjugate* of the denominator, then simplify. For example, if we have $2+3i$, the conjugate is $2-3i$.
}
\end{enumerate}

\end{document}