\documentclass{extbook}[14pt]
\usepackage{multicol, enumerate, enumitem, hyperref, color, soul, setspace, parskip, fancyhdr, amssymb, amsthm, amsmath, bbm, latexsym, units, mathtools}
\everymath{\displaystyle}
\usepackage[headsep=0.5cm,headheight=0cm, left=1 in,right= 1 in,top= 1 in,bottom= 1 in]{geometry}
\usepackage{dashrule}  % Package to use the command below to create lines between items
\newcommand{\litem}[1]{\item #1

\rule{\textwidth}{0.4pt}}
\pagestyle{fancy}
\lhead{}
\chead{Answer Key for Progress Quiz 1 Version B}
\rhead{}
\lfoot{3114-1073}
\cfoot{}
\rfoot{Fall 2020}
\begin{document}
\textbf{This key should allow you to understand why you choose the option you did (beyond just getting a question right or wrong). \href{https://xronos.clas.ufl.edu/mac1105spring2020/courseDescriptionAndMisc/Exams/LearningFromResults}{More instructions on how to use this key can be found here}.}

\textbf{If you have a suggestion to make the keys better, \href{https://forms.gle/CZkbZmPbC9XALEE88}{please fill out the short survey here}.}

\textit{Note: This key is auto-generated and may contain issues and/or errors. The keys are reviewed after each exam to ensure grading is done accurately. If there are issues (like duplicate options), they are noted in the offline gradebook. The keys are a work-in-progress to give students as many resources to improve as possible.}

\rule{\textwidth}{0.4pt}

\begin{enumerate}\litem{
Simplify the expression below into the form $a+bi$. Then, choose the intervals that $a$ and $b$ belong to.
\[ (-8 - 9 i)(-4 - 7 i) \]
The solution is \( -31 + 92 i \), which is option A.\begin{enumerate}[label=\Alph*.]
\item \( a \in [-31, -25] \text{ and } b \in [89, 98] \)

* $-31 + 92 i$, which is the correct option.
\item \( a \in [92, 97] \text{ and } b \in [-26, -19] \)

 $95 - 20 i$, which corresponds to adding a minus sign in the second term.
\item \( a \in [31, 33] \text{ and } b \in [62, 66] \)

 $32 + 63 i$, which corresponds to just multiplying the real terms to get the real part of the solution and the coefficients in the complex terms to get the complex part.
\item \( a \in [-31, -25] \text{ and } b \in [-99, -91] \)

 $-31 - 92 i$, which corresponds to adding a minus sign in both terms.
\item \( a \in [92, 97] \text{ and } b \in [13, 23] \)

 $95 + 20 i$, which corresponds to adding a minus sign in the first term.
\end{enumerate}

\textbf{General Comment:} You can treat $i$ as a variable and distribute. Just remember that $i^2=-1$, so you can continue to reduce after you distribute.
}
\litem{
Simplify the expression below and choose the interval the simplification is contained within.
\[ 10 - 18^2 + 14 \div 19 * 11 \div 9 \]
The solution is \( -313.099 \), which is option C.\begin{enumerate}[label=\Alph*.]
\item \( [-314.74, -313.67] \)

 -313.993, which corresponds to an Order of Operations error: not reading left-to-right for multiplication/division.
\item \( [333.63, 334.21] \)

 334.007, which corresponds to two Order of Operations errors.
\item \( [-313.97, -311.99] \)

* -313.099, this is the correct option
\item \( [334.86, 335.48] \)

 334.901, which corresponds to an Order of Operations error: multiplying by negative before squaring. For example: $(-3)^2 \neq -3^2$
\item \( \text{None of the above} \)

 You may have gotten this by making an unanticipated error. If you got a value that is not any of the others, please let the coordinator know so they can help you figure out what happened.
\end{enumerate}

\textbf{General Comment:} While you may remember (or were taught) PEMDAS is done in order, it is actually done as P/E/MD/AS. When we are at MD or AS, we read left to right.
}
\litem{
Simplify the expression below and choose the interval the simplification is contained within.
\[ 11 - 12^2 + 20 \div 17 * 6 \div 10 \]
The solution is \( -132.294 \), which is option C.\begin{enumerate}[label=\Alph*.]
\item \( [155.64, 156.28] \)

 155.706, which corresponds to an Order of Operations error: multiplying by negative before squaring. For example: $(-3)^2 \neq -3^2$
\item \( [154.41, 155.04] \)

 155.020, which corresponds to two Order of Operations errors.
\item \( [-132.54, -131.39] \)

* -132.294, this is the correct option
\item \( [-134.12, -132.81] \)

 -132.980, which corresponds to an Order of Operations error: not reading left-to-right for multiplication/division.
\item \( \text{None of the above} \)

 You may have gotten this by making an unanticipated error. If you got a value that is not any of the others, please let the coordinator know so they can help you figure out what happened.
\end{enumerate}

\textbf{General Comment:} While you may remember (or were taught) PEMDAS is done in order, it is actually done as P/E/MD/AS. When we are at MD or AS, we read left to right.
}
\litem{
Choose the \textbf{smallest} set of Real numbers that the number below belongs to.
\[ \sqrt{\frac{27225}{121}} \]
The solution is \( \text{Whole} \), which is option D.\begin{enumerate}[label=\Alph*.]
\item \( \text{Irrational} \)

These cannot be written as a fraction of Integers.
\item \( \text{Integer} \)

These are the negative and positive counting numbers (..., -3, -2, -1, 0, 1, 2, 3, ...)
\item \( \text{Rational} \)

These are numbers that can be written as fraction of Integers (e.g., -2/3)
\item \( \text{Whole} \)

* This is the correct option!
\item \( \text{Not a Real number} \)

These are Nonreal Complex numbers \textbf{OR} things that are not numbers (e.g., dividing by 0).
\end{enumerate}

\textbf{General Comment:} First, you \textbf{NEED} to simplify the expression. This question simplifies to $165$. 
 
 Be sure you look at the simplified fraction and not just the decimal expansion. Numbers such as 13, 17, and 19 provide \textbf{long but repeating/terminating decimal expansions!} 
 
 The only ways to *not* be a Real number are: dividing by 0 or taking the square root of a negative number. 
 
 Irrational numbers are more than just square root of 3: adding or subtracting values from square root of 3 is also irrational.
}
\litem{
Choose the \textbf{smallest} set of Complex numbers that the number below belongs to.
\[ \frac{-12}{-9}+\sqrt{-36}i \]
The solution is \( \text{Rational} \), which is option C.\begin{enumerate}[label=\Alph*.]
\item \( \text{Not a Complex Number} \)

This is not a number. The only non-Complex number we know is dividing by 0 as this is not a number!
\item \( \text{Pure Imaginary} \)

This is a Complex number $(a+bi)$ that \textbf{only} has an imaginary part like $2i$.
\item \( \text{Rational} \)

* This is the correct option!
\item \( \text{Nonreal Complex} \)

This is a Complex number $(a+bi)$ that is not Real (has $i$ as part of the number).
\item \( \text{Irrational} \)

These cannot be written as a fraction of Integers. Remember: $\pi$ is not an Integer!
\end{enumerate}

\textbf{General Comment:} Be sure to simplify $i^2 = -1$. This may remove the imaginary portion for your number. If you are having trouble, you may want to look at the \textit{Subgroups of the Real Numbers} section.
}
\litem{
Simplify the expression below into the form $a+bi$. Then, choose the intervals that $a$ and $b$ belong to.
\[ \frac{-45 + 88 i}{3 + 4 i} \]
The solution is \( 8.68  + 17.76 i \), which is option A.\begin{enumerate}[label=\Alph*.]
\item \( a \in [8, 9] \text{ and } b \in [17.5, 19] \)

* $8.68  + 17.76 i$, which is the correct option.
\item \( a \in [-16, -14.5] \text{ and } b \in [21, 22.5] \)

 $-15.00  + 22.00 i$, which corresponds to just dividing the first term by the first term and the second by the second.
\item \( a \in [-21, -18.5] \text{ and } b \in [2.5, 4] \)

 $-19.48  + 3.36 i$, which corresponds to forgetting to multiply the conjugate by the numerator and not computing the conjugate correctly.
\item \( a \in [216, 217.5] \text{ and } b \in [17.5, 19] \)

 $217.00  + 17.76 i$, which corresponds to forgetting to multiply the conjugate by the numerator and using a plus instead of a minus in the denominator.
\item \( a \in [8, 9] \text{ and } b \in [443, 444.5] \)

 $8.68  + 444.00 i$, which corresponds to forgetting to multiply the conjugate by the numerator.
\end{enumerate}

\textbf{General Comment:} Multiply the numerator and denominator by the *conjugate* of the denominator, then simplify. For example, if we have $2+3i$, the conjugate is $2-3i$.
}
\litem{
Simplify the expression below into the form $a+bi$. Then, choose the intervals that $a$ and $b$ belong to.
\[ \frac{-54 + 55 i}{8 + 7 i} \]
The solution is \( -0.42  + 7.24 i \), which is option E.\begin{enumerate}[label=\Alph*.]
\item \( a \in [-47.65, -46.45] \text{ and } b \in [6.85, 7.6] \)

 $-47.00  + 7.24 i$, which corresponds to forgetting to multiply the conjugate by the numerator and using a plus instead of a minus in the denominator.
\item \( a \in [-7.3, -7.05] \text{ and } b \in [0.5, 1.25] \)

 $-7.23  + 0.55 i$, which corresponds to forgetting to multiply the conjugate by the numerator and not computing the conjugate correctly.
\item \( a \in [-7.05, -6.7] \text{ and } b \in [7.25, 8.3] \)

 $-6.75  + 7.86 i$, which corresponds to just dividing the first term by the first term and the second by the second.
\item \( a \in [-0.8, -0.15] \text{ and } b \in [817.55, 818.3] \)

 $-0.42  + 818.00 i$, which corresponds to forgetting to multiply the conjugate by the numerator.
\item \( a \in [-0.8, -0.15] \text{ and } b \in [6.85, 7.6] \)

* $-0.42  + 7.24 i$, which is the correct option.
\end{enumerate}

\textbf{General Comment:} Multiply the numerator and denominator by the *conjugate* of the denominator, then simplify. For example, if we have $2+3i$, the conjugate is $2-3i$.
}
\litem{
Choose the \textbf{smallest} set of Real numbers that the number below belongs to.
\[ -\sqrt{\frac{-2002}{14}} \]
The solution is \( \text{Not a Real number} \), which is option C.\begin{enumerate}[label=\Alph*.]
\item \( \text{Rational} \)

These are numbers that can be written as fraction of Integers (e.g., -2/3)
\item \( \text{Whole} \)

These are the counting numbers with 0 (0, 1, 2, 3, ...)
\item \( \text{Not a Real number} \)

* This is the correct option!
\item \( \text{Integer} \)

These are the negative and positive counting numbers (..., -3, -2, -1, 0, 1, 2, 3, ...)
\item \( \text{Irrational} \)

These cannot be written as a fraction of Integers.
\end{enumerate}

\textbf{General Comment:} First, you \textbf{NEED} to simplify the expression. This question simplifies to $-\sqrt{143} i$. 
 
 Be sure you look at the simplified fraction and not just the decimal expansion. Numbers such as 13, 17, and 19 provide \textbf{long but repeating/terminating decimal expansions!} 
 
 The only ways to *not* be a Real number are: dividing by 0 or taking the square root of a negative number. 
 
 Irrational numbers are more than just square root of 3: adding or subtracting values from square root of 3 is also irrational.
}
\litem{
Simplify the expression below into the form $a+bi$. Then, choose the intervals that $a$ and $b$ belong to.
\[ (6 + 7 i)(9 + 2 i) \]
The solution is \( 40 + 75 i \), which is option E.\begin{enumerate}[label=\Alph*.]
\item \( a \in [67, 77] \text{ and } b \in [-54, -46] \)

 $68 - 51 i$, which corresponds to adding a minus sign in the first term.
\item \( a \in [48, 56] \text{ and } b \in [12, 16] \)

 $54 + 14 i$, which corresponds to just multiplying the real terms to get the real part of the solution and the coefficients in the complex terms to get the complex part.
\item \( a \in [67, 77] \text{ and } b \in [51, 53] \)

 $68 + 51 i$, which corresponds to adding a minus sign in the second term.
\item \( a \in [40, 47] \text{ and } b \in [-78, -74] \)

 $40 - 75 i$, which corresponds to adding a minus sign in both terms.
\item \( a \in [40, 47] \text{ and } b \in [74, 79] \)

* $40 + 75 i$, which is the correct option.
\end{enumerate}

\textbf{General Comment:} You can treat $i$ as a variable and distribute. Just remember that $i^2=-1$, so you can continue to reduce after you distribute.
}
\litem{
Choose the \textbf{smallest} set of Complex numbers that the number below belongs to.
\[ \sqrt{\frac{1890}{15}}+\sqrt{143} i \]
The solution is \( \text{Nonreal Complex} \), which is option B.\begin{enumerate}[label=\Alph*.]
\item \( \text{Not a Complex Number} \)

This is not a number. The only non-Complex number we know is dividing by 0 as this is not a number!
\item \( \text{Nonreal Complex} \)

* This is the correct option!
\item \( \text{Pure Imaginary} \)

This is a Complex number $(a+bi)$ that \textbf{only} has an imaginary part like $2i$.
\item \( \text{Rational} \)

These are numbers that can be written as fraction of Integers (e.g., -2/3 + 5)
\item \( \text{Irrational} \)

These cannot be written as a fraction of Integers. Remember: $\pi$ is not an Integer!
\end{enumerate}

\textbf{General Comment:} Be sure to simplify $i^2 = -1$. This may remove the imaginary portion for your number. If you are having trouble, you may want to look at the \textit{Subgroups of the Real Numbers} section.
}
\end{enumerate}

\end{document}