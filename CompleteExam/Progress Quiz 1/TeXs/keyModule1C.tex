\documentclass{extbook}[14pt]
\usepackage{multicol, enumerate, enumitem, hyperref, color, soul, setspace, parskip, fancyhdr, amssymb, amsthm, amsmath, bbm, latexsym, units, mathtools}
\everymath{\displaystyle}
\usepackage[headsep=0.5cm,headheight=0cm, left=1 in,right= 1 in,top= 1 in,bottom= 1 in]{geometry}
\usepackage{dashrule}  % Package to use the command below to create lines between items
\newcommand{\litem}[1]{\item #1

\rule{\textwidth}{0.4pt}}
\pagestyle{fancy}
\lhead{}
\chead{Answer Key for Progress Quiz 1 Version C}
\rhead{}
\lfoot{3939-9803}
\cfoot{}
\rfoot{Fall 2020}
\begin{document}
\textbf{This key should allow you to understand why you choose the option you did (beyond just getting a question right or wrong). \href{https://xronos.clas.ufl.edu/mac1105spring2020/courseDescriptionAndMisc/Exams/LearningFromResults}{More instructions on how to use this key can be found here}.}

\textbf{If you have a suggestion to make the keys better, \href{https://forms.gle/CZkbZmPbC9XALEE88}{please fill out the short survey here}.}

\textit{Note: This key is auto-generated and may contain issues and/or errors. The keys are reviewed after each exam to ensure grading is done accurately. If there are issues (like duplicate options), they are noted in the offline gradebook. The keys are a work-in-progress to give students as many resources to improve as possible.}

\rule{\textwidth}{0.4pt}

\begin{enumerate}
\litem{
Simplify the expression below into the form $a+bi$. Then, choose the intervals that $a$ and $b$ belong to.
\[ \frac{18 - 44 i}{-1 - 8 i} \]
The solution is \( 5.14  + 2.89 i \), which is option B.\begin{enumerate}[label=\Alph*.]
\item \( a \in [-18.5, -17.5] \text{ and } b \in [5, 6.5] \)
 $-18.00  + 5.50 i$, which corresponds to just dividing the first term by the first term and the second by the second.
\item \( a \in [4.5, 6.5] \text{ and } b \in [1.5, 3.5] \)
* $5.14  + 2.89 i$, which is the correct option.
\item \( a \in [-6.5, -5] \text{ and } b \in [-2, -0.5] \)
 $-5.69  - 1.54 i$, which corresponds to forgetting to multiply the conjugate by the numerator and not computing the conjugate correctly.
\item \( a \in [333.5, 334.5] \text{ and } b \in [1.5, 3.5] \)
 $334.00  + 2.89 i$, which corresponds to forgetting to multiply the conjugate by the numerator and using a plus instead of a minus in the denominator.
\item \( a \in [4.5, 6.5] \text{ and } b \in [187.5, 189.5] \)
 $5.14  + 188.00 i$, which corresponds to forgetting to multiply the conjugate by the numerator.
\end{enumerate}

\textbf{General Comment:} Multiply the numerator and denominator by the *conjugate* of the denominator, then simplify. For example, if we have $2+3i$, the conjugate is $2-3i$.
}
\litem{
Simplify the expression below into the form $a+bi$. Then, choose the intervals that $a$ and $b$ belong to.
\[ (3 + 6 i)(8 - 7 i) \]
The solution is \( 66 + 27 i \), which is option A.\begin{enumerate}[label=\Alph*.]
\item \( a \in [63, 70] \text{ and } b \in [19, 29] \)
* $66 + 27 i$, which is the correct option.
\item \( a \in [63, 70] \text{ and } b \in [-27, -23] \)
 $66 - 27 i$, which corresponds to adding a minus sign in both terms.
\item \( a \in [-18, -15] \text{ and } b \in [-69, -68] \)
 $-18 - 69 i$, which corresponds to adding a minus sign in the first term.
\item \( a \in [-18, -15] \text{ and } b \in [68, 70] \)
 $-18 + 69 i$, which corresponds to adding a minus sign in the second term.
\item \( a \in [22, 28] \text{ and } b \in [-45, -41] \)
 $24 - 42 i$, which corresponds to just multiplying the real terms to get the real part of the solution and the coefficients in the complex terms to get the complex part.
\end{enumerate}

\textbf{General Comment:} You can treat $i$ as a variable and distribute. Just remember that $i^2=-1$, so you can continue to reduce after you distribute.
}
\litem{
Simplify the expression below and choose the interval the simplification is contained within.
\[ 4 - 7^2 + 8 \div 20 * 11 \div 13 \]
The solution is \( -44.662 \), which is option A.\begin{enumerate}[label=\Alph*.]
\item \( [-44.79, -44.62] \)
* -44.662, this is the correct option
\item \( [52.89, 53.08] \)
 53.003, which corresponds to two Order of Operations errors.
\item \( [53.1, 53.74] \)
 53.338, which corresponds to an Order of Operations error: multiplying by negative before squaring. For example: $(-3)^2 \neq -3^2$
\item \( [-45.14, -44.87] \)
 -44.997, which corresponds to an Order of Operations error: not reading left-to-right for multiplication/division.
\item \( \text{None of the above} \)
 You may have gotten this by making an unanticipated error. If you got a value that is not any of the others, please let the coordinator know so they can help you figure out what happened.
\end{enumerate}

\textbf{General Comment:} While you may remember (or were taught) PEMDAS is done in order, it is actually done as P/E/MD/AS. When we are at MD or AS, we read left to right.
}
\litem{
Choose the \textbf{smallest} set of Complex numbers that the number below belongs to.
\[ \sqrt{\frac{324}{0}}+\sqrt{165} i \]
The solution is \( \text{Not a Complex Number} \), which is option C.\begin{enumerate}[label=\Alph*.]
\item \( \text{Irrational} \)
These cannot be written as a fraction of Integers. Remember: $\pi$ is not an Integer!
\item \( \text{Rational} \)
These are numbers that can be written as fraction of Integers (e.g., -2/3 + 5)
\item \( \text{Not a Complex Number} \)
* This is the correct option!
\item \( \text{Pure Imaginary} \)
This is a Complex number $(a+bi)$ that \textbf{only} has an imaginary part like $2i$.
\item \( \text{Nonreal Complex} \)
This is a Complex number $(a+bi)$ that is not Real (has $i$ as part of the number).
\end{enumerate}

\textbf{General Comment:} Be sure to simplify $i^2 = -1$. This may remove the imaginary portion for your number. If you are having trouble, you may want to look at the \textit{Subgroups of the Real Numbers} section.
}
\litem{
Choose the \textbf{smallest} set of Real numbers that the number below belongs to.
\[ -\sqrt{\frac{6}{0}} \]
The solution is \( \text{Not a Real number} \), which is option D.\begin{enumerate}[label=\Alph*.]
\item \( \text{Irrational} \)
These cannot be written as a fraction of Integers.
\item \( \text{Rational} \)
These are numbers that can be written as fraction of Integers (e.g., -2/3)
\item \( \text{Whole} \)
These are the counting numbers with 0 (0, 1, 2, 3, ...)
\item \( \text{Not a Real number} \)
* This is the correct option!
\item \( \text{Integer} \)
These are the negative and positive counting numbers (..., -3, -2, -1, 0, 1, 2, 3, ...)
\end{enumerate}

\textbf{General Comment:} First, you \textbf{NEED} to simplify the expression. This question simplifies to $-\sqrt{\frac{6}{0}}$. 
 
 Be sure you look at the simplified fraction and not just the decimal expansion. Numbers such as 13, 17, and 19 provide \textbf{long but repeating/terminating decimal expansions!} 
 
 The only ways to *not* be a Real number are: dividing by 0 or taking the square root of a negative number. 
 
 Irrational numbers are more than just square root of 3: adding or subtracting values from square root of 3 is also irrational.
}
\litem{
Simplify the expression below into the form $a+bi$. Then, choose the intervals that $a$ and $b$ belong to.
\[ \frac{63 + 11 i}{-4 - 3 i} \]
The solution is \( -11.40  + 5.80 i \), which is option E.\begin{enumerate}[label=\Alph*.]
\item \( a \in [-17, -14] \text{ and } b \in [-4.5, -3] \)
 $-15.75  - 3.67 i$, which corresponds to just dividing the first term by the first term and the second by the second.
\item \( a \in [-285.5, -284] \text{ and } b \in [5.5, 6.5] \)
 $-285.00  + 5.80 i$, which corresponds to forgetting to multiply the conjugate by the numerator and using a plus instead of a minus in the denominator.
\item \( a \in [-9.5, -8] \text{ and } b \in [-10.5, -8.5] \)
 $-8.76  - 9.32 i$, which corresponds to forgetting to multiply the conjugate by the numerator and not computing the conjugate correctly.
\item \( a \in [-12.5, -11] \text{ and } b \in [144.5, 146.5] \)
 $-11.40  + 145.00 i$, which corresponds to forgetting to multiply the conjugate by the numerator.
\item \( a \in [-12.5, -11] \text{ and } b \in [5.5, 6.5] \)
* $-11.40  + 5.80 i$, which is the correct option.
\end{enumerate}

\textbf{General Comment:} Multiply the numerator and denominator by the *conjugate* of the denominator, then simplify. For example, if we have $2+3i$, the conjugate is $2-3i$.
}
\litem{
Simplify the expression below into the form $a+bi$. Then, choose the intervals that $a$ and $b$ belong to.
\[ (8 + 7 i)(-3 - 9 i) \]
The solution is \( 39 - 93 i \), which is option B.\begin{enumerate}[label=\Alph*.]
\item \( a \in [-31, -21] \text{ and } b \in [-66, -56] \)
 $-24 - 63 i$, which corresponds to just multiplying the real terms to get the real part of the solution and the coefficients in the complex terms to get the complex part.
\item \( a \in [37, 42] \text{ and } b \in [-96, -91] \)
* $39 - 93 i$, which is the correct option.
\item \( a \in [-93, -85] \text{ and } b \in [51, 52] \)
 $-87 + 51 i$, which corresponds to adding a minus sign in the second term.
\item \( a \in [37, 42] \text{ and } b \in [88, 97] \)
 $39 + 93 i$, which corresponds to adding a minus sign in both terms.
\item \( a \in [-93, -85] \text{ and } b \in [-56, -50] \)
 $-87 - 51 i$, which corresponds to adding a minus sign in the first term.
\end{enumerate}

\textbf{General Comment:} You can treat $i$ as a variable and distribute. Just remember that $i^2=-1$, so you can continue to reduce after you distribute.
}
\litem{
Simplify the expression below and choose the interval the simplification is contained within.
\[ 5 - 9 \div 10 * 3 - (7 * 19) \]
The solution is \( -130.700 \), which is option A.\begin{enumerate}[label=\Alph*.]
\item \( [-132, -129.9] \)
* -130.700, which is the correct option.
\item \( [-129.1, -127.2] \)
 -128.300, which corresponds to an Order of Operations error: not reading left-to-right for multiplication/division.
\item \( [-89.6, -88.4] \)
 -89.300, which corresponds to not distributing a negative correctly.
\item \( [136.5, 137.8] \)
 137.700, which corresponds to not distributing addition and subtraction correctly.
\item \( \text{None of the above} \)
 You may have gotten this by making an unanticipated error. If you got a value that is not any of the others, please let the coordinator know so they can help you figure out what happened.
\end{enumerate}

\textbf{General Comment:} While you may remember (or were taught) PEMDAS is done in order, it is actually done as P/E/MD/AS. When we are at MD or AS, we read left to right.
}
\litem{
Choose the \textbf{smallest} set of Complex numbers that the number below belongs to.
\[ \frac{-5}{10}+\sqrt{90} i \]
The solution is \( \text{Nonreal Complex} \), which is option D.\begin{enumerate}[label=\Alph*.]
\item \( \text{Pure Imaginary} \)
This is a Complex number $(a+bi)$ that \textbf{only} has an imaginary part like $2i$.
\item \( \text{Not a Complex Number} \)
This is not a number. The only non-Complex number we know is dividing by 0 as this is not a number!
\item \( \text{Irrational} \)
These cannot be written as a fraction of Integers. Remember: $\pi$ is not an Integer!
\item \( \text{Nonreal Complex} \)
* This is the correct option!
\item \( \text{Rational} \)
These are numbers that can be written as fraction of Integers (e.g., -2/3 + 5)
\end{enumerate}

\textbf{General Comment:} Be sure to simplify $i^2 = -1$. This may remove the imaginary portion for your number. If you are having trouble, you may want to look at the \textit{Subgroups of the Real Numbers} section.
}
\litem{
Choose the \textbf{smallest} set of Real numbers that the number below belongs to.
\[ -\sqrt{\frac{900}{10}} \]
The solution is \( \text{Irrational} \), which is option D.\begin{enumerate}[label=\Alph*.]
\item \( \text{Integer} \)
These are the negative and positive counting numbers (..., -3, -2, -1, 0, 1, 2, 3, ...)
\item \( \text{Whole} \)
These are the counting numbers with 0 (0, 1, 2, 3, ...)
\item \( \text{Not a Real number} \)
These are Nonreal Complex numbers \textbf{OR} things that are not numbers (e.g., dividing by 0).
\item \( \text{Irrational} \)
* This is the correct option!
\item \( \text{Rational} \)
These are numbers that can be written as fraction of Integers (e.g., -2/3)
\end{enumerate}

\textbf{General Comment:} First, you \textbf{NEED} to simplify the expression. This question simplifies to $-\sqrt{90}$. 
 
 Be sure you look at the simplified fraction and not just the decimal expansion. Numbers such as 13, 17, and 19 provide \textbf{long but repeating/terminating decimal expansions!} 
 
 The only ways to *not* be a Real number are: dividing by 0 or taking the square root of a negative number. 
 
 Irrational numbers are more than just square root of 3: adding or subtracting values from square root of 3 is also irrational.
}
\end{enumerate}

\end{document}