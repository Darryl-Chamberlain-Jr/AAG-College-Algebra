\documentclass{extbook}[14pt]
\usepackage{multicol, enumerate, enumitem, hyperref, color, soul, setspace, parskip, fancyhdr, amssymb, amsthm, amsmath, latexsym, units, mathtools}
\everymath{\displaystyle}
\usepackage[headsep=0.5cm,headheight=0cm, left=1 in,right= 1 in,top= 1 in,bottom= 1 in]{geometry}
\usepackage{dashrule}  % Package to use the command below to create lines between items
\newcommand{\litem}[1]{\item #1

\rule{\textwidth}{0.4pt}}
\pagestyle{fancy}
\lhead{}
\chead{Answer Key for Progress Quiz 1 Version C}
\rhead{}
\lfoot{4082-7053}
\cfoot{}
\rfoot{test}
\begin{document}
\textbf{This key should allow you to understand why you choose the option you did (beyond just getting a question right or wrong). \href{https://xronos.clas.ufl.edu/mac1105spring2020/courseDescriptionAndMisc/Exams/LearningFromResults}{More instructions on how to use this key can be found here}.}

\textbf{If you have a suggestion to make the keys better, \href{https://forms.gle/CZkbZmPbC9XALEE88}{please fill out the short survey here}.}

\textit{Note: This key is auto-generated and may contain issues and/or errors. The keys are reviewed after each exam to ensure grading is done accurately. If there are issues (like duplicate options), they are noted in the offline gradebook. The keys are a work-in-progress to give students as many resources to improve as possible.}

\rule{\textwidth}{0.4pt}

\begin{enumerate}\litem{
Choose the \textbf{smallest} set of Real numbers that the number below belongs to.
\[ \sqrt{\frac{4096}{64}} \]The solution is \( \text{Whole} \), which is option C.\begin{enumerate}[label=\Alph*.]
\item \( \text{Rational} \)

These are numbers that can be written as fraction of Integers (e.g., -2/3)
\item \( \text{Not a Real number} \)

These are Nonreal Complex numbers \textbf{OR} things that are not numbers (e.g., dividing by 0).
\item \( \text{Whole} \)

* This is the correct option!
\item \( \text{Integer} \)

These are the negative and positive counting numbers (..., -3, -2, -1, 0, 1, 2, 3, ...)
\item \( \text{Irrational} \)

These cannot be written as a fraction of Integers.
\end{enumerate}

\textbf{General Comment:} First, you \textbf{NEED} to simplify the expression. This question simplifies to $64$. 
 
 Be sure you look at the simplified fraction and not just the decimal expansion. Numbers such as 13, 17, and 19 provide \textbf{long but repeating/terminating decimal expansions!} 
 
 The only ways to *not* be a Real number are: dividing by 0 or taking the square root of a negative number. 
 
 Irrational numbers are more than just square root of 3: adding or subtracting values from square root of 3 is also irrational.
}
\litem{
Simplify the expression below and choose the interval the simplification is contained within.
\[ 12 - 10 \div 17 * 15 - (6 * 18) \]The solution is \( -104.824 \), which is option C.\begin{enumerate}[label=\Alph*.]
\item \( [-53.82, -46.82] \)

 -50.824, which corresponds to not distributing a negative correctly.
\item \( [115.96, 121.96] \)

 119.961, which corresponds to not distributing addition and subtraction correctly.
\item \( [-110.82, -103.82] \)

* -104.824, which is the correct option.
\item \( [-104.04, -92.04] \)

 -96.039, which corresponds to an Order of Operations error: not reading left-to-right for multiplication/division.
\item \( \text{None of the above} \)

 You may have gotten this by making an unanticipated error. If you got a value that is not any of the others, please let the coordinator know so they can help you figure out what happened.
\end{enumerate}

\textbf{General Comment:} While you may remember (or were taught) PEMDAS is done in order, it is actually done as P/E/MD/AS. When we are at MD or AS, we read left to right.
}
\litem{
Simplify the expression below and choose the interval the simplification is contained within.
\[ 12 - 7^2 + 19 \div 14 * 17 \div 10 \]The solution is \( -34.693 \), which is option D.\begin{enumerate}[label=\Alph*.]
\item \( [63.08, 64.09] \)

 63.307, which corresponds to an Order of Operations error: multiplying by negative before squaring. For example: $(-3)^2 \neq -3^2$
\item \( [-38.16, -36.7] \)

 -36.992, which corresponds to an Order of Operations error: not reading left-to-right for multiplication/division.
\item \( [60.5, 61.5] \)

 61.008, which corresponds to two Order of Operations errors.
\item \( [-34.8, -34.29] \)

* -34.693, this is the correct option
\item \( \text{None of the above} \)

 You may have gotten this by making an unanticipated error. If you got a value that is not any of the others, please let the coordinator know so they can help you figure out what happened.
\end{enumerate}

\textbf{General Comment:} While you may remember (or were taught) PEMDAS is done in order, it is actually done as P/E/MD/AS. When we are at MD or AS, we read left to right.
}
\litem{
Choose the \textbf{smallest} set of Complex numbers that the number below belongs to.
\[ \frac{\sqrt{165}}{16}+3i^2 \]The solution is \( \text{Irrational} \), which is option E.\begin{enumerate}[label=\Alph*.]
\item \( \text{Rational} \)

These are numbers that can be written as fraction of Integers (e.g., -2/3 + 5)
\item \( \text{Pure Imaginary} \)

This is a Complex number $(a+bi)$ that \textbf{only} has an imaginary part like $2i$.
\item \( \text{Not a Complex Number} \)

This is not a number. The only non-Complex number we know is dividing by 0 as this is not a number!
\item \( \text{Nonreal Complex} \)

This is a Complex number $(a+bi)$ that is not Real (has $i$ as part of the number).
\item \( \text{Irrational} \)

* This is the correct option!
\end{enumerate}

\textbf{General Comment:} Be sure to simplify $i^2 = -1$. This may remove the imaginary portion for your number. If you are having trouble, you may want to look at the \textit{Subgroups of the Real Numbers} section.
}
\litem{
Simplify the expression below into the form $a+bi$. Then, choose the intervals that $a$ and $b$ belong to.
\[ (8 + 7 i)(9 + 10 i) \]The solution is \( 2 + 143 i \), which is option D.\begin{enumerate}[label=\Alph*.]
\item \( a \in [140, 147] \text{ and } b \in [-17, -15] \)

 $142 - 17 i$, which corresponds to adding a minus sign in the second term.
\item \( a \in [1, 7] \text{ and } b \in [-152, -141] \)

 $2 - 143 i$, which corresponds to adding a minus sign in both terms.
\item \( a \in [140, 147] \text{ and } b \in [14, 18] \)

 $142 + 17 i$, which corresponds to adding a minus sign in the first term.
\item \( a \in [1, 7] \text{ and } b \in [142, 146] \)

* $2 + 143 i$, which is the correct option.
\item \( a \in [70, 77] \text{ and } b \in [64, 79] \)

 $72 + 70 i$, which corresponds to just multiplying the real terms to get the real part of the solution and the coefficients in the complex terms to get the complex part.
\end{enumerate}

\textbf{General Comment:} You can treat $i$ as a variable and distribute. Just remember that $i^2=-1$, so you can continue to reduce after you distribute.
}
\litem{
Simplify the expression below into the form $a+bi$. Then, choose the intervals that $a$ and $b$ belong to.
\[ \frac{-45 + 77 i}{2 - 3 i} \]The solution is \( -24.69  + 1.46 i \), which is option C.\begin{enumerate}[label=\Alph*.]
\item \( a \in [-23, -22] \text{ and } b \in [-26.5, -25] \)

 $-22.50  - 25.67 i$, which corresponds to just dividing the first term by the first term and the second by the second.
\item \( a \in [-25, -24.5] \text{ and } b \in [18, 20] \)

 $-24.69  + 19.00 i$, which corresponds to forgetting to multiply the conjugate by the numerator.
\item \( a \in [-25, -24.5] \text{ and } b \in [0.5, 2] \)

* $-24.69  + 1.46 i$, which is the correct option.
\item \( a \in [-321.5, -319.5] \text{ and } b \in [0.5, 2] \)

 $-321.00  + 1.46 i$, which corresponds to forgetting to multiply the conjugate by the numerator and using a plus instead of a minus in the denominator.
\item \( a \in [10, 11.5] \text{ and } b \in [21.5, 23] \)

 $10.85  + 22.23 i$, which corresponds to forgetting to multiply the conjugate by the numerator and not computing the conjugate correctly.
\end{enumerate}

\textbf{General Comment:} Multiply the numerator and denominator by the *conjugate* of the denominator, then simplify. For example, if we have $2+3i$, the conjugate is $2-3i$.
}
\litem{
Choose the \textbf{smallest} set of Complex numbers that the number below belongs to.
\[ \sqrt{\frac{361}{256}}+\sqrt{198} i \]The solution is \( \text{Nonreal Complex} \), which is option C.\begin{enumerate}[label=\Alph*.]
\item \( \text{Pure Imaginary} \)

This is a Complex number $(a+bi)$ that \textbf{only} has an imaginary part like $2i$.
\item \( \text{Not a Complex Number} \)

This is not a number. The only non-Complex number we know is dividing by 0 as this is not a number!
\item \( \text{Nonreal Complex} \)

* This is the correct option!
\item \( \text{Rational} \)

These are numbers that can be written as fraction of Integers (e.g., -2/3 + 5)
\item \( \text{Irrational} \)

These cannot be written as a fraction of Integers. Remember: $\pi$ is not an Integer!
\end{enumerate}

\textbf{General Comment:} Be sure to simplify $i^2 = -1$. This may remove the imaginary portion for your number. If you are having trouble, you may want to look at the \textit{Subgroups of the Real Numbers} section.
}
\litem{
Simplify the expression below into the form $a+bi$. Then, choose the intervals that $a$ and $b$ belong to.
\[ (4 - 8 i)(6 + 7 i) \]The solution is \( 80 - 20 i \), which is option E.\begin{enumerate}[label=\Alph*.]
\item \( a \in [-36, -28] \text{ and } b \in [70, 79] \)

 $-32 + 76 i$, which corresponds to adding a minus sign in the first term.
\item \( a \in [23, 28] \text{ and } b \in [-59, -52] \)

 $24 - 56 i$, which corresponds to just multiplying the real terms to get the real part of the solution and the coefficients in the complex terms to get the complex part.
\item \( a \in [-36, -28] \text{ and } b \in [-83, -70] \)

 $-32 - 76 i$, which corresponds to adding a minus sign in the second term.
\item \( a \in [75, 87] \text{ and } b \in [17, 21] \)

 $80 + 20 i$, which corresponds to adding a minus sign in both terms.
\item \( a \in [75, 87] \text{ and } b \in [-24, -18] \)

* $80 - 20 i$, which is the correct option.
\end{enumerate}

\textbf{General Comment:} You can treat $i$ as a variable and distribute. Just remember that $i^2=-1$, so you can continue to reduce after you distribute.
}
\litem{
Simplify the expression below into the form $a+bi$. Then, choose the intervals that $a$ and $b$ belong to.
\[ \frac{-45 - 44 i}{2 + 7 i} \]The solution is \( -7.51  + 4.28 i \), which is option E.\begin{enumerate}[label=\Alph*.]
\item \( a \in [-9, -6.5] \text{ and } b \in [226, 228.5] \)

 $-7.51  + 227.00 i$, which corresponds to forgetting to multiply the conjugate by the numerator.
\item \( a \in [-399, -397] \text{ and } b \in [4, 5] \)

 $-398.00  + 4.28 i$, which corresponds to forgetting to multiply the conjugate by the numerator and using a plus instead of a minus in the denominator.
\item \( a \in [3, 5] \text{ and } b \in [-9, -7] \)

 $4.11  - 7.60 i$, which corresponds to forgetting to multiply the conjugate by the numerator and not computing the conjugate correctly.
\item \( a \in [-23, -21.5] \text{ and } b \in [-7, -6] \)

 $-22.50  - 6.29 i$, which corresponds to just dividing the first term by the first term and the second by the second.
\item \( a \in [-9, -6.5] \text{ and } b \in [4, 5] \)

* $-7.51  + 4.28 i$, which is the correct option.
\end{enumerate}

\textbf{General Comment:} Multiply the numerator and denominator by the *conjugate* of the denominator, then simplify. For example, if we have $2+3i$, the conjugate is $2-3i$.
}
\litem{
Choose the \textbf{smallest} set of Real numbers that the number below belongs to.
\[ -\sqrt{\frac{1680}{15}} \]The solution is \( \text{Irrational} \), which is option C.\begin{enumerate}[label=\Alph*.]
\item \( \text{Rational} \)

These are numbers that can be written as fraction of Integers (e.g., -2/3)
\item \( \text{Whole} \)

These are the counting numbers with 0 (0, 1, 2, 3, ...)
\item \( \text{Irrational} \)

* This is the correct option!
\item \( \text{Integer} \)

These are the negative and positive counting numbers (..., -3, -2, -1, 0, 1, 2, 3, ...)
\item \( \text{Not a Real number} \)

These are Nonreal Complex numbers \textbf{OR} things that are not numbers (e.g., dividing by 0).
\end{enumerate}

\textbf{General Comment:} First, you \textbf{NEED} to simplify the expression. This question simplifies to $-\sqrt{112}$. 
 
 Be sure you look at the simplified fraction and not just the decimal expansion. Numbers such as 13, 17, and 19 provide \textbf{long but repeating/terminating decimal expansions!} 
 
 The only ways to *not* be a Real number are: dividing by 0 or taking the square root of a negative number. 
 
 Irrational numbers are more than just square root of 3: adding or subtracting values from square root of 3 is also irrational.
}
\end{enumerate}

\end{document}