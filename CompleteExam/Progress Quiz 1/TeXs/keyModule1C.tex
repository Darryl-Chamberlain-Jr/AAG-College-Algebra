\documentclass{extbook}[14pt]
\usepackage{multicol, enumerate, enumitem, hyperref, color, soul, setspace, parskip, fancyhdr, amssymb, amsthm, amsmath, latexsym, units, mathtools}
\everymath{\displaystyle}
\usepackage[headsep=0.5cm,headheight=0cm, left=1 in,right= 1 in,top= 1 in,bottom= 1 in]{geometry}
\usepackage{dashrule}  % Package to use the command below to create lines between items
\newcommand{\litem}[1]{\item #1

\rule{\textwidth}{0.4pt}}
\pagestyle{fancy}
\lhead{}
\chead{Answer Key for Progress Quiz 1 Version C}
\rhead{}
\lfoot{5899-4682}
\cfoot{}
\rfoot{Spring 2021}
\begin{document}
\textbf{This key should allow you to understand why you choose the option you did (beyond just getting a question right or wrong). \href{https://xronos.clas.ufl.edu/mac1105spring2020/courseDescriptionAndMisc/Exams/LearningFromResults}{More instructions on how to use this key can be found here}.}

\textbf{If you have a suggestion to make the keys better, \href{https://forms.gle/CZkbZmPbC9XALEE88}{please fill out the short survey here}.}

\textit{Note: This key is auto-generated and may contain issues and/or errors. The keys are reviewed after each exam to ensure grading is done accurately. If there are issues (like duplicate options), they are noted in the offline gradebook. The keys are a work-in-progress to give students as many resources to improve as possible.}

\rule{\textwidth}{0.4pt}

\begin{enumerate}\litem{
Choose the \textbf{smallest} set of Real numbers that the number below belongs to.
\[ -\sqrt{\frac{625}{49}} \]The solution is \( \text{Rational} \), which is option D.\begin{enumerate}[label=\Alph*.]
\item \( \text{Integer} \)

These are the negative and positive counting numbers (..., -3, -2, -1, 0, 1, 2, 3, ...)
\item \( \text{Irrational} \)

These cannot be written as a fraction of Integers.
\item \( \text{Not a Real number} \)

These are Nonreal Complex numbers \textbf{OR} things that are not numbers (e.g., dividing by 0).
\item \( \text{Rational} \)

* This is the correct option!
\item \( \text{Whole} \)

These are the counting numbers with 0 (0, 1, 2, 3, ...)
\end{enumerate}

\textbf{General Comment:} First, you \textbf{NEED} to simplify the expression. This question simplifies to $-\frac{25}{7}$. 
 
 Be sure you look at the simplified fraction and not just the decimal expansion. Numbers such as 13, 17, and 19 provide \textbf{long but repeating/terminating decimal expansions!} 
 
 The only ways to *not* be a Real number are: dividing by 0 or taking the square root of a negative number. 
 
 Irrational numbers are more than just square root of 3: adding or subtracting values from square root of 3 is also irrational.
}
\litem{
Choose the \textbf{smallest} set of Real numbers that the number below belongs to.
\[ \sqrt{\frac{81}{25}} \]The solution is \( \text{Rational} \), which is option D.\begin{enumerate}[label=\Alph*.]
\item \( \text{Whole} \)

These are the counting numbers with 0 (0, 1, 2, 3, ...)
\item \( \text{Not a Real number} \)

These are Nonreal Complex numbers \textbf{OR} things that are not numbers (e.g., dividing by 0).
\item \( \text{Integer} \)

These are the negative and positive counting numbers (..., -3, -2, -1, 0, 1, 2, 3, ...)
\item \( \text{Rational} \)

* This is the correct option!
\item \( \text{Irrational} \)

These cannot be written as a fraction of Integers.
\end{enumerate}

\textbf{General Comment:} First, you \textbf{NEED} to simplify the expression. This question simplifies to $\frac{9}{5}$. 
 
 Be sure you look at the simplified fraction and not just the decimal expansion. Numbers such as 13, 17, and 19 provide \textbf{long but repeating/terminating decimal expansions!} 
 
 The only ways to *not* be a Real number are: dividing by 0 or taking the square root of a negative number. 
 
 Irrational numbers are more than just square root of 3: adding or subtracting values from square root of 3 is also irrational.
}
\litem{
Simplify the expression below into the form $a+bi$. Then, choose the intervals that $a$ and $b$ belong to.
\[ (-5 + 9 i)(8 + 2 i) \]The solution is \( -58 + 62 i \), which is option B.\begin{enumerate}[label=\Alph*.]
\item \( a \in [-60, -51] \text{ and } b \in [-63, -59] \)

 $-58 - 62 i$, which corresponds to adding a minus sign in both terms.
\item \( a \in [-60, -51] \text{ and } b \in [61, 68] \)

* $-58 + 62 i$, which is the correct option.
\item \( a \in [-23, -17] \text{ and } b \in [78, 86] \)

 $-22 + 82 i$, which corresponds to adding a minus sign in the second term.
\item \( a \in [-23, -17] \text{ and } b \in [-87, -73] \)

 $-22 - 82 i$, which corresponds to adding a minus sign in the first term.
\item \( a \in [-42, -34] \text{ and } b \in [12, 20] \)

 $-40 + 18 i$, which corresponds to just multiplying the real terms to get the real part of the solution and the coefficients in the complex terms to get the complex part.
\end{enumerate}

\textbf{General Comment:} You can treat $i$ as a variable and distribute. Just remember that $i^2=-1$, so you can continue to reduce after you distribute.
}
\litem{
Simplify the expression below into the form $a+bi$. Then, choose the intervals that $a$ and $b$ belong to.
\[ \frac{9 + 77 i}{4 - 3 i} \]The solution is \( -7.80  + 13.40 i \), which is option B.\begin{enumerate}[label=\Alph*.]
\item \( a \in [1.5, 3.5] \text{ and } b \in [-26.5, -25] \)

 $2.25  - 25.67 i$, which corresponds to just dividing the first term by the first term and the second by the second.
\item \( a \in [-9.5, -6.5] \text{ and } b \in [13, 15] \)

* $-7.80  + 13.40 i$, which is the correct option.
\item \( a \in [10, 11] \text{ and } b \in [11, 12] \)

 $10.68  + 11.24 i$, which corresponds to forgetting to multiply the conjugate by the numerator and not computing the conjugate correctly.
\item \( a \in [-9.5, -6.5] \text{ and } b \in [334.5, 336] \)

 $-7.80  + 335.00 i$, which corresponds to forgetting to multiply the conjugate by the numerator.
\item \( a \in [-195.5, -194.5] \text{ and } b \in [13, 15] \)

 $-195.00  + 13.40 i$, which corresponds to forgetting to multiply the conjugate by the numerator and using a plus instead of a minus in the denominator.
\end{enumerate}

\textbf{General Comment:} Multiply the numerator and denominator by the *conjugate* of the denominator, then simplify. For example, if we have $2+3i$, the conjugate is $2-3i$.
}
\litem{
Choose the \textbf{smallest} set of Complex numbers that the number below belongs to.
\[ \frac{-12}{12}+25i^2 \]The solution is \( \text{Rational} \), which is option E.\begin{enumerate}[label=\Alph*.]
\item \( \text{Not a Complex Number} \)

This is not a number. The only non-Complex number we know is dividing by 0 as this is not a number!
\item \( \text{Pure Imaginary} \)

This is a Complex number $(a+bi)$ that \textbf{only} has an imaginary part like $2i$.
\item \( \text{Nonreal Complex} \)

This is a Complex number $(a+bi)$ that is not Real (has $i$ as part of the number).
\item \( \text{Irrational} \)

These cannot be written as a fraction of Integers. Remember: $\pi$ is not an Integer!
\item \( \text{Rational} \)

* This is the correct option!
\end{enumerate}

\textbf{General Comment:} Be sure to simplify $i^2 = -1$. This may remove the imaginary portion for your number. If you are having trouble, you may want to look at the \textit{Subgroups of the Real Numbers} section.
}
\litem{
Simplify the expression below and choose the interval the simplification is contained within.
\[ 17 - 14 \div 20 * 9 - (3 * 19) \]The solution is \( -46.300 \), which is option A.\begin{enumerate}[label=\Alph*.]
\item \( [-47.3, -43.3] \)

* -46.300, which is the correct option.
\item \( [-44.08, -37.08] \)

 -40.078, which corresponds to an Order of Operations error: not reading left-to-right for multiplication/division.
\item \( [70.92, 75.92] \)

 73.922, which corresponds to not distributing addition and subtraction correctly.
\item \( [142.3, 151.3] \)

 146.300, which corresponds to not distributing a negative correctly.
\item \( \text{None of the above} \)

 You may have gotten this by making an unanticipated error. If you got a value that is not any of the others, please let the coordinator know so they can help you figure out what happened.
\end{enumerate}

\textbf{General Comment:} While you may remember (or were taught) PEMDAS is done in order, it is actually done as P/E/MD/AS. When we are at MD or AS, we read left to right.
}
\litem{
Simplify the expression below and choose the interval the simplification is contained within.
\[ 17 - 10 \div 7 * 18 - (9 * 4) \]The solution is \( -44.714 \), which is option C.\begin{enumerate}[label=\Alph*.]
\item \( [-19.08, -17.08] \)

 -19.079, which corresponds to an Order of Operations error: not reading left-to-right for multiplication/division.
\item \( [-71.86, -66.86] \)

 -70.857, which corresponds to not distributing a negative correctly.
\item \( [-46.71, -43.71] \)

* -44.714, which is the correct option.
\item \( [49.92, 54.92] \)

 52.921, which corresponds to not distributing addition and subtraction correctly.
\item \( \text{None of the above} \)

 You may have gotten this by making an unanticipated error. If you got a value that is not any of the others, please let the coordinator know so they can help you figure out what happened.
\end{enumerate}

\textbf{General Comment:} While you may remember (or were taught) PEMDAS is done in order, it is actually done as P/E/MD/AS. When we are at MD or AS, we read left to right.
}
\litem{
Choose the \textbf{smallest} set of Complex numbers that the number below belongs to.
\[ \frac{-11}{-6}+64i^2 \]The solution is \( \text{Rational} \), which is option C.\begin{enumerate}[label=\Alph*.]
\item \( \text{Irrational} \)

These cannot be written as a fraction of Integers. Remember: $\pi$ is not an Integer!
\item \( \text{Not a Complex Number} \)

This is not a number. The only non-Complex number we know is dividing by 0 as this is not a number!
\item \( \text{Rational} \)

* This is the correct option!
\item \( \text{Pure Imaginary} \)

This is a Complex number $(a+bi)$ that \textbf{only} has an imaginary part like $2i$.
\item \( \text{Nonreal Complex} \)

This is a Complex number $(a+bi)$ that is not Real (has $i$ as part of the number).
\end{enumerate}

\textbf{General Comment:} Be sure to simplify $i^2 = -1$. This may remove the imaginary portion for your number. If you are having trouble, you may want to look at the \textit{Subgroups of the Real Numbers} section.
}
\litem{
Simplify the expression below into the form $a+bi$. Then, choose the intervals that $a$ and $b$ belong to.
\[ (-3 + 8 i)(6 + 4 i) \]The solution is \( -50 + 36 i \), which is option E.\begin{enumerate}[label=\Alph*.]
\item \( a \in [5, 15] \text{ and } b \in [-62.5, -59.6] \)

 $14 - 60 i$, which corresponds to adding a minus sign in the first term.
\item \( a \in [5, 15] \text{ and } b \in [58.5, 60.7] \)

 $14 + 60 i$, which corresponds to adding a minus sign in the second term.
\item \( a \in [-20, -12] \text{ and } b \in [29.1, 34.4] \)

 $-18 + 32 i$, which corresponds to just multiplying the real terms to get the real part of the solution and the coefficients in the complex terms to get the complex part.
\item \( a \in [-52, -44] \text{ and } b \in [-36.3, -35] \)

 $-50 - 36 i$, which corresponds to adding a minus sign in both terms.
\item \( a \in [-52, -44] \text{ and } b \in [35.9, 37.1] \)

* $-50 + 36 i$, which is the correct option.
\end{enumerate}

\textbf{General Comment:} You can treat $i$ as a variable and distribute. Just remember that $i^2=-1$, so you can continue to reduce after you distribute.
}
\litem{
Simplify the expression below into the form $a+bi$. Then, choose the intervals that $a$ and $b$ belong to.
\[ \frac{54 - 88 i}{7 + i} \]The solution is \( 5.80  - 13.40 i \), which is option C.\begin{enumerate}[label=\Alph*.]
\item \( a \in [289.5, 291] \text{ and } b \in [-14, -11.5] \)

 $290.00  - 13.40 i$, which corresponds to forgetting to multiply the conjugate by the numerator and using a plus instead of a minus in the denominator.
\item \( a \in [6.5, 8] \text{ and } b \in [-89, -87.5] \)

 $7.71  - 88.00 i$, which corresponds to just dividing the first term by the first term and the second by the second.
\item \( a \in [4, 7] \text{ and } b \in [-14, -11.5] \)

* $5.80  - 13.40 i$, which is the correct option.
\item \( a \in [8.5, 11] \text{ and } b \in [-12, -10] \)

 $9.32  - 11.24 i$, which corresponds to forgetting to multiply the conjugate by the numerator and not computing the conjugate correctly.
\item \( a \in [4, 7] \text{ and } b \in [-670.5, -669.5] \)

 $5.80  - 670.00 i$, which corresponds to forgetting to multiply the conjugate by the numerator.
\end{enumerate}

\textbf{General Comment:} Multiply the numerator and denominator by the *conjugate* of the denominator, then simplify. For example, if we have $2+3i$, the conjugate is $2-3i$.
}
\end{enumerate}

\end{document}