\documentclass{extbook}[14pt]
\usepackage{multicol, enumerate, enumitem, hyperref, color, soul, setspace, parskip, fancyhdr, amssymb, amsthm, amsmath, bbm, latexsym, units, mathtools}
\everymath{\displaystyle}
\usepackage[headsep=0.5cm,headheight=0cm, left=1 in,right= 1 in,top= 1 in,bottom= 1 in]{geometry}
\usepackage{dashrule}  % Package to use the command below to create lines between items
\newcommand{\litem}[1]{\item #1

\rule{\textwidth}{0.4pt}}
\pagestyle{fancy}
\lhead{}
\chead{Answer Key for Progress Quiz 1 Version C}
\rhead{}
\lfoot{3735-1698}
\cfoot{}
\rfoot{Spring 2021}
\begin{document}
\textbf{This key should allow you to understand why you choose the option you did (beyond just getting a question right or wrong). \href{https://xronos.clas.ufl.edu/mac1105spring2020/courseDescriptionAndMisc/Exams/LearningFromResults}{More instructions on how to use this key can be found here}.}

\textbf{If you have a suggestion to make the keys better, \href{https://forms.gle/CZkbZmPbC9XALEE88}{please fill out the short survey here}.}

\textit{Note: This key is auto-generated and may contain issues and/or errors. The keys are reviewed after each exam to ensure grading is done accurately. If there are issues (like duplicate options), they are noted in the offline gradebook. The keys are a work-in-progress to give students as many resources to improve as possible.}

\rule{\textwidth}{0.4pt}

\begin{enumerate}\litem{
Simplify the expression below into the form $a+bi$. Then, choose the intervals that $a$ and $b$ belong to.
\[ \frac{9 + 44 i}{-6 - 7 i} \]

The solution is \( -4.26  - 2.36 i \), which is option E.\begin{enumerate}[label=\Alph*.]
\item \( a \in [2, 4] \text{ and } b \in [-5, -3.5] \)

 $2.99  - 3.85 i$, which corresponds to forgetting to multiply the conjugate by the numerator and not computing the conjugate correctly.
\item \( a \in [-363.5, -361.5] \text{ and } b \in [-3, -1.5] \)

 $-362.00  - 2.36 i$, which corresponds to forgetting to multiply the conjugate by the numerator and using a plus instead of a minus in the denominator.
\item \( a \in [-5.5, -3] \text{ and } b \in [-201.5, -200.5] \)

 $-4.26  - 201.00 i$, which corresponds to forgetting to multiply the conjugate by the numerator.
\item \( a \in [-2, -0.5] \text{ and } b \in [-7, -4.5] \)

 $-1.50  - 6.29 i$, which corresponds to just dividing the first term by the first term and the second by the second.
\item \( a \in [-5.5, -3] \text{ and } b \in [-3, -1.5] \)

* $-4.26  - 2.36 i$, which is the correct option.
\end{enumerate}

\textbf{General Comment:} Multiply the numerator and denominator by the *conjugate* of the denominator, then simplify. For example, if we have $2+3i$, the conjugate is $2-3i$.
}
\litem{
Choose the \textbf{smallest} set of Real numbers that the number below belongs to.
\[ -\sqrt{\frac{6}{0}} \]

The solution is \( \text{Not a Real number} \), which is option D.\begin{enumerate}[label=\Alph*.]
\item \( \text{Whole} \)

These are the counting numbers with 0 (0, 1, 2, 3, ...)
\item \( \text{Rational} \)

These are numbers that can be written as fraction of Integers (e.g., -2/3)
\item \( \text{Irrational} \)

These cannot be written as a fraction of Integers.
\item \( \text{Not a Real number} \)

* This is the correct option!
\item \( \text{Integer} \)

These are the negative and positive counting numbers (..., -3, -2, -1, 0, 1, 2, 3, ...)
\end{enumerate}

\textbf{General Comment:} First, you \textbf{NEED} to simplify the expression. This question simplifies to $-\sqrt{\frac{6}{0}}$. 
 
 Be sure you look at the simplified fraction and not just the decimal expansion. Numbers such as 13, 17, and 19 provide \textbf{long but repeating/terminating decimal expansions!} 
 
 The only ways to *not* be a Real number are: dividing by 0 or taking the square root of a negative number. 
 
 Irrational numbers are more than just square root of 3: adding or subtracting values from square root of 3 is also irrational.
}
\litem{
Choose the \textbf{smallest} set of Complex numbers that the number below belongs to.
\[ \sqrt{\frac{0}{529}}+\sqrt{4}i \]

The solution is \( \text{Pure Imaginary} \), which is option B.\begin{enumerate}[label=\Alph*.]
\item \( \text{Rational} \)

These are numbers that can be written as fraction of Integers (e.g., -2/3 + 5)
\item \( \text{Pure Imaginary} \)

* This is the correct option!
\item \( \text{Not a Complex Number} \)

This is not a number. The only non-Complex number we know is dividing by 0 as this is not a number!
\item \( \text{Nonreal Complex} \)

This is a Complex number $(a+bi)$ that is not Real (has $i$ as part of the number).
\item \( \text{Irrational} \)

These cannot be written as a fraction of Integers. Remember: $\pi$ is not an Integer!
\end{enumerate}

\textbf{General Comment:} Be sure to simplify $i^2 = -1$. This may remove the imaginary portion for your number. If you are having trouble, you may want to look at the \textit{Subgroups of the Real Numbers} section.
}
\litem{
Simplify the expression below into the form $a+bi$. Then, choose the intervals that $a$ and $b$ belong to.
\[ (10 - 4 i)(3 - 9 i) \]

The solution is \( -6 - 102 i \), which is option D.\begin{enumerate}[label=\Alph*.]
\item \( a \in [30, 35] \text{ and } b \in [34, 43] \)

 $30 + 36 i$, which corresponds to just multiplying the real terms to get the real part of the solution and the coefficients in the complex terms to get the complex part.
\item \( a \in [64, 68] \text{ and } b \in [77, 84] \)

 $66 + 78 i$, which corresponds to adding a minus sign in the second term.
\item \( a \in [-13, -5] \text{ and } b \in [101, 103] \)

 $-6 + 102 i$, which corresponds to adding a minus sign in both terms.
\item \( a \in [-13, -5] \text{ and } b \in [-104, -95] \)

* $-6 - 102 i$, which is the correct option.
\item \( a \in [64, 68] \text{ and } b \in [-80, -72] \)

 $66 - 78 i$, which corresponds to adding a minus sign in the first term.
\end{enumerate}

\textbf{General Comment:} You can treat $i$ as a variable and distribute. Just remember that $i^2=-1$, so you can continue to reduce after you distribute.
}
\litem{
Simplify the expression below into the form $a+bi$. Then, choose the intervals that $a$ and $b$ belong to.
\[ (10 - 8 i)(-4 + 7 i) \]

The solution is \( 16 + 102 i \), which is option A.\begin{enumerate}[label=\Alph*.]
\item \( a \in [12, 17] \text{ and } b \in [94, 103] \)

* $16 + 102 i$, which is the correct option.
\item \( a \in [-43, -37] \text{ and } b \in [-60, -48] \)

 $-40 - 56 i$, which corresponds to just multiplying the real terms to get the real part of the solution and the coefficients in the complex terms to get the complex part.
\item \( a \in [-102, -94] \text{ and } b \in [-40, -34] \)

 $-96 - 38 i$, which corresponds to adding a minus sign in the second term.
\item \( a \in [-102, -94] \text{ and } b \in [36, 42] \)

 $-96 + 38 i$, which corresponds to adding a minus sign in the first term.
\item \( a \in [12, 17] \text{ and } b \in [-103, -100] \)

 $16 - 102 i$, which corresponds to adding a minus sign in both terms.
\end{enumerate}

\textbf{General Comment:} You can treat $i$ as a variable and distribute. Just remember that $i^2=-1$, so you can continue to reduce after you distribute.
}
\litem{
Simplify the expression below and choose the interval the simplification is contained within.
\[ 8 - 6^2 + 5 \div 15 * 4 \div 13 \]

The solution is \( -27.897 \), which is option A.\begin{enumerate}[label=\Alph*.]
\item \( [-27.9, -27.84] \)

* -27.897, this is the correct option
\item \( [43.98, 44.01] \)

 44.006, which corresponds to two Order of Operations errors.
\item \( [-28.02, -27.98] \)

 -27.994, which corresponds to an Order of Operations error: not reading left-to-right for multiplication/division.
\item \( [44.1, 44.19] \)

 44.103, which corresponds to an Order of Operations error: multiplying by negative before squaring. For example: $(-3)^2 \neq -3^2$
\item \( \text{None of the above} \)

 You may have gotten this by making an unanticipated error. If you got a value that is not any of the others, please let the coordinator know so they can help you figure out what happened.
\end{enumerate}

\textbf{General Comment:} While you may remember (or were taught) PEMDAS is done in order, it is actually done as P/E/MD/AS. When we are at MD or AS, we read left to right.
}
\litem{
Choose the \textbf{smallest} set of Complex numbers that the number below belongs to.
\[ -\sqrt{\frac{196}{169}} + 16i^2 \]

The solution is \( \text{Rational} \), which is option C.\begin{enumerate}[label=\Alph*.]
\item \( \text{Not a Complex Number} \)

This is not a number. The only non-Complex number we know is dividing by 0 as this is not a number!
\item \( \text{Irrational} \)

These cannot be written as a fraction of Integers. Remember: $\pi$ is not an Integer!
\item \( \text{Rational} \)

* This is the correct option!
\item \( \text{Nonreal Complex} \)

This is a Complex number $(a+bi)$ that is not Real (has $i$ as part of the number).
\item \( \text{Pure Imaginary} \)

This is a Complex number $(a+bi)$ that \textbf{only} has an imaginary part like $2i$.
\end{enumerate}

\textbf{General Comment:} Be sure to simplify $i^2 = -1$. This may remove the imaginary portion for your number. If you are having trouble, you may want to look at the \textit{Subgroups of the Real Numbers} section.
}
\litem{
Simplify the expression below and choose the interval the simplification is contained within.
\[ 3 - 20 \div 10 * 4 - (16 * 7) \]

The solution is \( -117.000 \), which is option B.\begin{enumerate}[label=\Alph*.]
\item \( [110.5, 119.5] \)

 114.500, which corresponds to not distributing addition and subtraction correctly.
\item \( [-121, -116] \)

* -117.000, which is the correct option.
\item \( [-110.5, -105.5] \)

 -109.500, which corresponds to an Order of Operations error: not reading left-to-right for multiplication/division.
\item \( [-147, -144] \)

 -147.000, which corresponds to not distributing a negative correctly.
\item \( \text{None of the above} \)

 You may have gotten this by making an unanticipated error. If you got a value that is not any of the others, please let the coordinator know so they can help you figure out what happened.
\end{enumerate}

\textbf{General Comment:} While you may remember (or were taught) PEMDAS is done in order, it is actually done as P/E/MD/AS. When we are at MD or AS, we read left to right.
}
\litem{
Choose the \textbf{smallest} set of Real numbers that the number below belongs to.
\[ -\sqrt{\frac{1120}{10}} \]

The solution is \( \text{Irrational} \), which is option A.\begin{enumerate}[label=\Alph*.]
\item \( \text{Irrational} \)

* This is the correct option!
\item \( \text{Rational} \)

These are numbers that can be written as fraction of Integers (e.g., -2/3)
\item \( \text{Whole} \)

These are the counting numbers with 0 (0, 1, 2, 3, ...)
\item \( \text{Not a Real number} \)

These are Nonreal Complex numbers \textbf{OR} things that are not numbers (e.g., dividing by 0).
\item \( \text{Integer} \)

These are the negative and positive counting numbers (..., -3, -2, -1, 0, 1, 2, 3, ...)
\end{enumerate}

\textbf{General Comment:} First, you \textbf{NEED} to simplify the expression. This question simplifies to $-\sqrt{112}$. 
 
 Be sure you look at the simplified fraction and not just the decimal expansion. Numbers such as 13, 17, and 19 provide \textbf{long but repeating/terminating decimal expansions!} 
 
 The only ways to *not* be a Real number are: dividing by 0 or taking the square root of a negative number. 
 
 Irrational numbers are more than just square root of 3: adding or subtracting values from square root of 3 is also irrational.
}
\litem{
Simplify the expression below into the form $a+bi$. Then, choose the intervals that $a$ and $b$ belong to.
\[ \frac{9 + 55 i}{-6 + 8 i} \]

The solution is \( 3.86  - 4.02 i \), which is option A.\begin{enumerate}[label=\Alph*.]
\item \( a \in [3.5, 4.5] \text{ and } b \in [-5, -3.5] \)

* $3.86  - 4.02 i$, which is the correct option.
\item \( a \in [-5.5, -3.5] \text{ and } b \in [-3, -1] \)

 $-4.94  - 2.58 i$, which corresponds to forgetting to multiply the conjugate by the numerator and not computing the conjugate correctly.
\item \( a \in [385, 386.5] \text{ and } b \in [-5, -3.5] \)

 $386.00  - 4.02 i$, which corresponds to forgetting to multiply the conjugate by the numerator and using a plus instead of a minus in the denominator.
\item \( a \in [-2.5, -0.5] \text{ and } b \in [6, 8] \)

 $-1.50  + 6.88 i$, which corresponds to just dividing the first term by the first term and the second by the second.
\item \( a \in [3.5, 4.5] \text{ and } b \in [-402.5, -401] \)

 $3.86  - 402.00 i$, which corresponds to forgetting to multiply the conjugate by the numerator.
\end{enumerate}

\textbf{General Comment:} Multiply the numerator and denominator by the *conjugate* of the denominator, then simplify. For example, if we have $2+3i$, the conjugate is $2-3i$.
}
\end{enumerate}

\end{document}