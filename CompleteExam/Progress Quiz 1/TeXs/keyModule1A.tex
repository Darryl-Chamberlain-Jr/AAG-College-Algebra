\documentclass{extbook}[14pt]
\usepackage{multicol, enumerate, enumitem, hyperref, color, soul, setspace, parskip, fancyhdr, amssymb, amsthm, amsmath, bbm, latexsym, units, mathtools}
\everymath{\displaystyle}
\usepackage[headsep=0.5cm,headheight=0cm, left=1 in,right= 1 in,top= 1 in,bottom= 1 in]{geometry}
\usepackage{dashrule}  % Package to use the command below to create lines between items
\newcommand{\litem}[1]{\item #1

\rule{\textwidth}{0.4pt}}
\pagestyle{fancy}
\lhead{}
\chead{Answer Key for Progress Quiz 1 Version A}
\rhead{}
\lfoot{2654-6976}
\cfoot{}
\rfoot{Fall 2020}
\begin{document}
\textbf{This key should allow you to understand why you choose the option you did (beyond just getting a question right or wrong). \href{https://xronos.clas.ufl.edu/mac1105spring2020/courseDescriptionAndMisc/Exams/LearningFromResults}{More instructions on how to use this key can be found here}.}

\textbf{If you have a suggestion to make the keys better, \href{https://forms.gle/CZkbZmPbC9XALEE88}{please fill out the short survey here}.}

\textit{Note: This key is auto-generated and may contain issues and/or errors. The keys are reviewed after each exam to ensure grading is done accurately. If there are issues (like duplicate options), they are noted in the offline gradebook. The keys are a work-in-progress to give students as many resources to improve as possible.}

\rule{\textwidth}{0.4pt}

\begin{enumerate}\litem{
Simplify the expression below and choose the interval the simplification is contained within.
\[ 19 - 11 \div 12 * 17 - (6 * 18) \]
The solution is \( -104.583 \), which is option D.\begin{enumerate}[label=\Alph*.]
\item \( [-51.5, -42.5] \)

 -46.500, which corresponds to not distributing a negative correctly.
\item \( [122.95, 129.95] \)

 126.946, which corresponds to not distributing addition and subtraction correctly.
\item \( [-93.05, -87.05] \)

 -89.054, which corresponds to an Order of Operations error: not reading left-to-right for multiplication/division.
\item \( [-107.58, -102.58] \)

* -104.583, which is the correct option.
\item \( \text{None of the above} \)

 You may have gotten this by making an unanticipated error. If you got a value that is not any of the others, please let the coordinator know so they can help you figure out what happened.
\end{enumerate}

\textbf{General Comment:} While you may remember (or were taught) PEMDAS is done in order, it is actually done as P/E/MD/AS. When we are at MD or AS, we read left to right.
}
\litem{
Simplify the expression below into the form $a+bi$. Then, choose the intervals that $a$ and $b$ belong to.
\[ (2 + 4 i)(10 + 5 i) \]
The solution is \( 0 + 50 i \), which is option B.\begin{enumerate}[label=\Alph*.]
\item \( a \in [15, 24] \text{ and } b \in [16, 21] \)

 $20 + 20 i$, which corresponds to just multiplying the real terms to get the real part of the solution and the coefficients in the complex terms to get the complex part.
\item \( a \in [-3, 7] \text{ and } b \in [50, 51] \)

* $0 + 50 i$, which is the correct option.
\item \( a \in [36, 45] \text{ and } b \in [27, 34] \)

 $40 + 30 i$, which corresponds to adding a minus sign in the second term.
\item \( a \in [-3, 7] \text{ and } b \in [-53, -49] \)

 $0 - 50 i$, which corresponds to adding a minus sign in both terms.
\item \( a \in [36, 45] \text{ and } b \in [-33, -25] \)

 $40 - 30 i$, which corresponds to adding a minus sign in the first term.
\end{enumerate}

\textbf{General Comment:} You can treat $i$ as a variable and distribute. Just remember that $i^2=-1$, so you can continue to reduce after you distribute.
}
\litem{
Choose the \textbf{smallest} set of Complex numbers that the number below belongs to.
\[ \frac{-22}{16}+\sqrt{126} i \]
The solution is \( \text{Nonreal Complex} \), which is option D.\begin{enumerate}[label=\Alph*.]
\item \( \text{Not a Complex Number} \)

This is not a number. The only non-Complex number we know is dividing by 0 as this is not a number!
\item \( \text{Rational} \)

These are numbers that can be written as fraction of Integers (e.g., -2/3 + 5)
\item \( \text{Irrational} \)

These cannot be written as a fraction of Integers. Remember: $\pi$ is not an Integer!
\item \( \text{Nonreal Complex} \)

* This is the correct option!
\item \( \text{Pure Imaginary} \)

This is a Complex number $(a+bi)$ that \textbf{only} has an imaginary part like $2i$.
\end{enumerate}

\textbf{General Comment:} Be sure to simplify $i^2 = -1$. This may remove the imaginary portion for your number. If you are having trouble, you may want to look at the \textit{Subgroups of the Real Numbers} section.
}
\litem{
Choose the \textbf{smallest} set of Real numbers that the number below belongs to.
\[ -\sqrt{\frac{40000}{100}} \]
The solution is \( \text{Integer} \), which is option D.\begin{enumerate}[label=\Alph*.]
\item \( \text{Whole} \)

These are the counting numbers with 0 (0, 1, 2, 3, ...)
\item \( \text{Irrational} \)

These cannot be written as a fraction of Integers.
\item \( \text{Rational} \)

These are numbers that can be written as fraction of Integers (e.g., -2/3)
\item \( \text{Integer} \)

* This is the correct option!
\item \( \text{Not a Real number} \)

These are Nonreal Complex numbers \textbf{OR} things that are not numbers (e.g., dividing by 0).
\end{enumerate}

\textbf{General Comment:} First, you \textbf{NEED} to simplify the expression. This question simplifies to $-200$. 
 
 Be sure you look at the simplified fraction and not just the decimal expansion. Numbers such as 13, 17, and 19 provide \textbf{long but repeating/terminating decimal expansions!} 
 
 The only ways to *not* be a Real number are: dividing by 0 or taking the square root of a negative number. 
 
 Irrational numbers are more than just square root of 3: adding or subtracting values from square root of 3 is also irrational.
}
\litem{
Simplify the expression below into the form $a+bi$. Then, choose the intervals that $a$ and $b$ belong to.
\[ \frac{9 - 33 i}{7 - 6 i} \]
The solution is \( 3.07  - 2.08 i \), which is option E.\begin{enumerate}[label=\Alph*.]
\item \( a \in [2.5, 4] \text{ and } b \in [-178.5, -176.5] \)

 $3.07  - 177.00 i$, which corresponds to forgetting to multiply the conjugate by the numerator.
\item \( a \in [260.5, 262] \text{ and } b \in [-3, -1.5] \)

 $261.00  - 2.08 i$, which corresponds to forgetting to multiply the conjugate by the numerator and using a plus instead of a minus in the denominator.
\item \( a \in [0.5, 2.5] \text{ and } b \in [4, 7] \)

 $1.29  + 5.50 i$, which corresponds to just dividing the first term by the first term and the second by the second.
\item \( a \in [-2.5, -1] \text{ and } b \in [-4.5, -2.5] \)

 $-1.59  - 3.35 i$, which corresponds to forgetting to multiply the conjugate by the numerator and not computing the conjugate correctly.
\item \( a \in [2.5, 4] \text{ and } b \in [-3, -1.5] \)

* $3.07  - 2.08 i$, which is the correct option.
\end{enumerate}

\textbf{General Comment:} Multiply the numerator and denominator by the *conjugate* of the denominator, then simplify. For example, if we have $2+3i$, the conjugate is $2-3i$.
}
\litem{
Simplify the expression below and choose the interval the simplification is contained within.
\[ 2 - 17^2 + 10 \div 4 * 18 \div 19 \]
The solution is \( -284.632 \), which is option A.\begin{enumerate}[label=\Alph*.]
\item \( [-285.19, -284.38] \)

* -284.632, this is the correct option
\item \( [-288.43, -285.77] \)

 -286.993, which corresponds to an Order of Operations error: not reading left-to-right for multiplication/division.
\item \( [289.87, 291.34] \)

 291.007, which corresponds to two Order of Operations errors.
\item \( [293.29, 293.53] \)

 293.368, which corresponds to an Order of Operations error: multiplying by negative before squaring. For example: $(-3)^2 \neq -3^2$
\item \( \text{None of the above} \)

 You may have gotten this by making an unanticipated error. If you got a value that is not any of the others, please let the coordinator know so they can help you figure out what happened.
\end{enumerate}

\textbf{General Comment:} While you may remember (or were taught) PEMDAS is done in order, it is actually done as P/E/MD/AS. When we are at MD or AS, we read left to right.
}
\litem{
Choose the \textbf{smallest} set of Complex numbers that the number below belongs to.
\[ -\sqrt{\frac{1232}{8}}+7i^2 \]
The solution is \( \text{Irrational} \), which is option D.\begin{enumerate}[label=\Alph*.]
\item \( \text{Not a Complex Number} \)

This is not a number. The only non-Complex number we know is dividing by 0 as this is not a number!
\item \( \text{Nonreal Complex} \)

This is a Complex number $(a+bi)$ that is not Real (has $i$ as part of the number).
\item \( \text{Pure Imaginary} \)

This is a Complex number $(a+bi)$ that \textbf{only} has an imaginary part like $2i$.
\item \( \text{Irrational} \)

* This is the correct option!
\item \( \text{Rational} \)

These are numbers that can be written as fraction of Integers (e.g., -2/3 + 5)
\end{enumerate}

\textbf{General Comment:} Be sure to simplify $i^2 = -1$. This may remove the imaginary portion for your number. If you are having trouble, you may want to look at the \textit{Subgroups of the Real Numbers} section.
}
\litem{
Choose the \textbf{smallest} set of Real numbers that the number below belongs to.
\[ -\sqrt{\frac{-600}{10}} \]
The solution is \( \text{Not a Real number} \), which is option A.\begin{enumerate}[label=\Alph*.]
\item \( \text{Not a Real number} \)

* This is the correct option!
\item \( \text{Whole} \)

These are the counting numbers with 0 (0, 1, 2, 3, ...)
\item \( \text{Irrational} \)

These cannot be written as a fraction of Integers.
\item \( \text{Rational} \)

These are numbers that can be written as fraction of Integers (e.g., -2/3)
\item \( \text{Integer} \)

These are the negative and positive counting numbers (..., -3, -2, -1, 0, 1, 2, 3, ...)
\end{enumerate}

\textbf{General Comment:} First, you \textbf{NEED} to simplify the expression. This question simplifies to $-\sqrt{60} i$. 
 
 Be sure you look at the simplified fraction and not just the decimal expansion. Numbers such as 13, 17, and 19 provide \textbf{long but repeating/terminating decimal expansions!} 
 
 The only ways to *not* be a Real number are: dividing by 0 or taking the square root of a negative number. 
 
 Irrational numbers are more than just square root of 3: adding or subtracting values from square root of 3 is also irrational.
}
\litem{
Simplify the expression below into the form $a+bi$. Then, choose the intervals that $a$ and $b$ belong to.
\[ (-7 - 10 i)(-5 - 6 i) \]
The solution is \( -25 + 92 i \), which is option A.\begin{enumerate}[label=\Alph*.]
\item \( a \in [-28, -23] \text{ and } b \in [91, 95] \)

* $-25 + 92 i$, which is the correct option.
\item \( a \in [-28, -23] \text{ and } b \in [-92, -85] \)

 $-25 - 92 i$, which corresponds to adding a minus sign in both terms.
\item \( a \in [95, 101] \text{ and } b \in [-15, -3] \)

 $95 - 8 i$, which corresponds to adding a minus sign in the first term.
\item \( a \in [33, 41] \text{ and } b \in [60, 61] \)

 $35 + 60 i$, which corresponds to just multiplying the real terms to get the real part of the solution and the coefficients in the complex terms to get the complex part.
\item \( a \in [95, 101] \text{ and } b \in [4, 10] \)

 $95 + 8 i$, which corresponds to adding a minus sign in the second term.
\end{enumerate}

\textbf{General Comment:} You can treat $i$ as a variable and distribute. Just remember that $i^2=-1$, so you can continue to reduce after you distribute.
}
\litem{
Simplify the expression below into the form $a+bi$. Then, choose the intervals that $a$ and $b$ belong to.
\[ \frac{36 + 22 i}{-7 - 8 i} \]
The solution is \( -3.79  + 1.19 i \), which is option C.\begin{enumerate}[label=\Alph*.]
\item \( a \in [-428.5, -427.5] \text{ and } b \in [1, 1.5] \)

 $-428.00  + 1.19 i$, which corresponds to forgetting to multiply the conjugate by the numerator and using a plus instead of a minus in the denominator.
\item \( a \in [-2, 0] \text{ and } b \in [-4, -3.5] \)

 $-0.67  - 3.91 i$, which corresponds to forgetting to multiply the conjugate by the numerator and not computing the conjugate correctly.
\item \( a \in [-4.5, -2.5] \text{ and } b \in [1, 1.5] \)

* $-3.79  + 1.19 i$, which is the correct option.
\item \( a \in [-4.5, -2.5] \text{ and } b \in [133.5, 135] \)

 $-3.79  + 134.00 i$, which corresponds to forgetting to multiply the conjugate by the numerator.
\item \( a \in [-6, -4.5] \text{ and } b \in [-3, -2.5] \)

 $-5.14  - 2.75 i$, which corresponds to just dividing the first term by the first term and the second by the second.
\end{enumerate}

\textbf{General Comment:} Multiply the numerator and denominator by the *conjugate* of the denominator, then simplify. For example, if we have $2+3i$, the conjugate is $2-3i$.
}
\end{enumerate}

\end{document}