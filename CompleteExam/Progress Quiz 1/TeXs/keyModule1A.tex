\documentclass{extbook}[14pt]
\usepackage{multicol, enumerate, enumitem, hyperref, color, soul, setspace, parskip, fancyhdr, amssymb, amsthm, amsmath, latexsym, units, mathtools}
\everymath{\displaystyle}
\usepackage[headsep=0.5cm,headheight=0cm, left=1 in,right= 1 in,top= 1 in,bottom= 1 in]{geometry}
\usepackage{dashrule}  % Package to use the command below to create lines between items
\newcommand{\litem}[1]{\item #1

\rule{\textwidth}{0.4pt}}
\pagestyle{fancy}
\lhead{}
\chead{Answer Key for Progress Quiz 1 Version A}
\rhead{}
\lfoot{4082-7053}
\cfoot{}
\rfoot{test}
\begin{document}
\textbf{This key should allow you to understand why you choose the option you did (beyond just getting a question right or wrong). \href{https://xronos.clas.ufl.edu/mac1105spring2020/courseDescriptionAndMisc/Exams/LearningFromResults}{More instructions on how to use this key can be found here}.}

\textbf{If you have a suggestion to make the keys better, \href{https://forms.gle/CZkbZmPbC9XALEE88}{please fill out the short survey here}.}

\textit{Note: This key is auto-generated and may contain issues and/or errors. The keys are reviewed after each exam to ensure grading is done accurately. If there are issues (like duplicate options), they are noted in the offline gradebook. The keys are a work-in-progress to give students as many resources to improve as possible.}

\rule{\textwidth}{0.4pt}

\begin{enumerate}\litem{
Choose the \textbf{smallest} set of Real numbers that the number below belongs to.
\[ -\sqrt{\frac{9216}{144}} \]The solution is \( \text{Integer} \), which is option E.\begin{enumerate}[label=\Alph*.]
\item \( \text{Irrational} \)

These cannot be written as a fraction of Integers.
\item \( \text{Whole} \)

These are the counting numbers with 0 (0, 1, 2, 3, ...)
\item \( \text{Rational} \)

These are numbers that can be written as fraction of Integers (e.g., -2/3)
\item \( \text{Not a Real number} \)

These are Nonreal Complex numbers \textbf{OR} things that are not numbers (e.g., dividing by 0).
\item \( \text{Integer} \)

* This is the correct option!
\end{enumerate}

\textbf{General Comment:} First, you \textbf{NEED} to simplify the expression. This question simplifies to $-96$. 
 
 Be sure you look at the simplified fraction and not just the decimal expansion. Numbers such as 13, 17, and 19 provide \textbf{long but repeating/terminating decimal expansions!} 
 
 The only ways to *not* be a Real number are: dividing by 0 or taking the square root of a negative number. 
 
 Irrational numbers are more than just square root of 3: adding or subtracting values from square root of 3 is also irrational.
}
\litem{
Simplify the expression below and choose the interval the simplification is contained within.
\[ 1 - 18 \div 8 * 13 - (15 * 7) \]The solution is \( -133.250 \), which is option C.\begin{enumerate}[label=\Alph*.]
\item \( [-105.17, -98.17] \)

 -104.173, which corresponds to an Order of Operations error: not reading left-to-right for multiplication/division.
\item \( [103.83, 106.83] \)

 105.827, which corresponds to not distributing addition and subtraction correctly.
\item \( [-133.25, -127.25] \)

* -133.250, which is the correct option.
\item \( [-306.75, -297.75] \)

 -302.750, which corresponds to not distributing a negative correctly.
\item \( \text{None of the above} \)

 You may have gotten this by making an unanticipated error. If you got a value that is not any of the others, please let the coordinator know so they can help you figure out what happened.
\end{enumerate}

\textbf{General Comment:} While you may remember (or were taught) PEMDAS is done in order, it is actually done as P/E/MD/AS. When we are at MD or AS, we read left to right.
}
\litem{
Simplify the expression below and choose the interval the simplification is contained within.
\[ 10 - 7 \div 5 * 12 - (14 * 4) \]The solution is \( -62.800 \), which is option C.\begin{enumerate}[label=\Alph*.]
\item \( [-49.12, -44.12] \)

 -46.117, which corresponds to an Order of Operations error: not reading left-to-right for multiplication/division.
\item \( [-85.2, -81.2] \)

 -83.200, which corresponds to not distributing a negative correctly.
\item \( [-64.8, -61.8] \)

* -62.800, which is the correct option.
\item \( [62.88, 66.88] \)

 65.883, which corresponds to not distributing addition and subtraction correctly.
\item \( \text{None of the above} \)

 You may have gotten this by making an unanticipated error. If you got a value that is not any of the others, please let the coordinator know so they can help you figure out what happened.
\end{enumerate}

\textbf{General Comment:} While you may remember (or were taught) PEMDAS is done in order, it is actually done as P/E/MD/AS. When we are at MD or AS, we read left to right.
}
\litem{
Choose the \textbf{smallest} set of Complex numbers that the number below belongs to.
\[ \sqrt{\frac{1584}{12}}+\sqrt{156} i \]The solution is \( \text{Nonreal Complex} \), which is option A.\begin{enumerate}[label=\Alph*.]
\item \( \text{Nonreal Complex} \)

* This is the correct option!
\item \( \text{Pure Imaginary} \)

This is a Complex number $(a+bi)$ that \textbf{only} has an imaginary part like $2i$.
\item \( \text{Irrational} \)

These cannot be written as a fraction of Integers. Remember: $\pi$ is not an Integer!
\item \( \text{Not a Complex Number} \)

This is not a number. The only non-Complex number we know is dividing by 0 as this is not a number!
\item \( \text{Rational} \)

These are numbers that can be written as fraction of Integers (e.g., -2/3 + 5)
\end{enumerate}

\textbf{General Comment:} Be sure to simplify $i^2 = -1$. This may remove the imaginary portion for your number. If you are having trouble, you may want to look at the \textit{Subgroups of the Real Numbers} section.
}
\litem{
Simplify the expression below into the form $a+bi$. Then, choose the intervals that $a$ and $b$ belong to.
\[ (3 + 9 i)(5 - 4 i) \]The solution is \( 51 + 33 i \), which is option D.\begin{enumerate}[label=\Alph*.]
\item \( a \in [49, 55] \text{ and } b \in [-35.3, -32.9] \)

 $51 - 33 i$, which corresponds to adding a minus sign in both terms.
\item \( a \in [-26, -18] \text{ and } b \in [56.9, 59.9] \)

 $-21 + 57 i$, which corresponds to adding a minus sign in the second term.
\item \( a \in [12, 17] \text{ and } b \in [-37.1, -35.5] \)

 $15 - 36 i$, which corresponds to just multiplying the real terms to get the real part of the solution and the coefficients in the complex terms to get the complex part.
\item \( a \in [49, 55] \text{ and } b \in [32.3, 33.6] \)

* $51 + 33 i$, which is the correct option.
\item \( a \in [-26, -18] \text{ and } b \in [-57.3, -53.8] \)

 $-21 - 57 i$, which corresponds to adding a minus sign in the first term.
\end{enumerate}

\textbf{General Comment:} You can treat $i$ as a variable and distribute. Just remember that $i^2=-1$, so you can continue to reduce after you distribute.
}
\litem{
Simplify the expression below into the form $a+bi$. Then, choose the intervals that $a$ and $b$ belong to.
\[ \frac{63 + 11 i}{2 - 8 i} \]The solution is \( 0.56  + 7.74 i \), which is option B.\begin{enumerate}[label=\Alph*.]
\item \( a \in [37.5, 39] \text{ and } b \in [7, 8.5] \)

 $38.00  + 7.74 i$, which corresponds to forgetting to multiply the conjugate by the numerator and using a plus instead of a minus in the denominator.
\item \( a \in [-0.5, 1] \text{ and } b \in [7, 8.5] \)

* $0.56  + 7.74 i$, which is the correct option.
\item \( a \in [30, 33] \text{ and } b \in [-1.5, -1] \)

 $31.50  - 1.38 i$, which corresponds to just dividing the first term by the first term and the second by the second.
\item \( a \in [-0.5, 1] \text{ and } b \in [525.5, 527] \)

 $0.56  + 526.00 i$, which corresponds to forgetting to multiply the conjugate by the numerator.
\item \( a \in [1.5, 4.5] \text{ and } b \in [-7.5, -6.5] \)

 $3.15  - 7.09 i$, which corresponds to forgetting to multiply the conjugate by the numerator and not computing the conjugate correctly.
\end{enumerate}

\textbf{General Comment:} Multiply the numerator and denominator by the *conjugate* of the denominator, then simplify. For example, if we have $2+3i$, the conjugate is $2-3i$.
}
\litem{
Choose the \textbf{smallest} set of Complex numbers that the number below belongs to.
\[ \frac{-13}{20}+\sqrt{234} i \]The solution is \( \text{Nonreal Complex} \), which is option B.\begin{enumerate}[label=\Alph*.]
\item \( \text{Pure Imaginary} \)

This is a Complex number $(a+bi)$ that \textbf{only} has an imaginary part like $2i$.
\item \( \text{Nonreal Complex} \)

* This is the correct option!
\item \( \text{Not a Complex Number} \)

This is not a number. The only non-Complex number we know is dividing by 0 as this is not a number!
\item \( \text{Rational} \)

These are numbers that can be written as fraction of Integers (e.g., -2/3 + 5)
\item \( \text{Irrational} \)

These cannot be written as a fraction of Integers. Remember: $\pi$ is not an Integer!
\end{enumerate}

\textbf{General Comment:} Be sure to simplify $i^2 = -1$. This may remove the imaginary portion for your number. If you are having trouble, you may want to look at the \textit{Subgroups of the Real Numbers} section.
}
\litem{
Simplify the expression below into the form $a+bi$. Then, choose the intervals that $a$ and $b$ belong to.
\[ (4 + 9 i)(-6 - 3 i) \]The solution is \( 3 - 66 i \), which is option E.\begin{enumerate}[label=\Alph*.]
\item \( a \in [-57, -48] \text{ and } b \in [-45, -39] \)

 $-51 - 42 i$, which corresponds to adding a minus sign in the second term.
\item \( a \in [3, 5] \text{ and } b \in [60, 69] \)

 $3 + 66 i$, which corresponds to adding a minus sign in both terms.
\item \( a \in [-29, -17] \text{ and } b \in [-32, -22] \)

 $-24 - 27 i$, which corresponds to just multiplying the real terms to get the real part of the solution and the coefficients in the complex terms to get the complex part.
\item \( a \in [-57, -48] \text{ and } b \in [38, 44] \)

 $-51 + 42 i$, which corresponds to adding a minus sign in the first term.
\item \( a \in [3, 5] \text{ and } b \in [-68, -62] \)

* $3 - 66 i$, which is the correct option.
\end{enumerate}

\textbf{General Comment:} You can treat $i$ as a variable and distribute. Just remember that $i^2=-1$, so you can continue to reduce after you distribute.
}
\litem{
Simplify the expression below into the form $a+bi$. Then, choose the intervals that $a$ and $b$ belong to.
\[ \frac{-63 - 44 i}{-5 + i} \]The solution is \( 10.42  + 10.88 i \), which is option E.\begin{enumerate}[label=\Alph*.]
\item \( a \in [270.5, 272.5] \text{ and } b \in [10, 11.5] \)

 $271.00  + 10.88 i$, which corresponds to forgetting to multiply the conjugate by the numerator and using a plus instead of a minus in the denominator.
\item \( a \in [9, 11] \text{ and } b \in [280.5, 284] \)

 $10.42  + 283.00 i$, which corresponds to forgetting to multiply the conjugate by the numerator.
\item \( a \in [12, 13.5] \text{ and } b \in [-44.5, -43] \)

 $12.60  - 44.00 i$, which corresponds to just dividing the first term by the first term and the second by the second.
\item \( a \in [13, 15.5] \text{ and } b \in [5.5, 6.5] \)

 $13.81  + 6.04 i$, which corresponds to forgetting to multiply the conjugate by the numerator and not computing the conjugate correctly.
\item \( a \in [9, 11] \text{ and } b \in [10, 11.5] \)

* $10.42  + 10.88 i$, which is the correct option.
\end{enumerate}

\textbf{General Comment:} Multiply the numerator and denominator by the *conjugate* of the denominator, then simplify. For example, if we have $2+3i$, the conjugate is $2-3i$.
}
\litem{
Choose the \textbf{smallest} set of Real numbers that the number below belongs to.
\[ -\sqrt{\frac{279841}{529}} \]The solution is \( \text{Integer} \), which is option C.\begin{enumerate}[label=\Alph*.]
\item \( \text{Irrational} \)

These cannot be written as a fraction of Integers.
\item \( \text{Not a Real number} \)

These are Nonreal Complex numbers \textbf{OR} things that are not numbers (e.g., dividing by 0).
\item \( \text{Integer} \)

* This is the correct option!
\item \( \text{Whole} \)

These are the counting numbers with 0 (0, 1, 2, 3, ...)
\item \( \text{Rational} \)

These are numbers that can be written as fraction of Integers (e.g., -2/3)
\end{enumerate}

\textbf{General Comment:} First, you \textbf{NEED} to simplify the expression. This question simplifies to $-529$. 
 
 Be sure you look at the simplified fraction and not just the decimal expansion. Numbers such as 13, 17, and 19 provide \textbf{long but repeating/terminating decimal expansions!} 
 
 The only ways to *not* be a Real number are: dividing by 0 or taking the square root of a negative number. 
 
 Irrational numbers are more than just square root of 3: adding or subtracting values from square root of 3 is also irrational.
}
\end{enumerate}

\end{document}