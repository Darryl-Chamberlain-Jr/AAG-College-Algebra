\documentclass{extbook}[14pt]
\usepackage{multicol, enumerate, enumitem, hyperref, color, soul, setspace, parskip, fancyhdr, amssymb, amsthm, amsmath, latexsym, units, mathtools}
\everymath{\displaystyle}
\usepackage[headsep=0.5cm,headheight=0cm, left=1 in,right= 1 in,top= 1 in,bottom= 1 in]{geometry}
\usepackage{dashrule}  % Package to use the command below to create lines between items
\newcommand{\litem}[1]{\item #1

\rule{\textwidth}{0.4pt}}
\pagestyle{fancy}
\lhead{}
\chead{Answer Key for Module8 Version ALL}
\rhead{}
\lfoot{2439-3803}
\cfoot{}
\rfoot{test}
\begin{document}
\textbf{This key should allow you to understand why you choose the option you did (beyond just getting a question right or wrong). \href{https://xronos.clas.ufl.edu/mac1105spring2020/courseDescriptionAndMisc/Exams/LearningFromResults}{More instructions on how to use this key can be found here}.}

\textbf{If you have a suggestion to make the keys better, \href{https://forms.gle/CZkbZmPbC9XALEE88}{please fill out the short survey here}.}

\textit{Note: This key is auto-generated and may contain issues and/or errors. The keys are reviewed after each exam to ensure grading is done accurately. If there are issues (like duplicate options), they are noted in the offline gradebook. The keys are a work-in-progress to give students as many resources to improve as possible.}

\rule{\textwidth}{0.4pt}

\begin{enumerate}\litem{
Solve the equation below for $x$.
\[ 2^{5x+2} = \left(\frac{1}{9}\right)^{4x-5} \]The solution is \( x = 0.783 \).\begin{enumerate}[label=\Alph*.]
\textbf{Plausible alternative answers include:}$x = -0.571$, which corresponds to distributing the $\ln(base)$ to the first term of the exponent only.
* $x = 0.783$, which is the correct option.
$x = -7.000$, which corresponds to solving the numerators as equal while ignoring the bases are different.
$x = 9.600$, which corresponds to distributing the $\ln(base)$ to the second term of the exponent only.
This corresponds to believing there is no solution since the bases are not powers of each other.
\end{enumerate}

\textbf{General Comment:} \textbf{General Comments:} This question was written so that the bases could not be written the same. You will need to take the log of both sides.
}
\litem{
Solve the equation below for $x$.
\[  14 = \sqrt[5]{\frac{29}{e^{4x}}} \]The solution is \( x = -2.457, \text{ which does not fit in any of the interval options.} \).\begin{enumerate}[label=\Alph*.]
\textbf{Plausible alternative answers include:}$x = 2.457$, which is the negative of the correct solution.
$x = -18.342$, which corresponds to thinking you don't need to take the natural log of both sides before reducing, as if the right side already has a natural log.
$x = -0.478$, which corresponds to treating any root as a square root.
This corresponds to believing you cannot solve the equation.
* $x = -2.457$ is the correct solution and does not fit in any of the other intervals.
\end{enumerate}

\textbf{General Comment:} \textbf{General Comments}: After using the properties of logarithmic functions to break up the right-hand side, use $\ln(e) = 1$ to reduce the question to a linear function to solve. You can put $\ln(29)$ into a calculator if you are having trouble.
}
\litem{
Describe the Domain of the function below.
\[ f(x) = -\log_2{(x+2)}+7 \]The solution is \( (-2, \infty) \).\begin{enumerate}[label=\Alph*.]
\textbf{Plausible alternative answers include:}* $(-2, \infty)$, which is the correct option.
$[7, \infty)$, which corresponds to using the vertical shift when shifting the Domain AND including the endpoint.
$(-\infty, 2)$, which corresponds to flipping the Domain. Remember: the general for is $a*\log(x-h)+k$, \textbf{where $a$ does not affect the domain}.
$(-\infty, -7]$, which corresponds to using the negative vertical shift AND including the endpoint AND flipping the domain.
This corresponds to thinking of the range of the log function (or the domain of the exponential function).
\end{enumerate}

\textbf{General Comment:} \textbf{General Comments}: The domain of a basic logarithmic function is $(0, \infty)$ and the Range is $(-\infty, \infty)$. We can use shifts when finding the Domain, but the Range will always be all Real numbers.
}
\litem{
Solve the equation below for $x$.
\[ \log_{4}{(3x+8)}+6 = 2 \]The solution is \( x = -2.665 \).\begin{enumerate}[label=\Alph*.]
\textbf{Plausible alternative answers include:}$x = 82.667$, which corresponds to reversing the base and exponent when converting.
* $x = -2.665$, which is the correct option.
$x = 2.667$, which corresponds to ignoring the vertical shift when converting to exponential form.
$x = 88.000$, which corresponds to reversing the base and exponent when converting and reversing the value with $x$.
Corresponds to believing a negative coefficient within the log equation means there is no Real solution.
\end{enumerate}

\textbf{General Comment:} \textbf{General Comments:} First, get the equation in the form $\log_b{(cx+d)} = a$. Then, convert to $b^a = cx+d$ and solve.
}
\litem{
Describe the Domain of the function below.
\[ f(x) = -\log_2{(x-6)}+7 \]The solution is \( (6, \infty) \).\begin{enumerate}[label=\Alph*.]
\textbf{Plausible alternative answers include:}$(-\infty, -7]$, which corresponds to using the negative vertical shift AND including the endpoint AND flipping the domain.
$[7, \infty)$, which corresponds to using the vertical shift when shifting the Domain AND including the endpoint.
* $(6, \infty)$, which is the correct option.
$(-\infty, -6)$, which corresponds to flipping the Domain. Remember: the general for is $a*\log(x-h)+k$, \textbf{where $a$ does not affect the domain}.
This corresponds to thinking of the range of the log function (or the domain of the exponential function).
\end{enumerate}

\textbf{General Comment:} \textbf{General Comments}: The domain of a basic logarithmic function is $(0, \infty)$ and the Range is $(-\infty, \infty)$. We can use shifts when finding the Domain, but the Range will always be all Real numbers.
}
\litem{
Solve the equation below for $x$.
\[ 3^{-5x+3} = 125^{-2x-4} \]The solution is \( x = -5.430 \).\begin{enumerate}[label=\Alph*.]
\textbf{Plausible alternative answers include:}$x = 7.536$, which corresponds to distributing the $\ln(base)$ to the second term of the exponent only.
* $x = -5.430$, which is the correct option.
$x = -1.681$, which corresponds to distributing the $\ln(base)$ to the first term of the exponent only.
$x = 2.333$, which corresponds to solving the numerators as equal while ignoring the bases are different.
This corresponds to believing there is no solution since the bases are not powers of each other.
\end{enumerate}

\textbf{General Comment:} \textbf{General Comments:} This question was written so that the bases could not be written the same. You will need to take the log of both sides.
}
\litem{
Describe the Range of the function below.
\[ f(x) = e^{x-1}-4 \]The solution is \( (-4, \infty) \).\begin{enumerate}[label=\Alph*.]
\textbf{Plausible alternative answers include:}* $(-4, \infty)$, which is the correct option.
$(-\infty, 4]$, which corresponds to using the negative vertical shift AND flipping the Range interval AND including the endpoint.
$[-4, \infty)$, which corresponds to including the endpoint.
$(-\infty, 4)$, which corresponds to using the negative vertical shift AND flipping the Range interval.
This corresponds to confusing range of an exponential function with the domain of an exponential function.
\end{enumerate}

\textbf{General Comment:} \textbf{General Comments}: Domain of a basic exponential function is $(-\infty, \infty)$ while the Range is $(0, \infty)$. We can shift these intervals [and even flip when $a<0$!] to find the new Domain/Range.
}
\litem{
Solve the equation below for $x$.
\[  11 = \ln{\sqrt[6]{\frac{30}{e^{3x}}}} \]The solution is \( x = -20.866, \text{ which does not fit in any of the interval options.} \).\begin{enumerate}[label=\Alph*.]
\textbf{Plausible alternative answers include:}$x = -6.200$, which corresponds to treating any root as a square root.
$x = -5.930$, which corresponds to thinking you need to take the natural log of the left side before reducing.
$x = 20.866$, which is the negative of the correct solution.
This corresponds to believing you cannot solve the equation.
*$x = -20.866$ is the correct solution and does not fit in any of the other intervals.
\end{enumerate}

\textbf{General Comment:} \textbf{General Comments}: After using the properties of logarithmic functions to break up the right-hand side, use $\ln(e) = 1$ to reduce the question to a linear function to solve. You can put $\ln(30)$ into a calculator if you are having trouble.
}
\litem{
Solve the equation below for $x$.
\[ \log_{4}{(2x+6)}+4 = 3 \]The solution is \( x = -2.875 \).\begin{enumerate}[label=\Alph*.]
\textbf{Plausible alternative answers include:}$x = 3.500$, which corresponds to reversing the base and exponent when converting and reversing the value with $x$.
$x = -2.500$, which corresponds to reversing the base and exponent when converting.
$x = 29.000$, which corresponds to ignoring the vertical shift when converting to exponential form.
* $x = -2.875$, which is the correct option.
Corresponds to believing a negative coefficient within the log equation means there is no Real solution.
\end{enumerate}

\textbf{General Comment:} \textbf{General Comments:} First, get the equation in the form $\log_b{(cx+d)} = a$. Then, convert to $b^a = cx+d$ and solve.
}
\litem{
Describe the Domain of the function below.
\[ f(x) = -e^{x-9}-3 \]The solution is \( (-\infty, \infty) \).\begin{enumerate}[label=\Alph*.]
\textbf{Plausible alternative answers include:}$(-\infty, -3)$, which corresponds to using the correct vertical shift *if we wanted the Range*.
$(3, \infty)$, which corresponds to using the negative vertical shift AND flipping the Range interval.
$(-\infty, -3]$, which corresponds to using the correct vertical shift *if we wanted the Range* AND including the endpoint.
$[3, \infty)$, which corresponds to using the negative vertical shift AND flipping the Range interval AND including the endpoint.
* This is the correct option.
\end{enumerate}

\textbf{General Comment:} \textbf{General Comments}: Domain of a basic exponential function is $(-\infty, \infty)$ while the Range is $(0, \infty)$. We can shift these intervals [and even flip when $a<0$!] to find the new Domain/Range.
}
\litem{
Solve the equation below for $x$.
\[ 4^{-4x-4} = \left(\frac{1}{25}\right)^{2x-2} \]The solution is \( x = 13.425 \).\begin{enumerate}[label=\Alph*.]
\textbf{Plausible alternative answers include:}$x = -1.997$, which corresponds to distributing the $\ln(base)$ to the second term of the exponent only.
$x = -0.333$, which corresponds to solving the numerators as equal while ignoring the bases are different.
* $x = 13.425$, which is the correct option.
$x = 2.241$, which corresponds to distributing the $\ln(base)$ to the first term of the exponent only.
This corresponds to believing there is no solution since the bases are not powers of each other.
\end{enumerate}

\textbf{General Comment:} \textbf{General Comments:} This question was written so that the bases could not be written the same. You will need to take the log of both sides.
}
\litem{
Solve the equation below for $x$.
\[  14 = \ln{\sqrt[7]{\frac{5}{e^{4x}}}} \]The solution is \( x = -24.098 \).\begin{enumerate}[label=\Alph*.]
\textbf{Plausible alternative answers include:}* $x = -24.098$, which is the correct option.
$x = -5.021$, which corresponds to thinking you need to take the natural log of on the left before reducing.
$x = -6.598$, which corresponds to treating any root as a square root.
This corresponds to believing you cannot solve the equation.
This corresponds to making an unexpected error.
\end{enumerate}

\textbf{General Comment:} \textbf{General Comments}: After using the properties of logarithmic functions to break up the right-hand side, use $\ln(e) = 1$ to reduce the question to a linear function to solve. You can put $\ln(5)$ into a calculator if you are having trouble.
}
\litem{
Describe the Domain of the function below.
\[ f(x) = -\log_2{(x-8)}+4 \]The solution is \( (8, \infty) \).\begin{enumerate}[label=\Alph*.]
\textbf{Plausible alternative answers include:}$[4, \infty)$, which corresponds to using the vertical shift when shifting the Domain AND including the endpoint.
$(-\infty, -8)$, which corresponds to flipping the Domain. Remember: the general for is $a*\log(x-h)+k$, \textbf{where $a$ does not affect the domain}.
* $(8, \infty)$, which is the correct option.
$(-\infty, -4]$, which corresponds to using the negative vertical shift AND including the endpoint AND flipping the domain.
This corresponds to thinking of the range of the log function (or the domain of the exponential function).
\end{enumerate}

\textbf{General Comment:} \textbf{General Comments}: The domain of a basic logarithmic function is $(0, \infty)$ and the Range is $(-\infty, \infty)$. We can use shifts when finding the Domain, but the Range will always be all Real numbers.
}
\litem{
Solve the equation below for $x$.
\[ \log_{2}{(-3x+6)}+5 = 2 \]The solution is \( x = 1.958 \).\begin{enumerate}[label=\Alph*.]
\textbf{Plausible alternative answers include:}$x = 0.667$, which corresponds to ignoring the vertical shift when converting to exponential form.
$x = -1.000$, which corresponds to reversing the base and exponent when converting.
$x = -5.000$, which corresponds to reversing the base and exponent when converting and reversing the value with $x$.
* $x = 1.958$, which is the correct option.
Corresponds to believing a negative coefficient within the log equation means there is no Real solution.
\end{enumerate}

\textbf{General Comment:} \textbf{General Comments:} First, get the equation in the form $\log_b{(cx+d)} = a$. Then, convert to $b^a = cx+d$ and solve.
}
\litem{
Describe the Domain of the function below.
\[ f(x) = -\log_2{(x+4)}+7 \]The solution is \( (-4, \infty) \).\begin{enumerate}[label=\Alph*.]
\textbf{Plausible alternative answers include:}* $(-4, \infty)$, which is the correct option.
$[7, \infty)$, which corresponds to using the vertical shift when shifting the Domain AND including the endpoint.
$(-\infty, 4)$, which corresponds to flipping the Domain. Remember: the general for is $a*\log(x-h)+k$, \textbf{where $a$ does not affect the domain}.
$(-\infty, -7]$, which corresponds to using the negative vertical shift AND including the endpoint AND flipping the domain.
This corresponds to thinking of the range of the log function (or the domain of the exponential function).
\end{enumerate}

\textbf{General Comment:} \textbf{General Comments}: The domain of a basic logarithmic function is $(0, \infty)$ and the Range is $(-\infty, \infty)$. We can use shifts when finding the Domain, but the Range will always be all Real numbers.
}
\litem{
Solve the equation below for $x$.
\[ 5^{4x+2} = 16^{3x+5} \]The solution is \( x = -5.662 \).\begin{enumerate}[label=\Alph*.]
\textbf{Plausible alternative answers include:}$x = 10.644$, which corresponds to distributing the $\ln(base)$ to the second term of the exponent only.
$x = -1.596$, which corresponds to distributing the $\ln(base)$ to the first term of the exponent only.
* $x = -5.662$, which is the correct option.
$x = 3.000$, which corresponds to solving the numerators as equal while ignoring the bases are different.
This corresponds to believing there is no solution since the bases are not powers of each other.
\end{enumerate}

\textbf{General Comment:} \textbf{General Comments:} This question was written so that the bases could not be written the same. You will need to take the log of both sides.
}
\litem{
Describe the Domain of the function below.
\[ f(x) = e^{x-2}-7 \]The solution is \( (-\infty, \infty) \).\begin{enumerate}[label=\Alph*.]
\textbf{Plausible alternative answers include:}$(7, \infty)$, which corresponds to using the negative vertical shift AND flipping the Range interval.
$[7, \infty)$, which corresponds to using the negative vertical shift AND flipping the Range interval AND including the endpoint.
$(-\infty, -7]$, which corresponds to using the correct vertical shift *if we wanted the Range* AND including the endpoint.
$(-\infty, -7)$, which corresponds to using the correct vertical shift *if we wanted the Range*.
* This is the correct option.
\end{enumerate}

\textbf{General Comment:} \textbf{General Comments}: Domain of a basic exponential function is $(-\infty, \infty)$ while the Range is $(0, \infty)$. We can shift these intervals [and even flip when $a<0$!] to find the new Domain/Range.
}
\litem{
Solve the equation below for $x$.
\[  15 = \ln{\sqrt[6]{\frac{28}{e^{6x}}}} \]The solution is \( x = -14.445, \text{ which does not fit in any of the interval options.} \).\begin{enumerate}[label=\Alph*.]
\textbf{Plausible alternative answers include:}$x = 14.445$, which is the negative of the correct solution.
$x = -4.445$, which corresponds to treating any root as a square root.
$x = -3.263$, which corresponds to thinking you need to take the natural log of the left side before reducing.
This corresponds to believing you cannot solve the equation.
*$x = -14.445$ is the correct solution and does not fit in any of the other intervals.
\end{enumerate}

\textbf{General Comment:} \textbf{General Comments}: After using the properties of logarithmic functions to break up the right-hand side, use $\ln(e) = 1$ to reduce the question to a linear function to solve. You can put $\ln(28)$ into a calculator if you are having trouble.
}
\litem{
Solve the equation below for $x$.
\[ \log_{4}{(-3x+5)}+6 = 3 \]The solution is \( x = 1.661 \).\begin{enumerate}[label=\Alph*.]
\textbf{Plausible alternative answers include:}$x = -28.667$, which corresponds to reversing the base and exponent when converting and reversing the value with $x$.
* $x = 1.661$, which is the correct option.
$x = -19.667$, which corresponds to ignoring the vertical shift when converting to exponential form.
$x = -25.333$, which corresponds to reversing the base and exponent when converting.
Corresponds to believing a negative coefficient within the log equation means there is no Real solution.
\end{enumerate}

\textbf{General Comment:} \textbf{General Comments:} First, get the equation in the form $\log_b{(cx+d)} = a$. Then, convert to $b^a = cx+d$ and solve.
}
\litem{
Describe the Domain of the function below.
\[ f(x) = -e^{x-9}+9 \]The solution is \( (-\infty, \infty) \).\begin{enumerate}[label=\Alph*.]
\textbf{Plausible alternative answers include:}$(-9, \infty)$, which corresponds to using the negative vertical shift AND flipping the Range interval.
$(-\infty, 9)$, which corresponds to using the correct vertical shift *if we wanted the Range*.
$(-\infty, 9]$, which corresponds to using the correct vertical shift *if we wanted the Range* AND including the endpoint.
$[-9, \infty)$, which corresponds to using the negative vertical shift AND flipping the Range interval AND including the endpoint.
* This is the correct option.
\end{enumerate}

\textbf{General Comment:} \textbf{General Comments}: Domain of a basic exponential function is $(-\infty, \infty)$ while the Range is $(0, \infty)$. We can shift these intervals [and even flip when $a<0$!] to find the new Domain/Range.
}
\litem{
Solve the equation below for $x$.
\[ 2^{4x-5} = \left(\frac{1}{125}\right)^{5x+5} \]The solution is \( x = -0.768 \).\begin{enumerate}[label=\Alph*.]
\textbf{Plausible alternative answers include:}* $x = -0.768$, which is the correct option.
$x = 0.372$, which corresponds to distributing the $\ln(base)$ to the first term of the exponent only.
$x = -10.000$, which corresponds to solving the numerators as equal while ignoring the bases are different.
$x = 20.676$, which corresponds to distributing the $\ln(base)$ to the second term of the exponent only.
This corresponds to believing there is no solution since the bases are not powers of each other.
\end{enumerate}

\textbf{General Comment:} \textbf{General Comments:} This question was written so that the bases could not be written the same. You will need to take the log of both sides.
}
\litem{
Solve the equation below for $x$.
\[  15 = \sqrt[4]{\frac{11}{e^{3x}}} \]The solution is \( x = -2.811, \text{ which does not fit in any of the interval options.} \).\begin{enumerate}[label=\Alph*.]
\textbf{Plausible alternative answers include:}$x = -1.006$, which corresponds to treating any root as a square root.
$x = 2.811$, which is the negative of the correct solution.
$x = -20.799$, which corresponds to thinking you don't need to take the natural log of both sides before reducing, as if the right side already has a natural log.
This corresponds to believing you cannot solve the equation.
* $x = -2.811$ is the correct solution and does not fit in any of the other intervals.
\end{enumerate}

\textbf{General Comment:} \textbf{General Comments}: After using the properties of logarithmic functions to break up the right-hand side, use $\ln(e) = 1$ to reduce the question to a linear function to solve. You can put $\ln(11)$ into a calculator if you are having trouble.
}
\litem{
Describe the Range of the function below.
\[ f(x) = \log_2{(x+5)}+2 \]The solution is \( (\infty, \infty) \).\begin{enumerate}[label=\Alph*.]
\textbf{Plausible alternative answers include:}$[5, \infty)$, which corresponds to using the negative of the horizontal shift AND including the endpoint.
$(-\infty, 2)$, which corresponds to using the vertical shift while the Range is $(-\infty, \infty)$.
$[2, \infty)$, which corresponds to using the flipped Domain AND including the endpoint.
$(-\infty, -2)$, which corresponds to using the using the negative of vertical shift on $(0, \infty)$.
*This is the correct option.
\end{enumerate}

\textbf{General Comment:} \textbf{General Comments}: The domain of a basic logarithmic function is $(0, \infty)$ and the Range is $(-\infty, \infty)$. We can use shifts when finding the Domain, but the Range will always be all Real numbers.
}
\litem{
Solve the equation below for $x$.
\[ \log_{2}{(-2x+5)}+6 = 2 \]The solution is \( x = 2.469 \).\begin{enumerate}[label=\Alph*.]
\textbf{Plausible alternative answers include:}$x = -10.500$, which corresponds to reversing the base and exponent when converting and reversing the value with $x$.
$x = -5.500$, which corresponds to reversing the base and exponent when converting.
$x = 0.500$, which corresponds to ignoring the vertical shift when converting to exponential form.
* $x = 2.469$, which is the correct option.
Corresponds to believing a negative coefficient within the log equation means there is no Real solution.
\end{enumerate}

\textbf{General Comment:} \textbf{General Comments:} First, get the equation in the form $\log_b{(cx+d)} = a$. Then, convert to $b^a = cx+d$ and solve.
}
\litem{
Describe the Domain of the function below.
\[ f(x) = \log_2{(x+5)}-7 \]The solution is \( (-5, \infty) \).\begin{enumerate}[label=\Alph*.]
\textbf{Plausible alternative answers include:}$(-\infty, 5)$, which corresponds to flipping the Domain. Remember: the general for is $a*\log(x-h)+k$, \textbf{where $a$ does not affect the domain}.
* $(-5, \infty)$, which is the correct option.
$(-\infty, 7]$, which corresponds to using the negative vertical shift AND including the endpoint AND flipping the domain.
$[-7, \infty)$, which corresponds to using the vertical shift when shifting the Domain AND including the endpoint.
This corresponds to thinking of the range of the log function (or the domain of the exponential function).
\end{enumerate}

\textbf{General Comment:} \textbf{General Comments}: The domain of a basic logarithmic function is $(0, \infty)$ and the Range is $(-\infty, \infty)$. We can use shifts when finding the Domain, but the Range will always be all Real numbers.
}
\litem{
Solve the equation below for $x$.
\[ 2^{2x+4} = \left(\frac{1}{27}\right)^{3x+3} \]The solution is \( x = -1.123 \).\begin{enumerate}[label=\Alph*.]
\textbf{Plausible alternative answers include:}$x = -0.089$, which corresponds to distributing the $\ln(base)$ to the first term of the exponent only.
$x = 12.660$, which corresponds to distributing the $\ln(base)$ to the second term of the exponent only.
$x = 1.000$, which corresponds to solving the numerators as equal while ignoring the bases are different.
* $x = -1.123$, which is the correct option.
This corresponds to believing there is no solution since the bases are not powers of each other.
\end{enumerate}

\textbf{General Comment:} \textbf{General Comments:} This question was written so that the bases could not be written the same. You will need to take the log of both sides.
}
\litem{
Describe the Range of the function below.
\[ f(x) = -e^{x+8}+5 \]The solution is \( (-\infty, 5) \).\begin{enumerate}[label=\Alph*.]
\textbf{Plausible alternative answers include:}$(-5, \infty)$, which corresponds to using the negative vertical shift AND flipping the Range interval.
* $(-\infty, 5)$, which is the correct option.
$[-5, \infty)$, which corresponds to using the negative vertical shift AND flipping the Range interval AND including the endpoint.
$(-\infty, 5]$, which corresponds to including the endpoint.
This corresponds to confusing range of an exponential function with the domain of an exponential function.
\end{enumerate}

\textbf{General Comment:} \textbf{General Comments}: Domain of a basic exponential function is $(-\infty, \infty)$ while the Range is $(0, \infty)$. We can shift these intervals [and even flip when $a<0$!] to find the new Domain/Range.
}
\litem{
Solve the equation below for $x$.
\[  5 = \ln{\sqrt[3]{\frac{10}{e^{7x}}}} \]The solution is \( x = -1.814 \).\begin{enumerate}[label=\Alph*.]
\textbf{Plausible alternative answers include:}$x = -1.019$, which corresponds to thinking you need to take the natural log of on the left before reducing.
* $x = -1.814$, which is the correct option.
$x = -1.100$, which corresponds to treating any root as a square root.
This corresponds to believing you cannot solve the equation.
This corresponds to making an unexpected error.
\end{enumerate}

\textbf{General Comment:} \textbf{General Comments}: After using the properties of logarithmic functions to break up the right-hand side, use $\ln(e) = 1$ to reduce the question to a linear function to solve. You can put $\ln(10)$ into a calculator if you are having trouble.
}
\litem{
Solve the equation below for $x$.
\[ \log_{4}{(4x+6)}+6 = 2 \]The solution is \( x = -1.499 \).\begin{enumerate}[label=\Alph*.]
\textbf{Plausible alternative answers include:}* $x = -1.499$, which is the correct option.
$x = 65.500$, which corresponds to reversing the base and exponent when converting and reversing the value with $x$.
$x = 62.500$, which corresponds to reversing the base and exponent when converting.
$x = 2.500$, which corresponds to ignoring the vertical shift when converting to exponential form.
Corresponds to believing a negative coefficient within the log equation means there is no Real solution.
\end{enumerate}

\textbf{General Comment:} \textbf{General Comments:} First, get the equation in the form $\log_b{(cx+d)} = a$. Then, convert to $b^a = cx+d$ and solve.
}
\litem{
Describe the Range of the function below.
\[ f(x) = -e^{x+3}-5 \]The solution is \( (-\infty, -5) \).\begin{enumerate}[label=\Alph*.]
\textbf{Plausible alternative answers include:}$(5, \infty)$, which corresponds to using the negative vertical shift AND flipping the Range interval.
$(-\infty, -5]$, which corresponds to including the endpoint.
$[5, \infty)$, which corresponds to using the negative vertical shift AND flipping the Range interval AND including the endpoint.
* $(-\infty, -5)$, which is the correct option.
This corresponds to confusing range of an exponential function with the domain of an exponential function.
\end{enumerate}

\textbf{General Comment:} \textbf{General Comments}: Domain of a basic exponential function is $(-\infty, \infty)$ while the Range is $(0, \infty)$. We can shift these intervals [and even flip when $a<0$!] to find the new Domain/Range.
}
\end{enumerate}

\end{document}