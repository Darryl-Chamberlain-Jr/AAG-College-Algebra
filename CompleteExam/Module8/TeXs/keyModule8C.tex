\documentclass{extbook}[14pt]
\usepackage{multicol, enumerate, enumitem, hyperref, color, soul, setspace, parskip, fancyhdr, amssymb, amsthm, amsmath, bbm, latexsym, units, mathtools}
\everymath{\displaystyle}
\usepackage[headsep=0.5cm,headheight=0cm, left=1 in,right= 1 in,top= 1 in,bottom= 1 in]{geometry}
\usepackage{dashrule}  % Package to use the command below to create lines between items
\newcommand{\litem}[1]{\item #1

\rule{\textwidth}{0.4pt}}
\pagestyle{fancy}
\lhead{}
\chead{Answer Key for Module8 Version C}
\rhead{}
\lfoot{5107-4344}
\cfoot{}
\rfoot{Fall 2020}
\begin{document}
\textbf{This key should allow you to understand why you choose the option you did (beyond just getting a question right or wrong). \href{https://xronos.clas.ufl.edu/mac1105spring2020/courseDescriptionAndMisc/Exams/LearningFromResults}{More instructions on how to use this key can be found here}.}

\textbf{If you have a suggestion to make the keys better, \href{https://forms.gle/CZkbZmPbC9XALEE88}{please fill out the short survey here}.}

\textit{Note: This key is auto-generated and may contain issues and/or errors. The keys are reviewed after each exam to ensure grading is done accurately. If there are issues (like duplicate options), they are noted in the offline gradebook. The keys are a work-in-progress to give students as many resources to improve as possible.}

\rule{\textwidth}{0.4pt}

\begin{enumerate}\litem{
Which of the following intervals describes the Range of the function below?
\[ f(x) = -\log_2{(x+3)}-1 \]
The solution is \( (\infty, \infty) \), which is option E.\begin{enumerate}[label=\Alph*.]
\item \( [a, \infty), a \in [2.2, 3.7] \)

$[3, \infty)$, which corresponds to using the negative of the horizontal shift AND including the endpoint.
\item \( [a, \infty), a \in [-3.65, -2.38] \)

$[-1, \infty)$, which corresponds to using the flipped Domain AND including the endpoint.
\item \( (-\infty, a), a \in [-1.42, -0.39] \)

$(-\infty, -1)$, which corresponds to using the vertical shift while the Range is $(-\infty, \infty)$.
\item \( (-\infty, a), a \in [0.51, 1.74] \)

$(-\infty, 1)$, which corresponds to using the using the negative of vertical shift on $(0, \infty)$.
\item \( (-\infty, \infty) \)

*This is the correct option.
\end{enumerate}

\textbf{General Comment:} \textbf{General Comments}: The domain of a basic logarithmic function is $(0, \infty)$ and the Range is $(-\infty, \infty)$. We can use shifts when finding the Domain, but the Range will always be all Real numbers.
}
\litem{
Which of the following intervals describes the Domain of the function below?
\[ f(x) = -e^{x-3}+1 \]
The solution is \( (-\infty, \infty) \), which is option E.\begin{enumerate}[label=\Alph*.]
\item \( (-\infty, a], a \in [-0.5, 2.1] \)

$(-\infty, 1]$, which corresponds to using the correct vertical shift *if we wanted the Range* AND including the endpoint.
\item \( (-\infty, a), a \in [-0.5, 2.1] \)

$(-\infty, 1)$, which corresponds to using the correct vertical shift *if we wanted the Range*.
\item \( [a, \infty), a \in [-1.2, 0] \)

$[-1, \infty)$, which corresponds to using the negative vertical shift AND flipping the Range interval AND including the endpoint.
\item \( (a, \infty), a \in [-1.2, 0] \)

$(-1, \infty)$, which corresponds to using the negative vertical shift AND flipping the Range interval.
\item \( (-\infty, \infty) \)

* This is the correct option.
\end{enumerate}

\textbf{General Comment:} \textbf{General Comments}: Domain of a basic exponential function is $(-\infty, \infty)$ while the Range is $(0, \infty)$. We can shift these intervals [and even flip when $a<0$!] to find the new Domain/Range.
}
\litem{
Solve the equation for $x$ and choose the interval that contains the solution (if it exists).
\[ 3^{5x-3} = 64^{3x+4} \]
The solution is \( x = -2.854 \), which is option C.\begin{enumerate}[label=\Alph*.]
\item \( x \in [9.6, 10.8] \)

$x = 9.966$, which corresponds to distributing the $\ln(base)$ to the second term of the exponent only.
\item \( x \in [3.2, 4.4] \)

$x = 3.500$, which corresponds to solving the numerators as equal while ignoring the bases are different.
\item \( x \in [-3.3, -2.7] \)

* $x = -2.854$, which is the correct option.
\item \( x \in [-1.1, 0.3] \)

$x = -1.002$, which corresponds to distributing the $\ln(base)$ to the first term of the exponent only.
\item \( \text{There is no Real solution to the equation.} \)

This corresponds to believing there is no solution since the bases are not powers of each other.
\end{enumerate}

\textbf{General Comment:} \textbf{General Comments:} This question was written so that the bases could not be written the same. You will need to take the log of both sides.
}
\litem{
Which of the following intervals describes the Range of the function below?
\[ f(x) = -\log_2{(x-9)}-6 \]
The solution is \( (\infty, \infty) \), which is option E.\begin{enumerate}[label=\Alph*.]
\item \( [a, \infty), a \in [-9.7, -8.3] \)

$[-9, \infty)$, which corresponds to using the negative of the horizontal shift AND including the endpoint.
\item \( (-\infty, a), a \in [3.3, 8.4] \)

$(-\infty, 6)$, which corresponds to using the using the negative of vertical shift on $(0, \infty)$.
\item \( [a, \infty), a \in [8.4, 10.4] \)

$[-6, \infty)$, which corresponds to using the flipped Domain AND including the endpoint.
\item \( (-\infty, a), a \in [-7.4, -5.4] \)

$(-\infty, -6)$, which corresponds to using the vertical shift while the Range is $(-\infty, \infty)$.
\item \( (-\infty, \infty) \)

*This is the correct option.
\end{enumerate}

\textbf{General Comment:} \textbf{General Comments}: The domain of a basic logarithmic function is $(0, \infty)$ and the Range is $(-\infty, \infty)$. We can use shifts when finding the Domain, but the Range will always be all Real numbers.
}
\litem{
 Solve the equation for $x$ and choose the interval that contains $x$ (if it exists).
\[  25 = \sqrt[7]{\frac{16}{e^{3x}}} \]
The solution is \( x = -6.587 \), which is option A.\begin{enumerate}[label=\Alph*.]
\item \( x \in [-8.59, -5.59] \)

* $x = -6.587$, which is the correct option.
\item \( x \in [-2.22, -0.22] \)

$x = -1.222$, which corresponds to treating any root as a square root.
\item \( x \in [-60.26, -56.26] \)

$x = -59.258$, which corresponds to thinking you don't need to take the natural log of both sides before reducing, as if the equation already had a natural log on the right side.
\item \( \text{There is no Real solution to the equation.} \)

This corresponds to believing you cannot solve the equation.
\item \( \text{None of the above.} \)

This corresponds to making an unexpected error.
\end{enumerate}

\textbf{General Comment:} \textbf{General Comments}: After using the properties of logarithmic functions to break up the right-hand side, use $\ln(e) = 1$ to reduce the question to a linear function to solve. You can put $\ln(16)$ into a calculator if you are having trouble.
}
\litem{
Solve the equation for $x$ and choose the interval that contains the solution (if it exists).
\[ \log_{3}{(-4x+8)}+5 = 2 \]
The solution is \( x = 1.991 \), which is option C.\begin{enumerate}[label=\Alph*.]
\item \( x \in [8.55, 8.88] \)

$x = 8.750$, which corresponds to reversing the base and exponent when converting.
\item \( x \in [-0.77, 0.4] \)

$x = -0.250$, which corresponds to ignoring the vertical shift when converting to exponential form.
\item \( x \in [0.96, 3] \)

* $x = 1.991$, which is the correct option.
\item \( x \in [3.22, 5.46] \)

$x = 4.750$, which corresponds to reversing the base and exponent when converting and reversing the value with $x$.
\item \( \text{There is no Real solution to the equation.} \)

Corresponds to believing a negative coefficient within the log equation means there is no Real solution.
\end{enumerate}

\textbf{General Comment:} \textbf{General Comments:} First, get the equation in the form $\log_b{(cx+d)} = a$. Then, convert to $b^a = cx+d$ and solve.
}
\litem{
 Solve the equation for $x$ and choose the interval that contains $x$ (if it exists).
\[  12 = \ln{\sqrt[6]{\frac{10}{e^{9x}}}} \]
The solution is \( x = -7.744 \), which is option C.\begin{enumerate}[label=\Alph*.]
\item \( x \in [-2.5, -2] \)

$x = -2.411$, which corresponds to treating any root as a square root.
\item \( x \in [-2.1, -1.5] \)

$x = -1.912$, which corresponds to thinking you need to take the natural log of on the left before reducing.
\item \( x \in [-8.5, -6.5] \)

* $x = -7.744$, which is the correct option.
\item \( \text{There is no Real solution to the equation.} \)

This corresponds to believing you cannot solve the equation.
\item \( \text{None of the above.} \)

This corresponds to making an unexpected error.
\end{enumerate}

\textbf{General Comment:} \textbf{General Comments}: After using the properties of logarithmic functions to break up the right-hand side, use $\ln(e) = 1$ to reduce the question to a linear function to solve. You can put $\ln(10)$ into a calculator if you are having trouble.
}
\litem{
Which of the following intervals describes the Range of the function below?
\[ f(x) = e^{x-7}-7 \]
The solution is \( (-7, \infty) \), which is option D.\begin{enumerate}[label=\Alph*.]
\item \( (-\infty, a), a \in [7, 8] \)

$(-\infty, 7)$, which corresponds to using the negative vertical shift AND flipping the Range interval.
\item \( [a, \infty), a \in [-7, -4] \)

$[-7, \infty)$, which corresponds to including the endpoint.
\item \( (-\infty, a], a \in [7, 8] \)

$(-\infty, 7]$, which corresponds to using the negative vertical shift AND flipping the Range interval AND including the endpoint.
\item \( (a, \infty), a \in [-7, -4] \)

* $(-7, \infty)$, which is the correct option.
\item \( (-\infty, \infty) \)

This corresponds to confusing range of an exponential function with the domain of an exponential function.
\end{enumerate}

\textbf{General Comment:} \textbf{General Comments}: Domain of a basic exponential function is $(-\infty, \infty)$ while the Range is $(0, \infty)$. We can shift these intervals [and even flip when $a<0$!] to find the new Domain/Range.
}
\litem{
Solve the equation for $x$ and choose the interval that contains the solution (if it exists).
\[ 4^{-4x-3} = 125^{-3x-5} \]
The solution is \( x = -2.235 \), which is option B.\begin{enumerate}[label=\Alph*.]
\item \( x \in [18.98, 22.98] \)

$x = 19.983$, which corresponds to distributing the $\ln(base)$ to the second term of the exponent only.
\item \( x \in [-2.24, -1.24] \)

* $x = -2.235$, which is the correct option.
\item \( x \in [1, 5] \)

$x = 2.000$, which corresponds to solving the numerators as equal while ignoring the bases are different.
\item \( x \in [-1.22, 1.78] \)

$x = -0.224$, which corresponds to distributing the $\ln(base)$ to the first term of the exponent only.
\item \( \text{There is no Real solution to the equation.} \)

This corresponds to believing there is no solution since the bases are not powers of each other.
\end{enumerate}

\textbf{General Comment:} \textbf{General Comments:} This question was written so that the bases could not be written the same. You will need to take the log of both sides.
}
\litem{
Solve the equation for $x$ and choose the interval that contains the solution (if it exists).
\[ \log_{5}{(-4x+6)}+5 = 2 \]
The solution is \( x = 1.498 \), which is option C.\begin{enumerate}[label=\Alph*.]
\item \( x \in [-9.75, 1.25] \)

$x = -4.750$, which corresponds to ignoring the vertical shift when converting to exponential form.
\item \( x \in [61.25, 69.25] \)

$x = 62.250$, which corresponds to reversing the base and exponent when converting.
\item \( x \in [-0.5, 2.5] \)

* $x = 1.498$, which is the correct option.
\item \( x \in [57.25, 60.25] \)

$x = 59.250$, which corresponds to reversing the base and exponent when converting and reversing the value with $x$.
\item \( \text{There is no Real solution to the equation.} \)

Corresponds to believing a negative coefficient within the log equation means there is no Real solution.
\end{enumerate}

\textbf{General Comment:} \textbf{General Comments:} First, get the equation in the form $\log_b{(cx+d)} = a$. Then, convert to $b^a = cx+d$ and solve.
}
\end{enumerate}

\end{document}