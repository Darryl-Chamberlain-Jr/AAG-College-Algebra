\documentclass{extbook}[14pt]
\usepackage{multicol, enumerate, enumitem, hyperref, color, soul, setspace, parskip, fancyhdr, amssymb, amsthm, amsmath, bbm, latexsym, units, mathtools}
\everymath{\displaystyle}
\usepackage[headsep=0.5cm,headheight=0cm, left=1 in,right= 1 in,top= 1 in,bottom= 1 in]{geometry}
\usepackage{dashrule}  % Package to use the command below to create lines between items
\newcommand{\litem}[1]{\item #1

\rule{\textwidth}{0.4pt}}
\pagestyle{fancy}
\lhead{}
\chead{Answer Key for Module8 Version B}
\rhead{}
\lfoot{1569-1502}
\cfoot{}
\rfoot{testing}
\begin{document}
\textbf{This key should allow you to understand why you choose the option you did (beyond just getting a question right or wrong). \href{https://xronos.clas.ufl.edu/mac1105spring2020/courseDescriptionAndMisc/Exams/LearningFromResults}{More instructions on how to use this key can be found here}.}

\textbf{If you have a suggestion to make the keys better, \href{https://forms.gle/CZkbZmPbC9XALEE88}{please fill out the short survey here}.}

\textit{Note: This key is auto-generated and may contain issues and/or errors. The keys are reviewed after each exam to ensure grading is done accurately. If there are issues (like duplicate options), they are noted in the offline gradebook. The keys are a work-in-progress to give students as many resources to improve as possible.}

\rule{\textwidth}{0.4pt}

\begin{enumerate}\litem{
Solve the equation for $x$ and choose the interval that contains the solution (if it exists).
\[ \log_{2}{(2x+6)}+6 = 2 \]The solution is \( x = -2.969 \), which is option A.\begin{enumerate}[label=\Alph*.]
\item \( x \in [-5.1, -1.9] \)

* $x = -2.969$, which is the correct option.
\item \( x \in [3.5, 5.6] \)

$x = 5.000$, which corresponds to reversing the base and exponent when converting.
\item \( x \in [9.5, 12.2] \)

$x = 11.000$, which corresponds to reversing the base and exponent when converting and reversing the value with $x$.
\item \( x \in [-2.3, 1.4] \)

$x = -1.000$, which corresponds to ignoring the vertical shift when converting to exponential form.
\item \( \text{There is no Real solution to the equation.} \)

Corresponds to believing a negative coefficient within the log equation means there is no Real solution.
\end{enumerate}

\textbf{General Comment:} \textbf{General Comments:} First, get the equation in the form $\log_b{(cx+d)} = a$. Then, convert to $b^a = cx+d$ and solve.
}
\litem{
Which of the following intervals describes the Range of the function below?
\[ f(x) = -e^{x+3}+1 \]The solution is \( (-\infty, 1) \), which is option B.\begin{enumerate}[label=\Alph*.]
\item \( (a, \infty), a \in [-4.5, -0.7] \)

$(-1, \infty)$, which corresponds to using the negative vertical shift AND flipping the Range interval.
\item \( (-\infty, a), a \in [0.7, 1.7] \)

* $(-\infty, 1)$, which is the correct option.
\item \( [a, \infty), a \in [-4.5, -0.7] \)

$[-1, \infty)$, which corresponds to using the negative vertical shift AND flipping the Range interval AND including the endpoint.
\item \( (-\infty, a], a \in [0.7, 1.7] \)

$(-\infty, 1]$, which corresponds to including the endpoint.
\item \( (-\infty, \infty) \)

This corresponds to confusing range of an exponential function with the domain of an exponential function.
\end{enumerate}

\textbf{General Comment:} \textbf{General Comments}: Domain of a basic exponential function is $(-\infty, \infty)$ while the Range is $(0, \infty)$. We can shift these intervals [and even flip when $a<0$!] to find the new Domain/Range.
}
\litem{
 Solve the equation for $x$ and choose the interval that contains $x$ (if it exists).
\[  21 = \ln{\sqrt[5]{\frac{20}{e^{8x}}}} \]The solution is \( x = -12.751, \text{ which does not fit in any of the interval options.} \), which is option E.\begin{enumerate}[label=\Alph*.]
\item \( x \in [-3.28, 0.72] \)

$x = -2.277$, which corresponds to thinking you need to take the natural log of the left side before reducing.
\item \( x \in [10.75, 15.75] \)

$x = 12.751$, which is the negative of the correct solution.
\item \( x \in [-7.88, -2.88] \)

$x = -4.876$, which corresponds to treating any root as a square root.
\item \( \text{There is no Real solution to the equation.} \)

This corresponds to believing you cannot solve the equation.
\item \( \text{None of the above.} \)

*$x = -12.751$ is the correct solution and does not fit in any of the other intervals.
\end{enumerate}

\textbf{General Comment:} \textbf{General Comments}: After using the properties of logarithmic functions to break up the right-hand side, use $\ln(e) = 1$ to reduce the question to a linear function to solve. You can put $\ln(20)$ into a calculator if you are having trouble.
}
\litem{
Solve the equation for $x$ and choose the interval that contains the solution (if it exists).
\[ 5^{3x+3} = 49^{5x-5} \]The solution is \( x = 1.660 \), which is option A.\begin{enumerate}[label=\Alph*.]
\item \( x \in [1, 2.3] \)

* $x = 1.660$, which is the correct option.
\item \( x \in [11.4, 12.7] \)

$x = 12.144$, which corresponds to distributing the $\ln(base)$ to the second term of the exponent only.
\item \( x \in [0, 1.6] \)

$x = 0.547$, which corresponds to distributing the $\ln(base)$ to the first term of the exponent only.
\item \( x \in [3.6, 5.1] \)

$x = 4.000$, which corresponds to solving the numerators as equal while ignoring the bases are different.
\item \( \text{There is no Real solution to the equation.} \)

This corresponds to believing there is no solution since the bases are not powers of each other.
\end{enumerate}

\textbf{General Comment:} \textbf{General Comments:} This question was written so that the bases could not be written the same. You will need to take the log of both sides.
}
\litem{
Which of the following intervals describes the Range of the function below?
\[ f(x) = \log_2{(x-1)}-7 \]The solution is \( (\infty, \infty) \), which is option E.\begin{enumerate}[label=\Alph*.]
\item \( [a, \infty), a \in [-4, 0] \)

$[-1, \infty)$, which corresponds to using the negative of the horizontal shift AND including the endpoint.
\item \( (-\infty, a), a \in [4, 12] \)

$(-\infty, 7)$, which corresponds to using the using the negative of vertical shift on $(0, \infty)$.
\item \( (-\infty, a), a \in [-11, -4] \)

$(-\infty, -7)$, which corresponds to using the vertical shift while the Range is $(-\infty, \infty)$.
\item \( [a, \infty), a \in [0, 2] \)

$[-7, \infty)$, which corresponds to using the flipped Domain AND including the endpoint.
\item \( (-\infty, \infty) \)

*This is the correct option.
\end{enumerate}

\textbf{General Comment:} \textbf{General Comments}: The domain of a basic logarithmic function is $(0, \infty)$ and the Range is $(-\infty, \infty)$. We can use shifts when finding the Domain, but the Range will always be all Real numbers.
}
\litem{
 Solve the equation for $x$ and choose the interval that contains $x$ (if it exists).
\[  22 = \sqrt[7]{\frac{20}{e^{7x}}} \]The solution is \( x = -2.663 \), which is option A.\begin{enumerate}[label=\Alph*.]
\item \( x \in [-3.66, -1.66] \)

* $x = -2.663$, which is the correct option.
\item \( x \in [-1.46, 0.54] \)

$x = -0.455$, which corresponds to treating any root as a square root.
\item \( x \in [-22.43, -20.43] \)

$x = -22.428$, which corresponds to thinking you don't need to take the natural log of both sides before reducing, as if the equation already had a natural log on the right side.
\item \( \text{There is no Real solution to the equation.} \)

This corresponds to believing you cannot solve the equation.
\item \( \text{None of the above.} \)

This corresponds to making an unexpected error.
\end{enumerate}

\textbf{General Comment:} \textbf{General Comments}: After using the properties of logarithmic functions to break up the right-hand side, use $\ln(e) = 1$ to reduce the question to a linear function to solve. You can put $\ln(20)$ into a calculator if you are having trouble.
}
\litem{
Which of the following intervals describes the Range of the function below?
\[ f(x) = -\log_2{(x-8)}+9 \]The solution is \( (\infty, \infty) \), which is option E.\begin{enumerate}[label=\Alph*.]
\item \( [a, \infty), a \in [7.91, 8.66] \)

$[9, \infty)$, which corresponds to using the flipped Domain AND including the endpoint.
\item \( [a, \infty), a \in [-8.45, -8] \)

$[-8, \infty)$, which corresponds to using the negative of the horizontal shift AND including the endpoint.
\item \( (-\infty, a), a \in [-9.26, -8.87] \)

$(-\infty, -9)$, which corresponds to using the using the negative of vertical shift on $(0, \infty)$.
\item \( (-\infty, a), a \in [8.15, 9.56] \)

$(-\infty, 9)$, which corresponds to using the vertical shift while the Range is $(-\infty, \infty)$.
\item \( (-\infty, \infty) \)

*This is the correct option.
\end{enumerate}

\textbf{General Comment:} \textbf{General Comments}: The domain of a basic logarithmic function is $(0, \infty)$ and the Range is $(-\infty, \infty)$. We can use shifts when finding the Domain, but the Range will always be all Real numbers.
}
\litem{
Solve the equation for $x$ and choose the interval that contains the solution (if it exists).
\[ 5^{2x-2} = \left(\frac{1}{36}\right)^{4x+5} \]The solution is \( x = -0.837 \), which is option C.\begin{enumerate}[label=\Alph*.]
\item \( x \in [6.29, 8.16] \)

$x = 7.349$, which corresponds to distributing the $\ln(base)$ to the second term of the exponent only.
\item \( x \in [0.12, 0.54] \)

$x = 0.399$, which corresponds to distributing the $\ln(base)$ to the first term of the exponent only.
\item \( x \in [-0.88, -0.5] \)

* $x = -0.837$, which is the correct option.
\item \( x \in [-3.67, -3.35] \)

$x = -3.500$, which corresponds to solving the numerators as equal while ignoring the bases are different.
\item \( \text{There is no Real solution to the equation.} \)

This corresponds to believing there is no solution since the bases are not powers of each other.
\end{enumerate}

\textbf{General Comment:} \textbf{General Comments:} This question was written so that the bases could not be written the same. You will need to take the log of both sides.
}
\litem{
Which of the following intervals describes the Domain of the function below?
\[ f(x) = e^{x+2}+1 \]The solution is \( (-\infty, \infty) \), which is option E.\begin{enumerate}[label=\Alph*.]
\item \( [a, \infty), a \in [-3, 0] \)

$[-1, \infty)$, which corresponds to using the negative vertical shift AND flipping the Range interval AND including the endpoint.
\item \( (-\infty, a], a \in [0, 3] \)

$(-\infty, 1]$, which corresponds to using the correct vertical shift *if we wanted the Range* AND including the endpoint.
\item \( (a, \infty), a \in [-3, 0] \)

$(-1, \infty)$, which corresponds to using the negative vertical shift AND flipping the Range interval.
\item \( (-\infty, a), a \in [0, 3] \)

$(-\infty, 1)$, which corresponds to using the correct vertical shift *if we wanted the Range*.
\item \( (-\infty, \infty) \)

* This is the correct option.
\end{enumerate}

\textbf{General Comment:} \textbf{General Comments}: Domain of a basic exponential function is $(-\infty, \infty)$ while the Range is $(0, \infty)$. We can shift these intervals [and even flip when $a<0$!] to find the new Domain/Range.
}
\litem{
Solve the equation for $x$ and choose the interval that contains the solution (if it exists).
\[ \log_{4}{(3x+7)}+5 = 2 \]The solution is \( x = -2.328 \), which is option D.\begin{enumerate}[label=\Alph*.]
\item \( x \in [0, 6] \)

$x = 3.000$, which corresponds to ignoring the vertical shift when converting to exponential form.
\item \( x \in [21.67, 26.67] \)

$x = 24.667$, which corresponds to reversing the base and exponent when converting.
\item \( x \in [28.33, 32.33] \)

$x = 29.333$, which corresponds to reversing the base and exponent when converting and reversing the value with $x$.
\item \( x \in [-5.33, -1.33] \)

* $x = -2.328$, which is the correct option.
\item \( \text{There is no Real solution to the equation.} \)

Corresponds to believing a negative coefficient within the log equation means there is no Real solution.
\end{enumerate}

\textbf{General Comment:} \textbf{General Comments:} First, get the equation in the form $\log_b{(cx+d)} = a$. Then, convert to $b^a = cx+d$ and solve.
}
\end{enumerate}

\end{document}