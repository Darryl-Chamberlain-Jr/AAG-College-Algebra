\documentclass{extbook}[14pt]
\usepackage{multicol, enumerate, enumitem, hyperref, color, soul, setspace, parskip, fancyhdr, amssymb, amsthm, amsmath, latexsym, units, mathtools}
\everymath{\displaystyle}
\usepackage[headsep=0.5cm,headheight=0cm, left=1 in,right= 1 in,top= 1 in,bottom= 1 in]{geometry}
\usepackage{dashrule}  % Package to use the command below to create lines between items
\newcommand{\litem}[1]{\item #1

\rule{\textwidth}{0.4pt}}
\pagestyle{fancy}
\lhead{}
\chead{Answer Key for Makeup Progress Quiz 2 Version A}
\rhead{}
\lfoot{5763-3522}
\cfoot{}
\rfoot{Spring 2021}
\begin{document}
\textbf{This key should allow you to understand why you choose the option you did (beyond just getting a question right or wrong). \href{https://xronos.clas.ufl.edu/mac1105spring2020/courseDescriptionAndMisc/Exams/LearningFromResults}{More instructions on how to use this key can be found here}.}

\textbf{If you have a suggestion to make the keys better, \href{https://forms.gle/CZkbZmPbC9XALEE88}{please fill out the short survey here}.}

\textit{Note: This key is auto-generated and may contain issues and/or errors. The keys are reviewed after each exam to ensure grading is done accurately. If there are issues (like duplicate options), they are noted in the offline gradebook. The keys are a work-in-progress to give students as many resources to improve as possible.}

\rule{\textwidth}{0.4pt}

\begin{enumerate}\litem{
Which of the following intervals describes the Domain of the function below?
\[ f(x) = -\log_2{(x+1)}-8 \]The solution is \( (-1, \infty) \), which is option D.\begin{enumerate}[label=\Alph*.]
\item \( (-\infty, a], a \in [7.34, 9.69] \)

$(-\infty, 8]$, which corresponds to using the negative vertical shift AND including the endpoint AND flipping the domain.
\item \( (-\infty, a), a \in [0.12, 1.91] \)

$(-\infty, 1)$, which corresponds to flipping the Domain. Remember: the general for is $a*\log(x-h)+k$, \textbf{where $a$ does not affect the domain}.
\item \( [a, \infty), a \in [-8.29, -7.01] \)

$[-8, \infty)$, which corresponds to using the vertical shift when shifting the Domain AND including the endpoint.
\item \( (a, \infty), a \in [-1.93, -0.7] \)

* $(-1, \infty)$, which is the correct option.
\item \( (-\infty, \infty) \)

This corresponds to thinking of the range of the log function (or the domain of the exponential function).
\end{enumerate}

\textbf{General Comment:} \textbf{General Comments}: The domain of a basic logarithmic function is $(0, \infty)$ and the Range is $(-\infty, \infty)$. We can use shifts when finding the Domain, but the Range will always be all Real numbers.
}
\litem{
Solve the equation for $x$ and choose the interval that contains the solution (if it exists).
\[ 5^{-4x-2} = \left(\frac{1}{343}\right)^{-2x-5} \]The solution is \( x = -1.789 \), which is option D.\begin{enumerate}[label=\Alph*.]
\item \( x \in [-16.7, -14.6] \)

$x = -16.204$, which corresponds to distributing the $\ln(base)$ to the second term of the exponent only.
\item \( x \in [-1, 0.4] \)

$x = 0.166$, which corresponds to distributing the $\ln(base)$ to the first term of the exponent only.
\item \( x \in [1.1, 3.1] \)

$x = 1.500$, which corresponds to solving the numerators as equal while ignoring the bases are different.
\item \( x \in [-1.9, -0.8] \)

* $x = -1.789$, which is the correct option.
\item \( \text{There is no Real solution to the equation.} \)

This corresponds to believing there is no solution since the bases are not powers of each other.
\end{enumerate}

\textbf{General Comment:} \textbf{General Comments:} This question was written so that the bases could not be written the same. You will need to take the log of both sides.
}
\litem{
Solve the equation for $x$ and choose the interval that contains the solution (if it exists).
\[ \log_{4}{(-2x+7)}+6 = 3 \]The solution is \( x = 3.492 \), which is option A.\begin{enumerate}[label=\Alph*.]
\item \( x \in [1.49, 7.49] \)

* $x = 3.492$, which is the correct option.
\item \( x \in [-33.5, -23.5] \)

$x = -28.500$, which corresponds to ignoring the vertical shift when converting to exponential form.
\item \( x \in [-37, -30] \)

$x = -37.000$, which corresponds to reversing the base and exponent when converting.
\item \( x \in [-50, -40] \)

$x = -44.000$, which corresponds to reversing the base and exponent when converting and reversing the value with $x$.
\item \( \text{There is no Real solution to the equation.} \)

Corresponds to believing a negative coefficient within the log equation means there is no Real solution.
\end{enumerate}

\textbf{General Comment:} \textbf{General Comments:} First, get the equation in the form $\log_b{(cx+d)} = a$. Then, convert to $b^a = cx+d$ and solve.
}
\litem{
Which of the following intervals describes the Range of the function below?
\[ f(x) = e^{x+2}+2 \]The solution is \( (2, \infty) \), which is option B.\begin{enumerate}[label=\Alph*.]
\item \( (-\infty, a], a \in [-3.9, 1.9] \)

$(-\infty, -2]$, which corresponds to using the negative vertical shift AND flipping the Range interval AND including the endpoint.
\item \( (a, \infty), a \in [-1.2, 2.7] \)

* $(2, \infty)$, which is the correct option.
\item \( [a, \infty), a \in [-1.2, 2.7] \)

$[2, \infty)$, which corresponds to including the endpoint.
\item \( (-\infty, a), a \in [-3.9, 1.9] \)

$(-\infty, -2)$, which corresponds to using the negative vertical shift AND flipping the Range interval.
\item \( (-\infty, \infty) \)

This corresponds to confusing range of an exponential function with the domain of an exponential function.
\end{enumerate}

\textbf{General Comment:} \textbf{General Comments}: Domain of a basic exponential function is $(-\infty, \infty)$ while the Range is $(0, \infty)$. We can shift these intervals [and even flip when $a<0$!] to find the new Domain/Range.
}
\litem{
 Solve the equation for $x$ and choose the interval that contains $x$ (if it exists).
\[  18 = \sqrt[5]{\frac{20}{e^{3x}}} \]The solution is \( x = -3.819, \text{ which does not fit in any of the interval options.} \), which is option E.\begin{enumerate}[label=\Alph*.]
\item \( x \in [-0.93, 0.07] \)

$x = -0.928$, which corresponds to treating any root as a square root.
\item \( x \in [0.82, 7.82] \)

$x = 3.819$, which is the negative of the correct solution.
\item \( x \in [-31, -28] \)

$x = -30.999$, which corresponds to thinking you don't need to take the natural log of both sides before reducing, as if the right side already has a natural log.
\item \( \text{There is no Real solution to the equation.} \)

This corresponds to believing you cannot solve the equation.
\item \( \text{None of the above.} \)

* $x = -3.819$ is the correct solution and does not fit in any of the other intervals.
\end{enumerate}

\textbf{General Comment:} \textbf{General Comments}: After using the properties of logarithmic functions to break up the right-hand side, use $\ln(e) = 1$ to reduce the question to a linear function to solve. You can put $\ln(20)$ into a calculator if you are having trouble.
}
\litem{
Solve the equation for $x$ and choose the interval that contains the solution (if it exists).
\[ 4^{-3x-3} = 49^{-2x+3} \]The solution is \( x = 4.368 \), which is option A.\begin{enumerate}[label=\Alph*.]
\item \( x \in [3.8, 5.1] \)

* $x = 4.368$, which is the correct option.
\item \( x \in [-7.3, -5.6] \)

$x = -6.000$, which corresponds to solving the numerators as equal while ignoring the bases are different.
\item \( x \in [-17.7, -12.8] \)

$x = -15.834$, which corresponds to distributing the $\ln(base)$ to the second term of the exponent only.
\item \( x \in [0.7, 3.7] \)

$x = 1.655$, which corresponds to distributing the $\ln(base)$ to the first term of the exponent only.
\item \( \text{There is no Real solution to the equation.} \)

This corresponds to believing there is no solution since the bases are not powers of each other.
\end{enumerate}

\textbf{General Comment:} \textbf{General Comments:} This question was written so that the bases could not be written the same. You will need to take the log of both sides.
}
\litem{
 Solve the equation for $x$ and choose the interval that contains $x$ (if it exists).
\[  23 = \sqrt[3]{\frac{5}{e^{9x}}} \]The solution is \( x = -0.866 \), which is option B.\begin{enumerate}[label=\Alph*.]
\item \( x \in [-0.7, 0.5] \)

$x = -0.518$, which corresponds to treating any root as a square root.
\item \( x \in [-2.1, -0.7] \)

* $x = -0.866$, which is the correct option.
\item \( x \in [-8.5, -7.2] \)

$x = -7.845$, which corresponds to thinking you don't need to take the natural log of both sides before reducing, as if the equation already had a natural log on the right side.
\item \( \text{There is no Real solution to the equation.} \)

This corresponds to believing you cannot solve the equation.
\item \( \text{None of the above.} \)

This corresponds to making an unexpected error.
\end{enumerate}

\textbf{General Comment:} \textbf{General Comments}: After using the properties of logarithmic functions to break up the right-hand side, use $\ln(e) = 1$ to reduce the question to a linear function to solve. You can put $\ln(5)$ into a calculator if you are having trouble.
}
\litem{
Solve the equation for $x$ and choose the interval that contains the solution (if it exists).
\[ \log_{3}{(2x+5)}+5 = 2 \]The solution is \( x = -2.481 \), which is option A.\begin{enumerate}[label=\Alph*.]
\item \( x \in [-2.7, -1] \)

* $x = -2.481$, which is the correct option.
\item \( x \in [-17.4, -14.2] \)

$x = -16.000$, which corresponds to reversing the base and exponent when converting.
\item \( x \in [-1.4, 3.9] \)

$x = 2.000$, which corresponds to ignoring the vertical shift when converting to exponential form.
\item \( x \in [-11.3, -9.8] \)

$x = -11.000$, which corresponds to reversing the base and exponent when converting and reversing the value with $x$.
\item \( \text{There is no Real solution to the equation.} \)

Corresponds to believing a negative coefficient within the log equation means there is no Real solution.
\end{enumerate}

\textbf{General Comment:} \textbf{General Comments:} First, get the equation in the form $\log_b{(cx+d)} = a$. Then, convert to $b^a = cx+d$ and solve.
}
\litem{
Which of the following intervals describes the Range of the function below?
\[ f(x) = e^{x-2}+4 \]The solution is \( (4, \infty) \), which is option C.\begin{enumerate}[label=\Alph*.]
\item \( (-\infty, a), a \in [-4, 1] \)

$(-\infty, -4)$, which corresponds to using the negative vertical shift AND flipping the Range interval.
\item \( (-\infty, a], a \in [-4, 1] \)

$(-\infty, -4]$, which corresponds to using the negative vertical shift AND flipping the Range interval AND including the endpoint.
\item \( (a, \infty), a \in [3, 6] \)

* $(4, \infty)$, which is the correct option.
\item \( [a, \infty), a \in [3, 6] \)

$[4, \infty)$, which corresponds to including the endpoint.
\item \( (-\infty, \infty) \)

This corresponds to confusing range of an exponential function with the domain of an exponential function.
\end{enumerate}

\textbf{General Comment:} \textbf{General Comments}: Domain of a basic exponential function is $(-\infty, \infty)$ while the Range is $(0, \infty)$. We can shift these intervals [and even flip when $a<0$!] to find the new Domain/Range.
}
\litem{
Which of the following intervals describes the Domain of the function below?
\[ f(x) = -\log_2{(x+6)}-5 \]The solution is \( (-6, \infty) \), which is option D.\begin{enumerate}[label=\Alph*.]
\item \( (-\infty, a], a \in [4.86, 5.8] \)

$(-\infty, 5]$, which corresponds to using the negative vertical shift AND including the endpoint AND flipping the domain.
\item \( (-\infty, a), a \in [5.98, 6.26] \)

$(-\infty, 6)$, which corresponds to flipping the Domain. Remember: the general for is $a*\log(x-h)+k$, \textbf{where $a$ does not affect the domain}.
\item \( [a, \infty), a \in [-5.28, -4.62] \)

$[-5, \infty)$, which corresponds to using the vertical shift when shifting the Domain AND including the endpoint.
\item \( (a, \infty), a \in [-7.54, -5.54] \)

* $(-6, \infty)$, which is the correct option.
\item \( (-\infty, \infty) \)

This corresponds to thinking of the range of the log function (or the domain of the exponential function).
\end{enumerate}

\textbf{General Comment:} \textbf{General Comments}: The domain of a basic logarithmic function is $(0, \infty)$ and the Range is $(-\infty, \infty)$. We can use shifts when finding the Domain, but the Range will always be all Real numbers.
}
\end{enumerate}

\end{document}