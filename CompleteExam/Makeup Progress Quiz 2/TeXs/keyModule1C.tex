\documentclass{extbook}[14pt]
\usepackage{multicol, enumerate, enumitem, hyperref, color, soul, setspace, parskip, fancyhdr, amssymb, amsthm, amsmath, latexsym, units, mathtools}
\everymath{\displaystyle}
\usepackage[headsep=0.5cm,headheight=0cm, left=1 in,right= 1 in,top= 1 in,bottom= 1 in]{geometry}
\usepackage{dashrule}  % Package to use the command below to create lines between items
\newcommand{\litem}[1]{\item #1

\rule{\textwidth}{0.4pt}}
\pagestyle{fancy}
\lhead{}
\chead{Answer Key for Makeup Progress Quiz 2 Version C}
\rhead{}
\lfoot{5763-3522}
\cfoot{}
\rfoot{Spring 2021}
\begin{document}
\textbf{This key should allow you to understand why you choose the option you did (beyond just getting a question right or wrong). \href{https://xronos.clas.ufl.edu/mac1105spring2020/courseDescriptionAndMisc/Exams/LearningFromResults}{More instructions on how to use this key can be found here}.}

\textbf{If you have a suggestion to make the keys better, \href{https://forms.gle/CZkbZmPbC9XALEE88}{please fill out the short survey here}.}

\textit{Note: This key is auto-generated and may contain issues and/or errors. The keys are reviewed after each exam to ensure grading is done accurately. If there are issues (like duplicate options), they are noted in the offline gradebook. The keys are a work-in-progress to give students as many resources to improve as possible.}

\rule{\textwidth}{0.4pt}

\begin{enumerate}\litem{
Choose the \textbf{smallest} set of Complex numbers that the number below belongs to.
\[ \frac{\sqrt{91}}{12}+4i^2 \]The solution is \( \text{Irrational} \), which is option C.\begin{enumerate}[label=\Alph*.]
\item \( \text{Nonreal Complex} \)

This is a Complex number $(a+bi)$ that is not Real (has $i$ as part of the number).
\item \( \text{Not a Complex Number} \)

This is not a number. The only non-Complex number we know is dividing by 0 as this is not a number!
\item \( \text{Irrational} \)

* This is the correct option!
\item \( \text{Rational} \)

These are numbers that can be written as fraction of Integers (e.g., -2/3 + 5)
\item \( \text{Pure Imaginary} \)

This is a Complex number $(a+bi)$ that \textbf{only} has an imaginary part like $2i$.
\end{enumerate}

\textbf{General Comment:} Be sure to simplify $i^2 = -1$. This may remove the imaginary portion for your number. If you are having trouble, you may want to look at the \textit{Subgroups of the Real Numbers} section.
}
\litem{
Simplify the expression below into the form $a+bi$. Then, choose the intervals that $a$ and $b$ belong to.
\[ (-8 + 9 i)(10 + 2 i) \]The solution is \( -98 + 74 i \), which is option B.\begin{enumerate}[label=\Alph*.]
\item \( a \in [-63, -56] \text{ and } b \in [106, 112] \)

 $-62 + 106 i$, which corresponds to adding a minus sign in the second term.
\item \( a \in [-101, -97] \text{ and } b \in [71, 75] \)

* $-98 + 74 i$, which is the correct option.
\item \( a \in [-81, -75] \text{ and } b \in [16, 19] \)

 $-80 + 18 i$, which corresponds to just multiplying the real terms to get the real part of the solution and the coefficients in the complex terms to get the complex part.
\item \( a \in [-101, -97] \text{ and } b \in [-78, -68] \)

 $-98 - 74 i$, which corresponds to adding a minus sign in both terms.
\item \( a \in [-63, -56] \text{ and } b \in [-109, -104] \)

 $-62 - 106 i$, which corresponds to adding a minus sign in the first term.
\end{enumerate}

\textbf{General Comment:} You can treat $i$ as a variable and distribute. Just remember that $i^2=-1$, so you can continue to reduce after you distribute.
}
\litem{
Simplify the expression below and choose the interval the simplification is contained within.
\[ 19 - 11 \div 13 * 3 - (17 * 6) \]The solution is \( -85.538 \), which is option B.\begin{enumerate}[label=\Alph*.]
\item \( [119.25, 120.99] \)

 120.718, which corresponds to not distributing addition and subtraction correctly.
\item \( [-86.05, -84.92] \)

* -85.538, which is the correct option.
\item \( [-4.14, -2.99] \)

 -3.231, which corresponds to not distributing a negative correctly.
\item \( [-84.79, -83.07] \)

 -83.282, which corresponds to an Order of Operations error: not reading left-to-right for multiplication/division.
\item \( \text{None of the above} \)

 You may have gotten this by making an unanticipated error. If you got a value that is not any of the others, please let the coordinator know so they can help you figure out what happened.
\end{enumerate}

\textbf{General Comment:} While you may remember (or were taught) PEMDAS is done in order, it is actually done as P/E/MD/AS. When we are at MD or AS, we read left to right.
}
\litem{
Simplify the expression below into the form $a+bi$. Then, choose the intervals that $a$ and $b$ belong to.
\[ \frac{-54 - 33 i}{1 + 4 i} \]The solution is \( -10.94  + 10.76 i \), which is option A.\begin{enumerate}[label=\Alph*.]
\item \( a \in [-12.5, -10.5] \text{ and } b \in [10.5, 11.5] \)

* $-10.94  + 10.76 i$, which is the correct option.
\item \( a \in [-55.5, -53] \text{ and } b \in [-8.5, -7.5] \)

 $-54.00  - 8.25 i$, which corresponds to just dividing the first term by the first term and the second by the second.
\item \( a \in [-187, -185.5] \text{ and } b \in [10.5, 11.5] \)

 $-186.00  + 10.76 i$, which corresponds to forgetting to multiply the conjugate by the numerator and using a plus instead of a minus in the denominator.
\item \( a \in [-12.5, -10.5] \text{ and } b \in [181.5, 183.5] \)

 $-10.94  + 183.00 i$, which corresponds to forgetting to multiply the conjugate by the numerator.
\item \( a \in [4, 6.5] \text{ and } b \in [-15.5, -14] \)

 $4.59  - 14.65 i$, which corresponds to forgetting to multiply the conjugate by the numerator and not computing the conjugate correctly.
\end{enumerate}

\textbf{General Comment:} Multiply the numerator and denominator by the *conjugate* of the denominator, then simplify. For example, if we have $2+3i$, the conjugate is $2-3i$.
}
\litem{
Choose the \textbf{smallest} set of Real numbers that the number below belongs to.
\[ \sqrt{\frac{1056}{8}} \]The solution is \( \text{Irrational} \), which is option E.\begin{enumerate}[label=\Alph*.]
\item \( \text{Whole} \)

These are the counting numbers with 0 (0, 1, 2, 3, ...)
\item \( \text{Rational} \)

These are numbers that can be written as fraction of Integers (e.g., -2/3)
\item \( \text{Not a Real number} \)

These are Nonreal Complex numbers \textbf{OR} things that are not numbers (e.g., dividing by 0).
\item \( \text{Integer} \)

These are the negative and positive counting numbers (..., -3, -2, -1, 0, 1, 2, 3, ...)
\item \( \text{Irrational} \)

* This is the correct option!
\end{enumerate}

\textbf{General Comment:} First, you \textbf{NEED} to simplify the expression. This question simplifies to $\sqrt{132}$. 
 
 Be sure you look at the simplified fraction and not just the decimal expansion. Numbers such as 13, 17, and 19 provide \textbf{long but repeating/terminating decimal expansions!} 
 
 The only ways to *not* be a Real number are: dividing by 0 or taking the square root of a negative number. 
 
 Irrational numbers are more than just square root of 3: adding or subtracting values from square root of 3 is also irrational.
}
\litem{
Simplify the expression below into the form $a+bi$. Then, choose the intervals that $a$ and $b$ belong to.
\[ \frac{18 + 44 i}{-7 + 6 i} \]The solution is \( 1.62  - 4.89 i \), which is option B.\begin{enumerate}[label=\Alph*.]
\item \( a \in [-4.5, -2] \text{ and } b \in [6.5, 8] \)

 $-2.57  + 7.33 i$, which corresponds to just dividing the first term by the first term and the second by the second.
\item \( a \in [0.5, 3] \text{ and } b \in [-5.5, -4.5] \)

* $1.62  - 4.89 i$, which is the correct option.
\item \( a \in [0.5, 3] \text{ and } b \in [-417, -415] \)

 $1.62  - 416.00 i$, which corresponds to forgetting to multiply the conjugate by the numerator.
\item \( a \in [-5.5, -4.5] \text{ and } b \in [-3, -1] \)

 $-4.59  - 2.35 i$, which corresponds to forgetting to multiply the conjugate by the numerator and not computing the conjugate correctly.
\item \( a \in [137.5, 139] \text{ and } b \in [-5.5, -4.5] \)

 $138.00  - 4.89 i$, which corresponds to forgetting to multiply the conjugate by the numerator and using a plus instead of a minus in the denominator.
\end{enumerate}

\textbf{General Comment:} Multiply the numerator and denominator by the *conjugate* of the denominator, then simplify. For example, if we have $2+3i$, the conjugate is $2-3i$.
}
\litem{
Simplify the expression below and choose the interval the simplification is contained within.
\[ 10 - 12^2 + 17 \div 20 * 5 \div 9 \]The solution is \( -133.528 \), which is option B.\begin{enumerate}[label=\Alph*.]
\item \( [154.38, 154.75] \)

 154.472, which corresponds to an Order of Operations error: multiplying by negative before squaring. For example: $(-3)^2 \neq -3^2$
\item \( [-133.6, -133.17] \)

* -133.528, this is the correct option
\item \( [153.79, 154.07] \)

 154.019, which corresponds to two Order of Operations errors.
\item \( [-134.22, -133.93] \)

 -133.981, which corresponds to an Order of Operations error: not reading left-to-right for multiplication/division.
\item \( \text{None of the above} \)

 You may have gotten this by making an unanticipated error. If you got a value that is not any of the others, please let the coordinator know so they can help you figure out what happened.
\end{enumerate}

\textbf{General Comment:} While you may remember (or were taught) PEMDAS is done in order, it is actually done as P/E/MD/AS. When we are at MD or AS, we read left to right.
}
\litem{
Choose the \textbf{smallest} set of Complex numbers that the number below belongs to.
\[ \frac{0}{6 \pi}+\sqrt{5}i \]The solution is \( \text{Pure Imaginary} \), which is option C.\begin{enumerate}[label=\Alph*.]
\item \( \text{Irrational} \)

These cannot be written as a fraction of Integers. Remember: $\pi$ is not an Integer!
\item \( \text{Not a Complex Number} \)

This is not a number. The only non-Complex number we know is dividing by 0 as this is not a number!
\item \( \text{Pure Imaginary} \)

* This is the correct option!
\item \( \text{Nonreal Complex} \)

This is a Complex number $(a+bi)$ that is not Real (has $i$ as part of the number).
\item \( \text{Rational} \)

These are numbers that can be written as fraction of Integers (e.g., -2/3 + 5)
\end{enumerate}

\textbf{General Comment:} Be sure to simplify $i^2 = -1$. This may remove the imaginary portion for your number. If you are having trouble, you may want to look at the \textit{Subgroups of the Real Numbers} section.
}
\litem{
Simplify the expression below into the form $a+bi$. Then, choose the intervals that $a$ and $b$ belong to.
\[ (8 + 10 i)(-2 - 9 i) \]The solution is \( 74 - 92 i \), which is option A.\begin{enumerate}[label=\Alph*.]
\item \( a \in [72, 80] \text{ and } b \in [-95.3, -90.7] \)

* $74 - 92 i$, which is the correct option.
\item \( a \in [72, 80] \text{ and } b \in [90.7, 94.9] \)

 $74 + 92 i$, which corresponds to adding a minus sign in both terms.
\item \( a \in [-107, -104] \text{ and } b \in [-53.1, -48.6] \)

 $-106 - 52 i$, which corresponds to adding a minus sign in the first term.
\item \( a \in [-20, -11] \text{ and } b \in [-90.6, -89.3] \)

 $-16 - 90 i$, which corresponds to just multiplying the real terms to get the real part of the solution and the coefficients in the complex terms to get the complex part.
\item \( a \in [-107, -104] \text{ and } b \in [49.2, 52.2] \)

 $-106 + 52 i$, which corresponds to adding a minus sign in the second term.
\end{enumerate}

\textbf{General Comment:} You can treat $i$ as a variable and distribute. Just remember that $i^2=-1$, so you can continue to reduce after you distribute.
}
\litem{
Choose the \textbf{smallest} set of Real numbers that the number below belongs to.
\[ \sqrt{\frac{560}{8}} \]The solution is \( \text{Irrational} \), which is option B.\begin{enumerate}[label=\Alph*.]
\item \( \text{Rational} \)

These are numbers that can be written as fraction of Integers (e.g., -2/3)
\item \( \text{Irrational} \)

* This is the correct option!
\item \( \text{Not a Real number} \)

These are Nonreal Complex numbers \textbf{OR} things that are not numbers (e.g., dividing by 0).
\item \( \text{Whole} \)

These are the counting numbers with 0 (0, 1, 2, 3, ...)
\item \( \text{Integer} \)

These are the negative and positive counting numbers (..., -3, -2, -1, 0, 1, 2, 3, ...)
\end{enumerate}

\textbf{General Comment:} First, you \textbf{NEED} to simplify the expression. This question simplifies to $\sqrt{70}$. 
 
 Be sure you look at the simplified fraction and not just the decimal expansion. Numbers such as 13, 17, and 19 provide \textbf{long but repeating/terminating decimal expansions!} 
 
 The only ways to *not* be a Real number are: dividing by 0 or taking the square root of a negative number. 
 
 Irrational numbers are more than just square root of 3: adding or subtracting values from square root of 3 is also irrational.
}
\end{enumerate}

\end{document}