\documentclass[14pt]{extbook}
\usepackage{multicol, enumerate, enumitem, hyperref, color, soul, setspace, parskip, fancyhdr} %General Packages
\usepackage{amssymb, amsthm, amsmath, latexsym, units, mathtools} %Math Packages
\everymath{\displaystyle} %All math in Display Style
% Packages with additional options
\usepackage[headsep=0.5cm,headheight=12pt, left=1 in,right= 1 in,top= 1 in,bottom= 1 in]{geometry}
\usepackage[usenames,dvipsnames]{xcolor}
\usepackage{dashrule}  % Package to use the command below to create lines between items
\newcommand{\litem}[1]{\item#1\hspace*{-1cm}\rule{\textwidth}{0.4pt}}
\pagestyle{fancy}
\lhead{Makeup Progress Quiz 2}
\chead{}
\rhead{Version B}
\lfoot{5763-3522}
\cfoot{}
\rfoot{Spring 2021}
\begin{document}

\begin{enumerate}
\litem{
Factor the polynomial below completely. Then, choose the intervals the zeros of the polynomial belong to, where $z_1 \leq z_2 \leq z_3$. \textit{To make the problem easier, all zeros are between -5 and 5.}\[ f(x) = 12x^{3} -29 x^{2} -15 x + 50 \]\begin{enumerate}[label=\Alph*.]
\item \( z_1 \in [-1.61, -0.88], \text{   }  z_2 \in [1.57, 1.67], \text{   and   } z_3 \in [1.37, 2.08] \)
\item \( z_1 \in [-1.19, -0.67], \text{   }  z_2 \in [0.5, 0.8], \text{   and   } z_3 \in [1.37, 2.08] \)
\item \( z_1 \in [-2.28, -1.85], \text{   }  z_2 \in [-1.78, -1.59], \text{   and   } z_3 \in [1.21, 1.48] \)
\item \( z_1 \in [-2.28, -1.85], \text{   }  z_2 \in [-0.44, -0.3], \text{   and   } z_3 \in [4.54, 5.07] \)
\item \( z_1 \in [-2.28, -1.85], \text{   }  z_2 \in [-0.61, -0.43], \text{   and   } z_3 \in [0.26, 1.13] \)

\end{enumerate} }
\litem{
Factor the polynomial below completely, knowing that $x+4$ is a factor. Then, choose the intervals the zeros of the polynomial belong to, where $z_1 \leq z_2 \leq z_3 \leq z_4$. \textit{To make the problem easier, all zeros are between -5 and 5.}\[ f(x) = 12x^{4} +13 x^{3} -253 x^{2} -512 x -240 \]\begin{enumerate}[label=\Alph*.]
\item \( z_1 \in [-4.6, -3.2], \text{   }  z_2 \in [-2.95, -0.11], z_3 \in [-2, 0.6], \text{   and   } z_4 \in [4.06, 5.01] \)
\item \( z_1 \in [-4.6, -3.2], \text{   }  z_2 \in [-2.95, -0.11], z_3 \in [-2, 0.6], \text{   and   } z_4 \in [4.06, 5.01] \)
\item \( z_1 \in [-6, -4.3], \text{   }  z_2 \in [0.36, 1.25], z_3 \in [1.1, 2.4], \text{   and   } z_4 \in [3.9, 4.7] \)
\item \( z_1 \in [-6, -4.3], \text{   }  z_2 \in [-0.27, 0.43], z_3 \in [2.4, 3.3], \text{   and   } z_4 \in [3.9, 4.7] \)
\item \( z_1 \in [-6, -4.3], \text{   }  z_2 \in [0.36, 1.25], z_3 \in [1.1, 2.4], \text{   and   } z_4 \in [3.9, 4.7] \)

\end{enumerate} }
\litem{
Factor the polynomial below completely, knowing that $x-5$ is a factor. Then, choose the intervals the zeros of the polynomial belong to, where $z_1 \leq z_2 \leq z_3 \leq z_4$. \textit{To make the problem easier, all zeros are between -5 and 5.}\[ f(x) = 8x^{4} -58 x^{3} +79 x^{2} +85 x -150 \]\begin{enumerate}[label=\Alph*.]
\item \( z_1 \in [-1.2, -0.72], \text{   }  z_2 \in [0.4, 1.1], z_3 \in [1.72, 2.14], \text{   and   } z_4 \in [4.96, 5.07] \)
\item \( z_1 \in [-1.77, -1.12], \text{   }  z_2 \in [1.1, 1.9], z_3 \in [1.72, 2.14], \text{   and   } z_4 \in [4.96, 5.07] \)
\item \( z_1 \in [-5.99, -4.14], \text{   }  z_2 \in [-4.1, -2.4], z_3 \in [-2.21, -1.94], \text{   and   } z_4 \in [0.62, 0.74] \)
\item \( z_1 \in [-5.99, -4.14], \text{   }  z_2 \in [-2.3, -0.6], z_3 \in [-1.78, -1.34], \text{   and   } z_4 \in [1.19, 1.27] \)
\item \( z_1 \in [-5.99, -4.14], \text{   }  z_2 \in [-2.3, -0.6], z_3 \in [-0.9, -0.57], \text{   and   } z_4 \in [0.74, 0.8] \)

\end{enumerate} }
\litem{
Perform the division below. Then, find the intervals that correspond to the quotient in the form $ax^2+bx+c$ and remainder $r$.\[ \frac{15x^{3} +38 x^{2} -37}{x + 2} \]\begin{enumerate}[label=\Alph*.]
\item \( a \in [12, 16], b \in [-9, -4], c \in [21, 25], \text{ and } r \in [-104, -97]. \)
\item \( a \in [-34, -24], b \in [98, 100], c \in [-197, -194], \text{ and } r \in [349, 357]. \)
\item \( a \in [12, 16], b \in [4, 12], c \in [-21, -9], \text{ and } r \in [-8, -1]. \)
\item \( a \in [12, 16], b \in [60, 70], c \in [133, 144], \text{ and } r \in [234, 237]. \)
\item \( a \in [-34, -24], b \in [-24, -20], c \in [-44, -42], \text{ and } r \in [-132, -121]. \)

\end{enumerate} }
\litem{
Perform the division below. Then, find the intervals that correspond to the quotient in the form $ax^2+bx+c$ and remainder $r$.\[ \frac{8x^{3} -14 x^{2} -19 x + 25}{x -2} \]\begin{enumerate}[label=\Alph*.]
\item \( a \in [6, 10], \text{   } b \in [-8, -1], \text{   } c \in [-32, -17], \text{   and   } r \in [-3, 6]. \)
\item \( a \in [6, 10], \text{   } b \in [-35, -27], \text{   } c \in [40, 43], \text{   and   } r \in [-64, -52]. \)
\item \( a \in [16, 24], \text{   } b \in [-46, -38], \text{   } c \in [73, 74], \text{   and   } r \in [-124, -115]. \)
\item \( a \in [6, 10], \text{   } b \in [-5, 6], \text{   } c \in [-19, -9], \text{   and   } r \in [-5, -2]. \)
\item \( a \in [16, 24], \text{   } b \in [13, 19], \text{   } c \in [15, 21], \text{   and   } r \in [57, 62]. \)

\end{enumerate} }
\litem{
Factor the polynomial below completely. Then, choose the intervals the zeros of the polynomial belong to, where $z_1 \leq z_2 \leq z_3$. \textit{To make the problem easier, all zeros are between -5 and 5.}\[ f(x) = 6x^{3} -29 x^{2} +14 x + 24 \]\begin{enumerate}[label=\Alph*.]
\item \( z_1 \in [-5.5, -3.9], \text{   }  z_2 \in [-1, -0.5], \text{   and   } z_3 \in [1.27, 1.57] \)
\item \( z_1 \in [-1, -0.3], \text{   }  z_2 \in [1.4, 1.9], \text{   and   } z_3 \in [3.82, 4.15] \)
\item \( z_1 \in [-5.5, -3.9], \text{   }  z_2 \in [-3.9, -2.7], \text{   and   } z_3 \in [0.17, 0.38] \)
\item \( z_1 \in [-5.5, -3.9], \text{   }  z_2 \in [-1.9, -0.7], \text{   and   } z_3 \in [0.42, 0.76] \)
\item \( z_1 \in [-2.2, -1.2], \text{   }  z_2 \in [0.3, 0.9], \text{   and   } z_3 \in [3.82, 4.15] \)

\end{enumerate} }
\litem{
What are the \textit{possible Integer} roots of the polynomial below?\[ f(x) = 4x^{3} +2 x^{2} +3 x + 5 \]\begin{enumerate}[label=\Alph*.]
\item \( \pm 1,\pm 5 \)
\item \( \pm 1,\pm 2,\pm 4 \)
\item \( \text{ All combinations of: }\frac{\pm 1,\pm 5}{\pm 1,\pm 2,\pm 4} \)
\item \( \text{ All combinations of: }\frac{\pm 1,\pm 2,\pm 4}{\pm 1,\pm 5} \)
\item \( \text{There is no formula or theorem that tells us all possible Integer roots.} \)

\end{enumerate} }
\litem{
Perform the division below. Then, find the intervals that correspond to the quotient in the form $ax^2+bx+c$ and remainder $r$.\[ \frac{20x^{3} +65 x^{2} -41}{x + 3} \]\begin{enumerate}[label=\Alph*.]
\item \( a \in [15, 24], b \in [-1, 9], c \in [-18, -12], \text{ and } r \in [2, 8]. \)
\item \( a \in [-63, -58], b \in [-119, -114], c \in [-347, -343], \text{ and } r \in [-1078, -1070]. \)
\item \( a \in [15, 24], b \in [-15, -10], c \in [60, 64], \text{ and } r \in [-282, -272]. \)
\item \( a \in [-63, -58], b \in [242, 247], c \in [-739, -729], \text{ and } r \in [2163, 2166]. \)
\item \( a \in [15, 24], b \in [124, 129], c \in [372, 376], \text{ and } r \in [1082, 1085]. \)

\end{enumerate} }
\litem{
Perform the division below. Then, find the intervals that correspond to the quotient in the form $ax^2+bx+c$ and remainder $r$.\[ \frac{20x^{3} -17 x^{2} -40 x -15}{x -2} \]\begin{enumerate}[label=\Alph*.]
\item \( a \in [17, 28], \text{   } b \in [-58, -56], \text{   } c \in [73, 84], \text{   and   } r \in [-166, -161]. \)
\item \( a \in [17, 28], \text{   } b \in [1, 4], \text{   } c \in [-38, -33], \text{   and   } r \in [-52, -50]. \)
\item \( a \in [17, 28], \text{   } b \in [22, 24], \text{   } c \in [6, 10], \text{   and   } r \in [-3, -2]. \)
\item \( a \in [37, 44], \text{   } b \in [-98, -90], \text{   } c \in [151, 158], \text{   and   } r \in [-325, -322]. \)
\item \( a \in [37, 44], \text{   } b \in [60, 71], \text{   } c \in [82, 89], \text{   and   } r \in [153, 162]. \)

\end{enumerate} }
\litem{
What are the \textit{possible Rational} roots of the polynomial below?\[ f(x) = 5x^{4} +5 x^{3} +7 x^{2} +6 x + 2 \]\begin{enumerate}[label=\Alph*.]
\item \( \pm 1,\pm 5 \)
\item \( \text{ All combinations of: }\frac{\pm 1,\pm 2}{\pm 1,\pm 5} \)
\item \( \text{ All combinations of: }\frac{\pm 1,\pm 5}{\pm 1,\pm 2} \)
\item \( \pm 1,\pm 2 \)
\item \( \text{ There is no formula or theorem that tells us all possible Rational roots.} \)

\end{enumerate} }
\end{enumerate}

\end{document}