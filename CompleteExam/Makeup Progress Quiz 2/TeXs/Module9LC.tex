\documentclass[14pt]{extbook}
\usepackage{multicol, enumerate, enumitem, hyperref, color, soul, setspace, parskip, fancyhdr} %General Packages
\usepackage{amssymb, amsthm, amsmath, latexsym, units, mathtools} %Math Packages
\everymath{\displaystyle} %All math in Display Style
% Packages with additional options
\usepackage[headsep=0.5cm,headheight=12pt, left=1 in,right= 1 in,top= 1 in,bottom= 1 in]{geometry}
\usepackage[usenames,dvipsnames]{xcolor}
\usepackage{dashrule}  % Package to use the command below to create lines between items
\newcommand{\litem}[1]{\item#1\hspace*{-1cm}\rule{\textwidth}{0.4pt}}
\pagestyle{fancy}
\lhead{Makeup Progress Quiz 2}
\chead{}
\rhead{Version C}
\lfoot{5763-3522}
\cfoot{}
\rfoot{Spring 2021}
\begin{document}

\begin{enumerate}
\litem{
Find the inverse of the function below (if it exists). Then, evaluate the inverse at $x = 10$ and choose the interval that $f^{-1}(10)$ belongs to.\[ f(x) = 3 x^2 - 4 \]\begin{enumerate}[label=\Alph*.]
\item \( f^{-1}(10) \in [1.98, 2.31] \)
\item \( f^{-1}(10) \in [3.1, 3.42] \)
\item \( f^{-1}(10) \in [3.43, 4.59] \)
\item \( f^{-1}(10) \in [1.18, 1.51] \)
\item \( \text{ The function is not invertible for all Real numbers. } \)

\end{enumerate} }
\litem{
Subtract the following functions, then choose the domain of the resulting function from the list below.\[ f(x) = 2x^{4} +5 x^{3} +7 x^{2} +5 x + 8 \text{ and } g(x) = \sqrt{-5x-18}  \]\begin{enumerate}[label=\Alph*.]
\item \( \text{ The domain is all Real numbers except } x = a, \text{ where } a \in [-9.2, -2.2] \)
\item \( \text{ The domain is all Real numbers less than or equal to } x = a, \text{ where } a \in [-4.6, -1.6] \)
\item \( \text{ The domain is all Real numbers greater than or equal to } x = a, \text{ where } a \in [-7, 3] \)
\item \( \text{ The domain is all Real numbers except } x = a \text{ and } x = b, \text{ where } a \in [4.8, 7.8] \text{ and } b \in [4.6, 7.6] \)
\item \( \text{ The domain is all Real numbers. } \)

\end{enumerate} }
\litem{
Determine whether the function below is 1-1.\[ f(x) = \sqrt{5 x - 31} \]\begin{enumerate}[label=\Alph*.]
\item \( \text{No, because there is a $y$-value that goes to 2 different $x$-values.} \)
\item \( \text{No, because the domain of the function is not $(-\infty, \infty)$.} \)
\item \( \text{Yes, the function is 1-1.} \)
\item \( \text{No, because the range of the function is not $(-\infty, \infty)$.} \)
\item \( \text{No, because there is an $x$-value that goes to 2 different $y$-values.} \)

\end{enumerate} }
\litem{
Find the inverse of the function below. Then, evaluate the inverse at $x = 9$ and choose the interval that $f^{-1}(9)$ belongs to.\[ f(x) = e^{x+4}+5 \]\begin{enumerate}[label=\Alph*.]
\item \( f^{-1}(9) \in [7.36, 7.62] \)
\item \( f^{-1}(9) \in [-2.73, -2.58] \)
\item \( f^{-1}(9) \in [5.01, 5.74] \)
\item \( f^{-1}(9) \in [7.59, 7.68] \)
\item \( f^{-1}(9) \in [6.57, 7.46] \)

\end{enumerate} }
\litem{
Find the inverse of the function below. Then, evaluate the inverse at $x = 7$ and choose the interval that $f^{-1}(7)$ belongs to.\[ f(x) = e^{x+5}+2 \]\begin{enumerate}[label=\Alph*.]
\item \( f^{-1}(7) \in [5.44, 7.75] \)
\item \( f^{-1}(7) \in [-4.57, -3.11] \)
\item \( f^{-1}(7) \in [4.44, 4.74] \)
\item \( f^{-1}(7) \in [3.99, 4.23] \)
\item \( f^{-1}(7) \in [1.65, 3.06] \)

\end{enumerate} }
\litem{
Determine whether the function below is 1-1.\[ f(x) = \sqrt{6 x - 38} \]\begin{enumerate}[label=\Alph*.]
\item \( \text{No, because the range of the function is not $(-\infty, \infty)$.} \)
\item \( \text{Yes, the function is 1-1.} \)
\item \( \text{No, because the domain of the function is not $(-\infty, \infty)$.} \)
\item \( \text{No, because there is an $x$-value that goes to 2 different $y$-values.} \)
\item \( \text{No, because there is a $y$-value that goes to 2 different $x$-values.} \)

\end{enumerate} }
\litem{
Subtract the following functions, then choose the domain of the resulting function from the list below.\[ f(x) = 9x^{3} +6 x^{2} +5 x + 8 \text{ and } g(x) = \sqrt{4x-30}  \]\begin{enumerate}[label=\Alph*.]
\item \( \text{ The domain is all Real numbers except } x = a, \text{ where } a \in [-4.25, 9.75] \)
\item \( \text{ The domain is all Real numbers greater than or equal to } x = a, \text{ where } a \in [4.5, 14.5] \)
\item \( \text{ The domain is all Real numbers less than or equal to } x = a, \text{ where } a \in [3.8, 6.8] \)
\item \( \text{ The domain is all Real numbers except } x = a \text{ and } x = b, \text{ where } a \in [5.4, 6.4] \text{ and } b \in [-8.8, 0.2] \)
\item \( \text{ The domain is all Real numbers. } \)

\end{enumerate} }
\litem{
Choose the interval below that $f$ composed with $g$ at $x=1$ is in.\[ f(x) = -2x^{3} +3 x^{2} -4 x + 3 \text{ and } g(x) = -3x^{3} +3 x^{2} +2 x \]\begin{enumerate}[label=\Alph*.]
\item \( (f \circ g)(1) \in [-11.4, -7.3] \)
\item \( (f \circ g)(1) \in [-7.6, -4.8] \)
\item \( (f \circ g)(1) \in [-1.7, 4.1] \)
\item \( (f \circ g)(1) \in [-5.5, -2.9] \)
\item \( \text{It is not possible to compose the two functions.} \)

\end{enumerate} }
\litem{
Choose the interval below that $f$ composed with $g$ at $x=-1$ is in.\[ f(x) = -2x^{3} -1 x^{2} -x -2 \text{ and } g(x) = 2x^{3} -1 x^{2} -3 x \]\begin{enumerate}[label=\Alph*.]
\item \( (f \circ g)(-1) \in [8.9, 14.2] \)
\item \( (f \circ g)(-1) \in [-3.2, -0.6] \)
\item \( (f \circ g)(-1) \in [-1.5, 2.9] \)
\item \( (f \circ g)(-1) \in [-13.8, -11.2] \)
\item \( \text{It is not possible to compose the two functions.} \)

\end{enumerate} }
\litem{
Find the inverse of the function below (if it exists). Then, evaluate the inverse at $x = 10$ and choose the interval the $f^{-1}(10)$ belongs to.\[ f(x) = \sqrt[3]{3 x - 5} \]\begin{enumerate}[label=\Alph*.]
\item \( f^{-1}(10) \in [-331.67, -324.67] \)
\item \( f^{-1}(10) \in [-337, -332] \)
\item \( f^{-1}(10) \in [332, 338] \)
\item \( f^{-1}(10) \in [326.67, 334.67] \)
\item \( \text{ The function is not invertible for all Real numbers. } \)

\end{enumerate} }
\end{enumerate}

\end{document}