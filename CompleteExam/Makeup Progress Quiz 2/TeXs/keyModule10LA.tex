\documentclass{extbook}[14pt]
\usepackage{multicol, enumerate, enumitem, hyperref, color, soul, setspace, parskip, fancyhdr, amssymb, amsthm, amsmath, latexsym, units, mathtools}
\everymath{\displaystyle}
\usepackage[headsep=0.5cm,headheight=0cm, left=1 in,right= 1 in,top= 1 in,bottom= 1 in]{geometry}
\usepackage{dashrule}  % Package to use the command below to create lines between items
\newcommand{\litem}[1]{\item #1

\rule{\textwidth}{0.4pt}}
\pagestyle{fancy}
\lhead{}
\chead{Answer Key for Makeup Progress Quiz 2 Version A}
\rhead{}
\lfoot{5763-3522}
\cfoot{}
\rfoot{Spring 2021}
\begin{document}
\textbf{This key should allow you to understand why you choose the option you did (beyond just getting a question right or wrong). \href{https://xronos.clas.ufl.edu/mac1105spring2020/courseDescriptionAndMisc/Exams/LearningFromResults}{More instructions on how to use this key can be found here}.}

\textbf{If you have a suggestion to make the keys better, \href{https://forms.gle/CZkbZmPbC9XALEE88}{please fill out the short survey here}.}

\textit{Note: This key is auto-generated and may contain issues and/or errors. The keys are reviewed after each exam to ensure grading is done accurately. If there are issues (like duplicate options), they are noted in the offline gradebook. The keys are a work-in-progress to give students as many resources to improve as possible.}

\rule{\textwidth}{0.4pt}

\begin{enumerate}\litem{
Factor the polynomial below completely. Then, choose the intervals the zeros of the polynomial belong to, where $z_1 \leq z_2 \leq z_3$. \textit{To make the problem easier, all zeros are between -5 and 5.}
\[ f(x) = 10x^{3} -23 x^{2} -88 x + 80 \]The solution is \( [-2.5, 0.8, 4] \), which is option E.\begin{enumerate}[label=\Alph*.]
\item \( z_1 \in [-5, -3], \text{   }  z_2 \in [-1.35, -1.06], \text{   and   } z_3 \in [-0.2, 1.4] \)

 Distractor 3: Corresponds to negatives of all zeros AND inversing rational roots.
\item \( z_1 \in [-0.4, 1.6], \text{   }  z_2 \in [1.11, 1.3], \text{   and   } z_3 \in [3.2, 4.7] \)

 Distractor 2: Corresponds to inversing rational roots.
\item \( z_1 \in [-5, -3], \text{   }  z_2 \in [-0.93, -0.79], \text{   and   } z_3 \in [2, 3] \)

 Distractor 1: Corresponds to negatives of all zeros.
\item \( z_1 \in [-5, -3], \text{   }  z_2 \in [-0.43, -0.09], \text{   and   } z_3 \in [4.1, 5.5] \)

 Distractor 4: Corresponds to moving factors from one rational to another.
\item \( z_1 \in [-2.5, -1.5], \text{   }  z_2 \in [0.6, 0.82], \text{   and   } z_3 \in [3.2, 4.7] \)

* This is the solution!
\end{enumerate}

\textbf{General Comment:} Remember to try the middle-most integers first as these normally are the zeros. Also, once you get it to a quadratic, you can use your other factoring techniques to finish factoring.
}
\litem{
Factor the polynomial below completely, knowing that $x+5$ is a factor. Then, choose the intervals the zeros of the polynomial belong to, where $z_1 \leq z_2 \leq z_3 \leq z_4$. \textit{To make the problem easier, all zeros are between -5 and 5.}
\[ f(x) = 6x^{4} +13 x^{3} -144 x^{2} -325 x -150 \]The solution is \( [-5, -1.5, -0.6666666666666666, 5] \), which is option B.\begin{enumerate}[label=\Alph*.]
\item \( z_1 \in [-5, -4], \text{   }  z_2 \in [0.56, 0.73], z_3 \in [-0.3, 2.1], \text{   and   } z_4 \in [5, 7] \)

 Distractor 3: Corresponds to negatives of all zeros AND inversing rational roots.
\item \( z_1 \in [-5, -4], \text{   }  z_2 \in [-1.55, -1.48], z_3 \in [-2.7, -0.6], \text{   and   } z_4 \in [5, 7] \)

* This is the solution!
\item \( z_1 \in [-5, -4], \text{   }  z_2 \in [0.56, 0.73], z_3 \in [-0.3, 2.1], \text{   and   } z_4 \in [5, 7] \)

 Distractor 1: Corresponds to negatives of all zeros.
\item \( z_1 \in [-5, -4], \text{   }  z_2 \in [0.21, 0.5], z_3 \in [2.8, 3.9], \text{   and   } z_4 \in [5, 7] \)

 Distractor 4: Corresponds to moving factors from one rational to another.
\item \( z_1 \in [-5, -4], \text{   }  z_2 \in [-1.55, -1.48], z_3 \in [-2.7, -0.6], \text{   and   } z_4 \in [5, 7] \)

 Distractor 2: Corresponds to inversing rational roots.
\end{enumerate}

\textbf{General Comment:} Remember to try the middle-most integers first as these normally are the zeros. Also, once you get it to a quadratic, you can use your other factoring techniques to finish factoring.
}
\litem{
Factor the polynomial below completely, knowing that $x+5$ is a factor. Then, choose the intervals the zeros of the polynomial belong to, where $z_1 \leq z_2 \leq z_3 \leq z_4$. \textit{To make the problem easier, all zeros are between -5 and 5.}
\[ f(x) = 15x^{4} +154 x^{3} +461 x^{2} +290 x -200 \]The solution is \( [-5, -4, -1.6666666666666667, 0.4] \), which is option D.\begin{enumerate}[label=\Alph*.]
\item \( z_1 \in [-3.21, -1.62], \text{   }  z_2 \in [-0.5, 1], z_3 \in [3, 4.3], \text{   and   } z_4 \in [4, 6] \)

 Distractor 3: Corresponds to negatives of all zeros AND inversing rational roots.
\item \( z_1 \in [-0.56, -0.26], \text{   }  z_2 \in [0.9, 3], z_3 \in [3, 4.3], \text{   and   } z_4 \in [4, 6] \)

 Distractor 1: Corresponds to negatives of all zeros.
\item \( z_1 \in [-5.03, -4.39], \text{   }  z_2 \in [-4.5, -3.5], z_3 \in [-0.7, 0.1], \text{   and   } z_4 \in [2.5, 4.5] \)

 Distractor 2: Corresponds to inversing rational roots.
\item \( z_1 \in [-5.03, -4.39], \text{   }  z_2 \in [-4.5, -3.5], z_3 \in [-2.5, -1.1], \text{   and   } z_4 \in [0.4, 1.4] \)

* This is the solution!
\item \( z_1 \in [-0.27, 0.65], \text{   }  z_2 \in [3.8, 4.1], z_3 \in [4.9, 5.5], \text{   and   } z_4 \in [4, 6] \)

 Distractor 4: Corresponds to moving factors from one rational to another.
\end{enumerate}

\textbf{General Comment:} Remember to try the middle-most integers first as these normally are the zeros. Also, once you get it to a quadratic, you can use your other factoring techniques to finish factoring.
}
\litem{
Perform the division below. Then, find the intervals that correspond to the quotient in the form $ax^2+bx+c$ and remainder $r$.
\[ \frac{15x^{3} +35 x^{2} -15}{x + 2} \]The solution is \( 15x^{2} +5 x -10 + \frac{5}{x + 2} \), which is option E.\begin{enumerate}[label=\Alph*.]
\item \( a \in [12, 18], b \in [-13, -6], c \in [23, 32], \text{ and } r \in [-106, -103]. \)

 You multipled by the synthetic number and subtracted rather than adding during synthetic division.
\item \( a \in [-30, -27], b \in [-26, -21], c \in [-55, -49], \text{ and } r \in [-115, -112]. \)

 You divided by the opposite of the factor AND multipled the first factor rather than just bringing it down.
\item \( a \in [12, 18], b \in [63, 67], c \in [124, 131], \text{ and } r \in [241, 248]. \)

 You divided by the opposite of the factor.
\item \( a \in [-30, -27], b \in [92, 96], c \in [-191, -185], \text{ and } r \in [362, 368]. \)

 You multipled by the synthetic number rather than bringing the first factor down.
\item \( a \in [12, 18], b \in [4, 10], c \in [-13, -5], \text{ and } r \in [4, 7]. \)

* This is the solution!
\end{enumerate}

\textbf{General Comment:} Be sure to synthetically divide by the zero of the denominator! Also, make sure to include 0 placeholders for missing terms.
}
\litem{
Perform the division below. Then, find the intervals that correspond to the quotient in the form $ax^2+bx+c$ and remainder $r$.
\[ \frac{12x^{3} +44 x^{2} +4 x -57}{x + 3} \]The solution is \( 12x^{2} +8 x -20 + \frac{3}{x + 3} \), which is option E.\begin{enumerate}[label=\Alph*.]
\item \( a \in [-41, -27], \text{   } b \in [146, 153], \text{   } c \in [-456, -449], \text{   and   } r \in [1295, 1303]. \)

 You multiplied by the synthetic number rather than bringing the first factor down.
\item \( a \in [10, 15], \text{   } b \in [-4, 5], \text{   } c \in [20, 21], \text{   and   } r \in [-147, -136]. \)

 You multiplied by the synthetic number and subtracted rather than adding during synthetic division.
\item \( a \in [10, 15], \text{   } b \in [79, 85], \text{   } c \in [239, 252], \text{   and   } r \in [673, 676]. \)

 You divided by the opposite of the factor.
\item \( a \in [-41, -27], \text{   } b \in [-65, -61], \text{   } c \in [-189, -186], \text{   and   } r \in [-621, -617]. \)

 You divided by the opposite of the factor AND multiplied the first factor rather than just bringing it down.
\item \( a \in [10, 15], \text{   } b \in [8, 9], \text{   } c \in [-23, -15], \text{   and   } r \in [-1, 6]. \)

* This is the solution!
\end{enumerate}

\textbf{General Comment:} Be sure to synthetically divide by the zero of the denominator!
}
\litem{
Factor the polynomial below completely. Then, choose the intervals the zeros of the polynomial belong to, where $z_1 \leq z_2 \leq z_3$. \textit{To make the problem easier, all zeros are between -5 and 5.}
\[ f(x) = 10x^{3} +47 x^{2} +16 x -48 \]The solution is \( [-4, -1.5, 0.8] \), which is option D.\begin{enumerate}[label=\Alph*.]
\item \( z_1 \in [-0.5, -0.16], \text{   }  z_2 \in [2.92, 3.62], \text{   and   } z_3 \in [3.5, 4.58] \)

 Distractor 4: Corresponds to moving factors from one rational to another.
\item \( z_1 \in [-4.21, -3.82], \text{   }  z_2 \in [-0.83, -0.6], \text{   and   } z_3 \in [1.18, 1.38] \)

 Distractor 2: Corresponds to inversing rational roots.
\item \( z_1 \in [-1.65, -0.91], \text{   }  z_2 \in [0.19, 1.25], \text{   and   } z_3 \in [3.5, 4.58] \)

 Distractor 3: Corresponds to negatives of all zeros AND inversing rational roots.
\item \( z_1 \in [-4.21, -3.82], \text{   }  z_2 \in [-1.76, -1.17], \text{   and   } z_3 \in [0.53, 0.83] \)

* This is the solution!
\item \( z_1 \in [-0.82, -0.77], \text{   }  z_2 \in [1.31, 1.88], \text{   and   } z_3 \in [3.5, 4.58] \)

 Distractor 1: Corresponds to negatives of all zeros.
\end{enumerate}

\textbf{General Comment:} Remember to try the middle-most integers first as these normally are the zeros. Also, once you get it to a quadratic, you can use your other factoring techniques to finish factoring.
}
\litem{
What are the \textit{possible Rational} roots of the polynomial below?
\[ f(x) = 6x^{2} +7 x + 4 \]The solution is \( \text{ All combinations of: }\frac{\pm 1,\pm 2,\pm 4}{\pm 1,\pm 2,\pm 3,\pm 6} \), which is option C.\begin{enumerate}[label=\Alph*.]
\item \( \pm 1,\pm 2,\pm 3,\pm 6 \)

 Distractor 1: Corresponds to the plus or minus factors of a1 only.
\item \( \text{ All combinations of: }\frac{\pm 1,\pm 2,\pm 3,\pm 6}{\pm 1,\pm 2,\pm 4} \)

 Distractor 3: Corresponds to the plus or minus of the inverse quotient (an/a0) of the factors. 
\item \( \text{ All combinations of: }\frac{\pm 1,\pm 2,\pm 4}{\pm 1,\pm 2,\pm 3,\pm 6} \)

* This is the solution \textbf{since we asked for the possible Rational roots}!
\item \( \pm 1,\pm 2,\pm 4 \)

This would have been the solution \textbf{if asked for the possible Integer roots}!
\item \( \text{ There is no formula or theorem that tells us all possible Rational roots.} \)

 Distractor 4: Corresponds to not recalling the theorem for rational roots of a polynomial.
\end{enumerate}

\textbf{General Comment:} We have a way to find the possible Rational roots. The possible Integer roots are the Integers in this list.
}
\litem{
Perform the division below. Then, find the intervals that correspond to the quotient in the form $ax^2+bx+c$ and remainder $r$.
\[ \frac{8x^{3} +42 x^{2} -53}{x + 5} \]The solution is \( 8x^{2} +2 x -10 + \frac{-3}{x + 5} \), which is option D.\begin{enumerate}[label=\Alph*.]
\item \( a \in [-42, -37], b \in [-165, -155], c \in [-792, -788], \text{ and } r \in [-4006, -3999]. \)

 You divided by the opposite of the factor AND multipled the first factor rather than just bringing it down.
\item \( a \in [8, 14], b \in [-7, -2], c \in [33, 37], \text{ and } r \in [-270, -265]. \)

 You multipled by the synthetic number and subtracted rather than adding during synthetic division.
\item \( a \in [-42, -37], b \in [242, 244], c \in [-1217, -1206], \text{ and } r \in [5996, 5999]. \)

 You multipled by the synthetic number rather than bringing the first factor down.
\item \( a \in [8, 14], b \in [0, 6], c \in [-11, -4], \text{ and } r \in [-4, 2]. \)

* This is the solution!
\item \( a \in [8, 14], b \in [78, 85], c \in [404, 411], \text{ and } r \in [1995, 1999]. \)

 You divided by the opposite of the factor.
\end{enumerate}

\textbf{General Comment:} Be sure to synthetically divide by the zero of the denominator! Also, make sure to include 0 placeholders for missing terms.
}
\litem{
Perform the division below. Then, find the intervals that correspond to the quotient in the form $ax^2+bx+c$ and remainder $r$.
\[ \frac{4x^{3} -4 x^{2} -32 x + 45}{x + 3} \]The solution is \( 4x^{2} -16 x + 16 + \frac{-3}{x + 3} \), which is option B.\begin{enumerate}[label=\Alph*.]
\item \( a \in [2, 7], \text{   } b \in [6, 14], \text{   } c \in [-9, -5], \text{   and   } r \in [13, 24]. \)

 You divided by the opposite of the factor.
\item \( a \in [2, 7], \text{   } b \in [-17, -8], \text{   } c \in [13, 18], \text{   and   } r \in [-6, -1]. \)

* This is the solution!
\item \( a \in [2, 7], \text{   } b \in [-22, -19], \text{   } c \in [43, 50], \text{   and   } r \in [-151, -144]. \)

 You multiplied by the synthetic number and subtracted rather than adding during synthetic division.
\item \( a \in [-16, -6], \text{   } b \in [32, 35], \text{   } c \in [-128, -127], \text{   and   } r \in [428, 432]. \)

 You multiplied by the synthetic number rather than bringing the first factor down.
\item \( a \in [-16, -6], \text{   } b \in [-45, -36], \text{   } c \in [-153, -146], \text{   and   } r \in [-417, -404]. \)

 You divided by the opposite of the factor AND multiplied the first factor rather than just bringing it down.
\end{enumerate}

\textbf{General Comment:} Be sure to synthetically divide by the zero of the denominator!
}
\litem{
What are the \textit{possible Rational} roots of the polynomial below?
\[ f(x) = 6x^{2} +5 x + 5 \]The solution is \( \text{ All combinations of: }\frac{\pm 1,\pm 5}{\pm 1,\pm 2,\pm 3,\pm 6} \), which is option A.\begin{enumerate}[label=\Alph*.]
\item \( \text{ All combinations of: }\frac{\pm 1,\pm 5}{\pm 1,\pm 2,\pm 3,\pm 6} \)

* This is the solution \textbf{since we asked for the possible Rational roots}!
\item \( \pm 1,\pm 2,\pm 3,\pm 6 \)

 Distractor 1: Corresponds to the plus or minus factors of a1 only.
\item \( \pm 1,\pm 5 \)

This would have been the solution \textbf{if asked for the possible Integer roots}!
\item \( \text{ All combinations of: }\frac{\pm 1,\pm 2,\pm 3,\pm 6}{\pm 1,\pm 5} \)

 Distractor 3: Corresponds to the plus or minus of the inverse quotient (an/a0) of the factors. 
\item \( \text{ There is no formula or theorem that tells us all possible Rational roots.} \)

 Distractor 4: Corresponds to not recalling the theorem for rational roots of a polynomial.
\end{enumerate}

\textbf{General Comment:} We have a way to find the possible Rational roots. The possible Integer roots are the Integers in this list.
}
\end{enumerate}

\end{document}