\documentclass{extbook}[14pt]
\usepackage{multicol, enumerate, enumitem, hyperref, color, soul, setspace, parskip, fancyhdr, amssymb, amsthm, amsmath, latexsym, units, mathtools}
\everymath{\displaystyle}
\usepackage[headsep=0.5cm,headheight=0cm, left=1 in,right= 1 in,top= 1 in,bottom= 1 in]{geometry}
\usepackage{dashrule}  % Package to use the command below to create lines between items
\newcommand{\litem}[1]{\item #1

\rule{\textwidth}{0.4pt}}
\pagestyle{fancy}
\lhead{}
\chead{Answer Key for Makeup Progress Quiz 2 Version C}
\rhead{}
\lfoot{5763-3522}
\cfoot{}
\rfoot{Spring 2021}
\begin{document}
\textbf{This key should allow you to understand why you choose the option you did (beyond just getting a question right or wrong). \href{https://xronos.clas.ufl.edu/mac1105spring2020/courseDescriptionAndMisc/Exams/LearningFromResults}{More instructions on how to use this key can be found here}.}

\textbf{If you have a suggestion to make the keys better, \href{https://forms.gle/CZkbZmPbC9XALEE88}{please fill out the short survey here}.}

\textit{Note: This key is auto-generated and may contain issues and/or errors. The keys are reviewed after each exam to ensure grading is done accurately. If there are issues (like duplicate options), they are noted in the offline gradebook. The keys are a work-in-progress to give students as many resources to improve as possible.}

\rule{\textwidth}{0.4pt}

\begin{enumerate}\litem{
Solve the linear inequality below. Then, choose the constant and interval combination that describes the solution set.
\[ 7x + 3 < 8x -9 \]The solution is \( (12.0, \infty) \), which is option D.\begin{enumerate}[label=\Alph*.]
\item \( (-\infty, a), \text{ where } a \in [-13, -10] \)

 $(-\infty, -12.0)$, which corresponds to switching the direction of the interval AND negating the endpoint. You likely did this if you did not flip the inequality when dividing by a negative as well as not moving values over to a side properly.
\item \( (a, \infty), \text{ where } a \in [-18, -8] \)

 $(-12.0, \infty)$, which corresponds to negating the endpoint of the solution.
\item \( (-\infty, a), \text{ where } a \in [7, 13] \)

 $(-\infty, 12.0)$, which corresponds to switching the direction of the interval. You likely did this if you did not flip the inequality when dividing by a negative!
\item \( (a, \infty), \text{ where } a \in [9, 14] \)

* $(12.0, \infty)$, which is the correct option.
\item \( \text{None of the above}. \)

You may have chosen this if you thought the inequality did not match the ends of the intervals.
\end{enumerate}

\textbf{General Comment:} Remember that less/greater than or equal to includes the endpoint, while less/greater do not. Also, remember that you need to flip the inequality when you multiply or divide by a negative.
}
\litem{
Using an interval or intervals, describe all the $x$-values within or including a distance of the given values.
\[ \text{ Less than } 4 \text{ units from the number } -4. \]The solution is \( (-8, 0) \), which is option C.\begin{enumerate}[label=\Alph*.]
\item \( [-8, 0] \)

This describes the values no more than 4 from -4
\item \( (-\infty, -8) \cup (0, \infty) \)

This describes the values more than 4 from -4
\item \( (-8, 0) \)

This describes the values less than 4 from -4
\item \( (-\infty, -8] \cup [0, \infty) \)

This describes the values no less than 4 from -4
\item \( \text{None of the above} \)

You likely thought the values in the interval were not correct.
\end{enumerate}

\textbf{General Comment:} When thinking about this language, it helps to draw a number line and try points.
}
\litem{
Solve the linear inequality below. Then, choose the constant and interval combination that describes the solution set.
\[ -8 + 8 x > 9 x \text{ or } 4 + 9 x < 10 x \]The solution is \( (-\infty, -8.0) \text{ or } (4.0, \infty) \), which is option C.\begin{enumerate}[label=\Alph*.]
\item \( (-\infty, a) \cup (b, \infty), \text{ where } a \in [-5, -2] \text{ and } b \in [5, 9] \)

Corresponds to inverting the inequality and negating the solution.
\item \( (-\infty, a] \cup [b, \infty), \text{ where } a \in [-8, -7] \text{ and } b \in [2, 6] \)

Corresponds to including the endpoints (when they should be excluded).
\item \( (-\infty, a) \cup (b, \infty), \text{ where } a \in [-9, -7] \text{ and } b \in [3, 6] \)

 * Correct option.
\item \( (-\infty, a] \cup [b, \infty), \text{ where } a \in [-5, -1] \text{ and } b \in [7, 12] \)

Corresponds to including the endpoints AND negating.
\item \( (-\infty, \infty) \)

Corresponds to the variable canceling, which does not happen in this instance.
\end{enumerate}

\textbf{General Comment:} When multiplying or dividing by a negative, flip the sign.
}
\litem{
Solve the linear inequality below. Then, choose the constant and interval combination that describes the solution set.
\[ 5x + 5 \geq 10x -3 \]The solution is \( (-\infty, 1.6] \), which is option A.\begin{enumerate}[label=\Alph*.]
\item \( (-\infty, a], \text{ where } a \in [-1.4, 6.6] \)

* $(-\infty, 1.6]$, which is the correct option.
\item \( (-\infty, a], \text{ where } a \in [-1.6, -0.6] \)

 $(-\infty, -1.6]$, which corresponds to negating the endpoint of the solution.
\item \( [a, \infty), \text{ where } a \in [-0.6, 3.4] \)

 $[1.6, \infty)$, which corresponds to switching the direction of the interval. You likely did this if you did not flip the inequality when dividing by a negative!
\item \( [a, \infty), \text{ where } a \in [-3.4, -0.4] \)

 $[-1.6, \infty)$, which corresponds to switching the direction of the interval AND negating the endpoint. You likely did this if you did not flip the inequality when dividing by a negative as well as not moving values over to a side properly.
\item \( \text{None of the above}. \)

You may have chosen this if you thought the inequality did not match the ends of the intervals.
\end{enumerate}

\textbf{General Comment:} Remember that less/greater than or equal to includes the endpoint, while less/greater do not. Also, remember that you need to flip the inequality when you multiply or divide by a negative.
}
\litem{
Using an interval or intervals, describe all the $x$-values within or including a distance of the given values.
\[ \text{ Less than } 8 \text{ units from the number } -1. \]The solution is \( (-9, 7) \), which is option A.\begin{enumerate}[label=\Alph*.]
\item \( (-9, 7) \)

This describes the values less than 8 from -1
\item \( (-\infty, -9] \cup [7, \infty) \)

This describes the values no less than 8 from -1
\item \( [-9, 7] \)

This describes the values no more than 8 from -1
\item \( (-\infty, -9) \cup (7, \infty) \)

This describes the values more than 8 from -1
\item \( \text{None of the above} \)

You likely thought the values in the interval were not correct.
\end{enumerate}

\textbf{General Comment:} When thinking about this language, it helps to draw a number line and try points.
}
\litem{
Solve the linear inequality below. Then, choose the constant and interval combination that describes the solution set.
\[ \frac{4}{8} - \frac{6}{9} x \geq \frac{-3}{3} x + \frac{7}{6} \]The solution is \( [2.0, \infty) \), which is option C.\begin{enumerate}[label=\Alph*.]
\item \( [a, \infty), \text{ where } a \in [-2, -1] \)

 $[-2.0, \infty)$, which corresponds to negating the endpoint of the solution.
\item \( (-\infty, a], \text{ where } a \in [2, 7] \)

 $(-\infty, 2.0]$, which corresponds to switching the direction of the interval. You likely did this if you did not flip the inequality when dividing by a negative!
\item \( [a, \infty), \text{ where } a \in [2, 3] \)

* $[2.0, \infty)$, which is the correct option.
\item \( (-\infty, a], \text{ where } a \in [-3, -1] \)

 $(-\infty, -2.0]$, which corresponds to switching the direction of the interval AND negating the endpoint. You likely did this if you did not flip the inequality when dividing by a negative as well as not moving values over to a side properly.
\item \( \text{None of the above}. \)

You may have chosen this if you thought the inequality did not match the ends of the intervals.
\end{enumerate}

\textbf{General Comment:} Remember that less/greater than or equal to includes the endpoint, while less/greater do not. Also, remember that you need to flip the inequality when you multiply or divide by a negative.
}
\litem{
Solve the linear inequality below. Then, choose the constant and interval combination that describes the solution set.
\[ -9 + 6 x < \frac{56 x + 9}{9} \leq -7 + 6 x \]The solution is \( \text{None of the above.} \), which is option E.\begin{enumerate}[label=\Alph*.]
\item \( (a, b], \text{ where } a \in [42, 47] \text{ and } b \in [33, 41] \)

$(45.00, 36.00]$, which is the correct interval but negatives of the actual endpoints.
\item \( (-\infty, a) \cup [b, \infty), \text{ where } a \in [42, 49] \text{ and } b \in [35, 41] \)

$(-\infty, 45.00) \cup [36.00, \infty)$, which corresponds to displaying the and-inequality as an or-inequality and getting negatives of the actual endpoints.
\item \( [a, b), \text{ where } a \in [45, 51] \text{ and } b \in [36, 39] \)

$[45.00, 36.00)$, which corresponds to flipping the inequality and getting negatives of the actual endpoints.
\item \( (-\infty, a] \cup (b, \infty), \text{ where } a \in [44, 50] \text{ and } b \in [34, 39] \)

$(-\infty, 45.00] \cup (36.00, \infty)$, which corresponds to displaying the and-inequality as an or-inequality AND flipping the inequality AND getting negatives of the actual endpoints.
\item \( \text{None of the above.} \)

* This is correct as the answer should be $(-45.00, -36.00]$.
\end{enumerate}

\textbf{General Comment:} To solve, you will need to break up the compound inequality into two inequalities. Be sure to keep track of the inequality! It may be best to draw a number line and graph your solution.
}
\litem{
Solve the linear inequality below. Then, choose the constant and interval combination that describes the solution set.
\[ -4 + 8 x < \frac{76 x + 9}{9} \leq 9 + 3 x \]The solution is \( \text{None of the above.} \), which is option E.\begin{enumerate}[label=\Alph*.]
\item \( (a, b], \text{ where } a \in [10.25, 13.25] \text{ and } b \in [-1.47, -0.47] \)

$(11.25, -1.47]$, which is the correct interval but negatives of the actual endpoints.
\item \( [a, b), \text{ where } a \in [11.25, 13.25] \text{ and } b \in [-4.47, -0.47] \)

$[11.25, -1.47)$, which corresponds to flipping the inequality and getting negatives of the actual endpoints.
\item \( (-\infty, a] \cup (b, \infty), \text{ where } a \in [9.25, 19.25] \text{ and } b \in [-4.47, 0.53] \)

$(-\infty, 11.25] \cup (-1.47, \infty)$, which corresponds to displaying the and-inequality as an or-inequality AND flipping the inequality AND getting negatives of the actual endpoints.
\item \( (-\infty, a) \cup [b, \infty), \text{ where } a \in [10.25, 13.25] \text{ and } b \in [-3.47, -0.47] \)

$(-\infty, 11.25) \cup [-1.47, \infty)$, which corresponds to displaying the and-inequality as an or-inequality and getting negatives of the actual endpoints.
\item \( \text{None of the above.} \)

* This is correct as the answer should be $(-11.25, 1.47]$.
\end{enumerate}

\textbf{General Comment:} To solve, you will need to break up the compound inequality into two inequalities. Be sure to keep track of the inequality! It may be best to draw a number line and graph your solution.
}
\litem{
Solve the linear inequality below. Then, choose the constant and interval combination that describes the solution set.
\[ -6 + 3 x > 5 x \text{ or } -5 + 3 x < 6 x \]The solution is \( (-\infty, -3.0) \text{ or } (-1.667, \infty) \), which is option B.\begin{enumerate}[label=\Alph*.]
\item \( (-\infty, a] \cup [b, \infty), \text{ where } a \in [-5, 0] \text{ and } b \in [-6.67, -0.67] \)

Corresponds to including the endpoints (when they should be excluded).
\item \( (-\infty, a) \cup (b, \infty), \text{ where } a \in [-5, -1] \text{ and } b \in [-4.67, 0.33] \)

 * Correct option.
\item \( (-\infty, a) \cup (b, \infty), \text{ where } a \in [-0.33, 5.67] \text{ and } b \in [-1, 6] \)

Corresponds to inverting the inequality and negating the solution.
\item \( (-\infty, a] \cup [b, \infty), \text{ where } a \in [-2.33, 2.67] \text{ and } b \in [3, 5] \)

Corresponds to including the endpoints AND negating.
\item \( (-\infty, \infty) \)

Corresponds to the variable canceling, which does not happen in this instance.
\end{enumerate}

\textbf{General Comment:} When multiplying or dividing by a negative, flip the sign.
}
\litem{
Solve the linear inequality below. Then, choose the constant and interval combination that describes the solution set.
\[ \frac{5}{9} - \frac{6}{4} x < \frac{-5}{8} x - \frac{4}{5} \]The solution is \( (1.549, \infty) \), which is option D.\begin{enumerate}[label=\Alph*.]
\item \( (a, \infty), \text{ where } a \in [-1.55, 1.45] \)

 $(-1.549, \infty)$, which corresponds to negating the endpoint of the solution.
\item \( (-\infty, a), \text{ where } a \in [-4.55, 0.45] \)

 $(-\infty, -1.549)$, which corresponds to switching the direction of the interval AND negating the endpoint. You likely did this if you did not flip the inequality when dividing by a negative as well as not moving values over to a side properly.
\item \( (-\infty, a), \text{ where } a \in [0.55, 4.55] \)

 $(-\infty, 1.549)$, which corresponds to switching the direction of the interval. You likely did this if you did not flip the inequality when dividing by a negative!
\item \( (a, \infty), \text{ where } a \in [0.55, 3.55] \)

* $(1.549, \infty)$, which is the correct option.
\item \( \text{None of the above}. \)

You may have chosen this if you thought the inequality did not match the ends of the intervals.
\end{enumerate}

\textbf{General Comment:} Remember that less/greater than or equal to includes the endpoint, while less/greater do not. Also, remember that you need to flip the inequality when you multiply or divide by a negative.
}
\end{enumerate}

\end{document}