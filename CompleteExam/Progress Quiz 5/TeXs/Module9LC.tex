\documentclass[14pt]{extbook}
\usepackage{multicol, enumerate, enumitem, hyperref, color, soul, setspace, parskip, fancyhdr} %General Packages
\usepackage{amssymb, amsthm, amsmath, bbm, latexsym, units, mathtools} %Math Packages
\everymath{\displaystyle} %All math in Display Style
% Packages with additional options
\usepackage[headsep=0.5cm,headheight=12pt, left=1 in,right= 1 in,top= 1 in,bottom= 1 in]{geometry}
\usepackage[usenames,dvipsnames]{xcolor}
\usepackage{dashrule}  % Package to use the command below to create lines between items
\newcommand{\litem}[1]{\item#1\hspace*{-1cm}\rule{\textwidth}{0.4pt}}
\pagestyle{fancy}
\lhead{Progress Quiz 5}
\chead{}
\rhead{Version C}
\lfoot{9912-2038}
\cfoot{}
\rfoot{Spring 2021}
\begin{document}

\begin{enumerate}
\litem{
Find the inverse of the function below (if it exists). Then, evaluate the inverse at $x = -12$ and choose the interval that $f^{-1}(-12)$ belongs to.\[ f(x) = 4 x^2 - 5 \]\begin{enumerate}[label=\Alph*.]
\item \( f^{-1}(-12) \in [2.15, 3.38] \)
\item \( f^{-1}(-12) \in [1.31, 1.64] \)
\item \( f^{-1}(-12) \in [5.6, 7.09] \)
\item \( f^{-1}(-12) \in [1.78, 2.28] \)
\item \( \text{ The function is not invertible for all Real numbers. } \)

\end{enumerate} }
\litem{
Choose the interval below that $f$ composed with $g$ at $x=-1$ is in.\[ f(x) = -2x^{3} -4 x^{2} -4 x \text{ and } g(x) = 2x^{3} -3 x^{2} -4 x \]\begin{enumerate}[label=\Alph*.]
\item \( (f \circ g)(-1) \in [0, 4] \)
\item \( (f \circ g)(-1) \in [-4, 1] \)
\item \( (f \circ g)(-1) \in [-4, 1] \)
\item \( (f \circ g)(-1) \in [-13, -9] \)
\item \( \text{It is not possible to compose the two functions.} \)

\end{enumerate} }
\litem{
Find the inverse of the function below. Then, evaluate the inverse at $x = 8$ and choose the interval that $f^{-1}(8)$ belongs to.\[ f(x) = e^{x+3}+2 \]\begin{enumerate}[label=\Alph*.]
\item \( f^{-1}(8) \in [4.35, 4.46] \)
\item \( f^{-1}(8) \in [4.53, 4.96] \)
\item \( f^{-1}(8) \in [3.57, 3.73] \)
\item \( f^{-1}(8) \in [4.25, 4.31] \)
\item \( f^{-1}(8) \in [-1.3, -1.13] \)

\end{enumerate} }
\litem{
Determine whether the function below is 1-1.\[ f(x) = \sqrt{4 x - 16} \]\begin{enumerate}[label=\Alph*.]
\item \( \text{No, because the range of the function is not $(-\infty, \infty)$.} \)
\item \( \text{Yes, the function is 1-1.} \)
\item \( \text{No, because the domain of the function is not $(-\infty, \infty)$.} \)
\item \( \text{No, because there is an $x$-value that goes to 2 different $y$-values.} \)
\item \( \text{No, because there is a $y$-value that goes to 2 different $x$-values.} \)

\end{enumerate} }
\litem{
Determine whether the function below is 1-1.\[ f(x) = -12 x^2 + 11 x + 56 \]\begin{enumerate}[label=\Alph*.]
\item \( \text{No, because the range of the function is not $(-\infty, \infty)$.} \)
\item \( \text{No, because there is an $x$-value that goes to 2 different $y$-values.} \)
\item \( \text{Yes, the function is 1-1.} \)
\item \( \text{No, because there is a $y$-value that goes to 2 different $x$-values.} \)
\item \( \text{No, because the domain of the function is not $(-\infty, \infty)$.} \)

\end{enumerate} }
\litem{
Choose the interval below that $f$ composed with $g$ at $x=-1$ is in.\[ f(x) = -2x^{3} +3 x^{2} +x -4 \text{ and } g(x) = 2x^{3} +2 x^{2} +2 x \]\begin{enumerate}[label=\Alph*.]
\item \( (f \circ g)(-1) \in [31, 36] \)
\item \( (f \circ g)(-1) \in [21, 27] \)
\item \( (f \circ g)(-1) \in [-8, -1] \)
\item \( (f \circ g)(-1) \in [-2, 5] \)
\item \( \text{It is not possible to compose the two functions.} \)

\end{enumerate} }
\litem{
Find the inverse of the function below (if it exists). Then, evaluate the inverse at $x = -15$ and choose the interval the $f^{-1}(-15)$ belongs to.\[ f(x) = \sqrt[3]{5 x - 4} \]\begin{enumerate}[label=\Alph*.]
\item \( f^{-1}(-15) \in [-674.97, -673.41] \)
\item \( f^{-1}(-15) \in [673.81, 675.29] \)
\item \( f^{-1}(-15) \in [-677.11, -675.68] \)
\item \( f^{-1}(-15) \in [675.31, 676.63] \)
\item \( \text{ The function is not invertible for all Real numbers. } \)

\end{enumerate} }
\litem{
Find the inverse of the function below. Then, evaluate the inverse at $x = 9$ and choose the interval that $f^{-1}(9)$ belongs to.\[ f(x) = \ln{(x-4)}+2 \]\begin{enumerate}[label=\Alph*.]
\item \( f^{-1}(9) \in [59875.14, 59882.14] \)
\item \( f^{-1}(9) \in [442414.39, 442417.39] \)
\item \( f^{-1}(9) \in [1096.63, 1103.63] \)
\item \( f^{-1}(9) \in [144.41, 155.41] \)
\item \( f^{-1}(9) \in [1087.63, 1093.63] \)

\end{enumerate} }
\litem{
Add the following functions, then choose the domain of the resulting function from the list below.\[ f(x) = 9x^{3} + x^{2} +5 x + 7 \text{ and } g(x) = \sqrt{6x+36}  \]\begin{enumerate}[label=\Alph*.]
\item \( \text{ The domain is all Real numbers less than or equal to } x = a, \text{ where } a \in [1, 4] \)
\item \( \text{ The domain is all Real numbers except } x = a, \text{ where } a \in [1.8, 7.8] \)
\item \( \text{ The domain is all Real numbers greater than or equal to } x = a, \text{ where } a \in [-6, -4] \)
\item \( \text{ The domain is all Real numbers except } x = a \text{ and } x = b, \text{ where } a \in [-0.4, 6.6] \text{ and } b \in [4.33, 9.33] \)
\item \( \text{ The domain is all Real numbers. } \)

\end{enumerate} }
\litem{
Multiply the following functions, then choose the domain of the resulting function from the list below.\[ f(x) = 6x^{3} +6 x + 9 \text{ and } g(x) = 8x^{2} +5 x + 3 \]\begin{enumerate}[label=\Alph*.]
\item \( \text{ The domain is all Real numbers except } x = a, \text{ where } a \in [1.33, 6.33] \)
\item \( \text{ The domain is all Real numbers less than or equal to } x = a, \text{ where } a \in [-7.2, -3.2] \)
\item \( \text{ The domain is all Real numbers greater than or equal to } x = a, \text{ where } a \in [7.33, 9.33] \)
\item \( \text{ The domain is all Real numbers except } x = a \text{ and } x = b, \text{ where } a \in [-6.25, 5.75] \text{ and } b \in [-11.8, -1.8] \)
\item \( \text{ The domain is all Real numbers. } \)

\end{enumerate} }
\end{enumerate}

\end{document}