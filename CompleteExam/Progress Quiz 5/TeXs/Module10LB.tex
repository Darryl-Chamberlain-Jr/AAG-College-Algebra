\documentclass[14pt]{extbook}
\usepackage{multicol, enumerate, enumitem, hyperref, color, soul, setspace, parskip, fancyhdr} %General Packages
\usepackage{amssymb, amsthm, amsmath, bbm, latexsym, units, mathtools} %Math Packages
\everymath{\displaystyle} %All math in Display Style
% Packages with additional options
\usepackage[headsep=0.5cm,headheight=12pt, left=1 in,right= 1 in,top= 1 in,bottom= 1 in]{geometry}
\usepackage[usenames,dvipsnames]{xcolor}
\usepackage{dashrule}  % Package to use the command below to create lines between items
\newcommand{\litem}[1]{\item#1\hspace*{-1cm}\rule{\textwidth}{0.4pt}}
\pagestyle{fancy}
\lhead{Progress Quiz 5}
\chead{}
\rhead{Version B}
\lfoot{9912-2038}
\cfoot{}
\rfoot{Spring 2021}
\begin{document}

\begin{enumerate}
\litem{
What are the \textit{possible Integer} roots of the polynomial below?\[ f(x) = 6x^{3} +4 x^{2} +6 x + 7 \]\begin{enumerate}[label=\Alph*.]
\item \( \text{ All combinations of: }\frac{\pm 1,\pm 7}{\pm 1,\pm 2,\pm 3,\pm 6} \)
\item \( \text{ All combinations of: }\frac{\pm 1,\pm 2,\pm 3,\pm 6}{\pm 1,\pm 7} \)
\item \( \pm 1,\pm 2,\pm 3,\pm 6 \)
\item \( \pm 1,\pm 7 \)
\item \( \text{There is no formula or theorem that tells us all possible Integer roots.} \)

\end{enumerate} }
\litem{
Factor the polynomial below completely. Then, choose the intervals the zeros of the polynomial belong to, where $z_1 \leq z_2 \leq z_3$. \textit{To make the problem easier, all zeros are between -5 and 5.}\[ f(x) = 15x^{3} +91 x^{2} +84 x + 20 \]\begin{enumerate}[label=\Alph*.]
\item \( z_1 \in [0.07, 0.25], \text{   }  z_2 \in [1.86, 2.35], \text{   and   } z_3 \in [2, 7] \)
\item \( z_1 \in [-5.17, -4.99], \text{   }  z_2 \in [-2.52, -2.48], \text{   and   } z_3 \in [-1.5, -0.5] \)
\item \( z_1 \in [-5.17, -4.99], \text{   }  z_2 \in [-1.06, -0.1], \text{   and   } z_3 \in [-1.4, 1.6] \)
\item \( z_1 \in [1.27, 1.51], \text{   }  z_2 \in [2.02, 3.25], \text{   and   } z_3 \in [2, 7] \)
\item \( z_1 \in [0.32, 0.61], \text{   }  z_2 \in [0.61, 1.51], \text{   and   } z_3 \in [2, 7] \)

\end{enumerate} }
\litem{
Perform the division below. Then, find the intervals that correspond to the quotient in the form $ax^2+bx+c$ and remainder $r$.\[ \frac{10x^{3} +30 x^{2} -35}{x + 2} \]\begin{enumerate}[label=\Alph*.]
\item \( a \in [-22, -18], b \in [68, 72], c \in [-144, -136], \text{ and } r \in [241, 251]. \)
\item \( a \in [-22, -18], b \in [-10, -6], c \in [-28, -18], \text{ and } r \in [-81, -72]. \)
\item \( a \in [2, 11], b \in [48, 55], c \in [98, 107], \text{ and } r \in [158, 170]. \)
\item \( a \in [2, 11], b \in [-1, 6], c \in [-2, 3], \text{ and } r \in [-37, -31]. \)
\item \( a \in [2, 11], b \in [4, 11], c \in [-28, -18], \text{ and } r \in [-1, 9]. \)

\end{enumerate} }
\litem{
Factor the polynomial below completely. Then, choose the intervals the zeros of the polynomial belong to, where $z_1 \leq z_2 \leq z_3$. \textit{To make the problem easier, all zeros are between -5 and 5.}\[ f(x) = 12x^{3} +35 x^{2} +7 x -30 \]\begin{enumerate}[label=\Alph*.]
\item \( z_1 \in [-2.03, -1.79], \text{   }  z_2 \in [-2.18, -1.66], \text{   and   } z_3 \in [0.5, 1.3] \)
\item \( z_1 \in [-0.9, -0.7], \text{   }  z_2 \in [1.65, 1.76], \text{   and   } z_3 \in [1.8, 3.1] \)
\item \( z_1 \in [-0.57, 0.04], \text{   }  z_2 \in [1.99, 2.02], \text{   and   } z_3 \in [4, 6] \)
\item \( z_1 \in [-2.03, -1.79], \text{   }  z_2 \in [-0.82, -0.31], \text{   and   } z_3 \in [1.3, 1.5] \)
\item \( z_1 \in [-1.75, -1.19], \text{   }  z_2 \in [0.57, 0.61], \text{   and   } z_3 \in [1.8, 3.1] \)

\end{enumerate} }
\litem{
Perform the division below. Then, find the intervals that correspond to the quotient in the form $ax^2+bx+c$ and remainder $r$.\[ \frac{6x^{3} +26 x^{2} -30}{x + 4} \]\begin{enumerate}[label=\Alph*.]
\item \( a \in [4, 8], b \in [1, 8], c \in [-15, -6], \text{ and } r \in [-3, 6]. \)
\item \( a \in [4, 8], b \in [45, 55], c \in [198, 205], \text{ and } r \in [763, 774]. \)
\item \( a \in [4, 8], b \in [-6, 0], c \in [18, 25], \text{ and } r \in [-139, -127]. \)
\item \( a \in [-27, -23], b \in [-71, -64], c \in [-284, -277], \text{ and } r \in [-1154, -1149]. \)
\item \( a \in [-27, -23], b \in [118, 129], c \in [-490, -487], \text{ and } r \in [1922, 1924]. \)

\end{enumerate} }
\litem{
Perform the division below. Then, find the intervals that correspond to the quotient in the form $ax^2+bx+c$ and remainder $r$.\[ \frac{8x^{3} -18 x^{2} -6 x + 15}{x -2} \]\begin{enumerate}[label=\Alph*.]
\item \( a \in [14, 17], \text{   } b \in [13, 21], \text{   } c \in [21, 26], \text{   and   } r \in [57, 66]. \)
\item \( a \in [7, 10], \text{   } b \in [-4, 2], \text{   } c \in [-15, -5], \text{   and   } r \in [-6, -2]. \)
\item \( a \in [7, 10], \text{   } b \in [-38, -32], \text{   } c \in [56, 66], \text{   and   } r \in [-110, -108]. \)
\item \( a \in [7, 10], \text{   } b \in [-14, -6], \text{   } c \in [-21, -14], \text{   and   } r \in [-3, 6]. \)
\item \( a \in [14, 17], \text{   } b \in [-50, -46], \text{   } c \in [89, 95], \text{   and   } r \in [-179, -163]. \)

\end{enumerate} }
\litem{
Perform the division below. Then, find the intervals that correspond to the quotient in the form $ax^2+bx+c$ and remainder $r$.\[ \frac{8x^{3} -22 x^{2} -21 x + 49}{x -3} \]\begin{enumerate}[label=\Alph*.]
\item \( a \in [4, 14], \text{   } b \in [-1, 3], \text{   } c \in [-21, -13], \text{   and   } r \in [3, 8]. \)
\item \( a \in [4, 14], \text{   } b \in [-6, 0], \text{   } c \in [-33, -31], \text{   and   } r \in [-17, -13]. \)
\item \( a \in [21, 27], \text{   } b \in [45, 52], \text{   } c \in [127, 131], \text{   and   } r \in [436, 443]. \)
\item \( a \in [4, 14], \text{   } b \in [-49, -45], \text{   } c \in [117, 124], \text{   and   } r \in [-302, -299]. \)
\item \( a \in [21, 27], \text{   } b \in [-98, -90], \text{   } c \in [261, 269], \text{   and   } r \in [-736, -730]. \)

\end{enumerate} }
\litem{
Factor the polynomial below completely, knowing that $x+3$ is a factor. Then, choose the intervals the zeros of the polynomial belong to, where $z_1 \leq z_2 \leq z_3 \leq z_4$. \textit{To make the problem easier, all zeros are between -5 and 5.}\[ f(x) = 8x^{4} -22 x^{3} -53 x^{2} +205 x -150 \]\begin{enumerate}[label=\Alph*.]
\item \( z_1 \in [-5.07, -4.81], \text{   }  z_2 \in [-2.23, -1.27], z_3 \in [-0.67, -0.55], \text{   and   } z_4 \in [2.63, 3.42] \)
\item \( z_1 \in [-2.97, -2.29], \text{   }  z_2 \in [-2.23, -1.27], z_3 \in [-1.44, -1.1], \text{   and   } z_4 \in [2.63, 3.42] \)
\item \( z_1 \in [-3.31, -2.85], \text{   }  z_2 \in [0.01, 0.55], z_3 \in [0.76, 0.85], \text{   and   } z_4 \in [1.89, 2.33] \)
\item \( z_1 \in [-2.26, -0.67], \text{   }  z_2 \in [-0.89, -0.77], z_3 \in [-0.43, -0.2], \text{   and   } z_4 \in [2.63, 3.42] \)
\item \( z_1 \in [-3.31, -2.85], \text{   }  z_2 \in [1.13, 1.73], z_3 \in [1.98, 2.07], \text{   and   } z_4 \in [2.05, 2.5] \)

\end{enumerate} }
\litem{
Factor the polynomial below completely, knowing that $x-4$ is a factor. Then, choose the intervals the zeros of the polynomial belong to, where $z_1 \leq z_2 \leq z_3 \leq z_4$. \textit{To make the problem easier, all zeros are between -5 and 5.}\[ f(x) = 15x^{4} -11 x^{3} -318 x^{2} +528 x -160 \]\begin{enumerate}[label=\Alph*.]
\item \( z_1 \in [-4.68, -3.48], \text{   }  z_2 \in [-2.64, -2.37], z_3 \in [-0.83, -0.51], \text{   and   } z_4 \in [4.59, 5.75] \)
\item \( z_1 \in [-4.68, -3.48], \text{   }  z_2 \in [-4.04, -3.53], z_3 \in [-0.25, -0.12], \text{   and   } z_4 \in [4.59, 5.75] \)
\item \( z_1 \in [-5.34, -4.95], \text{   }  z_2 \in [0.32, 0.69], z_3 \in [1.29, 1.64], \text{   and   } z_4 \in [3.28, 4.56] \)
\item \( z_1 \in [-4.68, -3.48], \text{   }  z_2 \in [-1.54, -0.8], z_3 \in [-0.59, -0.26], \text{   and   } z_4 \in [4.59, 5.75] \)
\item \( z_1 \in [-5.34, -4.95], \text{   }  z_2 \in [0.72, 1.26], z_3 \in [2.04, 2.75], \text{   and   } z_4 \in [3.28, 4.56] \)

\end{enumerate} }
\litem{
What are the \textit{possible Integer} roots of the polynomial below?\[ f(x) = 5x^{4} +2 x^{3} +7 x^{2} +7 x + 2 \]\begin{enumerate}[label=\Alph*.]
\item \( \text{ All combinations of: }\frac{\pm 1,\pm 2}{\pm 1,\pm 5} \)
\item \( \pm 1,\pm 5 \)
\item \( \pm 1,\pm 2 \)
\item \( \text{ All combinations of: }\frac{\pm 1,\pm 5}{\pm 1,\pm 2} \)
\item \( \text{There is no formula or theorem that tells us all possible Integer roots.} \)

\end{enumerate} }
\end{enumerate}

\end{document}