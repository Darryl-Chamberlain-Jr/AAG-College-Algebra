\documentclass[14pt]{extbook}
\usepackage{multicol, enumerate, enumitem, hyperref, color, soul, setspace, parskip, fancyhdr} %General Packages
\usepackage{amssymb, amsthm, amsmath, bbm, latexsym, units, mathtools} %Math Packages
\everymath{\displaystyle} %All math in Display Style
% Packages with additional options
\usepackage[headsep=0.5cm,headheight=12pt, left=1 in,right= 1 in,top= 1 in,bottom= 1 in]{geometry}
\usepackage[usenames,dvipsnames]{xcolor}
\usepackage{dashrule}  % Package to use the command below to create lines between items
\newcommand{\litem}[1]{\item#1\hspace*{-1cm}\rule{\textwidth}{0.4pt}}
\pagestyle{fancy}
\lhead{Progress Quiz 5}
\chead{}
\rhead{Version C}
\lfoot{9912-2038}
\cfoot{}
\rfoot{Spring 2021}
\begin{document}

\begin{enumerate}
\litem{
What are the \textit{possible Rational} roots of the polynomial below?\[ f(x) = 5x^{3} +5 x^{2} +7 x + 4 \]\begin{enumerate}[label=\Alph*.]
\item \( \text{ All combinations of: }\frac{\pm 1,\pm 5}{\pm 1,\pm 2,\pm 4} \)
\item \( \pm 1,\pm 5 \)
\item \( \text{ All combinations of: }\frac{\pm 1,\pm 2,\pm 4}{\pm 1,\pm 5} \)
\item \( \pm 1,\pm 2,\pm 4 \)
\item \( \text{ There is no formula or theorem that tells us all possible Rational roots.} \)

\end{enumerate} }
\litem{
Factor the polynomial below completely. Then, choose the intervals the zeros of the polynomial belong to, where $z_1 \leq z_2 \leq z_3$. \textit{To make the problem easier, all zeros are between -5 and 5.}\[ f(x) = 15x^{3} -41 x^{2} -70 x -24 \]\begin{enumerate}[label=\Alph*.]
\item \( z_1 \in [-1.6, -0.3], \text{   }  z_2 \in [-1.13, -0.54], \text{   and   } z_3 \in [3.6, 5.1] \)
\item \( z_1 \in [-4.8, -3.7], \text{   }  z_2 \in [1.46, 1.55], \text{   and   } z_3 \in [1.4, 2.7] \)
\item \( z_1 \in [-4.8, -3.7], \text{   }  z_2 \in [-0.16, 0.36], \text{   and   } z_3 \in [2.9, 3.1] \)
\item \( z_1 \in [-2.4, -1.6], \text{   }  z_2 \in [-1.57, -1.04], \text{   and   } z_3 \in [3.6, 5.1] \)
\item \( z_1 \in [-4.8, -3.7], \text{   }  z_2 \in [0.52, 1.2], \text{   and   } z_3 \in [0.4, 1.3] \)

\end{enumerate} }
\litem{
Perform the division below. Then, find the intervals that correspond to the quotient in the form $ax^2+bx+c$ and remainder $r$.\[ \frac{15x^{3} -35 x^{2} + 22}{x -2} \]\begin{enumerate}[label=\Alph*.]
\item \( a \in [25, 31], b \in [22, 26], c \in [49, 57], \text{ and } r \in [120, 130]. \)
\item \( a \in [15, 20], b \in [-20, -17], c \in [-20, -14], \text{ and } r \in [-2, 3]. \)
\item \( a \in [15, 20], b \in [-10, -4], c \in [-15, -3], \text{ and } r \in [-2, 3]. \)
\item \( a \in [25, 31], b \in [-96, -89], c \in [188, 191], \text{ and } r \in [-358, -354]. \)
\item \( a \in [15, 20], b \in [-71, -60], c \in [129, 132], \text{ and } r \in [-242, -232]. \)

\end{enumerate} }
\litem{
Factor the polynomial below completely. Then, choose the intervals the zeros of the polynomial belong to, where $z_1 \leq z_2 \leq z_3$. \textit{To make the problem easier, all zeros are between -5 and 5.}\[ f(x) = 9x^{3} -45 x^{2} -16 x + 80 \]\begin{enumerate}[label=\Alph*.]
\item \( z_1 \in [-1.51, -1.32], \text{   }  z_2 \in [1.27, 1.74], \text{   and   } z_3 \in [4.92, 5.01] \)
\item \( z_1 \in [-5.05, -4.53], \text{   }  z_2 \in [-1.86, -1.08], \text{   and   } z_3 \in [1.31, 1.51] \)
\item \( z_1 \in [-5.05, -4.53], \text{   }  z_2 \in [-4.03, -3.89], \text{   and   } z_3 \in [0.28, 0.52] \)
\item \( z_1 \in [-1.02, -0.49], \text{   }  z_2 \in [0.69, 1.02], \text{   and   } z_3 \in [4.92, 5.01] \)
\item \( z_1 \in [-5.05, -4.53], \text{   }  z_2 \in [-1.22, -0.61], \text{   and   } z_3 \in [0.48, 0.81] \)

\end{enumerate} }
\litem{
Perform the division below. Then, find the intervals that correspond to the quotient in the form $ax^2+bx+c$ and remainder $r$.\[ \frac{10x^{3} -30 x + 16}{x + 2} \]\begin{enumerate}[label=\Alph*.]
\item \( a \in [-24, -19], b \in [38, 41], c \in [-116, -104], \text{ and } r \in [235, 238]. \)
\item \( a \in [-24, -19], b \in [-43, -34], c \in [-116, -104], \text{ and } r \in [-204, -203]. \)
\item \( a \in [10, 12], b \in [-33, -29], c \in [57, 63], \text{ and } r \in [-173, -160]. \)
\item \( a \in [10, 12], b \in [-20, -18], c \in [7, 15], \text{ and } r \in [-6, 2]. \)
\item \( a \in [10, 12], b \in [15, 24], c \in [7, 15], \text{ and } r \in [34, 39]. \)

\end{enumerate} }
\litem{
Perform the division below. Then, find the intervals that correspond to the quotient in the form $ax^2+bx+c$ and remainder $r$.\[ \frac{10x^{3} -34 x^{2} +6 x + 23}{x -3} \]\begin{enumerate}[label=\Alph*.]
\item \( a \in [8, 18], \text{   } b \in [-7, -1], \text{   } c \in [-8, 0], \text{   and   } r \in [5, 8]. \)
\item \( a \in [8, 18], \text{   } b \in [-17, -9], \text{   } c \in [-24, -20], \text{   and   } r \in [-21, -19]. \)
\item \( a \in [25, 32], \text{   } b \in [-127, -121], \text{   } c \in [374, 379], \text{   and   } r \in [-1113, -1110]. \)
\item \( a \in [8, 18], \text{   } b \in [-64, -62], \text{   } c \in [196, 199], \text{   and   } r \in [-575, -565]. \)
\item \( a \in [25, 32], \text{   } b \in [56, 57], \text{   } c \in [173, 182], \text{   and   } r \in [540, 549]. \)

\end{enumerate} }
\litem{
Perform the division below. Then, find the intervals that correspond to the quotient in the form $ax^2+bx+c$ and remainder $r$.\[ \frac{6x^{3} +23 x^{2} -10 x -80}{x + 3} \]\begin{enumerate}[label=\Alph*.]
\item \( a \in [2, 10], \text{   } b \in [3, 12], \text{   } c \in [-25, -22], \text{   and   } r \in [-8, -4]. \)
\item \( a \in [-18, -10], \text{   } b \in [75, 78], \text{   } c \in [-245, -237], \text{   and   } r \in [638, 646]. \)
\item \( a \in [-18, -10], \text{   } b \in [-31, -28], \text{   } c \in [-104, -102], \text{   and   } r \in [-396, -386]. \)
\item \( a \in [2, 10], \text{   } b \in [-4, 0], \text{   } c \in [-7, -4], \text{   and   } r \in [-60, -51]. \)
\item \( a \in [2, 10], \text{   } b \in [35, 42], \text{   } c \in [113, 116], \text{   and   } r \in [254, 262]. \)

\end{enumerate} }
\litem{
Factor the polynomial below completely, knowing that $x+4$ is a factor. Then, choose the intervals the zeros of the polynomial belong to, where $z_1 \leq z_2 \leq z_3 \leq z_4$. \textit{To make the problem easier, all zeros are between -5 and 5.}\[ f(x) = 10x^{4} +43 x^{3} -67 x^{2} -376 x -240 \]\begin{enumerate}[label=\Alph*.]
\item \( z_1 \in [-3.8, -2.2], \text{   }  z_2 \in [-0.05, 0.7], z_3 \in [1.02, 1.73], \text{   and   } z_4 \in [3.97, 4.18] \)
\item \( z_1 \in [-3.8, -2.2], \text{   }  z_2 \in [0.6, 1.1], z_3 \in [2.11, 2.94], \text{   and   } z_4 \in [3.97, 4.18] \)
\item \( z_1 \in [-5.5, -3.5], \text{   }  z_2 \in [-1.3, -0.99], z_3 \in [-0.61, -0.35], \text{   and   } z_4 \in [2.93, 3.31] \)
\item \( z_1 \in [-3.8, -2.2], \text{   }  z_2 \in [-0.05, 0.7], z_3 \in [3.99, 4.29], \text{   and   } z_4 \in [4.71, 5.27] \)
\item \( z_1 \in [-5.5, -3.5], \text{   }  z_2 \in [-2.64, -2.14], z_3 \in [-0.89, -0.53], \text{   and   } z_4 \in [2.93, 3.31] \)

\end{enumerate} }
\litem{
Factor the polynomial below completely, knowing that $x+5$ is a factor. Then, choose the intervals the zeros of the polynomial belong to, where $z_1 \leq z_2 \leq z_3 \leq z_4$. \textit{To make the problem easier, all zeros are between -5 and 5.}\[ f(x) = 20x^{4} +201 x^{3} +648 x^{2} +775 x + 300 \]\begin{enumerate}[label=\Alph*.]
\item \( z_1 \in [-6.28, -4.33], \text{   }  z_2 \in [-3.6, -2], z_3 \in [-1.8, -0.5], \text{   and   } z_4 \in [-1.8, 0.2] \)
\item \( z_1 \in [-0.48, 0.33], \text{   }  z_2 \in [1.5, 4.8], z_3 \in [3.5, 5.1], \text{   and   } z_4 \in [5, 6] \)
\item \( z_1 \in [0.29, 1.75], \text{   }  z_2 \in [-0.4, 2.2], z_3 \in [2.1, 3.3], \text{   and   } z_4 \in [5, 6] \)
\item \( z_1 \in [-6.28, -4.33], \text{   }  z_2 \in [-3.6, -2], z_3 \in [-1.8, -0.5], \text{   and   } z_4 \in [-1.8, 0.2] \)
\item \( z_1 \in [0.29, 1.75], \text{   }  z_2 \in [-0.4, 2.2], z_3 \in [2.1, 3.3], \text{   and   } z_4 \in [5, 6] \)

\end{enumerate} }
\litem{
What are the \textit{possible Rational} roots of the polynomial below?\[ f(x) = 6x^{2} +5 x + 7 \]\begin{enumerate}[label=\Alph*.]
\item \( \text{ All combinations of: }\frac{\pm 1,\pm 2,\pm 3,\pm 6}{\pm 1,\pm 7} \)
\item \( \pm 1,\pm 2,\pm 3,\pm 6 \)
\item \( \pm 1,\pm 7 \)
\item \( \text{ All combinations of: }\frac{\pm 1,\pm 7}{\pm 1,\pm 2,\pm 3,\pm 6} \)
\item \( \text{ There is no formula or theorem that tells us all possible Rational roots.} \)

\end{enumerate} }
\end{enumerate}

\end{document}