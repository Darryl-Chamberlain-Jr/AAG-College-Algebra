\documentclass{extbook}[14pt]
\usepackage{multicol, enumerate, enumitem, hyperref, color, soul, setspace, parskip, fancyhdr, amssymb, amsthm, amsmath, bbm, latexsym, units, mathtools}
\everymath{\displaystyle}
\usepackage[headsep=0.5cm,headheight=0cm, left=1 in,right= 1 in,top= 1 in,bottom= 1 in]{geometry}
\usepackage{dashrule}  % Package to use the command below to create lines between items
\newcommand{\litem}[1]{\item #1

\rule{\textwidth}{0.4pt}}
\pagestyle{fancy}
\lhead{}
\chead{Answer Key for Progress Quiz 5 Version A}
\rhead{}
\lfoot{9912-2038}
\cfoot{}
\rfoot{Spring 2021}
\begin{document}
\textbf{This key should allow you to understand why you choose the option you did (beyond just getting a question right or wrong). \href{https://xronos.clas.ufl.edu/mac1105spring2020/courseDescriptionAndMisc/Exams/LearningFromResults}{More instructions on how to use this key can be found here}.}

\textbf{If you have a suggestion to make the keys better, \href{https://forms.gle/CZkbZmPbC9XALEE88}{please fill out the short survey here}.}

\textit{Note: This key is auto-generated and may contain issues and/or errors. The keys are reviewed after each exam to ensure grading is done accurately. If there are issues (like duplicate options), they are noted in the offline gradebook. The keys are a work-in-progress to give students as many resources to improve as possible.}

\rule{\textwidth}{0.4pt}

\begin{enumerate}\litem{
Choose the \textbf{smallest} set of Complex numbers that the number below belongs to.
\[ \frac{2}{-4}+\sqrt{-25}i \]The solution is \( \text{Rational} \), which is option E.\begin{enumerate}[label=\Alph*.]
\item \( \text{Not a Complex Number} \)

This is not a number. The only non-Complex number we know is dividing by 0 as this is not a number!
\item \( \text{Irrational} \)

These cannot be written as a fraction of Integers. Remember: $\pi$ is not an Integer!
\item \( \text{Pure Imaginary} \)

This is a Complex number $(a+bi)$ that \textbf{only} has an imaginary part like $2i$.
\item \( \text{Nonreal Complex} \)

This is a Complex number $(a+bi)$ that is not Real (has $i$ as part of the number).
\item \( \text{Rational} \)

* This is the correct option!
\end{enumerate}

\textbf{General Comment:} Be sure to simplify $i^2 = -1$. This may remove the imaginary portion for your number. If you are having trouble, you may want to look at the \textit{Subgroups of the Real Numbers} section.
}
\litem{
Simplify the expression below into the form $a+bi$. Then, choose the intervals that $a$ and $b$ belong to.
\[ (-8 + 9 i)(4 - 5 i) \]The solution is \( 13 + 76 i \), which is option A.\begin{enumerate}[label=\Alph*.]
\item \( a \in [13, 17] \text{ and } b \in [73, 78] \)

* $13 + 76 i$, which is the correct option.
\item \( a \in [13, 17] \text{ and } b \in [-79, -74] \)

 $13 - 76 i$, which corresponds to adding a minus sign in both terms.
\item \( a \in [-39, -30] \text{ and } b \in [-47, -40] \)

 $-32 - 45 i$, which corresponds to just multiplying the real terms to get the real part of the solution and the coefficients in the complex terms to get the complex part.
\item \( a \in [-83, -73] \text{ and } b \in [0, 6] \)

 $-77 + 4 i$, which corresponds to adding a minus sign in the first term.
\item \( a \in [-83, -73] \text{ and } b \in [-6, 2] \)

 $-77 - 4 i$, which corresponds to adding a minus sign in the second term.
\end{enumerate}

\textbf{General Comment:} You can treat $i$ as a variable and distribute. Just remember that $i^2=-1$, so you can continue to reduce after you distribute.
}
\litem{
Simplify the expression below into the form $a+bi$. Then, choose the intervals that $a$ and $b$ belong to.
\[ \frac{54 + 33 i}{-4 - 5 i} \]The solution is \( -9.29  + 3.37 i \), which is option B.\begin{enumerate}[label=\Alph*.]
\item \( a \in [-1.5, 0] \text{ and } b \in [-11, -8] \)

 $-1.24  - 9.80 i$, which corresponds to forgetting to multiply the conjugate by the numerator and not computing the conjugate correctly.
\item \( a \in [-10, -9] \text{ and } b \in [3, 3.5] \)

* $-9.29  + 3.37 i$, which is the correct option.
\item \( a \in [-10, -9] \text{ and } b \in [137.5, 139] \)

 $-9.29  + 138.00 i$, which corresponds to forgetting to multiply the conjugate by the numerator.
\item \( a \in [-381.5, -379.5] \text{ and } b \in [3, 3.5] \)

 $-381.00  + 3.37 i$, which corresponds to forgetting to multiply the conjugate by the numerator and using a plus instead of a minus in the denominator.
\item \( a \in [-14.5, -13] \text{ and } b \in [-8, -6] \)

 $-13.50  - 6.60 i$, which corresponds to just dividing the first term by the first term and the second by the second.
\end{enumerate}

\textbf{General Comment:} Multiply the numerator and denominator by the *conjugate* of the denominator, then simplify. For example, if we have $2+3i$, the conjugate is $2-3i$.
}
\litem{
Simplify the expression below into the form $a+bi$. Then, choose the intervals that $a$ and $b$ belong to.
\[ \frac{-9 + 44 i}{2 - 5 i} \]The solution is \( -8.21  + 1.48 i \), which is option D.\begin{enumerate}[label=\Alph*.]
\item \( a \in [-239, -237] \text{ and } b \in [1, 2] \)

 $-238.00  + 1.48 i$, which corresponds to forgetting to multiply the conjugate by the numerator and using a plus instead of a minus in the denominator.
\item \( a \in [-5.5, -3.5] \text{ and } b \in [-9.5, -8] \)

 $-4.50  - 8.80 i$, which corresponds to just dividing the first term by the first term and the second by the second.
\item \( a \in [6, 8] \text{ and } b \in [3.5, 5.5] \)

 $6.97  + 4.59 i$, which corresponds to forgetting to multiply the conjugate by the numerator and not computing the conjugate correctly.
\item \( a \in [-9.5, -7.5] \text{ and } b \in [1, 2] \)

* $-8.21  + 1.48 i$, which is the correct option.
\item \( a \in [-9.5, -7.5] \text{ and } b \in [42, 43.5] \)

 $-8.21  + 43.00 i$, which corresponds to forgetting to multiply the conjugate by the numerator.
\end{enumerate}

\textbf{General Comment:} Multiply the numerator and denominator by the *conjugate* of the denominator, then simplify. For example, if we have $2+3i$, the conjugate is $2-3i$.
}
\litem{
Simplify the expression below and choose the interval the simplification is contained within.
\[ 4 - 2^2 + 5 \div 17 * 15 \div 6 \]The solution is \( 0.735 \), which is option B.\begin{enumerate}[label=\Alph*.]
\item \( [-0.33, 0.08] \)

 0.003, which corresponds to an Order of Operations error: not reading left-to-right for multiplication/division.
\item \( [0.55, 1.32] \)

* 0.735, this is the correct option
\item \( [7.98, 8.4] \)

 8.003, which corresponds to two Order of Operations errors.
\item \( [8.09, 9.27] \)

 8.735, which corresponds to an Order of Operations error: multiplying by negative before squaring. For example: $(-3)^2 \neq -3^2$
\item \( \text{None of the above} \)

 You may have gotten this by making an unanticipated error. If you got a value that is not any of the others, please let the coordinator know so they can help you figure out what happened.
\end{enumerate}

\textbf{General Comment:} While you may remember (or were taught) PEMDAS is done in order, it is actually done as P/E/MD/AS. When we are at MD or AS, we read left to right.
}
\litem{
Choose the \textbf{smallest} set of Real numbers that the number below belongs to.
\[ -\sqrt{\frac{73984}{256}} \]The solution is \( \text{Integer} \), which is option A.\begin{enumerate}[label=\Alph*.]
\item \( \text{Integer} \)

* This is the correct option!
\item \( \text{Irrational} \)

These cannot be written as a fraction of Integers.
\item \( \text{Whole} \)

These are the counting numbers with 0 (0, 1, 2, 3, ...)
\item \( \text{Rational} \)

These are numbers that can be written as fraction of Integers (e.g., -2/3)
\item \( \text{Not a Real number} \)

These are Nonreal Complex numbers \textbf{OR} things that are not numbers (e.g., dividing by 0).
\end{enumerate}

\textbf{General Comment:} First, you \textbf{NEED} to simplify the expression. This question simplifies to $-272$. 
 
 Be sure you look at the simplified fraction and not just the decimal expansion. Numbers such as 13, 17, and 19 provide \textbf{long but repeating/terminating decimal expansions!} 
 
 The only ways to *not* be a Real number are: dividing by 0 or taking the square root of a negative number. 
 
 Irrational numbers are more than just square root of 3: adding or subtracting values from square root of 3 is also irrational.
}
\litem{
Choose the \textbf{smallest} set of Real numbers that the number below belongs to.
\[ \sqrt{\frac{2916}{36}} \]The solution is \( \text{Whole} \), which is option E.\begin{enumerate}[label=\Alph*.]
\item \( \text{Irrational} \)

These cannot be written as a fraction of Integers.
\item \( \text{Not a Real number} \)

These are Nonreal Complex numbers \textbf{OR} things that are not numbers (e.g., dividing by 0).
\item \( \text{Rational} \)

These are numbers that can be written as fraction of Integers (e.g., -2/3)
\item \( \text{Integer} \)

These are the negative and positive counting numbers (..., -3, -2, -1, 0, 1, 2, 3, ...)
\item \( \text{Whole} \)

* This is the correct option!
\end{enumerate}

\textbf{General Comment:} First, you \textbf{NEED} to simplify the expression. This question simplifies to $54$. 
 
 Be sure you look at the simplified fraction and not just the decimal expansion. Numbers such as 13, 17, and 19 provide \textbf{long but repeating/terminating decimal expansions!} 
 
 The only ways to *not* be a Real number are: dividing by 0 or taking the square root of a negative number. 
 
 Irrational numbers are more than just square root of 3: adding or subtracting values from square root of 3 is also irrational.
}
\litem{
Choose the \textbf{smallest} set of Complex numbers that the number below belongs to.
\[ \sqrt{\frac{144}{121}} + 64i^2 \]The solution is \( \text{Rational} \), which is option E.\begin{enumerate}[label=\Alph*.]
\item \( \text{Irrational} \)

These cannot be written as a fraction of Integers. Remember: $\pi$ is not an Integer!
\item \( \text{Pure Imaginary} \)

This is a Complex number $(a+bi)$ that \textbf{only} has an imaginary part like $2i$.
\item \( \text{Nonreal Complex} \)

This is a Complex number $(a+bi)$ that is not Real (has $i$ as part of the number).
\item \( \text{Not a Complex Number} \)

This is not a number. The only non-Complex number we know is dividing by 0 as this is not a number!
\item \( \text{Rational} \)

* This is the correct option!
\end{enumerate}

\textbf{General Comment:} Be sure to simplify $i^2 = -1$. This may remove the imaginary portion for your number. If you are having trouble, you may want to look at the \textit{Subgroups of the Real Numbers} section.
}
\litem{
Simplify the expression below into the form $a+bi$. Then, choose the intervals that $a$ and $b$ belong to.
\[ (-4 + 8 i)(-9 + 5 i) \]The solution is \( -4 - 92 i \), which is option A.\begin{enumerate}[label=\Alph*.]
\item \( a \in [-5, 1] \text{ and } b \in [-92, -91] \)

* $-4 - 92 i$, which is the correct option.
\item \( a \in [-5, 1] \text{ and } b \in [92, 94] \)

 $-4 + 92 i$, which corresponds to adding a minus sign in both terms.
\item \( a \in [29, 42] \text{ and } b \in [34, 42] \)

 $36 + 40 i$, which corresponds to just multiplying the real terms to get the real part of the solution and the coefficients in the complex terms to get the complex part.
\item \( a \in [75, 77] \text{ and } b \in [51, 57] \)

 $76 + 52 i$, which corresponds to adding a minus sign in the first term.
\item \( a \in [75, 77] \text{ and } b \in [-56, -47] \)

 $76 - 52 i$, which corresponds to adding a minus sign in the second term.
\end{enumerate}

\textbf{General Comment:} You can treat $i$ as a variable and distribute. Just remember that $i^2=-1$, so you can continue to reduce after you distribute.
}
\litem{
Simplify the expression below and choose the interval the simplification is contained within.
\[ 2 - 16^2 + 3 \div 5 * 9 \div 14 \]The solution is \( -253.614 \), which is option A.\begin{enumerate}[label=\Alph*.]
\item \( [-253.67, -253.48] \)

* -253.614, this is the correct option
\item \( [257.67, 258.35] \)

 258.005, which corresponds to two Order of Operations errors.
\item \( [258.24, 258.57] \)

 258.386, which corresponds to an Order of Operations error: multiplying by negative before squaring. For example: $(-3)^2 \neq -3^2$
\item \( [-254.36, -253.95] \)

 -253.995, which corresponds to an Order of Operations error: not reading left-to-right for multiplication/division.
\item \( \text{None of the above} \)

 You may have gotten this by making an unanticipated error. If you got a value that is not any of the others, please let the coordinator know so they can help you figure out what happened.
\end{enumerate}

\textbf{General Comment:} While you may remember (or were taught) PEMDAS is done in order, it is actually done as P/E/MD/AS. When we are at MD or AS, we read left to right.
}
\end{enumerate}

\end{document}