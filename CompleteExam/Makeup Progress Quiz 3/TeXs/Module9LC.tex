\documentclass[14pt]{extbook}
\usepackage{multicol, enumerate, enumitem, hyperref, color, soul, setspace, parskip, fancyhdr} %General Packages
\usepackage{amssymb, amsthm, amsmath, bbm, latexsym, units, mathtools} %Math Packages
\everymath{\displaystyle} %All math in Display Style
% Packages with additional options
\usepackage[headsep=0.5cm,headheight=12pt, left=1 in,right= 1 in,top= 1 in,bottom= 1 in]{geometry}
\usepackage[usenames,dvipsnames]{xcolor}
\usepackage{dashrule}  % Package to use the command below to create lines between items
\newcommand{\litem}[1]{\item#1\hspace*{-1cm}\rule{\textwidth}{0.4pt}}
\pagestyle{fancy}
\lhead{Makeup Progress Quiz 3}
\chead{}
\rhead{Version C}
\lfoot{4315-3397}
\cfoot{}
\rfoot{Fall 2020}
\begin{document}

\begin{enumerate}
\litem{
Find the inverse of the function below (if it exists). Then, evaluate the inverse at $x = 13$ and choose the interval the $f^{-1}(13)$ belongs to.\[ f(x) = \sqrt[3]{5 x - 4} \]\begin{enumerate}[label=\Alph*.]
\item \( f^{-1}(13) \in [-439.6, -437.3] \)
\item \( f^{-1}(13) \in [440, 441] \)
\item \( f^{-1}(13) \in [-442.7, -439] \)
\item \( f^{-1}(13) \in [435.5, 439] \)
\item \( \text{ The function is not invertible for all Real numbers. } \)

\end{enumerate} }
\litem{
Subtract the following functions, then choose the domain of the resulting function from the list below.\[ f(x) = \frac{2}{5x+34} \text{ and } g(x) = x^{4} +4 x^{3} + x^{2} +7 x \]\begin{enumerate}[label=\Alph*.]
\item \( \text{ The domain is all Real numbers greater than or equal to } x = a, \text{ where } a \in [-6.5, 7.5] \)
\item \( \text{ The domain is all Real numbers except } x = a, \text{ where } a \in [-6.8, -2.8] \)
\item \( \text{ The domain is all Real numbers less than or equal to } x = a, \text{ where } a \in [2.2, 7.2] \)
\item \( \text{ The domain is all Real numbers except } x = a \text{ and } x = b, \text{ where } a \in [1.75, 4.75] \text{ and } b \in [-5.67, -1.67] \)
\item \( \text{ The domain is all Real numbers. } \)

\end{enumerate} }
\litem{
Add the following functions, then choose the domain of the resulting function from the list below.\[ f(x) = \sqrt{-6x-18}  \text{ and } g(x) = 4x^{2} +2 x + 4 \]\begin{enumerate}[label=\Alph*.]
\item \( \text{ The domain is all Real numbers except } x = a, \text{ where } a \in [-7.67, -3.67] \)
\item \( \text{ The domain is all Real numbers less than or equal to } x = a, \text{ where } a \in [-4, -1] \)
\item \( \text{ The domain is all Real numbers greater than or equal to } x = a, \text{ where } a \in [-7.4, -2.4] \)
\item \( \text{ The domain is all Real numbers except } x = a \text{ and } x = b, \text{ where } a \in [-6.2, -1.2] \text{ and } b \in [-8.17, -1.17] \)
\item \( \text{ The domain is all Real numbers. } \)

\end{enumerate} }
\litem{
Choose the interval below that $f$ composed with $g$ at $x=1$ is in.\[ f(x) = -3x^{3} +3 x^{2} +4 x \text{ and } g(x) = x^{3} -1 x^{2} -2 x \]\begin{enumerate}[label=\Alph*.]
\item \( (f \circ g)(1) \in [14, 20] \)
\item \( (f \circ g)(1) \in [28, 31] \)
\item \( (f \circ g)(1) \in [47, 52] \)
\item \( (f \circ g)(1) \in [37, 46] \)
\item \( \text{It is not possible to compose the two functions.} \)

\end{enumerate} }
\litem{
Choose the interval below that $f$ composed with $g$ at $x=-1$ is in.\[ f(x) = x^{3} +2 x^{2} +4 x \text{ and } g(x) = -3x^{3} -1 x^{2} +2 x \]\begin{enumerate}[label=\Alph*.]
\item \( (f \circ g)(-1) \in [56, 65] \)
\item \( (f \circ g)(-1) \in [0, 1] \)
\item \( (f \circ g)(-1) \in [6, 14] \)
\item \( (f \circ g)(-1) \in [64, 69] \)
\item \( \text{It is not possible to compose the two functions.} \)

\end{enumerate} }
\litem{
Find the inverse of the function below (if it exists). Then, evaluate the inverse at $x = -11$ and choose the interval that $f^{-1}(-11)$ belongs to.\[ f(x) = 5 x^2 - 3 \]\begin{enumerate}[label=\Alph*.]
\item \( f^{-1}(-11) \in [0.99, 1.41] \)
\item \( f^{-1}(-11) \in [2.23, 2.55] \)
\item \( f^{-1}(-11) \in [1.51, 1.87] \)
\item \( f^{-1}(-11) \in [5.15, 5.38] \)
\item \( \text{ The function is not invertible for all Real numbers. } \)

\end{enumerate} }
\litem{
Determine whether the function below is 1-1.\[ f(x) = 16 x^2 + 32 x - 425 \]\begin{enumerate}[label=\Alph*.]
\item \( \text{No, because there is a $y$-value that goes to 2 different $x$-values.} \)
\item \( \text{No, because the domain of the function is not $(-\infty, \infty)$.} \)
\item \( \text{Yes, the function is 1-1.} \)
\item \( \text{No, because there is an $x$-value that goes to 2 different $y$-values.} \)
\item \( \text{No, because the range of the function is not $(-\infty, \infty)$.} \)

\end{enumerate} }
\litem{
Find the inverse of the function below. Then, evaluate the inverse at $x = 6$ and choose the interval that $f^{-1}(6)$ belongs to.\[ f(x) = \ln{(x+4)}+2 \]\begin{enumerate}[label=\Alph*.]
\item \( f^{-1}(6) \in [53.6, 65.6] \)
\item \( f^{-1}(6) \in [22024.47, 22031.47] \)
\item \( f^{-1}(6) \in [7.39, 12.39] \)
\item \( f^{-1}(6) \in [49.6, 54.6] \)
\item \( f^{-1}(6) \in [2974.96, 2983.96] \)

\end{enumerate} }
\litem{
Determine whether the function below is 1-1.\[ f(x) = 25 x^2 - 250 x + 625 \]\begin{enumerate}[label=\Alph*.]
\item \( \text{Yes, the function is 1-1.} \)
\item \( \text{No, because there is a $y$-value that goes to 2 different $x$-values.} \)
\item \( \text{No, because the range of the function is not $(-\infty, \infty)$.} \)
\item \( \text{No, because there is an $x$-value that goes to 2 different $y$-values.} \)
\item \( \text{No, because the domain of the function is not $(-\infty, \infty)$.} \)

\end{enumerate} }
\litem{
Find the inverse of the function below. Then, evaluate the inverse at $x = 7$ and choose the interval that $f^{-1}(7)$ belongs to.\[ f(x) = \ln{(x-2)}-2 \]\begin{enumerate}[label=\Alph*.]
\item \( f^{-1}(7) \in [148.41, 155.41] \)
\item \( f^{-1}(7) \in [8099.08, 8103.08] \)
\item \( f^{-1}(7) \in [8102.08, 8107.08] \)
\item \( f^{-1}(7) \in [146.41, 148.41] \)
\item \( f^{-1}(7) \in [8099.08, 8103.08] \)

\end{enumerate} }
\end{enumerate}

\end{document}