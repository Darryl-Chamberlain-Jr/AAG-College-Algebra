\documentclass{extbook}[14pt]
\usepackage{multicol, enumerate, enumitem, hyperref, color, soul, setspace, parskip, fancyhdr, amssymb, amsthm, amsmath, bbm, latexsym, units, mathtools}
\everymath{\displaystyle}
\usepackage[headsep=0.5cm,headheight=0cm, left=1 in,right= 1 in,top= 1 in,bottom= 1 in]{geometry}
\usepackage{dashrule}  % Package to use the command below to create lines between items
\newcommand{\litem}[1]{\item #1

\rule{\textwidth}{0.4pt}}
\pagestyle{fancy}
\lhead{}
\chead{Answer Key for Makeup Progress Quiz 3 Version C}
\rhead{}
\lfoot{4315-3397}
\cfoot{}
\rfoot{Fall 2020}
\begin{document}
\textbf{This key should allow you to understand why you choose the option you did (beyond just getting a question right or wrong). \href{https://xronos.clas.ufl.edu/mac1105spring2020/courseDescriptionAndMisc/Exams/LearningFromResults}{More instructions on how to use this key can be found here}.}

\textbf{If you have a suggestion to make the keys better, \href{https://forms.gle/CZkbZmPbC9XALEE88}{please fill out the short survey here}.}

\textit{Note: This key is auto-generated and may contain issues and/or errors. The keys are reviewed after each exam to ensure grading is done accurately. If there are issues (like duplicate options), they are noted in the offline gradebook. The keys are a work-in-progress to give students as many resources to improve as possible.}

\rule{\textwidth}{0.4pt}

\begin{enumerate}\litem{
Which of the following intervals describes the Range of the function below?
\[ f(x) = -\log_2{(x+9)}-6 \]

The solution is \( (\infty, \infty) \), which is option E.\begin{enumerate}[label=\Alph*.]
\item \( [a, \infty), a \in [-11.3, -7.6] \)

$[-6, \infty)$, which corresponds to using the flipped Domain AND including the endpoint.
\item \( (-\infty, a), a \in [-8.6, -4.1] \)

$(-\infty, -6)$, which corresponds to using the vertical shift while the Range is $(-\infty, \infty)$.
\item \( [a, \infty), a \in [8.1, 9.6] \)

$[9, \infty)$, which corresponds to using the negative of the horizontal shift AND including the endpoint.
\item \( (-\infty, a), a \in [5, 7.7] \)

$(-\infty, 6)$, which corresponds to using the using the negative of vertical shift on $(0, \infty)$.
\item \( (-\infty, \infty) \)

*This is the correct option.
\end{enumerate}

\textbf{General Comment:} \textbf{General Comments}: The domain of a basic logarithmic function is $(0, \infty)$ and the Range is $(-\infty, \infty)$. We can use shifts when finding the Domain, but the Range will always be all Real numbers.
}
\litem{
Which of the following intervals describes the Domain of the function below?
\[ f(x) = -e^{x-3}+9 \]

The solution is \( (-\infty, \infty) \), which is option E.\begin{enumerate}[label=\Alph*.]
\item \( (-\infty, a), a \in [4, 11] \)

$(-\infty, 9)$, which corresponds to using the correct vertical shift *if we wanted the Range*.
\item \( [a, \infty), a \in [-17, -7] \)

$[-9, \infty)$, which corresponds to using the negative vertical shift AND flipping the Range interval AND including the endpoint.
\item \( (a, \infty), a \in [-17, -7] \)

$(-9, \infty)$, which corresponds to using the negative vertical shift AND flipping the Range interval.
\item \( (-\infty, a], a \in [4, 11] \)

$(-\infty, 9]$, which corresponds to using the correct vertical shift *if we wanted the Range* AND including the endpoint.
\item \( (-\infty, \infty) \)

* This is the correct option.
\end{enumerate}

\textbf{General Comment:} \textbf{General Comments}: Domain of a basic exponential function is $(-\infty, \infty)$ while the Range is $(0, \infty)$. We can shift these intervals [and even flip when $a<0$!] to find the new Domain/Range.
}
\litem{
Solve the equation for $x$ and choose the interval that contains the solution (if it exists).
\[ 5^{4x-2} = \left(\frac{1}{9}\right)^{-3x+5} \]

The solution is \( x = 50.462 \), which is option B.\begin{enumerate}[label=\Alph*.]
\item \( x \in [-46.48, -38.48] \)

$x = -45.478$, which corresponds to distributing the $\ln(base)$ to the first term of the exponent only.
\item \( x \in [48.46, 53.46] \)

* $x = 50.462$, which is the correct option.
\item \( x \in [0, 3] \)

$x = 1.000$, which corresponds to solving the numerators as equal while ignoring the bases are different.
\item \( x \in [-1.11, 0.89] \)

$x = -1.110$, which corresponds to distributing the $\ln(base)$ to the second term of the exponent only.
\item \( \text{There is no Real solution to the equation.} \)

This corresponds to believing there is no solution since the bases are not powers of each other.
\end{enumerate}

\textbf{General Comment:} \textbf{General Comments:} This question was written so that the bases could not be written the same. You will need to take the log of both sides.
}
\litem{
Which of the following intervals describes the Range of the function below?
\[ f(x) = e^{x+4}-6 \]

The solution is \( (-6, \infty) \), which is option B.\begin{enumerate}[label=\Alph*.]
\item \( (-\infty, a), a \in [5, 11] \)

$(-\infty, 6)$, which corresponds to using the negative vertical shift AND flipping the Range interval.
\item \( (a, \infty), a \in [-8, -3] \)

* $(-6, \infty)$, which is the correct option.
\item \( [a, \infty), a \in [-8, -3] \)

$[-6, \infty)$, which corresponds to including the endpoint.
\item \( (-\infty, a], a \in [5, 11] \)

$(-\infty, 6]$, which corresponds to using the negative vertical shift AND flipping the Range interval AND including the endpoint.
\item \( (-\infty, \infty) \)

This corresponds to confusing range of an exponential function with the domain of an exponential function.
\end{enumerate}

\textbf{General Comment:} \textbf{General Comments}: Domain of a basic exponential function is $(-\infty, \infty)$ while the Range is $(0, \infty)$. We can shift these intervals [and even flip when $a<0$!] to find the new Domain/Range.
}
\litem{
 Solve the equation for $x$ and choose the interval that contains $x$ (if it exists).
\[  19 = \sqrt[7]{\frac{17}{e^{5x}}} \]

The solution is \( x = -3.556, \text{ which does not fit in any of the interval options.} \), which is option E.\begin{enumerate}[label=\Alph*.]
\item \( x \in [-2.61, 0.39] \)

$x = -0.611$, which corresponds to treating any root as a square root.
\item \( x \in [-29.17, -26.17] \)

$x = -27.167$, which corresponds to thinking you don't need to take the natural log of both sides before reducing, as if the right side already has a natural log.
\item \( x \in [2.56, 4.56] \)

$x = 3.556$, which is the negative of the correct solution.
\item \( \text{There is no Real solution to the equation.} \)

This corresponds to believing you cannot solve the equation.
\item \( \text{None of the above.} \)

* $x = -3.556$ is the correct solution and does not fit in any of the other intervals.
\end{enumerate}

\textbf{General Comment:} \textbf{General Comments}: After using the properties of logarithmic functions to break up the right-hand side, use $\ln(e) = 1$ to reduce the question to a linear function to solve. You can put $\ln(17)$ into a calculator if you are having trouble.
}
\litem{
Which of the following intervals describes the Domain of the function below?
\[ f(x) = -\log_2{(x-3)}-5 \]

The solution is \( (3, \infty) \), which is option B.\begin{enumerate}[label=\Alph*.]
\item \( [a, \infty), a \in [-5.15, -4.66] \)

$[-5, \infty)$, which corresponds to using the vertical shift when shifting the Domain AND including the endpoint.
\item \( (a, \infty), a \in [2.7, 3.27] \)

* $(3, \infty)$, which is the correct option.
\item \( (-\infty, a), a \in [-4.67, -2.29] \)

$(-\infty, -3)$, which corresponds to flipping the Domain. Remember: the general for is $a*\log(x-h)+k$, \textbf{where $a$ does not affect the domain}.
\item \( (-\infty, a], a \in [4.82, 5.22] \)

$(-\infty, 5]$, which corresponds to using the negative vertical shift AND including the endpoint AND flipping the domain.
\item \( (-\infty, \infty) \)

This corresponds to thinking of the range of the log function (or the domain of the exponential function).
\end{enumerate}

\textbf{General Comment:} \textbf{General Comments}: The domain of a basic logarithmic function is $(0, \infty)$ and the Range is $(-\infty, \infty)$. We can use shifts when finding the Domain, but the Range will always be all Real numbers.
}
\litem{
 Solve the equation for $x$ and choose the interval that contains $x$ (if it exists).
\[  18 = \sqrt[3]{\frac{30}{e^{5x}}} \]

The solution is \( x = -1.054, \text{ which does not fit in any of the interval options.} \), which is option E.\begin{enumerate}[label=\Alph*.]
\item \( x \in [-0.76, 0.08] \)

$x = -0.476$, which corresponds to treating any root as a square root.
\item \( x \in [0.31, 2.69] \)

$x = 1.054$, which is the negative of the correct solution.
\item \( x \in [-12.33, -10.36] \)

$x = -11.480$, which corresponds to thinking you don't need to take the natural log of both sides before reducing, as if the right side already has a natural log.
\item \( \text{There is no Real solution to the equation.} \)

This corresponds to believing you cannot solve the equation.
\item \( \text{None of the above.} \)

* $x = -1.054$ is the correct solution and does not fit in any of the other intervals.
\end{enumerate}

\textbf{General Comment:} \textbf{General Comments}: After using the properties of logarithmic functions to break up the right-hand side, use $\ln(e) = 1$ to reduce the question to a linear function to solve. You can put $\ln(30)$ into a calculator if you are having trouble.
}
\litem{
Solve the equation for $x$ and choose the interval that contains the solution (if it exists).
\[ 5^{-3x+3} = \left(\frac{1}{49}\right)^{-2x-3} \]

The solution is \( x = -0.543 \), which is option B.\begin{enumerate}[label=\Alph*.]
\item \( x \in [-0.4, 2.2] \)

$x = 0.476$, which corresponds to distributing the $\ln(base)$ to the first term of the exponent only.
\item \( x \in [-1.1, -0.1] \)

* $x = -0.543$, which is the correct option.
\item \( x \in [-8.7, -6.1] \)

$x = -6.847$, which corresponds to distributing the $\ln(base)$ to the second term of the exponent only.
\item \( x \in [5.6, 6.7] \)

$x = 6.000$, which corresponds to solving the numerators as equal while ignoring the bases are different.
\item \( \text{There is no Real solution to the equation.} \)

This corresponds to believing there is no solution since the bases are not powers of each other.
\end{enumerate}

\textbf{General Comment:} \textbf{General Comments:} This question was written so that the bases could not be written the same. You will need to take the log of both sides.
}
\litem{
Solve the equation for $x$ and choose the interval that contains the solution (if it exists).
\[ \log_{5}{(4x+5)}+4 = 3 \]

The solution is \( x = -1.200 \), which is option C.\begin{enumerate}[label=\Alph*.]
\item \( x \in [-1.81, -1.28] \)

$x = -1.500$, which corresponds to reversing the base and exponent when converting.
\item \( x \in [29.85, 30.19] \)

$x = 30.000$, which corresponds to ignoring the vertical shift when converting to exponential form.
\item \( x \in [-1.32, -1.04] \)

* $x = -1.200$, which is the correct option.
\item \( x \in [0.71, 1.13] \)

$x = 1.000$, which corresponds to reversing the base and exponent when converting and reversing the value with $x$.
\item \( \text{There is no Real solution to the equation.} \)

Corresponds to believing a negative coefficient within the log equation means there is no Real solution.
\end{enumerate}

\textbf{General Comment:} \textbf{General Comments:} First, get the equation in the form $\log_b{(cx+d)} = a$. Then, convert to $b^a = cx+d$ and solve.
}
\litem{
Solve the equation for $x$ and choose the interval that contains the solution (if it exists).
\[ \log_{3}{(-2x+7)}+5 = 3 \]

The solution is \( x = 3.444 \), which is option B.\begin{enumerate}[label=\Alph*.]
\item \( x \in [-12, -6] \)

$x = -10.000$, which corresponds to ignoring the vertical shift when converting to exponential form.
\item \( x \in [2.44, 6.44] \)

* $x = 3.444$, which is the correct option.
\item \( x \in [4.5, 14.5] \)

$x = 7.500$, which corresponds to reversing the base and exponent when converting.
\item \( x \in [-0.5, 2.5] \)

$x = 0.500$, which corresponds to reversing the base and exponent when converting and reversing the value with $x$.
\item \( \text{There is no Real solution to the equation.} \)

Corresponds to believing a negative coefficient within the log equation means there is no Real solution.
\end{enumerate}

\textbf{General Comment:} \textbf{General Comments:} First, get the equation in the form $\log_b{(cx+d)} = a$. Then, convert to $b^a = cx+d$ and solve.
}
\end{enumerate}

\end{document}