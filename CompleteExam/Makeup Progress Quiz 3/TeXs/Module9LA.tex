\documentclass[14pt]{extbook}
\usepackage{multicol, enumerate, enumitem, hyperref, color, soul, setspace, parskip, fancyhdr} %General Packages
\usepackage{amssymb, amsthm, amsmath, bbm, latexsym, units, mathtools} %Math Packages
\everymath{\displaystyle} %All math in Display Style
% Packages with additional options
\usepackage[headsep=0.5cm,headheight=12pt, left=1 in,right= 1 in,top= 1 in,bottom= 1 in]{geometry}
\usepackage[usenames,dvipsnames]{xcolor}
\usepackage{dashrule}  % Package to use the command below to create lines between items
\newcommand{\litem}[1]{\item#1\hspace*{-1cm}\rule{\textwidth}{0.4pt}}
\pagestyle{fancy}
\lhead{Makeup Progress Quiz 3}
\chead{}
\rhead{Version A}
\lfoot{4315-3397}
\cfoot{}
\rfoot{Fall 2020}
\begin{document}

\begin{enumerate}
\litem{
Find the inverse of the function below (if it exists). Then, evaluate the inverse at $x = 14$ and choose the interval the $f^{-1}(14)$ belongs to.\[ f(x) = \sqrt[3]{5 x - 3} \]\begin{enumerate}[label=\Alph*.]
\item \( f^{-1}(14) \in [-550.33, -549.18] \)
\item \( f^{-1}(14) \in [548, 548.78] \)
\item \( f^{-1}(14) \in [-548.46, -546.6] \)
\item \( f^{-1}(14) \in [548.34, 551.9] \)
\item \( \text{ The function is not invertible for all Real numbers. } \)

\end{enumerate} }
\litem{
Multiply the following functions, then choose the domain of the resulting function from the list below.\[ f(x) = 5x^{3} +8 x^{2} +4 x + 3 \text{ and } g(x) = \sqrt{4x+17}  \]\begin{enumerate}[label=\Alph*.]
\item \( \text{ The domain is all Real numbers greater than or equal to } x = a, \text{ where } a \in [-6.25, 4.75] \)
\item \( \text{ The domain is all Real numbers except } x = a, \text{ where } a \in [0.4, 6.4] \)
\item \( \text{ The domain is all Real numbers less than or equal to } x = a, \text{ where } a \in [4.8, 5.8] \)
\item \( \text{ The domain is all Real numbers except } x = a \text{ and } x = b, \text{ where } a \in [2.2, 5.2] \text{ and } b \in [-7.67, -2.67] \)
\item \( \text{ The domain is all Real numbers. } \)

\end{enumerate} }
\litem{
Subtract the following functions, then choose the domain of the resulting function from the list below.\[ f(x) = \sqrt{-4x-12}  \text{ and } g(x) = 9x + 5 \]\begin{enumerate}[label=\Alph*.]
\item \( \text{ The domain is all Real numbers greater than or equal to } x = a, \text{ where } a \in [-7.5, -2.5] \)
\item \( \text{ The domain is all Real numbers less than or equal to } x = a, \text{ where } a \in [-4, 6] \)
\item \( \text{ The domain is all Real numbers except } x = a, \text{ where } a \in [1.8, 5.8] \)
\item \( \text{ The domain is all Real numbers except } x = a \text{ and } x = b, \text{ where } a \in [-4.67, -1.67] \text{ and } b \in [2.25, 10.25] \)
\item \( \text{ The domain is all Real numbers. } \)

\end{enumerate} }
\litem{
Choose the interval below that $f$ composed with $g$ at $x=-2$ is in.\[ f(x) = 3x^{3} +3 x^{2} -2 x + 4 \text{ and } g(x) = x^{3} +4 x^{2} +3 x \]\begin{enumerate}[label=\Alph*.]
\item \( (f \circ g)(-2) \in [31, 37] \)
\item \( (f \circ g)(-2) \in [-13, -10] \)
\item \( (f \circ g)(-2) \in [19, 29] \)
\item \( (f \circ g)(-2) \in [-24, -20] \)
\item \( \text{It is not possible to compose the two functions.} \)

\end{enumerate} }
\litem{
Choose the interval below that $f$ composed with $g$ at $x=2$ is in.\[ f(x) = -x^{3} +3 x^{2} -3 x -1 \text{ and } g(x) = -x^{3} +3 x^{2} -x \]\begin{enumerate}[label=\Alph*.]
\item \( (f \circ g)(2) \in [-7, -1] \)
\item \( (f \circ g)(2) \in [66, 69] \)
\item \( (f \circ g)(2) \in [56, 59] \)
\item \( (f \circ g)(2) \in [-10, -6] \)
\item \( \text{It is not possible to compose the two functions.} \)

\end{enumerate} }
\litem{
Find the inverse of the function below (if it exists). Then, evaluate the inverse at $x = -12$ and choose the interval that $f^{-1}(-12)$ belongs to.\[ f(x) = 3 x^2 + 5 \]\begin{enumerate}[label=\Alph*.]
\item \( f^{-1}(-12) \in [2.13, 2.95] \)
\item \( f^{-1}(-12) \in [1.24, 1.71] \)
\item \( f^{-1}(-12) \in [5.19, 6.21] \)
\item \( f^{-1}(-12) \in [3.3, 3.8] \)
\item \( \text{ The function is not invertible for all Real numbers. } \)

\end{enumerate} }
\litem{
Determine whether the function below is 1-1.\[ f(x) = -15 x^2 + 212 x - 672 \]\begin{enumerate}[label=\Alph*.]
\item \( \text{No, because the range of the function is not $(-\infty, \infty)$.} \)
\item \( \text{No, because there is a $y$-value that goes to 2 different $x$-values.} \)
\item \( \text{No, because there is an $x$-value that goes to 2 different $y$-values.} \)
\item \( \text{No, because the domain of the function is not $(-\infty, \infty)$.} \)
\item \( \text{Yes, the function is 1-1.} \)

\end{enumerate} }
\litem{
Find the inverse of the function below. Then, evaluate the inverse at $x = 9$ and choose the interval that $f^{-1}(9)$ belongs to.\[ f(x) = e^{x+4}+3 \]\begin{enumerate}[label=\Alph*.]
\item \( f^{-1}(9) \in [5.72, 5.84] \)
\item \( f^{-1}(9) \in [5.52, 5.71] \)
\item \( f^{-1}(9) \in [4.43, 4.65] \)
\item \( f^{-1}(9) \in [-2.43, -2.1] \)
\item \( f^{-1}(9) \in [5.36, 5.53] \)

\end{enumerate} }
\litem{
Determine whether the function below is 1-1.\[ f(x) = \sqrt{5 x - 21} \]\begin{enumerate}[label=\Alph*.]
\item \( \text{No, because the range of the function is not $(-\infty, \infty)$.} \)
\item \( \text{No, because there is a $y$-value that goes to 2 different $x$-values.} \)
\item \( \text{No, because the domain of the function is not $(-\infty, \infty)$.} \)
\item \( \text{Yes, the function is 1-1.} \)
\item \( \text{No, because there is an $x$-value that goes to 2 different $y$-values.} \)

\end{enumerate} }
\litem{
Find the inverse of the function below. Then, evaluate the inverse at $x = 9$ and choose the interval that $f^{-1}(9)$ belongs to.\[ f(x) = e^{x-4}+3 \]\begin{enumerate}[label=\Alph*.]
\item \( f^{-1}(9) \in [5.33, 5.56] \)
\item \( f^{-1}(9) \in [5.6, 5.89] \)
\item \( f^{-1}(9) \in [4.59, 4.75] \)
\item \( f^{-1}(9) \in [-2.44, -2.16] \)
\item \( f^{-1}(9) \in [5.5, 5.62] \)

\end{enumerate} }
\end{enumerate}

\end{document}