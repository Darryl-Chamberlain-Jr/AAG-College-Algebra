\documentclass{extbook}[14pt]
\usepackage{multicol, enumerate, enumitem, hyperref, color, soul, setspace, parskip, fancyhdr, amssymb, amsthm, amsmath, bbm, latexsym, units, mathtools}
\everymath{\displaystyle}
\usepackage[headsep=0.5cm,headheight=0cm, left=1 in,right= 1 in,top= 1 in,bottom= 1 in]{geometry}
\usepackage{dashrule}  % Package to use the command below to create lines between items
\newcommand{\litem}[1]{\item #1

\rule{\textwidth}{0.4pt}}
\pagestyle{fancy}
\lhead{}
\chead{Answer Key for Makeup Progress Quiz 3 Version B}
\rhead{}
\lfoot{4315-3397}
\cfoot{}
\rfoot{Fall 2020}
\begin{document}
\textbf{This key should allow you to understand why you choose the option you did (beyond just getting a question right or wrong). \href{https://xronos.clas.ufl.edu/mac1105spring2020/courseDescriptionAndMisc/Exams/LearningFromResults}{More instructions on how to use this key can be found here}.}

\textbf{If you have a suggestion to make the keys better, \href{https://forms.gle/CZkbZmPbC9XALEE88}{please fill out the short survey here}.}

\textit{Note: This key is auto-generated and may contain issues and/or errors. The keys are reviewed after each exam to ensure grading is done accurately. If there are issues (like duplicate options), they are noted in the offline gradebook. The keys are a work-in-progress to give students as many resources to improve as possible.}

\rule{\textwidth}{0.4pt}

\begin{enumerate}\litem{
Solve the linear inequality below. Then, choose the constant and interval combination that describes the solution set.
\[ -6 + 6 x < \frac{46 x - 6}{6} \leq 7 + 7 x \]

The solution is \( (-3.00, 12.00] \), which is option A.\begin{enumerate}[label=\Alph*.]
\item \( (a, b], \text{ where } a \in [-4, 2] \text{ and } b \in [6, 14] \)

* $(-3.00, 12.00]$, which is the correct option.
\item \( (-\infty, a] \cup (b, \infty), \text{ where } a \in [-7, 0] \text{ and } b \in [8, 14] \)

$(-\infty, -3.00] \cup (12.00, \infty)$, which corresponds to displaying the and-inequality as an or-inequality AND flipping the inequality.
\item \( [a, b), \text{ where } a \in [-3, -1] \text{ and } b \in [11, 14] \)

$[-3.00, 12.00)$, which corresponds to flipping the inequality.
\item \( (-\infty, a) \cup [b, \infty), \text{ where } a \in [-3, 0] \text{ and } b \in [12, 15] \)

$(-\infty, -3.00) \cup [12.00, \infty)$, which corresponds to displaying the and-inequality as an or-inequality.
\item \( \text{None of the above.} \)


\end{enumerate}

\textbf{General Comment:} To solve, you will need to break up the compound inequality into two inequalities. Be sure to keep track of the inequality! It may be best to draw a number line and graph your solution.
}
\litem{
Solve the linear inequality below. Then, choose the constant and interval combination that describes the solution set.
\[ -8 + 7 x > 8 x \text{ or } -9 - 3 x < 4 x \]

The solution is \( (-\infty, -8.0) \text{ or } (-1.286, \infty) \), which is option A.\begin{enumerate}[label=\Alph*.]
\item \( (-\infty, a) \cup (b, \infty), \text{ where } a \in [-8, -5] \text{ and } b \in [-4.29, 2.71] \)

 * Correct option.
\item \( (-\infty, a) \cup (b, \infty), \text{ where } a \in [-3.71, 2.29] \text{ and } b \in [7, 12] \)

Corresponds to inverting the inequality and negating the solution.
\item \( (-\infty, a] \cup [b, \infty), \text{ where } a \in [-0.71, 3.29] \text{ and } b \in [6, 11] \)

Corresponds to including the endpoints AND negating.
\item \( (-\infty, a] \cup [b, \infty), \text{ where } a \in [-12, -6] \text{ and } b \in [-3.29, 0.71] \)

Corresponds to including the endpoints (when they should be excluded).
\item \( (-\infty, \infty) \)

Corresponds to the variable canceling, which does not happen in this instance.
\end{enumerate}

\textbf{General Comment:} When multiplying or dividing by a negative, flip the sign.
}
\litem{
Solve the linear inequality below. Then, choose the constant and interval combination that describes the solution set.
\[ 8 - 9 x \leq \frac{-43 x - 5}{9} < 7 - 5 x \]

The solution is \( [2.03, 34.00) \), which is option B.\begin{enumerate}[label=\Alph*.]
\item \( (-\infty, a) \cup [b, \infty), \text{ where } a \in [1.03, 3.03] \text{ and } b \in [34, 38] \)

$(-\infty, 2.03) \cup [34.00, \infty)$, which corresponds to displaying the and-inequality as an or-inequality AND flipping the inequality.
\item \( [a, b), \text{ where } a \in [0.03, 9.03] \text{ and } b \in [34, 37] \)

$[2.03, 34.00)$, which is the correct option.
\item \( (a, b], \text{ where } a \in [-1.97, 4.03] \text{ and } b \in [33, 36] \)

$(2.03, 34.00]$, which corresponds to flipping the inequality.
\item \( (-\infty, a] \cup (b, \infty), \text{ where } a \in [-0.97, 6.03] \text{ and } b \in [30, 36] \)

$(-\infty, 2.03] \cup (34.00, \infty)$, which corresponds to displaying the and-inequality as an or-inequality.
\item \( \text{None of the above.} \)


\end{enumerate}

\textbf{General Comment:} To solve, you will need to break up the compound inequality into two inequalities. Be sure to keep track of the inequality! It may be best to draw a number line and graph your solution.
}
\litem{
Solve the linear inequality below. Then, choose the constant and interval combination that describes the solution set.
\[ -6 + 3 x > 4 x \text{ or } -7 + 5 x < 8 x \]

The solution is \( (-\infty, -6.0) \text{ or } (-2.333, \infty) \), which is option A.\begin{enumerate}[label=\Alph*.]
\item \( (-\infty, a) \cup (b, \infty), \text{ where } a \in [-6, -3] \text{ and } b \in [-4.33, 4.67] \)

 * Correct option.
\item \( (-\infty, a] \cup [b, \infty), \text{ where } a \in [-7, -3] \text{ and } b \in [-4.33, -0.33] \)

Corresponds to including the endpoints (when they should be excluded).
\item \( (-\infty, a] \cup [b, \infty), \text{ where } a \in [1.33, 6.33] \text{ and } b \in [2, 10] \)

Corresponds to including the endpoints AND negating.
\item \( (-\infty, a) \cup (b, \infty), \text{ where } a \in [0.33, 3.33] \text{ and } b \in [5, 9] \)

Corresponds to inverting the inequality and negating the solution.
\item \( (-\infty, \infty) \)

Corresponds to the variable canceling, which does not happen in this instance.
\end{enumerate}

\textbf{General Comment:} When multiplying or dividing by a negative, flip the sign.
}
\litem{
Solve the linear inequality below. Then, choose the constant and interval combination that describes the solution set.
\[ \frac{9}{8} - \frac{6}{5} x \leq \frac{5}{4} x - \frac{10}{7} \]

The solution is \( [1.042, \infty) \), which is option B.\begin{enumerate}[label=\Alph*.]
\item \( (-\infty, a], \text{ where } a \in [-1.9, -0.2] \)

 $(-\infty, -1.042]$, which corresponds to switching the direction of the interval AND negating the endpoint. You likely did this if you did not flip the inequality when dividing by a negative as well as not moving values over to a side properly.
\item \( [a, \infty), \text{ where } a \in [-0.96, 2.04] \)

* $[1.042, \infty)$, which is the correct option.
\item \( [a, \infty), \text{ where } a \in [-4.04, -0.04] \)

 $[-1.042, \infty)$, which corresponds to negating the endpoint of the solution.
\item \( (-\infty, a], \text{ where } a \in [0.2, 2.2] \)

 $(-\infty, 1.042]$, which corresponds to switching the direction of the interval. You likely did this if you did not flip the inequality when dividing by a negative!
\item \( \text{None of the above}. \)

You may have chosen this if you thought the inequality did not match the ends of the intervals.
\end{enumerate}

\textbf{General Comment:} Remember that less/greater than or equal to includes the endpoint, while less/greater do not. Also, remember that you need to flip the inequality when you multiply or divide by a negative.
}
\litem{
Solve the linear inequality below. Then, choose the constant and interval combination that describes the solution set.
\[ -6x -4 \geq -4x + 6 \]

The solution is \( (-\infty, -5.0] \), which is option D.\begin{enumerate}[label=\Alph*.]
\item \( [a, \infty), \text{ where } a \in [-6, -4] \)

 $[-5.0, \infty)$, which corresponds to switching the direction of the interval. You likely did this if you did not flip the inequality when dividing by a negative!
\item \( [a, \infty), \text{ where } a \in [-1, 15] \)

 $[5.0, \infty)$, which corresponds to switching the direction of the interval AND negating the endpoint. You likely did this if you did not flip the inequality when dividing by a negative as well as not moving values over to a side properly.
\item \( (-\infty, a], \text{ where } a \in [2, 7] \)

 $(-\infty, 5.0]$, which corresponds to negating the endpoint of the solution.
\item \( (-\infty, a], \text{ where } a \in [-5, -4] \)

* $(-\infty, -5.0]$, which is the correct option.
\item \( \text{None of the above}. \)

You may have chosen this if you thought the inequality did not match the ends of the intervals.
\end{enumerate}

\textbf{General Comment:} Remember that less/greater than or equal to includes the endpoint, while less/greater do not. Also, remember that you need to flip the inequality when you multiply or divide by a negative.
}
\litem{
Using an interval or intervals, describe all the $x$-values within or including a distance of the given values.
\[ \text{ More than } 5 \text{ units from the number } 6. \]

The solution is \( \text{None of the above} \), which is option E.\begin{enumerate}[label=\Alph*.]
\item \( (-\infty, -1) \cup (11, \infty) \)

This describes the values more than 6 from 5
\item \( (-1, 11) \)

This describes the values less than 6 from 5
\item \( (-\infty, -1] \cup [11, \infty) \)

This describes the values no less than 6 from 5
\item \( [-1, 11] \)

This describes the values no more than 6 from 5
\item \( \text{None of the above} \)

Options A-D described the values [more/less than] 6 units from 5, which is the reverse of what the question asked.
\end{enumerate}

\textbf{General Comment:} When thinking about this language, it helps to draw a number line and try points.
}
\litem{
Using an interval or intervals, describe all the $x$-values within or including a distance of the given values.
\[ \text{ More than } 5 \text{ units from the number } -5. \]

The solution is \( (-\infty, -10) \cup (0, \infty) \), which is option C.\begin{enumerate}[label=\Alph*.]
\item \( (-\infty, -10] \cup [0, \infty) \)

This describes the values no less than 5 from -5
\item \( (-10, 0) \)

This describes the values less than 5 from -5
\item \( (-\infty, -10) \cup (0, \infty) \)

This describes the values more than 5 from -5
\item \( [-10, 0] \)

This describes the values no more than 5 from -5
\item \( \text{None of the above} \)

You likely thought the values in the interval were not correct.
\end{enumerate}

\textbf{General Comment:} When thinking about this language, it helps to draw a number line and try points.
}
\litem{
Solve the linear inequality below. Then, choose the constant and interval combination that describes the solution set.
\[ \frac{5}{5} + \frac{3}{8} x > \frac{4}{9} x - \frac{7}{6} \]

The solution is \( (-\infty, 31.2) \), which is option C.\begin{enumerate}[label=\Alph*.]
\item \( (a, \infty), \text{ where } a \in [-31.2, -30.2] \)

 $(-31.2, \infty)$, which corresponds to switching the direction of the interval AND negating the endpoint. You likely did this if you did not flip the inequality when dividing by a negative as well as not moving values over to a side properly.
\item \( (-\infty, a), \text{ where } a \in [-31.2, -27.2] \)

 $(-\infty, -31.2)$, which corresponds to negating the endpoint of the solution.
\item \( (-\infty, a), \text{ where } a \in [29.2, 33.2] \)

* $(-\infty, 31.2)$, which is the correct option.
\item \( (a, \infty), \text{ where } a \in [30.2, 35.2] \)

 $(31.2, \infty)$, which corresponds to switching the direction of the interval. You likely did this if you did not flip the inequality when dividing by a negative!
\item \( \text{None of the above}. \)

You may have chosen this if you thought the inequality did not match the ends of the intervals.
\end{enumerate}

\textbf{General Comment:} Remember that less/greater than or equal to includes the endpoint, while less/greater do not. Also, remember that you need to flip the inequality when you multiply or divide by a negative.
}
\litem{
Solve the linear inequality below. Then, choose the constant and interval combination that describes the solution set.
\[ 3x + 7 \leq 4x -3 \]

The solution is \( [10.0, \infty) \), which is option D.\begin{enumerate}[label=\Alph*.]
\item \( (-\infty, a], \text{ where } a \in [4, 13] \)

 $(-\infty, 10.0]$, which corresponds to switching the direction of the interval. You likely did this if you did not flip the inequality when dividing by a negative!
\item \( (-\infty, a], \text{ where } a \in [-13, -9] \)

 $(-\infty, -10.0]$, which corresponds to switching the direction of the interval AND negating the endpoint. You likely did this if you did not flip the inequality when dividing by a negative as well as not moving values over to a side properly.
\item \( [a, \infty), \text{ where } a \in [-11, -8] \)

 $[-10.0, \infty)$, which corresponds to negating the endpoint of the solution.
\item \( [a, \infty), \text{ where } a \in [9, 12] \)

* $[10.0, \infty)$, which is the correct option.
\item \( \text{None of the above}. \)

You may have chosen this if you thought the inequality did not match the ends of the intervals.
\end{enumerate}

\textbf{General Comment:} Remember that less/greater than or equal to includes the endpoint, while less/greater do not. Also, remember that you need to flip the inequality when you multiply or divide by a negative.
}
\end{enumerate}

\end{document}