\documentclass{extbook}[14pt]
\usepackage{multicol, enumerate, enumitem, hyperref, color, soul, setspace, parskip, fancyhdr, amssymb, amsthm, amsmath, bbm, latexsym, units, mathtools}
\everymath{\displaystyle}
\usepackage[headsep=0.5cm,headheight=0cm, left=1 in,right= 1 in,top= 1 in,bottom= 1 in]{geometry}
\usepackage{dashrule}  % Package to use the command below to create lines between items
\newcommand{\litem}[1]{\item #1

\rule{\textwidth}{0.4pt}}
\pagestyle{fancy}
\lhead{}
\chead{Answer Key for Makeup Progress Quiz 3 Version C}
\rhead{}
\lfoot{4315-3397}
\cfoot{}
\rfoot{Fall 2020}
\begin{document}
\textbf{This key should allow you to understand why you choose the option you did (beyond just getting a question right or wrong). \href{https://xronos.clas.ufl.edu/mac1105spring2020/courseDescriptionAndMisc/Exams/LearningFromResults}{More instructions on how to use this key can be found here}.}

\textbf{If you have a suggestion to make the keys better, \href{https://forms.gle/CZkbZmPbC9XALEE88}{please fill out the short survey here}.}

\textit{Note: This key is auto-generated and may contain issues and/or errors. The keys are reviewed after each exam to ensure grading is done accurately. If there are issues (like duplicate options), they are noted in the offline gradebook. The keys are a work-in-progress to give students as many resources to improve as possible.}

\rule{\textwidth}{0.4pt}

\begin{enumerate}\litem{
Choose the \textbf{smallest} set of Real numbers that the number below belongs to.
\[ -\sqrt{\frac{13225}{529}} \]

The solution is \( \text{Integer} \), which is option B.\begin{enumerate}[label=\Alph*.]
\item \( \text{Irrational} \)

These cannot be written as a fraction of Integers.
\item \( \text{Integer} \)

* This is the correct option!
\item \( \text{Rational} \)

These are numbers that can be written as fraction of Integers (e.g., -2/3)
\item \( \text{Not a Real number} \)

These are Nonreal Complex numbers \textbf{OR} things that are not numbers (e.g., dividing by 0).
\item \( \text{Whole} \)

These are the counting numbers with 0 (0, 1, 2, 3, ...)
\end{enumerate}

\textbf{General Comment:} First, you \textbf{NEED} to simplify the expression. This question simplifies to $-115$. 
 
 Be sure you look at the simplified fraction and not just the decimal expansion. Numbers such as 13, 17, and 19 provide \textbf{long but repeating/terminating decimal expansions!} 
 
 The only ways to *not* be a Real number are: dividing by 0 or taking the square root of a negative number. 
 
 Irrational numbers are more than just square root of 3: adding or subtracting values from square root of 3 is also irrational.
}
\litem{
Choose the \textbf{smallest} set of Complex numbers that the number below belongs to.
\[ \sqrt{\frac{1872}{12}}+\sqrt{154} i \]

The solution is \( \text{Nonreal Complex} \), which is option C.\begin{enumerate}[label=\Alph*.]
\item \( \text{Irrational} \)

These cannot be written as a fraction of Integers. Remember: $\pi$ is not an Integer!
\item \( \text{Pure Imaginary} \)

This is a Complex number $(a+bi)$ that \textbf{only} has an imaginary part like $2i$.
\item \( \text{Nonreal Complex} \)

* This is the correct option!
\item \( \text{Rational} \)

These are numbers that can be written as fraction of Integers (e.g., -2/3 + 5)
\item \( \text{Not a Complex Number} \)

This is not a number. The only non-Complex number we know is dividing by 0 as this is not a number!
\end{enumerate}

\textbf{General Comment:} Be sure to simplify $i^2 = -1$. This may remove the imaginary portion for your number. If you are having trouble, you may want to look at the \textit{Subgroups of the Real Numbers} section.
}
\litem{
Choose the \textbf{smallest} set of Real numbers that the number below belongs to.
\[ -\sqrt{\frac{425}{5}} \]

The solution is \( \text{Irrational} \), which is option C.\begin{enumerate}[label=\Alph*.]
\item \( \text{Not a Real number} \)

These are Nonreal Complex numbers \textbf{OR} things that are not numbers (e.g., dividing by 0).
\item \( \text{Integer} \)

These are the negative and positive counting numbers (..., -3, -2, -1, 0, 1, 2, 3, ...)
\item \( \text{Irrational} \)

* This is the correct option!
\item \( \text{Rational} \)

These are numbers that can be written as fraction of Integers (e.g., -2/3)
\item \( \text{Whole} \)

These are the counting numbers with 0 (0, 1, 2, 3, ...)
\end{enumerate}

\textbf{General Comment:} First, you \textbf{NEED} to simplify the expression. This question simplifies to $-\sqrt{85}$. 
 
 Be sure you look at the simplified fraction and not just the decimal expansion. Numbers such as 13, 17, and 19 provide \textbf{long but repeating/terminating decimal expansions!} 
 
 The only ways to *not* be a Real number are: dividing by 0 or taking the square root of a negative number. 
 
 Irrational numbers are more than just square root of 3: adding or subtracting values from square root of 3 is also irrational.
}
\litem{
Simplify the expression below and choose the interval the simplification is contained within.
\[ 1 - 9^2 + 10 \div 19 * 12 \div 15 \]

The solution is \( -79.579 \), which is option A.\begin{enumerate}[label=\Alph*.]
\item \( [-79.87, -79.02] \)

* -79.579, this is the correct option
\item \( [-80.78, -79.83] \)

 -79.997, which corresponds to an Order of Operations error: not reading left-to-right for multiplication/division.
\item \( [81.56, 82.03] \)

 82.003, which corresponds to two Order of Operations errors.
\item \( [82.39, 82.81] \)

 82.421, which corresponds to an Order of Operations error: multiplying by negative before squaring. For example: $(-3)^2 \neq -3^2$
\item \( \text{None of the above} \)

 You may have gotten this by making an unanticipated error. If you got a value that is not any of the others, please let the coordinator know so they can help you figure out what happened.
\end{enumerate}

\textbf{General Comment:} While you may remember (or were taught) PEMDAS is done in order, it is actually done as P/E/MD/AS. When we are at MD or AS, we read left to right.
}
\litem{
Choose the \textbf{smallest} set of Complex numbers that the number below belongs to.
\[ \frac{0}{-9 \pi}+\sqrt{9}i \]

The solution is \( \text{Pure Imaginary} \), which is option E.\begin{enumerate}[label=\Alph*.]
\item \( \text{Not a Complex Number} \)

This is not a number. The only non-Complex number we know is dividing by 0 as this is not a number!
\item \( \text{Irrational} \)

These cannot be written as a fraction of Integers. Remember: $\pi$ is not an Integer!
\item \( \text{Rational} \)

These are numbers that can be written as fraction of Integers (e.g., -2/3 + 5)
\item \( \text{Nonreal Complex} \)

This is a Complex number $(a+bi)$ that is not Real (has $i$ as part of the number).
\item \( \text{Pure Imaginary} \)

* This is the correct option!
\end{enumerate}

\textbf{General Comment:} Be sure to simplify $i^2 = -1$. This may remove the imaginary portion for your number. If you are having trouble, you may want to look at the \textit{Subgroups of the Real Numbers} section.
}
\litem{
Simplify the expression below into the form $a+bi$. Then, choose the intervals that $a$ and $b$ belong to.
\[ \frac{-9 + 44 i}{-3 + 8 i} \]

The solution is \( 5.19  - 0.82 i \), which is option C.\begin{enumerate}[label=\Alph*.]
\item \( a \in [-5, -3.5] \text{ and } b \in [-4, -2] \)

 $-4.45  - 2.79 i$, which corresponds to forgetting to multiply the conjugate by the numerator and not computing the conjugate correctly.
\item \( a \in [378.5, 379.5] \text{ and } b \in [-1.5, 0.5] \)

 $379.00  - 0.82 i$, which corresponds to forgetting to multiply the conjugate by the numerator and using a plus instead of a minus in the denominator.
\item \( a \in [4.5, 6] \text{ and } b \in [-1.5, 0.5] \)

* $5.19  - 0.82 i$, which is the correct option.
\item \( a \in [2.5, 3.5] \text{ and } b \in [5, 6.5] \)

 $3.00  + 5.50 i$, which corresponds to just dividing the first term by the first term and the second by the second.
\item \( a \in [4.5, 6] \text{ and } b \in [-60.5, -59.5] \)

 $5.19  - 60.00 i$, which corresponds to forgetting to multiply the conjugate by the numerator.
\end{enumerate}

\textbf{General Comment:} Multiply the numerator and denominator by the *conjugate* of the denominator, then simplify. For example, if we have $2+3i$, the conjugate is $2-3i$.
}
\litem{
Simplify the expression below into the form $a+bi$. Then, choose the intervals that $a$ and $b$ belong to.
\[ (10 - 6 i)(5 - 8 i) \]

The solution is \( 2 - 110 i \), which is option E.\begin{enumerate}[label=\Alph*.]
\item \( a \in [96, 102] \text{ and } b \in [-50.68, -49.98] \)

 $98 - 50 i$, which corresponds to adding a minus sign in the first term.
\item \( a \in [96, 102] \text{ and } b \in [49.72, 50.97] \)

 $98 + 50 i$, which corresponds to adding a minus sign in the second term.
\item \( a \in [-2, 7] \text{ and } b \in [109.01, 111.53] \)

 $2 + 110 i$, which corresponds to adding a minus sign in both terms.
\item \( a \in [49, 55] \text{ and } b \in [46.97, 49.25] \)

 $50 + 48 i$, which corresponds to just multiplying the real terms to get the real part of the solution and the coefficients in the complex terms to get the complex part.
\item \( a \in [-2, 7] \text{ and } b \in [-110.1, -109.19] \)

* $2 - 110 i$, which is the correct option.
\end{enumerate}

\textbf{General Comment:} You can treat $i$ as a variable and distribute. Just remember that $i^2=-1$, so you can continue to reduce after you distribute.
}
\litem{
Simplify the expression below into the form $a+bi$. Then, choose the intervals that $a$ and $b$ belong to.
\[ \frac{-18 - 55 i}{-7 + 3 i} \]

The solution is \( -0.67  + 7.57 i \), which is option D.\begin{enumerate}[label=\Alph*.]
\item \( a \in [-39.5, -38.5] \text{ and } b \in [6, 8] \)

 $-39.00  + 7.57 i$, which corresponds to forgetting to multiply the conjugate by the numerator and using a plus instead of a minus in the denominator.
\item \( a \in [4, 6] \text{ and } b \in [4.5, 6.5] \)

 $5.02  + 5.71 i$, which corresponds to forgetting to multiply the conjugate by the numerator and not computing the conjugate correctly.
\item \( a \in [-2, 0] \text{ and } b \in [438.5, 439.5] \)

 $-0.67  + 439.00 i$, which corresponds to forgetting to multiply the conjugate by the numerator.
\item \( a \in [-2, 0] \text{ and } b \in [6, 8] \)

* $-0.67  + 7.57 i$, which is the correct option.
\item \( a \in [1.5, 3.5] \text{ and } b \in [-19, -18] \)

 $2.57  - 18.33 i$, which corresponds to just dividing the first term by the first term and the second by the second.
\end{enumerate}

\textbf{General Comment:} Multiply the numerator and denominator by the *conjugate* of the denominator, then simplify. For example, if we have $2+3i$, the conjugate is $2-3i$.
}
\litem{
Simplify the expression below and choose the interval the simplification is contained within.
\[ 1 - 6 \div 19 * 8 - (9 * 20) \]

The solution is \( -181.526 \), which is option D.\begin{enumerate}[label=\Alph*.]
\item \( [-215, -209.8] \)

 -210.526, which corresponds to not distributing a negative correctly.
\item \( [-179.4, -177.7] \)

 -179.039, which corresponds to an Order of Operations error: not reading left-to-right for multiplication/division.
\item \( [179.4, 181.8] \)

 180.961, which corresponds to not distributing addition and subtraction correctly.
\item \( [-185.5, -180.3] \)

* -181.526, which is the correct option.
\item \( \text{None of the above} \)

 You may have gotten this by making an unanticipated error. If you got a value that is not any of the others, please let the coordinator know so they can help you figure out what happened.
\end{enumerate}

\textbf{General Comment:} While you may remember (or were taught) PEMDAS is done in order, it is actually done as P/E/MD/AS. When we are at MD or AS, we read left to right.
}
\litem{
Simplify the expression below into the form $a+bi$. Then, choose the intervals that $a$ and $b$ belong to.
\[ (3 + 9 i)(-2 + 8 i) \]

The solution is \( -78 + 6 i \), which is option C.\begin{enumerate}[label=\Alph*.]
\item \( a \in [66, 69] \text{ and } b \in [42, 44] \)

 $66 + 42 i$, which corresponds to adding a minus sign in the first term.
\item \( a \in [66, 69] \text{ and } b \in [-45, -37] \)

 $66 - 42 i$, which corresponds to adding a minus sign in the second term.
\item \( a \in [-80, -77] \text{ and } b \in [5, 8] \)

* $-78 + 6 i$, which is the correct option.
\item \( a \in [-80, -77] \text{ and } b \in [-8, -4] \)

 $-78 - 6 i$, which corresponds to adding a minus sign in both terms.
\item \( a \in [-10, -1] \text{ and } b \in [67, 77] \)

 $-6 + 72 i$, which corresponds to just multiplying the real terms to get the real part of the solution and the coefficients in the complex terms to get the complex part.
\end{enumerate}

\textbf{General Comment:} You can treat $i$ as a variable and distribute. Just remember that $i^2=-1$, so you can continue to reduce after you distribute.
}
\end{enumerate}

\end{document}