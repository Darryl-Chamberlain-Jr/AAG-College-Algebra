\documentclass{extbook}[14pt]
\usepackage{multicol, enumerate, enumitem, hyperref, color, soul, setspace, parskip, fancyhdr, amssymb, amsthm, amsmath, bbm, latexsym, units, mathtools}
\everymath{\displaystyle}
\usepackage[headsep=0.5cm,headheight=0cm, left=1 in,right= 1 in,top= 1 in,bottom= 1 in]{geometry}
\usepackage{dashrule}  % Package to use the command below to create lines between items
\newcommand{\litem}[1]{\item #1

\rule{\textwidth}{0.4pt}}
\pagestyle{fancy}
\lhead{}
\chead{Answer Key for Makeup Progress Quiz 3 Version B}
\rhead{}
\lfoot{4315-3397}
\cfoot{}
\rfoot{Fall 2020}
\begin{document}
\textbf{This key should allow you to understand why you choose the option you did (beyond just getting a question right or wrong). \href{https://xronos.clas.ufl.edu/mac1105spring2020/courseDescriptionAndMisc/Exams/LearningFromResults}{More instructions on how to use this key can be found here}.}

\textbf{If you have a suggestion to make the keys better, \href{https://forms.gle/CZkbZmPbC9XALEE88}{please fill out the short survey here}.}

\textit{Note: This key is auto-generated and may contain issues and/or errors. The keys are reviewed after each exam to ensure grading is done accurately. If there are issues (like duplicate options), they are noted in the offline gradebook. The keys are a work-in-progress to give students as many resources to improve as possible.}

\rule{\textwidth}{0.4pt}

\begin{enumerate}\litem{
Perform the division below. Then, find the intervals that correspond to the quotient in the form $ax^2+bx+c$ and remainder $r$.
\[ \frac{16x^{3} +52 x^{2} -31}{x + 3} \]

The solution is \( 16x^{2} +4 x -12 + \frac{5}{x + 3} \), which is option E.\begin{enumerate}[label=\Alph*.]
\item \( a \in [-49, -46], b \in [-97, -87], c \in [-279, -272], \text{ and } r \in [-864, -858]. \)

 You divided by the opposite of the factor AND multipled the first factor rather than just bringing it down.
\item \( a \in [15, 18], b \in [-13, -6], c \in [46, 51], \text{ and } r \in [-227, -220]. \)

 You multipled by the synthetic number and subtracted rather than adding during synthetic division.
\item \( a \in [15, 18], b \in [100, 105], c \in [290, 305], \text{ and } r \in [868, 873]. \)

 You divided by the opposite of the factor.
\item \( a \in [-49, -46], b \in [194, 198], c \in [-590, -585], \text{ and } r \in [1732, 1740]. \)

 You multipled by the synthetic number rather than bringing the first factor down.
\item \( a \in [15, 18], b \in [3, 8], c \in [-15, -11], \text{ and } r \in [2, 9]. \)

* This is the solution!
\end{enumerate}

\textbf{General Comment:} Be sure to synthetically divide by the zero of the denominator! Also, make sure to include 0 placeholders for missing terms.
}
\litem{
What are the \textit{possible Rational} roots of the polynomial below?
\[ f(x) = 3x^{2} +3 x + 4 \]

The solution is \( \text{ All combinations of: }\frac{\pm 1,\pm 2,\pm 4}{\pm 1,\pm 3} \), which is option C.\begin{enumerate}[label=\Alph*.]
\item \( \pm 1,\pm 3 \)

 Distractor 1: Corresponds to the plus or minus factors of a1 only.
\item \( \text{ All combinations of: }\frac{\pm 1,\pm 3}{\pm 1,\pm 2,\pm 4} \)

 Distractor 3: Corresponds to the plus or minus of the inverse quotient (an/a0) of the factors. 
\item \( \text{ All combinations of: }\frac{\pm 1,\pm 2,\pm 4}{\pm 1,\pm 3} \)

* This is the solution \textbf{since we asked for the possible Rational roots}!
\item \( \pm 1,\pm 2,\pm 4 \)

This would have been the solution \textbf{if asked for the possible Integer roots}!
\item \( \text{ There is no formula or theorem that tells us all possible Rational roots.} \)

 Distractor 4: Corresponds to not recalling the theorem for rational roots of a polynomial.
\end{enumerate}

\textbf{General Comment:} We have a way to find the possible Rational roots. The possible Integer roots are the Integers in this list.
}
\litem{
Perform the division below. Then, find the intervals that correspond to the quotient in the form $ax^2+bx+c$ and remainder $r$.
\[ \frac{4x^{3} -27 x + 29}{x + 3} \]

The solution is \( 4x^{2} -12 x + 9 + \frac{2}{x + 3} \), which is option C.\begin{enumerate}[label=\Alph*.]
\item \( a \in [-14, -8], b \in [-39, -26], c \in [-138, -132], \text{ and } r \in [-381, -372]. \)

 You divided by the opposite of the factor AND multipled the first factor rather than just bringing it down.
\item \( a \in [-4, 11], b \in [10, 19], c \in [8, 12], \text{ and } r \in [49, 60]. \)

 You divided by the opposite of the factor.
\item \( a \in [-4, 11], b \in [-12, -10], c \in [8, 12], \text{ and } r \in [-2, 9]. \)

* This is the solution!
\item \( a \in [-14, -8], b \in [36, 37], c \in [-138, -132], \text{ and } r \in [433, 436]. \)

 You multipled by the synthetic number rather than bringing the first factor down.
\item \( a \in [-4, 11], b \in [-21, -15], c \in [35, 38], \text{ and } r \in [-124, -116]. \)

 You multipled by the synthetic number and subtracted rather than adding during synthetic division.
\end{enumerate}

\textbf{General Comment:} Be sure to synthetically divide by the zero of the denominator! Also, make sure to include 0 placeholders for missing terms.
}
\litem{
Perform the division below. Then, find the intervals that correspond to the quotient in the form $ax^2+bx+c$ and remainder $r$.
\[ \frac{15x^{3} -23 x^{2} -128 x -82}{x -4} \]

The solution is \( 15x^{2} +37 x + 20 + \frac{-2}{x -4} \), which is option E.\begin{enumerate}[label=\Alph*.]
\item \( a \in [57, 61], \text{   } b \in [-266, -261], \text{   } c \in [916, 931], \text{   and   } r \in [-3778, -3774]. \)

 You divided by the opposite of the factor AND multiplied the first factor rather than just bringing it down.
\item \( a \in [15, 19], \text{   } b \in [15, 23], \text{   } c \in [-66, -57], \text{   and   } r \in [-274, -263]. \)

 You multiplied by the synthetic number and subtracted rather than adding during synthetic division.
\item \( a \in [57, 61], \text{   } b \in [217, 223], \text{   } c \in [740, 741], \text{   and   } r \in [2872, 2882]. \)

 You multiplied by the synthetic number rather than bringing the first factor down.
\item \( a \in [15, 19], \text{   } b \in [-90, -82], \text{   } c \in [202, 209], \text{   and   } r \in [-899, -891]. \)

 You divided by the opposite of the factor.
\item \( a \in [15, 19], \text{   } b \in [36, 40], \text{   } c \in [16, 25], \text{   and   } r \in [-3, 4]. \)

* This is the solution!
\end{enumerate}

\textbf{General Comment:} Be sure to synthetically divide by the zero of the denominator!
}
\litem{
What are the \textit{possible Rational} roots of the polynomial below?
\[ f(x) = 7x^{3} +7 x^{2} +7 x + 6 \]

The solution is \( \text{ All combinations of: }\frac{\pm 1,\pm 2,\pm 3,\pm 6}{\pm 1,\pm 7} \), which is option C.\begin{enumerate}[label=\Alph*.]
\item \( \text{ All combinations of: }\frac{\pm 1,\pm 7}{\pm 1,\pm 2,\pm 3,\pm 6} \)

 Distractor 3: Corresponds to the plus or minus of the inverse quotient (an/a0) of the factors. 
\item \( \pm 1,\pm 2,\pm 3,\pm 6 \)

This would have been the solution \textbf{if asked for the possible Integer roots}!
\item \( \text{ All combinations of: }\frac{\pm 1,\pm 2,\pm 3,\pm 6}{\pm 1,\pm 7} \)

* This is the solution \textbf{since we asked for the possible Rational roots}!
\item \( \pm 1,\pm 7 \)

 Distractor 1: Corresponds to the plus or minus factors of a1 only.
\item \( \text{ There is no formula or theorem that tells us all possible Rational roots.} \)

 Distractor 4: Corresponds to not recalling the theorem for rational roots of a polynomial.
\end{enumerate}

\textbf{General Comment:} We have a way to find the possible Rational roots. The possible Integer roots are the Integers in this list.
}
\litem{
Perform the division below. Then, find the intervals that correspond to the quotient in the form $ax^2+bx+c$ and remainder $r$.
\[ \frac{8x^{3} -24 x^{2} -90 x + 55}{x -5} \]

The solution is \( 8x^{2} +16 x -10 + \frac{5}{x -5} \), which is option E.\begin{enumerate}[label=\Alph*.]
\item \( a \in [2, 10], \text{   } b \in [2, 15], \text{   } c \in [-59, -57], \text{   and   } r \in [-178, -174]. \)

 You multiplied by the synthetic number and subtracted rather than adding during synthetic division.
\item \( a \in [2, 10], \text{   } b \in [-67, -55], \text{   } c \in [229, 231], \text{   and   } r \in [-1098, -1092]. \)

 You divided by the opposite of the factor.
\item \( a \in [36, 43], \text{   } b \in [-226, -221], \text{   } c \in [1029, 1036], \text{   and   } r \in [-5107, -5090]. \)

 You divided by the opposite of the factor AND multiplied the first factor rather than just bringing it down.
\item \( a \in [36, 43], \text{   } b \in [174, 185], \text{   } c \in [787, 794], \text{   and   } r \in [4002, 4009]. \)

 You multiplied by the synthetic number rather than bringing the first factor down.
\item \( a \in [2, 10], \text{   } b \in [13, 19], \text{   } c \in [-14, -2], \text{   and   } r \in [5, 10]. \)

* This is the solution!
\end{enumerate}

\textbf{General Comment:} Be sure to synthetically divide by the zero of the denominator!
}
\litem{
Factor the polynomial below completely, knowing that $x-5$ is a factor. Then, choose the intervals the zeros of the polynomial belong to, where $z_1 \leq z_2 \leq z_3 \leq z_4$. \textit{To make the problem easier, all zeros are between -5 and 5.}
\[ f(x) = 16x^{4} -104 x^{3} +89 x^{2} +185 x -150 \]

The solution is \( [-1.25, 0.75, 2, 5] \), which is option D.\begin{enumerate}[label=\Alph*.]
\item \( z_1 \in [-6.2, -4.9], \text{   }  z_2 \in [-2.26, -1.59], z_3 \in [-0.76, -0.4], \text{   and   } z_4 \in [0.92, 1.62] \)

 Distractor 1: Corresponds to negatives of all zeros.
\item \( z_1 \in [-6.2, -4.9], \text{   }  z_2 \in [-4.02, -2.71], z_3 \in [-2.19, -1.95], \text{   and   } z_4 \in [0.16, 0.36] \)

 Distractor 4: Corresponds to moving factors from one rational to another.
\item \( z_1 \in [-1, 1.1], \text{   }  z_2 \in [1.31, 2.09], z_3 \in [1.94, 2.37], \text{   and   } z_4 \in [4.78, 5.29] \)

 Distractor 2: Corresponds to inversing rational roots.
\item \( z_1 \in [-2.4, -0.9], \text{   }  z_2 \in [0.72, 0.77], z_3 \in [1.94, 2.37], \text{   and   } z_4 \in [4.78, 5.29] \)

* This is the solution!
\item \( z_1 \in [-6.2, -4.9], \text{   }  z_2 \in [-2.26, -1.59], z_3 \in [-1.66, -1.22], \text{   and   } z_4 \in [0.75, 1.1] \)

 Distractor 3: Corresponds to negatives of all zeros AND inversing rational roots.
\end{enumerate}

\textbf{General Comment:} Remember to try the middle-most integers first as these normally are the zeros. Also, once you get it to a quadratic, you can use your other factoring techniques to finish factoring.
}
\litem{
Factor the polynomial below completely. Then, choose the intervals the zeros of the polynomial belong to, where $z_1 \leq z_2 \leq z_3$. \textit{To make the problem easier, all zeros are between -5 and 5.}
\[ f(x) = 10x^{3} -73 x^{2} +127 x -60 \]

The solution is \( [0.8, 1.5, 5] \), which is option A.\begin{enumerate}[label=\Alph*.]
\item \( z_1 \in [0.67, 1.13], \text{   }  z_2 \in [1.48, 1.57], \text{   and   } z_3 \in [5, 5.04] \)

* This is the solution!
\item \( z_1 \in [-5.34, -4.96], \text{   }  z_2 \in [-1.25, -1.09], \text{   and   } z_3 \in [-0.68, -0.6] \)

 Distractor 3: Corresponds to negatives of all zeros AND inversing rational roots.
\item \( z_1 \in [-5.34, -4.96], \text{   }  z_2 \in [-1.65, -1.42], \text{   and   } z_3 \in [-0.89, -0.68] \)

 Distractor 1: Corresponds to negatives of all zeros.
\item \( z_1 \in [-5.34, -4.96], \text{   }  z_2 \in [-3.32, -2.65], \text{   and   } z_3 \in [-0.55, -0.32] \)

 Distractor 4: Corresponds to moving factors from one rational to another.
\item \( z_1 \in [0.39, 0.75], \text{   }  z_2 \in [1.23, 1.43], \text{   and   } z_3 \in [5, 5.04] \)

 Distractor 2: Corresponds to inversing rational roots.
\end{enumerate}

\textbf{General Comment:} Remember to try the middle-most integers first as these normally are the zeros. Also, once you get it to a quadratic, you can use your other factoring techniques to finish factoring.
}
\litem{
Factor the polynomial below completely, knowing that $x-4$ is a factor. Then, choose the intervals the zeros of the polynomial belong to, where $z_1 \leq z_2 \leq z_3 \leq z_4$. \textit{To make the problem easier, all zeros are between -5 and 5.}
\[ f(x) = 10x^{4} -121 x^{3} +494 x^{2} -755 x + 300 \]

The solution is \( [0.6, 2.5, 4, 5] \), which is option B.\begin{enumerate}[label=\Alph*.]
\item \( z_1 \in [0.11, 0.57], \text{   }  z_2 \in [1.39, 2.41], z_3 \in [2.8, 4.2], \text{   and   } z_4 \in [4.93, 5.22] \)

 Distractor 2: Corresponds to inversing rational roots.
\item \( z_1 \in [0.59, 0.88], \text{   }  z_2 \in [2.44, 2.88], z_3 \in [2.8, 4.2], \text{   and   } z_4 \in [4.93, 5.22] \)

* This is the solution!
\item \( z_1 \in [-5.14, -4.94], \text{   }  z_2 \in [-4.7, -3.64], z_3 \in [-3.8, -2.4], \text{   and   } z_4 \in [-0.74, -0.46] \)

 Distractor 1: Corresponds to negatives of all zeros.
\item \( z_1 \in [-5.14, -4.94], \text{   }  z_2 \in [-5.01, -4.34], z_3 \in [-5.9, -3.7], \text{   and   } z_4 \in [-0.36, -0.19] \)

 Distractor 4: Corresponds to moving factors from one rational to another.
\item \( z_1 \in [-5.14, -4.94], \text{   }  z_2 \in [-4.7, -3.64], z_3 \in [-2, -1.5], \text{   and   } z_4 \in [-0.58, -0.34] \)

 Distractor 3: Corresponds to negatives of all zeros AND inversing rational roots.
\end{enumerate}

\textbf{General Comment:} Remember to try the middle-most integers first as these normally are the zeros. Also, once you get it to a quadratic, you can use your other factoring techniques to finish factoring.
}
\litem{
Factor the polynomial below completely. Then, choose the intervals the zeros of the polynomial belong to, where $z_1 \leq z_2 \leq z_3$. \textit{To make the problem easier, all zeros are between -5 and 5.}
\[ f(x) = 6x^{3} -13 x^{2} -13 x + 30 \]

The solution is \( [-1.5, 1.6666666666666667, 2] \), which is option C.\begin{enumerate}[label=\Alph*.]
\item \( z_1 \in [-2.6, -1.8], \text{   }  z_2 \in [-1.78, -1.42], \text{   and   } z_3 \in [1.04, 1.81] \)

 Distractor 1: Corresponds to negatives of all zeros.
\item \( z_1 \in [-1, -0.5], \text{   }  z_2 \in [0.53, 0.72], \text{   and   } z_3 \in [1.81, 2.35] \)

 Distractor 2: Corresponds to inversing rational roots.
\item \( z_1 \in [-1.7, -0.8], \text{   }  z_2 \in [1.53, 1.85], \text{   and   } z_3 \in [1.81, 2.35] \)

* This is the solution!
\item \( z_1 \in [-2.6, -1.8], \text{   }  z_2 \in [-0.81, -0.28], \text{   and   } z_3 \in [0.4, 0.81] \)

 Distractor 3: Corresponds to negatives of all zeros AND inversing rational roots.
\item \( z_1 \in [-2.6, -1.8], \text{   }  z_2 \in [-0.87, -0.65], \text{   and   } z_3 \in [2.29, 3.55] \)

 Distractor 4: Corresponds to moving factors from one rational to another.
\end{enumerate}

\textbf{General Comment:} Remember to try the middle-most integers first as these normally are the zeros. Also, once you get it to a quadratic, you can use your other factoring techniques to finish factoring.
}
\end{enumerate}

\end{document}