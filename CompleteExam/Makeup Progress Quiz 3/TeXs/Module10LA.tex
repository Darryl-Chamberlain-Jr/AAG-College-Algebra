\documentclass[14pt]{extbook}
\usepackage{multicol, enumerate, enumitem, hyperref, color, soul, setspace, parskip, fancyhdr} %General Packages
\usepackage{amssymb, amsthm, amsmath, bbm, latexsym, units, mathtools} %Math Packages
\everymath{\displaystyle} %All math in Display Style
% Packages with additional options
\usepackage[headsep=0.5cm,headheight=12pt, left=1 in,right= 1 in,top= 1 in,bottom= 1 in]{geometry}
\usepackage[usenames,dvipsnames]{xcolor}
\usepackage{dashrule}  % Package to use the command below to create lines between items
\newcommand{\litem}[1]{\item#1\hspace*{-1cm}\rule{\textwidth}{0.4pt}}
\pagestyle{fancy}
\lhead{Makeup Progress Quiz 3}
\chead{}
\rhead{Version A}
\lfoot{4315-3397}
\cfoot{}
\rfoot{Fall 2020}
\begin{document}

\begin{enumerate}
\litem{
Perform the division below. Then, find the intervals that correspond to the quotient in the form $ax^2+bx+c$ and remainder $r$.\[ \frac{9x^{3} -27 x + 23}{x + 2} \]\begin{enumerate}[label=\Alph*.]
\item \( a \in [6, 15], b \in [14, 26], c \in [6, 13], \text{ and } r \in [39, 43]. \)
\item \( a \in [6, 15], b \in [-32, -26], c \in [50, 62], \text{ and } r \in [-139, -137]. \)
\item \( a \in [-20, -13], b \in [-37, -32], c \in [-104, -98], \text{ and } r \in [-175, -173]. \)
\item \( a \in [-20, -13], b \in [36, 39], c \in [-104, -98], \text{ and } r \in [220, 223]. \)
\item \( a \in [6, 15], b \in [-18, -9], c \in [6, 13], \text{ and } r \in [0, 9]. \)

\end{enumerate} }
\litem{
What are the \textit{possible Integer} roots of the polynomial below?\[ f(x) = 5x^{3} +3 x^{2} +6 x + 3 \]\begin{enumerate}[label=\Alph*.]
\item \( \pm 1,\pm 3 \)
\item \( \text{ All combinations of: }\frac{\pm 1,\pm 3}{\pm 1,\pm 5} \)
\item \( \text{ All combinations of: }\frac{\pm 1,\pm 5}{\pm 1,\pm 3} \)
\item \( \pm 1,\pm 5 \)
\item \( \text{There is no formula or theorem that tells us all possible Integer roots.} \)

\end{enumerate} }
\litem{
Perform the division below. Then, find the intervals that correspond to the quotient in the form $ax^2+bx+c$ and remainder $r$.\[ \frac{6x^{3} +21 x^{2} -30}{x + 3} \]\begin{enumerate}[label=\Alph*.]
\item \( a \in [-18, -14], b \in [-33, -30], c \in [-99, -97], \text{ and } r \in [-327, -324]. \)
\item \( a \in [-1, 8], b \in [37, 44], c \in [116, 120], \text{ and } r \in [321, 324]. \)
\item \( a \in [-18, -14], b \in [73, 78], c \in [-226, -223], \text{ and } r \in [641, 647]. \)
\item \( a \in [-1, 8], b \in [-6, 1], c \in [10, 13], \text{ and } r \in [-79, -73]. \)
\item \( a \in [-1, 8], b \in [-1, 10], c \in [-14, -6], \text{ and } r \in [-7, -1]. \)

\end{enumerate} }
\litem{
Perform the division below. Then, find the intervals that correspond to the quotient in the form $ax^2+bx+c$ and remainder $r$.\[ \frac{20x^{3} -107 x^{2} +117 x -34}{x -4} \]\begin{enumerate}[label=\Alph*.]
\item \( a \in [20, 23], \text{   } b \in [-30, -25], \text{   } c \in [7, 10], \text{   and   } r \in [-1, 7]. \)
\item \( a \in [20, 23], \text{   } b \in [-187, -185], \text{   } c \in [863, 866], \text{   and   } r \in [-3499, -3493]. \)
\item \( a \in [75, 84], \text{   } b \in [-430, -426], \text{   } c \in [1821, 1828], \text{   and   } r \in [-7334, -7328]. \)
\item \( a \in [20, 23], \text{   } b \in [-52, -46], \text{   } c \in [-24, -22], \text{   and   } r \in [-108, -96]. \)
\item \( a \in [75, 84], \text{   } b \in [206, 220], \text{   } c \in [966, 972], \text{   and   } r \in [3841, 3844]. \)

\end{enumerate} }
\litem{
What are the \textit{possible Rational} roots of the polynomial below?\[ f(x) = 4x^{4} +2 x^{3} +7 x^{2} +6 x + 3 \]\begin{enumerate}[label=\Alph*.]
\item \( \text{ All combinations of: }\frac{\pm 1,\pm 3}{\pm 1,\pm 2,\pm 4} \)
\item \( \pm 1,\pm 3 \)
\item \( \pm 1,\pm 2,\pm 4 \)
\item \( \text{ All combinations of: }\frac{\pm 1,\pm 2,\pm 4}{\pm 1,\pm 3} \)
\item \( \text{ There is no formula or theorem that tells us all possible Rational roots.} \)

\end{enumerate} }
\litem{
Perform the division below. Then, find the intervals that correspond to the quotient in the form $ax^2+bx+c$ and remainder $r$.\[ \frac{20x^{3} -54 x^{2} -96 x -36}{x -4} \]\begin{enumerate}[label=\Alph*.]
\item \( a \in [17, 23], \text{   } b \in [-136, -127], \text{   } c \in [436, 441], \text{   and   } r \in [-1798, -1793]. \)
\item \( a \in [80, 83], \text{   } b \in [261, 270], \text{   } c \in [967, 969], \text{   and   } r \in [3833, 3837]. \)
\item \( a \in [17, 23], \text{   } b \in [26, 29], \text{   } c \in [5, 9], \text{   and   } r \in [-4, -2]. \)
\item \( a \in [17, 23], \text{   } b \in [4, 12], \text{   } c \in [-81, -75], \text{   and   } r \in [-271, -268]. \)
\item \( a \in [80, 83], \text{   } b \in [-375, -370], \text{   } c \in [1397, 1403], \text{   and   } r \in [-5638, -5635]. \)

\end{enumerate} }
\litem{
Factor the polynomial below completely, knowing that $x+3$ is a factor. Then, choose the intervals the zeros of the polynomial belong to, where $z_1 \leq z_2 \leq z_3 \leq z_4$. \textit{To make the problem easier, all zeros are between -5 and 5.}\[ f(x) = 20x^{4} +103 x^{3} -4 x^{2} -339 x + 180 \]\begin{enumerate}[label=\Alph*.]
\item \( z_1 \in [-2.04, -1.58], \text{   }  z_2 \in [-0.86, -0.8], z_3 \in [2.92, 3.3], \text{   and   } z_4 \in [3.77, 4.03] \)
\item \( z_1 \in [-3.47, -2.82], \text{   }  z_2 \in [-0.46, -0.21], z_3 \in [2.92, 3.3], \text{   and   } z_4 \in [3.77, 4.03] \)
\item \( z_1 \in [-4.67, -3.71], \text{   }  z_2 \in [-3.23, -2.91], z_3 \in [0.5, 0.64], \text{   and   } z_4 \in [0.85, 1.29] \)
\item \( z_1 \in [-4.67, -3.71], \text{   }  z_2 \in [-3.23, -2.91], z_3 \in [0.68, 1.09], \text{   and   } z_4 \in [1.31, 1.72] \)
\item \( z_1 \in [-1.44, -0.82], \text{   }  z_2 \in [-0.79, -0.4], z_3 \in [2.92, 3.3], \text{   and   } z_4 \in [3.77, 4.03] \)

\end{enumerate} }
\litem{
Factor the polynomial below completely. Then, choose the intervals the zeros of the polynomial belong to, where $z_1 \leq z_2 \leq z_3$. \textit{To make the problem easier, all zeros are between -5 and 5.}\[ f(x) = 8x^{3} -10 x^{2} -57 x + 45 \]\begin{enumerate}[label=\Alph*.]
\item \( z_1 \in [-3.07, -2.72], \text{   }  z_2 \in [-0.86, -0.71], \text{   and   } z_3 \in [2.34, 2.68] \)
\item \( z_1 \in [-3.07, -2.72], \text{   }  z_2 \in [-1.55, -1.2], \text{   and   } z_3 \in [-0.03, 0.76] \)
\item \( z_1 \in [-0.52, -0.05], \text{   }  z_2 \in [1.01, 1.41], \text{   and   } z_3 \in [2.8, 3.19] \)
\item \( z_1 \in [-2.84, -2.33], \text{   }  z_2 \in [0.4, 0.87], \text{   and   } z_3 \in [2.8, 3.19] \)
\item \( z_1 \in [-3.07, -2.72], \text{   }  z_2 \in [-0.63, -0.18], \text{   and   } z_3 \in [4.83, 5.39] \)

\end{enumerate} }
\litem{
Factor the polynomial below completely, knowing that $x-5$ is a factor. Then, choose the intervals the zeros of the polynomial belong to, where $z_1 \leq z_2 \leq z_3 \leq z_4$. \textit{To make the problem easier, all zeros are between -5 and 5.}\[ f(x) = 12x^{4} -5 x^{3} -325 x^{2} +125 x + 625 \]\begin{enumerate}[label=\Alph*.]
\item \( z_1 \in [-5, -4], \text{   }  z_2 \in [-1.25, -1.18], z_3 \in [1.62, 1.74], \text{   and   } z_4 \in [5, 9] \)
\item \( z_1 \in [-5, -4], \text{   }  z_2 \in [-1.69, -1.55], z_3 \in [1.17, 1.47], \text{   and   } z_4 \in [5, 9] \)
\item \( z_1 \in [-5, -4], \text{   }  z_2 \in [-0.57, -0.41], z_3 \in [4.99, 5.11], \text{   and   } z_4 \in [5, 9] \)
\item \( z_1 \in [-5, -4], \text{   }  z_2 \in [-0.74, -0.58], z_3 \in [0.76, 0.81], \text{   and   } z_4 \in [5, 9] \)
\item \( z_1 \in [-5, -4], \text{   }  z_2 \in [-0.94, -0.76], z_3 \in [0.49, 0.62], \text{   and   } z_4 \in [5, 9] \)

\end{enumerate} }
\litem{
Factor the polynomial below completely. Then, choose the intervals the zeros of the polynomial belong to, where $z_1 \leq z_2 \leq z_3$. \textit{To make the problem easier, all zeros are between -5 and 5.}\[ f(x) = 10x^{3} +39 x^{2} +18 x -27 \]\begin{enumerate}[label=\Alph*.]
\item \( z_1 \in [-1.67, -0.67], \text{   }  z_2 \in [0.52, 0.82], \text{   and   } z_3 \in [2, 3.2] \)
\item \( z_1 \in [-3, -2], \text{   }  z_2 \in [0.04, 0.48], \text{   and   } z_3 \in [2, 3.2] \)
\item \( z_1 \in [-3, -2], \text{   }  z_2 \in [-0.8, -0.48], \text{   and   } z_3 \in [1.3, 1.7] \)
\item \( z_1 \in [-3, -2], \text{   }  z_2 \in [-1.53, -1.4], \text{   and   } z_3 \in [0, 0.7] \)
\item \( z_1 \in [-0.6, 1.4], \text{   }  z_2 \in [1.33, 1.63], \text{   and   } z_3 \in [2, 3.2] \)

\end{enumerate} }
\end{enumerate}

\end{document}