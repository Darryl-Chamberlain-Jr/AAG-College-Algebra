\documentclass[14pt]{extbook}
\usepackage{multicol, enumerate, enumitem, hyperref, color, soul, setspace, parskip, fancyhdr} %General Packages
\usepackage{amssymb, amsthm, amsmath, bbm, latexsym, units, mathtools} %Math Packages
\everymath{\displaystyle} %All math in Display Style
% Packages with additional options
\usepackage[headsep=0.5cm,headheight=12pt, left=1 in,right= 1 in,top= 1 in,bottom= 1 in]{geometry}
\usepackage[usenames,dvipsnames]{xcolor}
\usepackage{dashrule}  % Package to use the command below to create lines between items
\newcommand{\litem}[1]{\item#1\hspace*{-1cm}\rule{\textwidth}{0.4pt}}
\pagestyle{fancy}
\lhead{Module9L}
\chead{}
\rhead{Version B}
\lfoot{2346-2618}
\cfoot{}
\rfoot{Fall 2020}
\begin{document}

\begin{enumerate}
\litem{
Add the following functions, then choose the domain of the resulting function from the list below.\[ f(x) = 7x^{2} +3 x + 1 \text{ and } g(x) = 6x + 4 \]\begin{enumerate}[label=\Alph*.]
\item \( \text{ The domain is all Real numbers except } x = a, \text{ where } a \in [2.6, 10.6] \)
\item \( \text{ The domain is all Real numbers less than or equal to } x = a, \text{ where } a \in [-8.6, 0.4] \)
\item \( \text{ The domain is all Real numbers greater than or equal to } x = a, \text{ where } a \in [-7, -4] \)
\item \( \text{ The domain is all Real numbers except } x = a \text{ and } x = b, \text{ where } a \in [-9.83, -4.83] \text{ and } b \in [-6.17, 1.83] \)
\item \( \text{ The domain is all Real numbers. } \)

\end{enumerate} }
\litem{
Find the inverse of the function below. Then, evaluate the inverse at $x = 8$ and choose the interval that $f^{-1}(8)$ belongs to.\[ f(x) = e^{x+2}-5 \]\begin{enumerate}[label=\Alph*.]
\item \( f^{-1}(8) \in [3.96, 4.92] \)
\item \( f^{-1}(8) \in [-4.14, -3.73] \)
\item \( f^{-1}(8) \in [0.09, 0.79] \)
\item \( f^{-1}(8) \in [-3.63, -2.86] \)
\item \( f^{-1}(8) \in [-3.11, -2.22] \)

\end{enumerate} }
\litem{
Choose the interval below that $f$ composed with $g$ at $x=1$ is in.\[ f(x) = -2x^{3} -1 x^{2} +4 x -4 \text{ and } g(x) = -x^{3} +3 x^{2} -x -3 \]\begin{enumerate}[label=\Alph*.]
\item \( (f \circ g)(1) \in [53, 61] \)
\item \( (f \circ g)(1) \in [0, 12] \)
\item \( (f \circ g)(1) \in [40, 46] \)
\item \( (f \circ g)(1) \in [-11, -7] \)
\item \( \text{It is not possible to compose the two functions.} \)

\end{enumerate} }
\litem{
Multiply the following functions, then choose the domain of the resulting function from the list below.\[ f(x) = \sqrt{6x-22}  \text{ and } g(x) = 3x^{4} +3 x^{3} +5 x^{2} +3 x + 3 \]\begin{enumerate}[label=\Alph*.]
\item \( \text{ The domain is all Real numbers greater than or equal to } x = a, \text{ where } a \in [-2.33, 6.67] \)
\item \( \text{ The domain is all Real numbers except } x = a, \text{ where } a \in [3.2, 12.2] \)
\item \( \text{ The domain is all Real numbers less than or equal to } x = a, \text{ where } a \in [4.83, 11.83] \)
\item \( \text{ The domain is all Real numbers except } x = a \text{ and } x = b, \text{ where } a \in [-0.17, 8.83] \text{ and } b \in [1.6, 6.6] \)
\item \( \text{ The domain is all Real numbers. } \)

\end{enumerate} }
\litem{
Find the inverse of the function below. Then, evaluate the inverse at $x = 7$ and choose the interval that $f^{-1}(7)$ belongs to.\[ f(x) = e^{x+3}-2 \]\begin{enumerate}[label=\Alph*.]
\item \( f^{-1}(7) \in [4.72, 5.49] \)
\item \( f^{-1}(7) \in [0.18, 0.6] \)
\item \( f^{-1}(7) \in [-0.62, -0.49] \)
\item \( f^{-1}(7) \in [-0.47, -0.2] \)
\item \( f^{-1}(7) \in [-1.09, -0.71] \)

\end{enumerate} }
\litem{
Choose the interval below that $f$ composed with $g$ at $x=1$ is in.\[ f(x) = 2x^{3} -4 x^{2} +x \text{ and } g(x) = 4x^{3} -2 x^{2} +x \]\begin{enumerate}[label=\Alph*.]
\item \( (f \circ g)(1) \in [20.5, 25.1] \)
\item \( (f \circ g)(1) \in [24.8, 28.5] \)
\item \( (f \circ g)(1) \in [-14.9, -11.7] \)
\item \( (f \circ g)(1) \in [-9.6, -6.6] \)
\item \( \text{It is not possible to compose the two functions.} \)

\end{enumerate} }
\litem{
Find the inverse of the function below (if it exists). Then, evaluate the inverse at $x = -15$ and choose the interval the $f^{-1}(-15)$ belongs to.\[ f(x) = \sqrt[3]{5 x + 4} \]\begin{enumerate}[label=\Alph*.]
\item \( f^{-1}(-15) \in [675.15, 675.9] \)
\item \( f^{-1}(-15) \in [674.07, 674.56] \)
\item \( f^{-1}(-15) \in [-676.04, -675.8] \)
\item \( f^{-1}(-15) \in [-674.37, -673.71] \)
\item \( \text{ The function is not invertible for all Real numbers. } \)

\end{enumerate} }
\litem{
Determine whether the function below is 1-1.\[ f(x) = (6 x - 29)^3 \]\begin{enumerate}[label=\Alph*.]
\item \( \text{No, because the domain of the function is not $(-\infty, \infty)$.} \)
\item \( \text{No, because there is an $x$-value that goes to 2 different $y$-values.} \)
\item \( \text{No, because the range of the function is not $(-\infty, \infty)$.} \)
\item \( \text{Yes, the function is 1-1.} \)
\item \( \text{No, because there is a $y$-value that goes to 2 different $x$-values.} \)

\end{enumerate} }
\litem{
Find the inverse of the function below (if it exists). Then, evaluate the inverse at $x = 10$ and choose the interval that $f^{-1}(10)$ belongs to.\[ f(x) = 2 x^2 + 3 \]\begin{enumerate}[label=\Alph*.]
\item \( f^{-1}(10) \in [2.77, 3.36] \)
\item \( f^{-1}(10) \in [1.73, 2.01] \)
\item \( f^{-1}(10) \in [2.51, 2.65] \)
\item \( f^{-1}(10) \in [4.62, 4.91] \)
\item \( \text{ The function is not invertible for all Real numbers. } \)

\end{enumerate} }
\litem{
Determine whether the function below is 1-1.\[ f(x) = 20 x^2 + 14 x - 528 \]\begin{enumerate}[label=\Alph*.]
\item \( \text{No, because there is an $x$-value that goes to 2 different $y$-values.} \)
\item \( \text{No, because the range of the function is not $(-\infty, \infty)$.} \)
\item \( \text{Yes, the function is 1-1.} \)
\item \( \text{No, because there is a $y$-value that goes to 2 different $x$-values.} \)
\item \( \text{No, because the domain of the function is not $(-\infty, \infty)$.} \)

\end{enumerate} }
\end{enumerate}

\end{document}