\documentclass{extbook}[14pt]
\usepackage{multicol, enumerate, enumitem, hyperref, color, soul, setspace, parskip, fancyhdr, amssymb, amsthm, amsmath, bbm, latexsym, units, mathtools}
\everymath{\displaystyle}
\usepackage[headsep=0.5cm,headheight=0cm, left=1 in,right= 1 in,top= 1 in,bottom= 1 in]{geometry}
\usepackage{dashrule}  % Package to use the command below to create lines between items
\newcommand{\litem}[1]{\item #1

\rule{\textwidth}{0.4pt}}
\pagestyle{fancy}
\lhead{}
\chead{Answer Key for Module9L Version C}
\rhead{}
\lfoot{2346-2618}
\cfoot{}
\rfoot{Fall 2020}
\begin{document}
\textbf{This key should allow you to understand why you choose the option you did (beyond just getting a question right or wrong). \href{https://xronos.clas.ufl.edu/mac1105spring2020/courseDescriptionAndMisc/Exams/LearningFromResults}{More instructions on how to use this key can be found here}.}

\textbf{If you have a suggestion to make the keys better, \href{https://forms.gle/CZkbZmPbC9XALEE88}{please fill out the short survey here}.}

\textit{Note: This key is auto-generated and may contain issues and/or errors. The keys are reviewed after each exam to ensure grading is done accurately. If there are issues (like duplicate options), they are noted in the offline gradebook. The keys are a work-in-progress to give students as many resources to improve as possible.}

\rule{\textwidth}{0.4pt}

\begin{enumerate}\litem{
Choose the interval below that $f$ composed with $g$ at $x=-1$ is in.
\[ f(x) = -2x^{3} +2 x^{2} +x \text{ and } g(x) = 2x^{3} -2 x^{2} -3 x -2 \]
The solution is \( 69.0 \), which is option C.\begin{enumerate}[label=\Alph*.]
\item \( (f \circ g)(-1) \in [21, 26] \)

 Distractor 1: Corresponds to reversing the composition.
\item \( (f \circ g)(-1) \in [71, 79] \)

 Distractor 2: Corresponds to being slightly off from the solution.
\item \( (f \circ g)(-1) \in [66, 70] \)

* This is the correct solution
\item \( (f \circ g)(-1) \in [13, 19] \)

 Distractor 3: Corresponds to being slightly off from the solution.
\item \( \text{It is not possible to compose the two functions.} \)


\end{enumerate}

\textbf{General Comment:} $f$ composed with $g$ at $x$ means $f(g(x))$. The order matters!
}
\litem{
Subtract the following functions, then choose the domain of the resulting function from the list below.
\[ f(x) = 7x + 6 \text{ and } g(x) = \frac{3}{3x-14} \]
The solution is \( \text{ The domain is all Real numbers except } x = 4.666666666666667 \), which is option C.\begin{enumerate}[label=\Alph*.]
\item \( \text{ The domain is all Real numbers less than or equal to } x = a, \text{ where } a \in [1.5, 7.5] \)


\item \( \text{ The domain is all Real numbers greater than or equal to } x = a, \text{ where } a \in [3, 8] \)


\item \( \text{ The domain is all Real numbers except } x = a, \text{ where } a \in [-3.33, 10.67] \)


\item \( \text{ The domain is all Real numbers except } x = a \text{ and } x = b, \text{ where } a \in [-5.8, -4.8] \text{ and } b \in [-0.33, 6.67] \)


\item \( \text{ The domain is all Real numbers. } \)


\end{enumerate}

\textbf{General Comment:} The new domain is the intersection of the previous domains.
}
\litem{
Find the inverse of the function below. Then, evaluate the inverse at $x = 7$ and choose the interval that $f^{-1}(7)$ belongs to.
\[ f(x) = e^{x+3}+3 \]
The solution is \( f^{-1}(7) = -1.614 \), which is option B.\begin{enumerate}[label=\Alph*.]
\item \( f^{-1}(7) \in [3.79, 5.03] \)

 This solution corresponds to distractor 3.
\item \( f^{-1}(7) \in [-1.91, -0.72] \)

 This is the solution.
\item \( f^{-1}(7) \in [5.14, 5.92] \)

 This solution corresponds to distractor 4.
\item \( f^{-1}(7) \in [3.79, 5.03] \)

 This solution corresponds to distractor 1.
\item \( f^{-1}(7) \in [5.14, 5.92] \)

 This solution corresponds to distractor 2.
\end{enumerate}

\textbf{General Comment:} Natural log and exponential functions always have an inverse. Once you switch the $x$ and $y$, use the conversion $ e^y = x \leftrightarrow y=\ln(x)$.
}
\litem{
Find the inverse of the function below (if it exists). Then, evaluate the inverse at $x = -11$ and choose the interval the $f^{-1}(-11)$ belongs to.
\[ f(x) = \sqrt[3]{3 x - 5} \]
The solution is \( -442.0 \), which is option D.\begin{enumerate}[label=\Alph*.]
\item \( f^{-1}(-11) \in [439.9, 442.5] \)

 This solution corresponds to distractor 2.
\item \( f^{-1}(-11) \in [443.6, 445.4] \)

 This solution corresponds to distractor 3.
\item \( f^{-1}(-11) \in [-446.5, -444.3] \)

 Distractor 1: This corresponds to 
\item \( f^{-1}(-11) \in [-442.2, -441.3] \)

* This is the correct solution.
\item \( \text{ The function is not invertible for all Real numbers. } \)

 This solution corresponds to distractor 4.
\end{enumerate}

\textbf{General Comment:} Be sure you check that the function is 1-1 before trying to find the inverse!
}
\litem{
Determine whether the function below is 1-1.
\[ f(x) = -30 x^2 - 251 x - 494 \]
The solution is \( \text{no} \), which is option A.\begin{enumerate}[label=\Alph*.]
\item \( \text{No, because there is a $y$-value that goes to 2 different $x$-values.} \)

* This is the solution.
\item \( \text{Yes, the function is 1-1.} \)

Corresponds to believing the function passes the Horizontal Line test.
\item \( \text{No, because the range of the function is not $(-\infty, \infty)$.} \)

Corresponds to believing 1-1 means the range is all Real numbers.
\item \( \text{No, because the domain of the function is not $(-\infty, \infty)$.} \)

Corresponds to believing 1-1 means the domain is all Real numbers.
\item \( \text{No, because there is an $x$-value that goes to 2 different $y$-values.} \)

Corresponds to the Vertical Line test, which checks if an expression is a function.
\end{enumerate}

\textbf{General Comment:} There are only two valid options: The function is 1-1 OR No because there is a $y$-value that goes to 2 different $x$-values.
}
\litem{
Determine whether the function below is 1-1.
\[ f(x) = (6 x + 30)^3 \]
The solution is \( \text{yes} \), which is option C.\begin{enumerate}[label=\Alph*.]
\item \( \text{No, because the domain of the function is not $(-\infty, \infty)$.} \)

Corresponds to believing 1-1 means the domain is all Real numbers.
\item \( \text{No, because there is a $y$-value that goes to 2 different $x$-values.} \)

Corresponds to the Horizontal Line test, which this function passes.
\item \( \text{Yes, the function is 1-1.} \)

* This is the solution.
\item \( \text{No, because there is an $x$-value that goes to 2 different $y$-values.} \)

Corresponds to the Vertical Line test, which checks if an expression is a function.
\item \( \text{No, because the range of the function is not $(-\infty, \infty)$.} \)

Corresponds to believing 1-1 means the range is all Real numbers.
\end{enumerate}

\textbf{General Comment:} There are only two valid options: The function is 1-1 OR No because there is a $y$-value that goes to 2 different $x$-values.
}
\litem{
Find the inverse of the function below. Then, evaluate the inverse at $x = 8$ and choose the interval that $f^{-1}(8)$ belongs to.
\[ f(x) = e^{x-3}-3 \]
The solution is \( f^{-1}(8) = 5.398 \), which is option D.\begin{enumerate}[label=\Alph*.]
\item \( f^{-1}(8) \in [-1.9, -0.73] \)

 This solution corresponds to distractor 4.
\item \( f^{-1}(8) \in [-0.66, 0.08] \)

 This solution corresponds to distractor 1.
\item \( f^{-1}(8) \in [-0.66, 0.08] \)

 This solution corresponds to distractor 3.
\item \( f^{-1}(8) \in [5.18, 5.8] \)

 This is the solution.
\item \( f^{-1}(8) \in [-1.9, -0.73] \)

 This solution corresponds to distractor 2.
\end{enumerate}

\textbf{General Comment:} Natural log and exponential functions always have an inverse. Once you switch the $x$ and $y$, use the conversion $ e^y = x \leftrightarrow y=\ln(x)$.
}
\litem{
Find the inverse of the function below (if it exists). Then, evaluate the inverse at $x = -11$ and choose the interval the $f^{-1}(-11)$ belongs to.
\[ f(x) = \sqrt[3]{5 x - 3} \]
The solution is \( -265.6 \), which is option B.\begin{enumerate}[label=\Alph*.]
\item \( f^{-1}(-11) \in [-267.24, -266.28] \)

 Distractor 1: This corresponds to 
\item \( f^{-1}(-11) \in [-266.11, -263.86] \)

* This is the correct solution.
\item \( f^{-1}(-11) \in [266.35, 267.67] \)

 This solution corresponds to distractor 3.
\item \( f^{-1}(-11) \in [265.37, 266.28] \)

 This solution corresponds to distractor 2.
\item \( \text{ The function is not invertible for all Real numbers. } \)

 This solution corresponds to distractor 4.
\end{enumerate}

\textbf{General Comment:} Be sure you check that the function is 1-1 before trying to find the inverse!
}
\litem{
Add the following functions, then choose the domain of the resulting function from the list below.
\[ f(x) = \frac{3}{5x-31} \text{ and } g(x) = 9x^{3} +6 x^{2} +8 x + 2 \]
The solution is \( \text{ The domain is all Real numbers except } x = 6.2 \), which is option C.\begin{enumerate}[label=\Alph*.]
\item \( \text{ The domain is all Real numbers less than or equal to } x = a, \text{ where } a \in [-6.2, -3.2] \)


\item \( \text{ The domain is all Real numbers greater than or equal to } x = a, \text{ where } a \in [4.25, 9.25] \)


\item \( \text{ The domain is all Real numbers except } x = a, \text{ where } a \in [4.2, 10.2] \)


\item \( \text{ The domain is all Real numbers except } x = a \text{ and } x = b, \text{ where } a \in [-0.4, 12.6] \text{ and } b \in [4.4, 6.4] \)


\item \( \text{ The domain is all Real numbers. } \)


\end{enumerate}

\textbf{General Comment:} The new domain is the intersection of the previous domains.
}
\litem{
Choose the interval below that $f$ composed with $g$ at $x=1$ is in.
\[ f(x) = 3x^{3} -2 x^{2} +x \text{ and } g(x) = 3x^{3} +2 x^{2} -4 x + 1 \]
The solution is \( 18.0 \), which is option A.\begin{enumerate}[label=\Alph*.]
\item \( (f \circ g)(1) \in [17, 19.3] \)

* This is the correct solution
\item \( (f \circ g)(1) \in [22.9, 26.7] \)

 Distractor 1: Corresponds to reversing the composition.
\item \( (f \circ g)(1) \in [22.9, 26.7] \)

 Distractor 2: Corresponds to being slightly off from the solution.
\item \( (f \circ g)(1) \in [31.1, 32.9] \)

 Distractor 3: Corresponds to being slightly off from the solution.
\item \( \text{It is not possible to compose the two functions.} \)


\end{enumerate}

\textbf{General Comment:} $f$ composed with $g$ at $x$ means $f(g(x))$. The order matters!
}
\end{enumerate}

\end{document}