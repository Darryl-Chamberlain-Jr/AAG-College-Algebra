\documentclass[14pt]{extbook}
\usepackage{multicol, enumerate, enumitem, hyperref, color, soul, setspace, parskip, fancyhdr} %General Packages
\usepackage{amssymb, amsthm, amsmath, bbm, latexsym, units, mathtools} %Math Packages
\everymath{\displaystyle} %All math in Display Style
% Packages with additional options
\usepackage[headsep=0.5cm,headheight=12pt, left=1 in,right= 1 in,top= 1 in,bottom= 1 in]{geometry}
\usepackage[usenames,dvipsnames]{xcolor}
\usepackage{dashrule}  % Package to use the command below to create lines between items
\newcommand{\litem}[1]{\item#1\hspace*{-1cm}\rule{\textwidth}{0.4pt}}
\pagestyle{fancy}
\lhead{Module9L}
\chead{}
\rhead{Version C}
\lfoot{3286-3678}
\cfoot{}
\rfoot{testing}
\begin{document}

\begin{enumerate}
\litem{
Determine whether the function below is 1-1.\[ f(x) = (6 x + 33)^3 \]\begin{enumerate}[label=\Alph*.]
\item \( \text{No, because the domain of the function is not $(-\infty, \infty)$.} \)
\item \( \text{No, because there is an $x$-value that goes to 2 different $y$-values.} \)
\item \( \text{No, because the range of the function is not $(-\infty, \infty)$.} \)
\item \( \text{No, because there is a $y$-value that goes to 2 different $x$-values.} \)
\item \( \text{Yes, the function is 1-1.} \)

\end{enumerate} }
\litem{
Choose the interval below that $f$ composed with $g$ at $x=-1$ is in.\[ f(x) = -2x^{3} +2 x^{2} +3 x \text{ and } g(x) = 2x^{3} +3 x^{2} -2 x -1 \]\begin{enumerate}[label=\Alph*.]
\item \( (f \circ g)(-1) \in [-3.2, 1.6] \)
\item \( (f \circ g)(-1) \in [1.9, 2.6] \)
\item \( (f \circ g)(-1) \in [2.4, 6] \)
\item \( (f \circ g)(-1) \in [6.2, 9] \)
\item \( \text{It is not possible to compose the two functions.} \)

\end{enumerate} }
\litem{
Subtract the following functions, then choose the domain of the resulting function from the list below.\[ f(x) = 4x + 5 \text{ and } g(x) = \sqrt{6x+32}  \]\begin{enumerate}[label=\Alph*.]
\item \( \text{ The domain is all Real numbers except } x = a, \text{ where } a \in [-8.67, -0.67] \)
\item \( \text{ The domain is all Real numbers less than or equal to } x = a, \text{ where } a \in [4.75, 7.75] \)
\item \( \text{ The domain is all Real numbers greater than or equal to } x = a, \text{ where } a \in [-8.33, -3.33] \)
\item \( \text{ The domain is all Real numbers except } x = a \text{ and } x = b, \text{ where } a \in [-0.83, 7.17] \text{ and } b \in [1.67, 12.67] \)
\item \( \text{ The domain is all Real numbers. } \)

\end{enumerate} }
\litem{
Find the inverse of the function below. Then, evaluate the inverse at $x = 7$ and choose the interval that $f^{-1}(7)$ belongs to.\[ f(x) = e^{x-3}+5 \]\begin{enumerate}[label=\Alph*.]
\item \( f^{-1}(7) \in [-2.42, -2.13] \)
\item \( f^{-1}(7) \in [7.47, 7.57] \)
\item \( f^{-1}(7) \in [7.08, 7.35] \)
\item \( f^{-1}(7) \in [3.44, 4] \)
\item \( f^{-1}(7) \in [6.34, 6.52] \)

\end{enumerate} }
\litem{
Find the inverse of the function below (if it exists). Then, evaluate the inverse at $x = 15$ and choose the interval that $f^{-1}(15)$ belongs to.\[ f(x) = 5 x^2 - 3 \]\begin{enumerate}[label=\Alph*.]
\item \( f^{-1}(15) \in [1.15, 1.57] \)
\item \( f^{-1}(15) \in [2.41, 3.18] \)
\item \( f^{-1}(15) \in [3.77, 4.14] \)
\item \( f^{-1}(15) \in [1.86, 2.05] \)
\item \( \text{ The function is not invertible for all Real numbers. } \)

\end{enumerate} }
\litem{
Subtract the following functions, then choose the domain of the resulting function from the list below.\[ f(x) = \sqrt{5x-31}  \text{ and } g(x) = 9x + 2 \]\begin{enumerate}[label=\Alph*.]
\item \( \text{ The domain is all Real numbers less than or equal to } x = a, \text{ where } a \in [1.33, 6.33] \)
\item \( \text{ The domain is all Real numbers except } x = a, \text{ where } a \in [2.25, 5.25] \)
\item \( \text{ The domain is all Real numbers greater than or equal to } x = a, \text{ where } a \in [4.2, 10.2] \)
\item \( \text{ The domain is all Real numbers except } x = a \text{ and } x = b, \text{ where } a \in [2.33, 4.33] \text{ and } b \in [-9.33, -5.33] \)
\item \( \text{ The domain is all Real numbers. } \)

\end{enumerate} }
\litem{
Choose the interval below that $f$ composed with $g$ at $x=1$ is in.\[ f(x) = 4x^{3} -3 x^{2} +2 x -3 \text{ and } g(x) = 4x^{3} -1 x^{2} -3 x -1 \]\begin{enumerate}[label=\Alph*.]
\item \( (f \circ g)(1) \in [-1, 0] \)
\item \( (f \circ g)(1) \in [0, 9] \)
\item \( (f \circ g)(1) \in [-12, -6] \)
\item \( (f \circ g)(1) \in [-8, -3] \)
\item \( \text{It is not possible to compose the two functions.} \)

\end{enumerate} }
\litem{
Find the inverse of the function below (if it exists). Then, evaluate the inverse at $x = -15$ and choose the interval that $f^{-1}(-15)$ belongs to.\[ f(x) = 2 x^2 - 3 \]\begin{enumerate}[label=\Alph*.]
\item \( f^{-1}(-15) \in [7.06, 7.48] \)
\item \( f^{-1}(-15) \in [4.07, 5.74] \)
\item \( f^{-1}(-15) \in [1.92, 2.72] \)
\item \( f^{-1}(-15) \in [2.94, 3.02] \)
\item \( \text{ The function is not invertible for all Real numbers. } \)

\end{enumerate} }
\litem{
Find the inverse of the function below. Then, evaluate the inverse at $x = 6$ and choose the interval that $f^{-1}(6)$ belongs to.\[ f(x) = e^{x-2}-3 \]\begin{enumerate}[label=\Alph*.]
\item \( f^{-1}(6) \in [-1.64, -1.28] \)
\item \( f^{-1}(6) \in [0.14, 0.28] \)
\item \( f^{-1}(6) \in [-2.08, -1.81] \)
\item \( f^{-1}(6) \in [-1.37, -0.78] \)
\item \( f^{-1}(6) \in [3.76, 4.25] \)

\end{enumerate} }
\litem{
Determine whether the function below is 1-1.\[ f(x) = \sqrt{3 x + 20} \]\begin{enumerate}[label=\Alph*.]
\item \( \text{No, because there is an $x$-value that goes to 2 different $y$-values.} \)
\item \( \text{No, because there is a $y$-value that goes to 2 different $x$-values.} \)
\item \( \text{No, because the range of the function is not $(-\infty, \infty)$.} \)
\item \( \text{No, because the domain of the function is not $(-\infty, \infty)$.} \)
\item \( \text{Yes, the function is 1-1.} \)

\end{enumerate} }
\end{enumerate}

\end{document}