\documentclass[14pt]{extbook}
\usepackage{multicol, enumerate, enumitem, hyperref, color, soul, setspace, parskip, fancyhdr} %General Packages
\usepackage{amssymb, amsthm, amsmath, bbm, latexsym, units, mathtools} %Math Packages
\everymath{\displaystyle} %All math in Display Style
% Packages with additional options
\usepackage[headsep=0.5cm,headheight=12pt, left=1 in,right= 1 in,top= 1 in,bottom= 1 in]{geometry}
\usepackage[usenames,dvipsnames]{xcolor}
\usepackage{dashrule}  % Package to use the command below to create lines between items
\newcommand{\litem}[1]{\item#1\hspace*{-1cm}\rule{\textwidth}{0.4pt}}
\pagestyle{fancy}
\lhead{Module9L}
\chead{}
\rhead{Version C}
\lfoot{2346-2618}
\cfoot{}
\rfoot{Fall 2020}
\begin{document}

\begin{enumerate}
\litem{
Choose the interval below that $f$ composed with $g$ at $x=-1$ is in.\[ f(x) = -2x^{3} +2 x^{2} +x \text{ and } g(x) = 2x^{3} -2 x^{2} -3 x -2 \]\begin{enumerate}[label=\Alph*.]
\item \( (f \circ g)(-1) \in [21, 26] \)
\item \( (f \circ g)(-1) \in [71, 79] \)
\item \( (f \circ g)(-1) \in [66, 70] \)
\item \( (f \circ g)(-1) \in [13, 19] \)
\item \( \text{It is not possible to compose the two functions.} \)

\end{enumerate} }
\litem{
Subtract the following functions, then choose the domain of the resulting function from the list below.\[ f(x) = 7x + 6 \text{ and } g(x) = \frac{3}{3x-14} \]\begin{enumerate}[label=\Alph*.]
\item \( \text{ The domain is all Real numbers less than or equal to } x = a, \text{ where } a \in [1.5, 7.5] \)
\item \( \text{ The domain is all Real numbers greater than or equal to } x = a, \text{ where } a \in [3, 8] \)
\item \( \text{ The domain is all Real numbers except } x = a, \text{ where } a \in [-3.33, 10.67] \)
\item \( \text{ The domain is all Real numbers except } x = a \text{ and } x = b, \text{ where } a \in [-5.8, -4.8] \text{ and } b \in [-0.33, 6.67] \)
\item \( \text{ The domain is all Real numbers. } \)

\end{enumerate} }
\litem{
Find the inverse of the function below. Then, evaluate the inverse at $x = 7$ and choose the interval that $f^{-1}(7)$ belongs to.\[ f(x) = e^{x+3}+3 \]\begin{enumerate}[label=\Alph*.]
\item \( f^{-1}(7) \in [3.79, 5.03] \)
\item \( f^{-1}(7) \in [-1.91, -0.72] \)
\item \( f^{-1}(7) \in [5.14, 5.92] \)
\item \( f^{-1}(7) \in [3.79, 5.03] \)
\item \( f^{-1}(7) \in [5.14, 5.92] \)

\end{enumerate} }
\litem{
Find the inverse of the function below (if it exists). Then, evaluate the inverse at $x = -11$ and choose the interval the $f^{-1}(-11)$ belongs to.\[ f(x) = \sqrt[3]{3 x - 5} \]\begin{enumerate}[label=\Alph*.]
\item \( f^{-1}(-11) \in [439.9, 442.5] \)
\item \( f^{-1}(-11) \in [443.6, 445.4] \)
\item \( f^{-1}(-11) \in [-446.5, -444.3] \)
\item \( f^{-1}(-11) \in [-442.2, -441.3] \)
\item \( \text{ The function is not invertible for all Real numbers. } \)

\end{enumerate} }
\litem{
Determine whether the function below is 1-1.\[ f(x) = -30 x^2 - 251 x - 494 \]\begin{enumerate}[label=\Alph*.]
\item \( \text{No, because there is a $y$-value that goes to 2 different $x$-values.} \)
\item \( \text{Yes, the function is 1-1.} \)
\item \( \text{No, because the range of the function is not $(-\infty, \infty)$.} \)
\item \( \text{No, because the domain of the function is not $(-\infty, \infty)$.} \)
\item \( \text{No, because there is an $x$-value that goes to 2 different $y$-values.} \)

\end{enumerate} }
\litem{
Determine whether the function below is 1-1.\[ f(x) = (6 x + 30)^3 \]\begin{enumerate}[label=\Alph*.]
\item \( \text{No, because the domain of the function is not $(-\infty, \infty)$.} \)
\item \( \text{No, because there is a $y$-value that goes to 2 different $x$-values.} \)
\item \( \text{Yes, the function is 1-1.} \)
\item \( \text{No, because there is an $x$-value that goes to 2 different $y$-values.} \)
\item \( \text{No, because the range of the function is not $(-\infty, \infty)$.} \)

\end{enumerate} }
\litem{
Find the inverse of the function below. Then, evaluate the inverse at $x = 8$ and choose the interval that $f^{-1}(8)$ belongs to.\[ f(x) = e^{x-3}-3 \]\begin{enumerate}[label=\Alph*.]
\item \( f^{-1}(8) \in [-1.9, -0.73] \)
\item \( f^{-1}(8) \in [-0.66, 0.08] \)
\item \( f^{-1}(8) \in [-0.66, 0.08] \)
\item \( f^{-1}(8) \in [5.18, 5.8] \)
\item \( f^{-1}(8) \in [-1.9, -0.73] \)

\end{enumerate} }
\litem{
Find the inverse of the function below (if it exists). Then, evaluate the inverse at $x = -11$ and choose the interval the $f^{-1}(-11)$ belongs to.\[ f(x) = \sqrt[3]{5 x - 3} \]\begin{enumerate}[label=\Alph*.]
\item \( f^{-1}(-11) \in [-267.24, -266.28] \)
\item \( f^{-1}(-11) \in [-266.11, -263.86] \)
\item \( f^{-1}(-11) \in [266.35, 267.67] \)
\item \( f^{-1}(-11) \in [265.37, 266.28] \)
\item \( \text{ The function is not invertible for all Real numbers. } \)

\end{enumerate} }
\litem{
Add the following functions, then choose the domain of the resulting function from the list below.\[ f(x) = \frac{3}{5x-31} \text{ and } g(x) = 9x^{3} +6 x^{2} +8 x + 2 \]\begin{enumerate}[label=\Alph*.]
\item \( \text{ The domain is all Real numbers less than or equal to } x = a, \text{ where } a \in [-6.2, -3.2] \)
\item \( \text{ The domain is all Real numbers greater than or equal to } x = a, \text{ where } a \in [4.25, 9.25] \)
\item \( \text{ The domain is all Real numbers except } x = a, \text{ where } a \in [4.2, 10.2] \)
\item \( \text{ The domain is all Real numbers except } x = a \text{ and } x = b, \text{ where } a \in [-0.4, 12.6] \text{ and } b \in [4.4, 6.4] \)
\item \( \text{ The domain is all Real numbers. } \)

\end{enumerate} }
\litem{
Choose the interval below that $f$ composed with $g$ at $x=1$ is in.\[ f(x) = 3x^{3} -2 x^{2} +x \text{ and } g(x) = 3x^{3} +2 x^{2} -4 x + 1 \]\begin{enumerate}[label=\Alph*.]
\item \( (f \circ g)(1) \in [17, 19.3] \)
\item \( (f \circ g)(1) \in [22.9, 26.7] \)
\item \( (f \circ g)(1) \in [22.9, 26.7] \)
\item \( (f \circ g)(1) \in [31.1, 32.9] \)
\item \( \text{It is not possible to compose the two functions.} \)

\end{enumerate} }
\end{enumerate}

\end{document}