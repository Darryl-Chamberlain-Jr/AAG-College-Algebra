\documentclass{extbook}[14pt]
\usepackage{multicol, enumerate, enumitem, hyperref, color, soul, setspace, parskip, fancyhdr, amssymb, amsthm, amsmath, bbm, latexsym, units, mathtools}
\everymath{\displaystyle}
\usepackage[headsep=0.5cm,headheight=0cm, left=1 in,right= 1 in,top= 1 in,bottom= 1 in]{geometry}
\usepackage{dashrule}  % Package to use the command below to create lines between items
\newcommand{\litem}[1]{\item #1

\rule{\textwidth}{0.4pt}}
\pagestyle{fancy}
\lhead{}
\chead{Answer Key for Progress Quiz 4 Version B}
\rhead{}
\lfoot{8448-1521}
\cfoot{}
\rfoot{Fall 2020}
\begin{document}
\textbf{This key should allow you to understand why you choose the option you did (beyond just getting a question right or wrong). \href{https://xronos.clas.ufl.edu/mac1105spring2020/courseDescriptionAndMisc/Exams/LearningFromResults}{More instructions on how to use this key can be found here}.}

\textbf{If you have a suggestion to make the keys better, \href{https://forms.gle/CZkbZmPbC9XALEE88}{please fill out the short survey here}.}

\textit{Note: This key is auto-generated and may contain issues and/or errors. The keys are reviewed after each exam to ensure grading is done accurately. If there are issues (like duplicate options), they are noted in the offline gradebook. The keys are a work-in-progress to give students as many resources to improve as possible.}

\rule{\textwidth}{0.4pt}

\begin{enumerate}\litem{
Choose the \textbf{smallest} set of Real numbers that the number below belongs to.
\[ \sqrt{\frac{-1560}{12}} \]
The solution is \( \text{Not a Real number} \), which is option D.\begin{enumerate}[label=\Alph*.]
\item \( \text{Whole} \)

These are the counting numbers with 0 (0, 1, 2, 3, ...)
\item \( \text{Irrational} \)

These cannot be written as a fraction of Integers.
\item \( \text{Integer} \)

These are the negative and positive counting numbers (..., -3, -2, -1, 0, 1, 2, 3, ...)
\item \( \text{Not a Real number} \)

* This is the correct option!
\item \( \text{Rational} \)

These are numbers that can be written as fraction of Integers (e.g., -2/3)
\end{enumerate}

\textbf{General Comment:} First, you \textbf{NEED} to simplify the expression. This question simplifies to $\sqrt{130} i$. 
 
 Be sure you look at the simplified fraction and not just the decimal expansion. Numbers such as 13, 17, and 19 provide \textbf{long but repeating/terminating decimal expansions!} 
 
 The only ways to *not* be a Real number are: dividing by 0 or taking the square root of a negative number. 
 
 Irrational numbers are more than just square root of 3: adding or subtracting values from square root of 3 is also irrational.
}
\litem{
Choose the \textbf{smallest} set of Complex numbers that the number below belongs to.
\[ \sqrt{\frac{-1716}{12}}+\sqrt{130} \]
The solution is \( \text{Nonreal Complex} \), which is option B.\begin{enumerate}[label=\Alph*.]
\item \( \text{Not a Complex Number} \)

This is not a number. The only non-Complex number we know is dividing by 0 as this is not a number!
\item \( \text{Nonreal Complex} \)

* This is the correct option!
\item \( \text{Pure Imaginary} \)

This is a Complex number $(a+bi)$ that \textbf{only} has an imaginary part like $2i$.
\item \( \text{Irrational} \)

These cannot be written as a fraction of Integers. Remember: $\pi$ is not an Integer!
\item \( \text{Rational} \)

These are numbers that can be written as fraction of Integers (e.g., -2/3 + 5)
\end{enumerate}

\textbf{General Comment:} Be sure to simplify $i^2 = -1$. This may remove the imaginary portion for your number. If you are having trouble, you may want to look at the \textit{Subgroups of the Real Numbers} section.
}
\litem{
Simplify the expression below into the form $a+bi$. Then, choose the intervals that $a$ and $b$ belong to.
\[ (-5 - 7 i)(9 + 3 i) \]
The solution is \( -24 - 78 i \), which is option A.\begin{enumerate}[label=\Alph*.]
\item \( a \in [-25, -20] \text{ and } b \in [-82, -74] \)

* $-24 - 78 i$, which is the correct option.
\item \( a \in [-67, -65] \text{ and } b \in [48, 49] \)

 $-66 + 48 i$, which corresponds to adding a minus sign in the first term.
\item \( a \in [-49, -40] \text{ and } b \in [-23, -19] \)

 $-45 - 21 i$, which corresponds to just multiplying the real terms to get the real part of the solution and the coefficients in the complex terms to get the complex part.
\item \( a \in [-25, -20] \text{ and } b \in [78, 80] \)

 $-24 + 78 i$, which corresponds to adding a minus sign in both terms.
\item \( a \in [-67, -65] \text{ and } b \in [-52, -47] \)

 $-66 - 48 i$, which corresponds to adding a minus sign in the second term.
\end{enumerate}

\textbf{General Comment:} You can treat $i$ as a variable and distribute. Just remember that $i^2=-1$, so you can continue to reduce after you distribute.
}
\litem{
Simplify the expression below into the form $a+bi$. Then, choose the intervals that $a$ and $b$ belong to.
\[ \frac{-63 + 44 i}{2 + 5 i} \]
The solution is \( 3.24  + 13.90 i \), which is option D.\begin{enumerate}[label=\Alph*.]
\item \( a \in [92, 94.5] \text{ and } b \in [13, 14.5] \)

 $94.00  + 13.90 i$, which corresponds to forgetting to multiply the conjugate by the numerator and using a plus instead of a minus in the denominator.
\item \( a \in [-32.5, -29] \text{ and } b \in [8.5, 10] \)

 $-31.50  + 8.80 i$, which corresponds to just dividing the first term by the first term and the second by the second.
\item \( a \in [2, 5] \text{ and } b \in [400.5, 404] \)

 $3.24  + 403.00 i$, which corresponds to forgetting to multiply the conjugate by the numerator.
\item \( a \in [2, 5] \text{ and } b \in [13, 14.5] \)

* $3.24  + 13.90 i$, which is the correct option.
\item \( a \in [-12.5, -11.5] \text{ and } b \in [-8.5, -7.5] \)

 $-11.93  - 7.83 i$, which corresponds to forgetting to multiply the conjugate by the numerator and not computing the conjugate correctly.
\end{enumerate}

\textbf{General Comment:} Multiply the numerator and denominator by the *conjugate* of the denominator, then simplify. For example, if we have $2+3i$, the conjugate is $2-3i$.
}
\litem{
Choose the \textbf{smallest} set of Complex numbers that the number below belongs to.
\[ \sqrt{\frac{2431}{11}}+\sqrt{119} i \]
The solution is \( \text{Nonreal Complex} \), which is option B.\begin{enumerate}[label=\Alph*.]
\item \( \text{Irrational} \)

These cannot be written as a fraction of Integers. Remember: $\pi$ is not an Integer!
\item \( \text{Nonreal Complex} \)

* This is the correct option!
\item \( \text{Not a Complex Number} \)

This is not a number. The only non-Complex number we know is dividing by 0 as this is not a number!
\item \( \text{Rational} \)

These are numbers that can be written as fraction of Integers (e.g., -2/3 + 5)
\item \( \text{Pure Imaginary} \)

This is a Complex number $(a+bi)$ that \textbf{only} has an imaginary part like $2i$.
\end{enumerate}

\textbf{General Comment:} Be sure to simplify $i^2 = -1$. This may remove the imaginary portion for your number. If you are having trouble, you may want to look at the \textit{Subgroups of the Real Numbers} section.
}
\litem{
Simplify the expression below and choose the interval the simplification is contained within.
\[ 9 - 7^2 + 5 \div 13 * 15 \div 3 \]
The solution is \( -38.077 \), which is option A.\begin{enumerate}[label=\Alph*.]
\item \( [-39.4, -35.9] \)

* -38.077, this is the correct option
\item \( [58.2, 60.9] \)

 59.923, which corresponds to an Order of Operations error: multiplying by negative before squaring. For example: $(-3)^2 \neq -3^2$
\item \( [56.7, 59.1] \)

 58.009, which corresponds to two Order of Operations errors.
\item \( [-41.6, -38.7] \)

 -39.991, which corresponds to an Order of Operations error: not reading left-to-right for multiplication/division.
\item \( \text{None of the above} \)

 You may have gotten this by making an unanticipated error. If you got a value that is not any of the others, please let the coordinator know so they can help you figure out what happened.
\end{enumerate}

\textbf{General Comment:} While you may remember (or were taught) PEMDAS is done in order, it is actually done as P/E/MD/AS. When we are at MD or AS, we read left to right.
}
\litem{
Simplify the expression below into the form $a+bi$. Then, choose the intervals that $a$ and $b$ belong to.
\[ \frac{45 - 33 i}{-8 - 7 i} \]
The solution is \( -1.14  + 5.12 i \), which is option C.\begin{enumerate}[label=\Alph*.]
\item \( a \in [-5.83, -5.57] \text{ and } b \in [4.5, 5] \)

 $-5.62  + 4.71 i$, which corresponds to just dividing the first term by the first term and the second by the second.
\item \( a \in [-1.2, -0.95] \text{ and } b \in [578.25, 579.6] \)

 $-1.14  + 579.00 i$, which corresponds to forgetting to multiply the conjugate by the numerator.
\item \( a \in [-1.2, -0.95] \text{ and } b \in [4.75, 5.25] \)

* $-1.14  + 5.12 i$, which is the correct option.
\item \( a \in [-129.11, -128.9] \text{ and } b \in [4.75, 5.25] \)

 $-129.00  + 5.12 i$, which corresponds to forgetting to multiply the conjugate by the numerator and using a plus instead of a minus in the denominator.
\item \( a \in [-5.48, -5.08] \text{ and } b \in [-0.9, 0.1] \)

 $-5.23  - 0.45 i$, which corresponds to forgetting to multiply the conjugate by the numerator and not computing the conjugate correctly.
\end{enumerate}

\textbf{General Comment:} Multiply the numerator and denominator by the *conjugate* of the denominator, then simplify. For example, if we have $2+3i$, the conjugate is $2-3i$.
}
\litem{
Simplify the expression below into the form $a+bi$. Then, choose the intervals that $a$ and $b$ belong to.
\[ (10 + 6 i)(9 + 4 i) \]
The solution is \( 66 + 94 i \), which is option B.\begin{enumerate}[label=\Alph*.]
\item \( a \in [109, 115] \text{ and } b \in [8, 22] \)

 $114 + 14 i$, which corresponds to adding a minus sign in the second term.
\item \( a \in [65, 67] \text{ and } b \in [93, 95] \)

* $66 + 94 i$, which is the correct option.
\item \( a \in [88, 98] \text{ and } b \in [17, 25] \)

 $90 + 24 i$, which corresponds to just multiplying the real terms to get the real part of the solution and the coefficients in the complex terms to get the complex part.
\item \( a \in [109, 115] \text{ and } b \in [-14, -11] \)

 $114 - 14 i$, which corresponds to adding a minus sign in the first term.
\item \( a \in [65, 67] \text{ and } b \in [-96, -90] \)

 $66 - 94 i$, which corresponds to adding a minus sign in both terms.
\end{enumerate}

\textbf{General Comment:} You can treat $i$ as a variable and distribute. Just remember that $i^2=-1$, so you can continue to reduce after you distribute.
}
\litem{
Choose the \textbf{smallest} set of Real numbers that the number below belongs to.
\[ -\sqrt{\frac{97344}{169}} \]
The solution is \( \text{Integer} \), which is option A.\begin{enumerate}[label=\Alph*.]
\item \( \text{Integer} \)

* This is the correct option!
\item \( \text{Rational} \)

These are numbers that can be written as fraction of Integers (e.g., -2/3)
\item \( \text{Not a Real number} \)

These are Nonreal Complex numbers \textbf{OR} things that are not numbers (e.g., dividing by 0).
\item \( \text{Whole} \)

These are the counting numbers with 0 (0, 1, 2, 3, ...)
\item \( \text{Irrational} \)

These cannot be written as a fraction of Integers.
\end{enumerate}

\textbf{General Comment:} First, you \textbf{NEED} to simplify the expression. This question simplifies to $-312$. 
 
 Be sure you look at the simplified fraction and not just the decimal expansion. Numbers such as 13, 17, and 19 provide \textbf{long but repeating/terminating decimal expansions!} 
 
 The only ways to *not* be a Real number are: dividing by 0 or taking the square root of a negative number. 
 
 Irrational numbers are more than just square root of 3: adding or subtracting values from square root of 3 is also irrational.
}
\litem{
Simplify the expression below and choose the interval the simplification is contained within.
\[ 10 - 8^2 + 7 \div 13 * 5 \div 16 \]
The solution is \( -53.832 \), which is option B.\begin{enumerate}[label=\Alph*.]
\item \( [73.94, 74.07] \)

 74.007, which corresponds to two Order of Operations errors.
\item \( [-53.94, -53.7] \)

* -53.832, this is the correct option
\item \( [-54.05, -53.87] \)

 -53.993, which corresponds to an Order of Operations error: not reading left-to-right for multiplication/division.
\item \( [74.12, 74.25] \)

 74.168, which corresponds to an Order of Operations error: multiplying by negative before squaring. For example: $(-3)^2 \neq -3^2$
\item \( \text{None of the above} \)

 You may have gotten this by making an unanticipated error. If you got a value that is not any of the others, please let the coordinator know so they can help you figure out what happened.
\end{enumerate}

\textbf{General Comment:} While you may remember (or were taught) PEMDAS is done in order, it is actually done as P/E/MD/AS. When we are at MD or AS, we read left to right.
}
\end{enumerate}

\end{document}