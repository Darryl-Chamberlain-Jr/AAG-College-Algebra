\documentclass{extbook}[14pt]
\usepackage{multicol, enumerate, enumitem, hyperref, color, soul, setspace, parskip, fancyhdr, amssymb, amsthm, amsmath, bbm, latexsym, units, mathtools}
\everymath{\displaystyle}
\usepackage[headsep=0.5cm,headheight=0cm, left=1 in,right= 1 in,top= 1 in,bottom= 1 in]{geometry}
\usepackage{dashrule}  % Package to use the command below to create lines between items
\newcommand{\litem}[1]{\item #1

\rule{\textwidth}{0.4pt}}
\pagestyle{fancy}
\lhead{}
\chead{Answer Key for Progress Quiz 4 Version B}
\rhead{}
\lfoot{4378-7085}
\cfoot{}
\rfoot{Fall 2020}
\begin{document}
\textbf{This key should allow you to understand why you choose the option you did (beyond just getting a question right or wrong). \href{https://xronos.clas.ufl.edu/mac1105spring2020/courseDescriptionAndMisc/Exams/LearningFromResults}{More instructions on how to use this key can be found here}.}

\textbf{If you have a suggestion to make the keys better, \href{https://forms.gle/CZkbZmPbC9XALEE88}{please fill out the short survey here}.}

\textit{Note: This key is auto-generated and may contain issues and/or errors. The keys are reviewed after each exam to ensure grading is done accurately. If there are issues (like duplicate options), they are noted in the offline gradebook. The keys are a work-in-progress to give students as many resources to improve as possible.}

\rule{\textwidth}{0.4pt}

\begin{enumerate}\litem{
Simplify the expression below and choose the interval the simplification is contained within.
\[ 17 - 8^2 + 3 \div 5 * 12 \div 15 \]
The solution is \( -46.520 \), which is option A.\begin{enumerate}[label=\Alph*.]
\item \( [-46.69, -46.05] \)

* -46.520, this is the correct option
\item \( [80.42, 81.06] \)

 81.003, which corresponds to two Order of Operations errors.
\item \( [-47.26, -46.63] \)

 -46.997, which corresponds to an Order of Operations error: not reading left-to-right for multiplication/division.
\item \( [81.42, 82.48] \)

 81.480, which corresponds to an Order of Operations error: multiplying by negative before squaring. For example: $(-3)^2 \neq -3^2$
\item \( \text{None of the above} \)

 You may have gotten this by making an unanticipated error. If you got a value that is not any of the others, please let the coordinator know so they can help you figure out what happened.
\end{enumerate}

\textbf{General Comment:} While you may remember (or were taught) PEMDAS is done in order, it is actually done as P/E/MD/AS. When we are at MD or AS, we read left to right.
}
\litem{
Simplify the expression below into the form $a+bi$. Then, choose the intervals that $a$ and $b$ belong to.
\[ \frac{45 + 33 i}{-1 - 4 i} \]
The solution is \( -10.41  + 8.65 i \), which is option B.\begin{enumerate}[label=\Alph*.]
\item \( a \in [-46.5, -44.5] \text{ and } b \in [-9.5, -7.5] \)

 $-45.00  - 8.25 i$, which corresponds to just dividing the first term by the first term and the second by the second.
\item \( a \in [-11, -10] \text{ and } b \in [8, 9.5] \)

* $-10.41  + 8.65 i$, which is the correct option.
\item \( a \in [4, 5.5] \text{ and } b \in [-13, -11] \)

 $5.12  - 12.53 i$, which corresponds to forgetting to multiply the conjugate by the numerator and not computing the conjugate correctly.
\item \( a \in [-178.5, -176.5] \text{ and } b \in [8, 9.5] \)

 $-177.00  + 8.65 i$, which corresponds to forgetting to multiply the conjugate by the numerator and using a plus instead of a minus in the denominator.
\item \( a \in [-11, -10] \text{ and } b \in [146, 147.5] \)

 $-10.41  + 147.00 i$, which corresponds to forgetting to multiply the conjugate by the numerator.
\end{enumerate}

\textbf{General Comment:} Multiply the numerator and denominator by the *conjugate* of the denominator, then simplify. For example, if we have $2+3i$, the conjugate is $2-3i$.
}
\litem{
Simplify the expression below into the form $a+bi$. Then, choose the intervals that $a$ and $b$ belong to.
\[ (-9 + 6 i)(5 - 7 i) \]
The solution is \( -3 + 93 i \), which is option D.\begin{enumerate}[label=\Alph*.]
\item \( a \in [-5, 4] \text{ and } b \in [-96, -87] \)

 $-3 - 93 i$, which corresponds to adding a minus sign in both terms.
\item \( a \in [-45, -39] \text{ and } b \in [-42, -39] \)

 $-45 - 42 i$, which corresponds to just multiplying the real terms to get the real part of the solution and the coefficients in the complex terms to get the complex part.
\item \( a \in [-87, -82] \text{ and } b \in [30, 36] \)

 $-87 + 33 i$, which corresponds to adding a minus sign in the first term.
\item \( a \in [-5, 4] \text{ and } b \in [92, 97] \)

* $-3 + 93 i$, which is the correct option.
\item \( a \in [-87, -82] \text{ and } b \in [-35, -24] \)

 $-87 - 33 i$, which corresponds to adding a minus sign in the second term.
\end{enumerate}

\textbf{General Comment:} You can treat $i$ as a variable and distribute. Just remember that $i^2=-1$, so you can continue to reduce after you distribute.
}
\litem{
Simplify the expression below into the form $a+bi$. Then, choose the intervals that $a$ and $b$ belong to.
\[ \frac{-9 - 88 i}{4 + 2 i} \]
The solution is \( -10.60  - 16.70 i \), which is option B.\begin{enumerate}[label=\Alph*.]
\item \( a \in [-11, -9.5] \text{ and } b \in [-335.5, -332.5] \)

 $-10.60  - 334.00 i$, which corresponds to forgetting to multiply the conjugate by the numerator.
\item \( a \in [-11, -9.5] \text{ and } b \in [-17, -15] \)

* $-10.60  - 16.70 i$, which is the correct option.
\item \( a \in [-212.5, -211.5] \text{ and } b \in [-17, -15] \)

 $-212.00  - 16.70 i$, which corresponds to forgetting to multiply the conjugate by the numerator and using a plus instead of a minus in the denominator.
\item \( a \in [6, 9] \text{ and } b \in [-19, -18] \)

 $7.00  - 18.50 i$, which corresponds to forgetting to multiply the conjugate by the numerator and not computing the conjugate correctly.
\item \( a \in [-4, -2] \text{ and } b \in [-44.5, -43] \)

 $-2.25  - 44.00 i$, which corresponds to just dividing the first term by the first term and the second by the second.
\end{enumerate}

\textbf{General Comment:} Multiply the numerator and denominator by the *conjugate* of the denominator, then simplify. For example, if we have $2+3i$, the conjugate is $2-3i$.
}
\litem{
Choose the \textbf{smallest} set of Complex numbers that the number below belongs to.
\[ \sqrt{\frac{0}{5}}+\sqrt{8}i \]
The solution is \( \text{Pure Imaginary} \), which is option A.\begin{enumerate}[label=\Alph*.]
\item \( \text{Pure Imaginary} \)

* This is the correct option!
\item \( \text{Not a Complex Number} \)

This is not a number. The only non-Complex number we know is dividing by 0 as this is not a number!
\item \( \text{Rational} \)

These are numbers that can be written as fraction of Integers (e.g., -2/3 + 5)
\item \( \text{Nonreal Complex} \)

This is a Complex number $(a+bi)$ that is not Real (has $i$ as part of the number).
\item \( \text{Irrational} \)

These cannot be written as a fraction of Integers. Remember: $\pi$ is not an Integer!
\end{enumerate}

\textbf{General Comment:} Be sure to simplify $i^2 = -1$. This may remove the imaginary portion for your number. If you are having trouble, you may want to look at the \textit{Subgroups of the Real Numbers} section.
}
\litem{
Choose the \textbf{smallest} set of Complex numbers that the number below belongs to.
\[ -\sqrt{\frac{2340}{12}}+3i^2 \]
The solution is \( \text{Irrational} \), which is option D.\begin{enumerate}[label=\Alph*.]
\item \( \text{Not a Complex Number} \)

This is not a number. The only non-Complex number we know is dividing by 0 as this is not a number!
\item \( \text{Nonreal Complex} \)

This is a Complex number $(a+bi)$ that is not Real (has $i$ as part of the number).
\item \( \text{Pure Imaginary} \)

This is a Complex number $(a+bi)$ that \textbf{only} has an imaginary part like $2i$.
\item \( \text{Irrational} \)

* This is the correct option!
\item \( \text{Rational} \)

These are numbers that can be written as fraction of Integers (e.g., -2/3 + 5)
\end{enumerate}

\textbf{General Comment:} Be sure to simplify $i^2 = -1$. This may remove the imaginary portion for your number. If you are having trouble, you may want to look at the \textit{Subgroups of the Real Numbers} section.
}
\litem{
Simplify the expression below into the form $a+bi$. Then, choose the intervals that $a$ and $b$ belong to.
\[ (-5 - 7 i)(4 - 6 i) \]
The solution is \( -62 + 2 i \), which is option B.\begin{enumerate}[label=\Alph*.]
\item \( a \in [-62, -56] \text{ and } b \in [-2.1, -0.2] \)

 $-62 - 2 i$, which corresponds to adding a minus sign in both terms.
\item \( a \in [-62, -56] \text{ and } b \in [1.2, 4.7] \)

* $-62 + 2 i$, which is the correct option.
\item \( a \in [18, 23] \text{ and } b \in [55.5, 58.3] \)

 $22 + 58 i$, which corresponds to adding a minus sign in the first term.
\item \( a \in [-24, -15] \text{ and } b \in [40.1, 42.2] \)

 $-20 + 42 i$, which corresponds to just multiplying the real terms to get the real part of the solution and the coefficients in the complex terms to get the complex part.
\item \( a \in [18, 23] \text{ and } b \in [-59.6, -55.1] \)

 $22 - 58 i$, which corresponds to adding a minus sign in the second term.
\end{enumerate}

\textbf{General Comment:} You can treat $i$ as a variable and distribute. Just remember that $i^2=-1$, so you can continue to reduce after you distribute.
}
\litem{
Choose the \textbf{smallest} set of Real numbers that the number below belongs to.
\[ \sqrt{\frac{-693}{7}} \]
The solution is \( \text{Not a Real number} \), which is option A.\begin{enumerate}[label=\Alph*.]
\item \( \text{Not a Real number} \)

* This is the correct option!
\item \( \text{Integer} \)

These are the negative and positive counting numbers (..., -3, -2, -1, 0, 1, 2, 3, ...)
\item \( \text{Whole} \)

These are the counting numbers with 0 (0, 1, 2, 3, ...)
\item \( \text{Irrational} \)

These cannot be written as a fraction of Integers.
\item \( \text{Rational} \)

These are numbers that can be written as fraction of Integers (e.g., -2/3)
\end{enumerate}

\textbf{General Comment:} First, you \textbf{NEED} to simplify the expression. This question simplifies to $\sqrt{99} i$. 
 
 Be sure you look at the simplified fraction and not just the decimal expansion. Numbers such as 13, 17, and 19 provide \textbf{long but repeating/terminating decimal expansions!} 
 
 The only ways to *not* be a Real number are: dividing by 0 or taking the square root of a negative number. 
 
 Irrational numbers are more than just square root of 3: adding or subtracting values from square root of 3 is also irrational.
}
\litem{
Choose the \textbf{smallest} set of Real numbers that the number below belongs to.
\[ -\sqrt{\frac{144}{169}} \]
The solution is \( \text{Rational} \), which is option D.\begin{enumerate}[label=\Alph*.]
\item \( \text{Not a Real number} \)

These are Nonreal Complex numbers \textbf{OR} things that are not numbers (e.g., dividing by 0).
\item \( \text{Whole} \)

These are the counting numbers with 0 (0, 1, 2, 3, ...)
\item \( \text{Irrational} \)

These cannot be written as a fraction of Integers.
\item \( \text{Rational} \)

* This is the correct option!
\item \( \text{Integer} \)

These are the negative and positive counting numbers (..., -3, -2, -1, 0, 1, 2, 3, ...)
\end{enumerate}

\textbf{General Comment:} First, you \textbf{NEED} to simplify the expression. This question simplifies to $-\frac{12}{13}$. 
 
 Be sure you look at the simplified fraction and not just the decimal expansion. Numbers such as 13, 17, and 19 provide \textbf{long but repeating/terminating decimal expansions!} 
 
 The only ways to *not* be a Real number are: dividing by 0 or taking the square root of a negative number. 
 
 Irrational numbers are more than just square root of 3: adding or subtracting values from square root of 3 is also irrational.
}
\litem{
Simplify the expression below and choose the interval the simplification is contained within.
\[ 3 - 6^2 + 12 \div 4 * 10 \div 11 \]
The solution is \( -30.273 \), which is option B.\begin{enumerate}[label=\Alph*.]
\item \( [40.6, 42.5] \)

 41.727, which corresponds to an Order of Operations error: multiplying by negative before squaring. For example: $(-3)^2 \neq -3^2$
\item \( [-32.9, -27.4] \)

* -30.273, this is the correct option
\item \( [-33.5, -31.3] \)

 -32.973, which corresponds to an Order of Operations error: not reading left-to-right for multiplication/division.
\item \( [38.3, 39.1] \)

 39.027, which corresponds to two Order of Operations errors.
\item \( \text{None of the above} \)

 You may have gotten this by making an unanticipated error. If you got a value that is not any of the others, please let the coordinator know so they can help you figure out what happened.
\end{enumerate}

\textbf{General Comment:} While you may remember (or were taught) PEMDAS is done in order, it is actually done as P/E/MD/AS. When we are at MD or AS, we read left to right.
}
\end{enumerate}

\end{document}