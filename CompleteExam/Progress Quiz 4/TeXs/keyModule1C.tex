\documentclass{extbook}[14pt]
\usepackage{multicol, enumerate, enumitem, hyperref, color, soul, setspace, parskip, fancyhdr, amssymb, amsthm, amsmath, bbm, latexsym, units, mathtools}
\everymath{\displaystyle}
\usepackage[headsep=0.5cm,headheight=0cm, left=1 in,right= 1 in,top= 1 in,bottom= 1 in]{geometry}
\usepackage{dashrule}  % Package to use the command below to create lines between items
\newcommand{\litem}[1]{\item #1

\rule{\textwidth}{0.4pt}}
\pagestyle{fancy}
\lhead{}
\chead{Answer Key for Progress Quiz 4 Version C}
\rhead{}
\lfoot{4378-7085}
\cfoot{}
\rfoot{Fall 2020}
\begin{document}
\textbf{This key should allow you to understand why you choose the option you did (beyond just getting a question right or wrong). \href{https://xronos.clas.ufl.edu/mac1105spring2020/courseDescriptionAndMisc/Exams/LearningFromResults}{More instructions on how to use this key can be found here}.}

\textbf{If you have a suggestion to make the keys better, \href{https://forms.gle/CZkbZmPbC9XALEE88}{please fill out the short survey here}.}

\textit{Note: This key is auto-generated and may contain issues and/or errors. The keys are reviewed after each exam to ensure grading is done accurately. If there are issues (like duplicate options), they are noted in the offline gradebook. The keys are a work-in-progress to give students as many resources to improve as possible.}

\rule{\textwidth}{0.4pt}

\begin{enumerate}\litem{
Simplify the expression below into the form $a+bi$. Then, choose the intervals that $a$ and $b$ belong to.
\[ (-9 - 3 i)(-8 + 4 i) \]
The solution is \( 84 - 12 i \), which is option B.\begin{enumerate}[label=\Alph*.]
\item \( a \in [60, 63] \text{ and } b \in [-61, -51] \)

 $60 - 60 i$, which corresponds to adding a minus sign in the first term.
\item \( a \in [84, 89] \text{ and } b \in [-17, -11] \)

* $84 - 12 i$, which is the correct option.
\item \( a \in [84, 89] \text{ and } b \in [8, 13] \)

 $84 + 12 i$, which corresponds to adding a minus sign in both terms.
\item \( a \in [70, 73] \text{ and } b \in [-17, -11] \)

 $72 - 12 i$, which corresponds to just multiplying the real terms to get the real part of the solution and the coefficients in the complex terms to get the complex part.
\item \( a \in [60, 63] \text{ and } b \in [57, 62] \)

 $60 + 60 i$, which corresponds to adding a minus sign in the second term.
\end{enumerate}

\textbf{General Comment:} You can treat $i$ as a variable and distribute. Just remember that $i^2=-1$, so you can continue to reduce after you distribute.
}
\litem{
Simplify the expression below and choose the interval the simplification is contained within.
\[ 5 - 4 \div 10 * 7 - (8 * 15) \]
The solution is \( -117.800 \), which is option B.\begin{enumerate}[label=\Alph*.]
\item \( [-90, -84] \)

 -87.000, which corresponds to not distributing a negative correctly.
\item \( [-122.8, -116.8] \)

* -117.800, which is the correct option.
\item \( [-116.06, -107.06] \)

 -115.057, which corresponds to an Order of Operations error: not reading left-to-right for multiplication/division.
\item \( [119.94, 128.94] \)

 124.943, which corresponds to not distributing addition and subtraction correctly.
\item \( \text{None of the above} \)

 You may have gotten this by making an unanticipated error. If you got a value that is not any of the others, please let the coordinator know so they can help you figure out what happened.
\end{enumerate}

\textbf{General Comment:} While you may remember (or were taught) PEMDAS is done in order, it is actually done as P/E/MD/AS. When we are at MD or AS, we read left to right.
}
\litem{
Choose the \textbf{smallest} set of Real numbers that the number below belongs to.
\[ \sqrt{\frac{2304}{36}} \]
The solution is \( \text{Whole} \), which is option A.\begin{enumerate}[label=\Alph*.]
\item \( \text{Whole} \)

* This is the correct option!
\item \( \text{Irrational} \)

These cannot be written as a fraction of Integers.
\item \( \text{Rational} \)

These are numbers that can be written as fraction of Integers (e.g., -2/3)
\item \( \text{Integer} \)

These are the negative and positive counting numbers (..., -3, -2, -1, 0, 1, 2, 3, ...)
\item \( \text{Not a Real number} \)

These are Nonreal Complex numbers \textbf{OR} things that are not numbers (e.g., dividing by 0).
\end{enumerate}

\textbf{General Comment:} First, you \textbf{NEED} to simplify the expression. This question simplifies to $48$. 
 
 Be sure you look at the simplified fraction and not just the decimal expansion. Numbers such as 13, 17, and 19 provide \textbf{long but repeating/terminating decimal expansions!} 
 
 The only ways to *not* be a Real number are: dividing by 0 or taking the square root of a negative number. 
 
 Irrational numbers are more than just square root of 3: adding or subtracting values from square root of 3 is also irrational.
}
\litem{
Choose the \textbf{smallest} set of Complex numbers that the number below belongs to.
\[ \sqrt{\frac{81}{625}}+\sqrt{85} i \]
The solution is \( \text{Nonreal Complex} \), which is option B.\begin{enumerate}[label=\Alph*.]
\item \( \text{Rational} \)

These are numbers that can be written as fraction of Integers (e.g., -2/3 + 5)
\item \( \text{Nonreal Complex} \)

* This is the correct option!
\item \( \text{Not a Complex Number} \)

This is not a number. The only non-Complex number we know is dividing by 0 as this is not a number!
\item \( \text{Pure Imaginary} \)

This is a Complex number $(a+bi)$ that \textbf{only} has an imaginary part like $2i$.
\item \( \text{Irrational} \)

These cannot be written as a fraction of Integers. Remember: $\pi$ is not an Integer!
\end{enumerate}

\textbf{General Comment:} Be sure to simplify $i^2 = -1$. This may remove the imaginary portion for your number. If you are having trouble, you may want to look at the \textit{Subgroups of the Real Numbers} section.
}
\litem{
Choose the \textbf{smallest} set of Real numbers that the number below belongs to.
\[ \sqrt{\frac{65025}{289}} \]
The solution is \( \text{Whole} \), which is option C.\begin{enumerate}[label=\Alph*.]
\item \( \text{Irrational} \)

These cannot be written as a fraction of Integers.
\item \( \text{Integer} \)

These are the negative and positive counting numbers (..., -3, -2, -1, 0, 1, 2, 3, ...)
\item \( \text{Whole} \)

* This is the correct option!
\item \( \text{Rational} \)

These are numbers that can be written as fraction of Integers (e.g., -2/3)
\item \( \text{Not a Real number} \)

These are Nonreal Complex numbers \textbf{OR} things that are not numbers (e.g., dividing by 0).
\end{enumerate}

\textbf{General Comment:} First, you \textbf{NEED} to simplify the expression. This question simplifies to $255$. 
 
 Be sure you look at the simplified fraction and not just the decimal expansion. Numbers such as 13, 17, and 19 provide \textbf{long but repeating/terminating decimal expansions!} 
 
 The only ways to *not* be a Real number are: dividing by 0 or taking the square root of a negative number. 
 
 Irrational numbers are more than just square root of 3: adding or subtracting values from square root of 3 is also irrational.
}
\litem{
Simplify the expression below into the form $a+bi$. Then, choose the intervals that $a$ and $b$ belong to.
\[ \frac{18 + 66 i}{-8 + 5 i} \]
The solution is \( 2.09  - 6.94 i \), which is option E.\begin{enumerate}[label=\Alph*.]
\item \( a \in [-6.5, -4.5] \text{ and } b \in [-5.5, -3.5] \)

 $-5.33  - 4.92 i$, which corresponds to forgetting to multiply the conjugate by the numerator and not computing the conjugate correctly.
\item \( a \in [184.5, 187.5] \text{ and } b \in [-8.5, -6] \)

 $186.00  - 6.94 i$, which corresponds to forgetting to multiply the conjugate by the numerator and using a plus instead of a minus in the denominator.
\item \( a \in [-3.5, -1] \text{ and } b \in [12.5, 14.5] \)

 $-2.25  + 13.20 i$, which corresponds to just dividing the first term by the first term and the second by the second.
\item \( a \in [1.5, 2.5] \text{ and } b \in [-619, -617.5] \)

 $2.09  - 618.00 i$, which corresponds to forgetting to multiply the conjugate by the numerator.
\item \( a \in [1.5, 2.5] \text{ and } b \in [-8.5, -6] \)

* $2.09  - 6.94 i$, which is the correct option.
\end{enumerate}

\textbf{General Comment:} Multiply the numerator and denominator by the *conjugate* of the denominator, then simplify. For example, if we have $2+3i$, the conjugate is $2-3i$.
}
\litem{
Simplify the expression below into the form $a+bi$. Then, choose the intervals that $a$ and $b$ belong to.
\[ (7 - 10 i)(2 + 8 i) \]
The solution is \( 94 + 36 i \), which is option E.\begin{enumerate}[label=\Alph*.]
\item \( a \in [-70, -65] \text{ and } b \in [-78, -70] \)

 $-66 - 76 i$, which corresponds to adding a minus sign in the second term.
\item \( a \in [11, 18] \text{ and } b \in [-83, -77] \)

 $14 - 80 i$, which corresponds to just multiplying the real terms to get the real part of the solution and the coefficients in the complex terms to get the complex part.
\item \( a \in [91, 99] \text{ and } b \in [-41, -34] \)

 $94 - 36 i$, which corresponds to adding a minus sign in both terms.
\item \( a \in [-70, -65] \text{ and } b \in [72, 78] \)

 $-66 + 76 i$, which corresponds to adding a minus sign in the first term.
\item \( a \in [91, 99] \text{ and } b \in [34, 37] \)

* $94 + 36 i$, which is the correct option.
\end{enumerate}

\textbf{General Comment:} You can treat $i$ as a variable and distribute. Just remember that $i^2=-1$, so you can continue to reduce after you distribute.
}
\litem{
Simplify the expression below into the form $a+bi$. Then, choose the intervals that $a$ and $b$ belong to.
\[ \frac{9 - 66 i}{3 + 5 i} \]
The solution is \( -8.91  - 7.15 i \), which is option B.\begin{enumerate}[label=\Alph*.]
\item \( a \in [9.5, 11] \text{ and } b \in [-5, -3.5] \)

 $10.50  - 4.50 i$, which corresponds to forgetting to multiply the conjugate by the numerator and not computing the conjugate correctly.
\item \( a \in [-10, -8.5] \text{ and } b \in [-8.5, -6] \)

* $-8.91  - 7.15 i$, which is the correct option.
\item \( a \in [-10, -8.5] \text{ and } b \in [-243.5, -242] \)

 $-8.91  - 243.00 i$, which corresponds to forgetting to multiply the conjugate by the numerator.
\item \( a \in [-304.5, -302] \text{ and } b \in [-8.5, -6] \)

 $-303.00  - 7.15 i$, which corresponds to forgetting to multiply the conjugate by the numerator and using a plus instead of a minus in the denominator.
\item \( a \in [2, 3.5] \text{ and } b \in [-13.5, -12.5] \)

 $3.00  - 13.20 i$, which corresponds to just dividing the first term by the first term and the second by the second.
\end{enumerate}

\textbf{General Comment:} Multiply the numerator and denominator by the *conjugate* of the denominator, then simplify. For example, if we have $2+3i$, the conjugate is $2-3i$.
}
\litem{
Simplify the expression below and choose the interval the simplification is contained within.
\[ 11 - 1^2 + 4 \div 8 * 10 \div 7 \]
The solution is \( 10.714 \), which is option C.\begin{enumerate}[label=\Alph*.]
\item \( [12.09, 14.12] \)

 12.714, which corresponds to an Order of Operations error: multiplying by negative before squaring. For example: $(-3)^2 \neq -3^2$
\item \( [9.13, 10.51] \)

 10.007, which corresponds to an Order of Operations error: not reading left-to-right for multiplication/division.
\item \( [10.06, 10.93] \)

* 10.714, this is the correct option
\item \( [11.32, 12.2] \)

 12.007, which corresponds to two Order of Operations errors.
\item \( \text{None of the above} \)

 You may have gotten this by making an unanticipated error. If you got a value that is not any of the others, please let the coordinator know so they can help you figure out what happened.
\end{enumerate}

\textbf{General Comment:} While you may remember (or were taught) PEMDAS is done in order, it is actually done as P/E/MD/AS. When we are at MD or AS, we read left to right.
}
\litem{
Choose the \textbf{smallest} set of Complex numbers that the number below belongs to.
\[ \frac{\sqrt{182}}{6}+\sqrt{-3}i \]
The solution is \( \text{Irrational} \), which is option B.\begin{enumerate}[label=\Alph*.]
\item \( \text{Nonreal Complex} \)

This is a Complex number $(a+bi)$ that is not Real (has $i$ as part of the number).
\item \( \text{Irrational} \)

* This is the correct option!
\item \( \text{Pure Imaginary} \)

This is a Complex number $(a+bi)$ that \textbf{only} has an imaginary part like $2i$.
\item \( \text{Rational} \)

These are numbers that can be written as fraction of Integers (e.g., -2/3 + 5)
\item \( \text{Not a Complex Number} \)

This is not a number. The only non-Complex number we know is dividing by 0 as this is not a number!
\end{enumerate}

\textbf{General Comment:} Be sure to simplify $i^2 = -1$. This may remove the imaginary portion for your number. If you are having trouble, you may want to look at the \textit{Subgroups of the Real Numbers} section.
}
\end{enumerate}

\end{document}