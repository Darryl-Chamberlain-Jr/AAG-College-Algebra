\documentclass{extbook}[14pt]
\usepackage{multicol, enumerate, enumitem, hyperref, color, soul, setspace, parskip, fancyhdr, amssymb, amsthm, amsmath, bbm, latexsym, units, mathtools}
\everymath{\displaystyle}
\usepackage[headsep=0.5cm,headheight=0cm, left=1 in,right= 1 in,top= 1 in,bottom= 1 in]{geometry}
\usepackage{dashrule}  % Package to use the command below to create lines between items
\newcommand{\litem}[1]{\item #1

\rule{\textwidth}{0.4pt}}
\pagestyle{fancy}
\lhead{}
\chead{Answer Key for Progress Quiz 4 Version C}
\rhead{}
\lfoot{9187-5854}
\cfoot{}
\rfoot{Spring 2021}
\begin{document}
\textbf{This key should allow you to understand why you choose the option you did (beyond just getting a question right or wrong). \href{https://xronos.clas.ufl.edu/mac1105spring2020/courseDescriptionAndMisc/Exams/LearningFromResults}{More instructions on how to use this key can be found here}.}

\textbf{If you have a suggestion to make the keys better, \href{https://forms.gle/CZkbZmPbC9XALEE88}{please fill out the short survey here}.}

\textit{Note: This key is auto-generated and may contain issues and/or errors. The keys are reviewed after each exam to ensure grading is done accurately. If there are issues (like duplicate options), they are noted in the offline gradebook. The keys are a work-in-progress to give students as many resources to improve as possible.}

\rule{\textwidth}{0.4pt}

\begin{enumerate}\litem{
Choose the \textbf{smallest} set of Complex numbers that the number below belongs to.
\[ \frac{-21}{0}+\sqrt{156} i \]The solution is \( \text{Not a Complex Number} \), which is option E.\begin{enumerate}[label=\Alph*.]
\item \( \text{Rational} \)

These are numbers that can be written as fraction of Integers (e.g., -2/3 + 5)
\item \( \text{Nonreal Complex} \)

This is a Complex number $(a+bi)$ that is not Real (has $i$ as part of the number).
\item \( \text{Pure Imaginary} \)

This is a Complex number $(a+bi)$ that \textbf{only} has an imaginary part like $2i$.
\item \( \text{Irrational} \)

These cannot be written as a fraction of Integers. Remember: $\pi$ is not an Integer!
\item \( \text{Not a Complex Number} \)

* This is the correct option!
\end{enumerate}

\textbf{General Comment:} Be sure to simplify $i^2 = -1$. This may remove the imaginary portion for your number. If you are having trouble, you may want to look at the \textit{Subgroups of the Real Numbers} section.
}
\litem{
Choose the \textbf{smallest} set of Complex numbers that the number below belongs to.
\[ \sqrt{\frac{-833}{0}} i+\sqrt{195}i \]The solution is \( \text{Not a Complex Number} \), which is option E.\begin{enumerate}[label=\Alph*.]
\item \( \text{Nonreal Complex} \)

This is a Complex number $(a+bi)$ that is not Real (has $i$ as part of the number).
\item \( \text{Rational} \)

These are numbers that can be written as fraction of Integers (e.g., -2/3 + 5)
\item \( \text{Irrational} \)

These cannot be written as a fraction of Integers. Remember: $\pi$ is not an Integer!
\item \( \text{Pure Imaginary} \)

This is a Complex number $(a+bi)$ that \textbf{only} has an imaginary part like $2i$.
\item \( \text{Not a Complex Number} \)

* This is the correct option!
\end{enumerate}

\textbf{General Comment:} Be sure to simplify $i^2 = -1$. This may remove the imaginary portion for your number. If you are having trouble, you may want to look at the \textit{Subgroups of the Real Numbers} section.
}
\litem{
Simplify the expression below and choose the interval the simplification is contained within.
\[ 14 - 3^2 + 11 \div 19 * 8 \div 9 \]The solution is \( 5.515 \), which is option A.\begin{enumerate}[label=\Alph*.]
\item \( [5.28, 5.63] \)

* 5.515, this is the correct option
\item \( [4.7, 5.38] \)

 5.008, which corresponds to an Order of Operations error: not reading left-to-right for multiplication/division.
\item \( [23.49, 23.58] \)

 23.515, which corresponds to an Order of Operations error: multiplying by negative before squaring. For example: $(-3)^2 \neq -3^2$
\item \( [22.8, 23.28] \)

 23.008, which corresponds to two Order of Operations errors.
\item \( \text{None of the above} \)

 You may have gotten this by making an unanticipated error. If you got a value that is not any of the others, please let the coordinator know so they can help you figure out what happened.
\end{enumerate}

\textbf{General Comment:} While you may remember (or were taught) PEMDAS is done in order, it is actually done as P/E/MD/AS. When we are at MD or AS, we read left to right.
}
\litem{
Simplify the expression below and choose the interval the simplification is contained within.
\[ 15 - 9 \div 6 * 11 - (7 * 19) \]The solution is \( -134.500 \), which is option A.\begin{enumerate}[label=\Alph*.]
\item \( [-136.5, -132.5] \)

* -134.500, which is the correct option.
\item \( [146.86, 151.86] \)

 147.864, which corresponds to not distributing addition and subtraction correctly.
\item \( [-165.5, -155.5] \)

 -161.500, which corresponds to not distributing a negative correctly.
\item \( [-118.14, -109.14] \)

 -118.136, which corresponds to an Order of Operations error: not reading left-to-right for multiplication/division.
\item \( \text{None of the above} \)

 You may have gotten this by making an unanticipated error. If you got a value that is not any of the others, please let the coordinator know so they can help you figure out what happened.
\end{enumerate}

\textbf{General Comment:} While you may remember (or were taught) PEMDAS is done in order, it is actually done as P/E/MD/AS. When we are at MD or AS, we read left to right.
}
\litem{
Simplify the expression below into the form $a+bi$. Then, choose the intervals that $a$ and $b$ belong to.
\[ (5 - 4 i)(-7 - 10 i) \]The solution is \( -75 - 22 i \), which is option B.\begin{enumerate}[label=\Alph*.]
\item \( a \in [2, 6] \text{ and } b \in [74, 82] \)

 $5 + 78 i$, which corresponds to adding a minus sign in the second term.
\item \( a \in [-78, -72] \text{ and } b \in [-24, -18] \)

* $-75 - 22 i$, which is the correct option.
\item \( a \in [-78, -72] \text{ and } b \in [22, 30] \)

 $-75 + 22 i$, which corresponds to adding a minus sign in both terms.
\item \( a \in [2, 6] \text{ and } b \in [-84, -77] \)

 $5 - 78 i$, which corresponds to adding a minus sign in the first term.
\item \( a \in [-36, -33] \text{ and } b \in [39, 47] \)

 $-35 + 40 i$, which corresponds to just multiplying the real terms to get the real part of the solution and the coefficients in the complex terms to get the complex part.
\end{enumerate}

\textbf{General Comment:} You can treat $i$ as a variable and distribute. Just remember that $i^2=-1$, so you can continue to reduce after you distribute.
}
\litem{
Choose the \textbf{smallest} set of Real numbers that the number below belongs to.
\[ -\sqrt{\frac{1848}{14}} \]The solution is \( \text{Irrational} \), which is option C.\begin{enumerate}[label=\Alph*.]
\item \( \text{Not a Real number} \)

These are Nonreal Complex numbers \textbf{OR} things that are not numbers (e.g., dividing by 0).
\item \( \text{Whole} \)

These are the counting numbers with 0 (0, 1, 2, 3, ...)
\item \( \text{Irrational} \)

* This is the correct option!
\item \( \text{Rational} \)

These are numbers that can be written as fraction of Integers (e.g., -2/3)
\item \( \text{Integer} \)

These are the negative and positive counting numbers (..., -3, -2, -1, 0, 1, 2, 3, ...)
\end{enumerate}

\textbf{General Comment:} First, you \textbf{NEED} to simplify the expression. This question simplifies to $-\sqrt{132}$. 
 
 Be sure you look at the simplified fraction and not just the decimal expansion. Numbers such as 13, 17, and 19 provide \textbf{long but repeating/terminating decimal expansions!} 
 
 The only ways to *not* be a Real number are: dividing by 0 or taking the square root of a negative number. 
 
 Irrational numbers are more than just square root of 3: adding or subtracting values from square root of 3 is also irrational.
}
\litem{
Simplify the expression below into the form $a+bi$. Then, choose the intervals that $a$ and $b$ belong to.
\[ \frac{-36 - 66 i}{2 + 7 i} \]The solution is \( -10.08  + 2.26 i \), which is option A.\begin{enumerate}[label=\Alph*.]
\item \( a \in [-11.5, -9] \text{ and } b \in [1.5, 3] \)

* $-10.08  + 2.26 i$, which is the correct option.
\item \( a \in [-18.5, -17.5] \text{ and } b \in [-10, -8] \)

 $-18.00  - 9.43 i$, which corresponds to just dividing the first term by the first term and the second by the second.
\item \( a \in [6.5, 8] \text{ and } b \in [-8.5, -7] \)

 $7.36  - 7.25 i$, which corresponds to forgetting to multiply the conjugate by the numerator and not computing the conjugate correctly.
\item \( a \in [-536, -533.5] \text{ and } b \in [1.5, 3] \)

 $-534.00  + 2.26 i$, which corresponds to forgetting to multiply the conjugate by the numerator and using a plus instead of a minus in the denominator.
\item \( a \in [-11.5, -9] \text{ and } b \in [119, 120.5] \)

 $-10.08  + 120.00 i$, which corresponds to forgetting to multiply the conjugate by the numerator.
\end{enumerate}

\textbf{General Comment:} Multiply the numerator and denominator by the *conjugate* of the denominator, then simplify. For example, if we have $2+3i$, the conjugate is $2-3i$.
}
\litem{
Choose the \textbf{smallest} set of Real numbers that the number below belongs to.
\[ -\sqrt{\frac{6}{0}} \]The solution is \( \text{Not a Real number} \), which is option A.\begin{enumerate}[label=\Alph*.]
\item \( \text{Not a Real number} \)

* This is the correct option!
\item \( \text{Irrational} \)

These cannot be written as a fraction of Integers.
\item \( \text{Whole} \)

These are the counting numbers with 0 (0, 1, 2, 3, ...)
\item \( \text{Rational} \)

These are numbers that can be written as fraction of Integers (e.g., -2/3)
\item \( \text{Integer} \)

These are the negative and positive counting numbers (..., -3, -2, -1, 0, 1, 2, 3, ...)
\end{enumerate}

\textbf{General Comment:} First, you \textbf{NEED} to simplify the expression. This question simplifies to $-\sqrt{\frac{6}{0}}$. 
 
 Be sure you look at the simplified fraction and not just the decimal expansion. Numbers such as 13, 17, and 19 provide \textbf{long but repeating/terminating decimal expansions!} 
 
 The only ways to *not* be a Real number are: dividing by 0 or taking the square root of a negative number. 
 
 Irrational numbers are more than just square root of 3: adding or subtracting values from square root of 3 is also irrational.
}
\litem{
Simplify the expression below into the form $a+bi$. Then, choose the intervals that $a$ and $b$ belong to.
\[ \frac{-72 + 55 i}{7 - 3 i} \]The solution is \( -11.53  + 2.91 i \), which is option E.\begin{enumerate}[label=\Alph*.]
\item \( a \in [-671, -668] \text{ and } b \in [1.5, 4] \)

 $-669.00  + 2.91 i$, which corresponds to forgetting to multiply the conjugate by the numerator and using a plus instead of a minus in the denominator.
\item \( a \in [-12.5, -10.5] \text{ and } b \in [168, 170] \)

 $-11.53  + 169.00 i$, which corresponds to forgetting to multiply the conjugate by the numerator.
\item \( a \in [-11, -9.5] \text{ and } b \in [-19, -17] \)

 $-10.29  - 18.33 i$, which corresponds to just dividing the first term by the first term and the second by the second.
\item \( a \in [-7, -5.5] \text{ and } b \in [9, 11] \)

 $-5.84  + 10.36 i$, which corresponds to forgetting to multiply the conjugate by the numerator and not computing the conjugate correctly.
\item \( a \in [-12.5, -10.5] \text{ and } b \in [1.5, 4] \)

* $-11.53  + 2.91 i$, which is the correct option.
\end{enumerate}

\textbf{General Comment:} Multiply the numerator and denominator by the *conjugate* of the denominator, then simplify. For example, if we have $2+3i$, the conjugate is $2-3i$.
}
\litem{
Simplify the expression below into the form $a+bi$. Then, choose the intervals that $a$ and $b$ belong to.
\[ (-2 - 4 i)(10 - 3 i) \]The solution is \( -32 - 34 i \), which is option B.\begin{enumerate}[label=\Alph*.]
\item \( a \in [-37, -28] \text{ and } b \in [29, 35] \)

 $-32 + 34 i$, which corresponds to adding a minus sign in both terms.
\item \( a \in [-37, -28] \text{ and } b \in [-34, -33] \)

* $-32 - 34 i$, which is the correct option.
\item \( a \in [-24, -16] \text{ and } b \in [12, 18] \)

 $-20 + 12 i$, which corresponds to just multiplying the real terms to get the real part of the solution and the coefficients in the complex terms to get the complex part.
\item \( a \in [-9, -5] \text{ and } b \in [-46, -42] \)

 $-8 - 46 i$, which corresponds to adding a minus sign in the second term.
\item \( a \in [-9, -5] \text{ and } b \in [40, 48] \)

 $-8 + 46 i$, which corresponds to adding a minus sign in the first term.
\end{enumerate}

\textbf{General Comment:} You can treat $i$ as a variable and distribute. Just remember that $i^2=-1$, so you can continue to reduce after you distribute.
}
\end{enumerate}

\end{document}