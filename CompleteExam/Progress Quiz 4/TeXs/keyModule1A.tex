\documentclass{extbook}[14pt]
\usepackage{multicol, enumerate, enumitem, hyperref, color, soul, setspace, parskip, fancyhdr, amssymb, amsthm, amsmath, bbm, latexsym, units, mathtools}
\everymath{\displaystyle}
\usepackage[headsep=0.5cm,headheight=0cm, left=1 in,right= 1 in,top= 1 in,bottom= 1 in]{geometry}
\usepackage{dashrule}  % Package to use the command below to create lines between items
\newcommand{\litem}[1]{\item #1

\rule{\textwidth}{0.4pt}}
\pagestyle{fancy}
\lhead{}
\chead{Answer Key for Progress Quiz 4 Version A}
\rhead{}
\lfoot{9187-5854}
\cfoot{}
\rfoot{Spring 2021}
\begin{document}
\textbf{This key should allow you to understand why you choose the option you did (beyond just getting a question right or wrong). \href{https://xronos.clas.ufl.edu/mac1105spring2020/courseDescriptionAndMisc/Exams/LearningFromResults}{More instructions on how to use this key can be found here}.}

\textbf{If you have a suggestion to make the keys better, \href{https://forms.gle/CZkbZmPbC9XALEE88}{please fill out the short survey here}.}

\textit{Note: This key is auto-generated and may contain issues and/or errors. The keys are reviewed after each exam to ensure grading is done accurately. If there are issues (like duplicate options), they are noted in the offline gradebook. The keys are a work-in-progress to give students as many resources to improve as possible.}

\rule{\textwidth}{0.4pt}

\begin{enumerate}\litem{
Choose the \textbf{smallest} set of Complex numbers that the number below belongs to.
\[ \sqrt{\frac{-315}{5}}+\sqrt{0}i \]The solution is \( \text{Pure Imaginary} \), which is option E.\begin{enumerate}[label=\Alph*.]
\item \( \text{Nonreal Complex} \)

This is a Complex number $(a+bi)$ that is not Real (has $i$ as part of the number).
\item \( \text{Irrational} \)

These cannot be written as a fraction of Integers. Remember: $\pi$ is not an Integer!
\item \( \text{Not a Complex Number} \)

This is not a number. The only non-Complex number we know is dividing by 0 as this is not a number!
\item \( \text{Rational} \)

These are numbers that can be written as fraction of Integers (e.g., -2/3 + 5)
\item \( \text{Pure Imaginary} \)

* This is the correct option!
\end{enumerate}

\textbf{General Comment:} Be sure to simplify $i^2 = -1$. This may remove the imaginary portion for your number. If you are having trouble, you may want to look at the \textit{Subgroups of the Real Numbers} section.
}
\litem{
Choose the \textbf{smallest} set of Complex numbers that the number below belongs to.
\[ \sqrt{\frac{625}{441}} + 36i^2 \]The solution is \( \text{Rational} \), which is option A.\begin{enumerate}[label=\Alph*.]
\item \( \text{Rational} \)

* This is the correct option!
\item \( \text{Nonreal Complex} \)

This is a Complex number $(a+bi)$ that is not Real (has $i$ as part of the number).
\item \( \text{Pure Imaginary} \)

This is a Complex number $(a+bi)$ that \textbf{only} has an imaginary part like $2i$.
\item \( \text{Not a Complex Number} \)

This is not a number. The only non-Complex number we know is dividing by 0 as this is not a number!
\item \( \text{Irrational} \)

These cannot be written as a fraction of Integers. Remember: $\pi$ is not an Integer!
\end{enumerate}

\textbf{General Comment:} Be sure to simplify $i^2 = -1$. This may remove the imaginary portion for your number. If you are having trouble, you may want to look at the \textit{Subgroups of the Real Numbers} section.
}
\litem{
Simplify the expression below and choose the interval the simplification is contained within.
\[ 19 - 7^2 + 9 \div 12 * 11 \div 18 \]The solution is \( -29.542 \), which is option D.\begin{enumerate}[label=\Alph*.]
\item \( [-30.09, -29.92] \)

 -29.996, which corresponds to an Order of Operations error: not reading left-to-right for multiplication/division.
\item \( [67.82, 68.26] \)

 68.004, which corresponds to two Order of Operations errors.
\item \( [68.41, 68.66] \)

 68.458, which corresponds to an Order of Operations error: multiplying by negative before squaring. For example: $(-3)^2 \neq -3^2$
\item \( [-29.59, -28.91] \)

* -29.542, this is the correct option
\item \( \text{None of the above} \)

 You may have gotten this by making an unanticipated error. If you got a value that is not any of the others, please let the coordinator know so they can help you figure out what happened.
\end{enumerate}

\textbf{General Comment:} While you may remember (or were taught) PEMDAS is done in order, it is actually done as P/E/MD/AS. When we are at MD or AS, we read left to right.
}
\litem{
Simplify the expression below and choose the interval the simplification is contained within.
\[ 13 - 1 \div 6 * 19 - (8 * 12) \]The solution is \( -86.167 \), which is option C.\begin{enumerate}[label=\Alph*.]
\item \( [21.8, 22.31] \)

 22.000, which corresponds to not distributing a negative correctly.
\item \( [-83.04, -81.13] \)

 -83.009, which corresponds to an Order of Operations error: not reading left-to-right for multiplication/division.
\item \( [-86.82, -85.92] \)

* -86.167, which is the correct option.
\item \( [108.47, 109.39] \)

 108.991, which corresponds to not distributing addition and subtraction correctly.
\item \( \text{None of the above} \)

 You may have gotten this by making an unanticipated error. If you got a value that is not any of the others, please let the coordinator know so they can help you figure out what happened.
\end{enumerate}

\textbf{General Comment:} While you may remember (or were taught) PEMDAS is done in order, it is actually done as P/E/MD/AS. When we are at MD or AS, we read left to right.
}
\litem{
Simplify the expression below into the form $a+bi$. Then, choose the intervals that $a$ and $b$ belong to.
\[ (-6 + 5 i)(-10 - 4 i) \]The solution is \( 80 - 26 i \), which is option A.\begin{enumerate}[label=\Alph*.]
\item \( a \in [74, 87] \text{ and } b \in [-26, -22] \)

* $80 - 26 i$, which is the correct option.
\item \( a \in [59, 65] \text{ and } b \in [-25, -13] \)

 $60 - 20 i$, which corresponds to just multiplying the real terms to get the real part of the solution and the coefficients in the complex terms to get the complex part.
\item \( a \in [74, 87] \text{ and } b \in [25, 28] \)

 $80 + 26 i$, which corresponds to adding a minus sign in both terms.
\item \( a \in [37, 45] \text{ and } b \in [-79, -72] \)

 $40 - 74 i$, which corresponds to adding a minus sign in the second term.
\item \( a \in [37, 45] \text{ and } b \in [72, 78] \)

 $40 + 74 i$, which corresponds to adding a minus sign in the first term.
\end{enumerate}

\textbf{General Comment:} You can treat $i$ as a variable and distribute. Just remember that $i^2=-1$, so you can continue to reduce after you distribute.
}
\litem{
Choose the \textbf{smallest} set of Real numbers that the number below belongs to.
\[ \sqrt{\frac{11664}{36}} \]The solution is \( \text{Whole} \), which is option B.\begin{enumerate}[label=\Alph*.]
\item \( \text{Irrational} \)

These cannot be written as a fraction of Integers.
\item \( \text{Whole} \)

* This is the correct option!
\item \( \text{Not a Real number} \)

These are Nonreal Complex numbers \textbf{OR} things that are not numbers (e.g., dividing by 0).
\item \( \text{Integer} \)

These are the negative and positive counting numbers (..., -3, -2, -1, 0, 1, 2, 3, ...)
\item \( \text{Rational} \)

These are numbers that can be written as fraction of Integers (e.g., -2/3)
\end{enumerate}

\textbf{General Comment:} First, you \textbf{NEED} to simplify the expression. This question simplifies to $108$. 
 
 Be sure you look at the simplified fraction and not just the decimal expansion. Numbers such as 13, 17, and 19 provide \textbf{long but repeating/terminating decimal expansions!} 
 
 The only ways to *not* be a Real number are: dividing by 0 or taking the square root of a negative number. 
 
 Irrational numbers are more than just square root of 3: adding or subtracting values from square root of 3 is also irrational.
}
\litem{
Simplify the expression below into the form $a+bi$. Then, choose the intervals that $a$ and $b$ belong to.
\[ \frac{36 + 33 i}{-7 + 8 i} \]The solution is \( 0.11  - 4.59 i \), which is option C.\begin{enumerate}[label=\Alph*.]
\item \( a \in [-5.3, -4.85] \text{ and } b \in [3, 5.5] \)

 $-5.14  + 4.12 i$, which corresponds to just dividing the first term by the first term and the second by the second.
\item \( a \in [-0.4, 0.8] \text{ and } b \in [-519.5, -518.5] \)

 $0.11  - 519.00 i$, which corresponds to forgetting to multiply the conjugate by the numerator.
\item \( a \in [-0.4, 0.8] \text{ and } b \in [-6, -4] \)

* $0.11  - 4.59 i$, which is the correct option.
\item \( a \in [-4.85, -4.05] \text{ and } b \in [-0.5, 1.5] \)

 $-4.57  + 0.50 i$, which corresponds to forgetting to multiply the conjugate by the numerator and not computing the conjugate correctly.
\item \( a \in [11.85, 12.7] \text{ and } b \in [-6, -4] \)

 $12.00  - 4.59 i$, which corresponds to forgetting to multiply the conjugate by the numerator and using a plus instead of a minus in the denominator.
\end{enumerate}

\textbf{General Comment:} Multiply the numerator and denominator by the *conjugate* of the denominator, then simplify. For example, if we have $2+3i$, the conjugate is $2-3i$.
}
\litem{
Choose the \textbf{smallest} set of Real numbers that the number below belongs to.
\[ \sqrt{\frac{-1989}{9}} \]The solution is \( \text{Not a Real number} \), which is option B.\begin{enumerate}[label=\Alph*.]
\item \( \text{Rational} \)

These are numbers that can be written as fraction of Integers (e.g., -2/3)
\item \( \text{Not a Real number} \)

* This is the correct option!
\item \( \text{Integer} \)

These are the negative and positive counting numbers (..., -3, -2, -1, 0, 1, 2, 3, ...)
\item \( \text{Irrational} \)

These cannot be written as a fraction of Integers.
\item \( \text{Whole} \)

These are the counting numbers with 0 (0, 1, 2, 3, ...)
\end{enumerate}

\textbf{General Comment:} First, you \textbf{NEED} to simplify the expression. This question simplifies to $\sqrt{221} i$. 
 
 Be sure you look at the simplified fraction and not just the decimal expansion. Numbers such as 13, 17, and 19 provide \textbf{long but repeating/terminating decimal expansions!} 
 
 The only ways to *not* be a Real number are: dividing by 0 or taking the square root of a negative number. 
 
 Irrational numbers are more than just square root of 3: adding or subtracting values from square root of 3 is also irrational.
}
\litem{
Simplify the expression below into the form $a+bi$. Then, choose the intervals that $a$ and $b$ belong to.
\[ \frac{54 - 44 i}{-2 - 8 i} \]The solution is \( 3.59  + 7.65 i \), which is option D.\begin{enumerate}[label=\Alph*.]
\item \( a \in [243.5, 244.5] \text{ and } b \in [6, 9] \)

 $244.00  + 7.65 i$, which corresponds to forgetting to multiply the conjugate by the numerator and using a plus instead of a minus in the denominator.
\item \( a \in [-28, -26.5] \text{ and } b \in [5, 7] \)

 $-27.00  + 5.50 i$, which corresponds to just dividing the first term by the first term and the second by the second.
\item \( a \in [2.5, 4.5] \text{ and } b \in [519.5, 520.5] \)

 $3.59  + 520.00 i$, which corresponds to forgetting to multiply the conjugate by the numerator.
\item \( a \in [2.5, 4.5] \text{ and } b \in [6, 9] \)

* $3.59  + 7.65 i$, which is the correct option.
\item \( a \in [-7, -6.5] \text{ and } b \in [-5.5, -4.5] \)

 $-6.76  - 5.06 i$, which corresponds to forgetting to multiply the conjugate by the numerator and not computing the conjugate correctly.
\end{enumerate}

\textbf{General Comment:} Multiply the numerator and denominator by the *conjugate* of the denominator, then simplify. For example, if we have $2+3i$, the conjugate is $2-3i$.
}
\litem{
Simplify the expression below into the form $a+bi$. Then, choose the intervals that $a$ and $b$ belong to.
\[ (5 - 3 i)(-10 + 6 i) \]The solution is \( -32 + 60 i \), which is option E.\begin{enumerate}[label=\Alph*.]
\item \( a \in [-73, -67] \text{ and } b \in [-2, 1] \)

 $-68 + 0 i$, which corresponds to adding a minus sign in the second term.
\item \( a \in [-55, -49] \text{ and } b \in [-21, -14] \)

 $-50 - 18 i$, which corresponds to just multiplying the real terms to get the real part of the solution and the coefficients in the complex terms to get the complex part.
\item \( a \in [-73, -67] \text{ and } b \in [-2, 1] \)

 $-68 + 0 i$, which corresponds to adding a minus sign in the first term.
\item \( a \in [-35, -29] \text{ and } b \in [-62, -59] \)

 $-32 - 60 i$, which corresponds to adding a minus sign in both terms.
\item \( a \in [-35, -29] \text{ and } b \in [58, 63] \)

* $-32 + 60 i$, which is the correct option.
\end{enumerate}

\textbf{General Comment:} You can treat $i$ as a variable and distribute. Just remember that $i^2=-1$, so you can continue to reduce after you distribute.
}
\end{enumerate}

\end{document}