\documentclass{extbook}[14pt]
\usepackage{multicol, enumerate, enumitem, hyperref, color, soul, setspace, parskip, fancyhdr, amssymb, amsthm, amsmath, bbm, latexsym, units, mathtools}
\everymath{\displaystyle}
\usepackage[headsep=0.5cm,headheight=0cm, left=1 in,right= 1 in,top= 1 in,bottom= 1 in]{geometry}
\usepackage{dashrule}  % Package to use the command below to create lines between items
\newcommand{\litem}[1]{\item #1

\rule{\textwidth}{0.4pt}}
\pagestyle{fancy}
\lhead{}
\chead{Answer Key for Progress Quiz 4 Version A}
\rhead{}
\lfoot{4378-7085}
\cfoot{}
\rfoot{Fall 2020}
\begin{document}
\textbf{This key should allow you to understand why you choose the option you did (beyond just getting a question right or wrong). \href{https://xronos.clas.ufl.edu/mac1105spring2020/courseDescriptionAndMisc/Exams/LearningFromResults}{More instructions on how to use this key can be found here}.}

\textbf{If you have a suggestion to make the keys better, \href{https://forms.gle/CZkbZmPbC9XALEE88}{please fill out the short survey here}.}

\textit{Note: This key is auto-generated and may contain issues and/or errors. The keys are reviewed after each exam to ensure grading is done accurately. If there are issues (like duplicate options), they are noted in the offline gradebook. The keys are a work-in-progress to give students as many resources to improve as possible.}

\rule{\textwidth}{0.4pt}

\begin{enumerate}\litem{
Simplify the expression below into the form $a+bi$. Then, choose the intervals that $a$ and $b$ belong to.
\[ \frac{36 + 66 i}{-2 - i} \]
The solution is \( -27.60  - 19.20 i \), which is option B.\begin{enumerate}[label=\Alph*.]
\item \( a \in [-140, -137.5] \text{ and } b \in [-21, -19] \)

 $-138.00  - 19.20 i$, which corresponds to forgetting to multiply the conjugate by the numerator and using a plus instead of a minus in the denominator.
\item \( a \in [-28, -27] \text{ and } b \in [-21, -19] \)

* $-27.60  - 19.20 i$, which is the correct option.
\item \( a \in [-28, -27] \text{ and } b \in [-97, -94.5] \)

 $-27.60  - 96.00 i$, which corresponds to forgetting to multiply the conjugate by the numerator.
\item \( a \in [-1.5, -0.5] \text{ and } b \in [-35, -33.5] \)

 $-1.20  - 33.60 i$, which corresponds to forgetting to multiply the conjugate by the numerator and not computing the conjugate correctly.
\item \( a \in [-18.5, -17.5] \text{ and } b \in [-67.5, -65] \)

 $-18.00  - 66.00 i$, which corresponds to just dividing the first term by the first term and the second by the second.
\end{enumerate}

\textbf{General Comment:} Multiply the numerator and denominator by the *conjugate* of the denominator, then simplify. For example, if we have $2+3i$, the conjugate is $2-3i$.
}
\litem{
Simplify the expression below into the form $a+bi$. Then, choose the intervals that $a$ and $b$ belong to.
\[ (-10 - 2 i)(-5 - 8 i) \]
The solution is \( 34 + 90 i \), which is option C.\begin{enumerate}[label=\Alph*.]
\item \( a \in [49, 52] \text{ and } b \in [15, 18] \)

 $50 + 16 i$, which corresponds to just multiplying the real terms to get the real part of the solution and the coefficients in the complex terms to get the complex part.
\item \( a \in [31, 38] \text{ and } b \in [-98, -88] \)

 $34 - 90 i$, which corresponds to adding a minus sign in both terms.
\item \( a \in [31, 38] \text{ and } b \in [89, 94] \)

* $34 + 90 i$, which is the correct option.
\item \( a \in [63, 71] \text{ and } b \in [69, 74] \)

 $66 + 70 i$, which corresponds to adding a minus sign in the first term.
\item \( a \in [63, 71] \text{ and } b \in [-71, -68] \)

 $66 - 70 i$, which corresponds to adding a minus sign in the second term.
\end{enumerate}

\textbf{General Comment:} You can treat $i$ as a variable and distribute. Just remember that $i^2=-1$, so you can continue to reduce after you distribute.
}
\litem{
Simplify the expression below and choose the interval the simplification is contained within.
\[ 4 - 5 \div 8 * 11 - (16 * 12) \]
The solution is \( -194.875 \), which is option A.\begin{enumerate}[label=\Alph*.]
\item \( [-197.88, -188.88] \)

* -194.875, which is the correct option.
\item \( [191.94, 199.94] \)

 195.943, which corresponds to not distributing addition and subtraction correctly.
\item \( [-191.06, -183.06] \)

 -188.057, which corresponds to an Order of Operations error: not reading left-to-right for multiplication/division.
\item \( [-228.5, -222.5] \)

 -226.500, which corresponds to not distributing a negative correctly.
\item \( \text{None of the above} \)

 You may have gotten this by making an unanticipated error. If you got a value that is not any of the others, please let the coordinator know so they can help you figure out what happened.
\end{enumerate}

\textbf{General Comment:} While you may remember (or were taught) PEMDAS is done in order, it is actually done as P/E/MD/AS. When we are at MD or AS, we read left to right.
}
\litem{
Choose the \textbf{smallest} set of Complex numbers that the number below belongs to.
\[ \frac{-20}{-9}+\sqrt{-16}i \]
The solution is \( \text{Rational} \), which is option C.\begin{enumerate}[label=\Alph*.]
\item \( \text{Irrational} \)

These cannot be written as a fraction of Integers. Remember: $\pi$ is not an Integer!
\item \( \text{Pure Imaginary} \)

This is a Complex number $(a+bi)$ that \textbf{only} has an imaginary part like $2i$.
\item \( \text{Rational} \)

* This is the correct option!
\item \( \text{Nonreal Complex} \)

This is a Complex number $(a+bi)$ that is not Real (has $i$ as part of the number).
\item \( \text{Not a Complex Number} \)

This is not a number. The only non-Complex number we know is dividing by 0 as this is not a number!
\end{enumerate}

\textbf{General Comment:} Be sure to simplify $i^2 = -1$. This may remove the imaginary portion for your number. If you are having trouble, you may want to look at the \textit{Subgroups of the Real Numbers} section.
}
\litem{
Simplify the expression below into the form $a+bi$. Then, choose the intervals that $a$ and $b$ belong to.
\[ \frac{54 + 11 i}{-2 - 3 i} \]
The solution is \( -10.85  + 10.77 i \), which is option B.\begin{enumerate}[label=\Alph*.]
\item \( a \in [-6.5, -5] \text{ and } b \in [-14.5, -12] \)

 $-5.77  - 14.15 i$, which corresponds to forgetting to multiply the conjugate by the numerator and not computing the conjugate correctly.
\item \( a \in [-11, -10.5] \text{ and } b \in [10, 12.5] \)

* $-10.85  + 10.77 i$, which is the correct option.
\item \( a \in [-142, -140] \text{ and } b \in [10, 12.5] \)

 $-141.00  + 10.77 i$, which corresponds to forgetting to multiply the conjugate by the numerator and using a plus instead of a minus in the denominator.
\item \( a \in [-11, -10.5] \text{ and } b \in [139.5, 140.5] \)

 $-10.85  + 140.00 i$, which corresponds to forgetting to multiply the conjugate by the numerator.
\item \( a \in [-28, -25.5] \text{ and } b \in [-4.5, -2.5] \)

 $-27.00  - 3.67 i$, which corresponds to just dividing the first term by the first term and the second by the second.
\end{enumerate}

\textbf{General Comment:} Multiply the numerator and denominator by the *conjugate* of the denominator, then simplify. For example, if we have $2+3i$, the conjugate is $2-3i$.
}
\litem{
Choose the \textbf{smallest} set of Complex numbers that the number below belongs to.
\[ \sqrt{\frac{1638}{0}}+\sqrt{176} i \]
The solution is \( \text{Not a Complex Number} \), which is option C.\begin{enumerate}[label=\Alph*.]
\item \( \text{Rational} \)

These are numbers that can be written as fraction of Integers (e.g., -2/3 + 5)
\item \( \text{Irrational} \)

These cannot be written as a fraction of Integers. Remember: $\pi$ is not an Integer!
\item \( \text{Not a Complex Number} \)

* This is the correct option!
\item \( \text{Pure Imaginary} \)

This is a Complex number $(a+bi)$ that \textbf{only} has an imaginary part like $2i$.
\item \( \text{Nonreal Complex} \)

This is a Complex number $(a+bi)$ that is not Real (has $i$ as part of the number).
\end{enumerate}

\textbf{General Comment:} Be sure to simplify $i^2 = -1$. This may remove the imaginary portion for your number. If you are having trouble, you may want to look at the \textit{Subgroups of the Real Numbers} section.
}
\litem{
Simplify the expression below into the form $a+bi$. Then, choose the intervals that $a$ and $b$ belong to.
\[ (10 - 5 i)(-4 + 3 i) \]
The solution is \( -25 + 50 i \), which is option B.\begin{enumerate}[label=\Alph*.]
\item \( a \in [-28, -22] \text{ and } b \in [-52, -49] \)

 $-25 - 50 i$, which corresponds to adding a minus sign in both terms.
\item \( a \in [-28, -22] \text{ and } b \in [49, 54] \)

* $-25 + 50 i$, which is the correct option.
\item \( a \in [-55, -54] \text{ and } b \in [-12, -9] \)

 $-55 - 10 i$, which corresponds to adding a minus sign in the second term.
\item \( a \in [-55, -54] \text{ and } b \in [9, 11] \)

 $-55 + 10 i$, which corresponds to adding a minus sign in the first term.
\item \( a \in [-44, -36] \text{ and } b \in [-15, -13] \)

 $-40 - 15 i$, which corresponds to just multiplying the real terms to get the real part of the solution and the coefficients in the complex terms to get the complex part.
\end{enumerate}

\textbf{General Comment:} You can treat $i$ as a variable and distribute. Just remember that $i^2=-1$, so you can continue to reduce after you distribute.
}
\litem{
Choose the \textbf{smallest} set of Real numbers that the number below belongs to.
\[ \sqrt{\frac{7}{0}} \]
The solution is \( \text{Not a Real number} \), which is option E.\begin{enumerate}[label=\Alph*.]
\item \( \text{Integer} \)

These are the negative and positive counting numbers (..., -3, -2, -1, 0, 1, 2, 3, ...)
\item \( \text{Whole} \)

These are the counting numbers with 0 (0, 1, 2, 3, ...)
\item \( \text{Irrational} \)

These cannot be written as a fraction of Integers.
\item \( \text{Rational} \)

These are numbers that can be written as fraction of Integers (e.g., -2/3)
\item \( \text{Not a Real number} \)

* This is the correct option!
\end{enumerate}

\textbf{General Comment:} First, you \textbf{NEED} to simplify the expression. This question simplifies to $\sqrt{\frac{7}{0}}$. 
 
 Be sure you look at the simplified fraction and not just the decimal expansion. Numbers such as 13, 17, and 19 provide \textbf{long but repeating/terminating decimal expansions!} 
 
 The only ways to *not* be a Real number are: dividing by 0 or taking the square root of a negative number. 
 
 Irrational numbers are more than just square root of 3: adding or subtracting values from square root of 3 is also irrational.
}
\litem{
Choose the \textbf{smallest} set of Real numbers that the number below belongs to.
\[ \sqrt{\frac{144400}{400}} \]
The solution is \( \text{Whole} \), which is option E.\begin{enumerate}[label=\Alph*.]
\item \( \text{Integer} \)

These are the negative and positive counting numbers (..., -3, -2, -1, 0, 1, 2, 3, ...)
\item \( \text{Irrational} \)

These cannot be written as a fraction of Integers.
\item \( \text{Rational} \)

These are numbers that can be written as fraction of Integers (e.g., -2/3)
\item \( \text{Not a Real number} \)

These are Nonreal Complex numbers \textbf{OR} things that are not numbers (e.g., dividing by 0).
\item \( \text{Whole} \)

* This is the correct option!
\end{enumerate}

\textbf{General Comment:} First, you \textbf{NEED} to simplify the expression. This question simplifies to $380$. 
 
 Be sure you look at the simplified fraction and not just the decimal expansion. Numbers such as 13, 17, and 19 provide \textbf{long but repeating/terminating decimal expansions!} 
 
 The only ways to *not* be a Real number are: dividing by 0 or taking the square root of a negative number. 
 
 Irrational numbers are more than just square root of 3: adding or subtracting values from square root of 3 is also irrational.
}
\litem{
Simplify the expression below and choose the interval the simplification is contained within.
\[ 16 - 18 \div 10 * 3 - (5 * 19) \]
The solution is \( -84.400 \), which is option B.\begin{enumerate}[label=\Alph*.]
\item \( [104.4, 109.4] \)

 106.400, which corresponds to not distributing a negative correctly.
\item \( [-84.4, -81.4] \)

* -84.400, which is the correct option.
\item \( [-81.6, -77.6] \)

 -79.600, which corresponds to an Order of Operations error: not reading left-to-right for multiplication/division.
\item \( [108.4, 112.4] \)

 110.400, which corresponds to not distributing addition and subtraction correctly.
\item \( \text{None of the above} \)

 You may have gotten this by making an unanticipated error. If you got a value that is not any of the others, please let the coordinator know so they can help you figure out what happened.
\end{enumerate}

\textbf{General Comment:} While you may remember (or were taught) PEMDAS is done in order, it is actually done as P/E/MD/AS. When we are at MD or AS, we read left to right.
}
\end{enumerate}

\end{document}