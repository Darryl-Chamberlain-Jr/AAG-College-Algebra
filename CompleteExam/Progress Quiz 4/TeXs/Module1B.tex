\documentclass[14pt]{extbook}
\usepackage{multicol, enumerate, enumitem, hyperref, color, soul, setspace, parskip, fancyhdr} %General Packages
\usepackage{amssymb, amsthm, amsmath, bbm, latexsym, units, mathtools} %Math Packages
\everymath{\displaystyle} %All math in Display Style
% Packages with additional options
\usepackage[headsep=0.5cm,headheight=12pt, left=1 in,right= 1 in,top= 1 in,bottom= 1 in]{geometry}
\usepackage[usenames,dvipsnames]{xcolor}
\usepackage{dashrule}  % Package to use the command below to create lines between items
\newcommand{\litem}[1]{\item#1\hspace*{-1cm}\rule{\textwidth}{0.4pt}}
\pagestyle{fancy}
\lhead{Progress Quiz 4}
\chead{}
\rhead{Version B}
\lfoot{4378-7085}
\cfoot{}
\rfoot{Fall 2020}
\begin{document}

\begin{enumerate}
\litem{
Simplify the expression below and choose the interval the simplification is contained within.\[ 17 - 8^2 + 3 \div 5 * 12 \div 15 \]\begin{enumerate}[label=\Alph*.]
\item \( [-46.69, -46.05] \)
\item \( [80.42, 81.06] \)
\item \( [-47.26, -46.63] \)
\item \( [81.42, 82.48] \)
\item \( \text{None of the above} \)

\end{enumerate} }
\litem{
Simplify the expression below into the form $a+bi$. Then, choose the intervals that $a$ and $b$ belong to.\[ \frac{45 + 33 i}{-1 - 4 i} \]\begin{enumerate}[label=\Alph*.]
\item \( a \in [-46.5, -44.5] \text{ and } b \in [-9.5, -7.5] \)
\item \( a \in [-11, -10] \text{ and } b \in [8, 9.5] \)
\item \( a \in [4, 5.5] \text{ and } b \in [-13, -11] \)
\item \( a \in [-178.5, -176.5] \text{ and } b \in [8, 9.5] \)
\item \( a \in [-11, -10] \text{ and } b \in [146, 147.5] \)

\end{enumerate} }
\litem{
Simplify the expression below into the form $a+bi$. Then, choose the intervals that $a$ and $b$ belong to.\[ (-9 + 6 i)(5 - 7 i) \]\begin{enumerate}[label=\Alph*.]
\item \( a \in [-5, 4] \text{ and } b \in [-96, -87] \)
\item \( a \in [-45, -39] \text{ and } b \in [-42, -39] \)
\item \( a \in [-87, -82] \text{ and } b \in [30, 36] \)
\item \( a \in [-5, 4] \text{ and } b \in [92, 97] \)
\item \( a \in [-87, -82] \text{ and } b \in [-35, -24] \)

\end{enumerate} }
\litem{
Simplify the expression below into the form $a+bi$. Then, choose the intervals that $a$ and $b$ belong to.\[ \frac{-9 - 88 i}{4 + 2 i} \]\begin{enumerate}[label=\Alph*.]
\item \( a \in [-11, -9.5] \text{ and } b \in [-335.5, -332.5] \)
\item \( a \in [-11, -9.5] \text{ and } b \in [-17, -15] \)
\item \( a \in [-212.5, -211.5] \text{ and } b \in [-17, -15] \)
\item \( a \in [6, 9] \text{ and } b \in [-19, -18] \)
\item \( a \in [-4, -2] \text{ and } b \in [-44.5, -43] \)

\end{enumerate} }
\litem{
Choose the \textbf{smallest} set of Complex numbers that the number below belongs to.\[ \sqrt{\frac{0}{5}}+\sqrt{8}i \]\begin{enumerate}[label=\Alph*.]
\item \( \text{Pure Imaginary} \)
\item \( \text{Not a Complex Number} \)
\item \( \text{Rational} \)
\item \( \text{Nonreal Complex} \)
\item \( \text{Irrational} \)

\end{enumerate} }
\litem{
Choose the \textbf{smallest} set of Complex numbers that the number below belongs to.\[ -\sqrt{\frac{2340}{12}}+3i^2 \]\begin{enumerate}[label=\Alph*.]
\item \( \text{Not a Complex Number} \)
\item \( \text{Nonreal Complex} \)
\item \( \text{Pure Imaginary} \)
\item \( \text{Irrational} \)
\item \( \text{Rational} \)

\end{enumerate} }
\litem{
Simplify the expression below into the form $a+bi$. Then, choose the intervals that $a$ and $b$ belong to.\[ (-5 - 7 i)(4 - 6 i) \]\begin{enumerate}[label=\Alph*.]
\item \( a \in [-62, -56] \text{ and } b \in [-2.1, -0.2] \)
\item \( a \in [-62, -56] \text{ and } b \in [1.2, 4.7] \)
\item \( a \in [18, 23] \text{ and } b \in [55.5, 58.3] \)
\item \( a \in [-24, -15] \text{ and } b \in [40.1, 42.2] \)
\item \( a \in [18, 23] \text{ and } b \in [-59.6, -55.1] \)

\end{enumerate} }
\litem{
Choose the \textbf{smallest} set of Real numbers that the number below belongs to.\[ \sqrt{\frac{-693}{7}} \]\begin{enumerate}[label=\Alph*.]
\item \( \text{Not a Real number} \)
\item \( \text{Integer} \)
\item \( \text{Whole} \)
\item \( \text{Irrational} \)
\item \( \text{Rational} \)

\end{enumerate} }
\litem{
Choose the \textbf{smallest} set of Real numbers that the number below belongs to.\[ -\sqrt{\frac{144}{169}} \]\begin{enumerate}[label=\Alph*.]
\item \( \text{Not a Real number} \)
\item \( \text{Whole} \)
\item \( \text{Irrational} \)
\item \( \text{Rational} \)
\item \( \text{Integer} \)

\end{enumerate} }
\litem{
Simplify the expression below and choose the interval the simplification is contained within.\[ 3 - 6^2 + 12 \div 4 * 10 \div 11 \]\begin{enumerate}[label=\Alph*.]
\item \( [40.6, 42.5] \)
\item \( [-32.9, -27.4] \)
\item \( [-33.5, -31.3] \)
\item \( [38.3, 39.1] \)
\item \( \text{None of the above} \)

\end{enumerate} }
\end{enumerate}

\end{document}