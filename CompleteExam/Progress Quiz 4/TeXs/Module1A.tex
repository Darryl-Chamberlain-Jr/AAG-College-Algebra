\documentclass[14pt]{extbook}
\usepackage{multicol, enumerate, enumitem, hyperref, color, soul, setspace, parskip, fancyhdr} %General Packages
\usepackage{amssymb, amsthm, amsmath, bbm, latexsym, units, mathtools} %Math Packages
\everymath{\displaystyle} %All math in Display Style
% Packages with additional options
\usepackage[headsep=0.5cm,headheight=12pt, left=1 in,right= 1 in,top= 1 in,bottom= 1 in]{geometry}
\usepackage[usenames,dvipsnames]{xcolor}
\usepackage{dashrule}  % Package to use the command below to create lines between items
\newcommand{\litem}[1]{\item#1\hspace*{-1cm}\rule{\textwidth}{0.4pt}}
\pagestyle{fancy}
\lhead{Progress Quiz 4}
\chead{}
\rhead{Version A}
\lfoot{4378-7085}
\cfoot{}
\rfoot{Fall 2020}
\begin{document}

\begin{enumerate}
\litem{
Simplify the expression below into the form $a+bi$. Then, choose the intervals that $a$ and $b$ belong to.\[ \frac{36 + 66 i}{-2 - i} \]\begin{enumerate}[label=\Alph*.]
\item \( a \in [-140, -137.5] \text{ and } b \in [-21, -19] \)
\item \( a \in [-28, -27] \text{ and } b \in [-21, -19] \)
\item \( a \in [-28, -27] \text{ and } b \in [-97, -94.5] \)
\item \( a \in [-1.5, -0.5] \text{ and } b \in [-35, -33.5] \)
\item \( a \in [-18.5, -17.5] \text{ and } b \in [-67.5, -65] \)

\end{enumerate} }
\litem{
Simplify the expression below into the form $a+bi$. Then, choose the intervals that $a$ and $b$ belong to.\[ (-10 - 2 i)(-5 - 8 i) \]\begin{enumerate}[label=\Alph*.]
\item \( a \in [49, 52] \text{ and } b \in [15, 18] \)
\item \( a \in [31, 38] \text{ and } b \in [-98, -88] \)
\item \( a \in [31, 38] \text{ and } b \in [89, 94] \)
\item \( a \in [63, 71] \text{ and } b \in [69, 74] \)
\item \( a \in [63, 71] \text{ and } b \in [-71, -68] \)

\end{enumerate} }
\litem{
Simplify the expression below and choose the interval the simplification is contained within.\[ 4 - 5 \div 8 * 11 - (16 * 12) \]\begin{enumerate}[label=\Alph*.]
\item \( [-197.88, -188.88] \)
\item \( [191.94, 199.94] \)
\item \( [-191.06, -183.06] \)
\item \( [-228.5, -222.5] \)
\item \( \text{None of the above} \)

\end{enumerate} }
\litem{
Choose the \textbf{smallest} set of Complex numbers that the number below belongs to.\[ \frac{-20}{-9}+\sqrt{-16}i \]\begin{enumerate}[label=\Alph*.]
\item \( \text{Irrational} \)
\item \( \text{Pure Imaginary} \)
\item \( \text{Rational} \)
\item \( \text{Nonreal Complex} \)
\item \( \text{Not a Complex Number} \)

\end{enumerate} }
\litem{
Simplify the expression below into the form $a+bi$. Then, choose the intervals that $a$ and $b$ belong to.\[ \frac{54 + 11 i}{-2 - 3 i} \]\begin{enumerate}[label=\Alph*.]
\item \( a \in [-6.5, -5] \text{ and } b \in [-14.5, -12] \)
\item \( a \in [-11, -10.5] \text{ and } b \in [10, 12.5] \)
\item \( a \in [-142, -140] \text{ and } b \in [10, 12.5] \)
\item \( a \in [-11, -10.5] \text{ and } b \in [139.5, 140.5] \)
\item \( a \in [-28, -25.5] \text{ and } b \in [-4.5, -2.5] \)

\end{enumerate} }
\litem{
Choose the \textbf{smallest} set of Complex numbers that the number below belongs to.\[ \sqrt{\frac{1638}{0}}+\sqrt{176} i \]\begin{enumerate}[label=\Alph*.]
\item \( \text{Rational} \)
\item \( \text{Irrational} \)
\item \( \text{Not a Complex Number} \)
\item \( \text{Pure Imaginary} \)
\item \( \text{Nonreal Complex} \)

\end{enumerate} }
\litem{
Simplify the expression below into the form $a+bi$. Then, choose the intervals that $a$ and $b$ belong to.\[ (10 - 5 i)(-4 + 3 i) \]\begin{enumerate}[label=\Alph*.]
\item \( a \in [-28, -22] \text{ and } b \in [-52, -49] \)
\item \( a \in [-28, -22] \text{ and } b \in [49, 54] \)
\item \( a \in [-55, -54] \text{ and } b \in [-12, -9] \)
\item \( a \in [-55, -54] \text{ and } b \in [9, 11] \)
\item \( a \in [-44, -36] \text{ and } b \in [-15, -13] \)

\end{enumerate} }
\litem{
Choose the \textbf{smallest} set of Real numbers that the number below belongs to.\[ \sqrt{\frac{7}{0}} \]\begin{enumerate}[label=\Alph*.]
\item \( \text{Integer} \)
\item \( \text{Whole} \)
\item \( \text{Irrational} \)
\item \( \text{Rational} \)
\item \( \text{Not a Real number} \)

\end{enumerate} }
\litem{
Choose the \textbf{smallest} set of Real numbers that the number below belongs to.\[ \sqrt{\frac{144400}{400}} \]\begin{enumerate}[label=\Alph*.]
\item \( \text{Integer} \)
\item \( \text{Irrational} \)
\item \( \text{Rational} \)
\item \( \text{Not a Real number} \)
\item \( \text{Whole} \)

\end{enumerate} }
\litem{
Simplify the expression below and choose the interval the simplification is contained within.\[ 16 - 18 \div 10 * 3 - (5 * 19) \]\begin{enumerate}[label=\Alph*.]
\item \( [104.4, 109.4] \)
\item \( [-84.4, -81.4] \)
\item \( [-81.6, -77.6] \)
\item \( [108.4, 112.4] \)
\item \( \text{None of the above} \)

\end{enumerate} }
\end{enumerate}

\end{document}