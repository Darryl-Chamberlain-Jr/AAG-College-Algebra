\documentclass{extbook}[14pt]
\usepackage{multicol, enumerate, enumitem, hyperref, color, soul, setspace, parskip, fancyhdr, amssymb, amsthm, amsmath, bbm, latexsym, units, mathtools}
\everymath{\displaystyle}
\usepackage[headsep=0.5cm,headheight=0cm, left=1 in,right= 1 in,top= 1 in,bottom= 1 in]{geometry}
\usepackage{dashrule}  % Package to use the command below to create lines between items
\newcommand{\litem}[1]{\item #1

\rule{\textwidth}{0.4pt}}
\pagestyle{fancy}
\lhead{}
\chead{Answer Key for Progress Quiz 4 Version B}
\rhead{}
\lfoot{9187-5854}
\cfoot{}
\rfoot{Spring 2021}
\begin{document}
\textbf{This key should allow you to understand why you choose the option you did (beyond just getting a question right or wrong). \href{https://xronos.clas.ufl.edu/mac1105spring2020/courseDescriptionAndMisc/Exams/LearningFromResults}{More instructions on how to use this key can be found here}.}

\textbf{If you have a suggestion to make the keys better, \href{https://forms.gle/CZkbZmPbC9XALEE88}{please fill out the short survey here}.}

\textit{Note: This key is auto-generated and may contain issues and/or errors. The keys are reviewed after each exam to ensure grading is done accurately. If there are issues (like duplicate options), they are noted in the offline gradebook. The keys are a work-in-progress to give students as many resources to improve as possible.}

\rule{\textwidth}{0.4pt}

\begin{enumerate}\litem{
Which of the following intervals describes the Range of the function below?
\[ f(x) = -\log_2{(x+4)}+3 \]The solution is \( (\infty, \infty) \), which is option E.\begin{enumerate}[label=\Alph*.]
\item \( (-\infty, a), a \in [2.59, 3.08] \)

$(-\infty, 3)$, which corresponds to using the vertical shift while the Range is $(-\infty, \infty)$.
\item \( (-\infty, a), a \in [-3.26, -2.78] \)

$(-\infty, -3)$, which corresponds to using the using the negative of vertical shift on $(0, \infty)$.
\item \( [a, \infty), a \in [-4.3, -3.91] \)

$[3, \infty)$, which corresponds to using the flipped Domain AND including the endpoint.
\item \( [a, \infty), a \in [3.58, 4.29] \)

$[4, \infty)$, which corresponds to using the negative of the horizontal shift AND including the endpoint.
\item \( (-\infty, \infty) \)

*This is the correct option.
\end{enumerate}

\textbf{General Comment:} \textbf{General Comments}: The domain of a basic logarithmic function is $(0, \infty)$ and the Range is $(-\infty, \infty)$. We can use shifts when finding the Domain, but the Range will always be all Real numbers.
}
\litem{
Solve the equation for $x$ and choose the interval that contains the solution (if it exists).
\[ 3^{-4x+2} = \left(\frac{1}{25}\right)^{2x+5} \]The solution is \( x = -8.952 \), which is option D.\begin{enumerate}[label=\Alph*.]
\item \( x \in [0.8, 1.7] \)

$x = 1.468$, which corresponds to distributing the $\ln(base)$ to the first term of the exponent only.
\item \( x \in [-0.8, 0.7] \)

$x = -0.500$, which corresponds to solving the numerators as equal while ignoring the bases are different.
\item \( x \in [2.4, 4.2] \)

$x = 3.049$, which corresponds to distributing the $\ln(base)$ to the second term of the exponent only.
\item \( x \in [-9.4, -8.5] \)

* $x = -8.952$, which is the correct option.
\item \( \text{There is no Real solution to the equation.} \)

This corresponds to believing there is no solution since the bases are not powers of each other.
\end{enumerate}

\textbf{General Comment:} \textbf{General Comments:} This question was written so that the bases could not be written the same. You will need to take the log of both sides.
}
\litem{
Which of the following intervals describes the Range of the function below?
\[ f(x) = e^{x+4}+9 \]The solution is \( (9, \infty) \), which is option D.\begin{enumerate}[label=\Alph*.]
\item \( [a, \infty), a \in [3, 17] \)

$[9, \infty)$, which corresponds to including the endpoint.
\item \( (-\infty, a], a \in [-10, -8] \)

$(-\infty, -9]$, which corresponds to using the negative vertical shift AND flipping the Range interval AND including the endpoint.
\item \( (-\infty, a), a \in [-10, -8] \)

$(-\infty, -9)$, which corresponds to using the negative vertical shift AND flipping the Range interval.
\item \( (a, \infty), a \in [3, 17] \)

* $(9, \infty)$, which is the correct option.
\item \( (-\infty, \infty) \)

This corresponds to confusing range of an exponential function with the domain of an exponential function.
\end{enumerate}

\textbf{General Comment:} \textbf{General Comments}: Domain of a basic exponential function is $(-\infty, \infty)$ while the Range is $(0, \infty)$. We can shift these intervals [and even flip when $a<0$!] to find the new Domain/Range.
}
\litem{
Solve the equation for $x$ and choose the interval that contains the solution (if it exists).
\[ \log_{3}{(-2x+5)}+4 = 3 \]The solution is \( x = 2.333 \), which is option D.\begin{enumerate}[label=\Alph*.]
\item \( x \in [-2.2, -1.3] \)

$x = -2.000$, which corresponds to reversing the base and exponent when converting and reversing the value with $x$.
\item \( x \in [-12.9, -10.2] \)

$x = -11.000$, which corresponds to ignoring the vertical shift when converting to exponential form.
\item \( x \in [2.9, 3.2] \)

$x = 3.000$, which corresponds to reversing the base and exponent when converting.
\item \( x \in [2.2, 2.9] \)

* $x = 2.333$, which is the correct option.
\item \( \text{There is no Real solution to the equation.} \)

Corresponds to believing a negative coefficient within the log equation means there is no Real solution.
\end{enumerate}

\textbf{General Comment:} \textbf{General Comments:} First, get the equation in the form $\log_b{(cx+d)} = a$. Then, convert to $b^a = cx+d$ and solve.
}
\litem{
Which of the following intervals describes the Domain of the function below?
\[ f(x) = e^{x-4}+3 \]The solution is \( (-\infty, \infty) \), which is option E.\begin{enumerate}[label=\Alph*.]
\item \( [a, \infty), a \in [-9, 0] \)

$[-3, \infty)$, which corresponds to using the negative vertical shift AND flipping the Range interval AND including the endpoint.
\item \( (-\infty, a], a \in [1, 9] \)

$(-\infty, 3]$, which corresponds to using the correct vertical shift *if we wanted the Range* AND including the endpoint.
\item \( (-\infty, a), a \in [1, 9] \)

$(-\infty, 3)$, which corresponds to using the correct vertical shift *if we wanted the Range*.
\item \( (a, \infty), a \in [-9, 0] \)

$(-3, \infty)$, which corresponds to using the negative vertical shift AND flipping the Range interval.
\item \( (-\infty, \infty) \)

* This is the correct option.
\end{enumerate}

\textbf{General Comment:} \textbf{General Comments}: Domain of a basic exponential function is $(-\infty, \infty)$ while the Range is $(0, \infty)$. We can shift these intervals [and even flip when $a<0$!] to find the new Domain/Range.
}
\litem{
 Solve the equation for $x$ and choose the interval that contains $x$ (if it exists).
\[  17 = \ln{\sqrt[4]{\frac{18}{e^{6x}}}} \]The solution is \( x = -10.852, \text{ which does not fit in any of the interval options.} \), which is option E.\begin{enumerate}[label=\Alph*.]
\item \( x \in [-2.4, -0.4] \)

$x = -2.371$, which corresponds to thinking you need to take the natural log of the left side before reducing.
\item \( x \in [-5.4, -5] \)

$x = -5.185$, which corresponds to treating any root as a square root.
\item \( x \in [9.9, 11.1] \)

$x = 10.852$, which is the negative of the correct solution.
\item \( \text{There is no Real solution to the equation.} \)

This corresponds to believing you cannot solve the equation.
\item \( \text{None of the above.} \)

*$x = -10.852$ is the correct solution and does not fit in any of the other intervals.
\end{enumerate}

\textbf{General Comment:} \textbf{General Comments}: After using the properties of logarithmic functions to break up the right-hand side, use $\ln(e) = 1$ to reduce the question to a linear function to solve. You can put $\ln(18)$ into a calculator if you are having trouble.
}
\litem{
Solve the equation for $x$ and choose the interval that contains the solution (if it exists).
\[ \log_{4}{(3x+6)}+4 = 2 \]The solution is \( x = -1.979 \), which is option C.\begin{enumerate}[label=\Alph*.]
\item \( x \in [7.33, 10.33] \)

$x = 7.333$, which corresponds to reversing the base and exponent when converting and reversing the value with $x$.
\item \( x \in [3.33, 4.33] \)

$x = 3.333$, which corresponds to ignoring the vertical shift when converting to exponential form.
\item \( x \in [-4.98, 1.02] \)

* $x = -1.979$, which is the correct option.
\item \( x \in [3.33, 4.33] \)

$x = 3.333$, which corresponds to reversing the base and exponent when converting.
\item \( \text{There is no Real solution to the equation.} \)

Corresponds to believing a negative coefficient within the log equation means there is no Real solution.
\end{enumerate}

\textbf{General Comment:} \textbf{General Comments:} First, get the equation in the form $\log_b{(cx+d)} = a$. Then, convert to $b^a = cx+d$ and solve.
}
\litem{
Which of the following intervals describes the Range of the function below?
\[ f(x) = -\log_2{(x+6)}-5 \]The solution is \( (\infty, \infty) \), which is option E.\begin{enumerate}[label=\Alph*.]
\item \( (-\infty, a), a \in [-5.47, -4.72] \)

$(-\infty, -5)$, which corresponds to using the vertical shift while the Range is $(-\infty, \infty)$.
\item \( (-\infty, a), a \in [4.87, 5.13] \)

$(-\infty, 5)$, which corresponds to using the using the negative of vertical shift on $(0, \infty)$.
\item \( [a, \infty), a \in [-6.12, -5.44] \)

$[-5, \infty)$, which corresponds to using the flipped Domain AND including the endpoint.
\item \( [a, \infty), a \in [5.64, 6.34] \)

$[6, \infty)$, which corresponds to using the negative of the horizontal shift AND including the endpoint.
\item \( (-\infty, \infty) \)

*This is the correct option.
\end{enumerate}

\textbf{General Comment:} \textbf{General Comments}: The domain of a basic logarithmic function is $(0, \infty)$ and the Range is $(-\infty, \infty)$. We can use shifts when finding the Domain, but the Range will always be all Real numbers.
}
\litem{
Solve the equation for $x$ and choose the interval that contains the solution (if it exists).
\[ 2^{-5x-3} = \left(\frac{1}{343}\right)^{-4x+4} \]The solution is \( x = 0.793 \), which is option C.\begin{enumerate}[label=\Alph*.]
\item \( x \in [-7.74, -6.68] \)

$x = -7.000$, which corresponds to solving the numerators as equal while ignoring the bases are different.
\item \( x \in [-1.03, 0.22] \)

$x = -0.261$, which corresponds to distributing the $\ln(base)$ to the first term of the exponent only.
\item \( x \in [-0.01, 1.2] \)

* $x = 0.793$, which is the correct option.
\item \( x \in [20.67, 21.65] \)

$x = 21.271$, which corresponds to distributing the $\ln(base)$ to the second term of the exponent only.
\item \( \text{There is no Real solution to the equation.} \)

This corresponds to believing there is no solution since the bases are not powers of each other.
\end{enumerate}

\textbf{General Comment:} \textbf{General Comments:} This question was written so that the bases could not be written the same. You will need to take the log of both sides.
}
\litem{
 Solve the equation for $x$ and choose the interval that contains $x$ (if it exists).
\[  14 = \ln{\sqrt[4]{\frac{25}{e^{9x}}}} \]The solution is \( x = -5.865, \text{ which does not fit in any of the interval options.} \), which is option E.\begin{enumerate}[label=\Alph*.]
\item \( x \in [4.4, 6.3] \)

$x = 5.865$, which is the negative of the correct solution.
\item \( x \in [-1.6, 0.1] \)

$x = -1.531$, which corresponds to thinking you need to take the natural log of the left side before reducing.
\item \( x \in [-3.2, -2.3] \)

$x = -2.753$, which corresponds to treating any root as a square root.
\item \( \text{There is no Real solution to the equation.} \)

This corresponds to believing you cannot solve the equation.
\item \( \text{None of the above.} \)

*$x = -5.865$ is the correct solution and does not fit in any of the other intervals.
\end{enumerate}

\textbf{General Comment:} \textbf{General Comments}: After using the properties of logarithmic functions to break up the right-hand side, use $\ln(e) = 1$ to reduce the question to a linear function to solve. You can put $\ln(25)$ into a calculator if you are having trouble.
}
\end{enumerate}

\end{document}