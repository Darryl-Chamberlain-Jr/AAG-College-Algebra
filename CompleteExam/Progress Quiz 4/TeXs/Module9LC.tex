\documentclass[14pt]{extbook}
\usepackage{multicol, enumerate, enumitem, hyperref, color, soul, setspace, parskip, fancyhdr} %General Packages
\usepackage{amssymb, amsthm, amsmath, bbm, latexsym, units, mathtools} %Math Packages
\everymath{\displaystyle} %All math in Display Style
% Packages with additional options
\usepackage[headsep=0.5cm,headheight=12pt, left=1 in,right= 1 in,top= 1 in,bottom= 1 in]{geometry}
\usepackage[usenames,dvipsnames]{xcolor}
\usepackage{dashrule}  % Package to use the command below to create lines between items
\newcommand{\litem}[1]{\item#1\hspace*{-1cm}\rule{\textwidth}{0.4pt}}
\pagestyle{fancy}
\lhead{Progress Quiz 4}
\chead{}
\rhead{Version C}
\lfoot{9187-5854}
\cfoot{}
\rfoot{Spring 2021}
\begin{document}

\begin{enumerate}
\litem{
Choose the interval below that $f$ composed with $g$ at $x=-1$ is in.\[ f(x) = 4x^{3} -2 x^{2} -4 x \text{ and } g(x) = 3x^{3} +3 x^{2} -x + 2 \]\begin{enumerate}[label=\Alph*.]
\item \( (f \circ g)(-1) \in [-11, -5] \)
\item \( (f \circ g)(-1) \in [75, 79] \)
\item \( (f \circ g)(-1) \in [81, 90] \)
\item \( (f \circ g)(-1) \in [-5, 0] \)
\item \( \text{It is not possible to compose the two functions.} \)

\end{enumerate} }
\litem{
Find the inverse of the function below (if it exists). Then, evaluate the inverse at $x = -15$ and choose the interval the $f^{-1}(-15)$ belongs to.\[ f(x) = \sqrt[3]{3 x + 5} \]\begin{enumerate}[label=\Alph*.]
\item \( f^{-1}(-15) \in [-1128.5, -1124.9] \)
\item \( f^{-1}(-15) \in [-1124.7, -1122.4] \)
\item \( f^{-1}(-15) \in [1122.2, 1125.6] \)
\item \( f^{-1}(-15) \in [1124.6, 1127.9] \)
\item \( \text{ The function is not invertible for all Real numbers. } \)

\end{enumerate} }
\litem{
Find the inverse of the function below (if it exists). Then, evaluate the inverse at $x = -10$ and choose the interval that $f^{-1}(-10)$ belongs to.\[ f(x) = 4 x^2 - 2 \]\begin{enumerate}[label=\Alph*.]
\item \( f^{-1}(-10) \in [1.58, 2.34] \)
\item \( f^{-1}(-10) \in [2.26, 2.49] \)
\item \( f^{-1}(-10) \in [4.29, 5.04] \)
\item \( f^{-1}(-10) \in [1.22, 1.62] \)
\item \( \text{ The function is not invertible for all Real numbers. } \)

\end{enumerate} }
\litem{
Choose the interval below that $f$ composed with $g$ at $x=-1$ is in.\[ f(x) = 2x^{3} +4 x^{2} +2 x + 1 \text{ and } g(x) = 2x^{3} +2 x^{2} +2 x \]\begin{enumerate}[label=\Alph*.]
\item \( (f \circ g)(-1) \in [9, 20] \)
\item \( (f \circ g)(-1) \in [-15, -9] \)
\item \( (f \circ g)(-1) \in [-4, 0] \)
\item \( (f \circ g)(-1) \in [4, 9] \)
\item \( \text{It is not possible to compose the two functions.} \)

\end{enumerate} }
\litem{
Determine whether the function below is 1-1.\[ f(x) = -24 x^2 - 270 x - 729 \]\begin{enumerate}[label=\Alph*.]
\item \( \text{Yes, the function is 1-1.} \)
\item \( \text{No, because the domain of the function is not $(-\infty, \infty)$.} \)
\item \( \text{No, because there is an $x$-value that goes to 2 different $y$-values.} \)
\item \( \text{No, because the range of the function is not $(-\infty, \infty)$.} \)
\item \( \text{No, because there is a $y$-value that goes to 2 different $x$-values.} \)

\end{enumerate} }
\litem{
Find the inverse of the function below. Then, evaluate the inverse at $x = 5$ and choose the interval that $f^{-1}(5)$ belongs to.\[ f(x) = \ln{(x+3)}-2 \]\begin{enumerate}[label=\Alph*.]
\item \( f^{-1}(5) \in [1090.63, 1097.63] \)
\item \( f^{-1}(5) \in [3.39, 6.39] \)
\item \( f^{-1}(5) \in [1097.63, 1106.63] \)
\item \( f^{-1}(5) \in [2977.96, 2983.96] \)
\item \( f^{-1}(5) \in [15.09, 18.09] \)

\end{enumerate} }
\litem{
Find the inverse of the function below. Then, evaluate the inverse at $x = 9$ and choose the interval that $f^{-1}(9)$ belongs to.\[ f(x) = \ln{(x-4)}-5 \]\begin{enumerate}[label=\Alph*.]
\item \( f^{-1}(9) \in [1202596.28, 1202605.28] \)
\item \( f^{-1}(9) \in [1202604.28, 1202609.28] \)
\item \( f^{-1}(9) \in [141.41, 144.41] \)
\item \( f^{-1}(9) \in [442408.39, 442416.39] \)
\item \( f^{-1}(9) \in [55.6, 61.6] \)

\end{enumerate} }
\litem{
Subtract the following functions, then choose the domain of the resulting function from the list below.\[ f(x) = 8x^{2} +5 x + 5 \text{ and } g(x) = 3x + 6 \]\begin{enumerate}[label=\Alph*.]
\item \( \text{ The domain is all Real numbers greater than or equal to } x = a, \text{ where } a \in [-8, -2] \)
\item \( \text{ The domain is all Real numbers less than or equal to } x = a, \text{ where } a \in [2.33, 3.33] \)
\item \( \text{ The domain is all Real numbers except } x = a, \text{ where } a \in [1.75, 6.75] \)
\item \( \text{ The domain is all Real numbers except } x = a \text{ and } x = b, \text{ where } a \in [4.2, 10.2] \text{ and } b \in [-7.17, -4.17] \)
\item \( \text{ The domain is all Real numbers. } \)

\end{enumerate} }
\litem{
Determine whether the function below is 1-1.\[ f(x) = (5 x - 23)^3 \]\begin{enumerate}[label=\Alph*.]
\item \( \text{Yes, the function is 1-1.} \)
\item \( \text{No, because the range of the function is not $(-\infty, \infty)$.} \)
\item \( \text{No, because there is an $x$-value that goes to 2 different $y$-values.} \)
\item \( \text{No, because the domain of the function is not $(-\infty, \infty)$.} \)
\item \( \text{No, because there is a $y$-value that goes to 2 different $x$-values.} \)

\end{enumerate} }
\litem{
Multiply the following functions, then choose the domain of the resulting function from the list below.\[ f(x) = x + 3 \text{ and } g(x) = 6x + 7 \]\begin{enumerate}[label=\Alph*.]
\item \( \text{ The domain is all Real numbers greater than or equal to } x = a, \text{ where } a \in [-10, -5] \)
\item \( \text{ The domain is all Real numbers except } x = a, \text{ where } a \in [-5.67, -2.67] \)
\item \( \text{ The domain is all Real numbers less than or equal to } x = a, \text{ where } a \in [-5.33, -0.33] \)
\item \( \text{ The domain is all Real numbers except } x = a \text{ and } x = b, \text{ where } a \in [4.6, 11.6] \text{ and } b \in [6.2, 11.2] \)
\item \( \text{ The domain is all Real numbers. } \)

\end{enumerate} }
\end{enumerate}

\end{document}