\documentclass{extbook}[14pt]
\usepackage{multicol, enumerate, enumitem, hyperref, color, soul, setspace, parskip, fancyhdr, amssymb, amsthm, amsmath, bbm, latexsym, units, mathtools}
\everymath{\displaystyle}
\usepackage[headsep=0.5cm,headheight=0cm, left=1 in,right= 1 in,top= 1 in,bottom= 1 in]{geometry}
\usepackage{dashrule}  % Package to use the command below to create lines between items
\newcommand{\litem}[1]{\item #1

\rule{\textwidth}{0.4pt}}
\pagestyle{fancy}
\lhead{}
\chead{Answer Key for Progress Quiz 4 Version C}
\rhead{}
\lfoot{9187-5854}
\cfoot{}
\rfoot{Spring 2021}
\begin{document}
\textbf{This key should allow you to understand why you choose the option you did (beyond just getting a question right or wrong). \href{https://xronos.clas.ufl.edu/mac1105spring2020/courseDescriptionAndMisc/Exams/LearningFromResults}{More instructions on how to use this key can be found here}.}

\textbf{If you have a suggestion to make the keys better, \href{https://forms.gle/CZkbZmPbC9XALEE88}{please fill out the short survey here}.}

\textit{Note: This key is auto-generated and may contain issues and/or errors. The keys are reviewed after each exam to ensure grading is done accurately. If there are issues (like duplicate options), they are noted in the offline gradebook. The keys are a work-in-progress to give students as many resources to improve as possible.}

\rule{\textwidth}{0.4pt}

\begin{enumerate}\litem{
Which of the following intervals describes the Range of the function below?
\[ f(x) = \log_2{(x+3)}+5 \]The solution is \( (\infty, \infty) \), which is option E.\begin{enumerate}[label=\Alph*.]
\item \( (-\infty, a), a \in [4.8, 5.2] \)

$(-\infty, 5)$, which corresponds to using the vertical shift while the Range is $(-\infty, \infty)$.
\item \( [a, \infty), a \in [-3.2, -2.3] \)

$[5, \infty)$, which corresponds to using the flipped Domain AND including the endpoint.
\item \( (-\infty, a), a \in [-7.4, -4.7] \)

$(-\infty, -5)$, which corresponds to using the using the negative of vertical shift on $(0, \infty)$.
\item \( [a, \infty), a \in [0.4, 4.1] \)

$[3, \infty)$, which corresponds to using the negative of the horizontal shift AND including the endpoint.
\item \( (-\infty, \infty) \)

*This is the correct option.
\end{enumerate}

\textbf{General Comment:} \textbf{General Comments}: The domain of a basic logarithmic function is $(0, \infty)$ and the Range is $(-\infty, \infty)$. We can use shifts when finding the Domain, but the Range will always be all Real numbers.
}
\litem{
Solve the equation for $x$ and choose the interval that contains the solution (if it exists).
\[ 2^{3x-4} = \left(\frac{1}{25}\right)^{4x+5} \]The solution is \( x = -0.891 \), which is option C.\begin{enumerate}[label=\Alph*.]
\item \( x \in [0, 1.3] \)

$x = 0.602$, which corresponds to distributing the $\ln(base)$ to the first term of the exponent only.
\item \( x \in [12.9, 14.8] \)

$x = 13.322$, which corresponds to distributing the $\ln(base)$ to the second term of the exponent only.
\item \( x \in [-2, 0.3] \)

* $x = -0.891$, which is the correct option.
\item \( x \in [-10.2, -7.2] \)

$x = -9.000$, which corresponds to solving the numerators as equal while ignoring the bases are different.
\item \( \text{There is no Real solution to the equation.} \)

This corresponds to believing there is no solution since the bases are not powers of each other.
\end{enumerate}

\textbf{General Comment:} \textbf{General Comments:} This question was written so that the bases could not be written the same. You will need to take the log of both sides.
}
\litem{
Which of the following intervals describes the Range of the function below?
\[ f(x) = e^{x+5}-6 \]The solution is \( (-6, \infty) \), which is option B.\begin{enumerate}[label=\Alph*.]
\item \( (-\infty, a), a \in [5, 8] \)

$(-\infty, 6)$, which corresponds to using the negative vertical shift AND flipping the Range interval.
\item \( (a, \infty), a \in [-7, 2] \)

* $(-6, \infty)$, which is the correct option.
\item \( [a, \infty), a \in [-7, 2] \)

$[-6, \infty)$, which corresponds to including the endpoint.
\item \( (-\infty, a], a \in [5, 8] \)

$(-\infty, 6]$, which corresponds to using the negative vertical shift AND flipping the Range interval AND including the endpoint.
\item \( (-\infty, \infty) \)

This corresponds to confusing range of an exponential function with the domain of an exponential function.
\end{enumerate}

\textbf{General Comment:} \textbf{General Comments}: Domain of a basic exponential function is $(-\infty, \infty)$ while the Range is $(0, \infty)$. We can shift these intervals [and even flip when $a<0$!] to find the new Domain/Range.
}
\litem{
Solve the equation for $x$ and choose the interval that contains the solution (if it exists).
\[ \log_{5}{(-4x+6)}+5 = 2 \]The solution is \( x = 1.498 \), which is option B.\begin{enumerate}[label=\Alph*.]
\item \( x \in [-5.75, -0.75] \)

$x = -4.750$, which corresponds to ignoring the vertical shift when converting to exponential form.
\item \( x \in [-1.5, 8.5] \)

* $x = 1.498$, which is the correct option.
\item \( x \in [62.25, 66.25] \)

$x = 62.250$, which corresponds to reversing the base and exponent when converting.
\item \( x \in [57.25, 61.25] \)

$x = 59.250$, which corresponds to reversing the base and exponent when converting and reversing the value with $x$.
\item \( \text{There is no Real solution to the equation.} \)

Corresponds to believing a negative coefficient within the log equation means there is no Real solution.
\end{enumerate}

\textbf{General Comment:} \textbf{General Comments:} First, get the equation in the form $\log_b{(cx+d)} = a$. Then, convert to $b^a = cx+d$ and solve.
}
\litem{
Which of the following intervals describes the Range of the function below?
\[ f(x) = -e^{x+1}-5 \]The solution is \( (-\infty, -5) \), which is option C.\begin{enumerate}[label=\Alph*.]
\item \( [a, \infty), a \in [1, 10] \)

$[5, \infty)$, which corresponds to using the negative vertical shift AND flipping the Range interval AND including the endpoint.
\item \( (a, \infty), a \in [1, 10] \)

$(5, \infty)$, which corresponds to using the negative vertical shift AND flipping the Range interval.
\item \( (-\infty, a), a \in [-12, -4] \)

* $(-\infty, -5)$, which is the correct option.
\item \( (-\infty, a], a \in [-12, -4] \)

$(-\infty, -5]$, which corresponds to including the endpoint.
\item \( (-\infty, \infty) \)

This corresponds to confusing range of an exponential function with the domain of an exponential function.
\end{enumerate}

\textbf{General Comment:} \textbf{General Comments}: Domain of a basic exponential function is $(-\infty, \infty)$ while the Range is $(0, \infty)$. We can shift these intervals [and even flip when $a<0$!] to find the new Domain/Range.
}
\litem{
 Solve the equation for $x$ and choose the interval that contains $x$ (if it exists).
\[  23 = \sqrt[4]{\frac{17}{e^{8x}}} \]The solution is \( x = -1.214, \text{ which does not fit in any of the interval options.} \), which is option E.\begin{enumerate}[label=\Alph*.]
\item \( x \in [-13.5, -11.7] \)

$x = -11.854$, which corresponds to thinking you don't need to take the natural log of both sides before reducing, as if the right side already has a natural log.
\item \( x \in [0.9, 1.4] \)

$x = 1.214$, which is the negative of the correct solution.
\item \( x \in [-1.1, 1.2] \)

$x = -0.430$, which corresponds to treating any root as a square root.
\item \( \text{There is no Real solution to the equation.} \)

This corresponds to believing you cannot solve the equation.
\item \( \text{None of the above.} \)

* $x = -1.214$ is the correct solution and does not fit in any of the other intervals.
\end{enumerate}

\textbf{General Comment:} \textbf{General Comments}: After using the properties of logarithmic functions to break up the right-hand side, use $\ln(e) = 1$ to reduce the question to a linear function to solve. You can put $\ln(17)$ into a calculator if you are having trouble.
}
\litem{
Solve the equation for $x$ and choose the interval that contains the solution (if it exists).
\[ \log_{4}{(-4x+6)}+4 = 3 \]The solution is \( x = 1.438 \), which is option D.\begin{enumerate}[label=\Alph*.]
\item \( x \in [-15.25, -13.05] \)

$x = -14.500$, which corresponds to ignoring the vertical shift when converting to exponential form.
\item \( x \in [-2.49, -1.09] \)

$x = -1.750$, which corresponds to reversing the base and exponent when converting and reversing the value with $x$.
\item \( x \in [1.19, 1.31] \)

$x = 1.250$, which corresponds to reversing the base and exponent when converting.
\item \( x \in [1.32, 4.05] \)

* $x = 1.438$, which is the correct option.
\item \( \text{There is no Real solution to the equation.} \)

Corresponds to believing a negative coefficient within the log equation means there is no Real solution.
\end{enumerate}

\textbf{General Comment:} \textbf{General Comments:} First, get the equation in the form $\log_b{(cx+d)} = a$. Then, convert to $b^a = cx+d$ and solve.
}
\litem{
Which of the following intervals describes the Domain of the function below?
\[ f(x) = -\log_2{(x+8)}-7 \]The solution is \( (-8, \infty) \), which is option B.\begin{enumerate}[label=\Alph*.]
\item \( (-\infty, a], a \in [6.85, 7.69] \)

$(-\infty, 7]$, which corresponds to using the negative vertical shift AND including the endpoint AND flipping the domain.
\item \( (a, \infty), a \in [-8.11, -7.73] \)

* $(-8, \infty)$, which is the correct option.
\item \( (-\infty, a), a \in [7.82, 8.09] \)

$(-\infty, 8)$, which corresponds to flipping the Domain. Remember: the general for is $a*\log(x-h)+k$, \textbf{where $a$ does not affect the domain}.
\item \( [a, \infty), a \in [-7.15, -6.67] \)

$[-7, \infty)$, which corresponds to using the vertical shift when shifting the Domain AND including the endpoint.
\item \( (-\infty, \infty) \)

This corresponds to thinking of the range of the log function (or the domain of the exponential function).
\end{enumerate}

\textbf{General Comment:} \textbf{General Comments}: The domain of a basic logarithmic function is $(0, \infty)$ and the Range is $(-\infty, \infty)$. We can use shifts when finding the Domain, but the Range will always be all Real numbers.
}
\litem{
Solve the equation for $x$ and choose the interval that contains the solution (if it exists).
\[ 2^{-4x-5} = \left(\frac{1}{25}\right)^{-2x+2} \]The solution is \( x = 0.323 \), which is option B.\begin{enumerate}[label=\Alph*.]
\item \( x \in [1.4, 1.9] \)

$x = 1.486$, which corresponds to distributing the $\ln(base)$ to the second term of the exponent only.
\item \( x \in [-0.3, 0.8] \)

* $x = 0.323$, which is the correct option.
\item \( x \in [-5.9, -2.9] \)

$x = -3.500$, which corresponds to solving the numerators as equal while ignoring the bases are different.
\item \( x \in [-1.3, -0.2] \)

$x = -0.760$, which corresponds to distributing the $\ln(base)$ to the first term of the exponent only.
\item \( \text{There is no Real solution to the equation.} \)

This corresponds to believing there is no solution since the bases are not powers of each other.
\end{enumerate}

\textbf{General Comment:} \textbf{General Comments:} This question was written so that the bases could not be written the same. You will need to take the log of both sides.
}
\litem{
 Solve the equation for $x$ and choose the interval that contains $x$ (if it exists).
\[  13 = \ln{\sqrt[3]{\frac{11}{e^{8x}}}} \]The solution is \( x = -4.575, \text{ which does not fit in any of the interval options.} \), which is option E.\begin{enumerate}[label=\Alph*.]
\item \( x \in [4.1, 5.7] \)

$x = 4.575$, which is the negative of the correct solution.
\item \( x \in [-2.8, 0.7] \)

$x = -1.262$, which corresponds to thinking you need to take the natural log of the left side before reducing.
\item \( x \in [-3.2, -1.6] \)

$x = -2.950$, which corresponds to treating any root as a square root.
\item \( \text{There is no Real solution to the equation.} \)

This corresponds to believing you cannot solve the equation.
\item \( \text{None of the above.} \)

*$x = -4.575$ is the correct solution and does not fit in any of the other intervals.
\end{enumerate}

\textbf{General Comment:} \textbf{General Comments}: After using the properties of logarithmic functions to break up the right-hand side, use $\ln(e) = 1$ to reduce the question to a linear function to solve. You can put $\ln(11)$ into a calculator if you are having trouble.
}
\end{enumerate}

\end{document}