\documentclass{extbook}[14pt]
\usepackage{multicol, enumerate, enumitem, hyperref, color, soul, setspace, parskip, fancyhdr, amssymb, amsthm, amsmath, bbm, latexsym, units, mathtools}
\everymath{\displaystyle}
\usepackage[headsep=0.5cm,headheight=0cm, left=1 in,right= 1 in,top= 1 in,bottom= 1 in]{geometry}
\usepackage{dashrule}  % Package to use the command below to create lines between items
\newcommand{\litem}[1]{\item #1

\rule{\textwidth}{0.4pt}}
\pagestyle{fancy}
\lhead{}
\chead{Answer Key for Progress Quiz 4 Version C}
\rhead{}
\lfoot{8448-1521}
\cfoot{}
\rfoot{Fall 2020}
\begin{document}
\textbf{This key should allow you to understand why you choose the option you did (beyond just getting a question right or wrong). \href{https://xronos.clas.ufl.edu/mac1105spring2020/courseDescriptionAndMisc/Exams/LearningFromResults}{More instructions on how to use this key can be found here}.}

\textbf{If you have a suggestion to make the keys better, \href{https://forms.gle/CZkbZmPbC9XALEE88}{please fill out the short survey here}.}

\textit{Note: This key is auto-generated and may contain issues and/or errors. The keys are reviewed after each exam to ensure grading is done accurately. If there are issues (like duplicate options), they are noted in the offline gradebook. The keys are a work-in-progress to give students as many resources to improve as possible.}

\rule{\textwidth}{0.4pt}

\begin{enumerate}\litem{
 Solve the equation for $x$ and choose the interval that contains $x$ (if it exists).
\[  7 = \ln{\sqrt[6]{\frac{6}{e^{3x}}}} \]
The solution is \( x = -13.403 \), which is option B.\begin{enumerate}[label=\Alph*.]
\item \( x \in [-4.36, -3.73] \)

$x = -4.069$, which corresponds to treating any root as a square root.
\item \( x \in [-13.7, -13.21] \)

* $x = -13.403$, which is the correct option.
\item \( x \in [-4.64, -4.1] \)

$x = -4.489$, which corresponds to thinking you need to take the natural log of on the left before reducing.
\item \( \text{There is no Real solution to the equation.} \)

This corresponds to believing you cannot solve the equation.
\item \( \text{None of the above.} \)

This corresponds to making an unexpected error.
\end{enumerate}

\textbf{General Comment:} \textbf{General Comments}: After using the properties of logarithmic functions to break up the right-hand side, use $\ln(e) = 1$ to reduce the question to a linear function to solve. You can put $\ln(6)$ into a calculator if you are having trouble.
}
\litem{
Solve the equation for $x$ and choose the interval that contains the solution (if it exists).
\[ \log_{4}{(-2x+6)}+6 = 3 \]
The solution is \( x = 2.992 \), which is option A.\begin{enumerate}[label=\Alph*.]
\item \( x \in [-4.01, 5.99] \)

* $x = 2.992$, which is the correct option.
\item \( x \in [-45.5, -39.5] \)

$x = -43.500$, which corresponds to reversing the base and exponent when converting and reversing the value with $x$.
\item \( x \in [-39.5, -32.5] \)

$x = -37.500$, which corresponds to reversing the base and exponent when converting.
\item \( x \in [-36, -26] \)

$x = -29.000$, which corresponds to ignoring the vertical shift when converting to exponential form.
\item \( \text{There is no Real solution to the equation.} \)

Corresponds to believing a negative coefficient within the log equation means there is no Real solution.
\end{enumerate}

\textbf{General Comment:} \textbf{General Comments:} First, get the equation in the form $\log_b{(cx+d)} = a$. Then, convert to $b^a = cx+d$ and solve.
}
\litem{
Solve the equation for $x$ and choose the interval that contains the solution (if it exists).
\[ 3^{5x+2} = 49^{2x-3} \]
The solution is \( x = 6.056 \), which is option C.\begin{enumerate}[label=\Alph*.]
\item \( x \in [-5.9, -4] \)

$x = -4.624$, which corresponds to distributing the $\ln(base)$ to the second term of the exponent only.
\item \( x \in [0.7, 2.6] \)

$x = 2.183$, which corresponds to distributing the $\ln(base)$ to the first term of the exponent only.
\item \( x \in [5.7, 6.4] \)

* $x = 6.056$, which is the correct option.
\item \( x \in [-2.5, -0.4] \)

$x = -1.667$, which corresponds to solving the numerators as equal while ignoring the bases are different.
\item \( \text{There is no Real solution to the equation.} \)

This corresponds to believing there is no solution since the bases are not powers of each other.
\end{enumerate}

\textbf{General Comment:} \textbf{General Comments:} This question was written so that the bases could not be written the same. You will need to take the log of both sides.
}
\litem{
Which of the following intervals describes the Range of the function below?
\[ f(x) = -e^{x+3}+1 \]
The solution is \( (-\infty, 1) \), which is option C.\begin{enumerate}[label=\Alph*.]
\item \( [a, \infty), a \in [-2.8, 0.9] \)

$[-1, \infty)$, which corresponds to using the negative vertical shift AND flipping the Range interval AND including the endpoint.
\item \( (a, \infty), a \in [-2.8, 0.9] \)

$(-1, \infty)$, which corresponds to using the negative vertical shift AND flipping the Range interval.
\item \( (-\infty, a), a \in [0.3, 1.8] \)

* $(-\infty, 1)$, which is the correct option.
\item \( (-\infty, a], a \in [0.3, 1.8] \)

$(-\infty, 1]$, which corresponds to including the endpoint.
\item \( (-\infty, \infty) \)

This corresponds to confusing range of an exponential function with the domain of an exponential function.
\end{enumerate}

\textbf{General Comment:} \textbf{General Comments}: Domain of a basic exponential function is $(-\infty, \infty)$ while the Range is $(0, \infty)$. We can shift these intervals [and even flip when $a<0$!] to find the new Domain/Range.
}
\litem{
Solve the equation for $x$ and choose the interval that contains the solution (if it exists).
\[ 5^{5x+5} = 216^{3x-5} \]
The solution is \( x = 4.323 \), which is option B.\begin{enumerate}[label=\Alph*.]
\item \( x \in [-6, -1] \)

$x = -5.000$, which corresponds to solving the numerators as equal while ignoring the bases are different.
\item \( x \in [4.32, 5.32] \)

* $x = 4.323$, which is the correct option.
\item \( x \in [-21.46, -13.46] \)

$x = -17.462$, which corresponds to distributing the $\ln(base)$ to the second term of the exponent only.
\item \( x \in [0.24, 4.24] \)

$x = 1.238$, which corresponds to distributing the $\ln(base)$ to the first term of the exponent only.
\item \( \text{There is no Real solution to the equation.} \)

This corresponds to believing there is no solution since the bases are not powers of each other.
\end{enumerate}

\textbf{General Comment:} \textbf{General Comments:} This question was written so that the bases could not be written the same. You will need to take the log of both sides.
}
\litem{
Solve the equation for $x$ and choose the interval that contains the solution (if it exists).
\[ \log_{2}{(3x+7)}+5 = 3 \]
The solution is \( x = -2.250 \), which is option C.\begin{enumerate}[label=\Alph*.]
\item \( x \in [-0.32, 0.65] \)

$x = 0.333$, which corresponds to ignoring the vertical shift when converting to exponential form.
\item \( x \in [3.66, 4.68] \)

$x = 3.667$, which corresponds to reversing the base and exponent when converting and reversing the value with $x$.
\item \( x \in [-3.04, -1.22] \)

* $x = -2.250$, which is the correct option.
\item \( x \in [-2, -0.63] \)

$x = -1.000$, which corresponds to reversing the base and exponent when converting.
\item \( \text{There is no Real solution to the equation.} \)

Corresponds to believing a negative coefficient within the log equation means there is no Real solution.
\end{enumerate}

\textbf{General Comment:} \textbf{General Comments:} First, get the equation in the form $\log_b{(cx+d)} = a$. Then, convert to $b^a = cx+d$ and solve.
}
\litem{
Which of the following intervals describes the Domain of the function below?
\[ f(x) = -\log_2{(x-5)}-6 \]
The solution is \( (5, \infty) \), which is option B.\begin{enumerate}[label=\Alph*.]
\item \( (-\infty, a], a \in [5.47, 6.75] \)

$(-\infty, 6]$, which corresponds to using the negative vertical shift AND including the endpoint AND flipping the domain.
\item \( (a, \infty), a \in [4.4, 5.21] \)

* $(5, \infty)$, which is the correct option.
\item \( [a, \infty), a \in [-6.36, -5.68] \)

$[-6, \infty)$, which corresponds to using the vertical shift when shifting the Domain AND including the endpoint.
\item \( (-\infty, a), a \in [-5.79, -4.59] \)

$(-\infty, -5)$, which corresponds to flipping the Domain. Remember: the general for is $a*\log(x-h)+k$, \textbf{where $a$ does not affect the domain}.
\item \( (-\infty, \infty) \)

This corresponds to thinking of the range of the log function (or the domain of the exponential function).
\end{enumerate}

\textbf{General Comment:} \textbf{General Comments}: The domain of a basic logarithmic function is $(0, \infty)$ and the Range is $(-\infty, \infty)$. We can use shifts when finding the Domain, but the Range will always be all Real numbers.
}
\litem{
 Solve the equation for $x$ and choose the interval that contains $x$ (if it exists).
\[  13 = \ln{\sqrt[5]{\frac{10}{e^{6x}}}} \]
The solution is \( x = -10.45, \text{ which does not fit in any of the interval options.} \), which is option E.\begin{enumerate}[label=\Alph*.]
\item \( x \in [-6.3, -2.7] \)

$x = -3.950$, which corresponds to treating any root as a square root.
\item \( x \in [9.9, 10.7] \)

$x = 10.450$, which is the negative of the correct solution.
\item \( x \in [-3.9, -1.6] \)

$x = -2.521$, which corresponds to thinking you need to take the natural log of the left side before reducing.
\item \( \text{There is no Real solution to the equation.} \)

This corresponds to believing you cannot solve the equation.
\item \( \text{None of the above.} \)

*$x = -10.450$ is the correct solution and does not fit in any of the other intervals.
\end{enumerate}

\textbf{General Comment:} \textbf{General Comments}: After using the properties of logarithmic functions to break up the right-hand side, use $\ln(e) = 1$ to reduce the question to a linear function to solve. You can put $\ln(10)$ into a calculator if you are having trouble.
}
\litem{
Which of the following intervals describes the Domain of the function below?
\[ f(x) = -e^{x+2}+1 \]
The solution is \( (-\infty, \infty) \), which is option E.\begin{enumerate}[label=\Alph*.]
\item \( [a, \infty), a \in [-1.91, -0.86] \)

$[-1, \infty)$, which corresponds to using the negative vertical shift AND flipping the Range interval AND including the endpoint.
\item \( (-\infty, a), a \in [0.85, 2.48] \)

$(-\infty, 1)$, which corresponds to using the correct vertical shift *if we wanted the Range*.
\item \( (-\infty, a], a \in [0.85, 2.48] \)

$(-\infty, 1]$, which corresponds to using the correct vertical shift *if we wanted the Range* AND including the endpoint.
\item \( (a, \infty), a \in [-1.91, -0.86] \)

$(-1, \infty)$, which corresponds to using the negative vertical shift AND flipping the Range interval.
\item \( (-\infty, \infty) \)

* This is the correct option.
\end{enumerate}

\textbf{General Comment:} \textbf{General Comments}: Domain of a basic exponential function is $(-\infty, \infty)$ while the Range is $(0, \infty)$. We can shift these intervals [and even flip when $a<0$!] to find the new Domain/Range.
}
\litem{
Which of the following intervals describes the Domain of the function below?
\[ f(x) = \log_2{(x+9)}+1 \]
The solution is \( (-9, \infty) \), which is option B.\begin{enumerate}[label=\Alph*.]
\item \( (-\infty, a], a \in [-1, 0] \)

$(-\infty, -1]$, which corresponds to using the negative vertical shift AND including the endpoint AND flipping the domain.
\item \( (a, \infty), a \in [-10, -8] \)

* $(-9, \infty)$, which is the correct option.
\item \( (-\infty, a), a \in [4, 10] \)

$(-\infty, 9)$, which corresponds to flipping the Domain. Remember: the general for is $a*\log(x-h)+k$, \textbf{where $a$ does not affect the domain}.
\item \( [a, \infty), a \in [1, 4] \)

$[1, \infty)$, which corresponds to using the vertical shift when shifting the Domain AND including the endpoint.
\item \( (-\infty, \infty) \)

This corresponds to thinking of the range of the log function (or the domain of the exponential function).
\end{enumerate}

\textbf{General Comment:} \textbf{General Comments}: The domain of a basic logarithmic function is $(0, \infty)$ and the Range is $(-\infty, \infty)$. We can use shifts when finding the Domain, but the Range will always be all Real numbers.
}
\end{enumerate}

\end{document}