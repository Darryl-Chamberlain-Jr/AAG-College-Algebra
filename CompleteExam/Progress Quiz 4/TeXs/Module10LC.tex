\documentclass[14pt]{extbook}
\usepackage{multicol, enumerate, enumitem, hyperref, color, soul, setspace, parskip, fancyhdr} %General Packages
\usepackage{amssymb, amsthm, amsmath, bbm, latexsym, units, mathtools} %Math Packages
\everymath{\displaystyle} %All math in Display Style
% Packages with additional options
\usepackage[headsep=0.5cm,headheight=12pt, left=1 in,right= 1 in,top= 1 in,bottom= 1 in]{geometry}
\usepackage[usenames,dvipsnames]{xcolor}
\usepackage{dashrule}  % Package to use the command below to create lines between items
\newcommand{\litem}[1]{\item#1\hspace*{-1cm}\rule{\textwidth}{0.4pt}}
\pagestyle{fancy}
\lhead{Progress Quiz 4}
\chead{}
\rhead{Version C}
\lfoot{9187-5854}
\cfoot{}
\rfoot{Spring 2021}
\begin{document}

\begin{enumerate}
\litem{
Factor the polynomial below completely. Then, choose the intervals the zeros of the polynomial belong to, where $z_1 \leq z_2 \leq z_3$. \textit{To make the problem easier, all zeros are between -5 and 5.}\[ f(x) = 6x^{3} + x^{2} -20 x -12 \]\begin{enumerate}[label=\Alph*.]
\item \( z_1 \in [-2.2, -1.8], \text{   }  z_2 \in [0.07, 0.47], \text{   and   } z_3 \in [2.8, 3.8] \)
\item \( z_1 \in [-1.6, -0.4], \text{   }  z_2 \in [-1.31, -0.3], \text{   and   } z_3 \in [1.6, 2.3] \)
\item \( z_1 \in [-2.2, -1.8], \text{   }  z_2 \in [0.49, 0.9], \text{   and   } z_3 \in [1, 1.9] \)
\item \( z_1 \in [-1.6, -0.4], \text{   }  z_2 \in [-1.31, -0.3], \text{   and   } z_3 \in [1.6, 2.3] \)
\item \( z_1 \in [-2.2, -1.8], \text{   }  z_2 \in [0.49, 0.9], \text{   and   } z_3 \in [1, 1.9] \)

\end{enumerate} }
\litem{
Perform the division below. Then, find the intervals that correspond to the quotient in the form $ax^2+bx+c$ and remainder $r$.\[ \frac{9x^{3} +27 x^{2} +8 x -24}{x + 2} \]\begin{enumerate}[label=\Alph*.]
\item \( a \in [8, 12], \text{   } b \in [-3, 5], \text{   } c \in [8, 11], \text{   and   } r \in [-49, -46]. \)
\item \( a \in [-20, -11], \text{   } b \in [58, 70], \text{   } c \in [-119, -115], \text{   and   } r \in [211, 215]. \)
\item \( a \in [8, 12], \text{   } b \in [41, 50], \text{   } c \in [94, 101], \text{   and   } r \in [172, 179]. \)
\item \( a \in [8, 12], \text{   } b \in [8, 12], \text{   } c \in [-10, -9], \text{   and   } r \in [-6, 0]. \)
\item \( a \in [-20, -11], \text{   } b \in [-14, -7], \text{   } c \in [-10, -9], \text{   and   } r \in [-46, -43]. \)

\end{enumerate} }
\litem{
What are the \textit{possible Rational} roots of the polynomial below?\[ f(x) = 5x^{4} +4 x^{3} +3 x^{2} +2 x + 7 \]\begin{enumerate}[label=\Alph*.]
\item \( \pm 1,\pm 7 \)
\item \( \text{ All combinations of: }\frac{\pm 1,\pm 7}{\pm 1,\pm 5} \)
\item \( \pm 1,\pm 5 \)
\item \( \text{ All combinations of: }\frac{\pm 1,\pm 5}{\pm 1,\pm 7} \)
\item \( \text{ There is no formula or theorem that tells us all possible Rational roots.} \)

\end{enumerate} }
\litem{
Factor the polynomial below completely. Then, choose the intervals the zeros of the polynomial belong to, where $z_1 \leq z_2 \leq z_3$. \textit{To make the problem easier, all zeros are between -5 and 5.}\[ f(x) = 12x^{3} +53 x^{2} -45 x -50 \]\begin{enumerate}[label=\Alph*.]
\item \( z_1 \in [-1.32, -1.24], \text{   }  z_2 \in [0.55, 1.16], \text{   and   } z_3 \in [4.1, 6.1] \)
\item \( z_1 \in [-0.87, -0.32], \text{   }  z_2 \in [1.26, 1.75], \text{   and   } z_3 \in [4.1, 6.1] \)
\item \( z_1 \in [-5.27, -4.64], \text{   }  z_2 \in [0.07, 0.66], \text{   and   } z_3 \in [4.1, 6.1] \)
\item \( z_1 \in [-5.27, -4.64], \text{   }  z_2 \in [-1.52, -1.4], \text{   and   } z_3 \in [0.4, 1.1] \)
\item \( z_1 \in [-5.27, -4.64], \text{   }  z_2 \in [-1.1, -0.47], \text{   and   } z_3 \in [0.9, 1.7] \)

\end{enumerate} }
\litem{
Perform the division below. Then, find the intervals that correspond to the quotient in the form $ax^2+bx+c$ and remainder $r$.\[ \frac{9x^{3} -27 x -22}{x -2} \]\begin{enumerate}[label=\Alph*.]
\item \( a \in [6, 13], b \in [-18, -10], c \in [5, 11], \text{ and } r \in [-43, -34]. \)
\item \( a \in [17, 22], b \in [32, 43], c \in [45, 46], \text{ and } r \in [64, 72]. \)
\item \( a \in [6, 13], b \in [6, 12], c \in [-18, -17], \text{ and } r \in [-43, -34]. \)
\item \( a \in [17, 22], b \in [-42, -33], c \in [45, 46], \text{ and } r \in [-113, -108]. \)
\item \( a \in [6, 13], b \in [15, 19], c \in [5, 11], \text{ and } r \in [-6, 1]. \)

\end{enumerate} }
\litem{
Perform the division below. Then, find the intervals that correspond to the quotient in the form $ax^2+bx+c$ and remainder $r$.\[ \frac{4x^{3} +34 x^{2} +80 x + 48}{x + 5} \]\begin{enumerate}[label=\Alph*.]
\item \( a \in [-23, -19], \text{   } b \in [131.5, 135.1], \text{   } c \in [-593, -583], \text{   and   } r \in [2997, 3001]. \)
\item \( a \in [1, 7], \text{   } b \in [11.2, 15.1], \text{   } c \in [4, 15], \text{   and   } r \in [-2, 1]. \)
\item \( a \in [1, 7], \text{   } b \in [51, 55.2], \text{   } c \in [346, 353], \text{   and   } r \in [1794, 1801]. \)
\item \( a \in [-23, -19], \text{   } b \in [-66.4, -64.7], \text{   } c \in [-250, -246], \text{   and   } r \in [-1203, -1201]. \)
\item \( a \in [1, 7], \text{   } b \in [7.1, 11.1], \text{   } c \in [17, 23], \text{   and   } r \in [-77, -63]. \)

\end{enumerate} }
\litem{
What are the \textit{possible Rational} roots of the polynomial below?\[ f(x) = 2x^{4} +4 x^{3} +6 x^{2} +4 x + 6 \]\begin{enumerate}[label=\Alph*.]
\item \( \text{ All combinations of: }\frac{\pm 1,\pm 2}{\pm 1,\pm 2,\pm 3,\pm 6} \)
\item \( \pm 1,\pm 2,\pm 3,\pm 6 \)
\item \( \text{ All combinations of: }\frac{\pm 1,\pm 2,\pm 3,\pm 6}{\pm 1,\pm 2} \)
\item \( \pm 1,\pm 2 \)
\item \( \text{ There is no formula or theorem that tells us all possible Rational roots.} \)

\end{enumerate} }
\litem{
Factor the polynomial below completely, knowing that $x+3$ is a factor. Then, choose the intervals the zeros of the polynomial belong to, where $z_1 \leq z_2 \leq z_3 \leq z_4$. \textit{To make the problem easier, all zeros are between -5 and 5.}\[ f(x) = 6x^{4} -7 x^{3} -43 x^{2} +84 x -36 \]\begin{enumerate}[label=\Alph*.]
\item \( z_1 \in [-3.15, -2.72], \text{   }  z_2 \in [-2.12, -1.91], z_3 \in [-0.4, -0.11], \text{   and   } z_4 \in [2.3, 4.1] \)
\item \( z_1 \in [-2.21, -1.48], \text{   }  z_2 \in [-1.99, -1.44], z_3 \in [-0.76, -0.63], \text{   and   } z_4 \in [2.3, 4.1] \)
\item \( z_1 \in [-2.21, -1.48], \text{   }  z_2 \in [-1.99, -1.44], z_3 \in [-0.76, -0.63], \text{   and   } z_4 \in [2.3, 4.1] \)
\item \( z_1 \in [-3.15, -2.72], \text{   }  z_2 \in [0.04, 0.94], z_3 \in [1.03, 1.69], \text{   and   } z_4 \in [1.1, 2.1] \)
\item \( z_1 \in [-3.15, -2.72], \text{   }  z_2 \in [0.04, 0.94], z_3 \in [1.03, 1.69], \text{   and   } z_4 \in [1.1, 2.1] \)

\end{enumerate} }
\litem{
Factor the polynomial below completely, knowing that $x+5$ is a factor. Then, choose the intervals the zeros of the polynomial belong to, where $z_1 \leq z_2 \leq z_3 \leq z_4$. \textit{To make the problem easier, all zeros are between -5 and 5.}\[ f(x) = 12x^{4} +119 x^{3} +390 x^{2} +525 x + 250 \]\begin{enumerate}[label=\Alph*.]
\item \( z_1 \in [1.1, 1.3], \text{   }  z_2 \in [1.38, 1.76], z_3 \in [1.6, 3.4], \text{   and   } z_4 \in [3.4, 5.5] \)
\item \( z_1 \in [-5.12, -4.81], \text{   }  z_2 \in [-2.07, -1.82], z_3 \in [-1.7, -1.1], \text{   and   } z_4 \in [-2.3, -0.7] \)
\item \( z_1 \in [0.36, 0.55], \text{   }  z_2 \in [1.82, 2.42], z_3 \in [3.7, 6.2], \text{   and   } z_4 \in [3.4, 5.5] \)
\item \( z_1 \in [-5.12, -4.81], \text{   }  z_2 \in [-2.07, -1.82], z_3 \in [-1.6, -0.1], \text{   and   } z_4 \in [-0.8, 0.5] \)
\item \( z_1 \in [0.5, 0.89], \text{   }  z_2 \in [0.52, 0.87], z_3 \in [1.6, 3.4], \text{   and   } z_4 \in [3.4, 5.5] \)

\end{enumerate} }
\litem{
Perform the division below. Then, find the intervals that correspond to the quotient in the form $ax^2+bx+c$ and remainder $r$.\[ \frac{20x^{3} +65 x^{2} -48}{x + 3} \]\begin{enumerate}[label=\Alph*.]
\item \( a \in [-60, -59], b \in [242, 246], c \in [-735, -734], \text{ and } r \in [2155, 2161]. \)
\item \( a \in [-60, -59], b \in [-115, -110], c \in [-346, -344], \text{ and } r \in [-1087, -1081]. \)
\item \( a \in [14, 22], b \in [2, 6], c \in [-20, -11], \text{ and } r \in [-9, -1]. \)
\item \( a \in [14, 22], b \in [120, 128], c \in [369, 376], \text{ and } r \in [1075, 1086]. \)
\item \( a \in [14, 22], b \in [-20, -11], c \in [59, 66], \text{ and } r \in [-288, -282]. \)

\end{enumerate} }
\end{enumerate}

\end{document}