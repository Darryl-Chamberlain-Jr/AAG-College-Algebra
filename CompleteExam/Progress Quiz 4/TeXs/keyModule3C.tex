\documentclass{extbook}[14pt]
\usepackage{multicol, enumerate, enumitem, hyperref, color, soul, setspace, parskip, fancyhdr, amssymb, amsthm, amsmath, bbm, latexsym, units, mathtools}
\everymath{\displaystyle}
\usepackage[headsep=0.5cm,headheight=0cm, left=1 in,right= 1 in,top= 1 in,bottom= 1 in]{geometry}
\usepackage{dashrule}  % Package to use the command below to create lines between items
\newcommand{\litem}[1]{\item #1

\rule{\textwidth}{0.4pt}}
\pagestyle{fancy}
\lhead{}
\chead{Answer Key for Progress Quiz 4 Version C}
\rhead{}
\lfoot{9187-5854}
\cfoot{}
\rfoot{Spring 2021}
\begin{document}
\textbf{This key should allow you to understand why you choose the option you did (beyond just getting a question right or wrong). \href{https://xronos.clas.ufl.edu/mac1105spring2020/courseDescriptionAndMisc/Exams/LearningFromResults}{More instructions on how to use this key can be found here}.}

\textbf{If you have a suggestion to make the keys better, \href{https://forms.gle/CZkbZmPbC9XALEE88}{please fill out the short survey here}.}

\textit{Note: This key is auto-generated and may contain issues and/or errors. The keys are reviewed after each exam to ensure grading is done accurately. If there are issues (like duplicate options), they are noted in the offline gradebook. The keys are a work-in-progress to give students as many resources to improve as possible.}

\rule{\textwidth}{0.4pt}

\begin{enumerate}\litem{
Solve the linear inequality below. Then, choose the constant and interval combination that describes the solution set.
\[ 3 - 3 x \leq \frac{-19 x + 5}{8} < 5 - 3 x \]The solution is \( [3.80, 7.00) \), which is option C.\begin{enumerate}[label=\Alph*.]
\item \( (-\infty, a) \cup [b, \infty), \text{ where } a \in [2.8, 5.8] \text{ and } b \in [5, 11] \)

$(-\infty, 3.80) \cup [7.00, \infty)$, which corresponds to displaying the and-inequality as an or-inequality AND flipping the inequality.
\item \( (a, b], \text{ where } a \in [1.8, 8.8] \text{ and } b \in [6, 8] \)

$(3.80, 7.00]$, which corresponds to flipping the inequality.
\item \( [a, b), \text{ where } a \in [3.8, 6.8] \text{ and } b \in [6, 8] \)

$[3.80, 7.00)$, which is the correct option.
\item \( (-\infty, a] \cup (b, \infty), \text{ where } a \in [-1.2, 12.8] \text{ and } b \in [5, 11] \)

$(-\infty, 3.80] \cup (7.00, \infty)$, which corresponds to displaying the and-inequality as an or-inequality.
\item \( \text{None of the above.} \)


\end{enumerate}

\textbf{General Comment:} To solve, you will need to break up the compound inequality into two inequalities. Be sure to keep track of the inequality! It may be best to draw a number line and graph your solution.
}
\litem{
Solve the linear inequality below. Then, choose the constant and interval combination that describes the solution set.
\[ \frac{10}{6} - \frac{3}{2} x \geq \frac{8}{7} x - \frac{8}{4} \]The solution is \( (-\infty, 1.387] \), which is option A.\begin{enumerate}[label=\Alph*.]
\item \( (-\infty, a], \text{ where } a \in [-0.61, 2.39] \)

* $(-\infty, 1.387]$, which is the correct option.
\item \( [a, \infty), \text{ where } a \in [-0.61, 8.39] \)

 $[1.387, \infty)$, which corresponds to switching the direction of the interval. You likely did this if you did not flip the inequality when dividing by a negative!
\item \( [a, \infty), \text{ where } a \in [-6.39, 0.61] \)

 $[-1.387, \infty)$, which corresponds to switching the direction of the interval AND negating the endpoint. You likely did this if you did not flip the inequality when dividing by a negative as well as not moving values over to a side properly.
\item \( (-\infty, a], \text{ where } a \in [-1.39, 0.61] \)

 $(-\infty, -1.387]$, which corresponds to negating the endpoint of the solution.
\item \( \text{None of the above}. \)

You may have chosen this if you thought the inequality did not match the ends of the intervals.
\end{enumerate}

\textbf{General Comment:} Remember that less/greater than or equal to includes the endpoint, while less/greater do not. Also, remember that you need to flip the inequality when you multiply or divide by a negative.
}
\litem{
Solve the linear inequality below. Then, choose the constant and interval combination that describes the solution set.
\[ -5 + 4 x > 7 x \text{ or } 9 + 7 x < 9 x \]The solution is \( (-\infty, -1.667) \text{ or } (4.5, \infty) \), which is option B.\begin{enumerate}[label=\Alph*.]
\item \( (-\infty, a] \cup [b, \infty), \text{ where } a \in [-5.5, -2.5] \text{ and } b \in [-3.33, 2.67] \)

Corresponds to including the endpoints AND negating.
\item \( (-\infty, a) \cup (b, \infty), \text{ where } a \in [-2.67, 2.33] \text{ and } b \in [3.5, 5.5] \)

 * Correct option.
\item \( (-\infty, a] \cup [b, \infty), \text{ where } a \in [-1.67, 5.33] \text{ and } b \in [2.5, 6.5] \)

Corresponds to including the endpoints (when they should be excluded).
\item \( (-\infty, a) \cup (b, \infty), \text{ where } a \in [-4.5, -3.5] \text{ and } b \in [-2.33, 2.67] \)

Corresponds to inverting the inequality and negating the solution.
\item \( (-\infty, \infty) \)

Corresponds to the variable canceling, which does not happen in this instance.
\end{enumerate}

\textbf{General Comment:} When multiplying or dividing by a negative, flip the sign.
}
\litem{
Solve the linear inequality below. Then, choose the constant and interval combination that describes the solution set.
\[ -9x -8 \leq 5x -9 \]The solution is \( [0.071, \infty) \), which is option A.\begin{enumerate}[label=\Alph*.]
\item \( [a, \infty), \text{ where } a \in [0.05, 0.66] \)

* $[0.071, \infty)$, which is the correct option.
\item \( (-\infty, a], \text{ where } a \in [0.03, 0.15] \)

 $(-\infty, 0.071]$, which corresponds to switching the direction of the interval. You likely did this if you did not flip the inequality when dividing by a negative!
\item \( [a, \infty), \text{ where } a \in [-0.25, -0.02] \)

 $[-0.071, \infty)$, which corresponds to negating the endpoint of the solution.
\item \( (-\infty, a], \text{ where } a \in [-0.18, 0.05] \)

 $(-\infty, -0.071]$, which corresponds to switching the direction of the interval AND negating the endpoint. You likely did this if you did not flip the inequality when dividing by a negative as well as not moving values over to a side properly.
\item \( \text{None of the above}. \)

You may have chosen this if you thought the inequality did not match the ends of the intervals.
\end{enumerate}

\textbf{General Comment:} Remember that less/greater than or equal to includes the endpoint, while less/greater do not. Also, remember that you need to flip the inequality when you multiply or divide by a negative.
}
\litem{
Using an interval or intervals, describe all the $x$-values within or including a distance of the given values.
\[ \text{ No more than } 9 \text{ units from the number } 1. \]The solution is \( [-8, 10] \), which is option C.\begin{enumerate}[label=\Alph*.]
\item \( (-8, 10) \)

This describes the values less than 9 from 1
\item \( (-\infty, -8] \cup [10, \infty) \)

This describes the values no less than 9 from 1
\item \( [-8, 10] \)

This describes the values no more than 9 from 1
\item \( (-\infty, -8) \cup (10, \infty) \)

This describes the values more than 9 from 1
\item \( \text{None of the above} \)

You likely thought the values in the interval were not correct.
\end{enumerate}

\textbf{General Comment:} When thinking about this language, it helps to draw a number line and try points.
}
\litem{
Solve the linear inequality below. Then, choose the constant and interval combination that describes the solution set.
\[ \frac{9}{8} + \frac{3}{4} x > \frac{5}{9} x + \frac{7}{3} \]The solution is \( (6.214, \infty) \), which is option C.\begin{enumerate}[label=\Alph*.]
\item \( (-\infty, a), \text{ where } a \in [-7.21, -5.21] \)

 $(-\infty, -6.214)$, which corresponds to switching the direction of the interval AND negating the endpoint. You likely did this if you did not flip the inequality when dividing by a negative as well as not moving values over to a side properly.
\item \( (-\infty, a), \text{ where } a \in [5.21, 9.21] \)

 $(-\infty, 6.214)$, which corresponds to switching the direction of the interval. You likely did this if you did not flip the inequality when dividing by a negative!
\item \( (a, \infty), \text{ where } a \in [6.21, 7.21] \)

* $(6.214, \infty)$, which is the correct option.
\item \( (a, \infty), \text{ where } a \in [-8.21, -5.21] \)

 $(-6.214, \infty)$, which corresponds to negating the endpoint of the solution.
\item \( \text{None of the above}. \)

You may have chosen this if you thought the inequality did not match the ends of the intervals.
\end{enumerate}

\textbf{General Comment:} Remember that less/greater than or equal to includes the endpoint, while less/greater do not. Also, remember that you need to flip the inequality when you multiply or divide by a negative.
}
\litem{
Using an interval or intervals, describe all the $x$-values within or including a distance of the given values.
\[ \text{ No less than } 4 \text{ units from the number } 6. \]The solution is \( (-\infty, 2] \cup [10, \infty) \), which is option C.\begin{enumerate}[label=\Alph*.]
\item \( (-\infty, 2) \cup (10, \infty) \)

This describes the values more than 4 from 6
\item \( (2, 10) \)

This describes the values less than 4 from 6
\item \( (-\infty, 2] \cup [10, \infty) \)

This describes the values no less than 4 from 6
\item \( [2, 10] \)

This describes the values no more than 4 from 6
\item \( \text{None of the above} \)

You likely thought the values in the interval were not correct.
\end{enumerate}

\textbf{General Comment:} When thinking about this language, it helps to draw a number line and try points.
}
\litem{
Solve the linear inequality below. Then, choose the constant and interval combination that describes the solution set.
\[ -4 + 4 x \leq \frac{16 x + 9}{3} < -8 - 3 x \]The solution is \( \text{None of the above.} \), which is option E.\begin{enumerate}[label=\Alph*.]
\item \( [a, b), \text{ where } a \in [4.25, 7.25] \text{ and } b \in [0.6, 2.3] \)

$[5.25, 1.32)$, which is the correct interval but negatives of the actual endpoints.
\item \( (-\infty, a] \cup (b, \infty), \text{ where } a \in [4.25, 6.25] \text{ and } b \in [0.2, 1.9] \)

$(-\infty, 5.25] \cup (1.32, \infty)$, which corresponds to displaying the and-inequality as an or-inequality and getting negatives of the actual endpoints.
\item \( (a, b], \text{ where } a \in [3.25, 8.25] \text{ and } b \in [0.7, 4.2] \)

$(5.25, 1.32]$, which corresponds to flipping the inequality and getting negatives of the actual endpoints.
\item \( (-\infty, a) \cup [b, \infty), \text{ where } a \in [3.25, 6.25] \text{ and } b \in [1.32, 2.32] \)

$(-\infty, 5.25) \cup [1.32, \infty)$, which corresponds to displaying the and-inequality as an or-inequality AND flipping the inequality AND getting negatives of the actual endpoints.
\item \( \text{None of the above.} \)

* This is correct as the answer should be $[-5.25, -1.32)$.
\end{enumerate}

\textbf{General Comment:} To solve, you will need to break up the compound inequality into two inequalities. Be sure to keep track of the inequality! It may be best to draw a number line and graph your solution.
}
\litem{
Solve the linear inequality below. Then, choose the constant and interval combination that describes the solution set.
\[ -4 + 8 x > 11 x \text{ or } 8 + 6 x < 9 x \]The solution is \( (-\infty, -1.333) \text{ or } (2.667, \infty) \), which is option C.\begin{enumerate}[label=\Alph*.]
\item \( (-\infty, a) \cup (b, \infty), \text{ where } a \in [-3.5, -2.2] \text{ and } b \in [-0.8, 1.4] \)

Corresponds to inverting the inequality and negating the solution.
\item \( (-\infty, a] \cup [b, \infty), \text{ where } a \in [-2.6, 1.4] \text{ and } b \in [1.4, 3.9] \)

Corresponds to including the endpoints (when they should be excluded).
\item \( (-\infty, a) \cup (b, \infty), \text{ where } a \in [-2.3, 0] \text{ and } b \in [2, 4.4] \)

 * Correct option.
\item \( (-\infty, a] \cup [b, \infty), \text{ where } a \in [-2.9, -1.9] \text{ and } b \in [1.3, 2.6] \)

Corresponds to including the endpoints AND negating.
\item \( (-\infty, \infty) \)

Corresponds to the variable canceling, which does not happen in this instance.
\end{enumerate}

\textbf{General Comment:} When multiplying or dividing by a negative, flip the sign.
}
\litem{
Solve the linear inequality below. Then, choose the constant and interval combination that describes the solution set.
\[ -7x -3 \leq -4x + 9 \]The solution is \( [-4.0, \infty) \), which is option D.\begin{enumerate}[label=\Alph*.]
\item \( [a, \infty), \text{ where } a \in [4, 11] \)

 $[4.0, \infty)$, which corresponds to negating the endpoint of the solution.
\item \( (-\infty, a], \text{ where } a \in [4, 5] \)

 $(-\infty, 4.0]$, which corresponds to switching the direction of the interval AND negating the endpoint. You likely did this if you did not flip the inequality when dividing by a negative as well as not moving values over to a side properly.
\item \( (-\infty, a], \text{ where } a \in [-9, 1] \)

 $(-\infty, -4.0]$, which corresponds to switching the direction of the interval. You likely did this if you did not flip the inequality when dividing by a negative!
\item \( [a, \infty), \text{ where } a \in [-9, -1] \)

* $[-4.0, \infty)$, which is the correct option.
\item \( \text{None of the above}. \)

You may have chosen this if you thought the inequality did not match the ends of the intervals.
\end{enumerate}

\textbf{General Comment:} Remember that less/greater than or equal to includes the endpoint, while less/greater do not. Also, remember that you need to flip the inequality when you multiply or divide by a negative.
}
\end{enumerate}

\end{document}