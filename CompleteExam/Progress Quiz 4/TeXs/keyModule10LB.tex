\documentclass{extbook}[14pt]
\usepackage{multicol, enumerate, enumitem, hyperref, color, soul, setspace, parskip, fancyhdr, amssymb, amsthm, amsmath, bbm, latexsym, units, mathtools}
\everymath{\displaystyle}
\usepackage[headsep=0.5cm,headheight=0cm, left=1 in,right= 1 in,top= 1 in,bottom= 1 in]{geometry}
\usepackage{dashrule}  % Package to use the command below to create lines between items
\newcommand{\litem}[1]{\item #1

\rule{\textwidth}{0.4pt}}
\pagestyle{fancy}
\lhead{}
\chead{Answer Key for Progress Quiz 4 Version B}
\rhead{}
\lfoot{9187-5854}
\cfoot{}
\rfoot{Spring 2021}
\begin{document}
\textbf{This key should allow you to understand why you choose the option you did (beyond just getting a question right or wrong). \href{https://xronos.clas.ufl.edu/mac1105spring2020/courseDescriptionAndMisc/Exams/LearningFromResults}{More instructions on how to use this key can be found here}.}

\textbf{If you have a suggestion to make the keys better, \href{https://forms.gle/CZkbZmPbC9XALEE88}{please fill out the short survey here}.}

\textit{Note: This key is auto-generated and may contain issues and/or errors. The keys are reviewed after each exam to ensure grading is done accurately. If there are issues (like duplicate options), they are noted in the offline gradebook. The keys are a work-in-progress to give students as many resources to improve as possible.}

\rule{\textwidth}{0.4pt}

\begin{enumerate}\litem{
Factor the polynomial below completely. Then, choose the intervals the zeros of the polynomial belong to, where $z_1 \leq z_2 \leq z_3$. \textit{To make the problem easier, all zeros are between -5 and 5.}
\[ f(x) = 15x^{3} +29 x^{2} -8 x -12 \]The solution is \( [-2, -0.6, 0.6666666666666666] \), which is option B.\begin{enumerate}[label=\Alph*.]
\item \( z_1 \in [-1.99, -0.93], \text{   }  z_2 \in [1.5, 1.9], \text{   and   } z_3 \in [1.73, 2.12] \)

 Distractor 3: Corresponds to negatives of all zeros AND inversing rational roots.
\item \( z_1 \in [-2.22, -1.58], \text{   }  z_2 \in [-1.3, -0.2], \text{   and   } z_3 \in [0.38, 0.79] \)

* This is the solution!
\item \( z_1 \in [-2.22, -1.58], \text{   }  z_2 \in [-2.4, -0.7], \text{   and   } z_3 \in [1.34, 1.69] \)

 Distractor 2: Corresponds to inversing rational roots.
\item \( z_1 \in [-1.14, -0.57], \text{   }  z_2 \in [0.4, 0.7], \text{   and   } z_3 \in [1.73, 2.12] \)

 Distractor 1: Corresponds to negatives of all zeros.
\item \( z_1 \in [-0.38, -0.12], \text{   }  z_2 \in [1.9, 2.8], \text{   and   } z_3 \in [2.34, 3.32] \)

 Distractor 4: Corresponds to moving factors from one rational to another.
\end{enumerate}

\textbf{General Comment:} Remember to try the middle-most integers first as these normally are the zeros. Also, once you get it to a quadratic, you can use your other factoring techniques to finish factoring.
}
\litem{
Perform the division below. Then, find the intervals that correspond to the quotient in the form $ax^2+bx+c$ and remainder $r$.
\[ \frac{9x^{3} +33 x^{2} -32 x -84}{x + 4} \]The solution is \( 9x^{2} -3 x -20 + \frac{-4}{x + 4} \), which is option E.\begin{enumerate}[label=\Alph*.]
\item \( a \in [9, 11], \text{   } b \in [69, 73], \text{   } c \in [241, 248], \text{   and   } r \in [885, 895]. \)

 You divided by the opposite of the factor.
\item \( a \in [9, 11], \text{   } b \in [-13, -9], \text{   } c \in [27, 29], \text{   and   } r \in [-226, -213]. \)

 You multiplied by the synthetic number and subtracted rather than adding during synthetic division.
\item \( a \in [-40, -31], \text{   } b \in [-115, -107], \text{   } c \in [-478, -473], \text{   and   } r \in [-1990, -1981]. \)

 You divided by the opposite of the factor AND multiplied the first factor rather than just bringing it down.
\item \( a \in [-40, -31], \text{   } b \in [172, 178], \text{   } c \in [-743, -736], \text{   and   } r \in [2871, 2877]. \)

 You multiplied by the synthetic number rather than bringing the first factor down.
\item \( a \in [9, 11], \text{   } b \in [-3, 4], \text{   } c \in [-22, -18], \text{   and   } r \in [-7, -1]. \)

* This is the solution!
\end{enumerate}

\textbf{General Comment:} Be sure to synthetically divide by the zero of the denominator!
}
\litem{
What are the \textit{possible Rational} roots of the polynomial below?
\[ f(x) = 3x^{2} +4 x + 5 \]The solution is \( \text{ All combinations of: }\frac{\pm 1,\pm 5}{\pm 1,\pm 3} \), which is option C.\begin{enumerate}[label=\Alph*.]
\item \( \pm 1,\pm 5 \)

This would have been the solution \textbf{if asked for the possible Integer roots}!
\item \( \pm 1,\pm 3 \)

 Distractor 1: Corresponds to the plus or minus factors of a1 only.
\item \( \text{ All combinations of: }\frac{\pm 1,\pm 5}{\pm 1,\pm 3} \)

* This is the solution \textbf{since we asked for the possible Rational roots}!
\item \( \text{ All combinations of: }\frac{\pm 1,\pm 3}{\pm 1,\pm 5} \)

 Distractor 3: Corresponds to the plus or minus of the inverse quotient (an/a0) of the factors. 
\item \( \text{ There is no formula or theorem that tells us all possible Rational roots.} \)

 Distractor 4: Corresponds to not recalling the theorem for rational roots of a polynomial.
\end{enumerate}

\textbf{General Comment:} We have a way to find the possible Rational roots. The possible Integer roots are the Integers in this list.
}
\litem{
Factor the polynomial below completely. Then, choose the intervals the zeros of the polynomial belong to, where $z_1 \leq z_2 \leq z_3$. \textit{To make the problem easier, all zeros are between -5 and 5.}
\[ f(x) = 15x^{3} -59 x^{2} -10 x + 24 \]The solution is \( [-0.6666666666666666, 0.6, 4] \), which is option A.\begin{enumerate}[label=\Alph*.]
\item \( z_1 \in [-1.14, 0.07], \text{   }  z_2 \in [0.17, 0.63], \text{   and   } z_3 \in [3.67, 4.06] \)

* This is the solution!
\item \( z_1 \in [-4.12, -3.78], \text{   }  z_2 \in [-2, -1.4], \text{   and   } z_3 \in [1.48, 1.6] \)

 Distractor 3: Corresponds to negatives of all zeros AND inversing rational roots.
\item \( z_1 \in [-4.12, -3.78], \text{   }  z_2 \in [-0.58, -0.1], \text{   and   } z_3 \in [1.82, 2.21] \)

 Distractor 4: Corresponds to moving factors from one rational to another.
\item \( z_1 \in [-1.66, -1.07], \text{   }  z_2 \in [1.16, 1.88], \text{   and   } z_3 \in [3.67, 4.06] \)

 Distractor 2: Corresponds to inversing rational roots.
\item \( z_1 \in [-4.12, -3.78], \text{   }  z_2 \in [-0.9, -0.39], \text{   and   } z_3 \in [-0.04, 0.82] \)

 Distractor 1: Corresponds to negatives of all zeros.
\end{enumerate}

\textbf{General Comment:} Remember to try the middle-most integers first as these normally are the zeros. Also, once you get it to a quadratic, you can use your other factoring techniques to finish factoring.
}
\litem{
Perform the division below. Then, find the intervals that correspond to the quotient in the form $ax^2+bx+c$ and remainder $r$.
\[ \frac{15x^{3} +62 x^{2} -36}{x + 4} \]The solution is \( 15x^{2} +2 x -8 + \frac{-4}{x + 4} \), which is option D.\begin{enumerate}[label=\Alph*.]
\item \( a \in [13, 23], b \in [-13, -8], c \in [65, 70], \text{ and } r \in [-361, -360]. \)

 You multipled by the synthetic number and subtracted rather than adding during synthetic division.
\item \( a \in [-65, -55], b \in [298, 305], c \in [-1214, -1206], \text{ and } r \in [4794, 4797]. \)

 You multipled by the synthetic number rather than bringing the first factor down.
\item \( a \in [13, 23], b \in [120, 125], c \in [487, 489], \text{ and } r \in [1916, 1918]. \)

 You divided by the opposite of the factor.
\item \( a \in [13, 23], b \in [0, 7], c \in [-16, -2], \text{ and } r \in [-7, 0]. \)

* This is the solution!
\item \( a \in [-65, -55], b \in [-181, -177], c \in [-712, -710], \text{ and } r \in [-2886, -2876]. \)

 You divided by the opposite of the factor AND multipled the first factor rather than just bringing it down.
\end{enumerate}

\textbf{General Comment:} Be sure to synthetically divide by the zero of the denominator! Also, make sure to include 0 placeholders for missing terms.
}
\litem{
Perform the division below. Then, find the intervals that correspond to the quotient in the form $ax^2+bx+c$ and remainder $r$.
\[ \frac{15x^{3} +25 x^{2} -20 x -18}{x + 2} \]The solution is \( 15x^{2} -5 x -10 + \frac{2}{x + 2} \), which is option E.\begin{enumerate}[label=\Alph*.]
\item \( a \in [-36, -24], \text{   } b \in [-38, -30], \text{   } c \in [-91, -88], \text{   and   } r \in [-198, -194]. \)

 You divided by the opposite of the factor AND multiplied the first factor rather than just bringing it down.
\item \( a \in [-36, -24], \text{   } b \in [85, 88], \text{   } c \in [-192, -185], \text{   and   } r \in [361, 363]. \)

 You multiplied by the synthetic number rather than bringing the first factor down.
\item \( a \in [15, 19], \text{   } b \in [-25, -15], \text{   } c \in [38, 45], \text{   and   } r \in [-145, -132]. \)

 You multiplied by the synthetic number and subtracted rather than adding during synthetic division.
\item \( a \in [15, 19], \text{   } b \in [51, 57], \text{   } c \in [85, 91], \text{   and   } r \in [159, 165]. \)

 You divided by the opposite of the factor.
\item \( a \in [15, 19], \text{   } b \in [-9, -2], \text{   } c \in [-15, -6], \text{   and   } r \in [-2, 5]. \)

* This is the solution!
\end{enumerate}

\textbf{General Comment:} Be sure to synthetically divide by the zero of the denominator!
}
\litem{
What are the \textit{possible Rational} roots of the polynomial below?
\[ f(x) = 4x^{4} +7 x^{3} +2 x^{2} +3 x + 2 \]The solution is \( \text{ All combinations of: }\frac{\pm 1,\pm 2}{\pm 1,\pm 2,\pm 4} \), which is option C.\begin{enumerate}[label=\Alph*.]
\item \( \pm 1,\pm 2 \)

This would have been the solution \textbf{if asked for the possible Integer roots}!
\item \( \text{ All combinations of: }\frac{\pm 1,\pm 2,\pm 4}{\pm 1,\pm 2} \)

 Distractor 3: Corresponds to the plus or minus of the inverse quotient (an/a0) of the factors. 
\item \( \text{ All combinations of: }\frac{\pm 1,\pm 2}{\pm 1,\pm 2,\pm 4} \)

* This is the solution \textbf{since we asked for the possible Rational roots}!
\item \( \pm 1,\pm 2,\pm 4 \)

 Distractor 1: Corresponds to the plus or minus factors of a1 only.
\item \( \text{ There is no formula or theorem that tells us all possible Rational roots.} \)

 Distractor 4: Corresponds to not recalling the theorem for rational roots of a polynomial.
\end{enumerate}

\textbf{General Comment:} We have a way to find the possible Rational roots. The possible Integer roots are the Integers in this list.
}
\litem{
Factor the polynomial below completely, knowing that $x-3$ is a factor. Then, choose the intervals the zeros of the polynomial belong to, where $z_1 \leq z_2 \leq z_3 \leq z_4$. \textit{To make the problem easier, all zeros are between -5 and 5.}
\[ f(x) = 9x^{4} +27 x^{3} -61 x^{2} -243 x -180 \]The solution is \( [-3, -1.6666666666666667, -1.3333333333333333, 3] \), which is option C.\begin{enumerate}[label=\Alph*.]
\item \( z_1 \in [-5, -2], \text{   }  z_2 \in [1.28, 1.34], z_3 \in [1.25, 2.2], \text{   and   } z_4 \in [2, 3.1] \)

 Distractor 1: Corresponds to negatives of all zeros.
\item \( z_1 \in [-5, -2], \text{   }  z_2 \in [0.53, 0.58], z_3 \in [2.1, 3.67], \text{   and   } z_4 \in [3.7, 4.4] \)

 Distractor 4: Corresponds to moving factors from one rational to another.
\item \( z_1 \in [-5, -2], \text{   }  z_2 \in [-1.67, -1.62], z_3 \in [-1.85, -0.77], \text{   and   } z_4 \in [2, 3.1] \)

* This is the solution!
\item \( z_1 \in [-5, -2], \text{   }  z_2 \in [0.6, 0.65], z_3 \in [-0.08, 1.5], \text{   and   } z_4 \in [2, 3.1] \)

 Distractor 3: Corresponds to negatives of all zeros AND inversing rational roots.
\item \( z_1 \in [-5, -2], \text{   }  z_2 \in [-0.76, -0.75], z_3 \in [-0.84, -0.43], \text{   and   } z_4 \in [2, 3.1] \)

 Distractor 2: Corresponds to inversing rational roots.
\end{enumerate}

\textbf{General Comment:} Remember to try the middle-most integers first as these normally are the zeros. Also, once you get it to a quadratic, you can use your other factoring techniques to finish factoring.
}
\litem{
Factor the polynomial below completely, knowing that $x+3$ is a factor. Then, choose the intervals the zeros of the polynomial belong to, where $z_1 \leq z_2 \leq z_3 \leq z_4$. \textit{To make the problem easier, all zeros are between -5 and 5.}
\[ f(x) = 15x^{4} -13 x^{3} -155 x^{2} +117 x + 180 \]The solution is \( [-3, -0.8, 1.6666666666666667, 3] \), which is option E.\begin{enumerate}[label=\Alph*.]
\item \( z_1 \in [-4, 0], \text{   }  z_2 \in [-1.3, -1.22], z_3 \in [0.58, 0.63], \text{   and   } z_4 \in [1, 7] \)

 Distractor 2: Corresponds to inversing rational roots.
\item \( z_1 \in [-4, 0], \text{   }  z_2 \in [-0.62, -0.45], z_3 \in [1.11, 1.26], \text{   and   } z_4 \in [1, 7] \)

 Distractor 3: Corresponds to negatives of all zeros AND inversing rational roots.
\item \( z_1 \in [-5, -4], \text{   }  z_2 \in [-3.13, -2.79], z_3 \in [0.04, 0.31], \text{   and   } z_4 \in [1, 7] \)

 Distractor 4: Corresponds to moving factors from one rational to another.
\item \( z_1 \in [-4, 0], \text{   }  z_2 \in [-1.77, -1.51], z_3 \in [0.76, 0.97], \text{   and   } z_4 \in [1, 7] \)

 Distractor 1: Corresponds to negatives of all zeros.
\item \( z_1 \in [-4, 0], \text{   }  z_2 \in [-1.12, -0.75], z_3 \in [1.44, 1.72], \text{   and   } z_4 \in [1, 7] \)

* This is the solution!
\end{enumerate}

\textbf{General Comment:} Remember to try the middle-most integers first as these normally are the zeros. Also, once you get it to a quadratic, you can use your other factoring techniques to finish factoring.
}
\litem{
Perform the division below. Then, find the intervals that correspond to the quotient in the form $ax^2+bx+c$ and remainder $r$.
\[ \frac{4x^{3} -49 x -65}{x -4} \]The solution is \( 4x^{2} +16 x + 15 + \frac{-5}{x -4} \), which is option A.\begin{enumerate}[label=\Alph*.]
\item \( a \in [3, 9], b \in [15, 17.3], c \in [12, 22], \text{ and } r \in [-5, -3]. \)

* This is the solution!
\item \( a \in [10, 24], b \in [-65.5, -61.7], c \in [206, 213], \text{ and } r \in [-897, -884]. \)

 You divided by the opposite of the factor AND multipled the first factor rather than just bringing it down.
\item \( a \in [3, 9], b \in [-16.7, -14.7], c \in [12, 22], \text{ and } r \in [-126, -123]. \)

 You divided by the opposite of the factor.
\item \( a \in [10, 24], b \in [62.6, 66.2], c \in [206, 213], \text{ and } r \in [759, 766]. \)

 You multipled by the synthetic number rather than bringing the first factor down.
\item \( a \in [3, 9], b \in [11, 13], c \in [-17, -5], \text{ and } r \in [-105, -102]. \)

 You multipled by the synthetic number and subtracted rather than adding during synthetic division.
\end{enumerate}

\textbf{General Comment:} Be sure to synthetically divide by the zero of the denominator! Also, make sure to include 0 placeholders for missing terms.
}
\end{enumerate}

\end{document}