\documentclass{extbook}[14pt]
\usepackage{multicol, enumerate, enumitem, hyperref, color, soul, setspace, parskip, fancyhdr, amssymb, amsthm, amsmath, bbm, latexsym, units, mathtools}
\everymath{\displaystyle}
\usepackage[headsep=0.5cm,headheight=0cm, left=1 in,right= 1 in,top= 1 in,bottom= 1 in]{geometry}
\usepackage{dashrule}  % Package to use the command below to create lines between items
\newcommand{\litem}[1]{\item #1

\rule{\textwidth}{0.4pt}}
\pagestyle{fancy}
\lhead{}
\chead{Answer Key for Progress Quiz 4 Version A}
\rhead{}
\lfoot{4378-7085}
\cfoot{}
\rfoot{Fall 2020}
\begin{document}
\textbf{This key should allow you to understand why you choose the option you did (beyond just getting a question right or wrong). \href{https://xronos.clas.ufl.edu/mac1105spring2020/courseDescriptionAndMisc/Exams/LearningFromResults}{More instructions on how to use this key can be found here}.}

\textbf{If you have a suggestion to make the keys better, \href{https://forms.gle/CZkbZmPbC9XALEE88}{please fill out the short survey here}.}

\textit{Note: This key is auto-generated and may contain issues and/or errors. The keys are reviewed after each exam to ensure grading is done accurately. If there are issues (like duplicate options), they are noted in the offline gradebook. The keys are a work-in-progress to give students as many resources to improve as possible.}

\rule{\textwidth}{0.4pt}

\begin{enumerate}\litem{
Solve the linear inequality below. Then, choose the constant and interval combination that describes the solution set.
\[ -5x + 4 \leq 4x -10 \]
The solution is \( [1.556, \infty) \), which is option C.\begin{enumerate}[label=\Alph*.]
\item \( (-\infty, a], \text{ where } a \in [0.56, 2.56] \)

 $(-\infty, 1.556]$, which corresponds to switching the direction of the interval. You likely did this if you did not flip the inequality when dividing by a negative!
\item \( [a, \infty), \text{ where } a \in [-3.4, -0.1] \)

 $[-1.556, \infty)$, which corresponds to negating the endpoint of the solution.
\item \( [a, \infty), \text{ where } a \in [0.7, 4.6] \)

* $[1.556, \infty)$, which is the correct option.
\item \( (-\infty, a], \text{ where } a \in [-11.56, 1.44] \)

 $(-\infty, -1.556]$, which corresponds to switching the direction of the interval AND negating the endpoint. You likely did this if you did not flip the inequality when dividing by a negative as well as not moving values over to a side properly.
\item \( \text{None of the above}. \)

You may have chosen this if you thought the inequality did not match the ends of the intervals.
\end{enumerate}

\textbf{General Comment:} Remember that less/greater than or equal to includes the endpoint, while less/greater do not. Also, remember that you need to flip the inequality when you multiply or divide by a negative.
}
\litem{
Solve the linear inequality below. Then, choose the constant and interval combination that describes the solution set.
\[ \frac{3}{9} - \frac{4}{8} x \leq \frac{5}{5} x + \frac{9}{4} \]
The solution is \( [-1.278, \infty) \), which is option B.\begin{enumerate}[label=\Alph*.]
\item \( (-\infty, a], \text{ where } a \in [-2.28, 0.72] \)

 $(-\infty, -1.278]$, which corresponds to switching the direction of the interval. You likely did this if you did not flip the inequality when dividing by a negative!
\item \( [a, \infty), \text{ where } a \in [-2.5, -1] \)

* $[-1.278, \infty)$, which is the correct option.
\item \( (-\infty, a], \text{ where } a \in [0.28, 4.28] \)

 $(-\infty, 1.278]$, which corresponds to switching the direction of the interval AND negating the endpoint. You likely did this if you did not flip the inequality when dividing by a negative as well as not moving values over to a side properly.
\item \( [a, \infty), \text{ where } a \in [-0.6, 1.5] \)

 $[1.278, \infty)$, which corresponds to negating the endpoint of the solution.
\item \( \text{None of the above}. \)

You may have chosen this if you thought the inequality did not match the ends of the intervals.
\end{enumerate}

\textbf{General Comment:} Remember that less/greater than or equal to includes the endpoint, while less/greater do not. Also, remember that you need to flip the inequality when you multiply or divide by a negative.
}
\litem{
Solve the linear inequality below. Then, choose the constant and interval combination that describes the solution set.
\[ -8x + 5 < 4x + 4 \]
The solution is \( (0.083, \infty) \), which is option B.\begin{enumerate}[label=\Alph*.]
\item \( (a, \infty), \text{ where } a \in [-0.14, -0.02] \)

 $(-0.083, \infty)$, which corresponds to negating the endpoint of the solution.
\item \( (a, \infty), \text{ where } a \in [0.05, 0.17] \)

* $(0.083, \infty)$, which is the correct option.
\item \( (-\infty, a), \text{ where } a \in [-0.36, -0.06] \)

 $(-\infty, -0.083)$, which corresponds to switching the direction of the interval AND negating the endpoint. You likely did this if you did not flip the inequality when dividing by a negative as well as not moving values over to a side properly.
\item \( (-\infty, a), \text{ where } a \in [-0.03, 0.41] \)

 $(-\infty, 0.083)$, which corresponds to switching the direction of the interval. You likely did this if you did not flip the inequality when dividing by a negative!
\item \( \text{None of the above}. \)

You may have chosen this if you thought the inequality did not match the ends of the intervals.
\end{enumerate}

\textbf{General Comment:} Remember that less/greater than or equal to includes the endpoint, while less/greater do not. Also, remember that you need to flip the inequality when you multiply or divide by a negative.
}
\litem{
Solve the linear inequality below. Then, choose the constant and interval combination that describes the solution set.
\[ 7 - 9 x < \frac{-20 x + 4}{5} \leq 7 - 5 x \]
The solution is \( \text{None of the above.} \), which is option E.\begin{enumerate}[label=\Alph*.]
\item \( (-\infty, a] \cup (b, \infty), \text{ where } a \in [-1.24, -0.24] \text{ and } b \in [-6.2, -5.2] \)

$(-\infty, -1.24] \cup (-6.20, \infty)$, which corresponds to displaying the and-inequality as an or-inequality AND flipping the inequality AND getting negatives of the actual endpoints.
\item \( [a, b), \text{ where } a \in [-6.24, -0.24] \text{ and } b \in [-9.2, -5.2] \)

$[-1.24, -6.20)$, which corresponds to flipping the inequality and getting negatives of the actual endpoints.
\item \( (-\infty, a) \cup [b, \infty), \text{ where } a \in [-2.6, 0.2] \text{ and } b \in [-9.2, -5.2] \)

$(-\infty, -1.24) \cup [-6.20, \infty)$, which corresponds to displaying the and-inequality as an or-inequality and getting negatives of the actual endpoints.
\item \( (a, b], \text{ where } a \in [-1.7, 0.2] \text{ and } b \in [-8.2, -5.2] \)

$(-1.24, -6.20]$, which is the correct interval but negatives of the actual endpoints.
\item \( \text{None of the above.} \)

* This is correct as the answer should be $(1.24, 6.20]$.
\end{enumerate}

\textbf{General Comment:} To solve, you will need to break up the compound inequality into two inequalities. Be sure to keep track of the inequality! It may be best to draw a number line and graph your solution.
}
\litem{
Solve the linear inequality below. Then, choose the constant and interval combination that describes the solution set.
\[ \frac{-5}{9} - \frac{4}{8} x \leq \frac{4}{4} x + \frac{8}{3} \]
The solution is \( [-2.148, \infty) \), which is option A.\begin{enumerate}[label=\Alph*.]
\item \( [a, \infty), \text{ where } a \in [-4.15, 1.85] \)

* $[-2.148, \infty)$, which is the correct option.
\item \( (-\infty, a], \text{ where } a \in [-2.15, -1.15] \)

 $(-\infty, -2.148]$, which corresponds to switching the direction of the interval. You likely did this if you did not flip the inequality when dividing by a negative!
\item \( [a, \infty), \text{ where } a \in [1.15, 6.15] \)

 $[2.148, \infty)$, which corresponds to negating the endpoint of the solution.
\item \( (-\infty, a], \text{ where } a \in [0.15, 5.15] \)

 $(-\infty, 2.148]$, which corresponds to switching the direction of the interval AND negating the endpoint. You likely did this if you did not flip the inequality when dividing by a negative as well as not moving values over to a side properly.
\item \( \text{None of the above}. \)

You may have chosen this if you thought the inequality did not match the ends of the intervals.
\end{enumerate}

\textbf{General Comment:} Remember that less/greater than or equal to includes the endpoint, while less/greater do not. Also, remember that you need to flip the inequality when you multiply or divide by a negative.
}
\litem{
Using an interval or intervals, describe all the $x$-values within or including a distance of the given values.
\[ \text{ More than } 6 \text{ units from the number } -3. \]
The solution is \( (-\infty, -9) \cup (3, \infty) \), which is option D.\begin{enumerate}[label=\Alph*.]
\item \( (-9, 3) \)

This describes the values less than 6 from -3
\item \( [-9, 3] \)

This describes the values no more than 6 from -3
\item \( (-\infty, -9] \cup [3, \infty) \)

This describes the values no less than 6 from -3
\item \( (-\infty, -9) \cup (3, \infty) \)

This describes the values more than 6 from -3
\item \( \text{None of the above} \)

You likely thought the values in the interval were not correct.
\end{enumerate}

\textbf{General Comment:} When thinking about this language, it helps to draw a number line and try points.
}
\litem{
Solve the linear inequality below. Then, choose the constant and interval combination that describes the solution set.
\[ -6 - 5 x \leq \frac{-25 x - 5}{6} < -7 - 6 x \]
The solution is \( [-6.20, -3.36) \), which is option B.\begin{enumerate}[label=\Alph*.]
\item \( (a, b], \text{ where } a \in [-9.2, -1.2] \text{ and } b \in [-8.36, -2.36] \)

$(-6.20, -3.36]$, which corresponds to flipping the inequality.
\item \( [a, b), \text{ where } a \in [-8.2, -4.2] \text{ and } b \in [-6.36, 2.64] \)

$[-6.20, -3.36)$, which is the correct option.
\item \( (-\infty, a) \cup [b, \infty), \text{ where } a \in [-6.2, -5.2] \text{ and } b \in [-8.36, 1.64] \)

$(-\infty, -6.20) \cup [-3.36, \infty)$, which corresponds to displaying the and-inequality as an or-inequality AND flipping the inequality.
\item \( (-\infty, a] \cup (b, \infty), \text{ where } a \in [-6.2, -4.2] \text{ and } b \in [-5.36, -1.36] \)

$(-\infty, -6.20] \cup (-3.36, \infty)$, which corresponds to displaying the and-inequality as an or-inequality.
\item \( \text{None of the above.} \)


\end{enumerate}

\textbf{General Comment:} To solve, you will need to break up the compound inequality into two inequalities. Be sure to keep track of the inequality! It may be best to draw a number line and graph your solution.
}
\litem{
Solve the linear inequality below. Then, choose the constant and interval combination that describes the solution set.
\[ 8 + 5 x > 6 x \text{ or } 9 + 4 x < 5 x \]
The solution is \( (-\infty, 8.0) \text{ or } (9.0, \infty) \), which is option D.\begin{enumerate}[label=\Alph*.]
\item \( (-\infty, a] \cup [b, \infty), \text{ where } a \in [8, 11] \text{ and } b \in [6, 12] \)

Corresponds to including the endpoints (when they should be excluded).
\item \( (-\infty, a) \cup (b, \infty), \text{ where } a \in [-12, -6] \text{ and } b \in [-10, -6] \)

Corresponds to inverting the inequality and negating the solution.
\item \( (-\infty, a] \cup [b, \infty), \text{ where } a \in [-10, -7] \text{ and } b \in [-9, -7] \)

Corresponds to including the endpoints AND negating.
\item \( (-\infty, a) \cup (b, \infty), \text{ where } a \in [6, 9] \text{ and } b \in [8, 13] \)

 * Correct option.
\item \( (-\infty, \infty) \)

Corresponds to the variable canceling, which does not happen in this instance.
\end{enumerate}

\textbf{General Comment:} When multiplying or dividing by a negative, flip the sign.
}
\litem{
Using an interval or intervals, describe all the $x$-values within or including a distance of the given values.
\[ \text{ More than } 3 \text{ units from the number } 4. \]
The solution is \( (-\infty, 1) \cup (7, \infty) \), which is option C.\begin{enumerate}[label=\Alph*.]
\item \( [1, 7] \)

This describes the values no more than 3 from 4
\item \( (-\infty, 1] \cup [7, \infty) \)

This describes the values no less than 3 from 4
\item \( (-\infty, 1) \cup (7, \infty) \)

This describes the values more than 3 from 4
\item \( (1, 7) \)

This describes the values less than 3 from 4
\item \( \text{None of the above} \)

You likely thought the values in the interval were not correct.
\end{enumerate}

\textbf{General Comment:} When thinking about this language, it helps to draw a number line and try points.
}
\litem{
Solve the linear inequality below. Then, choose the constant and interval combination that describes the solution set.
\[ -8 + 7 x > 9 x \text{ or } 6 + 9 x < 10 x \]
The solution is \( (-\infty, -4.0) \text{ or } (6.0, \infty) \), which is option C.\begin{enumerate}[label=\Alph*.]
\item \( (-\infty, a] \cup [b, \infty), \text{ where } a \in [-4.71, -3.7] \text{ and } b \in [5.3, 6.9] \)

Corresponds to including the endpoints (when they should be excluded).
\item \( (-\infty, a] \cup [b, \infty), \text{ where } a \in [-7.28, -5.67] \text{ and } b \in [2.8, 4.2] \)

Corresponds to including the endpoints AND negating.
\item \( (-\infty, a) \cup (b, \infty), \text{ where } a \in [-4.3, -3.4] \text{ and } b \in [5.6, 6.4] \)

 * Correct option.
\item \( (-\infty, a) \cup (b, \infty), \text{ where } a \in [-6.1, -5.6] \text{ and } b \in [2.4, 5.7] \)

Corresponds to inverting the inequality and negating the solution.
\item \( (-\infty, \infty) \)

Corresponds to the variable canceling, which does not happen in this instance.
\end{enumerate}

\textbf{General Comment:} When multiplying or dividing by a negative, flip the sign.
}
\end{enumerate}

\end{document}