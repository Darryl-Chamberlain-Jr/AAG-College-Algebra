\documentclass{extbook}[14pt]
\usepackage{multicol, enumerate, enumitem, hyperref, color, soul, setspace, parskip, fancyhdr, amssymb, amsthm, amsmath, bbm, latexsym, units, mathtools}
\everymath{\displaystyle}
\usepackage[headsep=0.5cm,headheight=0cm, left=1 in,right= 1 in,top= 1 in,bottom= 1 in]{geometry}
\usepackage{dashrule}  % Package to use the command below to create lines between items
\newcommand{\litem}[1]{\item #1

\rule{\textwidth}{0.4pt}}
\pagestyle{fancy}
\lhead{}
\chead{Answer Key for Progress Quiz 4 Version A}
\rhead{}
\lfoot{9187-5854}
\cfoot{}
\rfoot{Spring 2021}
\begin{document}
\textbf{This key should allow you to understand why you choose the option you did (beyond just getting a question right or wrong). \href{https://xronos.clas.ufl.edu/mac1105spring2020/courseDescriptionAndMisc/Exams/LearningFromResults}{More instructions on how to use this key can be found here}.}

\textbf{If you have a suggestion to make the keys better, \href{https://forms.gle/CZkbZmPbC9XALEE88}{please fill out the short survey here}.}

\textit{Note: This key is auto-generated and may contain issues and/or errors. The keys are reviewed after each exam to ensure grading is done accurately. If there are issues (like duplicate options), they are noted in the offline gradebook. The keys are a work-in-progress to give students as many resources to improve as possible.}

\rule{\textwidth}{0.4pt}

\begin{enumerate}\litem{
Solve the linear inequality below. Then, choose the constant and interval combination that describes the solution set.
\[ -5 - 3 x < \frac{-13 x - 3}{6} \leq 9 - 3 x \]The solution is \( \text{None of the above.} \), which is option E.\begin{enumerate}[label=\Alph*.]
\item \( (a, b], \text{ where } a \in [0.4, 9.4] \text{ and } b \in [-12.4, -7.4] \)

$(5.40, -11.40]$, which is the correct interval but negatives of the actual endpoints.
\item \( (-\infty, a] \cup (b, \infty), \text{ where } a \in [5.4, 9.4] \text{ and } b \in [-11.4, -9.4] \)

$(-\infty, 5.40] \cup (-11.40, \infty)$, which corresponds to displaying the and-inequality as an or-inequality AND flipping the inequality AND getting negatives of the actual endpoints.
\item \( [a, b), \text{ where } a \in [2.4, 7.4] \text{ and } b \in [-12.4, -6.4] \)

$[5.40, -11.40)$, which corresponds to flipping the inequality and getting negatives of the actual endpoints.
\item \( (-\infty, a) \cup [b, \infty), \text{ where } a \in [4.4, 7.4] \text{ and } b \in [-11.4, -10.4] \)

$(-\infty, 5.40) \cup [-11.40, \infty)$, which corresponds to displaying the and-inequality as an or-inequality and getting negatives of the actual endpoints.
\item \( \text{None of the above.} \)

* This is correct as the answer should be $(-5.40, 11.40]$.
\end{enumerate}

\textbf{General Comment:} To solve, you will need to break up the compound inequality into two inequalities. Be sure to keep track of the inequality! It may be best to draw a number line and graph your solution.
}
\litem{
Solve the linear inequality below. Then, choose the constant and interval combination that describes the solution set.
\[ \frac{6}{3} - \frac{8}{8} x > \frac{-3}{5} x - \frac{9}{6} \]The solution is \( (-\infty, 8.75) \), which is option D.\begin{enumerate}[label=\Alph*.]
\item \( (a, \infty), \text{ where } a \in [8.75, 9.75] \)

 $(8.75, \infty)$, which corresponds to switching the direction of the interval. You likely did this if you did not flip the inequality when dividing by a negative!
\item \( (a, \infty), \text{ where } a \in [-8.75, -5.75] \)

 $(-8.75, \infty)$, which corresponds to switching the direction of the interval AND negating the endpoint. You likely did this if you did not flip the inequality when dividing by a negative as well as not moving values over to a side properly.
\item \( (-\infty, a), \text{ where } a \in [-10.75, -7.75] \)

 $(-\infty, -8.75)$, which corresponds to negating the endpoint of the solution.
\item \( (-\infty, a), \text{ where } a \in [6.75, 10.75] \)

* $(-\infty, 8.75)$, which is the correct option.
\item \( \text{None of the above}. \)

You may have chosen this if you thought the inequality did not match the ends of the intervals.
\end{enumerate}

\textbf{General Comment:} Remember that less/greater than or equal to includes the endpoint, while less/greater do not. Also, remember that you need to flip the inequality when you multiply or divide by a negative.
}
\litem{
Solve the linear inequality below. Then, choose the constant and interval combination that describes the solution set.
\[ 7 + 8 x > 9 x \text{ or } 8 + 4 x < 5 x \]The solution is \( (-\infty, 7.0) \text{ or } (8.0, \infty) \), which is option B.\begin{enumerate}[label=\Alph*.]
\item \( (-\infty, a] \cup [b, \infty), \text{ where } a \in [-8, -6] \text{ and } b \in [-8, -4] \)

Corresponds to including the endpoints AND negating.
\item \( (-\infty, a) \cup (b, \infty), \text{ where } a \in [3, 8] \text{ and } b \in [8, 11] \)

 * Correct option.
\item \( (-\infty, a) \cup (b, \infty), \text{ where } a \in [-8, -5] \text{ and } b \in [-9, -1] \)

Corresponds to inverting the inequality and negating the solution.
\item \( (-\infty, a] \cup [b, \infty), \text{ where } a \in [4, 9] \text{ and } b \in [8, 11] \)

Corresponds to including the endpoints (when they should be excluded).
\item \( (-\infty, \infty) \)

Corresponds to the variable canceling, which does not happen in this instance.
\end{enumerate}

\textbf{General Comment:} When multiplying or dividing by a negative, flip the sign.
}
\litem{
Solve the linear inequality below. Then, choose the constant and interval combination that describes the solution set.
\[ 8x + 5 \geq 10x + 3 \]The solution is \( (-\infty, 1.0] \), which is option C.\begin{enumerate}[label=\Alph*.]
\item \( (-\infty, a], \text{ where } a \in [-2.8, 0.7] \)

 $(-\infty, -1.0]$, which corresponds to negating the endpoint of the solution.
\item \( [a, \infty), \text{ where } a \in [-0.7, 3.3] \)

 $[1.0, \infty)$, which corresponds to switching the direction of the interval. You likely did this if you did not flip the inequality when dividing by a negative!
\item \( (-\infty, a], \text{ where } a \in [-0.6, 5.4] \)

* $(-\infty, 1.0]$, which is the correct option.
\item \( [a, \infty), \text{ where } a \in [-1.6, 0.9] \)

 $[-1.0, \infty)$, which corresponds to switching the direction of the interval AND negating the endpoint. You likely did this if you did not flip the inequality when dividing by a negative as well as not moving values over to a side properly.
\item \( \text{None of the above}. \)

You may have chosen this if you thought the inequality did not match the ends of the intervals.
\end{enumerate}

\textbf{General Comment:} Remember that less/greater than or equal to includes the endpoint, while less/greater do not. Also, remember that you need to flip the inequality when you multiply or divide by a negative.
}
\litem{
Using an interval or intervals, describe all the $x$-values within or including a distance of the given values.
\[ \text{ No less than } 6 \text{ units from the number } -3. \]The solution is \( (-\infty, -9] \cup [3, \infty) \), which is option A.\begin{enumerate}[label=\Alph*.]
\item \( (-\infty, -9] \cup [3, \infty) \)

This describes the values no less than 6 from -3
\item \( [-9, 3] \)

This describes the values no more than 6 from -3
\item \( (-9, 3) \)

This describes the values less than 6 from -3
\item \( (-\infty, -9) \cup (3, \infty) \)

This describes the values more than 6 from -3
\item \( \text{None of the above} \)

You likely thought the values in the interval were not correct.
\end{enumerate}

\textbf{General Comment:} When thinking about this language, it helps to draw a number line and try points.
}
\litem{
Solve the linear inequality below. Then, choose the constant and interval combination that describes the solution set.
\[ \frac{8}{8} - \frac{8}{4} x \geq \frac{-7}{5} x - \frac{6}{9} \]The solution is \( (-\infty, 2.778] \), which is option A.\begin{enumerate}[label=\Alph*.]
\item \( (-\infty, a], \text{ where } a \in [-1.22, 3.78] \)

* $(-\infty, 2.778]$, which is the correct option.
\item \( (-\infty, a], \text{ where } a \in [-4.78, 0.22] \)

 $(-\infty, -2.778]$, which corresponds to negating the endpoint of the solution.
\item \( [a, \infty), \text{ where } a \in [-3.78, 0.22] \)

 $[-2.778, \infty)$, which corresponds to switching the direction of the interval AND negating the endpoint. You likely did this if you did not flip the inequality when dividing by a negative as well as not moving values over to a side properly.
\item \( [a, \infty), \text{ where } a \in [1.78, 5.78] \)

 $[2.778, \infty)$, which corresponds to switching the direction of the interval. You likely did this if you did not flip the inequality when dividing by a negative!
\item \( \text{None of the above}. \)

You may have chosen this if you thought the inequality did not match the ends of the intervals.
\end{enumerate}

\textbf{General Comment:} Remember that less/greater than or equal to includes the endpoint, while less/greater do not. Also, remember that you need to flip the inequality when you multiply or divide by a negative.
}
\litem{
Using an interval or intervals, describe all the $x$-values within or including a distance of the given values.
\[ \text{ No less than } 9 \text{ units from the number } 5. \]The solution is \( (-\infty, -4] \cup [14, \infty) \), which is option D.\begin{enumerate}[label=\Alph*.]
\item \( (-\infty, -4) \cup (14, \infty) \)

This describes the values more than 9 from 5
\item \( [-4, 14] \)

This describes the values no more than 9 from 5
\item \( (-4, 14) \)

This describes the values less than 9 from 5
\item \( (-\infty, -4] \cup [14, \infty) \)

This describes the values no less than 9 from 5
\item \( \text{None of the above} \)

You likely thought the values in the interval were not correct.
\end{enumerate}

\textbf{General Comment:} When thinking about this language, it helps to draw a number line and try points.
}
\litem{
Solve the linear inequality below. Then, choose the constant and interval combination that describes the solution set.
\[ -7 + 8 x < \frac{66 x + 4}{8} \leq 3 + 8 x \]The solution is \( \text{None of the above.} \), which is option E.\begin{enumerate}[label=\Alph*.]
\item \( [a, b), \text{ where } a \in [29, 34] \text{ and } b \in [-11, -3] \)

$[30.00, -10.00)$, which corresponds to flipping the inequality and getting negatives of the actual endpoints.
\item \( (-\infty, a] \cup (b, \infty), \text{ where } a \in [30, 34] \text{ and } b \in [-12, -8] \)

$(-\infty, 30.00] \cup (-10.00, \infty)$, which corresponds to displaying the and-inequality as an or-inequality AND flipping the inequality AND getting negatives of the actual endpoints.
\item \( (-\infty, a) \cup [b, \infty), \text{ where } a \in [29, 32] \text{ and } b \in [-13, -4] \)

$(-\infty, 30.00) \cup [-10.00, \infty)$, which corresponds to displaying the and-inequality as an or-inequality and getting negatives of the actual endpoints.
\item \( (a, b], \text{ where } a \in [30, 31] \text{ and } b \in [-12, -7] \)

$(30.00, -10.00]$, which is the correct interval but negatives of the actual endpoints.
\item \( \text{None of the above.} \)

* This is correct as the answer should be $(-30.00, 10.00]$.
\end{enumerate}

\textbf{General Comment:} To solve, you will need to break up the compound inequality into two inequalities. Be sure to keep track of the inequality! It may be best to draw a number line and graph your solution.
}
\litem{
Solve the linear inequality below. Then, choose the constant and interval combination that describes the solution set.
\[ -5 + 4 x > 7 x \text{ or } 9 + 9 x < 12 x \]The solution is \( (-\infty, -1.667) \text{ or } (3.0, \infty) \), which is option B.\begin{enumerate}[label=\Alph*.]
\item \( (-\infty, a] \cup [b, \infty), \text{ where } a \in [-6, -2] \text{ and } b \in [-0.33, 2.67] \)

Corresponds to including the endpoints AND negating.
\item \( (-\infty, a) \cup (b, \infty), \text{ where } a \in [-1.67, 2.33] \text{ and } b \in [2.68, 3.81] \)

 * Correct option.
\item \( (-\infty, a) \cup (b, \infty), \text{ where } a \in [-4, -2] \text{ and } b \in [1.04, 2.06] \)

Corresponds to inverting the inequality and negating the solution.
\item \( (-\infty, a] \cup [b, \infty), \text{ where } a \in [-1.67, 0.33] \text{ and } b \in [2, 6] \)

Corresponds to including the endpoints (when they should be excluded).
\item \( (-\infty, \infty) \)

Corresponds to the variable canceling, which does not happen in this instance.
\end{enumerate}

\textbf{General Comment:} When multiplying or dividing by a negative, flip the sign.
}
\litem{
Solve the linear inequality below. Then, choose the constant and interval combination that describes the solution set.
\[ -9x -7 > 8x -10 \]The solution is \( (-\infty, 0.176) \), which is option C.\begin{enumerate}[label=\Alph*.]
\item \( (a, \infty), \text{ where } a \in [-0.02, 0.51] \)

 $(0.176, \infty)$, which corresponds to switching the direction of the interval. You likely did this if you did not flip the inequality when dividing by a negative!
\item \( (-\infty, a), \text{ where } a \in [-0.46, 0.14] \)

 $(-\infty, -0.176)$, which corresponds to negating the endpoint of the solution.
\item \( (-\infty, a), \text{ where } a \in [0.11, 1.05] \)

* $(-\infty, 0.176)$, which is the correct option.
\item \( (a, \infty), \text{ where } a \in [-0.47, -0.17] \)

 $(-0.176, \infty)$, which corresponds to switching the direction of the interval AND negating the endpoint. You likely did this if you did not flip the inequality when dividing by a negative as well as not moving values over to a side properly.
\item \( \text{None of the above}. \)

You may have chosen this if you thought the inequality did not match the ends of the intervals.
\end{enumerate}

\textbf{General Comment:} Remember that less/greater than or equal to includes the endpoint, while less/greater do not. Also, remember that you need to flip the inequality when you multiply or divide by a negative.
}
\end{enumerate}

\end{document}