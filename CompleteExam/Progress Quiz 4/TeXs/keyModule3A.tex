\documentclass{extbook}[14pt]
\usepackage{multicol, enumerate, enumitem, hyperref, color, soul, setspace, parskip, fancyhdr, amssymb, amsthm, amsmath, bbm, latexsym, units, mathtools}
\everymath{\displaystyle}
\usepackage[headsep=0.5cm,headheight=0cm, left=1 in,right= 1 in,top= 1 in,bottom= 1 in]{geometry}
\usepackage{dashrule}  % Package to use the command below to create lines between items
\newcommand{\litem}[1]{\item #1

\rule{\textwidth}{0.4pt}}
\pagestyle{fancy}
\lhead{}
\chead{Answer Key for Progress Quiz 4 Version A}
\rhead{}
\lfoot{8448-1521}
\cfoot{}
\rfoot{Fall 2020}
\begin{document}
\textbf{This key should allow you to understand why you choose the option you did (beyond just getting a question right or wrong). \href{https://xronos.clas.ufl.edu/mac1105spring2020/courseDescriptionAndMisc/Exams/LearningFromResults}{More instructions on how to use this key can be found here}.}

\textbf{If you have a suggestion to make the keys better, \href{https://forms.gle/CZkbZmPbC9XALEE88}{please fill out the short survey here}.}

\textit{Note: This key is auto-generated and may contain issues and/or errors. The keys are reviewed after each exam to ensure grading is done accurately. If there are issues (like duplicate options), they are noted in the offline gradebook. The keys are a work-in-progress to give students as many resources to improve as possible.}

\rule{\textwidth}{0.4pt}

\begin{enumerate}\litem{
Solve the linear inequality below. Then, choose the constant and interval combination that describes the solution set.
\[ \frac{-6}{4} - \frac{7}{5} x > \frac{3}{6} x + \frac{3}{3} \]
The solution is \( (-\infty, -1.316) \), which is option C.\begin{enumerate}[label=\Alph*.]
\item \( (a, \infty), \text{ where } a \in [0.32, 3.32] \)

 $(1.316, \infty)$, which corresponds to switching the direction of the interval AND negating the endpoint. You likely did this if you did not flip the inequality when dividing by a negative as well as not moving values over to a side properly.
\item \( (-\infty, a), \text{ where } a \in [0.32, 2.32] \)

 $(-\infty, 1.316)$, which corresponds to negating the endpoint of the solution.
\item \( (-\infty, a), \text{ where } a \in [-6.32, 0.68] \)

* $(-\infty, -1.316)$, which is the correct option.
\item \( (a, \infty), \text{ where } a \in [-3.32, -0.32] \)

 $(-1.316, \infty)$, which corresponds to switching the direction of the interval. You likely did this if you did not flip the inequality when dividing by a negative!
\item \( \text{None of the above}. \)

You may have chosen this if you thought the inequality did not match the ends of the intervals.
\end{enumerate}

\textbf{General Comment:} Remember that less/greater than or equal to includes the endpoint, while less/greater do not. Also, remember that you need to flip the inequality when you multiply or divide by a negative.
}
\litem{
Using an interval or intervals, describe all the $x$-values within or including a distance of the given values.
\[ \text{ No less than } 8 \text{ units from the number } -4. \]
The solution is \( (-\infty, -12] \cup [4, \infty) \), which is option B.\begin{enumerate}[label=\Alph*.]
\item \( (-\infty, -12) \cup (4, \infty) \)

This describes the values more than 8 from -4
\item \( (-\infty, -12] \cup [4, \infty) \)

This describes the values no less than 8 from -4
\item \( [-12, 4] \)

This describes the values no more than 8 from -4
\item \( (-12, 4) \)

This describes the values less than 8 from -4
\item \( \text{None of the above} \)

You likely thought the values in the interval were not correct.
\end{enumerate}

\textbf{General Comment:} When thinking about this language, it helps to draw a number line and try points.
}
\litem{
Solve the linear inequality below. Then, choose the constant and interval combination that describes the solution set.
\[ -8 + 9 x < \frac{32 x + 6}{3} \leq -7 + 5 x \]
The solution is \( \text{None of the above.} \), which is option E.\begin{enumerate}[label=\Alph*.]
\item \( [a, b), \text{ where } a \in [4, 8] \text{ and } b \in [-1.41, 4.59] \)

$[6.00, 1.59)$, which corresponds to flipping the inequality and getting negatives of the actual endpoints.
\item \( (a, b], \text{ where } a \in [5, 8] \text{ and } b \in [0.59, 6.59] \)

$(6.00, 1.59]$, which is the correct interval but negatives of the actual endpoints.
\item \( (-\infty, a] \cup (b, \infty), \text{ where } a \in [6, 8] \text{ and } b \in [0.59, 7.59] \)

$(-\infty, 6.00] \cup (1.59, \infty)$, which corresponds to displaying the and-inequality as an or-inequality AND flipping the inequality AND getting negatives of the actual endpoints.
\item \( (-\infty, a) \cup [b, \infty), \text{ where } a \in [3, 8] \text{ and } b \in [0.6, 4] \)

$(-\infty, 6.00) \cup [1.59, \infty)$, which corresponds to displaying the and-inequality as an or-inequality and getting negatives of the actual endpoints.
\item \( \text{None of the above.} \)

* This is correct as the answer should be $(-6.00, -1.59]$.
\end{enumerate}

\textbf{General Comment:} To solve, you will need to break up the compound inequality into two inequalities. Be sure to keep track of the inequality! It may be best to draw a number line and graph your solution.
}
\litem{
Solve the linear inequality below. Then, choose the constant and interval combination that describes the solution set.
\[ \frac{-7}{5} - \frac{7}{8} x > \frac{-5}{6} x + \frac{4}{9} \]
The solution is \( (-\infty, -44.267) \), which is option A.\begin{enumerate}[label=\Alph*.]
\item \( (-\infty, a), \text{ where } a \in [-47.27, -42.27] \)

* $(-\infty, -44.267)$, which is the correct option.
\item \( (-\infty, a), \text{ where } a \in [43.27, 47.27] \)

 $(-\infty, 44.267)$, which corresponds to negating the endpoint of the solution.
\item \( (a, \infty), \text{ where } a \in [-44.27, -38.27] \)

 $(-44.267, \infty)$, which corresponds to switching the direction of the interval. You likely did this if you did not flip the inequality when dividing by a negative!
\item \( (a, \infty), \text{ where } a \in [42.27, 48.27] \)

 $(44.267, \infty)$, which corresponds to switching the direction of the interval AND negating the endpoint. You likely did this if you did not flip the inequality when dividing by a negative as well as not moving values over to a side properly.
\item \( \text{None of the above}. \)

You may have chosen this if you thought the inequality did not match the ends of the intervals.
\end{enumerate}

\textbf{General Comment:} Remember that less/greater than or equal to includes the endpoint, while less/greater do not. Also, remember that you need to flip the inequality when you multiply or divide by a negative.
}
\litem{
Solve the linear inequality below. Then, choose the constant and interval combination that describes the solution set.
\[ -9x + 4 \geq 6x -3 \]
The solution is \( (-\infty, 0.467] \), which is option A.\begin{enumerate}[label=\Alph*.]
\item \( (-\infty, a], \text{ where } a \in [0.3, 0.48] \)

* $(-\infty, 0.467]$, which is the correct option.
\item \( [a, \infty), \text{ where } a \in [-0.6, 0.1] \)

 $[-0.467, \infty)$, which corresponds to switching the direction of the interval AND negating the endpoint. You likely did this if you did not flip the inequality when dividing by a negative as well as not moving values over to a side properly.
\item \( (-\infty, a], \text{ where } a \in [-2.57, -0.12] \)

 $(-\infty, -0.467]$, which corresponds to negating the endpoint of the solution.
\item \( [a, \infty), \text{ where } a \in [0.2, 4.3] \)

 $[0.467, \infty)$, which corresponds to switching the direction of the interval. You likely did this if you did not flip the inequality when dividing by a negative!
\item \( \text{None of the above}. \)

You may have chosen this if you thought the inequality did not match the ends of the intervals.
\end{enumerate}

\textbf{General Comment:} Remember that less/greater than or equal to includes the endpoint, while less/greater do not. Also, remember that you need to flip the inequality when you multiply or divide by a negative.
}
\litem{
Solve the linear inequality below. Then, choose the constant and interval combination that describes the solution set.
\[ -9x + 7 \geq -4x + 10 \]
The solution is \( (-\infty, -0.6] \), which is option C.\begin{enumerate}[label=\Alph*.]
\item \( [a, \infty), \text{ where } a \in [0.6, 2.6] \)

 $[0.6, \infty)$, which corresponds to switching the direction of the interval AND negating the endpoint. You likely did this if you did not flip the inequality when dividing by a negative as well as not moving values over to a side properly.
\item \( (-\infty, a], \text{ where } a \in [0.6, 2.6] \)

 $(-\infty, 0.6]$, which corresponds to negating the endpoint of the solution.
\item \( (-\infty, a], \text{ where } a \in [-4.6, 0.4] \)

* $(-\infty, -0.6]$, which is the correct option.
\item \( [a, \infty), \text{ where } a \in [-3.6, 0.4] \)

 $[-0.6, \infty)$, which corresponds to switching the direction of the interval. You likely did this if you did not flip the inequality when dividing by a negative!
\item \( \text{None of the above}. \)

You may have chosen this if you thought the inequality did not match the ends of the intervals.
\end{enumerate}

\textbf{General Comment:} Remember that less/greater than or equal to includes the endpoint, while less/greater do not. Also, remember that you need to flip the inequality when you multiply or divide by a negative.
}
\litem{
Solve the linear inequality below. Then, choose the constant and interval combination that describes the solution set.
\[ -8 - 3 x > 5 x \text{ or } 5 + 8 x < 9 x \]
The solution is \( (-\infty, -1.0) \text{ or } (5.0, \infty) \), which is option C.\begin{enumerate}[label=\Alph*.]
\item \( (-\infty, a] \cup [b, \infty), \text{ where } a \in [-5.3, -4.5] \text{ and } b \in [-3, 2] \)

Corresponds to including the endpoints AND negating.
\item \( (-\infty, a) \cup (b, \infty), \text{ where } a \in [-5, -2] \text{ and } b \in [1, 2] \)

Corresponds to inverting the inequality and negating the solution.
\item \( (-\infty, a) \cup (b, \infty), \text{ where } a \in [-3, 1] \text{ and } b \in [2, 6] \)

 * Correct option.
\item \( (-\infty, a] \cup [b, \infty), \text{ where } a \in [-2.2, -0.5] \text{ and } b \in [3, 6] \)

Corresponds to including the endpoints (when they should be excluded).
\item \( (-\infty, \infty) \)

Corresponds to the variable canceling, which does not happen in this instance.
\end{enumerate}

\textbf{General Comment:} When multiplying or dividing by a negative, flip the sign.
}
\litem{
Using an interval or intervals, describe all the $x$-values within or including a distance of the given values.
\[ \text{ More than } 8 \text{ units from the number } -6. \]
The solution is \( (-\infty, -14) \cup (2, \infty) \), which is option C.\begin{enumerate}[label=\Alph*.]
\item \( [-14, 2] \)

This describes the values no more than 8 from -6
\item \( (-14, 2) \)

This describes the values less than 8 from -6
\item \( (-\infty, -14) \cup (2, \infty) \)

This describes the values more than 8 from -6
\item \( (-\infty, -14] \cup [2, \infty) \)

This describes the values no less than 8 from -6
\item \( \text{None of the above} \)

You likely thought the values in the interval were not correct.
\end{enumerate}

\textbf{General Comment:} When thinking about this language, it helps to draw a number line and try points.
}
\litem{
Solve the linear inequality below. Then, choose the constant and interval combination that describes the solution set.
\[ -5 + 7 x > 10 x \text{ or } 8 + 6 x < 8 x \]
The solution is \( (-\infty, -1.667) \text{ or } (4.0, \infty) \), which is option D.\begin{enumerate}[label=\Alph*.]
\item \( (-\infty, a] \cup [b, \infty), \text{ where } a \in [-4, -3] \text{ and } b \in [1.6, 2.5] \)

Corresponds to including the endpoints AND negating.
\item \( (-\infty, a) \cup (b, \infty), \text{ where } a \in [-4, -2] \text{ and } b \in [-0.33, 3.67] \)

Corresponds to inverting the inequality and negating the solution.
\item \( (-\infty, a] \cup [b, \infty), \text{ where } a \in [-3.67, 2.33] \text{ and } b \in [2.4, 4.1] \)

Corresponds to including the endpoints (when they should be excluded).
\item \( (-\infty, a) \cup (b, \infty), \text{ where } a \in [-1.67, 1.33] \text{ and } b \in [4, 8] \)

 * Correct option.
\item \( (-\infty, \infty) \)

Corresponds to the variable canceling, which does not happen in this instance.
\end{enumerate}

\textbf{General Comment:} When multiplying or dividing by a negative, flip the sign.
}
\litem{
Solve the linear inequality below. Then, choose the constant and interval combination that describes the solution set.
\[ -5 - 6 x < \frac{-34 x - 8}{6} \leq -3 - 6 x \]
The solution is \( \text{None of the above.} \), which is option E.\begin{enumerate}[label=\Alph*.]
\item \( (-\infty, a) \cup [b, \infty), \text{ where } a \in [7, 12] \text{ and } b \in [2, 7] \)

$(-\infty, 11.00) \cup [5.00, \infty)$, which corresponds to displaying the and-inequality as an or-inequality and getting negatives of the actual endpoints.
\item \( (a, b], \text{ where } a \in [11, 15] \text{ and } b \in [2, 8] \)

$(11.00, 5.00]$, which is the correct interval but negatives of the actual endpoints.
\item \( [a, b), \text{ where } a \in [9, 14] \text{ and } b \in [4, 8] \)

$[11.00, 5.00)$, which corresponds to flipping the inequality and getting negatives of the actual endpoints.
\item \( (-\infty, a] \cup (b, \infty), \text{ where } a \in [11, 13] \text{ and } b \in [3, 7] \)

$(-\infty, 11.00] \cup (5.00, \infty)$, which corresponds to displaying the and-inequality as an or-inequality AND flipping the inequality AND getting negatives of the actual endpoints.
\item \( \text{None of the above.} \)

* This is correct as the answer should be $(-11.00, -5.00]$.
\end{enumerate}

\textbf{General Comment:} To solve, you will need to break up the compound inequality into two inequalities. Be sure to keep track of the inequality! It may be best to draw a number line and graph your solution.
}
\end{enumerate}

\end{document}