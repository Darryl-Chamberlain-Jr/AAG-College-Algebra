\documentclass{extbook}[14pt]
\usepackage{multicol, enumerate, enumitem, hyperref, color, soul, setspace, parskip, fancyhdr, amssymb, amsthm, amsmath, bbm, latexsym, units, mathtools}
\everymath{\displaystyle}
\usepackage[headsep=0.5cm,headheight=0cm, left=1 in,right= 1 in,top= 1 in,bottom= 1 in]{geometry}
\usepackage{dashrule}  % Package to use the command below to create lines between items
\newcommand{\litem}[1]{\item #1

\rule{\textwidth}{0.4pt}}
\pagestyle{fancy}
\lhead{}
\chead{Answer Key for Progress Quiz 4 Version C}
\rhead{}
\lfoot{9187-5854}
\cfoot{}
\rfoot{Spring 2021}
\begin{document}
\textbf{This key should allow you to understand why you choose the option you did (beyond just getting a question right or wrong). \href{https://xronos.clas.ufl.edu/mac1105spring2020/courseDescriptionAndMisc/Exams/LearningFromResults}{More instructions on how to use this key can be found here}.}

\textbf{If you have a suggestion to make the keys better, \href{https://forms.gle/CZkbZmPbC9XALEE88}{please fill out the short survey here}.}

\textit{Note: This key is auto-generated and may contain issues and/or errors. The keys are reviewed after each exam to ensure grading is done accurately. If there are issues (like duplicate options), they are noted in the offline gradebook. The keys are a work-in-progress to give students as many resources to improve as possible.}

\rule{\textwidth}{0.4pt}

\begin{enumerate}\litem{
Choose the interval below that $f$ composed with $g$ at $x=-1$ is in.
\[ f(x) = 4x^{3} -2 x^{2} -4 x \text{ and } g(x) = 3x^{3} +3 x^{2} -x + 2 \]The solution is \( 78.0 \), which is option B.\begin{enumerate}[label=\Alph*.]
\item \( (f \circ g)(-1) \in [-11, -5] \)

 Distractor 1: Corresponds to reversing the composition.
\item \( (f \circ g)(-1) \in [75, 79] \)

* This is the correct solution
\item \( (f \circ g)(-1) \in [81, 90] \)

 Distractor 2: Corresponds to being slightly off from the solution.
\item \( (f \circ g)(-1) \in [-5, 0] \)

 Distractor 3: Corresponds to being slightly off from the solution.
\item \( \text{It is not possible to compose the two functions.} \)


\end{enumerate}

\textbf{General Comment:} $f$ composed with $g$ at $x$ means $f(g(x))$. The order matters!
}
\litem{
Find the inverse of the function below (if it exists). Then, evaluate the inverse at $x = -15$ and choose the interval the $f^{-1}(-15)$ belongs to.
\[ f(x) = \sqrt[3]{3 x + 5} \]The solution is \( -1126.6666666666667 \), which is option A.\begin{enumerate}[label=\Alph*.]
\item \( f^{-1}(-15) \in [-1128.5, -1124.9] \)

* This is the correct solution.
\item \( f^{-1}(-15) \in [-1124.7, -1122.4] \)

 Distractor 1: This corresponds to 
\item \( f^{-1}(-15) \in [1122.2, 1125.6] \)

 This solution corresponds to distractor 3.
\item \( f^{-1}(-15) \in [1124.6, 1127.9] \)

 This solution corresponds to distractor 2.
\item \( \text{ The function is not invertible for all Real numbers. } \)

 This solution corresponds to distractor 4.
\end{enumerate}

\textbf{General Comment:} Be sure you check that the function is 1-1 before trying to find the inverse!
}
\litem{
Find the inverse of the function below (if it exists). Then, evaluate the inverse at $x = -10$ and choose the interval that $f^{-1}(-10)$ belongs to.
\[ f(x) = 4 x^2 - 2 \]The solution is \( \text{ The function is not invertible for all Real numbers. } \), which is option E.\begin{enumerate}[label=\Alph*.]
\item \( f^{-1}(-10) \in [1.58, 2.34] \)

 Distractor 2: This corresponds to finding the (nonexistent) inverse and not subtracting by the vertical shift.
\item \( f^{-1}(-10) \in [2.26, 2.49] \)

 Distractor 3: This corresponds to finding the (nonexistent) inverse and dividing by a negative.
\item \( f^{-1}(-10) \in [4.29, 5.04] \)

 Distractor 4: This corresponds to both distractors 2 and 3.
\item \( f^{-1}(-10) \in [1.22, 1.62] \)

 Distractor 1: This corresponds to trying to find the inverse even though the function is not 1-1. 
\item \( \text{ The function is not invertible for all Real numbers. } \)

* This is the correct option.
\end{enumerate}

\textbf{General Comment:} Be sure you check that the function is 1-1 before trying to find the inverse!
}
\litem{
Choose the interval below that $f$ composed with $g$ at $x=-1$ is in.
\[ f(x) = 2x^{3} +4 x^{2} +2 x + 1 \text{ and } g(x) = 2x^{3} +2 x^{2} +2 x \]The solution is \( -3.0 \), which is option C.\begin{enumerate}[label=\Alph*.]
\item \( (f \circ g)(-1) \in [9, 20] \)

 Distractor 3: Corresponds to being slightly off from the solution.
\item \( (f \circ g)(-1) \in [-15, -9] \)

 Distractor 2: Corresponds to being slightly off from the solution.
\item \( (f \circ g)(-1) \in [-4, 0] \)

* This is the correct solution
\item \( (f \circ g)(-1) \in [4, 9] \)

 Distractor 1: Corresponds to reversing the composition.
\item \( \text{It is not possible to compose the two functions.} \)


\end{enumerate}

\textbf{General Comment:} $f$ composed with $g$ at $x$ means $f(g(x))$. The order matters!
}
\litem{
Determine whether the function below is 1-1.
\[ f(x) = -24 x^2 - 270 x - 729 \]The solution is \( \text{no} \), which is option E.\begin{enumerate}[label=\Alph*.]
\item \( \text{Yes, the function is 1-1.} \)

Corresponds to believing the function passes the Horizontal Line test.
\item \( \text{No, because the domain of the function is not $(-\infty, \infty)$.} \)

Corresponds to believing 1-1 means the domain is all Real numbers.
\item \( \text{No, because there is an $x$-value that goes to 2 different $y$-values.} \)

Corresponds to the Vertical Line test, which checks if an expression is a function.
\item \( \text{No, because the range of the function is not $(-\infty, \infty)$.} \)

Corresponds to believing 1-1 means the range is all Real numbers.
\item \( \text{No, because there is a $y$-value that goes to 2 different $x$-values.} \)

* This is the solution.
\end{enumerate}

\textbf{General Comment:} There are only two valid options: The function is 1-1 OR No because there is a $y$-value that goes to 2 different $x$-values.
}
\litem{
Find the inverse of the function below. Then, evaluate the inverse at $x = 5$ and choose the interval that $f^{-1}(5)$ belongs to.
\[ f(x) = \ln{(x+3)}-2 \]The solution is \( f^{-1}(5) = 1093.633 \), which is option A.\begin{enumerate}[label=\Alph*.]
\item \( f^{-1}(5) \in [1090.63, 1097.63] \)

 This is the solution.
\item \( f^{-1}(5) \in [3.39, 6.39] \)

 This solution corresponds to distractor 2.
\item \( f^{-1}(5) \in [1097.63, 1106.63] \)

 This solution corresponds to distractor 3.
\item \( f^{-1}(5) \in [2977.96, 2983.96] \)

 This solution corresponds to distractor 4.
\item \( f^{-1}(5) \in [15.09, 18.09] \)

 This solution corresponds to distractor 1.
\end{enumerate}

\textbf{General Comment:} Natural log and exponential functions always have an inverse. Once you switch the $x$ and $y$, use the conversion $ e^y = x \leftrightarrow y=\ln(x)$.
}
\litem{
Find the inverse of the function below. Then, evaluate the inverse at $x = 9$ and choose the interval that $f^{-1}(9)$ belongs to.
\[ f(x) = \ln{(x-4)}-5 \]The solution is \( f^{-1}(9) = 1202608.284 \), which is option B.\begin{enumerate}[label=\Alph*.]
\item \( f^{-1}(9) \in [1202596.28, 1202605.28] \)

 This solution corresponds to distractor 3.
\item \( f^{-1}(9) \in [1202604.28, 1202609.28] \)

 This is the solution.
\item \( f^{-1}(9) \in [141.41, 144.41] \)

 This solution corresponds to distractor 4.
\item \( f^{-1}(9) \in [442408.39, 442416.39] \)

 This solution corresponds to distractor 2.
\item \( f^{-1}(9) \in [55.6, 61.6] \)

 This solution corresponds to distractor 1.
\end{enumerate}

\textbf{General Comment:} Natural log and exponential functions always have an inverse. Once you switch the $x$ and $y$, use the conversion $ e^y = x \leftrightarrow y=\ln(x)$.
}
\litem{
Subtract the following functions, then choose the domain of the resulting function from the list below.
\[ f(x) = 8x^{2} +5 x + 5 \text{ and } g(x) = 3x + 6 \]The solution is \( (-\infty, \infty) \), which is option E.\begin{enumerate}[label=\Alph*.]
\item \( \text{ The domain is all Real numbers greater than or equal to } x = a, \text{ where } a \in [-8, -2] \)


\item \( \text{ The domain is all Real numbers less than or equal to } x = a, \text{ where } a \in [2.33, 3.33] \)


\item \( \text{ The domain is all Real numbers except } x = a, \text{ where } a \in [1.75, 6.75] \)


\item \( \text{ The domain is all Real numbers except } x = a \text{ and } x = b, \text{ where } a \in [4.2, 10.2] \text{ and } b \in [-7.17, -4.17] \)


\item \( \text{ The domain is all Real numbers. } \)


\end{enumerate}

\textbf{General Comment:} The new domain is the intersection of the previous domains.
}
\litem{
Determine whether the function below is 1-1.
\[ f(x) = (5 x - 23)^3 \]The solution is \( \text{yes} \), which is option A.\begin{enumerate}[label=\Alph*.]
\item \( \text{Yes, the function is 1-1.} \)

* This is the solution.
\item \( \text{No, because the range of the function is not $(-\infty, \infty)$.} \)

Corresponds to believing 1-1 means the range is all Real numbers.
\item \( \text{No, because there is an $x$-value that goes to 2 different $y$-values.} \)

Corresponds to the Vertical Line test, which checks if an expression is a function.
\item \( \text{No, because the domain of the function is not $(-\infty, \infty)$.} \)

Corresponds to believing 1-1 means the domain is all Real numbers.
\item \( \text{No, because there is a $y$-value that goes to 2 different $x$-values.} \)

Corresponds to the Horizontal Line test, which this function passes.
\end{enumerate}

\textbf{General Comment:} There are only two valid options: The function is 1-1 OR No because there is a $y$-value that goes to 2 different $x$-values.
}
\litem{
Multiply the following functions, then choose the domain of the resulting function from the list below.
\[ f(x) = x + 3 \text{ and } g(x) = 6x + 7 \]The solution is \( (-\infty, \infty) \), which is option E.\begin{enumerate}[label=\Alph*.]
\item \( \text{ The domain is all Real numbers greater than or equal to } x = a, \text{ where } a \in [-10, -5] \)


\item \( \text{ The domain is all Real numbers except } x = a, \text{ where } a \in [-5.67, -2.67] \)


\item \( \text{ The domain is all Real numbers less than or equal to } x = a, \text{ where } a \in [-5.33, -0.33] \)


\item \( \text{ The domain is all Real numbers except } x = a \text{ and } x = b, \text{ where } a \in [4.6, 11.6] \text{ and } b \in [6.2, 11.2] \)


\item \( \text{ The domain is all Real numbers. } \)


\end{enumerate}

\textbf{General Comment:} The new domain is the intersection of the previous domains.
}
\end{enumerate}

\end{document}