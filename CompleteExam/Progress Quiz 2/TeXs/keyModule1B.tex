\documentclass{extbook}[14pt]
\usepackage{multicol, enumerate, enumitem, hyperref, color, soul, setspace, parskip, fancyhdr, amssymb, amsthm, amsmath, bbm, latexsym, units, mathtools}
\everymath{\displaystyle}
\usepackage[headsep=0.5cm,headheight=0cm, left=1 in,right= 1 in,top= 1 in,bottom= 1 in]{geometry}
\usepackage{dashrule}  % Package to use the command below to create lines between items
\newcommand{\litem}[1]{\item #1

\rule{\textwidth}{0.4pt}}
\pagestyle{fancy}
\lhead{}
\chead{Answer Key for Progress Quiz 2 Version B}
\rhead{}
\lfoot{7862-5421}
\cfoot{}
\rfoot{Spring 2021}
\begin{document}
\textbf{This key should allow you to understand why you choose the option you did (beyond just getting a question right or wrong). \href{https://xronos.clas.ufl.edu/mac1105spring2020/courseDescriptionAndMisc/Exams/LearningFromResults}{More instructions on how to use this key can be found here}.}

\textbf{If you have a suggestion to make the keys better, \href{https://forms.gle/CZkbZmPbC9XALEE88}{please fill out the short survey here}.}

\textit{Note: This key is auto-generated and may contain issues and/or errors. The keys are reviewed after each exam to ensure grading is done accurately. If there are issues (like duplicate options), they are noted in the offline gradebook. The keys are a work-in-progress to give students as many resources to improve as possible.}

\rule{\textwidth}{0.4pt}

\begin{enumerate}\litem{
Choose the \textbf{smallest} set of Real numbers that the number below belongs to.
\[ -\sqrt{\frac{6}{0}} \]

The solution is \( \text{Not a Real number} \), which is option C.\begin{enumerate}[label=\Alph*.]
\item \( \text{Irrational} \)

These cannot be written as a fraction of Integers.
\item \( \text{Rational} \)

These are numbers that can be written as fraction of Integers (e.g., -2/3)
\item \( \text{Not a Real number} \)

* This is the correct option!
\item \( \text{Integer} \)

These are the negative and positive counting numbers (..., -3, -2, -1, 0, 1, 2, 3, ...)
\item \( \text{Whole} \)

These are the counting numbers with 0 (0, 1, 2, 3, ...)
\end{enumerate}

\textbf{General Comment:} First, you \textbf{NEED} to simplify the expression. This question simplifies to $-\sqrt{\frac{6}{0}}$. 
 
 Be sure you look at the simplified fraction and not just the decimal expansion. Numbers such as 13, 17, and 19 provide \textbf{long but repeating/terminating decimal expansions!} 
 
 The only ways to *not* be a Real number are: dividing by 0 or taking the square root of a negative number. 
 
 Irrational numbers are more than just square root of 3: adding or subtracting values from square root of 3 is also irrational.
}
\litem{
Simplify the expression below and choose the interval the simplification is contained within.
\[ 12 - 13^2 + 10 \div 3 * 11 \div 2 \]

The solution is \( -138.667 \), which is option D.\begin{enumerate}[label=\Alph*.]
\item \( [198.33, 200.33] \)

 199.333, which corresponds to an Order of Operations error: multiplying by negative before squaring. For example: $(-3)^2 \neq -3^2$
\item \( [-158.85, -154.85] \)

 -156.848, which corresponds to an Order of Operations error: not reading left-to-right for multiplication/division.
\item \( [181.15, 188.15] \)

 181.152, which corresponds to two Order of Operations errors.
\item \( [-143.67, -135.67] \)

* -138.667, this is the correct option
\item \( \text{None of the above} \)

 You may have gotten this by making an unanticipated error. If you got a value that is not any of the others, please let the coordinator know so they can help you figure out what happened.
\end{enumerate}

\textbf{General Comment:} While you may remember (or were taught) PEMDAS is done in order, it is actually done as P/E/MD/AS. When we are at MD or AS, we read left to right.
}
\litem{
Choose the \textbf{smallest} set of Real numbers that the number below belongs to.
\[ \sqrt{\frac{1190}{14}} \]

The solution is \( \text{Irrational} \), which is option A.\begin{enumerate}[label=\Alph*.]
\item \( \text{Irrational} \)

* This is the correct option!
\item \( \text{Integer} \)

These are the negative and positive counting numbers (..., -3, -2, -1, 0, 1, 2, 3, ...)
\item \( \text{Whole} \)

These are the counting numbers with 0 (0, 1, 2, 3, ...)
\item \( \text{Not a Real number} \)

These are Nonreal Complex numbers \textbf{OR} things that are not numbers (e.g., dividing by 0).
\item \( \text{Rational} \)

These are numbers that can be written as fraction of Integers (e.g., -2/3)
\end{enumerate}

\textbf{General Comment:} First, you \textbf{NEED} to simplify the expression. This question simplifies to $\sqrt{85}$. 
 
 Be sure you look at the simplified fraction and not just the decimal expansion. Numbers such as 13, 17, and 19 provide \textbf{long but repeating/terminating decimal expansions!} 
 
 The only ways to *not* be a Real number are: dividing by 0 or taking the square root of a negative number. 
 
 Irrational numbers are more than just square root of 3: adding or subtracting values from square root of 3 is also irrational.
}
\litem{
Simplify the expression below into the form $a+bi$. Then, choose the intervals that $a$ and $b$ belong to.
\[ \frac{63 - 55 i}{3 - 4 i} \]

The solution is \( 16.36  + 3.48 i \), which is option D.\begin{enumerate}[label=\Alph*.]
\item \( a \in [15.5, 18.5] \text{ and } b \in [86.5, 88.5] \)

 $16.36  + 87.00 i$, which corresponds to forgetting to multiply the conjugate by the numerator.
\item \( a \in [20, 22.5] \text{ and } b \in [13, 14.5] \)

 $21.00  + 13.75 i$, which corresponds to just dividing the first term by the first term and the second by the second.
\item \( a \in [-2, -1] \text{ and } b \in [-17.5, -16] \)

 $-1.24  - 16.68 i$, which corresponds to forgetting to multiply the conjugate by the numerator and not computing the conjugate correctly.
\item \( a \in [15.5, 18.5] \text{ and } b \in [2.5, 5] \)

* $16.36  + 3.48 i$, which is the correct option.
\item \( a \in [408.5, 409.5] \text{ and } b \in [2.5, 5] \)

 $409.00  + 3.48 i$, which corresponds to forgetting to multiply the conjugate by the numerator and using a plus instead of a minus in the denominator.
\end{enumerate}

\textbf{General Comment:} Multiply the numerator and denominator by the *conjugate* of the denominator, then simplify. For example, if we have $2+3i$, the conjugate is $2-3i$.
}
\litem{
Simplify the expression below into the form $a+bi$. Then, choose the intervals that $a$ and $b$ belong to.
\[ \frac{54 + 77 i}{4 - 3 i} \]

The solution is \( -0.60  + 18.80 i \), which is option D.\begin{enumerate}[label=\Alph*.]
\item \( a \in [17, 19.5] \text{ and } b \in [5, 6.5] \)

 $17.88  + 5.84 i$, which corresponds to forgetting to multiply the conjugate by the numerator and not computing the conjugate correctly.
\item \( a \in [-1.5, 0.5] \text{ and } b \in [469.5, 471.5] \)

 $-0.60  + 470.00 i$, which corresponds to forgetting to multiply the conjugate by the numerator.
\item \( a \in [-16, -14.5] \text{ and } b \in [18, 20] \)

 $-15.00  + 18.80 i$, which corresponds to forgetting to multiply the conjugate by the numerator and using a plus instead of a minus in the denominator.
\item \( a \in [-1.5, 0.5] \text{ and } b \in [18, 20] \)

* $-0.60  + 18.80 i$, which is the correct option.
\item \( a \in [13, 16] \text{ and } b \in [-26.5, -25.5] \)

 $13.50  - 25.67 i$, which corresponds to just dividing the first term by the first term and the second by the second.
\end{enumerate}

\textbf{General Comment:} Multiply the numerator and denominator by the *conjugate* of the denominator, then simplify. For example, if we have $2+3i$, the conjugate is $2-3i$.
}
\litem{
Simplify the expression below and choose the interval the simplification is contained within.
\[ 1 - 2 \div 4 * 9 - (16 * 5) \]

The solution is \( -83.500 \), which is option A.\begin{enumerate}[label=\Alph*.]
\item \( [-86.5, -81.5] \)

* -83.500, which is the correct option.
\item \( [-97.5, -93.5] \)

 -97.500, which corresponds to not distributing a negative correctly.
\item \( [-81.06, -78.06] \)

 -79.056, which corresponds to an Order of Operations error: not reading left-to-right for multiplication/division.
\item \( [76.94, 86.94] \)

 80.944, which corresponds to not distributing addition and subtraction correctly.
\item \( \text{None of the above} \)

 You may have gotten this by making an unanticipated error. If you got a value that is not any of the others, please let the coordinator know so they can help you figure out what happened.
\end{enumerate}

\textbf{General Comment:} While you may remember (or were taught) PEMDAS is done in order, it is actually done as P/E/MD/AS. When we are at MD or AS, we read left to right.
}
\litem{
Choose the \textbf{smallest} set of Complex numbers that the number below belongs to.
\[ \sqrt{\frac{169}{441}}+\sqrt{165} i \]

The solution is \( \text{Nonreal Complex} \), which is option D.\begin{enumerate}[label=\Alph*.]
\item \( \text{Pure Imaginary} \)

This is a Complex number $(a+bi)$ that \textbf{only} has an imaginary part like $2i$.
\item \( \text{Not a Complex Number} \)

This is not a number. The only non-Complex number we know is dividing by 0 as this is not a number!
\item \( \text{Irrational} \)

These cannot be written as a fraction of Integers. Remember: $\pi$ is not an Integer!
\item \( \text{Nonreal Complex} \)

* This is the correct option!
\item \( \text{Rational} \)

These are numbers that can be written as fraction of Integers (e.g., -2/3 + 5)
\end{enumerate}

\textbf{General Comment:} Be sure to simplify $i^2 = -1$. This may remove the imaginary portion for your number. If you are having trouble, you may want to look at the \textit{Subgroups of the Real Numbers} section.
}
\litem{
Simplify the expression below into the form $a+bi$. Then, choose the intervals that $a$ and $b$ belong to.
\[ (-6 - 5 i)(4 - 2 i) \]

The solution is \( -34 - 8 i \), which is option C.\begin{enumerate}[label=\Alph*.]
\item \( a \in [-18, -11] \text{ and } b \in [30.5, 33.9] \)

 $-14 + 32 i$, which corresponds to adding a minus sign in the first term.
\item \( a \in [-29, -18] \text{ and } b \in [9.4, 10.2] \)

 $-24 + 10 i$, which corresponds to just multiplying the real terms to get the real part of the solution and the coefficients in the complex terms to get the complex part.
\item \( a \in [-39, -31] \text{ and } b \in [-11.4, -6.3] \)

* $-34 - 8 i$, which is the correct option.
\item \( a \in [-39, -31] \text{ and } b \in [7.6, 8.1] \)

 $-34 + 8 i$, which corresponds to adding a minus sign in both terms.
\item \( a \in [-18, -11] \text{ and } b \in [-32.4, -31.5] \)

 $-14 - 32 i$, which corresponds to adding a minus sign in the second term.
\end{enumerate}

\textbf{General Comment:} You can treat $i$ as a variable and distribute. Just remember that $i^2=-1$, so you can continue to reduce after you distribute.
}
\litem{
Choose the \textbf{smallest} set of Complex numbers that the number below belongs to.
\[ -\sqrt{\frac{256}{529}} + 49i^2 \]

The solution is \( \text{Rational} \), which is option D.\begin{enumerate}[label=\Alph*.]
\item \( \text{Irrational} \)

These cannot be written as a fraction of Integers. Remember: $\pi$ is not an Integer!
\item \( \text{Pure Imaginary} \)

This is a Complex number $(a+bi)$ that \textbf{only} has an imaginary part like $2i$.
\item \( \text{Not a Complex Number} \)

This is not a number. The only non-Complex number we know is dividing by 0 as this is not a number!
\item \( \text{Rational} \)

* This is the correct option!
\item \( \text{Nonreal Complex} \)

This is a Complex number $(a+bi)$ that is not Real (has $i$ as part of the number).
\end{enumerate}

\textbf{General Comment:} Be sure to simplify $i^2 = -1$. This may remove the imaginary portion for your number. If you are having trouble, you may want to look at the \textit{Subgroups of the Real Numbers} section.
}
\litem{
Simplify the expression below into the form $a+bi$. Then, choose the intervals that $a$ and $b$ belong to.
\[ (10 + 4 i)(-7 + 6 i) \]

The solution is \( -94 + 32 i \), which is option C.\begin{enumerate}[label=\Alph*.]
\item \( a \in [-74, -69] \text{ and } b \in [22, 29] \)

 $-70 + 24 i$, which corresponds to just multiplying the real terms to get the real part of the solution and the coefficients in the complex terms to get the complex part.
\item \( a \in [-54, -43] \text{ and } b \in [-90, -81] \)

 $-46 - 88 i$, which corresponds to adding a minus sign in the second term.
\item \( a \in [-97, -92] \text{ and } b \in [29, 34] \)

* $-94 + 32 i$, which is the correct option.
\item \( a \in [-54, -43] \text{ and } b \in [85, 90] \)

 $-46 + 88 i$, which corresponds to adding a minus sign in the first term.
\item \( a \in [-97, -92] \text{ and } b \in [-37, -30] \)

 $-94 - 32 i$, which corresponds to adding a minus sign in both terms.
\end{enumerate}

\textbf{General Comment:} You can treat $i$ as a variable and distribute. Just remember that $i^2=-1$, so you can continue to reduce after you distribute.
}
\end{enumerate}

\end{document}