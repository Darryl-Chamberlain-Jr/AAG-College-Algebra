\documentclass{extbook}[14pt]
\usepackage{multicol, enumerate, enumitem, hyperref, color, soul, setspace, parskip, fancyhdr, amssymb, amsthm, amsmath, bbm, latexsym, units, mathtools}
\everymath{\displaystyle}
\usepackage[headsep=0.5cm,headheight=0cm, left=1 in,right= 1 in,top= 1 in,bottom= 1 in]{geometry}
\usepackage{dashrule}  % Package to use the command below to create lines between items
\newcommand{\litem}[1]{\item #1

\rule{\textwidth}{0.4pt}}
\pagestyle{fancy}
\lhead{}
\chead{Answer Key for Makeup Progress Quiz 1 Version B}
\rhead{}
\lfoot{6018-3080}
\cfoot{}
\rfoot{Spring 2021}
\begin{document}
\textbf{This key should allow you to understand why you choose the option you did (beyond just getting a question right or wrong). \href{https://xronos.clas.ufl.edu/mac1105spring2020/courseDescriptionAndMisc/Exams/LearningFromResults}{More instructions on how to use this key can be found here}.}

\textbf{If you have a suggestion to make the keys better, \href{https://forms.gle/CZkbZmPbC9XALEE88}{please fill out the short survey here}.}

\textit{Note: This key is auto-generated and may contain issues and/or errors. The keys are reviewed after each exam to ensure grading is done accurately. If there are issues (like duplicate options), they are noted in the offline gradebook. The keys are a work-in-progress to give students as many resources to improve as possible.}

\rule{\textwidth}{0.4pt}

\begin{enumerate}\litem{
Simplify the expression below and choose the interval the simplification is contained within.
\[ 10 - 7 \div 1 * 5 - (11 * 17) \]The solution is \( -212.000 \), which is option A.\begin{enumerate}[label=\Alph*.]
\item \( [-215, -203] \)

* -212.000, which is the correct option.
\item \( [184.6, 197.6] \)

 195.600, which corresponds to not distributing addition and subtraction correctly.
\item \( [-178.4, -177.4] \)

 -178.400, which corresponds to an Order of Operations error: not reading left-to-right for multiplication/division.
\item \( [-612, -610] \)

 -612.000, which corresponds to not distributing a negative correctly.
\item \( \text{None of the above} \)

 You may have gotten this by making an unanticipated error. If you got a value that is not any of the others, please let the coordinator know so they can help you figure out what happened.
\end{enumerate}

\textbf{General Comment:} While you may remember (or were taught) PEMDAS is done in order, it is actually done as P/E/MD/AS. When we are at MD or AS, we read left to right.
}
\litem{
Simplify the expression below into the form $a+bi$. Then, choose the intervals that $a$ and $b$ belong to.
\[ (-4 + 9 i)(-6 - 10 i) \]The solution is \( 114 - 14 i \), which is option B.\begin{enumerate}[label=\Alph*.]
\item \( a \in [24, 27] \text{ and } b \in [-93, -87] \)

 $24 - 90 i$, which corresponds to just multiplying the real terms to get the real part of the solution and the coefficients in the complex terms to get the complex part.
\item \( a \in [113, 116] \text{ and } b \in [-18, -9] \)

* $114 - 14 i$, which is the correct option.
\item \( a \in [-70, -65] \text{ and } b \in [93, 99] \)

 $-66 + 94 i$, which corresponds to adding a minus sign in the first term.
\item \( a \in [-70, -65] \text{ and } b \in [-95, -92] \)

 $-66 - 94 i$, which corresponds to adding a minus sign in the second term.
\item \( a \in [113, 116] \text{ and } b \in [11, 20] \)

 $114 + 14 i$, which corresponds to adding a minus sign in both terms.
\end{enumerate}

\textbf{General Comment:} You can treat $i$ as a variable and distribute. Just remember that $i^2=-1$, so you can continue to reduce after you distribute.
}
\litem{
Simplify the expression below into the form $a+bi$. Then, choose the intervals that $a$ and $b$ belong to.
\[ \frac{18 - 33 i}{-8 - i} \]The solution is \( -1.71  + 4.34 i \), which is option C.\begin{enumerate}[label=\Alph*.]
\item \( a \in [-1.9, -1.15] \text{ and } b \in [281.75, 282.6] \)

 $-1.71  + 282.00 i$, which corresponds to forgetting to multiply the conjugate by the numerator.
\item \( a \in [-3.95, -2.35] \text{ and } b \in [3.75, 3.85] \)

 $-2.72  + 3.78 i$, which corresponds to forgetting to multiply the conjugate by the numerator and not computing the conjugate correctly.
\item \( a \in [-1.9, -1.15] \text{ and } b \in [3.85, 4.4] \)

* $-1.71  + 4.34 i$, which is the correct option.
\item \( a \in [-2.65, -2.2] \text{ and } b \in [32.6, 33.65] \)

 $-2.25  + 33.00 i$, which corresponds to just dividing the first term by the first term and the second by the second.
\item \( a \in [-111.15, -110] \text{ and } b \in [3.85, 4.4] \)

 $-111.00  + 4.34 i$, which corresponds to forgetting to multiply the conjugate by the numerator and using a plus instead of a minus in the denominator.
\end{enumerate}

\textbf{General Comment:} Multiply the numerator and denominator by the *conjugate* of the denominator, then simplify. For example, if we have $2+3i$, the conjugate is $2-3i$.
}
\litem{
Choose the \textbf{smallest} set of Complex numbers that the number below belongs to.
\[ \frac{\sqrt{70}}{15}+\sqrt{-4}i \]The solution is \( \text{Irrational} \), which is option A.\begin{enumerate}[label=\Alph*.]
\item \( \text{Irrational} \)

* This is the correct option!
\item \( \text{Pure Imaginary} \)

This is a Complex number $(a+bi)$ that \textbf{only} has an imaginary part like $2i$.
\item \( \text{Not a Complex Number} \)

This is not a number. The only non-Complex number we know is dividing by 0 as this is not a number!
\item \( \text{Nonreal Complex} \)

This is a Complex number $(a+bi)$ that is not Real (has $i$ as part of the number).
\item \( \text{Rational} \)

These are numbers that can be written as fraction of Integers (e.g., -2/3 + 5)
\end{enumerate}

\textbf{General Comment:} Be sure to simplify $i^2 = -1$. This may remove the imaginary portion for your number. If you are having trouble, you may want to look at the \textit{Subgroups of the Real Numbers} section.
}
\litem{
Simplify the expression below into the form $a+bi$. Then, choose the intervals that $a$ and $b$ belong to.
\[ (4 + 9 i)(-8 + 5 i) \]The solution is \( -77 - 52 i \), which is option D.\begin{enumerate}[label=\Alph*.]
\item \( a \in [11, 21] \text{ and } b \in [91, 94] \)

 $13 + 92 i$, which corresponds to adding a minus sign in the first term.
\item \( a \in [-37, -30] \text{ and } b \in [45, 46] \)

 $-32 + 45 i$, which corresponds to just multiplying the real terms to get the real part of the solution and the coefficients in the complex terms to get the complex part.
\item \( a \in [-78, -75] \text{ and } b \in [52, 58] \)

 $-77 + 52 i$, which corresponds to adding a minus sign in both terms.
\item \( a \in [-78, -75] \text{ and } b \in [-56, -51] \)

* $-77 - 52 i$, which is the correct option.
\item \( a \in [11, 21] \text{ and } b \in [-93, -87] \)

 $13 - 92 i$, which corresponds to adding a minus sign in the second term.
\end{enumerate}

\textbf{General Comment:} You can treat $i$ as a variable and distribute. Just remember that $i^2=-1$, so you can continue to reduce after you distribute.
}
\litem{
Simplify the expression below into the form $a+bi$. Then, choose the intervals that $a$ and $b$ belong to.
\[ \frac{-9 + 22 i}{6 + 3 i} \]The solution is \( 0.27  + 3.53 i \), which is option D.\begin{enumerate}[label=\Alph*.]
\item \( a \in [11, 12.5] \text{ and } b \in [2.5, 5] \)

 $12.00  + 3.53 i$, which corresponds to forgetting to multiply the conjugate by the numerator and using a plus instead of a minus in the denominator.
\item \( a \in [-3, -2] \text{ and } b \in [1, 3] \)

 $-2.67  + 2.33 i$, which corresponds to forgetting to multiply the conjugate by the numerator and not computing the conjugate correctly.
\item \( a \in [-0.5, 0.5] \text{ and } b \in [158, 159.5] \)

 $0.27  + 159.00 i$, which corresponds to forgetting to multiply the conjugate by the numerator.
\item \( a \in [-0.5, 0.5] \text{ and } b \in [2.5, 5] \)

* $0.27  + 3.53 i$, which is the correct option.
\item \( a \in [-2, -1] \text{ and } b \in [5.5, 7.5] \)

 $-1.50  + 7.33 i$, which corresponds to just dividing the first term by the first term and the second by the second.
\end{enumerate}

\textbf{General Comment:} Multiply the numerator and denominator by the *conjugate* of the denominator, then simplify. For example, if we have $2+3i$, the conjugate is $2-3i$.
}
\litem{
Simplify the expression below and choose the interval the simplification is contained within.
\[ 4 - 7 \div 11 * 3 - (17 * 20) \]The solution is \( -337.909 \), which is option A.\begin{enumerate}[label=\Alph*.]
\item \( [-339.21, -337.19] \)

* -337.909, which is the correct option.
\item \( [-337.32, -335.8] \)

 -336.212, which corresponds to an Order of Operations error: not reading left-to-right for multiplication/division.
\item \( [343.13, 343.92] \)

 343.788, which corresponds to not distributing addition and subtraction correctly.
\item \( [-300.51, -297.85] \)

 -298.182, which corresponds to not distributing a negative correctly.
\item \( \text{None of the above} \)

 You may have gotten this by making an unanticipated error. If you got a value that is not any of the others, please let the coordinator know so they can help you figure out what happened.
\end{enumerate}

\textbf{General Comment:} While you may remember (or were taught) PEMDAS is done in order, it is actually done as P/E/MD/AS. When we are at MD or AS, we read left to right.
}
\litem{
Choose the \textbf{smallest} set of Real numbers that the number below belongs to.
\[ -\sqrt{\frac{-1859}{13}} \]The solution is \( \text{Not a Real number} \), which is option D.\begin{enumerate}[label=\Alph*.]
\item \( \text{Rational} \)

These are numbers that can be written as fraction of Integers (e.g., -2/3)
\item \( \text{Integer} \)

These are the negative and positive counting numbers (..., -3, -2, -1, 0, 1, 2, 3, ...)
\item \( \text{Irrational} \)

These cannot be written as a fraction of Integers.
\item \( \text{Not a Real number} \)

* This is the correct option!
\item \( \text{Whole} \)

These are the counting numbers with 0 (0, 1, 2, 3, ...)
\end{enumerate}

\textbf{General Comment:} First, you \textbf{NEED} to simplify the expression. This question simplifies to $-\sqrt{143} i$. 
 
 Be sure you look at the simplified fraction and not just the decimal expansion. Numbers such as 13, 17, and 19 provide \textbf{long but repeating/terminating decimal expansions!} 
 
 The only ways to *not* be a Real number are: dividing by 0 or taking the square root of a negative number. 
 
 Irrational numbers are more than just square root of 3: adding or subtracting values from square root of 3 is also irrational.
}
\litem{
Choose the \textbf{smallest} set of Real numbers that the number below belongs to.
\[ -\sqrt{\frac{-1176}{14}} \]The solution is \( \text{Not a Real number} \), which is option A.\begin{enumerate}[label=\Alph*.]
\item \( \text{Not a Real number} \)

* This is the correct option!
\item \( \text{Irrational} \)

These cannot be written as a fraction of Integers.
\item \( \text{Rational} \)

These are numbers that can be written as fraction of Integers (e.g., -2/3)
\item \( \text{Integer} \)

These are the negative and positive counting numbers (..., -3, -2, -1, 0, 1, 2, 3, ...)
\item \( \text{Whole} \)

These are the counting numbers with 0 (0, 1, 2, 3, ...)
\end{enumerate}

\textbf{General Comment:} First, you \textbf{NEED} to simplify the expression. This question simplifies to $-\sqrt{84} i$. 
 
 Be sure you look at the simplified fraction and not just the decimal expansion. Numbers such as 13, 17, and 19 provide \textbf{long but repeating/terminating decimal expansions!} 
 
 The only ways to *not* be a Real number are: dividing by 0 or taking the square root of a negative number. 
 
 Irrational numbers are more than just square root of 3: adding or subtracting values from square root of 3 is also irrational.
}
\litem{
Choose the \textbf{smallest} set of Complex numbers that the number below belongs to.
\[ \sqrt{\frac{-880}{5}}+\sqrt{126} \]The solution is \( \text{Nonreal Complex} \), which is option B.\begin{enumerate}[label=\Alph*.]
\item \( \text{Not a Complex Number} \)

This is not a number. The only non-Complex number we know is dividing by 0 as this is not a number!
\item \( \text{Nonreal Complex} \)

* This is the correct option!
\item \( \text{Rational} \)

These are numbers that can be written as fraction of Integers (e.g., -2/3 + 5)
\item \( \text{Pure Imaginary} \)

This is a Complex number $(a+bi)$ that \textbf{only} has an imaginary part like $2i$.
\item \( \text{Irrational} \)

These cannot be written as a fraction of Integers. Remember: $\pi$ is not an Integer!
\end{enumerate}

\textbf{General Comment:} Be sure to simplify $i^2 = -1$. This may remove the imaginary portion for your number. If you are having trouble, you may want to look at the \textit{Subgroups of the Real Numbers} section.
}
\end{enumerate}

\end{document}