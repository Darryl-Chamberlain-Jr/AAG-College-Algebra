\documentclass[14pt]{extbook}
\usepackage{multicol, enumerate, enumitem, hyperref, color, soul, setspace, parskip, fancyhdr} %General Packages
\usepackage{amssymb, amsthm, amsmath, bbm, latexsym, units, mathtools} %Math Packages
\everymath{\displaystyle} %All math in Display Style
% Packages with additional options
\usepackage[headsep=0.5cm,headheight=12pt, left=1 in,right= 1 in,top= 1 in,bottom= 1 in]{geometry}
\usepackage[usenames,dvipsnames]{xcolor}
\usepackage{dashrule}  % Package to use the command below to create lines between items
\newcommand{\litem}[1]{\item#1\hspace*{-1cm}\rule{\textwidth}{0.4pt}}
\pagestyle{fancy}
\lhead{Makeup Progress Quiz 1}
\chead{}
\rhead{Version B}
\lfoot{6018-3080}
\cfoot{}
\rfoot{Spring 2021}
\begin{document}

\begin{enumerate}
\litem{
Simplify the expression below and choose the interval the simplification is contained within.\[ 10 - 7 \div 1 * 5 - (11 * 17) \]\begin{enumerate}[label=\Alph*.]
\item \( [-215, -203] \)
\item \( [184.6, 197.6] \)
\item \( [-178.4, -177.4] \)
\item \( [-612, -610] \)
\item \( \text{None of the above} \)

\end{enumerate} }
\litem{
Simplify the expression below into the form $a+bi$. Then, choose the intervals that $a$ and $b$ belong to.\[ (-4 + 9 i)(-6 - 10 i) \]\begin{enumerate}[label=\Alph*.]
\item \( a \in [24, 27] \text{ and } b \in [-93, -87] \)
\item \( a \in [113, 116] \text{ and } b \in [-18, -9] \)
\item \( a \in [-70, -65] \text{ and } b \in [93, 99] \)
\item \( a \in [-70, -65] \text{ and } b \in [-95, -92] \)
\item \( a \in [113, 116] \text{ and } b \in [11, 20] \)

\end{enumerate} }
\litem{
Simplify the expression below into the form $a+bi$. Then, choose the intervals that $a$ and $b$ belong to.\[ \frac{18 - 33 i}{-8 - i} \]\begin{enumerate}[label=\Alph*.]
\item \( a \in [-1.9, -1.15] \text{ and } b \in [281.75, 282.6] \)
\item \( a \in [-3.95, -2.35] \text{ and } b \in [3.75, 3.85] \)
\item \( a \in [-1.9, -1.15] \text{ and } b \in [3.85, 4.4] \)
\item \( a \in [-2.65, -2.2] \text{ and } b \in [32.6, 33.65] \)
\item \( a \in [-111.15, -110] \text{ and } b \in [3.85, 4.4] \)

\end{enumerate} }
\litem{
Choose the \textbf{smallest} set of Complex numbers that the number below belongs to.\[ \frac{\sqrt{70}}{15}+\sqrt{-4}i \]\begin{enumerate}[label=\Alph*.]
\item \( \text{Irrational} \)
\item \( \text{Pure Imaginary} \)
\item \( \text{Not a Complex Number} \)
\item \( \text{Nonreal Complex} \)
\item \( \text{Rational} \)

\end{enumerate} }
\litem{
Simplify the expression below into the form $a+bi$. Then, choose the intervals that $a$ and $b$ belong to.\[ (4 + 9 i)(-8 + 5 i) \]\begin{enumerate}[label=\Alph*.]
\item \( a \in [11, 21] \text{ and } b \in [91, 94] \)
\item \( a \in [-37, -30] \text{ and } b \in [45, 46] \)
\item \( a \in [-78, -75] \text{ and } b \in [52, 58] \)
\item \( a \in [-78, -75] \text{ and } b \in [-56, -51] \)
\item \( a \in [11, 21] \text{ and } b \in [-93, -87] \)

\end{enumerate} }
\litem{
Simplify the expression below into the form $a+bi$. Then, choose the intervals that $a$ and $b$ belong to.\[ \frac{-9 + 22 i}{6 + 3 i} \]\begin{enumerate}[label=\Alph*.]
\item \( a \in [11, 12.5] \text{ and } b \in [2.5, 5] \)
\item \( a \in [-3, -2] \text{ and } b \in [1, 3] \)
\item \( a \in [-0.5, 0.5] \text{ and } b \in [158, 159.5] \)
\item \( a \in [-0.5, 0.5] \text{ and } b \in [2.5, 5] \)
\item \( a \in [-2, -1] \text{ and } b \in [5.5, 7.5] \)

\end{enumerate} }
\litem{
Simplify the expression below and choose the interval the simplification is contained within.\[ 4 - 7 \div 11 * 3 - (17 * 20) \]\begin{enumerate}[label=\Alph*.]
\item \( [-339.21, -337.19] \)
\item \( [-337.32, -335.8] \)
\item \( [343.13, 343.92] \)
\item \( [-300.51, -297.85] \)
\item \( \text{None of the above} \)

\end{enumerate} }
\litem{
Choose the \textbf{smallest} set of Real numbers that the number below belongs to.\[ -\sqrt{\frac{-1859}{13}} \]\begin{enumerate}[label=\Alph*.]
\item \( \text{Rational} \)
\item \( \text{Integer} \)
\item \( \text{Irrational} \)
\item \( \text{Not a Real number} \)
\item \( \text{Whole} \)

\end{enumerate} }
\litem{
Choose the \textbf{smallest} set of Real numbers that the number below belongs to.\[ -\sqrt{\frac{-1176}{14}} \]\begin{enumerate}[label=\Alph*.]
\item \( \text{Not a Real number} \)
\item \( \text{Irrational} \)
\item \( \text{Rational} \)
\item \( \text{Integer} \)
\item \( \text{Whole} \)

\end{enumerate} }
\litem{
Choose the \textbf{smallest} set of Complex numbers that the number below belongs to.\[ \sqrt{\frac{-880}{5}}+\sqrt{126} \]\begin{enumerate}[label=\Alph*.]
\item \( \text{Not a Complex Number} \)
\item \( \text{Nonreal Complex} \)
\item \( \text{Rational} \)
\item \( \text{Pure Imaginary} \)
\item \( \text{Irrational} \)

\end{enumerate} }
\end{enumerate}

\end{document}