\documentclass{extbook}[14pt]
\usepackage{multicol, enumerate, enumitem, hyperref, color, soul, setspace, parskip, fancyhdr, amssymb, amsthm, amsmath, bbm, latexsym, units, mathtools}
\everymath{\displaystyle}
\usepackage[headsep=0.5cm,headheight=0cm, left=1 in,right= 1 in,top= 1 in,bottom= 1 in]{geometry}
\usepackage{dashrule}  % Package to use the command below to create lines between items
\newcommand{\litem}[1]{\item #1

\rule{\textwidth}{0.4pt}}
\pagestyle{fancy}
\lhead{}
\chead{Answer Key for Makeup Progress Quiz 1 Version C}
\rhead{}
\lfoot{6018-3080}
\cfoot{}
\rfoot{Spring 2021}
\begin{document}
\textbf{This key should allow you to understand why you choose the option you did (beyond just getting a question right or wrong). \href{https://xronos.clas.ufl.edu/mac1105spring2020/courseDescriptionAndMisc/Exams/LearningFromResults}{More instructions on how to use this key can be found here}.}

\textbf{If you have a suggestion to make the keys better, \href{https://forms.gle/CZkbZmPbC9XALEE88}{please fill out the short survey here}.}

\textit{Note: This key is auto-generated and may contain issues and/or errors. The keys are reviewed after each exam to ensure grading is done accurately. If there are issues (like duplicate options), they are noted in the offline gradebook. The keys are a work-in-progress to give students as many resources to improve as possible.}

\rule{\textwidth}{0.4pt}

\begin{enumerate}\litem{
Which of the following intervals describes the Domain of the function below?
\[ f(x) = -e^{x-1}+4 \]The solution is \( (-\infty, \infty) \), which is option E.\begin{enumerate}[label=\Alph*.]
\item \( (-\infty, a), a \in [4, 5] \)

$(-\infty, 4)$, which corresponds to using the correct vertical shift *if we wanted the Range*.
\item \( [a, \infty), a \in [-11, -2] \)

$[-4, \infty)$, which corresponds to using the negative vertical shift AND flipping the Range interval AND including the endpoint.
\item \( (a, \infty), a \in [-11, -2] \)

$(-4, \infty)$, which corresponds to using the negative vertical shift AND flipping the Range interval.
\item \( (-\infty, a], a \in [4, 5] \)

$(-\infty, 4]$, which corresponds to using the correct vertical shift *if we wanted the Range* AND including the endpoint.
\item \( (-\infty, \infty) \)

* This is the correct option.
\end{enumerate}

\textbf{General Comment:} \textbf{General Comments}: Domain of a basic exponential function is $(-\infty, \infty)$ while the Range is $(0, \infty)$. We can shift these intervals [and even flip when $a<0$!] to find the new Domain/Range.
}
\litem{
 Solve the equation for $x$ and choose the interval that contains $x$ (if it exists).
\[  12 = \sqrt[4]{\frac{5}{e^{7x}}} \]The solution is \( x = -1.19, \text{ which does not fit in any of the interval options.} \), which is option E.\begin{enumerate}[label=\Alph*.]
\item \( x \in [-7.96, -6.34] \)

$x = -7.087$, which corresponds to thinking you don't need to take the natural log of both sides before reducing, as if the right side already has a natural log.
\item \( x \in [0.38, 1.29] \)

$x = 1.190$, which is the negative of the correct solution.
\item \( x \in [-0.6, -0.15] \)

$x = -0.480$, which corresponds to treating any root as a square root.
\item \( \text{There is no Real solution to the equation.} \)

This corresponds to believing you cannot solve the equation.
\item \( \text{None of the above.} \)

* $x = -1.190$ is the correct solution and does not fit in any of the other intervals.
\end{enumerate}

\textbf{General Comment:} \textbf{General Comments}: After using the properties of logarithmic functions to break up the right-hand side, use $\ln(e) = 1$ to reduce the question to a linear function to solve. You can put $\ln(5)$ into a calculator if you are having trouble.
}
\litem{
Solve the equation for $x$ and choose the interval that contains the solution (if it exists).
\[ 5^{5x-2} = 16^{4x-3} \]The solution is \( x = 1.676 \), which is option D.\begin{enumerate}[label=\Alph*.]
\item \( x \in [-6.5, -4.1] \)

$x = -5.099$, which corresponds to distributing the $\ln(base)$ to the second term of the exponent only.
\item \( x \in [-0.5, 1] \)

$x = 0.329$, which corresponds to distributing the $\ln(base)$ to the first term of the exponent only.
\item \( x \in [-2.1, -0.3] \)

$x = -1.000$, which corresponds to solving the numerators as equal while ignoring the bases are different.
\item \( x \in [1.6, 2.3] \)

* $x = 1.676$, which is the correct option.
\item \( \text{There is no Real solution to the equation.} \)

This corresponds to believing there is no solution since the bases are not powers of each other.
\end{enumerate}

\textbf{General Comment:} \textbf{General Comments:} This question was written so that the bases could not be written the same. You will need to take the log of both sides.
}
\litem{
Solve the equation for $x$ and choose the interval that contains the solution (if it exists).
\[ 4^{-3x+4} = \left(\frac{1}{25}\right)^{2x-5} \]The solution is \( x = 4.629 \), which is option D.\begin{enumerate}[label=\Alph*.]
\item \( x \in [-6, -2.4] \)

$x = -3.949$, which corresponds to distributing the $\ln(base)$ to the first term of the exponent only.
\item \( x \in [1.2, 2.2] \)

$x = 1.800$, which corresponds to solving the numerators as equal while ignoring the bases are different.
\item \( x \in [-2.3, 0] \)

$x = -2.110$, which corresponds to distributing the $\ln(base)$ to the second term of the exponent only.
\item \( x \in [2.5, 6.2] \)

* $x = 4.629$, which is the correct option.
\item \( \text{There is no Real solution to the equation.} \)

This corresponds to believing there is no solution since the bases are not powers of each other.
\end{enumerate}

\textbf{General Comment:} \textbf{General Comments:} This question was written so that the bases could not be written the same. You will need to take the log of both sides.
}
\litem{
Solve the equation for $x$ and choose the interval that contains the solution (if it exists).
\[ \log_{2}{(-3x+6)}+6 = 3 \]The solution is \( x = 1.958 \), which is option D.\begin{enumerate}[label=\Alph*.]
\item \( x \in [-0.89, 0.21] \)

$x = -0.667$, which corresponds to ignoring the vertical shift when converting to exponential form.
\item \( x \in [-1.18, -0.88] \)

$x = -1.000$, which corresponds to reversing the base and exponent when converting.
\item \( x \in [-5.26, -4.67] \)

$x = -5.000$, which corresponds to reversing the base and exponent when converting and reversing the value with $x$.
\item \( x \in [1.37, 2.03] \)

* $x = 1.958$, which is the correct option.
\item \( \text{There is no Real solution to the equation.} \)

Corresponds to believing a negative coefficient within the log equation means there is no Real solution.
\end{enumerate}

\textbf{General Comment:} \textbf{General Comments:} First, get the equation in the form $\log_b{(cx+d)} = a$. Then, convert to $b^a = cx+d$ and solve.
}
\litem{
Which of the following intervals describes the Domain of the function below?
\[ f(x) = e^{x+8}+3 \]The solution is \( (-\infty, \infty) \), which is option E.\begin{enumerate}[label=\Alph*.]
\item \( (-\infty, a), a \in [3, 7] \)

$(-\infty, 3)$, which corresponds to using the correct vertical shift *if we wanted the Range*.
\item \( [a, \infty), a \in [-4, 0] \)

$[-3, \infty)$, which corresponds to using the negative vertical shift AND flipping the Range interval AND including the endpoint.
\item \( (-\infty, a], a \in [3, 7] \)

$(-\infty, 3]$, which corresponds to using the correct vertical shift *if we wanted the Range* AND including the endpoint.
\item \( (a, \infty), a \in [-4, 0] \)

$(-3, \infty)$, which corresponds to using the negative vertical shift AND flipping the Range interval.
\item \( (-\infty, \infty) \)

* This is the correct option.
\end{enumerate}

\textbf{General Comment:} \textbf{General Comments}: Domain of a basic exponential function is $(-\infty, \infty)$ while the Range is $(0, \infty)$. We can shift these intervals [and even flip when $a<0$!] to find the new Domain/Range.
}
\litem{
Which of the following intervals describes the Domain of the function below?
\[ f(x) = -\log_2{(x+9)}-6 \]The solution is \( (-9, \infty) \), which is option B.\begin{enumerate}[label=\Alph*.]
\item \( (-\infty, a], a \in [5.4, 8.1] \)

$(-\infty, 6]$, which corresponds to using the negative vertical shift AND including the endpoint AND flipping the domain.
\item \( (a, \infty), a \in [-11, -8.1] \)

* $(-9, \infty)$, which is the correct option.
\item \( (-\infty, a), a \in [7.4, 11.7] \)

$(-\infty, 9)$, which corresponds to flipping the Domain. Remember: the general for is $a*\log(x-h)+k$, \textbf{where $a$ does not affect the domain}.
\item \( [a, \infty), a \in [-6.7, -4.6] \)

$[-6, \infty)$, which corresponds to using the vertical shift when shifting the Domain AND including the endpoint.
\item \( (-\infty, \infty) \)

This corresponds to thinking of the range of the log function (or the domain of the exponential function).
\end{enumerate}

\textbf{General Comment:} \textbf{General Comments}: The domain of a basic logarithmic function is $(0, \infty)$ and the Range is $(-\infty, \infty)$. We can use shifts when finding the Domain, but the Range will always be all Real numbers.
}
\litem{
Solve the equation for $x$ and choose the interval that contains the solution (if it exists).
\[ \log_{4}{(2x+7)}+4 = 3 \]The solution is \( x = -3.375 \), which is option B.\begin{enumerate}[label=\Alph*.]
\item \( x \in [-3.21, -2.13] \)

$x = -3.000$, which corresponds to reversing the base and exponent when converting.
\item \( x \in [-3.57, -3.05] \)

* $x = -3.375$, which is the correct option.
\item \( x \in [28.2, 29.22] \)

$x = 28.500$, which corresponds to ignoring the vertical shift when converting to exponential form.
\item \( x \in [3.97, 4.72] \)

$x = 4.000$, which corresponds to reversing the base and exponent when converting and reversing the value with $x$.
\item \( \text{There is no Real solution to the equation.} \)

Corresponds to believing a negative coefficient within the log equation means there is no Real solution.
\end{enumerate}

\textbf{General Comment:} \textbf{General Comments:} First, get the equation in the form $\log_b{(cx+d)} = a$. Then, convert to $b^a = cx+d$ and solve.
}
\litem{
 Solve the equation for $x$ and choose the interval that contains $x$ (if it exists).
\[  10 = \ln{\sqrt[6]{\frac{11}{e^{9x}}}} \]The solution is \( x = -6.4 \), which is option A.\begin{enumerate}[label=\Alph*.]
\item \( x \in [-6.46, -5.94] \)

* $x = -6.400$, which is the correct option.
\item \( x \in [-2.11, -1.85] \)

$x = -1.956$, which corresponds to treating any root as a square root.
\item \( x \in [-1.82, -1.5] \)

$x = -1.801$, which corresponds to thinking you need to take the natural log of on the left before reducing.
\item \( \text{There is no Real solution to the equation.} \)

This corresponds to believing you cannot solve the equation.
\item \( \text{None of the above.} \)

This corresponds to making an unexpected error.
\end{enumerate}

\textbf{General Comment:} \textbf{General Comments}: After using the properties of logarithmic functions to break up the right-hand side, use $\ln(e) = 1$ to reduce the question to a linear function to solve. You can put $\ln(11)$ into a calculator if you are having trouble.
}
\litem{
Which of the following intervals describes the Range of the function below?
\[ f(x) = -\log_2{(x+2)}+5 \]The solution is \( (\infty, \infty) \), which is option E.\begin{enumerate}[label=\Alph*.]
\item \( (-\infty, a), a \in [-8.6, -4.7] \)

$(-\infty, -5)$, which corresponds to using the using the negative of vertical shift on $(0, \infty)$.
\item \( (-\infty, a), a \in [3.2, 6.4] \)

$(-\infty, 5)$, which corresponds to using the vertical shift while the Range is $(-\infty, \infty)$.
\item \( [a, \infty), a \in [1.6, 3.9] \)

$[2, \infty)$, which corresponds to using the negative of the horizontal shift AND including the endpoint.
\item \( [a, \infty), a \in [-4.5, -1.1] \)

$[5, \infty)$, which corresponds to using the flipped Domain AND including the endpoint.
\item \( (-\infty, \infty) \)

*This is the correct option.
\end{enumerate}

\textbf{General Comment:} \textbf{General Comments}: The domain of a basic logarithmic function is $(0, \infty)$ and the Range is $(-\infty, \infty)$. We can use shifts when finding the Domain, but the Range will always be all Real numbers.
}
\end{enumerate}

\end{document}