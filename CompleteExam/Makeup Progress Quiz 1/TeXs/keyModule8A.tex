\documentclass{extbook}[14pt]
\usepackage{multicol, enumerate, enumitem, hyperref, color, soul, setspace, parskip, fancyhdr, amssymb, amsthm, amsmath, bbm, latexsym, units, mathtools}
\everymath{\displaystyle}
\usepackage[headsep=0.5cm,headheight=0cm, left=1 in,right= 1 in,top= 1 in,bottom= 1 in]{geometry}
\usepackage{dashrule}  % Package to use the command below to create lines between items
\newcommand{\litem}[1]{\item #1

\rule{\textwidth}{0.4pt}}
\pagestyle{fancy}
\lhead{}
\chead{Answer Key for Makeup Progress Quiz 1 Version A}
\rhead{}
\lfoot{6018-3080}
\cfoot{}
\rfoot{Spring 2021}
\begin{document}
\textbf{This key should allow you to understand why you choose the option you did (beyond just getting a question right or wrong). \href{https://xronos.clas.ufl.edu/mac1105spring2020/courseDescriptionAndMisc/Exams/LearningFromResults}{More instructions on how to use this key can be found here}.}

\textbf{If you have a suggestion to make the keys better, \href{https://forms.gle/CZkbZmPbC9XALEE88}{please fill out the short survey here}.}

\textit{Note: This key is auto-generated and may contain issues and/or errors. The keys are reviewed after each exam to ensure grading is done accurately. If there are issues (like duplicate options), they are noted in the offline gradebook. The keys are a work-in-progress to give students as many resources to improve as possible.}

\rule{\textwidth}{0.4pt}

\begin{enumerate}\litem{
Which of the following intervals describes the Range of the function below?
\[ f(x) = e^{x-4}+6 \]The solution is \( (6, \infty) \), which is option D.\begin{enumerate}[label=\Alph*.]
\item \( (-\infty, a), a \in [-10, 1] \)

$(-\infty, -6)$, which corresponds to using the negative vertical shift AND flipping the Range interval.
\item \( [a, \infty), a \in [-1, 7] \)

$[6, \infty)$, which corresponds to including the endpoint.
\item \( (-\infty, a], a \in [-10, 1] \)

$(-\infty, -6]$, which corresponds to using the negative vertical shift AND flipping the Range interval AND including the endpoint.
\item \( (a, \infty), a \in [-1, 7] \)

* $(6, \infty)$, which is the correct option.
\item \( (-\infty, \infty) \)

This corresponds to confusing range of an exponential function with the domain of an exponential function.
\end{enumerate}

\textbf{General Comment:} \textbf{General Comments}: Domain of a basic exponential function is $(-\infty, \infty)$ while the Range is $(0, \infty)$. We can shift these intervals [and even flip when $a<0$!] to find the new Domain/Range.
}
\litem{
 Solve the equation for $x$ and choose the interval that contains $x$ (if it exists).
\[  6 = \ln{\sqrt[5]{\frac{18}{e^{8x}}}} \]The solution is \( x = -3.389 \), which is option B.\begin{enumerate}[label=\Alph*.]
\item \( x \in [-1.59, -1.31] \)

$x = -1.481$, which corresponds to thinking you need to take the natural log of on the left before reducing.
\item \( x \in [-3.57, -3.05] \)

* $x = -3.389$, which is the correct option.
\item \( x \in [-1.17, -0.82] \)

$x = -1.139$, which corresponds to treating any root as a square root.
\item \( \text{There is no Real solution to the equation.} \)

This corresponds to believing you cannot solve the equation.
\item \( \text{None of the above.} \)

This corresponds to making an unexpected error.
\end{enumerate}

\textbf{General Comment:} \textbf{General Comments}: After using the properties of logarithmic functions to break up the right-hand side, use $\ln(e) = 1$ to reduce the question to a linear function to solve. You can put $\ln(18)$ into a calculator if you are having trouble.
}
\litem{
Solve the equation for $x$ and choose the interval that contains the solution (if it exists).
\[ 4^{3x-5} = \left(\frac{1}{27}\right)^{2x+4} \]The solution is \( x = -0.582 \), which is option B.\begin{enumerate}[label=\Alph*.]
\item \( x \in [-6.9, -3.8] \)

$x = -6.252$, which corresponds to distributing the $\ln(base)$ to the second term of the exponent only.
\item \( x \in [-2.8, -0.1] \)

* $x = -0.582$, which is the correct option.
\item \( x \in [8.6, 9.9] \)

$x = 9.000$, which corresponds to solving the numerators as equal while ignoring the bases are different.
\item \( x \in [0.7, 3.3] \)

$x = 0.837$, which corresponds to distributing the $\ln(base)$ to the first term of the exponent only.
\item \( \text{There is no Real solution to the equation.} \)

This corresponds to believing there is no solution since the bases are not powers of each other.
\end{enumerate}

\textbf{General Comment:} \textbf{General Comments:} This question was written so that the bases could not be written the same. You will need to take the log of both sides.
}
\litem{
Solve the equation for $x$ and choose the interval that contains the solution (if it exists).
\[ 4^{2x+5} = 49^{3x-5} \]The solution is \( x = 2.964 \), which is option D.\begin{enumerate}[label=\Alph*.]
\item \( x \in [-0.88, 2.12] \)

$x = 1.123$, which corresponds to distributing the $\ln(base)$ to the first term of the exponent only.
\item \( x \in [8, 14] \)

$x = 10.000$, which corresponds to solving the numerators as equal while ignoring the bases are different.
\item \( x \in [22.39, 29.39] \)

$x = 26.391$, which corresponds to distributing the $\ln(base)$ to the second term of the exponent only.
\item \( x \in [1.96, 5.96] \)

* $x = 2.964$, which is the correct option.
\item \( \text{There is no Real solution to the equation.} \)

This corresponds to believing there is no solution since the bases are not powers of each other.
\end{enumerate}

\textbf{General Comment:} \textbf{General Comments:} This question was written so that the bases could not be written the same. You will need to take the log of both sides.
}
\litem{
Solve the equation for $x$ and choose the interval that contains the solution (if it exists).
\[ \log_{5}{(-4x+5)}+5 = 2 \]The solution is \( x = 1.248 \), which is option A.\begin{enumerate}[label=\Alph*.]
\item \( x \in [-3.75, 4.25] \)

* $x = 1.248$, which is the correct option.
\item \( x \in [61, 68] \)

$x = 62.000$, which corresponds to reversing the base and exponent when converting.
\item \( x \in [-11, -2] \)

$x = -5.000$, which corresponds to ignoring the vertical shift when converting to exponential form.
\item \( x \in [59.5, 60.5] \)

$x = 59.500$, which corresponds to reversing the base and exponent when converting and reversing the value with $x$.
\item \( \text{There is no Real solution to the equation.} \)

Corresponds to believing a negative coefficient within the log equation means there is no Real solution.
\end{enumerate}

\textbf{General Comment:} \textbf{General Comments:} First, get the equation in the form $\log_b{(cx+d)} = a$. Then, convert to $b^a = cx+d$ and solve.
}
\litem{
Which of the following intervals describes the Domain of the function below?
\[ f(x) = e^{x+7}-4 \]The solution is \( (-\infty, \infty) \), which is option E.\begin{enumerate}[label=\Alph*.]
\item \( (-\infty, a), a \in [-8, 2] \)

$(-\infty, -4)$, which corresponds to using the correct vertical shift *if we wanted the Range*.
\item \( (a, \infty), a \in [2, 8] \)

$(4, \infty)$, which corresponds to using the negative vertical shift AND flipping the Range interval.
\item \( (-\infty, a], a \in [-8, 2] \)

$(-\infty, -4]$, which corresponds to using the correct vertical shift *if we wanted the Range* AND including the endpoint.
\item \( [a, \infty), a \in [2, 8] \)

$[4, \infty)$, which corresponds to using the negative vertical shift AND flipping the Range interval AND including the endpoint.
\item \( (-\infty, \infty) \)

* This is the correct option.
\end{enumerate}

\textbf{General Comment:} \textbf{General Comments}: Domain of a basic exponential function is $(-\infty, \infty)$ while the Range is $(0, \infty)$. We can shift these intervals [and even flip when $a<0$!] to find the new Domain/Range.
}
\litem{
Which of the following intervals describes the Range of the function below?
\[ f(x) = \log_2{(x-6)}-1 \]The solution is \( (\infty, \infty) \), which is option E.\begin{enumerate}[label=\Alph*.]
\item \( (-\infty, a), a \in [-1.8, -0.2] \)

$(-\infty, -1)$, which corresponds to using the vertical shift while the Range is $(-\infty, \infty)$.
\item \( [a, \infty), a \in [4.3, 8] \)

$[-1, \infty)$, which corresponds to using the flipped Domain AND including the endpoint.
\item \( [a, \infty), a \in [-9.6, -3.2] \)

$[-6, \infty)$, which corresponds to using the negative of the horizontal shift AND including the endpoint.
\item \( (-\infty, a), a \in [0.1, 3.8] \)

$(-\infty, 1)$, which corresponds to using the using the negative of vertical shift on $(0, \infty)$.
\item \( (-\infty, \infty) \)

*This is the correct option.
\end{enumerate}

\textbf{General Comment:} \textbf{General Comments}: The domain of a basic logarithmic function is $(0, \infty)$ and the Range is $(-\infty, \infty)$. We can use shifts when finding the Domain, but the Range will always be all Real numbers.
}
\litem{
Solve the equation for $x$ and choose the interval that contains the solution (if it exists).
\[ \log_{5}{(3x+7)}+5 = 3 \]The solution is \( x = -2.320 \), which is option D.\begin{enumerate}[label=\Alph*.]
\item \( x \in [-10.33, -5.33] \)

$x = -8.333$, which corresponds to reversing the base and exponent when converting and reversing the value with $x$.
\item \( x \in [39.33, 40.33] \)

$x = 39.333$, which corresponds to ignoring the vertical shift when converting to exponential form.
\item \( x \in [-17, -11] \)

$x = -13.000$, which corresponds to reversing the base and exponent when converting.
\item \( x \in [-5.32, 1.68] \)

* $x = -2.320$, which is the correct option.
\item \( \text{There is no Real solution to the equation.} \)

Corresponds to believing a negative coefficient within the log equation means there is no Real solution.
\end{enumerate}

\textbf{General Comment:} \textbf{General Comments:} First, get the equation in the form $\log_b{(cx+d)} = a$. Then, convert to $b^a = cx+d$ and solve.
}
\litem{
 Solve the equation for $x$ and choose the interval that contains $x$ (if it exists).
\[  13 = \sqrt[5]{\frac{18}{e^{9x}}} \]The solution is \( x = -1.104, \text{ which does not fit in any of the interval options.} \), which is option E.\begin{enumerate}[label=\Alph*.]
\item \( x \in [0.8, 1.7] \)

$x = 1.104$, which is the negative of the correct solution.
\item \( x \in [-8.7, -7.1] \)

$x = -7.543$, which corresponds to thinking you don't need to take the natural log of both sides before reducing, as if the right side already has a natural log.
\item \( x \in [-1, 0.8] \)

$x = -0.249$, which corresponds to treating any root as a square root.
\item \( \text{There is no Real solution to the equation.} \)

This corresponds to believing you cannot solve the equation.
\item \( \text{None of the above.} \)

* $x = -1.104$ is the correct solution and does not fit in any of the other intervals.
\end{enumerate}

\textbf{General Comment:} \textbf{General Comments}: After using the properties of logarithmic functions to break up the right-hand side, use $\ln(e) = 1$ to reduce the question to a linear function to solve. You can put $\ln(18)$ into a calculator if you are having trouble.
}
\litem{
Which of the following intervals describes the Range of the function below?
\[ f(x) = -\log_2{(x-2)}-5 \]The solution is \( (\infty, \infty) \), which is option E.\begin{enumerate}[label=\Alph*.]
\item \( [a, \infty), a \in [-0.4, 2.5] \)

$[-5, \infty)$, which corresponds to using the flipped Domain AND including the endpoint.
\item \( (-\infty, a), a \in [4.1, 7.1] \)

$(-\infty, 5)$, which corresponds to using the using the negative of vertical shift on $(0, \infty)$.
\item \( (-\infty, a), a \in [-7.7, -3.9] \)

$(-\infty, -5)$, which corresponds to using the vertical shift while the Range is $(-\infty, \infty)$.
\item \( [a, \infty), a \in [-3.4, -1.1] \)

$[-2, \infty)$, which corresponds to using the negative of the horizontal shift AND including the endpoint.
\item \( (-\infty, \infty) \)

*This is the correct option.
\end{enumerate}

\textbf{General Comment:} \textbf{General Comments}: The domain of a basic logarithmic function is $(0, \infty)$ and the Range is $(-\infty, \infty)$. We can use shifts when finding the Domain, but the Range will always be all Real numbers.
}
\end{enumerate}

\end{document}