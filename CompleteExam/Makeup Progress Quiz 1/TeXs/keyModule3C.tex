\documentclass{extbook}[14pt]
\usepackage{multicol, enumerate, enumitem, hyperref, color, soul, setspace, parskip, fancyhdr, amssymb, amsthm, amsmath, bbm, latexsym, units, mathtools}
\everymath{\displaystyle}
\usepackage[headsep=0.5cm,headheight=0cm, left=1 in,right= 1 in,top= 1 in,bottom= 1 in]{geometry}
\usepackage{dashrule}  % Package to use the command below to create lines between items
\newcommand{\litem}[1]{\item #1

\rule{\textwidth}{0.4pt}}
\pagestyle{fancy}
\lhead{}
\chead{Answer Key for Makeup Progress Quiz 1 Version C}
\rhead{}
\lfoot{6018-3080}
\cfoot{}
\rfoot{Spring 2021}
\begin{document}
\textbf{This key should allow you to understand why you choose the option you did (beyond just getting a question right or wrong). \href{https://xronos.clas.ufl.edu/mac1105spring2020/courseDescriptionAndMisc/Exams/LearningFromResults}{More instructions on how to use this key can be found here}.}

\textbf{If you have a suggestion to make the keys better, \href{https://forms.gle/CZkbZmPbC9XALEE88}{please fill out the short survey here}.}

\textit{Note: This key is auto-generated and may contain issues and/or errors. The keys are reviewed after each exam to ensure grading is done accurately. If there are issues (like duplicate options), they are noted in the offline gradebook. The keys are a work-in-progress to give students as many resources to improve as possible.}

\rule{\textwidth}{0.4pt}

\begin{enumerate}\litem{
Solve the linear inequality below. Then, choose the constant and interval combination that describes the solution set.
\[ \frac{-10}{2} - \frac{3}{7} x \leq \frac{8}{5} x + \frac{5}{9} \]The solution is \( [-2.739, \infty) \), which is option A.\begin{enumerate}[label=\Alph*.]
\item \( [a, \infty), \text{ where } a \in [-2.74, 1.26] \)

* $[-2.739, \infty)$, which is the correct option.
\item \( (-\infty, a], \text{ where } a \in [0.74, 3.74] \)

 $(-\infty, 2.739]$, which corresponds to switching the direction of the interval AND negating the endpoint. You likely did this if you did not flip the inequality when dividing by a negative as well as not moving values over to a side properly.
\item \( (-\infty, a], \text{ where } a \in [-3.74, -1.74] \)

 $(-\infty, -2.739]$, which corresponds to switching the direction of the interval. You likely did this if you did not flip the inequality when dividing by a negative!
\item \( [a, \infty), \text{ where } a \in [1.74, 5.74] \)

 $[2.739, \infty)$, which corresponds to negating the endpoint of the solution.
\item \( \text{None of the above}. \)

You may have chosen this if you thought the inequality did not match the ends of the intervals.
\end{enumerate}

\textbf{General Comment:} Remember that less/greater than or equal to includes the endpoint, while less/greater do not. Also, remember that you need to flip the inequality when you multiply or divide by a negative.
}
\litem{
Using an interval or intervals, describe all the $x$-values within or including a distance of the given values.
\[ \text{ More than } 5 \text{ units from the number } -8. \]The solution is \( (-\infty, -13) \cup (-3, \infty) \), which is option C.\begin{enumerate}[label=\Alph*.]
\item \( [-13, -3] \)

This describes the values no more than 5 from -8
\item \( (-\infty, -13] \cup [-3, \infty) \)

This describes the values no less than 5 from -8
\item \( (-\infty, -13) \cup (-3, \infty) \)

This describes the values more than 5 from -8
\item \( (-13, -3) \)

This describes the values less than 5 from -8
\item \( \text{None of the above} \)

You likely thought the values in the interval were not correct.
\end{enumerate}

\textbf{General Comment:} When thinking about this language, it helps to draw a number line and try points.
}
\litem{
Using an interval or intervals, describe all the $x$-values within or including a distance of the given values.
\[ \text{ No less than } 5 \text{ units from the number } -10. \]The solution is \( (-\infty, -15] \cup [-5, \infty) \), which is option B.\begin{enumerate}[label=\Alph*.]
\item \( (-\infty, -15) \cup (-5, \infty) \)

This describes the values more than 5 from -10
\item \( (-\infty, -15] \cup [-5, \infty) \)

This describes the values no less than 5 from -10
\item \( (-15, -5) \)

This describes the values less than 5 from -10
\item \( [-15, -5] \)

This describes the values no more than 5 from -10
\item \( \text{None of the above} \)

You likely thought the values in the interval were not correct.
\end{enumerate}

\textbf{General Comment:} When thinking about this language, it helps to draw a number line and try points.
}
\litem{
Solve the linear inequality below. Then, choose the constant and interval combination that describes the solution set.
\[ \frac{-6}{2} - \frac{7}{9} x < \frac{-3}{3} x + \frac{9}{8} \]The solution is \( (-\infty, 18.562) \), which is option A.\begin{enumerate}[label=\Alph*.]
\item \( (-\infty, a), \text{ where } a \in [15.56, 24.56] \)

* $(-\infty, 18.562)$, which is the correct option.
\item \( (-\infty, a), \text{ where } a \in [-20.56, -15.56] \)

 $(-\infty, -18.562)$, which corresponds to negating the endpoint of the solution.
\item \( (a, \infty), \text{ where } a \in [13.56, 21.56] \)

 $(18.562, \infty)$, which corresponds to switching the direction of the interval. You likely did this if you did not flip the inequality when dividing by a negative!
\item \( (a, \infty), \text{ where } a \in [-19.56, -15.56] \)

 $(-18.562, \infty)$, which corresponds to switching the direction of the interval AND negating the endpoint. You likely did this if you did not flip the inequality when dividing by a negative as well as not moving values over to a side properly.
\item \( \text{None of the above}. \)

You may have chosen this if you thought the inequality did not match the ends of the intervals.
\end{enumerate}

\textbf{General Comment:} Remember that less/greater than or equal to includes the endpoint, while less/greater do not. Also, remember that you need to flip the inequality when you multiply or divide by a negative.
}
\litem{
Solve the linear inequality below. Then, choose the constant and interval combination that describes the solution set.
\[ -3 + 6 x < \frac{29 x - 7}{4} \leq 8 + 4 x \]The solution is \( \text{None of the above.} \), which is option E.\begin{enumerate}[label=\Alph*.]
\item \( (-\infty, a) \cup [b, \infty), \text{ where } a \in [-0.2, 2.1] \text{ and } b \in [-6, -2] \)

$(-\infty, 1.00) \cup [-3.00, \infty)$, which corresponds to displaying the and-inequality as an or-inequality and getting negatives of the actual endpoints.
\item \( (a, b], \text{ where } a \in [0, 4] \text{ and } b \in [-5, -2] \)

$(1.00, -3.00]$, which is the correct interval but negatives of the actual endpoints.
\item \( [a, b), \text{ where } a \in [0.7, 2.2] \text{ and } b \in [-8, 0] \)

$[1.00, -3.00)$, which corresponds to flipping the inequality and getting negatives of the actual endpoints.
\item \( (-\infty, a] \cup (b, \infty), \text{ where } a \in [0.6, 2.5] \text{ and } b \in [-9, 0] \)

$(-\infty, 1.00] \cup (-3.00, \infty)$, which corresponds to displaying the and-inequality as an or-inequality AND flipping the inequality AND getting negatives of the actual endpoints.
\item \( \text{None of the above.} \)

* This is correct as the answer should be $(-1.00, 3.00]$.
\end{enumerate}

\textbf{General Comment:} To solve, you will need to break up the compound inequality into two inequalities. Be sure to keep track of the inequality! It may be best to draw a number line and graph your solution.
}
\litem{
Solve the linear inequality below. Then, choose the constant and interval combination that describes the solution set.
\[ -9 + 6 x > 8 x \text{ or } -3 + 3 x < 4 x \]The solution is \( (-\infty, -4.5) \text{ or } (-3.0, \infty) \), which is option A.\begin{enumerate}[label=\Alph*.]
\item \( (-\infty, a) \cup (b, \infty), \text{ where } a \in [-5.5, -1.5] \text{ and } b \in [-6, -1] \)

 * Correct option.
\item \( (-\infty, a] \cup [b, \infty), \text{ where } a \in [-4.5, -1.5] \text{ and } b \in [-6, 0] \)

Corresponds to including the endpoints (when they should be excluded).
\item \( (-\infty, a) \cup (b, \infty), \text{ where } a \in [-1, 5] \text{ and } b \in [2.5, 6.5] \)

Corresponds to inverting the inequality and negating the solution.
\item \( (-\infty, a] \cup [b, \infty), \text{ where } a \in [0, 8] \text{ and } b \in [-0.5, 5.5] \)

Corresponds to including the endpoints AND negating.
\item \( (-\infty, \infty) \)

Corresponds to the variable canceling, which does not happen in this instance.
\end{enumerate}

\textbf{General Comment:} When multiplying or dividing by a negative, flip the sign.
}
\litem{
Solve the linear inequality below. Then, choose the constant and interval combination that describes the solution set.
\[ -10x -3 \leq 8x + 10 \]The solution is \( [-0.722, \infty) \), which is option C.\begin{enumerate}[label=\Alph*.]
\item \( (-\infty, a], \text{ where } a \in [-1.05, 0.66] \)

 $(-\infty, -0.722]$, which corresponds to switching the direction of the interval. You likely did this if you did not flip the inequality when dividing by a negative!
\item \( (-\infty, a], \text{ where } a \in [0.02, 1.01] \)

 $(-\infty, 0.722]$, which corresponds to switching the direction of the interval AND negating the endpoint. You likely did this if you did not flip the inequality when dividing by a negative as well as not moving values over to a side properly.
\item \( [a, \infty), \text{ where } a \in [-1.14, 0.02] \)

* $[-0.722, \infty)$, which is the correct option.
\item \( [a, \infty), \text{ where } a \in [0.57, 2.51] \)

 $[0.722, \infty)$, which corresponds to negating the endpoint of the solution.
\item \( \text{None of the above}. \)

You may have chosen this if you thought the inequality did not match the ends of the intervals.
\end{enumerate}

\textbf{General Comment:} Remember that less/greater than or equal to includes the endpoint, while less/greater do not. Also, remember that you need to flip the inequality when you multiply or divide by a negative.
}
\litem{
Solve the linear inequality below. Then, choose the constant and interval combination that describes the solution set.
\[ -9 + 8 x < \frac{44 x - 6}{5} \leq 5 + 4 x \]The solution is \( (-9.75, 1.29] \), which is option C.\begin{enumerate}[label=\Alph*.]
\item \( (-\infty, a) \cup [b, \infty), \text{ where } a \in [-12.75, -5.75] \text{ and } b \in [1.29, 4.29] \)

$(-\infty, -9.75) \cup [1.29, \infty)$, which corresponds to displaying the and-inequality as an or-inequality.
\item \( (-\infty, a] \cup (b, \infty), \text{ where } a \in [-12.75, -8.75] \text{ and } b \in [-0.71, 2.29] \)

$(-\infty, -9.75] \cup (1.29, \infty)$, which corresponds to displaying the and-inequality as an or-inequality AND flipping the inequality.
\item \( (a, b], \text{ where } a \in [-11.75, -5.75] \text{ and } b \in [0, 3] \)

* $(-9.75, 1.29]$, which is the correct option.
\item \( [a, b), \text{ where } a \in [-11.75, -6.75] \text{ and } b \in [0.29, 4.29] \)

$[-9.75, 1.29)$, which corresponds to flipping the inequality.
\item \( \text{None of the above.} \)


\end{enumerate}

\textbf{General Comment:} To solve, you will need to break up the compound inequality into two inequalities. Be sure to keep track of the inequality! It may be best to draw a number line and graph your solution.
}
\litem{
Solve the linear inequality below. Then, choose the constant and interval combination that describes the solution set.
\[ -6 + 7 x > 8 x \text{ or } 8 + 8 x < 11 x \]The solution is \( (-\infty, -6.0) \text{ or } (2.667, \infty) \), which is option C.\begin{enumerate}[label=\Alph*.]
\item \( (-\infty, a) \cup (b, \infty), \text{ where } a \in [-3.67, -0.67] \text{ and } b \in [6, 9] \)

Corresponds to inverting the inequality and negating the solution.
\item \( (-\infty, a] \cup [b, \infty), \text{ where } a \in [-2.67, 3.33] \text{ and } b \in [4, 8] \)

Corresponds to including the endpoints AND negating.
\item \( (-\infty, a) \cup (b, \infty), \text{ where } a \in [-8, -4] \text{ and } b \in [0.67, 3.67] \)

 * Correct option.
\item \( (-\infty, a] \cup [b, \infty), \text{ where } a \in [-8, -4] \text{ and } b \in [-1.33, 3.67] \)

Corresponds to including the endpoints (when they should be excluded).
\item \( (-\infty, \infty) \)

Corresponds to the variable canceling, which does not happen in this instance.
\end{enumerate}

\textbf{General Comment:} When multiplying or dividing by a negative, flip the sign.
}
\litem{
Solve the linear inequality below. Then, choose the constant and interval combination that describes the solution set.
\[ 4x -10 \geq 10x -5 \]The solution is \( (-\infty, -0.833] \), which is option D.\begin{enumerate}[label=\Alph*.]
\item \( [a, \infty), \text{ where } a \in [-0.5, 2.7] \)

 $[0.833, \infty)$, which corresponds to switching the direction of the interval AND negating the endpoint. You likely did this if you did not flip the inequality when dividing by a negative as well as not moving values over to a side properly.
\item \( [a, \infty), \text{ where } a \in [-2.7, 0.4] \)

 $[-0.833, \infty)$, which corresponds to switching the direction of the interval. You likely did this if you did not flip the inequality when dividing by a negative!
\item \( (-\infty, a], \text{ where } a \in [0.7, 2.2] \)

 $(-\infty, 0.833]$, which corresponds to negating the endpoint of the solution.
\item \( (-\infty, a], \text{ where } a \in [-2.2, 0.8] \)

* $(-\infty, -0.833]$, which is the correct option.
\item \( \text{None of the above}. \)

You may have chosen this if you thought the inequality did not match the ends of the intervals.
\end{enumerate}

\textbf{General Comment:} Remember that less/greater than or equal to includes the endpoint, while less/greater do not. Also, remember that you need to flip the inequality when you multiply or divide by a negative.
}
\end{enumerate}

\end{document}