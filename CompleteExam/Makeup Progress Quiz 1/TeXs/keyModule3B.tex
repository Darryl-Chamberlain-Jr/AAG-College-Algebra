\documentclass{extbook}[14pt]
\usepackage{multicol, enumerate, enumitem, hyperref, color, soul, setspace, parskip, fancyhdr, amssymb, amsthm, amsmath, bbm, latexsym, units, mathtools}
\everymath{\displaystyle}
\usepackage[headsep=0.5cm,headheight=0cm, left=1 in,right= 1 in,top= 1 in,bottom= 1 in]{geometry}
\usepackage{dashrule}  % Package to use the command below to create lines between items
\newcommand{\litem}[1]{\item #1

\rule{\textwidth}{0.4pt}}
\pagestyle{fancy}
\lhead{}
\chead{Answer Key for Makeup Progress Quiz 1 Version B}
\rhead{}
\lfoot{6018-3080}
\cfoot{}
\rfoot{Spring 2021}
\begin{document}
\textbf{This key should allow you to understand why you choose the option you did (beyond just getting a question right or wrong). \href{https://xronos.clas.ufl.edu/mac1105spring2020/courseDescriptionAndMisc/Exams/LearningFromResults}{More instructions on how to use this key can be found here}.}

\textbf{If you have a suggestion to make the keys better, \href{https://forms.gle/CZkbZmPbC9XALEE88}{please fill out the short survey here}.}

\textit{Note: This key is auto-generated and may contain issues and/or errors. The keys are reviewed after each exam to ensure grading is done accurately. If there are issues (like duplicate options), they are noted in the offline gradebook. The keys are a work-in-progress to give students as many resources to improve as possible.}

\rule{\textwidth}{0.4pt}

\begin{enumerate}\litem{
Solve the linear inequality below. Then, choose the constant and interval combination that describes the solution set.
\[ \frac{-8}{6} - \frac{6}{5} x \leq \frac{8}{7} x + \frac{8}{2} \]The solution is \( [-2.276, \infty) \), which is option C.\begin{enumerate}[label=\Alph*.]
\item \( (-\infty, a], \text{ where } a \in [0.28, 4.28] \)

 $(-\infty, 2.276]$, which corresponds to switching the direction of the interval AND negating the endpoint. You likely did this if you did not flip the inequality when dividing by a negative as well as not moving values over to a side properly.
\item \( (-\infty, a], \text{ where } a \in [-3.28, 0.72] \)

 $(-\infty, -2.276]$, which corresponds to switching the direction of the interval. You likely did this if you did not flip the inequality when dividing by a negative!
\item \( [a, \infty), \text{ where } a \in [-3.28, -1.28] \)

* $[-2.276, \infty)$, which is the correct option.
\item \( [a, \infty), \text{ where } a \in [0.28, 6.28] \)

 $[2.276, \infty)$, which corresponds to negating the endpoint of the solution.
\item \( \text{None of the above}. \)

You may have chosen this if you thought the inequality did not match the ends of the intervals.
\end{enumerate}

\textbf{General Comment:} Remember that less/greater than or equal to includes the endpoint, while less/greater do not. Also, remember that you need to flip the inequality when you multiply or divide by a negative.
}
\litem{
Using an interval or intervals, describe all the $x$-values within or including a distance of the given values.
\[ \text{ No less than } 3 \text{ units from the number } 7. \]The solution is \( (-\infty, 4] \cup [10, \infty) \), which is option C.\begin{enumerate}[label=\Alph*.]
\item \( (4, 10) \)

This describes the values less than 3 from 7
\item \( (-\infty, 4) \cup (10, \infty) \)

This describes the values more than 3 from 7
\item \( (-\infty, 4] \cup [10, \infty) \)

This describes the values no less than 3 from 7
\item \( [4, 10] \)

This describes the values no more than 3 from 7
\item \( \text{None of the above} \)

You likely thought the values in the interval were not correct.
\end{enumerate}

\textbf{General Comment:} When thinking about this language, it helps to draw a number line and try points.
}
\litem{
Using an interval or intervals, describe all the $x$-values within or including a distance of the given values.
\[ \text{ No more than } 10 \text{ units from the number } 5. \]The solution is \( [-5, 15] \), which is option D.\begin{enumerate}[label=\Alph*.]
\item \( (-\infty, -5] \cup [15, \infty) \)

This describes the values no less than 10 from 5
\item \( (-\infty, -5) \cup (15, \infty) \)

This describes the values more than 10 from 5
\item \( (-5, 15) \)

This describes the values less than 10 from 5
\item \( [-5, 15] \)

This describes the values no more than 10 from 5
\item \( \text{None of the above} \)

You likely thought the values in the interval were not correct.
\end{enumerate}

\textbf{General Comment:} When thinking about this language, it helps to draw a number line and try points.
}
\litem{
Solve the linear inequality below. Then, choose the constant and interval combination that describes the solution set.
\[ \frac{-6}{9} - \frac{8}{5} x > \frac{-4}{7} x - \frac{9}{4} \]The solution is \( (-\infty, 1.539) \), which is option A.\begin{enumerate}[label=\Alph*.]
\item \( (-\infty, a), \text{ where } a \in [0.8, 3.9] \)

* $(-\infty, 1.539)$, which is the correct option.
\item \( (a, \infty), \text{ where } a \in [-1.54, 0.46] \)

 $(-1.539, \infty)$, which corresponds to switching the direction of the interval AND negating the endpoint. You likely did this if you did not flip the inequality when dividing by a negative as well as not moving values over to a side properly.
\item \( (-\infty, a), \text{ where } a \in [-3.7, -0.4] \)

 $(-\infty, -1.539)$, which corresponds to negating the endpoint of the solution.
\item \( (a, \infty), \text{ where } a \in [0.54, 3.54] \)

 $(1.539, \infty)$, which corresponds to switching the direction of the interval. You likely did this if you did not flip the inequality when dividing by a negative!
\item \( \text{None of the above}. \)

You may have chosen this if you thought the inequality did not match the ends of the intervals.
\end{enumerate}

\textbf{General Comment:} Remember that less/greater than or equal to includes the endpoint, while less/greater do not. Also, remember that you need to flip the inequality when you multiply or divide by a negative.
}
\litem{
Solve the linear inequality below. Then, choose the constant and interval combination that describes the solution set.
\[ 7 + 4 x \leq \frac{31 x - 5}{4} < 7 + 7 x \]The solution is \( [2.20, 11.00) \), which is option C.\begin{enumerate}[label=\Alph*.]
\item \( (a, b], \text{ where } a \in [1.2, 7.2] \text{ and } b \in [10, 14] \)

$(2.20, 11.00]$, which corresponds to flipping the inequality.
\item \( (-\infty, a) \cup [b, \infty), \text{ where } a \in [0.2, 7.2] \text{ and } b \in [11, 12] \)

$(-\infty, 2.20) \cup [11.00, \infty)$, which corresponds to displaying the and-inequality as an or-inequality AND flipping the inequality.
\item \( [a, b), \text{ where } a \in [2.2, 3.2] \text{ and } b \in [10, 13] \)

$[2.20, 11.00)$, which is the correct option.
\item \( (-\infty, a] \cup (b, \infty), \text{ where } a \in [-0.2, 3] \text{ and } b \in [11, 12] \)

$(-\infty, 2.20] \cup (11.00, \infty)$, which corresponds to displaying the and-inequality as an or-inequality.
\item \( \text{None of the above.} \)


\end{enumerate}

\textbf{General Comment:} To solve, you will need to break up the compound inequality into two inequalities. Be sure to keep track of the inequality! It may be best to draw a number line and graph your solution.
}
\litem{
Solve the linear inequality below. Then, choose the constant and interval combination that describes the solution set.
\[ -7 + 8 x > 9 x \text{ or } 5 + 5 x < 8 x \]The solution is \( (-\infty, -7.0) \text{ or } (1.667, \infty) \), which is option D.\begin{enumerate}[label=\Alph*.]
\item \( (-\infty, a] \cup [b, \infty), \text{ where } a \in [-5.67, 3.33] \text{ and } b \in [7, 8] \)

Corresponds to including the endpoints AND negating.
\item \( (-\infty, a) \cup (b, \infty), \text{ where } a \in [-1.67, -0.67] \text{ and } b \in [4, 10] \)

Corresponds to inverting the inequality and negating the solution.
\item \( (-\infty, a] \cup [b, \infty), \text{ where } a \in [-8, -5] \text{ and } b \in [-4.33, 3.67] \)

Corresponds to including the endpoints (when they should be excluded).
\item \( (-\infty, a) \cup (b, \infty), \text{ where } a \in [-8, -3] \text{ and } b \in [-1.33, 4.67] \)

 * Correct option.
\item \( (-\infty, \infty) \)

Corresponds to the variable canceling, which does not happen in this instance.
\end{enumerate}

\textbf{General Comment:} When multiplying or dividing by a negative, flip the sign.
}
\litem{
Solve the linear inequality below. Then, choose the constant and interval combination that describes the solution set.
\[ -8x + 3 \geq -3x + 8 \]The solution is \( (-\infty, -1.0] \), which is option A.\begin{enumerate}[label=\Alph*.]
\item \( (-\infty, a], \text{ where } a \in [-5.4, 0.9] \)

* $(-\infty, -1.0]$, which is the correct option.
\item \( [a, \infty), \text{ where } a \in [-2.2, -0.6] \)

 $[-1.0, \infty)$, which corresponds to switching the direction of the interval. You likely did this if you did not flip the inequality when dividing by a negative!
\item \( (-\infty, a], \text{ where } a \in [0.3, 1.2] \)

 $(-\infty, 1.0]$, which corresponds to negating the endpoint of the solution.
\item \( [a, \infty), \text{ where } a \in [0, 4.1] \)

 $[1.0, \infty)$, which corresponds to switching the direction of the interval AND negating the endpoint. You likely did this if you did not flip the inequality when dividing by a negative as well as not moving values over to a side properly.
\item \( \text{None of the above}. \)

You may have chosen this if you thought the inequality did not match the ends of the intervals.
\end{enumerate}

\textbf{General Comment:} Remember that less/greater than or equal to includes the endpoint, while less/greater do not. Also, remember that you need to flip the inequality when you multiply or divide by a negative.
}
\litem{
Solve the linear inequality below. Then, choose the constant and interval combination that describes the solution set.
\[ -6 - 3 x \leq \frac{-10 x - 4}{5} < 5 - 6 x \]The solution is \( [-5.20, 1.45) \), which is option C.\begin{enumerate}[label=\Alph*.]
\item \( (-\infty, a] \cup (b, \infty), \text{ where } a \in [-6.2, -4.2] \text{ and } b \in [0.45, 5.45] \)

$(-\infty, -5.20] \cup (1.45, \infty)$, which corresponds to displaying the and-inequality as an or-inequality.
\item \( (a, b], \text{ where } a \in [-6.2, -4.2] \text{ and } b \in [1.1, 1.6] \)

$(-5.20, 1.45]$, which corresponds to flipping the inequality.
\item \( [a, b), \text{ where } a \in [-5.2, -3.2] \text{ and } b \in [1.45, 7.45] \)

$[-5.20, 1.45)$, which is the correct option.
\item \( (-\infty, a) \cup [b, \infty), \text{ where } a \in [-7.2, -2.2] \text{ and } b \in [1.45, 3.45] \)

$(-\infty, -5.20) \cup [1.45, \infty)$, which corresponds to displaying the and-inequality as an or-inequality AND flipping the inequality.
\item \( \text{None of the above.} \)


\end{enumerate}

\textbf{General Comment:} To solve, you will need to break up the compound inequality into two inequalities. Be sure to keep track of the inequality! It may be best to draw a number line and graph your solution.
}
\litem{
Solve the linear inequality below. Then, choose the constant and interval combination that describes the solution set.
\[ -7 + 7 x > 8 x \text{ or } 5 + 6 x < 9 x \]The solution is \( (-\infty, -7.0) \text{ or } (1.667, \infty) \), which is option B.\begin{enumerate}[label=\Alph*.]
\item \( (-\infty, a] \cup [b, \infty), \text{ where } a \in [-7, -4] \text{ and } b \in [-0.33, 3.67] \)

Corresponds to including the endpoints (when they should be excluded).
\item \( (-\infty, a) \cup (b, \infty), \text{ where } a \in [-7, -5] \text{ and } b \in [-0.33, 6.67] \)

 * Correct option.
\item \( (-\infty, a) \cup (b, \infty), \text{ where } a \in [-1.67, -0.67] \text{ and } b \in [3, 12] \)

Corresponds to inverting the inequality and negating the solution.
\item \( (-\infty, a] \cup [b, \infty), \text{ where } a \in [-2.67, 0.33] \text{ and } b \in [7, 9] \)

Corresponds to including the endpoints AND negating.
\item \( (-\infty, \infty) \)

Corresponds to the variable canceling, which does not happen in this instance.
\end{enumerate}

\textbf{General Comment:} When multiplying or dividing by a negative, flip the sign.
}
\litem{
Solve the linear inequality below. Then, choose the constant and interval combination that describes the solution set.
\[ -3x -7 \leq 10x -6 \]The solution is \( [-0.077, \infty) \), which is option B.\begin{enumerate}[label=\Alph*.]
\item \( [a, \infty), \text{ where } a \in [0.01, 0.23] \)

 $[0.077, \infty)$, which corresponds to negating the endpoint of the solution.
\item \( [a, \infty), \text{ where } a \in [-0.27, -0.03] \)

* $[-0.077, \infty)$, which is the correct option.
\item \( (-\infty, a], \text{ where } a \in [-0.32, 0.05] \)

 $(-\infty, -0.077]$, which corresponds to switching the direction of the interval. You likely did this if you did not flip the inequality when dividing by a negative!
\item \( (-\infty, a], \text{ where } a \in [-0.04, 0.5] \)

 $(-\infty, 0.077]$, which corresponds to switching the direction of the interval AND negating the endpoint. You likely did this if you did not flip the inequality when dividing by a negative as well as not moving values over to a side properly.
\item \( \text{None of the above}. \)

You may have chosen this if you thought the inequality did not match the ends of the intervals.
\end{enumerate}

\textbf{General Comment:} Remember that less/greater than or equal to includes the endpoint, while less/greater do not. Also, remember that you need to flip the inequality when you multiply or divide by a negative.
}
\end{enumerate}

\end{document}