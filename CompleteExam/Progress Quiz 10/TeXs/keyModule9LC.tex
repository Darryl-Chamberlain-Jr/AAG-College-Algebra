\documentclass{extbook}[14pt]
\usepackage{multicol, enumerate, enumitem, hyperref, color, soul, setspace, parskip, fancyhdr, amssymb, amsthm, amsmath, bbm, latexsym, units, mathtools}
\everymath{\displaystyle}
\usepackage[headsep=0.5cm,headheight=0cm, left=1 in,right= 1 in,top= 1 in,bottom= 1 in]{geometry}
\usepackage{dashrule}  % Package to use the command below to create lines between items
\newcommand{\litem}[1]{\item #1

\rule{\textwidth}{0.4pt}}
\pagestyle{fancy}
\lhead{}
\chead{Answer Key for Progress Quiz 10 Version C}
\rhead{}
\lfoot{6232-9639}
\cfoot{}
\rfoot{Fall 2020}
\begin{document}
\textbf{This key should allow you to understand why you choose the option you did (beyond just getting a question right or wrong). \href{https://xronos.clas.ufl.edu/mac1105spring2020/courseDescriptionAndMisc/Exams/LearningFromResults}{More instructions on how to use this key can be found here}.}

\textbf{If you have a suggestion to make the keys better, \href{https://forms.gle/CZkbZmPbC9XALEE88}{please fill out the short survey here}.}

\textit{Note: This key is auto-generated and may contain issues and/or errors. The keys are reviewed after each exam to ensure grading is done accurately. If there are issues (like duplicate options), they are noted in the offline gradebook. The keys are a work-in-progress to give students as many resources to improve as possible.}

\rule{\textwidth}{0.4pt}

\begin{enumerate}\litem{
Choose the interval below that $f$ composed with $g$ at $x=-2$ is in.
\[ f(x) = -2x^{3} -3 x^{2} +3 x \text{ and } g(x) = -x^{3} -1 x^{2} +4 x + 1 \]

The solution is \( 18.0 \), which is option A.\begin{enumerate}[label=\Alph*.]
\item \( (f \circ g)(-2) \in [13, 21] \)

* This is the correct solution
\item \( (f \circ g)(-2) \in [9, 17] \)

 Distractor 2: Corresponds to being slightly off from the solution.
\item \( (f \circ g)(-2) \in [1, 3] \)

 Distractor 3: Corresponds to being slightly off from the solution.
\item \( (f \circ g)(-2) \in [-3, -1] \)

 Distractor 1: Corresponds to reversing the composition.
\item \( \text{It is not possible to compose the two functions.} \)


\end{enumerate}

\textbf{General Comment:} $f$ composed with $g$ at $x$ means $f(g(x))$. The order matters!
}
\litem{
Determine whether the function below is 1-1.
\[ f(x) = -9 x^2 - 75 x - 136 \]

The solution is \( \text{no} \), which is option E.\begin{enumerate}[label=\Alph*.]
\item \( \text{Yes, the function is 1-1.} \)

Corresponds to believing the function passes the Horizontal Line test.
\item \( \text{No, because the domain of the function is not $(-\infty, \infty)$.} \)

Corresponds to believing 1-1 means the domain is all Real numbers.
\item \( \text{No, because there is an $x$-value that goes to 2 different $y$-values.} \)

Corresponds to the Vertical Line test, which checks if an expression is a function.
\item \( \text{No, because the range of the function is not $(-\infty, \infty)$.} \)

Corresponds to believing 1-1 means the range is all Real numbers.
\item \( \text{No, because there is a $y$-value that goes to 2 different $x$-values.} \)

* This is the solution.
\end{enumerate}

\textbf{General Comment:} There are only two valid options: The function is 1-1 OR No because there is a $y$-value that goes to 2 different $x$-values.
}
\litem{
Determine whether the function below is 1-1.
\[ f(x) = 16 x^2 + 112 x + 196 \]

The solution is \( \text{no} \), which is option C.\begin{enumerate}[label=\Alph*.]
\item \( \text{Yes, the function is 1-1.} \)

Corresponds to believing the function passes the Horizontal Line test.
\item \( \text{No, because the domain of the function is not $(-\infty, \infty)$.} \)

Corresponds to believing 1-1 means the domain is all Real numbers.
\item \( \text{No, because there is a $y$-value that goes to 2 different $x$-values.} \)

* This is the solution.
\item \( \text{No, because there is an $x$-value that goes to 2 different $y$-values.} \)

Corresponds to the Vertical Line test, which checks if an expression is a function.
\item \( \text{No, because the range of the function is not $(-\infty, \infty)$.} \)

Corresponds to believing 1-1 means the range is all Real numbers.
\end{enumerate}

\textbf{General Comment:} There are only two valid options: The function is 1-1 OR No because there is a $y$-value that goes to 2 different $x$-values.
}
\litem{
Multiply the following functions, then choose the domain of the resulting function from the list below.
\[ f(x) = \frac{2}{6x+31} \text{ and } g(x) = \frac{5}{5x+24} \]

The solution is \( \text{ The domain is all Real numbers except } x = -5.166666666666667 \text{ and } x = -4.8 \), which is option D.\begin{enumerate}[label=\Alph*.]
\item \( \text{ The domain is all Real numbers greater than or equal to } x = a, \text{ where } a \in [5, 9] \)


\item \( \text{ The domain is all Real numbers except } x = a, \text{ where } a \in [-7.75, 1.25] \)


\item \( \text{ The domain is all Real numbers less than or equal to } x = a, \text{ where } a \in [-6.67, -2.67] \)


\item \( \text{ The domain is all Real numbers except } x = a \text{ and } x = b, \text{ where } a \in [-11.17, -3.17] \text{ and } b \in [-6.8, -2.8] \)


\item \( \text{ The domain is all Real numbers. } \)


\end{enumerate}

\textbf{General Comment:} The new domain is the intersection of the previous domains.
}
\litem{
Find the inverse of the function below. Then, evaluate the inverse at $x = 8$ and choose the interval that $f^{-1}(8)$ belongs to.
\[ f(x) = e^{x+3}+5 \]

The solution is \( f^{-1}(8) = -1.901 \), which is option A.\begin{enumerate}[label=\Alph*.]
\item \( f^{-1}(8) \in [-2.04, -1.82] \)

 This is the solution.
\item \( f^{-1}(8) \in [7.42, 7.68] \)

 This solution corresponds to distractor 2.
\item \( f^{-1}(8) \in [7.24, 7.41] \)

 This solution corresponds to distractor 4.
\item \( f^{-1}(8) \in [3.93, 4.35] \)

 This solution corresponds to distractor 1.
\item \( f^{-1}(8) \in [6.48, 6.61] \)

 This solution corresponds to distractor 3.
\end{enumerate}

\textbf{General Comment:} Natural log and exponential functions always have an inverse. Once you switch the $x$ and $y$, use the conversion $ e^y = x \leftrightarrow y=\ln(x)$.
}
\litem{
Choose the interval below that $f$ composed with $g$ at $x=-1$ is in.
\[ f(x) = -x^{3} -1 x^{2} +x + 1 \text{ and } g(x) = -3x^{3} -2 x^{2} +2 x \]

The solution is \( 0.0 \), which is option A.\begin{enumerate}[label=\Alph*.]
\item \( (f \circ g)(-1) \in [-4, 1] \)

* This is the correct solution
\item \( (f \circ g)(-1) \in [4, 12] \)

 Distractor 2: Corresponds to being slightly off from the solution.
\item \( (f \circ g)(-1) \in [-10, -2] \)

 Distractor 3: Corresponds to being slightly off from the solution.
\item \( (f \circ g)(-1) \in [-4, 1] \)

 Distractor 1: Corresponds to reversing the composition.
\item \( \text{It is not possible to compose the two functions.} \)


\end{enumerate}

\textbf{General Comment:} $f$ composed with $g$ at $x$ means $f(g(x))$. The order matters!
}
\litem{
Find the inverse of the function below. Then, evaluate the inverse at $x = 6$ and choose the interval that $f^{-1}(6)$ belongs to.
\[ f(x) = \ln{(x-2)}+4 \]

The solution is \( f^{-1}(6) = 9.389 \), which is option B.\begin{enumerate}[label=\Alph*.]
\item \( f^{-1}(6) \in [22027.47, 22032.47] \)

 This solution corresponds to distractor 1.
\item \( f^{-1}(6) \in [8.39, 13.39] \)

 This is the solution.
\item \( f^{-1}(6) \in [2982.96, 2986.96] \)

 This solution corresponds to distractor 2.
\item \( f^{-1}(6) \in [53.6, 59.6] \)

 This solution corresponds to distractor 4.
\item \( f^{-1}(6) \in [-1.61, 8.39] \)

 This solution corresponds to distractor 3.
\end{enumerate}

\textbf{General Comment:} Natural log and exponential functions always have an inverse. Once you switch the $x$ and $y$, use the conversion $ e^y = x \leftrightarrow y=\ln(x)$.
}
\litem{
Find the inverse of the function below (if it exists). Then, evaluate the inverse at $x = 14$ and choose the interval that $f^{-1}(14)$ belongs to.
\[ f(x) = 5 x^2 + 3 \]

The solution is \( \text{ The function is not invertible for all Real numbers. } \), which is option E.\begin{enumerate}[label=\Alph*.]
\item \( f^{-1}(14) \in [1.82, 1.88] \)

 Distractor 2: This corresponds to finding the (nonexistent) inverse and not subtracting by the vertical shift.
\item \( f^{-1}(14) \in [7.27, 7.53] \)

 Distractor 4: This corresponds to both distractors 2 and 3.
\item \( f^{-1}(14) \in [4.19, 4.63] \)

 Distractor 3: This corresponds to finding the (nonexistent) inverse and dividing by a negative.
\item \( f^{-1}(14) \in [1.11, 1.71] \)

 Distractor 1: This corresponds to trying to find the inverse even though the function is not 1-1. 
\item \( \text{ The function is not invertible for all Real numbers. } \)

* This is the correct option.
\end{enumerate}

\textbf{General Comment:} Be sure you check that the function is 1-1 before trying to find the inverse!
}
\litem{
Find the inverse of the function below (if it exists). Then, evaluate the inverse at $x = -15$ and choose the interval that $f^{-1}(-15)$ belongs to.
\[ f(x) = 4 x^2 - 2 \]

The solution is \( \text{ The function is not invertible for all Real numbers. } \), which is option E.\begin{enumerate}[label=\Alph*.]
\item \( f^{-1}(-15) \in [6.54, 7.25] \)

 Distractor 4: This corresponds to both distractors 2 and 3.
\item \( f^{-1}(-15) \in [4.53, 5.15] \)

 Distractor 3: This corresponds to finding the (nonexistent) inverse and dividing by a negative.
\item \( f^{-1}(-15) \in [1.27, 1.96] \)

 Distractor 1: This corresponds to trying to find the inverse even though the function is not 1-1. 
\item \( f^{-1}(-15) \in [1.81, 2.37] \)

 Distractor 2: This corresponds to finding the (nonexistent) inverse and not subtracting by the vertical shift.
\item \( \text{ The function is not invertible for all Real numbers. } \)

* This is the correct option.
\end{enumerate}

\textbf{General Comment:} Be sure you check that the function is 1-1 before trying to find the inverse!
}
\litem{
Add the following functions, then choose the domain of the resulting function from the list below.
\[ f(x) = 8x^{2} +6 x + 5 \text{ and } g(x) = 9x^{3} +6 x^{2} +4 x + 4 \]

The solution is \( (-\infty, \infty) \), which is option E.\begin{enumerate}[label=\Alph*.]
\item \( \text{ The domain is all Real numbers less than or equal to } x = a, \text{ where } a \in [-2.6, 0.4] \)


\item \( \text{ The domain is all Real numbers except } x = a, \text{ where } a \in [-9.2, -0.2] \)


\item \( \text{ The domain is all Real numbers greater than or equal to } x = a, \text{ where } a \in [-0.5, 6.5] \)


\item \( \text{ The domain is all Real numbers except } x = a \text{ and } x = b, \text{ where } a \in [-6.6, -0.6] \text{ and } b \in [-8.2, -5.2] \)


\item \( \text{ The domain is all Real numbers. } \)


\end{enumerate}

\textbf{General Comment:} The new domain is the intersection of the previous domains.
}
\end{enumerate}

\end{document}