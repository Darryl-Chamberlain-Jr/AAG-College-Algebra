\documentclass[14pt]{extbook}
\usepackage{multicol, enumerate, enumitem, hyperref, color, soul, setspace, parskip, fancyhdr} %General Packages
\usepackage{amssymb, amsthm, amsmath, bbm, latexsym, units, mathtools} %Math Packages
\everymath{\displaystyle} %All math in Display Style
% Packages with additional options
\usepackage[headsep=0.5cm,headheight=12pt, left=1 in,right= 1 in,top= 1 in,bottom= 1 in]{geometry}
\usepackage[usenames,dvipsnames]{xcolor}
\usepackage{dashrule}  % Package to use the command below to create lines between items
\newcommand{\litem}[1]{\item#1\hspace*{-1cm}\rule{\textwidth}{0.4pt}}
\pagestyle{fancy}
\lhead{Progress Quiz 10}
\chead{}
\rhead{Version B}
\lfoot{6232-9639}
\cfoot{}
\rfoot{Fall 2020}
\begin{document}

\begin{enumerate}
\litem{
Choose the interval below that $f$ composed with $g$ at $x=-1$ is in.\[ f(x) = -4x^{3} -3 x^{2} +3 x \text{ and } g(x) = 2x^{3} -2 x^{2} -2 x + 2 \]\begin{enumerate}[label=\Alph*.]
\item \( (f \circ g)(-1) \in [-1, 1] \)
\item \( (f \circ g)(-1) \in [-32, -25] \)
\item \( (f \circ g)(-1) \in [-15, -4] \)
\item \( (f \circ g)(-1) \in [-25, -12] \)
\item \( \text{It is not possible to compose the two functions.} \)

\end{enumerate} }
\litem{
Determine whether the function below is 1-1.\[ f(x) = (4 x - 26)^3 \]\begin{enumerate}[label=\Alph*.]
\item \( \text{No, because the domain of the function is not $(-\infty, \infty)$.} \)
\item \( \text{No, because there is a $y$-value that goes to 2 different $x$-values.} \)
\item \( \text{No, because the range of the function is not $(-\infty, \infty)$.} \)
\item \( \text{No, because there is an $x$-value that goes to 2 different $y$-values.} \)
\item \( \text{Yes, the function is 1-1.} \)

\end{enumerate} }
\litem{
Determine whether the function below is 1-1.\[ f(x) = -30 x^2 - 237 x - 405 \]\begin{enumerate}[label=\Alph*.]
\item \( \text{No, because there is an $x$-value that goes to 2 different $y$-values.} \)
\item \( \text{Yes, the function is 1-1.} \)
\item \( \text{No, because the domain of the function is not $(-\infty, \infty)$.} \)
\item \( \text{No, because there is a $y$-value that goes to 2 different $x$-values.} \)
\item \( \text{No, because the range of the function is not $(-\infty, \infty)$.} \)

\end{enumerate} }
\litem{
Subtract the following functions, then choose the domain of the resulting function from the list below.\[ f(x) = \frac{5}{4x-25} \text{ and } g(x) = 7x^{2} +7 x + 3 \]\begin{enumerate}[label=\Alph*.]
\item \( \text{ The domain is all Real numbers except } x = a, \text{ where } a \in [6.25, 11.25] \)
\item \( \text{ The domain is all Real numbers less than or equal to } x = a, \text{ where } a \in [-1, 8] \)
\item \( \text{ The domain is all Real numbers greater than or equal to } x = a, \text{ where } a \in [-10.5, -2.5] \)
\item \( \text{ The domain is all Real numbers except } x = a \text{ and } x = b, \text{ where } a \in [1.33, 8.33] \text{ and } b \in [-4.4, 4.6] \)
\item \( \text{ The domain is all Real numbers. } \)

\end{enumerate} }
\litem{
Find the inverse of the function below. Then, evaluate the inverse at $x = 10$ and choose the interval that $f^{-1}(10)$ belongs to.\[ f(x) = \ln{(x-5)}+4 \]\begin{enumerate}[label=\Alph*.]
\item \( f^{-1}(10) \in [149.41, 158.41] \)
\item \( f^{-1}(10) \in [407.43, 410.43] \)
\item \( f^{-1}(10) \in [3269020.37, 3269025.37] \)
\item \( f^{-1}(10) \in [392.43, 401.43] \)
\item \( f^{-1}(10) \in [1202608.28, 1202615.28] \)

\end{enumerate} }
\litem{
Choose the interval below that $f$ composed with $g$ at $x=-1$ is in.\[ f(x) = 2x^{3} +3 x^{2} +2 x \text{ and } g(x) = 4x^{3} -2 x^{2} -4 x \]\begin{enumerate}[label=\Alph*.]
\item \( (f \circ g)(-1) \in [-10, -4] \)
\item \( (f \circ g)(-1) \in [-3, 1] \)
\item \( (f \circ g)(-1) \in [-22, -14] \)
\item \( (f \circ g)(-1) \in [2, 12] \)
\item \( \text{It is not possible to compose the two functions.} \)

\end{enumerate} }
\litem{
Find the inverse of the function below. Then, evaluate the inverse at $x = 10$ and choose the interval that $f^{-1}(10)$ belongs to.\[ f(x) = \ln{(x+5)}-3 \]\begin{enumerate}[label=\Alph*.]
\item \( f^{-1}(10) \in [1091.63, 1092.63] \)
\item \( f^{-1}(10) \in [3269014.37, 3269017.37] \)
\item \( f^{-1}(10) \in [442413.39, 442420.39] \)
\item \( f^{-1}(10) \in [142.41, 152.41] \)
\item \( f^{-1}(10) \in [442404.39, 442411.39] \)

\end{enumerate} }
\litem{
Find the inverse of the function below (if it exists). Then, evaluate the inverse at $x = -12$ and choose the interval that $f^{-1}(-12)$ belongs to.\[ f(x) = 5 x^2 + 2 \]\begin{enumerate}[label=\Alph*.]
\item \( f^{-1}(-12) \in [4.65, 4.78] \)
\item \( f^{-1}(-12) \in [1, 1.65] \)
\item \( f^{-1}(-12) \in [3.55, 3.95] \)
\item \( f^{-1}(-12) \in [1.67, 1.76] \)
\item \( \text{ The function is not invertible for all Real numbers. } \)

\end{enumerate} }
\litem{
Find the inverse of the function below (if it exists). Then, evaluate the inverse at $x = -15$ and choose the interval the $f^{-1}(-15)$ belongs to.\[ f(x) = \sqrt[3]{4 x + 3} \]\begin{enumerate}[label=\Alph*.]
\item \( f^{-1}(-15) \in [844.42, 845.34] \)
\item \( f^{-1}(-15) \in [-845.53, -843.31] \)
\item \( f^{-1}(-15) \in [-843.52, -842.62] \)
\item \( f^{-1}(-15) \in [842.23, 844.38] \)
\item \( \text{ The function is not invertible for all Real numbers. } \)

\end{enumerate} }
\litem{
Subtract the following functions, then choose the domain of the resulting function from the list below.\[ f(x) = \sqrt{6x+24}  \text{ and } g(x) = x^{4} +9 x^{3} +8 x^{2} +x + 6 \]\begin{enumerate}[label=\Alph*.]
\item \( \text{ The domain is all Real numbers less than or equal to } x = a, \text{ where } a \in [0.75, 6.75] \)
\item \( \text{ The domain is all Real numbers greater than or equal to } x = a, \text{ where } a \in [-7, -3] \)
\item \( \text{ The domain is all Real numbers except } x = a, \text{ where } a \in [3.25, 5.25] \)
\item \( \text{ The domain is all Real numbers except } x = a \text{ and } x = b, \text{ where } a \in [-12.25, -5.25] \text{ and } b \in [-6.4, -4.4] \)
\item \( \text{ The domain is all Real numbers. } \)

\end{enumerate} }
\end{enumerate}

\end{document}