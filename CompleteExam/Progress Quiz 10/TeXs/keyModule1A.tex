\documentclass{extbook}[14pt]
\usepackage{multicol, enumerate, enumitem, hyperref, color, soul, setspace, parskip, fancyhdr, amssymb, amsthm, amsmath, latexsym, units, mathtools}
\everymath{\displaystyle}
\usepackage[headsep=0.5cm,headheight=0cm, left=1 in,right= 1 in,top= 1 in,bottom= 1 in]{geometry}
\usepackage{dashrule}  % Package to use the command below to create lines between items
\newcommand{\litem}[1]{\item #1

\rule{\textwidth}{0.4pt}}
\pagestyle{fancy}
\lhead{}
\chead{Answer Key for Progress Quiz 10 Version A}
\rhead{}
\lfoot{1995-1928}
\cfoot{}
\rfoot{test}
\begin{document}
\textbf{This key should allow you to understand why you choose the option you did (beyond just getting a question right or wrong). \href{https://xronos.clas.ufl.edu/mac1105spring2020/courseDescriptionAndMisc/Exams/LearningFromResults}{More instructions on how to use this key can be found here}.}

\textbf{If you have a suggestion to make the keys better, \href{https://forms.gle/CZkbZmPbC9XALEE88}{please fill out the short survey here}.}

\textit{Note: This key is auto-generated and may contain issues and/or errors. The keys are reviewed after each exam to ensure grading is done accurately. If there are issues (like duplicate options), they are noted in the offline gradebook. The keys are a work-in-progress to give students as many resources to improve as possible.}

\rule{\textwidth}{0.4pt}

\begin{enumerate}\litem{
Simplify the expression below into the form $a+bi$. Then, choose the intervals that $a$ and $b$ belong to.
\[ (2 - 4 i)(7 - 8 i) \]The solution is \( -18 - 44 i \), which is option E.\begin{enumerate}[label=\Alph*.]
\item \( a \in [45, 47] \text{ and } b \in [8, 20] \)

 $46 + 12 i$, which corresponds to adding a minus sign in the first term.
\item \( a \in [45, 47] \text{ and } b \in [-14, -5] \)

 $46 - 12 i$, which corresponds to adding a minus sign in the second term.
\item \( a \in [-21, -9] \text{ and } b \in [43, 50] \)

 $-18 + 44 i$, which corresponds to adding a minus sign in both terms.
\item \( a \in [12, 15] \text{ and } b \in [32, 37] \)

 $14 + 32 i$, which corresponds to just multiplying the real terms to get the real part of the solution and the coefficients in the complex terms to get the complex part.
\item \( a \in [-21, -9] \text{ and } b \in [-48, -39] \)

* $-18 - 44 i$, which is the correct option.
\end{enumerate}

\textbf{General Comment:} You can treat $i$ as a variable and distribute. Just remember that $i^2=-1$, so you can continue to reduce after you distribute.
}
\litem{
Choose the \textbf{smallest} set of Complex numbers that the number below belongs to.
\[ \sqrt{\frac{-616}{8}}+\sqrt{0}i \]The solution is \( \text{Pure Imaginary} \), which is option E.\begin{enumerate}[label=\Alph*.]
\item \( \text{Not a Complex Number} \)

This is not a number. The only non-Complex number we know is dividing by 0 as this is not a number!
\item \( \text{Rational} \)

These are numbers that can be written as fraction of Integers (e.g., -2/3 + 5)
\item \( \text{Irrational} \)

These cannot be written as a fraction of Integers. Remember: $\pi$ is not an Integer!
\item \( \text{Nonreal Complex} \)

This is a Complex number $(a+bi)$ that is not Real (has $i$ as part of the number).
\item \( \text{Pure Imaginary} \)

* This is the correct option!
\end{enumerate}

\textbf{General Comment:} Be sure to simplify $i^2 = -1$. This may remove the imaginary portion for your number. If you are having trouble, you may want to look at the \textit{Subgroups of the Real Numbers} section.
}
\litem{
Simplify the expression below and choose the interval the simplification is contained within.
\[ 11 - 2 \div 10 * 7 - (5 * 13) \]The solution is \( -55.400 \), which is option D.\begin{enumerate}[label=\Alph*.]
\item \( [-54.1, -53.3] \)

 -54.029, which corresponds to an Order of Operations error: not reading left-to-right for multiplication/division.
\item \( [74.4, 76.3] \)

 75.971, which corresponds to not distributing addition and subtraction correctly.
\item \( [59.4, 61.1] \)

 59.800, which corresponds to not distributing a negative correctly.
\item \( [-57.9, -54.9] \)

* -55.400, which is the correct option.
\item \( \text{None of the above} \)

 You may have gotten this by making an unanticipated error. If you got a value that is not any of the others, please let the coordinator know so they can help you figure out what happened.
\end{enumerate}

\textbf{General Comment:} While you may remember (or were taught) PEMDAS is done in order, it is actually done as P/E/MD/AS. When we are at MD or AS, we read left to right.
}
\litem{
Choose the \textbf{smallest} set of Complex numbers that the number below belongs to.
\[ -\sqrt{\frac{1638}{14}}+5i^2 \]The solution is \( \text{Irrational} \), which is option E.\begin{enumerate}[label=\Alph*.]
\item \( \text{Rational} \)

These are numbers that can be written as fraction of Integers (e.g., -2/3 + 5)
\item \( \text{Not a Complex Number} \)

This is not a number. The only non-Complex number we know is dividing by 0 as this is not a number!
\item \( \text{Nonreal Complex} \)

This is a Complex number $(a+bi)$ that is not Real (has $i$ as part of the number).
\item \( \text{Pure Imaginary} \)

This is a Complex number $(a+bi)$ that \textbf{only} has an imaginary part like $2i$.
\item \( \text{Irrational} \)

* This is the correct option!
\end{enumerate}

\textbf{General Comment:} Be sure to simplify $i^2 = -1$. This may remove the imaginary portion for your number. If you are having trouble, you may want to look at the \textit{Subgroups of the Real Numbers} section.
}
\litem{
Choose the \textbf{smallest} set of Real numbers that the number below belongs to.
\[ -\sqrt{\frac{529}{625}} \]The solution is \( \text{Rational} \), which is option E.\begin{enumerate}[label=\Alph*.]
\item \( \text{Not a Real number} \)

These are Nonreal Complex numbers \textbf{OR} things that are not numbers (e.g., dividing by 0).
\item \( \text{Irrational} \)

These cannot be written as a fraction of Integers.
\item \( \text{Whole} \)

These are the counting numbers with 0 (0, 1, 2, 3, ...)
\item \( \text{Integer} \)

These are the negative and positive counting numbers (..., -3, -2, -1, 0, 1, 2, 3, ...)
\item \( \text{Rational} \)

* This is the correct option!
\end{enumerate}

\textbf{General Comment:} First, you \textbf{NEED} to simplify the expression. This question simplifies to $-\frac{23}{25}$. 
 
 Be sure you look at the simplified fraction and not just the decimal expansion. Numbers such as 13, 17, and 19 provide \textbf{long but repeating/terminating decimal expansions!} 
 
 The only ways to *not* be a Real number are: dividing by 0 or taking the square root of a negative number. 
 
 Irrational numbers are more than just square root of 3: adding or subtracting values from square root of 3 is also irrational.
}
\litem{
Choose the \textbf{smallest} set of Real numbers that the number below belongs to.
\[ \sqrt{\frac{100}{81}} \]The solution is \( \text{Rational} \), which is option C.\begin{enumerate}[label=\Alph*.]
\item \( \text{Not a Real number} \)

These are Nonreal Complex numbers \textbf{OR} things that are not numbers (e.g., dividing by 0).
\item \( \text{Whole} \)

These are the counting numbers with 0 (0, 1, 2, 3, ...)
\item \( \text{Rational} \)

* This is the correct option!
\item \( \text{Irrational} \)

These cannot be written as a fraction of Integers.
\item \( \text{Integer} \)

These are the negative and positive counting numbers (..., -3, -2, -1, 0, 1, 2, 3, ...)
\end{enumerate}

\textbf{General Comment:} First, you \textbf{NEED} to simplify the expression. This question simplifies to $\frac{10}{9}$. 
 
 Be sure you look at the simplified fraction and not just the decimal expansion. Numbers such as 13, 17, and 19 provide \textbf{long but repeating/terminating decimal expansions!} 
 
 The only ways to *not* be a Real number are: dividing by 0 or taking the square root of a negative number. 
 
 Irrational numbers are more than just square root of 3: adding or subtracting values from square root of 3 is also irrational.
}
\litem{
Simplify the expression below into the form $a+bi$. Then, choose the intervals that $a$ and $b$ belong to.
\[ \frac{27 - 22 i}{5 + i} \]The solution is \( 4.35  - 5.27 i \), which is option D.\begin{enumerate}[label=\Alph*.]
\item \( a \in [3.95, 4.5] \text{ and } b \in [-138, -135.5] \)

 $4.35  - 137.00 i$, which corresponds to forgetting to multiply the conjugate by the numerator.
\item \( a \in [112.05, 113.3] \text{ and } b \in [-6.5, -4.5] \)

 $113.00  - 5.27 i$, which corresponds to forgetting to multiply the conjugate by the numerator and using a plus instead of a minus in the denominator.
\item \( a \in [6, 6.65] \text{ and } b \in [-4, -2] \)

 $6.04  - 3.19 i$, which corresponds to forgetting to multiply the conjugate by the numerator and not computing the conjugate correctly.
\item \( a \in [3.95, 4.5] \text{ and } b \in [-6.5, -4.5] \)

* $4.35  - 5.27 i$, which is the correct option.
\item \( a \in [5.15, 5.9] \text{ and } b \in [-23, -20.5] \)

 $5.40  - 22.00 i$, which corresponds to just dividing the first term by the first term and the second by the second.
\end{enumerate}

\textbf{General Comment:} Multiply the numerator and denominator by the *conjugate* of the denominator, then simplify. For example, if we have $2+3i$, the conjugate is $2-3i$.
}
\litem{
Simplify the expression below and choose the interval the simplification is contained within.
\[ 12 - 18 \div 6 * 20 - (5 * 16) \]The solution is \( -128.000 \), which is option D.\begin{enumerate}[label=\Alph*.]
\item \( [-68.15, -66.15] \)

 -68.150, which corresponds to an Order of Operations error: not reading left-to-right for multiplication/division.
\item \( [88.85, 96.85] \)

 91.850, which corresponds to not distributing addition and subtraction correctly.
\item \( [-849, -836] \)

 -848.000, which corresponds to not distributing a negative correctly.
\item \( [-131, -125] \)

* -128.000, which is the correct option.
\item \( \text{None of the above} \)

 You may have gotten this by making an unanticipated error. If you got a value that is not any of the others, please let the coordinator know so they can help you figure out what happened.
\end{enumerate}

\textbf{General Comment:} While you may remember (or were taught) PEMDAS is done in order, it is actually done as P/E/MD/AS. When we are at MD or AS, we read left to right.
}
\litem{
Simplify the expression below into the form $a+bi$. Then, choose the intervals that $a$ and $b$ belong to.
\[ \frac{45 + 44 i}{-2 + 3 i} \]The solution is \( 3.23  - 17.15 i \), which is option B.\begin{enumerate}[label=\Alph*.]
\item \( a \in [-23, -20.5] \text{ and } b \in [14.5, 16] \)

 $-22.50  + 14.67 i$, which corresponds to just dividing the first term by the first term and the second by the second.
\item \( a \in [3, 4] \text{ and } b \in [-18, -16.5] \)

* $3.23  - 17.15 i$, which is the correct option.
\item \( a \in [3, 4] \text{ and } b \in [-223.5, -222] \)

 $3.23  - 223.00 i$, which corresponds to forgetting to multiply the conjugate by the numerator.
\item \( a \in [41, 43.5] \text{ and } b \in [-18, -16.5] \)

 $42.00  - 17.15 i$, which corresponds to forgetting to multiply the conjugate by the numerator and using a plus instead of a minus in the denominator.
\item \( a \in [-18, -15.5] \text{ and } b \in [2.5, 4] \)

 $-17.08  + 3.62 i$, which corresponds to forgetting to multiply the conjugate by the numerator and not computing the conjugate correctly.
\end{enumerate}

\textbf{General Comment:} Multiply the numerator and denominator by the *conjugate* of the denominator, then simplify. For example, if we have $2+3i$, the conjugate is $2-3i$.
}
\litem{
Simplify the expression below into the form $a+bi$. Then, choose the intervals that $a$ and $b$ belong to.
\[ (-5 - 10 i)(9 + 7 i) \]The solution is \( 25 - 125 i \), which is option D.\begin{enumerate}[label=\Alph*.]
\item \( a \in [-46, -44] \text{ and } b \in [-75, -68] \)

 $-45 - 70 i$, which corresponds to just multiplying the real terms to get the real part of the solution and the coefficients in the complex terms to get the complex part.
\item \( a \in [19, 28] \text{ and } b \in [123, 130] \)

 $25 + 125 i$, which corresponds to adding a minus sign in both terms.
\item \( a \in [-117, -109] \text{ and } b \in [52, 60] \)

 $-115 + 55 i$, which corresponds to adding a minus sign in the first term.
\item \( a \in [19, 28] \text{ and } b \in [-126, -122] \)

* $25 - 125 i$, which is the correct option.
\item \( a \in [-117, -109] \text{ and } b \in [-56, -46] \)

 $-115 - 55 i$, which corresponds to adding a minus sign in the second term.
\end{enumerate}

\textbf{General Comment:} You can treat $i$ as a variable and distribute. Just remember that $i^2=-1$, so you can continue to reduce after you distribute.
}
\end{enumerate}

\end{document}