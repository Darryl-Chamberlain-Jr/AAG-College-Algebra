\documentclass[14pt]{extbook}
\usepackage{multicol, enumerate, enumitem, hyperref, color, soul, setspace, parskip, fancyhdr} %General Packages
\usepackage{amssymb, amsthm, amsmath, bbm, latexsym, units, mathtools} %Math Packages
\everymath{\displaystyle} %All math in Display Style
% Packages with additional options
\usepackage[headsep=0.5cm,headheight=12pt, left=1 in,right= 1 in,top= 1 in,bottom= 1 in]{geometry}
\usepackage[usenames,dvipsnames]{xcolor}
\usepackage{dashrule}  % Package to use the command below to create lines between items
\newcommand{\litem}[1]{\item#1\hspace*{-1cm}\rule{\textwidth}{0.4pt}}
\pagestyle{fancy}
\lhead{Progress Quiz 10}
\chead{}
\rhead{Version A}
\lfoot{6232-9639}
\cfoot{}
\rfoot{Fall 2020}
\begin{document}

\begin{enumerate}
\litem{
Choose the interval below that $f$ composed with $g$ at $x=1$ is in.\[ f(x) = x^{3} -1 x^{2} +3 x -3 \text{ and } g(x) = -x^{3} -3 x^{2} -x + 3 \]\begin{enumerate}[label=\Alph*.]
\item \( (f \circ g)(1) \in [-22, -17] \)
\item \( (f \circ g)(1) \in [-6, 0] \)
\item \( (f \circ g)(1) \in [-31, -27] \)
\item \( (f \circ g)(1) \in [3, 6] \)
\item \( \text{It is not possible to compose the two functions.} \)

\end{enumerate} }
\litem{
Determine whether the function below is 1-1.\[ f(x) = 16 x^2 + 128 x + 256 \]\begin{enumerate}[label=\Alph*.]
\item \( \text{Yes, the function is 1-1.} \)
\item \( \text{No, because there is an $x$-value that goes to 2 different $y$-values.} \)
\item \( \text{No, because there is a $y$-value that goes to 2 different $x$-values.} \)
\item \( \text{No, because the domain of the function is not $(-\infty, \infty)$.} \)
\item \( \text{No, because the range of the function is not $(-\infty, \infty)$.} \)

\end{enumerate} }
\litem{
Determine whether the function below is 1-1.\[ f(x) = (6 x - 30)^3 \]\begin{enumerate}[label=\Alph*.]
\item \( \text{Yes, the function is 1-1.} \)
\item \( \text{No, because there is an $x$-value that goes to 2 different $y$-values.} \)
\item \( \text{No, because the domain of the function is not $(-\infty, \infty)$.} \)
\item \( \text{No, because there is a $y$-value that goes to 2 different $x$-values.} \)
\item \( \text{No, because the range of the function is not $(-\infty, \infty)$.} \)

\end{enumerate} }
\litem{
Multiply the following functions, then choose the domain of the resulting function from the list below.\[ f(x) = 9x^{3} +6 x^{2} +9 x + 9 \text{ and } g(x) = \sqrt{-3x-7}  \]\begin{enumerate}[label=\Alph*.]
\item \( \text{ The domain is all Real numbers except } x = a, \text{ where } a \in [-4.25, -0.25] \)
\item \( \text{ The domain is all Real numbers greater than or equal to } x = a, \text{ where } a \in [3.25, 4.25] \)
\item \( \text{ The domain is all Real numbers less than or equal to } x = a, \text{ where } a \in [-3.33, -1.33] \)
\item \( \text{ The domain is all Real numbers except } x = a \text{ and } x = b, \text{ where } a \in [6.2, 7.2] \text{ and } b \in [-12.2, -4.2] \)
\item \( \text{ The domain is all Real numbers. } \)

\end{enumerate} }
\litem{
Find the inverse of the function below. Then, evaluate the inverse at $x = 10$ and choose the interval that $f^{-1}(10)$ belongs to.\[ f(x) = \ln{(x-2)}+5 \]\begin{enumerate}[label=\Alph*.]
\item \( f^{-1}(10) \in [142.41, 147.41] \)
\item \( f^{-1}(10) \in [3269015.37, 3269026.37] \)
\item \( f^{-1}(10) \in [2982.96, 2989.96] \)
\item \( f^{-1}(10) \in [149.41, 157.41] \)
\item \( f^{-1}(10) \in [162756.79, 162766.79] \)

\end{enumerate} }
\litem{
Choose the interval below that $f$ composed with $g$ at $x=-1$ is in.\[ f(x) = -2x^{3} -4 x^{2} -4 x \text{ and } g(x) = x^{3} +2 x^{2} -x -2 \]\begin{enumerate}[label=\Alph*.]
\item \( (f \circ g)(-1) \in [12, 15] \)
\item \( (f \circ g)(-1) \in [17, 22] \)
\item \( (f \circ g)(-1) \in [-2, 2] \)
\item \( (f \circ g)(-1) \in [-9, -5] \)
\item \( \text{It is not possible to compose the two functions.} \)

\end{enumerate} }
\litem{
Find the inverse of the function below. Then, evaluate the inverse at $x = 8$ and choose the interval that $f^{-1}(8)$ belongs to.\[ f(x) = \ln{(x-4)}-2 \]\begin{enumerate}[label=\Alph*.]
\item \( f^{-1}(8) \in [162752.79, 162759.79] \)
\item \( f^{-1}(8) \in [400.43, 410.43] \)
\item \( f^{-1}(8) \in [22026.47, 22035.47] \)
\item \( f^{-1}(8) \in [51.6, 54.6] \)
\item \( f^{-1}(8) \in [22021.47, 22023.47] \)

\end{enumerate} }
\litem{
Find the inverse of the function below (if it exists). Then, evaluate the inverse at $x = 15$ and choose the interval that $f^{-1}(15)$ belongs to.\[ f(x) = 3 x^2 - 2 \]\begin{enumerate}[label=\Alph*.]
\item \( f^{-1}(15) \in [1.72, 2.16] \)
\item \( f^{-1}(15) \in [7.7, 8.51] \)
\item \( f^{-1}(15) \in [2.3, 2.85] \)
\item \( f^{-1}(15) \in [4.85, 5.63] \)
\item \( \text{ The function is not invertible for all Real numbers. } \)

\end{enumerate} }
\litem{
Find the inverse of the function below (if it exists). Then, evaluate the inverse at $x = 12$ and choose the interval the $f^{-1}(12)$ belongs to.\[ f(x) = \sqrt[3]{4 x - 3} \]\begin{enumerate}[label=\Alph*.]
\item \( f^{-1}(12) \in [-433.14, -431.76] \)
\item \( f^{-1}(12) \in [430.43, 432.1] \)
\item \( f^{-1}(12) \in [-431.68, -430.43] \)
\item \( f^{-1}(12) \in [432.29, 434.48] \)
\item \( \text{ The function is not invertible for all Real numbers. } \)

\end{enumerate} }
\litem{
Add the following functions, then choose the domain of the resulting function from the list below.\[ f(x) = \frac{4}{4x+21} \text{ and } g(x) = \frac{1}{3x-19} \]\begin{enumerate}[label=\Alph*.]
\item \( \text{ The domain is all Real numbers greater than or equal to } x = a, \text{ where } a \in [-10.4, -4.4] \)
\item \( \text{ The domain is all Real numbers except } x = a, \text{ where } a \in [-9.25, -0.25] \)
\item \( \text{ The domain is all Real numbers less than or equal to } x = a, \text{ where } a \in [3.75, 7.75] \)
\item \( \text{ The domain is all Real numbers except } x = a \text{ and } x = b, \text{ where } a \in [-9.25, -3.25] \text{ and } b \in [4.33, 15.33] \)
\item \( \text{ The domain is all Real numbers. } \)

\end{enumerate} }
\end{enumerate}

\end{document}