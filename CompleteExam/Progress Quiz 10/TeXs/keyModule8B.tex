\documentclass{extbook}[14pt]
\usepackage{multicol, enumerate, enumitem, hyperref, color, soul, setspace, parskip, fancyhdr, amssymb, amsthm, amsmath, bbm, latexsym, units, mathtools}
\everymath{\displaystyle}
\usepackage[headsep=0.5cm,headheight=0cm, left=1 in,right= 1 in,top= 1 in,bottom= 1 in]{geometry}
\usepackage{dashrule}  % Package to use the command below to create lines between items
\newcommand{\litem}[1]{\item #1

\rule{\textwidth}{0.4pt}}
\pagestyle{fancy}
\lhead{}
\chead{Answer Key for Progress Quiz 10 Version B}
\rhead{}
\lfoot{6232-9639}
\cfoot{}
\rfoot{Fall 2020}
\begin{document}
\textbf{This key should allow you to understand why you choose the option you did (beyond just getting a question right or wrong). \href{https://xronos.clas.ufl.edu/mac1105spring2020/courseDescriptionAndMisc/Exams/LearningFromResults}{More instructions on how to use this key can be found here}.}

\textbf{If you have a suggestion to make the keys better, \href{https://forms.gle/CZkbZmPbC9XALEE88}{please fill out the short survey here}.}

\textit{Note: This key is auto-generated and may contain issues and/or errors. The keys are reviewed after each exam to ensure grading is done accurately. If there are issues (like duplicate options), they are noted in the offline gradebook. The keys are a work-in-progress to give students as many resources to improve as possible.}

\rule{\textwidth}{0.4pt}

\begin{enumerate}\litem{
Which of the following intervals describes the Range of the function below?
\[ f(x) = \log_2{(x-1)}-5 \]

The solution is \( (\infty, \infty) \), which is option E.\begin{enumerate}[label=\Alph*.]
\item \( [a, \infty), a \in [0.6, 2.4] \)

$[-5, \infty)$, which corresponds to using the flipped Domain AND including the endpoint.
\item \( [a, \infty), a \in [-4.8, 0.2] \)

$[-1, \infty)$, which corresponds to using the negative of the horizontal shift AND including the endpoint.
\item \( (-\infty, a), a \in [2.9, 5.6] \)

$(-\infty, 5)$, which corresponds to using the using the negative of vertical shift on $(0, \infty)$.
\item \( (-\infty, a), a \in [-7.2, -2.9] \)

$(-\infty, -5)$, which corresponds to using the vertical shift while the Range is $(-\infty, \infty)$.
\item \( (-\infty, \infty) \)

*This is the correct option.
\end{enumerate}

\textbf{General Comment:} \textbf{General Comments}: The domain of a basic logarithmic function is $(0, \infty)$ and the Range is $(-\infty, \infty)$. We can use shifts when finding the Domain, but the Range will always be all Real numbers.
}
\litem{
 Solve the equation for $x$ and choose the interval that contains $x$ (if it exists).
\[  19 = \ln{\sqrt[4]{\frac{12}{e^{4x}}}} \]

The solution is \( x = -18.379 \), which is option A.\begin{enumerate}[label=\Alph*.]
\item \( x \in [-21.38, -13.38] \)

* $x = -18.379$, which is the correct option.
\item \( x \in [-5.57, -0.57] \)

$x = -3.566$, which corresponds to thinking you need to take the natural log of on the left before reducing.
\item \( x \in [-8.88, -5.88] \)

$x = -8.879$, which corresponds to treating any root as a square root.
\item \( \text{There is no Real solution to the equation.} \)

This corresponds to believing you cannot solve the equation.
\item \( \text{None of the above.} \)

This corresponds to making an unexpected error.
\end{enumerate}

\textbf{General Comment:} \textbf{General Comments}: After using the properties of logarithmic functions to break up the right-hand side, use $\ln(e) = 1$ to reduce the question to a linear function to solve. You can put $\ln(12)$ into a calculator if you are having trouble.
}
\litem{
Which of the following intervals describes the Range of the function below?
\[ f(x) = -e^{x-7}-8 \]

The solution is \( (-\infty, -8) \), which is option B.\begin{enumerate}[label=\Alph*.]
\item \( [a, \infty), a \in [2, 13] \)

$[8, \infty)$, which corresponds to using the negative vertical shift AND flipping the Range interval AND including the endpoint.
\item \( (-\infty, a), a \in [-14, -1] \)

* $(-\infty, -8)$, which is the correct option.
\item \( (a, \infty), a \in [2, 13] \)

$(8, \infty)$, which corresponds to using the negative vertical shift AND flipping the Range interval.
\item \( (-\infty, a], a \in [-14, -1] \)

$(-\infty, -8]$, which corresponds to including the endpoint.
\item \( (-\infty, \infty) \)

This corresponds to confusing range of an exponential function with the domain of an exponential function.
\end{enumerate}

\textbf{General Comment:} \textbf{General Comments}: Domain of a basic exponential function is $(-\infty, \infty)$ while the Range is $(0, \infty)$. We can shift these intervals [and even flip when $a<0$!] to find the new Domain/Range.
}
\litem{
Solve the equation for $x$ and choose the interval that contains the solution (if it exists).
\[ 4^{-4x-4} = 25^{-3x+5} \]

The solution is \( x = 5.263 \), which is option D.\begin{enumerate}[label=\Alph*.]
\item \( x \in [-10, -6] \)

$x = -9.000$, which corresponds to solving the numerators as equal while ignoring the bases are different.
\item \( x \in [0.19, 5.19] \)

$x = 2.189$, which corresponds to distributing the $\ln(base)$ to the first term of the exponent only.
\item \( x \in [-23.64, -19.64] \)

$x = -21.640$, which corresponds to distributing the $\ln(base)$ to the second term of the exponent only.
\item \( x \in [4.26, 6.26] \)

* $x = 5.263$, which is the correct option.
\item \( \text{There is no Real solution to the equation.} \)

This corresponds to believing there is no solution since the bases are not powers of each other.
\end{enumerate}

\textbf{General Comment:} \textbf{General Comments:} This question was written so that the bases could not be written the same. You will need to take the log of both sides.
}
\litem{
Which of the following intervals describes the Range of the function below?
\[ f(x) = \log_2{(x+4)}+8 \]

The solution is \( (\infty, \infty) \), which is option E.\begin{enumerate}[label=\Alph*.]
\item \( [a, \infty), a \in [-4, 3] \)

$[8, \infty)$, which corresponds to using the flipped Domain AND including the endpoint.
\item \( [a, \infty), a \in [4, 6] \)

$[4, \infty)$, which corresponds to using the negative of the horizontal shift AND including the endpoint.
\item \( (-\infty, a), a \in [5, 9] \)

$(-\infty, 8)$, which corresponds to using the vertical shift while the Range is $(-\infty, \infty)$.
\item \( (-\infty, a), a \in [-10, -6] \)

$(-\infty, -8)$, which corresponds to using the using the negative of vertical shift on $(0, \infty)$.
\item \( (-\infty, \infty) \)

*This is the correct option.
\end{enumerate}

\textbf{General Comment:} \textbf{General Comments}: The domain of a basic logarithmic function is $(0, \infty)$ and the Range is $(-\infty, \infty)$. We can use shifts when finding the Domain, but the Range will always be all Real numbers.
}
\litem{
 Solve the equation for $x$ and choose the interval that contains $x$ (if it exists).
\[  13 = \sqrt[4]{\frac{21}{e^{9x}}} \]

The solution is \( x = -0.802, \text{ which does not fit in any of the interval options.} \), which is option E.\begin{enumerate}[label=\Alph*.]
\item \( x \in [-0.27, 0.3] \)

$x = -0.232$, which corresponds to treating any root as a square root.
\item \( x \in [-6.32, -5.96] \)

$x = -6.116$, which corresponds to thinking you don't need to take the natural log of both sides before reducing, as if the right side already has a natural log.
\item \( x \in [0.46, 0.88] \)

$x = 0.802$, which is the negative of the correct solution.
\item \( \text{There is no Real solution to the equation.} \)

This corresponds to believing you cannot solve the equation.
\item \( \text{None of the above.} \)

* $x = -0.802$ is the correct solution and does not fit in any of the other intervals.
\end{enumerate}

\textbf{General Comment:} \textbf{General Comments}: After using the properties of logarithmic functions to break up the right-hand side, use $\ln(e) = 1$ to reduce the question to a linear function to solve. You can put $\ln(21)$ into a calculator if you are having trouble.
}
\litem{
Solve the equation for $x$ and choose the interval that contains the solution (if it exists).
\[ \log_{5}{(2x+5)}+5 = 3 \]

The solution is \( x = -2.480 \), which is option A.\begin{enumerate}[label=\Alph*.]
\item \( x \in [-7.48, 0.52] \)

* $x = -2.480$, which is the correct option.
\item \( x \in [-15.5, -9.5] \)

$x = -13.500$, which corresponds to reversing the base and exponent when converting and reversing the value with $x$.
\item \( x \in [-27.5, -16.5] \)

$x = -18.500$, which corresponds to reversing the base and exponent when converting.
\item \( x \in [60, 61] \)

$x = 60.000$, which corresponds to ignoring the vertical shift when converting to exponential form.
\item \( \text{There is no Real solution to the equation.} \)

Corresponds to believing a negative coefficient within the log equation means there is no Real solution.
\end{enumerate}

\textbf{General Comment:} \textbf{General Comments:} First, get the equation in the form $\log_b{(cx+d)} = a$. Then, convert to $b^a = cx+d$ and solve.
}
\litem{
Solve the equation for $x$ and choose the interval that contains the solution (if it exists).
\[ 2^{3x-4} = 343^{4x+5} \]

The solution is \( x = -1.503 \), which is option B.\begin{enumerate}[label=\Alph*.]
\item \( x \in [-1.42, 0.58] \)

$x = -0.423$, which corresponds to distributing the $\ln(base)$ to the first term of the exponent only.
\item \( x \in [-3.5, -0.5] \)

* $x = -1.503$, which is the correct option.
\item \( x \in [-32.96, -29.96] \)

$x = -31.961$, which corresponds to distributing the $\ln(base)$ to the second term of the exponent only.
\item \( x \in [-12, -8] \)

$x = -9.000$, which corresponds to solving the numerators as equal while ignoring the bases are different.
\item \( \text{There is no Real solution to the equation.} \)

This corresponds to believing there is no solution since the bases are not powers of each other.
\end{enumerate}

\textbf{General Comment:} \textbf{General Comments:} This question was written so that the bases could not be written the same. You will need to take the log of both sides.
}
\litem{
Solve the equation for $x$ and choose the interval that contains the solution (if it exists).
\[ \log_{5}{(-2x+7)}+6 = 3 \]

The solution is \( x = 3.496 \), which is option A.\begin{enumerate}[label=\Alph*.]
\item \( x \in [-1.5, 7.5] \)

* $x = 3.496$, which is the correct option.
\item \( x \in [114, 119] \)

$x = 118.000$, which corresponds to reversing the base and exponent when converting and reversing the value with $x$.
\item \( x \in [-61, -55] \)

$x = -59.000$, which corresponds to ignoring the vertical shift when converting to exponential form.
\item \( x \in [119, 129] \)

$x = 125.000$, which corresponds to reversing the base and exponent when converting.
\item \( \text{There is no Real solution to the equation.} \)

Corresponds to believing a negative coefficient within the log equation means there is no Real solution.
\end{enumerate}

\textbf{General Comment:} \textbf{General Comments:} First, get the equation in the form $\log_b{(cx+d)} = a$. Then, convert to $b^a = cx+d$ and solve.
}
\litem{
Which of the following intervals describes the Range of the function below?
\[ f(x) = -e^{x+6}-7 \]

The solution is \( (-\infty, -7) \), which is option A.\begin{enumerate}[label=\Alph*.]
\item \( (-\infty, a), a \in [-12, -4] \)

* $(-\infty, -7)$, which is the correct option.
\item \( (-\infty, a], a \in [-12, -4] \)

$(-\infty, -7]$, which corresponds to including the endpoint.
\item \( [a, \infty), a \in [6, 11] \)

$[7, \infty)$, which corresponds to using the negative vertical shift AND flipping the Range interval AND including the endpoint.
\item \( (a, \infty), a \in [6, 11] \)

$(7, \infty)$, which corresponds to using the negative vertical shift AND flipping the Range interval.
\item \( (-\infty, \infty) \)

This corresponds to confusing range of an exponential function with the domain of an exponential function.
\end{enumerate}

\textbf{General Comment:} \textbf{General Comments}: Domain of a basic exponential function is $(-\infty, \infty)$ while the Range is $(0, \infty)$. We can shift these intervals [and even flip when $a<0$!] to find the new Domain/Range.
}
\end{enumerate}

\end{document}