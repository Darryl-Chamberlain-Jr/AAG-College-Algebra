\documentclass{extbook}[14pt]
\usepackage{multicol, enumerate, enumitem, hyperref, color, soul, setspace, parskip, fancyhdr, amssymb, amsthm, amsmath, bbm, latexsym, units, mathtools}
\everymath{\displaystyle}
\usepackage[headsep=0.5cm,headheight=0cm, left=1 in,right= 1 in,top= 1 in,bottom= 1 in]{geometry}
\usepackage{dashrule}  % Package to use the command below to create lines between items
\newcommand{\litem}[1]{\item #1

\rule{\textwidth}{0.4pt}}
\pagestyle{fancy}
\lhead{}
\chead{Answer Key for Progress Quiz 10 Version A}
\rhead{}
\lfoot{6232-9639}
\cfoot{}
\rfoot{Fall 2020}
\begin{document}
\textbf{This key should allow you to understand why you choose the option you did (beyond just getting a question right or wrong). \href{https://xronos.clas.ufl.edu/mac1105spring2020/courseDescriptionAndMisc/Exams/LearningFromResults}{More instructions on how to use this key can be found here}.}

\textbf{If you have a suggestion to make the keys better, \href{https://forms.gle/CZkbZmPbC9XALEE88}{please fill out the short survey here}.}

\textit{Note: This key is auto-generated and may contain issues and/or errors. The keys are reviewed after each exam to ensure grading is done accurately. If there are issues (like duplicate options), they are noted in the offline gradebook. The keys are a work-in-progress to give students as many resources to improve as possible.}

\rule{\textwidth}{0.4pt}

\begin{enumerate}\litem{
Using an interval or intervals, describe all the $x$-values within or including a distance of the given values.
\[ \text{ No more than } 2 \text{ units from the number } 5. \]

The solution is \( [3, 7] \), which is option A.\begin{enumerate}[label=\Alph*.]
\item \( [3, 7] \)

This describes the values no more than 2 from 5
\item \( (-\infty, 3) \cup (7, \infty) \)

This describes the values more than 2 from 5
\item \( (-\infty, 3] \cup [7, \infty) \)

This describes the values no less than 2 from 5
\item \( (3, 7) \)

This describes the values less than 2 from 5
\item \( \text{None of the above} \)

You likely thought the values in the interval were not correct.
\end{enumerate}

\textbf{General Comment:} When thinking about this language, it helps to draw a number line and try points.
}
\litem{
Solve the linear inequality below. Then, choose the constant and interval combination that describes the solution set.
\[ -6x -4 \geq 7x -10 \]

The solution is \( (-\infty, 0.462] \), which is option A.\begin{enumerate}[label=\Alph*.]
\item \( (-\infty, a], \text{ where } a \in [-0.05, 0.49] \)

* $(-\infty, 0.462]$, which is the correct option.
\item \( [a, \infty), \text{ where } a \in [-0.6, 0.3] \)

 $[-0.462, \infty)$, which corresponds to switching the direction of the interval AND negating the endpoint. You likely did this if you did not flip the inequality when dividing by a negative as well as not moving values over to a side properly.
\item \( [a, \infty), \text{ where } a \in [-0.1, 3.6] \)

 $[0.462, \infty)$, which corresponds to switching the direction of the interval. You likely did this if you did not flip the inequality when dividing by a negative!
\item \( (-\infty, a], \text{ where } a \in [-0.67, -0.25] \)

 $(-\infty, -0.462]$, which corresponds to negating the endpoint of the solution.
\item \( \text{None of the above}. \)

You may have chosen this if you thought the inequality did not match the ends of the intervals.
\end{enumerate}

\textbf{General Comment:} Remember that less/greater than or equal to includes the endpoint, while less/greater do not. Also, remember that you need to flip the inequality when you multiply or divide by a negative.
}
\litem{
Solve the linear inequality below. Then, choose the constant and interval combination that describes the solution set.
\[ -3 + 3 x > 4 x \text{ or } 4 + 9 x < 11 x \]

The solution is \( (-\infty, -3.0) \text{ or } (2.0, \infty) \), which is option A.\begin{enumerate}[label=\Alph*.]
\item \( (-\infty, a) \cup (b, \infty), \text{ where } a \in [-5.8, -2.6] \text{ and } b \in [1.32, 2.3] \)

 * Correct option.
\item \( (-\infty, a] \cup [b, \infty), \text{ where } a \in [-2.45, -1.74] \text{ and } b \in [2.05, 3.72] \)

Corresponds to including the endpoints AND negating.
\item \( (-\infty, a) \cup (b, \infty), \text{ where } a \in [-2.3, -0.4] \text{ and } b \in [2.3, 3.89] \)

Corresponds to inverting the inequality and negating the solution.
\item \( (-\infty, a] \cup [b, \infty), \text{ where } a \in [-3.75, -2.88] \text{ and } b \in [1.47, 2.04] \)

Corresponds to including the endpoints (when they should be excluded).
\item \( (-\infty, \infty) \)

Corresponds to the variable canceling, which does not happen in this instance.
\end{enumerate}

\textbf{General Comment:} When multiplying or dividing by a negative, flip the sign.
}
\litem{
Solve the linear inequality below. Then, choose the constant and interval combination that describes the solution set.
\[ -9 + 3 x < \frac{66 x - 9}{7} \leq 7 + 9 x \]

The solution is \( \text{None of the above.} \), which is option E.\begin{enumerate}[label=\Alph*.]
\item \( [a, b), \text{ where } a \in [1.2, 6.2] \text{ and } b \in [-23.33, -15.33] \)

$[1.20, -19.33)$, which corresponds to flipping the inequality and getting negatives of the actual endpoints.
\item \( (a, b], \text{ where } a \in [1.2, 8.2] \text{ and } b \in [-21.33, -13.33] \)

$(1.20, -19.33]$, which is the correct interval but negatives of the actual endpoints.
\item \( (-\infty, a) \cup [b, \infty), \text{ where } a \in [0, 3] \text{ and } b \in [-20.33, -14.33] \)

$(-\infty, 1.20) \cup [-19.33, \infty)$, which corresponds to displaying the and-inequality as an or-inequality and getting negatives of the actual endpoints.
\item \( (-\infty, a] \cup (b, \infty), \text{ where } a \in [1.2, 2.2] \text{ and } b \in [-21.33, -17.33] \)

$(-\infty, 1.20] \cup (-19.33, \infty)$, which corresponds to displaying the and-inequality as an or-inequality AND flipping the inequality AND getting negatives of the actual endpoints.
\item \( \text{None of the above.} \)

* This is correct as the answer should be $(-1.20, 19.33]$.
\end{enumerate}

\textbf{General Comment:} To solve, you will need to break up the compound inequality into two inequalities. Be sure to keep track of the inequality! It may be best to draw a number line and graph your solution.
}
\litem{
Solve the linear inequality below. Then, choose the constant and interval combination that describes the solution set.
\[ -4 + 7 x > 9 x \text{ or } 3 + 8 x < 9 x \]

The solution is \( (-\infty, -2.0) \text{ or } (3.0, \infty) \), which is option D.\begin{enumerate}[label=\Alph*.]
\item \( (-\infty, a] \cup [b, \infty), \text{ where } a \in [-3.8, -2.47] \text{ and } b \in [1.95, 2.67] \)

Corresponds to including the endpoints AND negating.
\item \( (-\infty, a] \cup [b, \infty), \text{ where } a \in [-2.46, -0.86] \text{ and } b \in [2.48, 3.2] \)

Corresponds to including the endpoints (when they should be excluded).
\item \( (-\infty, a) \cup (b, \infty), \text{ where } a \in [-5.1, -2.1] \text{ and } b \in [1.52, 2.35] \)

Corresponds to inverting the inequality and negating the solution.
\item \( (-\infty, a) \cup (b, \infty), \text{ where } a \in [-2.2, -1.9] \text{ and } b \in [2.47, 4.04] \)

 * Correct option.
\item \( (-\infty, \infty) \)

Corresponds to the variable canceling, which does not happen in this instance.
\end{enumerate}

\textbf{General Comment:} When multiplying or dividing by a negative, flip the sign.
}
\litem{
Using an interval or intervals, describe all the $x$-values within or including a distance of the given values.
\[ \text{ More than } 4 \text{ units from the number } -8. \]

The solution is \( (-\infty, -12) \cup (-4, \infty) \), which is option D.\begin{enumerate}[label=\Alph*.]
\item \( (-\infty, -12] \cup [-4, \infty) \)

This describes the values no less than 4 from -8
\item \( (-12, -4) \)

This describes the values less than 4 from -8
\item \( [-12, -4] \)

This describes the values no more than 4 from -8
\item \( (-\infty, -12) \cup (-4, \infty) \)

This describes the values more than 4 from -8
\item \( \text{None of the above} \)

You likely thought the values in the interval were not correct.
\end{enumerate}

\textbf{General Comment:} When thinking about this language, it helps to draw a number line and try points.
}
\litem{
Solve the linear inequality below. Then, choose the constant and interval combination that describes the solution set.
\[ -6 - 5 x < \frac{-25 x - 4}{7} \leq 8 - 4 x \]

The solution is \( \text{None of the above.} \), which is option E.\begin{enumerate}[label=\Alph*.]
\item \( [a, b), \text{ where } a \in [1.8, 6.8] \text{ and } b \in [-24, -19] \)

$[3.80, -20.00)$, which corresponds to flipping the inequality and getting negatives of the actual endpoints.
\item \( (a, b], \text{ where } a \in [3.8, 6.8] \text{ and } b \in [-22, -18] \)

$(3.80, -20.00]$, which is the correct interval but negatives of the actual endpoints.
\item \( (-\infty, a) \cup [b, \infty), \text{ where } a \in [0.8, 4.8] \text{ and } b \in [-23, -15] \)

$(-\infty, 3.80) \cup [-20.00, \infty)$, which corresponds to displaying the and-inequality as an or-inequality and getting negatives of the actual endpoints.
\item \( (-\infty, a] \cup (b, \infty), \text{ where } a \in [2.8, 4.8] \text{ and } b \in [-23, -17] \)

$(-\infty, 3.80] \cup (-20.00, \infty)$, which corresponds to displaying the and-inequality as an or-inequality AND flipping the inequality AND getting negatives of the actual endpoints.
\item \( \text{None of the above.} \)

* This is correct as the answer should be $(-3.80, 20.00]$.
\end{enumerate}

\textbf{General Comment:} To solve, you will need to break up the compound inequality into two inequalities. Be sure to keep track of the inequality! It may be best to draw a number line and graph your solution.
}
\litem{
Solve the linear inequality below. Then, choose the constant and interval combination that describes the solution set.
\[ \frac{7}{8} + \frac{5}{2} x < \frac{10}{3} x - \frac{6}{6} \]

The solution is \( (2.25, \infty) \), which is option B.\begin{enumerate}[label=\Alph*.]
\item \( (-\infty, a), \text{ where } a \in [1.25, 3.25] \)

 $(-\infty, 2.25)$, which corresponds to switching the direction of the interval. You likely did this if you did not flip the inequality when dividing by a negative!
\item \( (a, \infty), \text{ where } a \in [2.25, 6.25] \)

* $(2.25, \infty)$, which is the correct option.
\item \( (-\infty, a), \text{ where } a \in [-3.25, -1.25] \)

 $(-\infty, -2.25)$, which corresponds to switching the direction of the interval AND negating the endpoint. You likely did this if you did not flip the inequality when dividing by a negative as well as not moving values over to a side properly.
\item \( (a, \infty), \text{ where } a \in [-4.25, -0.25] \)

 $(-2.25, \infty)$, which corresponds to negating the endpoint of the solution.
\item \( \text{None of the above}. \)

You may have chosen this if you thought the inequality did not match the ends of the intervals.
\end{enumerate}

\textbf{General Comment:} Remember that less/greater than or equal to includes the endpoint, while less/greater do not. Also, remember that you need to flip the inequality when you multiply or divide by a negative.
}
\litem{
Solve the linear inequality below. Then, choose the constant and interval combination that describes the solution set.
\[ \frac{10}{2} - \frac{4}{8} x < \frac{9}{9} x + \frac{4}{5} \]

The solution is \( (2.8, \infty) \), which is option C.\begin{enumerate}[label=\Alph*.]
\item \( (-\infty, a), \text{ where } a \in [-2.8, 0.2] \)

 $(-\infty, -2.8)$, which corresponds to switching the direction of the interval AND negating the endpoint. You likely did this if you did not flip the inequality when dividing by a negative as well as not moving values over to a side properly.
\item \( (-\infty, a), \text{ where } a \in [2.8, 4.8] \)

 $(-\infty, 2.8)$, which corresponds to switching the direction of the interval. You likely did this if you did not flip the inequality when dividing by a negative!
\item \( (a, \infty), \text{ where } a \in [1.8, 4.8] \)

* $(2.8, \infty)$, which is the correct option.
\item \( (a, \infty), \text{ where } a \in [-4.8, 0.2] \)

 $(-2.8, \infty)$, which corresponds to negating the endpoint of the solution.
\item \( \text{None of the above}. \)

You may have chosen this if you thought the inequality did not match the ends of the intervals.
\end{enumerate}

\textbf{General Comment:} Remember that less/greater than or equal to includes the endpoint, while less/greater do not. Also, remember that you need to flip the inequality when you multiply or divide by a negative.
}
\litem{
Solve the linear inequality below. Then, choose the constant and interval combination that describes the solution set.
\[ -4x + 10 < 5x -8 \]

The solution is \( (2.0, \infty) \), which is option D.\begin{enumerate}[label=\Alph*.]
\item \( (-\infty, a), \text{ where } a \in [-2, -1] \)

 $(-\infty, -2.0)$, which corresponds to switching the direction of the interval AND negating the endpoint. You likely did this if you did not flip the inequality when dividing by a negative as well as not moving values over to a side properly.
\item \( (-\infty, a), \text{ where } a \in [0, 6] \)

 $(-\infty, 2.0)$, which corresponds to switching the direction of the interval. You likely did this if you did not flip the inequality when dividing by a negative!
\item \( (a, \infty), \text{ where } a \in [-8, -1] \)

 $(-2.0, \infty)$, which corresponds to negating the endpoint of the solution.
\item \( (a, \infty), \text{ where } a \in [0, 6] \)

* $(2.0, \infty)$, which is the correct option.
\item \( \text{None of the above}. \)

You may have chosen this if you thought the inequality did not match the ends of the intervals.
\end{enumerate}

\textbf{General Comment:} Remember that less/greater than or equal to includes the endpoint, while less/greater do not. Also, remember that you need to flip the inequality when you multiply or divide by a negative.
}
\end{enumerate}

\end{document}