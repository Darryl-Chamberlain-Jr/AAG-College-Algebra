\documentclass{extbook}[14pt]
\usepackage{multicol, enumerate, enumitem, hyperref, color, soul, setspace, parskip, fancyhdr, amssymb, amsthm, amsmath, bbm, latexsym, units, mathtools}
\everymath{\displaystyle}
\usepackage[headsep=0.5cm,headheight=0cm, left=1 in,right= 1 in,top= 1 in,bottom= 1 in]{geometry}
\usepackage{dashrule}  % Package to use the command below to create lines between items
\newcommand{\litem}[1]{\item #1

\rule{\textwidth}{0.4pt}}
\pagestyle{fancy}
\lhead{}
\chead{Answer Key for Progress Quiz 10 Version C}
\rhead{}
\lfoot{6232-9639}
\cfoot{}
\rfoot{Fall 2020}
\begin{document}
\textbf{This key should allow you to understand why you choose the option you did (beyond just getting a question right or wrong). \href{https://xronos.clas.ufl.edu/mac1105spring2020/courseDescriptionAndMisc/Exams/LearningFromResults}{More instructions on how to use this key can be found here}.}

\textbf{If you have a suggestion to make the keys better, \href{https://forms.gle/CZkbZmPbC9XALEE88}{please fill out the short survey here}.}

\textit{Note: This key is auto-generated and may contain issues and/or errors. The keys are reviewed after each exam to ensure grading is done accurately. If there are issues (like duplicate options), they are noted in the offline gradebook. The keys are a work-in-progress to give students as many resources to improve as possible.}

\rule{\textwidth}{0.4pt}

\begin{enumerate}\litem{
For the scenario below, find the variation constant $k$ of the model (if possible).

\begin{center}
    \textit{ In an alternative galaxy, the quartic of the time, $T$ (Earth years), required for a planet to orbit Sun $\chi$ increases as the quartic of the distance, $d$ (AUs), that the planet is from Sun $\chi$ increases. For example, when Ea's average distance from Sun $\chi$ is 5, it takes 93 Earth days to complete an orbit. }
\end{center}


The solution is \( k = 119688.322 \), which is option D.\begin{enumerate}[label=\Alph*.]
\item \( k = 4.028 \)

This copies the constant used in the homework.
\item \( k = 46753250625.000 \)

This corresponds to the model $T^{4} = \frac{k}{d^{4}}$.
\item \( k = 2.077 \)

This corresponds to the model $T^{1/4} = k d^{1/4}$.
\item \( k = 119688.322 \)

* This is the correct option corresponding to the model $T^{4} = k d^{4}$.
\item \( \text{Unable to compute the constant based on the information given.} \)

This corresponds to believing you cannot determine the type of model from the information given.
\end{enumerate}

\textbf{General Comment:} Since $T$ increases proportionally as $d$ increases, we know this is a direct variation model.
}
\litem{
For the scenario below, find the variation constant $k$ of the model (if possible).

\begin{center}
    \textit{ In an alternative galaxy, the quartic of the time, $T$ (Earth years), required for a planet to orbit Sun $\chi$ increases as the quartic of the distance, $d$ (AUs), that the planet is from Sun $\chi$ increases. For example, when Ea's average distance from Sun $\chi$ is 6, it takes 78 Earth days to complete an orbit. }
\end{center}


The solution is \( k = 28561.000 \), which is option C.\begin{enumerate}[label=\Alph*.]
\item \( k = 4.028 \)

This copies the constant used in the homework.
\item \( k = 1.899 \)

This corresponds to the model $T^{1/4} = k d^{1/4}$.
\item \( k = 28561.000 \)

* This is the correct option corresponding to the model $T^{4} = k d^{4}$.
\item \( k = 47971512576.000 \)

This corresponds to the model $T^{4} = \frac{k}{d^{4}}$.
\item \( \text{Unable to compute the constant based on the information given.} \)

This corresponds to believing you cannot determine the type of model from the information given.
\end{enumerate}

\textbf{General Comment:} Since $T$ increases proportionally as $d$ increases, we know this is a direct variation model.
}
\litem{
Choose the model type that would best describe the scenario below.

\begin{center}
    \textit{ Social distancing is a common tactic to counter potential epidemics. This is due to the exponential increase in number of people infected as the density of people living in an area increases. }
\end{center}


The solution is \( \text{None of the above} \), which is option D.\begin{enumerate}[label=\Alph*.]
\item \( \text{Direct variation} \)


\item \( \text{Indirect variation} \)


\item \( \text{Joint variation} \)


\item \( \text{None of the above} \)


\end{enumerate}

\textbf{General Comment:} This is an exponential variation, which grows significantly faster than any power function.
}
\litem{
A town has an initial population of 50000. The town's population for the next 10 years is provided below. Which type of function would be most appropriate to model the town's population?



\begin{tabular}{c|c|c|c|c|c|c|c|c|c}
\textbf{Year} & 1 & 2 & 3 & 4 & 5 & 6 & 7 & 8 & 9 \tabularnewline
\hline
\textbf{Pop.} & 50000 & 50034 & 50054 & 50069 & 50080 & 50089 & 50097 & 50103 & 50109
\end{tabular} 

The solution is \( \text{Logarithmic} \), which is option A.\begin{enumerate}[label=\Alph*.]
\item \( \text{Logarithmic} \)

This suggests the slowest of growths that we know.
\item \( \text{Linear} \)

This suggests a constant growth. You would be able to add or subtract the same amount year-to-year if this is the correct answer.
\item \( \text{Exponential} \)

This suggests the fastest of growths that we know.
\item \( \text{Non-Linear Power} \)

This suggests a growth faster than constant but slower than exponential.
\item \( \text{None of the above} \)

Please contact the coordinator to discuss why you believe none of the options model the population.
\end{enumerate}

\textbf{General Comment:} We are trying to compare the growth rate of the population. Growth rates can be characterized from slowest to fastest as: logarithmic, indirect, linear, direct, exponential. The best way to approach this is to first compare it to linear (is it linear, faster than linear, or slower than linear)? If faster, is it as fast as exponential? If slower, is it as slow as logarithmic?
}
\litem{
A town has an initial population of 100000. The town's population for the next 10 years is provided below. Which type of function would be most appropriate to model the town's population?



\begin{tabular}{c|c|c|c|c|c|c|c|c|c}
\textbf{Year} & 1 & 2 & 3 & 4 & 5 & 6 & 7 & 8 & 9 \tabularnewline
\hline
\textbf{Pop.} & 100000 & 99979 & 99967 & 99958 & 99951 & 99946 & 99941 & 99937 & 99934
\end{tabular} 

The solution is \( \text{Logarithmic} \), which is option B.\begin{enumerate}[label=\Alph*.]
\item \( \text{Non-Linear Power} \)

This suggests a growth faster than constant but slower than exponential.
\item \( \text{Logarithmic} \)

This suggests the slowest of growths that we know.
\item \( \text{Linear} \)

This suggests a constant growth. You would be able to add or subtract the same amount year-to-year if this is the correct answer.
\item \( \text{Exponential} \)

This suggests the fastest of growths that we know.
\item \( \text{None of the above} \)

Please contact the coordinator to discuss why you believe none of the options model the population.
\end{enumerate}

\textbf{General Comment:} We are trying to compare the growth rate of the population. Growth rates can be characterized from slowest to fastest as: logarithmic, indirect, linear, direct, exponential. The best way to approach this is to first compare it to linear (is it linear, faster than linear, or slower than linear)? If faster, is it as fast as exponential? If slower, is it as slow as logarithmic?
}
\litem{
Choose the model type that would best describe the scenario below.

\begin{center}
    \textit{ Social distancing is a common tactic to counter potential epidemics. This is due to the exponential increase in number of people infected as the density of people living in an area increases. }
\end{center}


The solution is \( \text{None of the above} \), which is option D.\begin{enumerate}[label=\Alph*.]
\item \( \text{Direct variation} \)


\item \( \text{Joint variation} \)


\item \( \text{Indirect variation} \)


\item \( \text{None of the above} \)


\end{enumerate}

\textbf{General Comment:} This is an exponential variation, which grows significantly faster than any power function.
}
\litem{
For the scenario below, model the rate of vibration (cm/s) of the string in terms of the length of the string. Then determine the variation constant $k$ of the model (if possible). The constant should be in terms of cm and s.

\begin{center}
    \textit{ The rate of vibration of a string under constant tension varies based on the type of string and the length of the string. The rate of vibration of string $\omega$ increases as the quartic length of the string decreases. For example, when string $\omega$ is 3 mm long, the rate of vibration is 36 cm/s. }
\end{center}


The solution is \( k = 0.29 \), which is option B.\begin{enumerate}[label=\Alph*.]
\item \( k = 2916.00 \)

This option uses the correct model, $R = \frac{k}{l^{4}}$, but does not convert from mm to cm so that the units match.
\item \( k = 0.29 \)

* This is the correct option, which corresponds to the model $R = \frac{k}{l^{4}}$ AND converts from mm to cm.
\item \( k = 4444.44 \)

This option uses the model $R = kl^{4}$ as if this is a direct variation.
\item \( k = 0.44 \)

This option uses the model $R = kl^{4}$ as if this is a direct variation AND does not convert from mm to cm so that the units match.
\item \( \text{None of the above.} \)

Talk with the coordinator if you chose this option.
\end{enumerate}

\textbf{General Comment:} The most common mistake on this question is to not convert mm to cm! When modeling, you need to make sure all of the units for your variables are compatible.
}
\litem{
For the scenario below, use the model for the volume of a cylinder as $V = \pi r^2 h$ to find the coefficient for the model of the new volume $V_{	ext{new}} = k r^2 h$.

\begin{center}
    \textit{ Pepsi wants to increase the volume of soda in their cans. They've decided to decrease the radius by 20 percent and decrease the height by 17 percent. They want to model the new volume based on the radius and height of the original cans. }
\end{center}


The solution is \( k = 1.66881 \), which is option C.\begin{enumerate}[label=\Alph*.]
\item \( k = 0.53120 \)

This corresponds to the model: $V = (0.80 r)^2 (0.83 h)$.
\item \( k = 0.00680 \)

This corresponds to the model: $V = (0.20 r)^2 (0.17 h)$.
\item \( k = 1.66881 \)

* This is the correct option and corresponds to the model: $V = \pi (0.80 r)^2 (0.83 h)$.
\item \( k = 0.02136 \)

This corresponds to the model: $V = \pi (0.20 r)^2 (0.17 h)$.
\item \( \text{None of the above.} \)

If you chose this, please talk with the coordinator to discuss why you believe none of the options are correct.
\end{enumerate}

\textbf{General Comment:} When calculating the new dimensions, you need to add/subtract from 100\%. For example, a 10\% increase in height would result in 110\% of the original height: $1.1h_{old} = h_{new}$.
}
\litem{
For the scenario below, use the model for the volume of a cylinder as $V = \pi r^2 h$ to find the coefficient for the model of the new volume $V_{	ext{new}} = k r^2 h$.

\begin{center}
    \textit{ Pepsi wants to increase the volume of soda in their cans. They've decided to decrease the radius by 20 percent and decrease the height by 16 percent. They want to model the new volume based on the radius and height of the original cans. }
\end{center}


The solution is \( k = 1.68892 \), which is option D.\begin{enumerate}[label=\Alph*.]
\item \( k = 0.00640 \)

This corresponds to the model: $V = (0.20 r)^2 (0.16 h)$.
\item \( k = 0.53760 \)

This corresponds to the model: $V = (0.80 r)^2 (0.84 h)$.
\item \( k = 0.02011 \)

This corresponds to the model: $V = \pi (0.20 r)^2 (0.16 h)$.
\item \( k = 1.68892 \)

* This is the correct option and corresponds to the model: $V = \pi (0.80 r)^2 (0.84 h)$.
\item \( \text{None of the above.} \)

If you chose this, please talk with the coordinator to discuss why you believe none of the options are correct.
\end{enumerate}

\textbf{General Comment:} When calculating the new dimensions, you need to add/subtract from 100\%. For example, a 10\% increase in height would result in 110\% of the original height: $1.1h_{old} = h_{new}$.
}
\litem{
For the scenario below, model the rate of vibration (cm/s) of the string in terms of the length of the string. Then determine the variation constant $k$ of the model (if possible). The constant should be in terms of cm and s.

\begin{center}
    \textit{ The rate of vibration of a string under constant tension varies based on the type of string and the length of the string. The rate of vibration of string $\omega$ increases as the quartic length of the string decreases. For example, when string $\omega$ is 5 mm long, the rate of vibration is 35 cm/s. }
\end{center}


The solution is \( k = 2.19 \), which is option D.\begin{enumerate}[label=\Alph*.]
\item \( k = 21875.00 \)

This option uses the correct model, $R = \frac{k}{l^{4}}$, but does not convert from mm to cm so that the units match.
\item \( k = 560.00 \)

This option uses the model $R = kl^{4}$ as if this is a direct variation.
\item \( k = 0.06 \)

This option uses the model $R = kl^{4}$ as if this is a direct variation AND does not convert from mm to cm so that the units match.
\item \( k = 2.19 \)

* This is the correct option, which corresponds to the model $R = \frac{k}{l^{4}}$ AND converts from mm to cm.
\item \( \text{None of the above.} \)

Talk with the coordinator if you chose this option.
\end{enumerate}

\textbf{General Comment:} The most common mistake on this question is to not convert mm to cm! When modeling, you need to make sure all of the units for your variables are compatible.
}
\end{enumerate}

\end{document}