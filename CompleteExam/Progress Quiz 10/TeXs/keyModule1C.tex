\documentclass{extbook}[14pt]
\usepackage{multicol, enumerate, enumitem, hyperref, color, soul, setspace, parskip, fancyhdr, amssymb, amsthm, amsmath, bbm, latexsym, units, mathtools}
\everymath{\displaystyle}
\usepackage[headsep=0.5cm,headheight=0cm, left=1 in,right= 1 in,top= 1 in,bottom= 1 in]{geometry}
\usepackage{dashrule}  % Package to use the command below to create lines between items
\newcommand{\litem}[1]{\item #1

\rule{\textwidth}{0.4pt}}
\pagestyle{fancy}
\lhead{}
\chead{Answer Key for Progress Quiz 10 Version C}
\rhead{}
\lfoot{6232-9639}
\cfoot{}
\rfoot{Fall 2020}
\begin{document}
\textbf{This key should allow you to understand why you choose the option you did (beyond just getting a question right or wrong). \href{https://xronos.clas.ufl.edu/mac1105spring2020/courseDescriptionAndMisc/Exams/LearningFromResults}{More instructions on how to use this key can be found here}.}

\textbf{If you have a suggestion to make the keys better, \href{https://forms.gle/CZkbZmPbC9XALEE88}{please fill out the short survey here}.}

\textit{Note: This key is auto-generated and may contain issues and/or errors. The keys are reviewed after each exam to ensure grading is done accurately. If there are issues (like duplicate options), they are noted in the offline gradebook. The keys are a work-in-progress to give students as many resources to improve as possible.}

\rule{\textwidth}{0.4pt}

\begin{enumerate}\litem{
Choose the \textbf{smallest} set of Complex numbers that the number below belongs to.
\[ \sqrt{\frac{0}{49}}+\sqrt{4}i \]

The solution is \( \text{Pure Imaginary} \), which is option E.\begin{enumerate}[label=\Alph*.]
\item \( \text{Nonreal Complex} \)

This is a Complex number $(a+bi)$ that is not Real (has $i$ as part of the number).
\item \( \text{Irrational} \)

These cannot be written as a fraction of Integers. Remember: $\pi$ is not an Integer!
\item \( \text{Not a Complex Number} \)

This is not a number. The only non-Complex number we know is dividing by 0 as this is not a number!
\item \( \text{Rational} \)

These are numbers that can be written as fraction of Integers (e.g., -2/3 + 5)
\item \( \text{Pure Imaginary} \)

* This is the correct option!
\end{enumerate}

\textbf{General Comment:} Be sure to simplify $i^2 = -1$. This may remove the imaginary portion for your number. If you are having trouble, you may want to look at the \textit{Subgroups of the Real Numbers} section.
}
\litem{
Simplify the expression below and choose the interval the simplification is contained within.
\[ 2 - 10^2 + 11 \div 8 * 6 \div 5 \]

The solution is \( -96.350 \), which is option C.\begin{enumerate}[label=\Alph*.]
\item \( [-100.3, -96.6] \)

 -97.954, which corresponds to an Order of Operations error: not reading left-to-right for multiplication/division.
\item \( [103, 109.3] \)

 103.650, which corresponds to an Order of Operations error: multiplying by negative before squaring. For example: $(-3)^2 \neq -3^2$
\item \( [-97.5, -94.3] \)

* -96.350, this is the correct option
\item \( [100.5, 102.9] \)

 102.046, which corresponds to two Order of Operations errors.
\item \( \text{None of the above} \)

 You may have gotten this by making an unanticipated error. If you got a value that is not any of the others, please let the coordinator know so they can help you figure out what happened.
\end{enumerate}

\textbf{General Comment:} While you may remember (or were taught) PEMDAS is done in order, it is actually done as P/E/MD/AS. When we are at MD or AS, we read left to right.
}
\litem{
Choose the \textbf{smallest} set of Real numbers that the number below belongs to.
\[ \sqrt{\frac{23104}{361}} \]

The solution is \( \text{Whole} \), which is option D.\begin{enumerate}[label=\Alph*.]
\item \( \text{Rational} \)

These are numbers that can be written as fraction of Integers (e.g., -2/3)
\item \( \text{Integer} \)

These are the negative and positive counting numbers (..., -3, -2, -1, 0, 1, 2, 3, ...)
\item \( \text{Not a Real number} \)

These are Nonreal Complex numbers \textbf{OR} things that are not numbers (e.g., dividing by 0).
\item \( \text{Whole} \)

* This is the correct option!
\item \( \text{Irrational} \)

These cannot be written as a fraction of Integers.
\end{enumerate}

\textbf{General Comment:} First, you \textbf{NEED} to simplify the expression. This question simplifies to $152$. 
 
 Be sure you look at the simplified fraction and not just the decimal expansion. Numbers such as 13, 17, and 19 provide \textbf{long but repeating/terminating decimal expansions!} 
 
 The only ways to *not* be a Real number are: dividing by 0 or taking the square root of a negative number. 
 
 Irrational numbers are more than just square root of 3: adding or subtracting values from square root of 3 is also irrational.
}
\litem{
Simplify the expression below into the form $a+bi$. Then, choose the intervals that $a$ and $b$ belong to.
\[ (5 + 9 i)(10 + 7 i) \]

The solution is \( -13 + 125 i \), which is option D.\begin{enumerate}[label=\Alph*.]
\item \( a \in [-13, -9] \text{ and } b \in [-133, -123] \)

 $-13 - 125 i$, which corresponds to adding a minus sign in both terms.
\item \( a \in [112, 117] \text{ and } b \in [-55, -52] \)

 $113 - 55 i$, which corresponds to adding a minus sign in the first term.
\item \( a \in [112, 117] \text{ and } b \in [55, 61] \)

 $113 + 55 i$, which corresponds to adding a minus sign in the second term.
\item \( a \in [-13, -9] \text{ and } b \in [123, 128] \)

* $-13 + 125 i$, which is the correct option.
\item \( a \in [49, 52] \text{ and } b \in [63, 66] \)

 $50 + 63 i$, which corresponds to just multiplying the real terms to get the real part of the solution and the coefficients in the complex terms to get the complex part.
\end{enumerate}

\textbf{General Comment:} You can treat $i$ as a variable and distribute. Just remember that $i^2=-1$, so you can continue to reduce after you distribute.
}
\litem{
Choose the \textbf{smallest} set of Complex numbers that the number below belongs to.
\[ \sqrt{\frac{1584}{0}}+\sqrt{112} i \]

The solution is \( \text{Not a Complex Number} \), which is option B.\begin{enumerate}[label=\Alph*.]
\item \( \text{Irrational} \)

These cannot be written as a fraction of Integers. Remember: $\pi$ is not an Integer!
\item \( \text{Not a Complex Number} \)

* This is the correct option!
\item \( \text{Rational} \)

These are numbers that can be written as fraction of Integers (e.g., -2/3 + 5)
\item \( \text{Pure Imaginary} \)

This is a Complex number $(a+bi)$ that \textbf{only} has an imaginary part like $2i$.
\item \( \text{Nonreal Complex} \)

This is a Complex number $(a+bi)$ that is not Real (has $i$ as part of the number).
\end{enumerate}

\textbf{General Comment:} Be sure to simplify $i^2 = -1$. This may remove the imaginary portion for your number. If you are having trouble, you may want to look at the \textit{Subgroups of the Real Numbers} section.
}
\litem{
Simplify the expression below into the form $a+bi$. Then, choose the intervals that $a$ and $b$ belong to.
\[ (4 - 10 i)(8 - 2 i) \]

The solution is \( 12 - 88 i \), which is option C.\begin{enumerate}[label=\Alph*.]
\item \( a \in [32, 34] \text{ and } b \in [18, 26] \)

 $32 + 20 i$, which corresponds to just multiplying the real terms to get the real part of the solution and the coefficients in the complex terms to get the complex part.
\item \( a \in [51, 58] \text{ and } b \in [68, 79] \)

 $52 + 72 i$, which corresponds to adding a minus sign in the first term.
\item \( a \in [11, 15] \text{ and } b \in [-93, -85] \)

* $12 - 88 i$, which is the correct option.
\item \( a \in [51, 58] \text{ and } b \in [-74, -69] \)

 $52 - 72 i$, which corresponds to adding a minus sign in the second term.
\item \( a \in [11, 15] \text{ and } b \in [84, 90] \)

 $12 + 88 i$, which corresponds to adding a minus sign in both terms.
\end{enumerate}

\textbf{General Comment:} You can treat $i$ as a variable and distribute. Just remember that $i^2=-1$, so you can continue to reduce after you distribute.
}
\litem{
Choose the \textbf{smallest} set of Real numbers that the number below belongs to.
\[ \sqrt{\frac{2730}{14}} \]

The solution is \( \text{Irrational} \), which is option D.\begin{enumerate}[label=\Alph*.]
\item \( \text{Integer} \)

These are the negative and positive counting numbers (..., -3, -2, -1, 0, 1, 2, 3, ...)
\item \( \text{Rational} \)

These are numbers that can be written as fraction of Integers (e.g., -2/3)
\item \( \text{Not a Real number} \)

These are Nonreal Complex numbers \textbf{OR} things that are not numbers (e.g., dividing by 0).
\item \( \text{Irrational} \)

* This is the correct option!
\item \( \text{Whole} \)

These are the counting numbers with 0 (0, 1, 2, 3, ...)
\end{enumerate}

\textbf{General Comment:} First, you \textbf{NEED} to simplify the expression. This question simplifies to $\sqrt{195}$. 
 
 Be sure you look at the simplified fraction and not just the decimal expansion. Numbers such as 13, 17, and 19 provide \textbf{long but repeating/terminating decimal expansions!} 
 
 The only ways to *not* be a Real number are: dividing by 0 or taking the square root of a negative number. 
 
 Irrational numbers are more than just square root of 3: adding or subtracting values from square root of 3 is also irrational.
}
\litem{
Simplify the expression below into the form $a+bi$. Then, choose the intervals that $a$ and $b$ belong to.
\[ \frac{18 + 44 i}{-1 + 7 i} \]

The solution is \( 5.80  - 3.40 i \), which is option D.\begin{enumerate}[label=\Alph*.]
\item \( a \in [-18.5, -17.5] \text{ and } b \in [5.5, 6.5] \)

 $-18.00  + 6.29 i$, which corresponds to just dividing the first term by the first term and the second by the second.
\item \( a \in [289, 290.5] \text{ and } b \in [-4, -3] \)

 $290.00  - 3.40 i$, which corresponds to forgetting to multiply the conjugate by the numerator and using a plus instead of a minus in the denominator.
\item \( a \in [-7, -6.5] \text{ and } b \in [1, 3] \)

 $-6.52  + 1.64 i$, which corresponds to forgetting to multiply the conjugate by the numerator and not computing the conjugate correctly.
\item \( a \in [4, 6.5] \text{ and } b \in [-4, -3] \)

* $5.80  - 3.40 i$, which is the correct option.
\item \( a \in [4, 6.5] \text{ and } b \in [-171, -169] \)

 $5.80  - 170.00 i$, which corresponds to forgetting to multiply the conjugate by the numerator.
\end{enumerate}

\textbf{General Comment:} Multiply the numerator and denominator by the *conjugate* of the denominator, then simplify. For example, if we have $2+3i$, the conjugate is $2-3i$.
}
\litem{
Simplify the expression below and choose the interval the simplification is contained within.
\[ 15 - 16^2 + 6 \div 8 * 20 \div 18 \]

The solution is \( -240.167 \), which is option C.\begin{enumerate}[label=\Alph*.]
\item \( [-241.14, -240.53] \)

 -240.998, which corresponds to an Order of Operations error: not reading left-to-right for multiplication/division.
\item \( [270.95, 271.64] \)

 271.002, which corresponds to two Order of Operations errors.
\item \( [-240.18, -239.57] \)

* -240.167, this is the correct option
\item \( [271.36, 272.12] \)

 271.833, which corresponds to an Order of Operations error: multiplying by negative before squaring. For example: $(-3)^2 \neq -3^2$
\item \( \text{None of the above} \)

 You may have gotten this by making an unanticipated error. If you got a value that is not any of the others, please let the coordinator know so they can help you figure out what happened.
\end{enumerate}

\textbf{General Comment:} While you may remember (or were taught) PEMDAS is done in order, it is actually done as P/E/MD/AS. When we are at MD or AS, we read left to right.
}
\litem{
Simplify the expression below into the form $a+bi$. Then, choose the intervals that $a$ and $b$ belong to.
\[ \frac{18 - 33 i}{1 - 7 i} \]

The solution is \( 4.98  + 1.86 i \), which is option D.\begin{enumerate}[label=\Alph*.]
\item \( a \in [247.5, 249.5] \text{ and } b \in [1, 3] \)

 $249.00  + 1.86 i$, which corresponds to forgetting to multiply the conjugate by the numerator and using a plus instead of a minus in the denominator.
\item \( a \in [-6.5, -3.5] \text{ and } b \in [-3.5, -2.5] \)

 $-4.26  - 3.18 i$, which corresponds to forgetting to multiply the conjugate by the numerator and not computing the conjugate correctly.
\item \( a \in [17, 19] \text{ and } b \in [4, 5] \)

 $18.00  + 4.71 i$, which corresponds to just dividing the first term by the first term and the second by the second.
\item \( a \in [3.5, 5.5] \text{ and } b \in [1, 3] \)

* $4.98  + 1.86 i$, which is the correct option.
\item \( a \in [3.5, 5.5] \text{ and } b \in [92, 93.5] \)

 $4.98  + 93.00 i$, which corresponds to forgetting to multiply the conjugate by the numerator.
\end{enumerate}

\textbf{General Comment:} Multiply the numerator and denominator by the *conjugate* of the denominator, then simplify. For example, if we have $2+3i$, the conjugate is $2-3i$.
}
\end{enumerate}

\end{document}