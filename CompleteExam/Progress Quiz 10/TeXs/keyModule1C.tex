\documentclass{extbook}[14pt]
\usepackage{multicol, enumerate, enumitem, hyperref, color, soul, setspace, parskip, fancyhdr, amssymb, amsthm, amsmath, latexsym, units, mathtools}
\everymath{\displaystyle}
\usepackage[headsep=0.5cm,headheight=0cm, left=1 in,right= 1 in,top= 1 in,bottom= 1 in]{geometry}
\usepackage{dashrule}  % Package to use the command below to create lines between items
\newcommand{\litem}[1]{\item #1

\rule{\textwidth}{0.4pt}}
\pagestyle{fancy}
\lhead{}
\chead{Answer Key for Progress Quiz 10 Version C}
\rhead{}
\lfoot{1995-1928}
\cfoot{}
\rfoot{test}
\begin{document}
\textbf{This key should allow you to understand why you choose the option you did (beyond just getting a question right or wrong). \href{https://xronos.clas.ufl.edu/mac1105spring2020/courseDescriptionAndMisc/Exams/LearningFromResults}{More instructions on how to use this key can be found here}.}

\textbf{If you have a suggestion to make the keys better, \href{https://forms.gle/CZkbZmPbC9XALEE88}{please fill out the short survey here}.}

\textit{Note: This key is auto-generated and may contain issues and/or errors. The keys are reviewed after each exam to ensure grading is done accurately. If there are issues (like duplicate options), they are noted in the offline gradebook. The keys are a work-in-progress to give students as many resources to improve as possible.}

\rule{\textwidth}{0.4pt}

\begin{enumerate}\litem{
Simplify the expression below into the form $a+bi$. Then, choose the intervals that $a$ and $b$ belong to.
\[ (3 + 4 i)(-9 - 10 i) \]The solution is \( 13 - 66 i \), which is option A.\begin{enumerate}[label=\Alph*.]
\item \( a \in [13, 21] \text{ and } b \in [-70, -58] \)

* $13 - 66 i$, which is the correct option.
\item \( a \in [-70, -63] \text{ and } b \in [2, 14] \)

 $-67 + 6 i$, which corresponds to adding a minus sign in the first term.
\item \( a \in [13, 21] \text{ and } b \in [65, 70] \)

 $13 + 66 i$, which corresponds to adding a minus sign in both terms.
\item \( a \in [-27, -26] \text{ and } b \in [-43, -38] \)

 $-27 - 40 i$, which corresponds to just multiplying the real terms to get the real part of the solution and the coefficients in the complex terms to get the complex part.
\item \( a \in [-70, -63] \text{ and } b \in [-10, -1] \)

 $-67 - 6 i$, which corresponds to adding a minus sign in the second term.
\end{enumerate}

\textbf{General Comment:} You can treat $i$ as a variable and distribute. Just remember that $i^2=-1$, so you can continue to reduce after you distribute.
}
\litem{
Choose the \textbf{smallest} set of Complex numbers that the number below belongs to.
\[ \frac{\sqrt{154}}{19}+3i^2 \]The solution is \( \text{Irrational} \), which is option B.\begin{enumerate}[label=\Alph*.]
\item \( \text{Nonreal Complex} \)

This is a Complex number $(a+bi)$ that is not Real (has $i$ as part of the number).
\item \( \text{Irrational} \)

* This is the correct option!
\item \( \text{Rational} \)

These are numbers that can be written as fraction of Integers (e.g., -2/3 + 5)
\item \( \text{Not a Complex Number} \)

This is not a number. The only non-Complex number we know is dividing by 0 as this is not a number!
\item \( \text{Pure Imaginary} \)

This is a Complex number $(a+bi)$ that \textbf{only} has an imaginary part like $2i$.
\end{enumerate}

\textbf{General Comment:} Be sure to simplify $i^2 = -1$. This may remove the imaginary portion for your number. If you are having trouble, you may want to look at the \textit{Subgroups of the Real Numbers} section.
}
\litem{
Simplify the expression below and choose the interval the simplification is contained within.
\[ 17 - 5 \div 10 * 2 - (13 * 15) \]The solution is \( -179.000 \), which is option A.\begin{enumerate}[label=\Alph*.]
\item \( [-179.78, -178.41] \)

* -179.000, which is the correct option.
\item \( [44.85, 45.3] \)

 45.000, which corresponds to not distributing a negative correctly.
\item \( [210.63, 212.51] \)

 211.750, which corresponds to not distributing addition and subtraction correctly.
\item \( [-178.61, -178.11] \)

 -178.250, which corresponds to an Order of Operations error: not reading left-to-right for multiplication/division.
\item \( \text{None of the above} \)

 You may have gotten this by making an unanticipated error. If you got a value that is not any of the others, please let the coordinator know so they can help you figure out what happened.
\end{enumerate}

\textbf{General Comment:} While you may remember (or were taught) PEMDAS is done in order, it is actually done as P/E/MD/AS. When we are at MD or AS, we read left to right.
}
\litem{
Choose the \textbf{smallest} set of Complex numbers that the number below belongs to.
\[ \frac{-4}{-19}+\sqrt{-36}i \]The solution is \( \text{Rational} \), which is option A.\begin{enumerate}[label=\Alph*.]
\item \( \text{Rational} \)

* This is the correct option!
\item \( \text{Not a Complex Number} \)

This is not a number. The only non-Complex number we know is dividing by 0 as this is not a number!
\item \( \text{Nonreal Complex} \)

This is a Complex number $(a+bi)$ that is not Real (has $i$ as part of the number).
\item \( \text{Pure Imaginary} \)

This is a Complex number $(a+bi)$ that \textbf{only} has an imaginary part like $2i$.
\item \( \text{Irrational} \)

These cannot be written as a fraction of Integers. Remember: $\pi$ is not an Integer!
\end{enumerate}

\textbf{General Comment:} Be sure to simplify $i^2 = -1$. This may remove the imaginary portion for your number. If you are having trouble, you may want to look at the \textit{Subgroups of the Real Numbers} section.
}
\litem{
Choose the \textbf{smallest} set of Real numbers that the number below belongs to.
\[ -\sqrt{\frac{1456}{7}} \]The solution is \( \text{Irrational} \), which is option D.\begin{enumerate}[label=\Alph*.]
\item \( \text{Integer} \)

These are the negative and positive counting numbers (..., -3, -2, -1, 0, 1, 2, 3, ...)
\item \( \text{Whole} \)

These are the counting numbers with 0 (0, 1, 2, 3, ...)
\item \( \text{Not a Real number} \)

These are Nonreal Complex numbers \textbf{OR} things that are not numbers (e.g., dividing by 0).
\item \( \text{Irrational} \)

* This is the correct option!
\item \( \text{Rational} \)

These are numbers that can be written as fraction of Integers (e.g., -2/3)
\end{enumerate}

\textbf{General Comment:} First, you \textbf{NEED} to simplify the expression. This question simplifies to $-\sqrt{208}$. 
 
 Be sure you look at the simplified fraction and not just the decimal expansion. Numbers such as 13, 17, and 19 provide \textbf{long but repeating/terminating decimal expansions!} 
 
 The only ways to *not* be a Real number are: dividing by 0 or taking the square root of a negative number. 
 
 Irrational numbers are more than just square root of 3: adding or subtracting values from square root of 3 is also irrational.
}
\litem{
Choose the \textbf{smallest} set of Real numbers that the number below belongs to.
\[ -\sqrt{\frac{-1870}{10}} \]The solution is \( \text{Not a Real number} \), which is option E.\begin{enumerate}[label=\Alph*.]
\item \( \text{Integer} \)

These are the negative and positive counting numbers (..., -3, -2, -1, 0, 1, 2, 3, ...)
\item \( \text{Rational} \)

These are numbers that can be written as fraction of Integers (e.g., -2/3)
\item \( \text{Whole} \)

These are the counting numbers with 0 (0, 1, 2, 3, ...)
\item \( \text{Irrational} \)

These cannot be written as a fraction of Integers.
\item \( \text{Not a Real number} \)

* This is the correct option!
\end{enumerate}

\textbf{General Comment:} First, you \textbf{NEED} to simplify the expression. This question simplifies to $-\sqrt{187} i$. 
 
 Be sure you look at the simplified fraction and not just the decimal expansion. Numbers such as 13, 17, and 19 provide \textbf{long but repeating/terminating decimal expansions!} 
 
 The only ways to *not* be a Real number are: dividing by 0 or taking the square root of a negative number. 
 
 Irrational numbers are more than just square root of 3: adding or subtracting values from square root of 3 is also irrational.
}
\litem{
Simplify the expression below into the form $a+bi$. Then, choose the intervals that $a$ and $b$ belong to.
\[ \frac{-54 + 88 i}{-4 - 5 i} \]The solution is \( -5.46  - 15.17 i \), which is option B.\begin{enumerate}[label=\Alph*.]
\item \( a \in [12, 14] \text{ and } b \in [-18.5, -16.5] \)

 $13.50  - 17.60 i$, which corresponds to just dividing the first term by the first term and the second by the second.
\item \( a \in [-6, -4.5] \text{ and } b \in [-16, -14] \)

* $-5.46  - 15.17 i$, which is the correct option.
\item \( a \in [-224.5, -223.5] \text{ and } b \in [-16, -14] \)

 $-224.00  - 15.17 i$, which corresponds to forgetting to multiply the conjugate by the numerator and using a plus instead of a minus in the denominator.
\item \( a \in [15.5, 16.5] \text{ and } b \in [-3, -0.5] \)

 $16.00  - 2.00 i$, which corresponds to forgetting to multiply the conjugate by the numerator and not computing the conjugate correctly.
\item \( a \in [-6, -4.5] \text{ and } b \in [-623.5, -621] \)

 $-5.46  - 622.00 i$, which corresponds to forgetting to multiply the conjugate by the numerator.
\end{enumerate}

\textbf{General Comment:} Multiply the numerator and denominator by the *conjugate* of the denominator, then simplify. For example, if we have $2+3i$, the conjugate is $2-3i$.
}
\litem{
Simplify the expression below and choose the interval the simplification is contained within.
\[ 18 - 6 \div 8 * 4 - (15 * 19) \]The solution is \( -270.000 \), which is option B.\begin{enumerate}[label=\Alph*.]
\item \( [-267.9, -266.2] \)

 -267.188, which corresponds to an Order of Operations error: not reading left-to-right for multiplication/division.
\item \( [-271.8, -269.1] \)

* -270.000, which is the correct option.
\item \( [-1, 2.4] \)

 0.000, which corresponds to not distributing a negative correctly.
\item \( [302.4, 305] \)

 302.812, which corresponds to not distributing addition and subtraction correctly.
\item \( \text{None of the above} \)

 You may have gotten this by making an unanticipated error. If you got a value that is not any of the others, please let the coordinator know so they can help you figure out what happened.
\end{enumerate}

\textbf{General Comment:} While you may remember (or were taught) PEMDAS is done in order, it is actually done as P/E/MD/AS. When we are at MD or AS, we read left to right.
}
\litem{
Simplify the expression below into the form $a+bi$. Then, choose the intervals that $a$ and $b$ belong to.
\[ \frac{-72 - 22 i}{7 - i} \]The solution is \( -9.64  - 4.52 i \), which is option E.\begin{enumerate}[label=\Alph*.]
\item \( a \in [-9.9, -9.54] \text{ and } b \in [-227, -224.5] \)

 $-9.64  - 226.00 i$, which corresponds to forgetting to multiply the conjugate by the numerator.
\item \( a \in [-10.42, -10.19] \text{ and } b \in [20.5, 22.5] \)

 $-10.29  + 22.00 i$, which corresponds to just dividing the first term by the first term and the second by the second.
\item \( a \in [-482.38, -481.94] \text{ and } b \in [-6, -4] \)

 $-482.00  - 4.52 i$, which corresponds to forgetting to multiply the conjugate by the numerator and using a plus instead of a minus in the denominator.
\item \( a \in [-10.64, -10.47] \text{ and } b \in [-2, -1] \)

 $-10.52  - 1.64 i$, which corresponds to forgetting to multiply the conjugate by the numerator and not computing the conjugate correctly.
\item \( a \in [-9.9, -9.54] \text{ and } b \in [-6, -4] \)

* $-9.64  - 4.52 i$, which is the correct option.
\end{enumerate}

\textbf{General Comment:} Multiply the numerator and denominator by the *conjugate* of the denominator, then simplify. For example, if we have $2+3i$, the conjugate is $2-3i$.
}
\litem{
Simplify the expression below into the form $a+bi$. Then, choose the intervals that $a$ and $b$ belong to.
\[ (-2 - 3 i)(-7 - 9 i) \]The solution is \( -13 + 39 i \), which is option D.\begin{enumerate}[label=\Alph*.]
\item \( a \in [38, 47] \text{ and } b \in [-4, 0] \)

 $41 - 3 i$, which corresponds to adding a minus sign in the first term.
\item \( a \in [14, 17] \text{ and } b \in [24, 30] \)

 $14 + 27 i$, which corresponds to just multiplying the real terms to get the real part of the solution and the coefficients in the complex terms to get the complex part.
\item \( a \in [38, 47] \text{ and } b \in [0, 7] \)

 $41 + 3 i$, which corresponds to adding a minus sign in the second term.
\item \( a \in [-17, -10] \text{ and } b \in [39, 45] \)

* $-13 + 39 i$, which is the correct option.
\item \( a \in [-17, -10] \text{ and } b \in [-42, -38] \)

 $-13 - 39 i$, which corresponds to adding a minus sign in both terms.
\end{enumerate}

\textbf{General Comment:} You can treat $i$ as a variable and distribute. Just remember that $i^2=-1$, so you can continue to reduce after you distribute.
}
\end{enumerate}

\end{document}