\documentclass{extbook}[14pt]
\usepackage{multicol, enumerate, enumitem, hyperref, color, soul, setspace, parskip, fancyhdr, amssymb, amsthm, amsmath, bbm, latexsym, units, mathtools}
\everymath{\displaystyle}
\usepackage[headsep=0.5cm,headheight=0cm, left=1 in,right= 1 in,top= 1 in,bottom= 1 in]{geometry}
\usepackage{dashrule}  % Package to use the command below to create lines between items
\newcommand{\litem}[1]{\item #1

\rule{\textwidth}{0.4pt}}
\pagestyle{fancy}
\lhead{}
\chead{Answer Key for Progress Quiz 10 Version B}
\rhead{}
\lfoot{6232-9639}
\cfoot{}
\rfoot{Fall 2020}
\begin{document}
\textbf{This key should allow you to understand why you choose the option you did (beyond just getting a question right or wrong). \href{https://xronos.clas.ufl.edu/mac1105spring2020/courseDescriptionAndMisc/Exams/LearningFromResults}{More instructions on how to use this key can be found here}.}

\textbf{If you have a suggestion to make the keys better, \href{https://forms.gle/CZkbZmPbC9XALEE88}{please fill out the short survey here}.}

\textit{Note: This key is auto-generated and may contain issues and/or errors. The keys are reviewed after each exam to ensure grading is done accurately. If there are issues (like duplicate options), they are noted in the offline gradebook. The keys are a work-in-progress to give students as many resources to improve as possible.}

\rule{\textwidth}{0.4pt}

\begin{enumerate}\litem{
Using an interval or intervals, describe all the $x$-values within or including a distance of the given values.
\[ \text{ More than } 10 \text{ units from the number } -5. \]

The solution is \( (-\infty, -15) \cup (5, \infty) \), which is option A.\begin{enumerate}[label=\Alph*.]
\item \( (-\infty, -15) \cup (5, \infty) \)

This describes the values more than 10 from -5
\item \( (-\infty, -15] \cup [5, \infty) \)

This describes the values no less than 10 from -5
\item \( (-15, 5) \)

This describes the values less than 10 from -5
\item \( [-15, 5] \)

This describes the values no more than 10 from -5
\item \( \text{None of the above} \)

You likely thought the values in the interval were not correct.
\end{enumerate}

\textbf{General Comment:} When thinking about this language, it helps to draw a number line and try points.
}
\litem{
Solve the linear inequality below. Then, choose the constant and interval combination that describes the solution set.
\[ -4x -5 < 3x + 6 \]

The solution is \( (-1.571, \infty) \), which is option C.\begin{enumerate}[label=\Alph*.]
\item \( (-\infty, a), \text{ where } a \in [1.4, 2.7] \)

 $(-\infty, 1.571)$, which corresponds to switching the direction of the interval AND negating the endpoint. You likely did this if you did not flip the inequality when dividing by a negative as well as not moving values over to a side properly.
\item \( (-\infty, a), \text{ where } a \in [-3.2, -0.6] \)

 $(-\infty, -1.571)$, which corresponds to switching the direction of the interval. You likely did this if you did not flip the inequality when dividing by a negative!
\item \( (a, \infty), \text{ where } a \in [-3.57, -0.57] \)

* $(-1.571, \infty)$, which is the correct option.
\item \( (a, \infty), \text{ where } a \in [-1.43, 7.57] \)

 $(1.571, \infty)$, which corresponds to negating the endpoint of the solution.
\item \( \text{None of the above}. \)

You may have chosen this if you thought the inequality did not match the ends of the intervals.
\end{enumerate}

\textbf{General Comment:} Remember that less/greater than or equal to includes the endpoint, while less/greater do not. Also, remember that you need to flip the inequality when you multiply or divide by a negative.
}
\litem{
Solve the linear inequality below. Then, choose the constant and interval combination that describes the solution set.
\[ -7 + 7 x > 8 x \text{ or } -7 + 3 x < 5 x \]

The solution is \( (-\infty, -7.0) \text{ or } (-3.5, \infty) \), which is option B.\begin{enumerate}[label=\Alph*.]
\item \( (-\infty, a) \cup (b, \infty), \text{ where } a \in [3.5, 9.5] \text{ and } b \in [6, 8] \)

Corresponds to inverting the inequality and negating the solution.
\item \( (-\infty, a) \cup (b, \infty), \text{ where } a \in [-7, -5] \text{ and } b \in [-6.5, 1.5] \)

 * Correct option.
\item \( (-\infty, a] \cup [b, \infty), \text{ where } a \in [3.5, 7.5] \text{ and } b \in [6, 9] \)

Corresponds to including the endpoints AND negating.
\item \( (-\infty, a] \cup [b, \infty), \text{ where } a \in [-8, -4] \text{ and } b \in [-4.5, 3.5] \)

Corresponds to including the endpoints (when they should be excluded).
\item \( (-\infty, \infty) \)

Corresponds to the variable canceling, which does not happen in this instance.
\end{enumerate}

\textbf{General Comment:} When multiplying or dividing by a negative, flip the sign.
}
\litem{
Solve the linear inequality below. Then, choose the constant and interval combination that describes the solution set.
\[ -6 - 8 x < \frac{-16 x + 7}{6} \leq 5 - 3 x \]

The solution is \( (-1.34, 11.50] \), which is option B.\begin{enumerate}[label=\Alph*.]
\item \( (-\infty, a] \cup (b, \infty), \text{ where } a \in [-5.34, 0.66] \text{ and } b \in [9.5, 13.5] \)

$(-\infty, -1.34] \cup (11.50, \infty)$, which corresponds to displaying the and-inequality as an or-inequality AND flipping the inequality.
\item \( (a, b], \text{ where } a \in [-2.34, 0.66] \text{ and } b \in [6.5, 14.5] \)

* $(-1.34, 11.50]$, which is the correct option.
\item \( [a, b), \text{ where } a \in [-1.6, -1.2] \text{ and } b \in [11.5, 20.5] \)

$[-1.34, 11.50)$, which corresponds to flipping the inequality.
\item \( (-\infty, a) \cup [b, \infty), \text{ where } a \in [-1.9, 0.5] \text{ and } b \in [8.5, 13.5] \)

$(-\infty, -1.34) \cup [11.50, \infty)$, which corresponds to displaying the and-inequality as an or-inequality.
\item \( \text{None of the above.} \)


\end{enumerate}

\textbf{General Comment:} To solve, you will need to break up the compound inequality into two inequalities. Be sure to keep track of the inequality! It may be best to draw a number line and graph your solution.
}
\litem{
Solve the linear inequality below. Then, choose the constant and interval combination that describes the solution set.
\[ -9 + 7 x > 8 x \text{ or } -4 + 6 x < 9 x \]

The solution is \( (-\infty, -9.0) \text{ or } (-1.333, \infty) \), which is option D.\begin{enumerate}[label=\Alph*.]
\item \( (-\infty, a] \cup [b, \infty), \text{ where } a \in [-13, -8] \text{ and } b \in [-4.33, 3.67] \)

Corresponds to including the endpoints (when they should be excluded).
\item \( (-\infty, a) \cup (b, \infty), \text{ where } a \in [-2.67, 4.33] \text{ and } b \in [7, 14] \)

Corresponds to inverting the inequality and negating the solution.
\item \( (-\infty, a] \cup [b, \infty), \text{ where } a \in [0.33, 2.33] \text{ and } b \in [7, 15] \)

Corresponds to including the endpoints AND negating.
\item \( (-\infty, a) \cup (b, \infty), \text{ where } a \in [-11, -8] \text{ and } b \in [-5.33, 4.67] \)

 * Correct option.
\item \( (-\infty, \infty) \)

Corresponds to the variable canceling, which does not happen in this instance.
\end{enumerate}

\textbf{General Comment:} When multiplying or dividing by a negative, flip the sign.
}
\litem{
Using an interval or intervals, describe all the $x$-values within or including a distance of the given values.
\[ \text{ Less than } 7 \text{ units from the number } 10. \]

The solution is \( \text{None of the above} \), which is option E.\begin{enumerate}[label=\Alph*.]
\item \( (-3, 17) \)

This describes the values less than 10 from 7
\item \( (-\infty, -3) \cup (17, \infty) \)

This describes the values more than 10 from 7
\item \( (-\infty, -3] \cup [17, \infty) \)

This describes the values no less than 10 from 7
\item \( [-3, 17] \)

This describes the values no more than 10 from 7
\item \( \text{None of the above} \)

Options A-D described the values [more/less than] 10 units from 7, which is the reverse of what the question asked.
\end{enumerate}

\textbf{General Comment:} When thinking about this language, it helps to draw a number line and try points.
}
\litem{
Solve the linear inequality below. Then, choose the constant and interval combination that describes the solution set.
\[ 4 - 6 x < \frac{-20 x - 6}{5} \leq 4 - 5 x \]

The solution is \( \text{None of the above.} \), which is option E.\begin{enumerate}[label=\Alph*.]
\item \( (-\infty, a) \cup [b, \infty), \text{ where } a \in [-4.6, -1.6] \text{ and } b \in [-7.2, -2.2] \)

$(-\infty, -2.60) \cup [-5.20, \infty)$, which corresponds to displaying the and-inequality as an or-inequality and getting negatives of the actual endpoints.
\item \( (a, b], \text{ where } a \in [-2.6, -1.6] \text{ and } b \in [-8.2, -2.2] \)

$(-2.60, -5.20]$, which is the correct interval but negatives of the actual endpoints.
\item \( [a, b), \text{ where } a \in [-4.6, 1.4] \text{ and } b \in [-10.2, -2.2] \)

$[-2.60, -5.20)$, which corresponds to flipping the inequality and getting negatives of the actual endpoints.
\item \( (-\infty, a] \cup (b, \infty), \text{ where } a \in [-8.6, 0.4] \text{ and } b \in [-5.2, -1.2] \)

$(-\infty, -2.60] \cup (-5.20, \infty)$, which corresponds to displaying the and-inequality as an or-inequality AND flipping the inequality AND getting negatives of the actual endpoints.
\item \( \text{None of the above.} \)

* This is correct as the answer should be $(2.60, 5.20]$.
\end{enumerate}

\textbf{General Comment:} To solve, you will need to break up the compound inequality into two inequalities. Be sure to keep track of the inequality! It may be best to draw a number line and graph your solution.
}
\litem{
Solve the linear inequality below. Then, choose the constant and interval combination that describes the solution set.
\[ \frac{5}{3} - \frac{9}{6} x \leq \frac{-3}{7} x - \frac{5}{5} \]

The solution is \( [2.489, \infty) \), which is option A.\begin{enumerate}[label=\Alph*.]
\item \( [a, \infty), \text{ where } a \in [1.49, 3.49] \)

* $[2.489, \infty)$, which is the correct option.
\item \( (-\infty, a], \text{ where } a \in [1.49, 6.49] \)

 $(-\infty, 2.489]$, which corresponds to switching the direction of the interval. You likely did this if you did not flip the inequality when dividing by a negative!
\item \( [a, \infty), \text{ where } a \in [-5.49, -0.49] \)

 $[-2.489, \infty)$, which corresponds to negating the endpoint of the solution.
\item \( (-\infty, a], \text{ where } a \in [-6.49, 0.51] \)

 $(-\infty, -2.489]$, which corresponds to switching the direction of the interval AND negating the endpoint. You likely did this if you did not flip the inequality when dividing by a negative as well as not moving values over to a side properly.
\item \( \text{None of the above}. \)

You may have chosen this if you thought the inequality did not match the ends of the intervals.
\end{enumerate}

\textbf{General Comment:} Remember that less/greater than or equal to includes the endpoint, while less/greater do not. Also, remember that you need to flip the inequality when you multiply or divide by a negative.
}
\litem{
Solve the linear inequality below. Then, choose the constant and interval combination that describes the solution set.
\[ \frac{-6}{2} - \frac{8}{8} x \leq \frac{-4}{3} x - \frac{5}{4} \]

The solution is \( (-\infty, 5.25] \), which is option A.\begin{enumerate}[label=\Alph*.]
\item \( (-\infty, a], \text{ where } a \in [1.25, 8.25] \)

* $(-\infty, 5.25]$, which is the correct option.
\item \( [a, \infty), \text{ where } a \in [-10.25, -3.25] \)

 $[-5.25, \infty)$, which corresponds to switching the direction of the interval AND negating the endpoint. You likely did this if you did not flip the inequality when dividing by a negative as well as not moving values over to a side properly.
\item \( [a, \infty), \text{ where } a \in [4.25, 6.25] \)

 $[5.25, \infty)$, which corresponds to switching the direction of the interval. You likely did this if you did not flip the inequality when dividing by a negative!
\item \( (-\infty, a], \text{ where } a \in [-7.25, -3.25] \)

 $(-\infty, -5.25]$, which corresponds to negating the endpoint of the solution.
\item \( \text{None of the above}. \)

You may have chosen this if you thought the inequality did not match the ends of the intervals.
\end{enumerate}

\textbf{General Comment:} Remember that less/greater than or equal to includes the endpoint, while less/greater do not. Also, remember that you need to flip the inequality when you multiply or divide by a negative.
}
\litem{
Solve the linear inequality below. Then, choose the constant and interval combination that describes the solution set.
\[ 4x + 3 \leq 6x -5 \]

The solution is \( [4.0, \infty) \), which is option D.\begin{enumerate}[label=\Alph*.]
\item \( (-\infty, a], \text{ where } a \in [1, 6] \)

 $(-\infty, 4.0]$, which corresponds to switching the direction of the interval. You likely did this if you did not flip the inequality when dividing by a negative!
\item \( [a, \infty), \text{ where } a \in [-5, -2] \)

 $[-4.0, \infty)$, which corresponds to negating the endpoint of the solution.
\item \( (-\infty, a], \text{ where } a \in [-8, 0] \)

 $(-\infty, -4.0]$, which corresponds to switching the direction of the interval AND negating the endpoint. You likely did this if you did not flip the inequality when dividing by a negative as well as not moving values over to a side properly.
\item \( [a, \infty), \text{ where } a \in [1, 5] \)

* $[4.0, \infty)$, which is the correct option.
\item \( \text{None of the above}. \)

You may have chosen this if you thought the inequality did not match the ends of the intervals.
\end{enumerate}

\textbf{General Comment:} Remember that less/greater than or equal to includes the endpoint, while less/greater do not. Also, remember that you need to flip the inequality when you multiply or divide by a negative.
}
\end{enumerate}

\end{document}