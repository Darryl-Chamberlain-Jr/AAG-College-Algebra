\documentclass{extbook}[14pt]
\usepackage{multicol, enumerate, enumitem, hyperref, color, soul, setspace, parskip, fancyhdr, amssymb, amsthm, amsmath, bbm, latexsym, units, mathtools}
\everymath{\displaystyle}
\usepackage[headsep=0.5cm,headheight=0cm, left=1 in,right= 1 in,top= 1 in,bottom= 1 in]{geometry}
\usepackage{dashrule}  % Package to use the command below to create lines between items
\newcommand{\litem}[1]{\item #1

\rule{\textwidth}{0.4pt}}
\pagestyle{fancy}
\lhead{}
\chead{Answer Key for Progress Quiz 10 Version B}
\rhead{}
\lfoot{6232-9639}
\cfoot{}
\rfoot{Fall 2020}
\begin{document}
\textbf{This key should allow you to understand why you choose the option you did (beyond just getting a question right or wrong). \href{https://xronos.clas.ufl.edu/mac1105spring2020/courseDescriptionAndMisc/Exams/LearningFromResults}{More instructions on how to use this key can be found here}.}

\textbf{If you have a suggestion to make the keys better, \href{https://forms.gle/CZkbZmPbC9XALEE88}{please fill out the short survey here}.}

\textit{Note: This key is auto-generated and may contain issues and/or errors. The keys are reviewed after each exam to ensure grading is done accurately. If there are issues (like duplicate options), they are noted in the offline gradebook. The keys are a work-in-progress to give students as many resources to improve as possible.}

\rule{\textwidth}{0.4pt}

\begin{enumerate}\litem{
Choose the \textbf{smallest} set of Complex numbers that the number below belongs to.
\[ \frac{13}{-7}+9i^2 \]

The solution is \( \text{Rational} \), which is option E.\begin{enumerate}[label=\Alph*.]
\item \( \text{Pure Imaginary} \)

This is a Complex number $(a+bi)$ that \textbf{only} has an imaginary part like $2i$.
\item \( \text{Nonreal Complex} \)

This is a Complex number $(a+bi)$ that is not Real (has $i$ as part of the number).
\item \( \text{Not a Complex Number} \)

This is not a number. The only non-Complex number we know is dividing by 0 as this is not a number!
\item \( \text{Irrational} \)

These cannot be written as a fraction of Integers. Remember: $\pi$ is not an Integer!
\item \( \text{Rational} \)

* This is the correct option!
\end{enumerate}

\textbf{General Comment:} Be sure to simplify $i^2 = -1$. This may remove the imaginary portion for your number. If you are having trouble, you may want to look at the \textit{Subgroups of the Real Numbers} section.
}
\litem{
Simplify the expression below and choose the interval the simplification is contained within.
\[ 9 - 15^2 + 11 \div 4 * 14 \div 17 \]

The solution is \( -213.735 \), which is option C.\begin{enumerate}[label=\Alph*.]
\item \( [-216.12, -215.29] \)

 -215.988, which corresponds to an Order of Operations error: not reading left-to-right for multiplication/division.
\item \( [235.09, 236.38] \)

 236.265, which corresponds to an Order of Operations error: multiplying by negative before squaring. For example: $(-3)^2 \neq -3^2$
\item \( [-214.19, -213.5] \)

* -213.735, this is the correct option
\item \( [231.73, 234.55] \)

 234.012, which corresponds to two Order of Operations errors.
\item \( \text{None of the above} \)

 You may have gotten this by making an unanticipated error. If you got a value that is not any of the others, please let the coordinator know so they can help you figure out what happened.
\end{enumerate}

\textbf{General Comment:} While you may remember (or were taught) PEMDAS is done in order, it is actually done as P/E/MD/AS. When we are at MD or AS, we read left to right.
}
\litem{
Choose the \textbf{smallest} set of Real numbers that the number below belongs to.
\[ \sqrt{\frac{130321}{361}} \]

The solution is \( \text{Whole} \), which is option C.\begin{enumerate}[label=\Alph*.]
\item \( \text{Irrational} \)

These cannot be written as a fraction of Integers.
\item \( \text{Not a Real number} \)

These are Nonreal Complex numbers \textbf{OR} things that are not numbers (e.g., dividing by 0).
\item \( \text{Whole} \)

* This is the correct option!
\item \( \text{Rational} \)

These are numbers that can be written as fraction of Integers (e.g., -2/3)
\item \( \text{Integer} \)

These are the negative and positive counting numbers (..., -3, -2, -1, 0, 1, 2, 3, ...)
\end{enumerate}

\textbf{General Comment:} First, you \textbf{NEED} to simplify the expression. This question simplifies to $361$. 
 
 Be sure you look at the simplified fraction and not just the decimal expansion. Numbers such as 13, 17, and 19 provide \textbf{long but repeating/terminating decimal expansions!} 
 
 The only ways to *not* be a Real number are: dividing by 0 or taking the square root of a negative number. 
 
 Irrational numbers are more than just square root of 3: adding or subtracting values from square root of 3 is also irrational.
}
\litem{
Simplify the expression below into the form $a+bi$. Then, choose the intervals that $a$ and $b$ belong to.
\[ (-5 - 6 i)(-10 - 2 i) \]

The solution is \( 38 + 70 i \), which is option D.\begin{enumerate}[label=\Alph*.]
\item \( a \in [33, 39] \text{ and } b \in [-76, -67] \)

 $38 - 70 i$, which corresponds to adding a minus sign in both terms.
\item \( a \in [54, 63] \text{ and } b \in [47, 54] \)

 $62 + 50 i$, which corresponds to adding a minus sign in the second term.
\item \( a \in [54, 63] \text{ and } b \in [-54, -47] \)

 $62 - 50 i$, which corresponds to adding a minus sign in the first term.
\item \( a \in [33, 39] \text{ and } b \in [70, 74] \)

* $38 + 70 i$, which is the correct option.
\item \( a \in [47, 53] \text{ and } b \in [8, 17] \)

 $50 + 12 i$, which corresponds to just multiplying the real terms to get the real part of the solution and the coefficients in the complex terms to get the complex part.
\end{enumerate}

\textbf{General Comment:} You can treat $i$ as a variable and distribute. Just remember that $i^2=-1$, so you can continue to reduce after you distribute.
}
\litem{
Choose the \textbf{smallest} set of Complex numbers that the number below belongs to.
\[ \sqrt{\frac{0}{625}}+\sqrt{2}i \]

The solution is \( \text{Pure Imaginary} \), which is option D.\begin{enumerate}[label=\Alph*.]
\item \( \text{Not a Complex Number} \)

This is not a number. The only non-Complex number we know is dividing by 0 as this is not a number!
\item \( \text{Nonreal Complex} \)

This is a Complex number $(a+bi)$ that is not Real (has $i$ as part of the number).
\item \( \text{Irrational} \)

These cannot be written as a fraction of Integers. Remember: $\pi$ is not an Integer!
\item \( \text{Pure Imaginary} \)

* This is the correct option!
\item \( \text{Rational} \)

These are numbers that can be written as fraction of Integers (e.g., -2/3 + 5)
\end{enumerate}

\textbf{General Comment:} Be sure to simplify $i^2 = -1$. This may remove the imaginary portion for your number. If you are having trouble, you may want to look at the \textit{Subgroups of the Real Numbers} section.
}
\litem{
Simplify the expression below into the form $a+bi$. Then, choose the intervals that $a$ and $b$ belong to.
\[ (6 + 7 i)(-3 + 5 i) \]

The solution is \( -53 + 9 i \), which is option B.\begin{enumerate}[label=\Alph*.]
\item \( a \in [-21, -16] \text{ and } b \in [33, 38] \)

 $-18 + 35 i$, which corresponds to just multiplying the real terms to get the real part of the solution and the coefficients in the complex terms to get the complex part.
\item \( a \in [-53, -51] \text{ and } b \in [4, 14] \)

* $-53 + 9 i$, which is the correct option.
\item \( a \in [13, 18] \text{ and } b \in [51, 53] \)

 $17 + 51 i$, which corresponds to adding a minus sign in the first term.
\item \( a \in [-53, -51] \text{ and } b \in [-13, -6] \)

 $-53 - 9 i$, which corresponds to adding a minus sign in both terms.
\item \( a \in [13, 18] \text{ and } b \in [-53, -50] \)

 $17 - 51 i$, which corresponds to adding a minus sign in the second term.
\end{enumerate}

\textbf{General Comment:} You can treat $i$ as a variable and distribute. Just remember that $i^2=-1$, so you can continue to reduce after you distribute.
}
\litem{
Choose the \textbf{smallest} set of Real numbers that the number below belongs to.
\[ -\sqrt{\frac{81}{196}} \]

The solution is \( \text{Rational} \), which is option B.\begin{enumerate}[label=\Alph*.]
\item \( \text{Integer} \)

These are the negative and positive counting numbers (..., -3, -2, -1, 0, 1, 2, 3, ...)
\item \( \text{Rational} \)

* This is the correct option!
\item \( \text{Not a Real number} \)

These are Nonreal Complex numbers \textbf{OR} things that are not numbers (e.g., dividing by 0).
\item \( \text{Whole} \)

These are the counting numbers with 0 (0, 1, 2, 3, ...)
\item \( \text{Irrational} \)

These cannot be written as a fraction of Integers.
\end{enumerate}

\textbf{General Comment:} First, you \textbf{NEED} to simplify the expression. This question simplifies to $-\frac{9}{14}$. 
 
 Be sure you look at the simplified fraction and not just the decimal expansion. Numbers such as 13, 17, and 19 provide \textbf{long but repeating/terminating decimal expansions!} 
 
 The only ways to *not* be a Real number are: dividing by 0 or taking the square root of a negative number. 
 
 Irrational numbers are more than just square root of 3: adding or subtracting values from square root of 3 is also irrational.
}
\litem{
Simplify the expression below into the form $a+bi$. Then, choose the intervals that $a$ and $b$ belong to.
\[ \frac{72 - 66 i}{4 - 3 i} \]

The solution is \( 19.44  - 1.92 i \), which is option D.\begin{enumerate}[label=\Alph*.]
\item \( a \in [485.5, 486.5] \text{ and } b \in [-4, -0.5] \)

 $486.00  - 1.92 i$, which corresponds to forgetting to multiply the conjugate by the numerator and using a plus instead of a minus in the denominator.
\item \( a \in [17.5, 19] \text{ and } b \in [21.5, 22.5] \)

 $18.00  + 22.00 i$, which corresponds to just dividing the first term by the first term and the second by the second.
\item \( a \in [2, 4] \text{ and } b \in [-19.5, -19] \)

 $3.60  - 19.20 i$, which corresponds to forgetting to multiply the conjugate by the numerator and not computing the conjugate correctly.
\item \( a \in [18.5, 20] \text{ and } b \in [-4, -0.5] \)

* $19.44  - 1.92 i$, which is the correct option.
\item \( a \in [18.5, 20] \text{ and } b \in [-48.5, -47] \)

 $19.44  - 48.00 i$, which corresponds to forgetting to multiply the conjugate by the numerator.
\end{enumerate}

\textbf{General Comment:} Multiply the numerator and denominator by the *conjugate* of the denominator, then simplify. For example, if we have $2+3i$, the conjugate is $2-3i$.
}
\litem{
Simplify the expression below and choose the interval the simplification is contained within.
\[ 7 - 9 \div 5 * 18 - (1 * 2) \]

The solution is \( -27.400 \), which is option A.\begin{enumerate}[label=\Alph*.]
\item \( [-30.4, -24.4] \)

* -27.400, which is the correct option.
\item \( [-57.8, -45.8] \)

 -52.800, which corresponds to not distributing a negative correctly.
\item \( [2.9, 6.9] \)

 4.900, which corresponds to an Order of Operations error: not reading left-to-right for multiplication/division.
\item \( [6.9, 13.9] \)

 8.900, which corresponds to not distributing addition and subtraction correctly.
\item \( \text{None of the above} \)

 You may have gotten this by making an unanticipated error. If you got a value that is not any of the others, please let the coordinator know so they can help you figure out what happened.
\end{enumerate}

\textbf{General Comment:} While you may remember (or were taught) PEMDAS is done in order, it is actually done as P/E/MD/AS. When we are at MD or AS, we read left to right.
}
\litem{
Simplify the expression below into the form $a+bi$. Then, choose the intervals that $a$ and $b$ belong to.
\[ \frac{-27 + 22 i}{-5 - 8 i} \]

The solution is \( -0.46  - 3.66 i \), which is option D.\begin{enumerate}[label=\Alph*.]
\item \( a \in [-1.5, 1] \text{ and } b \in [-326.5, -325.5] \)

 $-0.46  - 326.00 i$, which corresponds to forgetting to multiply the conjugate by the numerator.
\item \( a \in [-42, -40.5] \text{ and } b \in [-4.5, -3] \)

 $-41.00  - 3.66 i$, which corresponds to forgetting to multiply the conjugate by the numerator and using a plus instead of a minus in the denominator.
\item \( a \in [4.5, 7] \text{ and } b \in [-3.5, -2.5] \)

 $5.40  - 2.75 i$, which corresponds to just dividing the first term by the first term and the second by the second.
\item \( a \in [-1.5, 1] \text{ and } b \in [-4.5, -3] \)

* $-0.46  - 3.66 i$, which is the correct option.
\item \( a \in [3, 5] \text{ and } b \in [0.5, 2] \)

 $3.49  + 1.19 i$, which corresponds to forgetting to multiply the conjugate by the numerator and not computing the conjugate correctly.
\end{enumerate}

\textbf{General Comment:} Multiply the numerator and denominator by the *conjugate* of the denominator, then simplify. For example, if we have $2+3i$, the conjugate is $2-3i$.
}
\end{enumerate}

\end{document}