\documentclass[14pt]{extbook}
\usepackage{multicol, enumerate, enumitem, hyperref, color, soul, setspace, parskip, fancyhdr} %General Packages
\usepackage{amssymb, amsthm, amsmath, bbm, latexsym, units, mathtools} %Math Packages
\everymath{\displaystyle} %All math in Display Style
% Packages with additional options
\usepackage[headsep=0.5cm,headheight=12pt, left=1 in,right= 1 in,top= 1 in,bottom= 1 in]{geometry}
\usepackage[usenames,dvipsnames]{xcolor}
\usepackage{dashrule}  % Package to use the command below to create lines between items
\newcommand{\litem}[1]{\item#1\hspace*{-1cm}\rule{\textwidth}{0.4pt}}
\pagestyle{fancy}
\lhead{Progress Quiz 10}
\chead{}
\rhead{Version A}
\lfoot{6232-9639}
\cfoot{}
\rfoot{Fall 2020}
\begin{document}

\begin{enumerate}
\litem{
A town has an initial population of 40000. The town's population for the next 10 years is provided below. Which type of function would be most appropriate to model the town's population?


\begin{tabular}{c|c|c|c|c|c|c|c|c|c}
\textbf{Year} & 1 & 2 & 3 & 4 & 5 & 6 & 7 & 8 & 9 \tabularnewline
\hline
\textbf{Pop.} & 40000 & 40027 & 40043 & 40055 & 40064 & 40071 & 40077 & 40083 & 40087
\end{tabular} \begin{enumerate}[label=\Alph*.]
\item \( \text{Non-Linear Power} \)
\item \( \text{Linear} \)
\item \( \text{Logarithmic} \)
\item \( \text{Exponential} \)
\item \( \text{None of the above} \)

\end{enumerate} }
\litem{
For the information provided below, construct a linear model that describes the total distance of the path, $D$, in terms of the time spent on a particular path \textit{if we know that the time spent on each path was equal}.
\begin{center}
    \textit{ A bicyclist is training for a race on a hilly path. Their bike keeps track of their speed at any time, but not the distance traveled. Their speed traveling up a hill is 5 mph, 10 mph when traveling down a hill, and 6 mph when traveling along a flat portion. }
\end{center}
\begin{enumerate}[label=\Alph*.]
\item \( 0.467 t \)
\item \( 21 t \)
\item \( 300 t \)
\item \( \text{The model can be found with the information provided, but isn't options 1-3.} \)
\item \( \text{The model cannot be found with the information provided.} \)

\end{enumerate} }
\litem{
Using the situation below, construct a linear model that describes the cost of the coffee beans $C(h)$ in terms of the weight of the high-quality coffee beans $h$.
\begin{center}
    \textit{ Veronica needs to prepare 170 of blended coffee beans selling for \$6.84 per pound. She has a high-quality bean that sells for \$7.49 a pound and a low-quality bean that sells for \$4.77 a pound. }
\end{center}
\begin{enumerate}[label=\Alph*.]
\item \( C(h) = 2.72 h + 810.90 \)
\item \( C(h) = -2.72 h + 1273.30 \)
\item \( C(h) = 7.49 h \)
\item \( C(h) = 6.13 h \)
\item \( \text{None of the above.} \)

\end{enumerate} }
\litem{
What is the \textbf{best} way to describe the domain of the scenario below?
\begin{center}
    \textit{ Bridges on highways often have expansion joints, which are small gaps in the roadway between one bridge section and the next. The gaps are put there so the bridge will have room to expand when the weather gets hot. Assume the gap width varies constantly with the temperature. Suppose a bridge has a gap of 1.3 cm when the temperature is 22 degrees C and that the gap narrows to 0.9 cm when the temperature warms to 30 degrees C. }
\end{center}
\begin{enumerate}[label=\Alph*.]
\item \( \text{Subset of the Natural numbers} \)
\item \( \text{Subset of the Rational numbers} \)
\item \( \text{Subset of the Integers} \)
\item \( \text{Proper subset of the Real numbers} \)
\item \( \text{There is no restricted domain in this scenario} \)

\end{enumerate} }
\litem{
What is the \textbf{best} way to describe the domain of the scenario below?
\begin{center}
    \textit{ Fred is a store manager at Publix. The store normally orders two pallets of water bottles a week and sells 1000 bottles per day. However, a hurricane is coming and Fred expects water bottle sales to increase tenfold for three days, then decrease by half of normal sales for four days. How many more pallets of water bottles should Fred order the week before the hurricane? }
\end{center}
\begin{enumerate}[label=\Alph*.]
\item \( \text{Proper subset of the Real numbers} \)
\item \( \text{Subset of the Natural numbers} \)
\item \( \text{Subset of the Rational numbers} \)
\item \( \text{Subset of the Integers} \)
\item \( \text{There is no restricted domain in this scenario} \)

\end{enumerate} }
\litem{
For the information provided below, construct a linear model that describes the total distance of the path, $D$, in terms of the time spent on a particular path \textit{if we know that all parts of the path are equal length}.
\begin{center}
    \textit{ A bicyclist is training for a race on a hilly path. Their bike keeps track of their speed at any time, but not the distance traveled. Their speed traveling up a hill is 3 mph, 8 mph when traveling down a hill, and 5 mph when traveling along a flat portion. }
\end{center}
\begin{enumerate}[label=\Alph*.]
\item \( 120 t \)
\item \( 0.658 t \)
\item \( 16 t \)
\item \( \text{The model can be found with the information provided, but isn't options 1-3.} \)
\item \( \text{The model cannot be found with the information provided.} \)

\end{enumerate} }
\litem{
Using the situation below, construct a linear model that describes the cost of the coffee beans $C(h)$ in terms of the weight of the high-quality coffee beans $h$.
\begin{center}
    \textit{ Veronica needs to prepare 200 of blended coffee beans selling for \$3.38 per pound. She has a high-quality bean that sells for \$4.15 a pound and a low-quality bean that sells for \$2.77 a pound. }
\end{center}
\begin{enumerate}[label=\Alph*.]
\item \( C(h) = 1.38 h + 554.00 \)
\item \( C(h) = 4.15 h \)
\item \( C(h) = 3.46 h \)
\item \( C(h) = -1.38 h + 830.00 \)
\item \( \text{None of the above.} \)

\end{enumerate} }
\litem{
For the information provided below, construct a linear model that describes her total budget, $B$, as a function of the number of months, $x$ she is at UF.
\begin{center}
    \textit{ Aubrey is a college student going into her first year at UF. She will receive Bright Futures, which covers her tuition plus a \$800 educational expense each year. Before college, Aubrey saved up \$11000. She knows she will need to pay \$700 in rent a month, \$50 for food a week, and \$64 in other weekly expenses. }
\end{center}
\begin{enumerate}[label=\Alph*.]
\item \( B(x) = 11800 - 814 x \)
\item \( B(x) = 11000 x + 800 \)
\item \( B(x) = 11800 - 1156 x \)
\item \( B(x) = 800 x + 11000 \)
\item \( \text{None of the above.} \)

\end{enumerate} }
\litem{
For the information provided below, construct a linear model that describes her total budget, $B$, as a function of the number of months, $x$ she is at UF.
\begin{center}
    \textit{ Aubrey is a college student going into her first year at UF. She will receive Bright Futures, which covers her tuition plus a \$600 educational expense each year. Before college, Aubrey saved up \$9000. She knows she will need to pay \$1200 in rent a month, \$60 for food a week, and \$64 in other weekly expenses. }
\end{center}
\begin{enumerate}[label=\Alph*.]
\item \( B(x) = 9600 - 1324 x \)
\item \( B(x) = 9000 x + 600 \)
\item \( B(x) = 9600 - 1696 x \)
\item \( B(x) = 600 x + 9000 \)
\item \( \text{None of the above.} \)

\end{enumerate} }
\litem{
A town has an initial population of 50000. The town's population for the next 10 years is provided below. Which type of function would be most appropriate to model the town's population?


\begin{tabular}{c|c|c|c|c|c|c|c|c|c}
\textbf{Year} & 1 & 2 & 3 & 4 & 5 & 6 & 7 & 8 & 9 \tabularnewline
\hline
\textbf{Pop.} & 50040 & 50080 & 50120 & 50160 & 50200 & 50240 & 50280 & 50320 & 50360
\end{tabular} \begin{enumerate}[label=\Alph*.]
\item \( \text{Exponential} \)
\item \( \text{Linear} \)
\item \( \text{Non-Linear Power} \)
\item \( \text{Logarithmic} \)
\item \( \text{None of the above} \)

\end{enumerate} }
\end{enumerate}

\end{document}