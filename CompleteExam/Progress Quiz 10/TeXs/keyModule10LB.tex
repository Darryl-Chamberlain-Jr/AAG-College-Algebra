\documentclass{extbook}[14pt]
\usepackage{multicol, enumerate, enumitem, hyperref, color, soul, setspace, parskip, fancyhdr, amssymb, amsthm, amsmath, bbm, latexsym, units, mathtools}
\everymath{\displaystyle}
\usepackage[headsep=0.5cm,headheight=0cm, left=1 in,right= 1 in,top= 1 in,bottom= 1 in]{geometry}
\usepackage{dashrule}  % Package to use the command below to create lines between items
\newcommand{\litem}[1]{\item #1

\rule{\textwidth}{0.4pt}}
\pagestyle{fancy}
\lhead{}
\chead{Answer Key for Progress Quiz 10 Version B}
\rhead{}
\lfoot{6232-9639}
\cfoot{}
\rfoot{Fall 2020}
\begin{document}
\textbf{This key should allow you to understand why you choose the option you did (beyond just getting a question right or wrong). \href{https://xronos.clas.ufl.edu/mac1105spring2020/courseDescriptionAndMisc/Exams/LearningFromResults}{More instructions on how to use this key can be found here}.}

\textbf{If you have a suggestion to make the keys better, \href{https://forms.gle/CZkbZmPbC9XALEE88}{please fill out the short survey here}.}

\textit{Note: This key is auto-generated and may contain issues and/or errors. The keys are reviewed after each exam to ensure grading is done accurately. If there are issues (like duplicate options), they are noted in the offline gradebook. The keys are a work-in-progress to give students as many resources to improve as possible.}

\rule{\textwidth}{0.4pt}

\begin{enumerate}\litem{
Perform the division below. Then, find the intervals that correspond to the quotient in the form $ax^2+bx+c$ and remainder $r$.
\[ \frac{16x^{3} +52 x^{2} -130 x + 47}{x + 5} \]

The solution is \( 16x^{2} -28 x + 10 + \frac{-3}{x + 5} \), which is option A.\begin{enumerate}[label=\Alph*.]
\item \( a \in [12, 25], \text{   } b \in [-32, -26], \text{   } c \in [8, 15], \text{   and   } r \in [-5, -2]. \)

* This is the solution!
\item \( a \in [12, 25], \text{   } b \in [128, 133], \text{   } c \in [530, 531], \text{   and   } r \in [2697, 2702]. \)

 You divided by the opposite of the factor.
\item \( a \in [-84, -79], \text{   } b \in [-352, -340], \text{   } c \in [-1875, -1866], \text{   and   } r \in [-9309, -9300]. \)

 You divided by the opposite of the factor AND multiplied the first factor rather than just bringing it down.
\item \( a \in [-84, -79], \text{   } b \in [451, 455], \text{   } c \in [-2392, -2386], \text{   and   } r \in [11988, 11999]. \)

 You multiplied by the synthetic number rather than bringing the first factor down.
\item \( a \in [12, 25], \text{   } b \in [-46, -40], \text{   } c \in [133, 135], \text{   and   } r \in [-762, -755]. \)

 You multiplied by the synthetic number and subtracted rather than adding during synthetic division.
\end{enumerate}

\textbf{General Comment:} Be sure to synthetically divide by the zero of the denominator!
}
\litem{
What are the \textit{possible Integer} roots of the polynomial below?
\[ f(x) = 4x^{4} +4 x^{3} +2 x^{2} +4 x + 6 \]

The solution is \( \pm 1,\pm 2,\pm 3,\pm 6 \), which is option B.\begin{enumerate}[label=\Alph*.]
\item \( \text{ All combinations of: }\frac{\pm 1,\pm 2,\pm 3,\pm 6}{\pm 1,\pm 2,\pm 4} \)

This would have been the solution \textbf{if asked for the possible Rational roots}!
\item \( \pm 1,\pm 2,\pm 3,\pm 6 \)

* This is the solution \textbf{since we asked for the possible Integer roots}!
\item \( \pm 1,\pm 2,\pm 4 \)

 Distractor 1: Corresponds to the plus or minus factors of a1 only.
\item \( \text{ All combinations of: }\frac{\pm 1,\pm 2,\pm 4}{\pm 1,\pm 2,\pm 3,\pm 6} \)

 Distractor 3: Corresponds to the plus or minus of the inverse quotient (an/a0) of the factors. 
\item \( \text{There is no formula or theorem that tells us all possible Integer roots.} \)

 Distractor 4: Corresponds to not recognizing Integers as a subset of Rationals.
\end{enumerate}

\textbf{General Comment:} We have a way to find the possible Rational roots. The possible Integer roots are the Integers in this list.
}
\litem{
Factor the polynomial below completely, knowing that $x-5$ is a factor. Then, choose the intervals the zeros of the polynomial belong to, where $z_1 \leq z_2 \leq z_3 \leq z_4$. \textit{To make the problem easier, all zeros are between -5 and 5.}
\[ f(x) = 12x^{4} -43 x^{3} -111 x^{2} +106 x + 120 \]

The solution is \( [-2, -0.75, 1.3333333333333333, 5] \), which is option C.\begin{enumerate}[label=\Alph*.]
\item \( z_1 \in [-2, 0], \text{   }  z_2 \in [-2.64, -1.1], z_3 \in [0.46, 0.96], \text{   and   } z_4 \in [4, 6] \)

 Distractor 2: Corresponds to inversing rational roots.
\item \( z_1 \in [-6, -4], \text{   }  z_2 \in [-2.64, -1.1], z_3 \in [0.46, 0.96], \text{   and   } z_4 \in [0, 3] \)

 Distractor 1: Corresponds to negatives of all zeros.
\item \( z_1 \in [-2, 0], \text{   }  z_2 \in [-1.33, 0.04], z_3 \in [1.19, 1.59], \text{   and   } z_4 \in [4, 6] \)

* This is the solution!
\item \( z_1 \in [-6, -4], \text{   }  z_2 \in [-1.33, 0.04], z_3 \in [1.19, 1.59], \text{   and   } z_4 \in [0, 3] \)

 Distractor 3: Corresponds to negatives of all zeros AND inversing rational roots.
\item \( z_1 \in [-6, -4], \text{   }  z_2 \in [-4.24, -3.31], z_3 \in [-0.22, 0.52], \text{   and   } z_4 \in [0, 3] \)

 Distractor 4: Corresponds to moving factors from one rational to another.
\end{enumerate}

\textbf{General Comment:} Remember to try the middle-most integers first as these normally are the zeros. Also, once you get it to a quadratic, you can use your other factoring techniques to finish factoring.
}
\litem{
Perform the division below. Then, find the intervals that correspond to the quotient in the form $ax^2+bx+c$ and remainder $r$.
\[ \frac{8x^{3} +34 x^{2} -39 x -42}{x + 5} \]

The solution is \( 8x^{2} -6 x -9 + \frac{3}{x + 5} \), which is option E.\begin{enumerate}[label=\Alph*.]
\item \( a \in [4, 12], \text{   } b \in [-16, -12], \text{   } c \in [44, 51], \text{   and   } r \in [-317, -309]. \)

 You multiplied by the synthetic number and subtracted rather than adding during synthetic division.
\item \( a \in [4, 12], \text{   } b \in [69, 80], \text{   } c \in [330, 332], \text{   and   } r \in [1613, 1619]. \)

 You divided by the opposite of the factor.
\item \( a \in [-41, -37], \text{   } b \in [-166, -164], \text{   } c \in [-876, -862], \text{   and   } r \in [-4388, -4381]. \)

 You divided by the opposite of the factor AND multiplied the first factor rather than just bringing it down.
\item \( a \in [-41, -37], \text{   } b \in [230, 235], \text{   } c \in [-1209, -1207], \text{   and   } r \in [6002, 6005]. \)

 You multiplied by the synthetic number rather than bringing the first factor down.
\item \( a \in [4, 12], \text{   } b \in [-7, -2], \text{   } c \in [-11, -7], \text{   and   } r \in [1, 8]. \)

* This is the solution!
\end{enumerate}

\textbf{General Comment:} Be sure to synthetically divide by the zero of the denominator!
}
\litem{
Perform the division below. Then, find the intervals that correspond to the quotient in the form $ax^2+bx+c$ and remainder $r$.
\[ \frac{12x^{3} +52 x^{2} -69}{x + 4} \]

The solution is \( 12x^{2} +4 x -16 + \frac{-5}{x + 4} \), which is option E.\begin{enumerate}[label=\Alph*.]
\item \( a \in [-50, -47], b \in [-140, -138], c \in [-560, -555], \text{ and } r \in [-2312, -2308]. \)

 You divided by the opposite of the factor AND multipled the first factor rather than just bringing it down.
\item \( a \in [8, 15], b \in [-11, -7], c \in [40, 42], \text{ and } r \in [-275, -265]. \)

 You multipled by the synthetic number and subtracted rather than adding during synthetic division.
\item \( a \in [-50, -47], b \in [238, 246], c \in [-981, -973], \text{ and } r \in [3835, 3838]. \)

 You multipled by the synthetic number rather than bringing the first factor down.
\item \( a \in [8, 15], b \in [99, 108], c \in [400, 403], \text{ and } r \in [1529, 1533]. \)

 You divided by the opposite of the factor.
\item \( a \in [8, 15], b \in [4, 8], c \in [-16, -15], \text{ and } r \in [-5, 0]. \)

* This is the solution!
\end{enumerate}

\textbf{General Comment:} Be sure to synthetically divide by the zero of the denominator! Also, make sure to include 0 placeholders for missing terms.
}
\litem{
What are the \textit{possible Integer} roots of the polynomial below?
\[ f(x) = 3x^{4} +4 x^{3} +6 x^{2} +3 x + 4 \]

The solution is \( \pm 1,\pm 2,\pm 4 \), which is option D.\begin{enumerate}[label=\Alph*.]
\item \( \pm 1,\pm 3 \)

 Distractor 1: Corresponds to the plus or minus factors of a1 only.
\item \( \text{ All combinations of: }\frac{\pm 1,\pm 3}{\pm 1,\pm 2,\pm 4} \)

 Distractor 3: Corresponds to the plus or minus of the inverse quotient (an/a0) of the factors. 
\item \( \text{ All combinations of: }\frac{\pm 1,\pm 2,\pm 4}{\pm 1,\pm 3} \)

This would have been the solution \textbf{if asked for the possible Rational roots}!
\item \( \pm 1,\pm 2,\pm 4 \)

* This is the solution \textbf{since we asked for the possible Integer roots}!
\item \( \text{There is no formula or theorem that tells us all possible Integer roots.} \)

 Distractor 4: Corresponds to not recognizing Integers as a subset of Rationals.
\end{enumerate}

\textbf{General Comment:} We have a way to find the possible Rational roots. The possible Integer roots are the Integers in this list.
}
\litem{
Perform the division below. Then, find the intervals that correspond to the quotient in the form $ax^2+bx+c$ and remainder $r$.
\[ \frac{4x^{3} -27 x + 32}{x + 3} \]

The solution is \( 4x^{2} -12 x + 9 + \frac{5}{x + 3} \), which is option B.\begin{enumerate}[label=\Alph*.]
\item \( a \in [4, 6], b \in [-16, -13], c \in [34, 40], \text{ and } r \in [-116, -115]. \)

 You multipled by the synthetic number and subtracted rather than adding during synthetic division.
\item \( a \in [4, 6], b \in [-12, -7], c \in [6, 12], \text{ and } r \in [1, 10]. \)

* This is the solution!
\item \( a \in [-13, -10], b \in [36, 38], c \in [-139, -125], \text{ and } r \in [433, 443]. \)

 You multipled by the synthetic number rather than bringing the first factor down.
\item \( a \in [4, 6], b \in [11, 17], c \in [6, 12], \text{ and } r \in [56, 61]. \)

 You divided by the opposite of the factor.
\item \( a \in [-13, -10], b \in [-41, -35], c \in [-139, -125], \text{ and } r \in [-375, -371]. \)

 You divided by the opposite of the factor AND multipled the first factor rather than just bringing it down.
\end{enumerate}

\textbf{General Comment:} Be sure to synthetically divide by the zero of the denominator! Also, make sure to include 0 placeholders for missing terms.
}
\litem{
Factor the polynomial below completely. Then, choose the intervals the zeros of the polynomial belong to, where $z_1 \leq z_2 \leq z_3$. \textit{To make the problem easier, all zeros are between -5 and 5.}
\[ f(x) = 12x^{3} +37 x^{2} -59 x -60 \]

The solution is \( [-4, -0.75, 1.6666666666666667] \), which is option B.\begin{enumerate}[label=\Alph*.]
\item \( z_1 \in [-0.9, 0.4], \text{   }  z_2 \in [1.17, 1.85], \text{   and   } z_3 \in [3.5, 5.1] \)

 Distractor 3: Corresponds to negatives of all zeros AND inversing rational roots.
\item \( z_1 \in [-4.1, -3], \text{   }  z_2 \in [-1.14, -0.72], \text{   and   } z_3 \in [0.9, 2.6] \)

* This is the solution!
\item \( z_1 \in [-4.1, -3], \text{   }  z_2 \in [-1.69, -0.93], \text{   and   } z_3 \in [-0.6, 1.5] \)

 Distractor 2: Corresponds to inversing rational roots.
\item \( z_1 \in [-5.2, -4.8], \text{   }  z_2 \in [-0.41, 0.52], \text{   and   } z_3 \in [3.5, 5.1] \)

 Distractor 4: Corresponds to moving factors from one rational to another.
\item \( z_1 \in [-2.5, -1], \text{   }  z_2 \in [0.55, 0.98], \text{   and   } z_3 \in [3.5, 5.1] \)

 Distractor 1: Corresponds to negatives of all zeros.
\end{enumerate}

\textbf{General Comment:} Remember to try the middle-most integers first as these normally are the zeros. Also, once you get it to a quadratic, you can use your other factoring techniques to finish factoring.
}
\litem{
Factor the polynomial below completely. Then, choose the intervals the zeros of the polynomial belong to, where $z_1 \leq z_2 \leq z_3$. \textit{To make the problem easier, all zeros are between -5 and 5.}
\[ f(x) = 8x^{3} -62 x^{2} +145 x -100 \]

The solution is \( [1.25, 2.5, 4] \), which is option D.\begin{enumerate}[label=\Alph*.]
\item \( z_1 \in [-4.3, -3.9], \text{   }  z_2 \in [-1.4, -0.6], \text{   and   } z_3 \in [-0.48, 0.04] \)

 Distractor 3: Corresponds to negatives of all zeros AND inversing rational roots.
\item \( z_1 \in [-0.7, 0.8], \text{   }  z_2 \in [0.1, 0.9], \text{   and   } z_3 \in [3.87, 4.22] \)

 Distractor 2: Corresponds to inversing rational roots.
\item \( z_1 \in [-5.9, -4.2], \text{   }  z_2 \in [-5.5, -3.6], \text{   and   } z_3 \in [-0.75, -0.42] \)

 Distractor 4: Corresponds to moving factors from one rational to another.
\item \( z_1 \in [0.6, 1.4], \text{   }  z_2 \in [1.8, 2.7], \text{   and   } z_3 \in [3.87, 4.22] \)

* This is the solution!
\item \( z_1 \in [-4.3, -3.9], \text{   }  z_2 \in [-2.8, -1.7], \text{   and   } z_3 \in [-1.31, -1.11] \)

 Distractor 1: Corresponds to negatives of all zeros.
\end{enumerate}

\textbf{General Comment:} Remember to try the middle-most integers first as these normally are the zeros. Also, once you get it to a quadratic, you can use your other factoring techniques to finish factoring.
}
\litem{
Factor the polynomial below completely, knowing that $x-2$ is a factor. Then, choose the intervals the zeros of the polynomial belong to, where $z_1 \leq z_2 \leq z_3 \leq z_4$. \textit{To make the problem easier, all zeros are between -5 and 5.}
\[ f(x) = 15x^{4} +46 x^{3} -84 x^{2} -120 x -32 \]

The solution is \( [-4, -0.6666666666666666, -0.4, 2] \), which is option C.\begin{enumerate}[label=\Alph*.]
\item \( z_1 \in [-7, -3], \text{   }  z_2 \in [-2.5, -2.17], z_3 \in [-1.76, -0.91], \text{   and   } z_4 \in [2, 3] \)

 Distractor 2: Corresponds to inversing rational roots.
\item \( z_1 \in [-3, 0], \text{   }  z_2 \in [0.18, 0.56], z_3 \in [0.59, 1.05], \text{   and   } z_4 \in [4, 5] \)

 Distractor 1: Corresponds to negatives of all zeros.
\item \( z_1 \in [-7, -3], \text{   }  z_2 \in [-0.71, -0.23], z_3 \in [-0.44, -0.2], \text{   and   } z_4 \in [2, 3] \)

* This is the solution!
\item \( z_1 \in [-3, 0], \text{   }  z_2 \in [1.35, 1.94], z_3 \in [2.33, 2.69], \text{   and   } z_4 \in [4, 5] \)

 Distractor 3: Corresponds to negatives of all zeros AND inversing rational roots.
\item \( z_1 \in [-3, 0], \text{   }  z_2 \in [-0.21, 0.26], z_3 \in [1.42, 2.32], \text{   and   } z_4 \in [4, 5] \)

 Distractor 4: Corresponds to moving factors from one rational to another.
\end{enumerate}

\textbf{General Comment:} Remember to try the middle-most integers first as these normally are the zeros. Also, once you get it to a quadratic, you can use your other factoring techniques to finish factoring.
}
\end{enumerate}

\end{document}