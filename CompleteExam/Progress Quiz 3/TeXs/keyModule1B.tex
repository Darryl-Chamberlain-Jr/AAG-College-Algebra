\documentclass{extbook}[14pt]
\usepackage{multicol, enumerate, enumitem, hyperref, color, soul, setspace, parskip, fancyhdr, amssymb, amsthm, amsmath, bbm, latexsym, units, mathtools}
\everymath{\displaystyle}
\usepackage[headsep=0.5cm,headheight=0cm, left=1 in,right= 1 in,top= 1 in,bottom= 1 in]{geometry}
\usepackage{dashrule}  % Package to use the command below to create lines between items
\newcommand{\litem}[1]{\item #1

\rule{\textwidth}{0.4pt}}
\pagestyle{fancy}
\lhead{}
\chead{Answer Key for Progress Quiz 3 Version B}
\rhead{}
\lfoot{}
\cfoot{}
\rfoot{Fall 2020}
\begin{document}
\textbf{This key should allow you to understand why you choose the option you did (beyond just getting a question right or wrong). \href{https://xronos.clas.ufl.edu/mac1105spring2020/courseDescriptionAndMisc/Exams/LearningFromResults}{More instructions on how to use this key can be found here}.}

\textbf{If you have a suggestion to make the keys better, \href{https://forms.gle/CZkbZmPbC9XALEE88}{please fill out the short survey here}.}

\textit{Note: This key is auto-generated and may contain issues and/or errors. The keys are reviewed after each exam to ensure grading is done accurately. If there are issues (like duplicate options), they are noted in the offline gradebook. The keys are a work-in-progress to give students as many resources to improve as possible.}

\rule{\textwidth}{0.4pt}

\begin{enumerate}\litem{
Simplify the expression below and choose the interval the simplification is contained within.
\[ 16 - 9^2 + 4 \div 17 * 14 \div 3 \]
The solution is \( -63.902 \), which is option C.\begin{enumerate}[label=\Alph*.]
\item \( [-65.1, -64.36] \)

 -64.994, which corresponds to an Order of Operations error: not reading left-to-right for multiplication/division.
\item \( [97.97, 98.99] \)

 98.098, which corresponds to an Order of Operations error: multiplying by negative before squaring. For example: $(-3)^2 \neq -3^2$
\item \( [-64.37, -63.54] \)

* -63.902, this is the correct option
\item \( [96, 97.36] \)

 97.006, which corresponds to two Order of Operations errors.
\item \( \text{None of the above} \)

 You may have gotten this by making an unanticipated error. If you got a value that is not any of the others, please let the coordinator know so they can help you figure out what happened.
\end{enumerate}

\textbf{General Comment:} While you may remember (or were taught) PEMDAS is done in order, it is actually done as P/E/MD/AS. When we are at MD or AS, we read left to right.
}
\litem{
Choose the \textbf{smallest} set of Real numbers that the number below belongs to.
\[ \sqrt{\frac{3600}{100}} \]
The solution is \( \text{Whole} \), which is option B.\begin{enumerate}[label=\Alph*.]
\item \( \text{Irrational} \)

These cannot be written as a fraction of Integers.
\item \( \text{Whole} \)

* This is the correct option!
\item \( \text{Not a Real number} \)

These are Nonreal Complex numbers \textbf{OR} things that are not numbers (e.g., dividing by 0).
\item \( \text{Integer} \)

These are the negative and positive counting numbers (..., -3, -2, -1, 0, 1, 2, 3, ...)
\item \( \text{Rational} \)

These are numbers that can be written as fraction of Integers (e.g., -2/3)
\end{enumerate}

\textbf{General Comment:} First, you \textbf{NEED} to simplify the expression. This question simplifies to $60$. 
 
 Be sure you look at the simplified fraction and not just the decimal expansion. Numbers such as 13, 17, and 19 provide \textbf{long but repeating/terminating decimal expansions!} 
 
 The only ways to *not* be a Real number are: dividing by 0 or taking the square root of a negative number. 
 
 Irrational numbers are more than just square root of 3: adding or subtracting values from square root of 3 is also irrational.
}
\litem{
Choose the \textbf{smallest} set of Complex numbers that the number below belongs to.
\[ \sqrt{\frac{64}{225}}+\sqrt{63} i \]
The solution is \( \text{Nonreal Complex} \), which is option B.\begin{enumerate}[label=\Alph*.]
\item \( \text{Rational} \)

These are numbers that can be written as fraction of Integers (e.g., -2/3 + 5)
\item \( \text{Nonreal Complex} \)

* This is the correct option!
\item \( \text{Not a Complex Number} \)

This is not a number. The only non-Complex number we know is dividing by 0 as this is not a number!
\item \( \text{Irrational} \)

These cannot be written as a fraction of Integers. Remember: $\pi$ is not an Integer!
\item \( \text{Pure Imaginary} \)

This is a Complex number $(a+bi)$ that \textbf{only} has an imaginary part like $2i$.
\end{enumerate}

\textbf{General Comment:} Be sure to simplify $i^2 = -1$. This may remove the imaginary portion for your number. If you are having trouble, you may want to look at the \textit{Subgroups of the Real Numbers} section.
}
\litem{
Simplify the expression below and choose the interval the simplification is contained within.
\[ 2 - 10 \div 18 * 14 - (1 * 15) \]
The solution is \( -20.778 \), which is option D.\begin{enumerate}[label=\Alph*.]
\item \( [-105.67, -98.67] \)

 -101.667, which corresponds to not distributing a negative correctly.
\item \( [-13.04, -8.04] \)

 -13.040, which corresponds to an Order of Operations error: not reading left-to-right for multiplication/division.
\item \( [9.96, 23.96] \)

 16.960, which corresponds to not distributing addition and subtraction correctly.
\item \( [-21.78, -15.78] \)

* -20.778, which is the correct option.
\item \( \text{None of the above} \)

 You may have gotten this by making an unanticipated error. If you got a value that is not any of the others, please let the coordinator know so they can help you figure out what happened.
\end{enumerate}

\textbf{General Comment:} While you may remember (or were taught) PEMDAS is done in order, it is actually done as P/E/MD/AS. When we are at MD or AS, we read left to right.
}
\litem{
Simplify the expression below into the form $a+bi$. Then, choose the intervals that $a$ and $b$ belong to.
\[ (6 + 7 i)(-5 + 4 i) \]
The solution is \( -58 - 11 i \), which is option A.\begin{enumerate}[label=\Alph*.]
\item \( a \in [-61, -51] \text{ and } b \in [-14, -4] \)

* $-58 - 11 i$, which is the correct option.
\item \( a \in [-3, -1] \text{ and } b \in [49, 62] \)

 $-2 + 59 i$, which corresponds to adding a minus sign in the first term.
\item \( a \in [-61, -51] \text{ and } b \in [10, 14] \)

 $-58 + 11 i$, which corresponds to adding a minus sign in both terms.
\item \( a \in [-3, -1] \text{ and } b \in [-61, -58] \)

 $-2 - 59 i$, which corresponds to adding a minus sign in the second term.
\item \( a \in [-32, -25] \text{ and } b \in [27, 34] \)

 $-30 + 28 i$, which corresponds to just multiplying the real terms to get the real part of the solution and the coefficients in the complex terms to get the complex part.
\end{enumerate}

\textbf{General Comment:} You can treat $i$ as a variable and distribute. Just remember that $i^2=-1$, so you can continue to reduce after you distribute.
}
\litem{
Choose the \textbf{smallest} set of Real numbers that the number below belongs to.
\[ \sqrt{\frac{20736}{144}} \]
The solution is \( \text{Whole} \), which is option C.\begin{enumerate}[label=\Alph*.]
\item \( \text{Irrational} \)

These cannot be written as a fraction of Integers.
\item \( \text{Rational} \)

These are numbers that can be written as fraction of Integers (e.g., -2/3)
\item \( \text{Whole} \)

* This is the correct option!
\item \( \text{Not a Real number} \)

These are Nonreal Complex numbers \textbf{OR} things that are not numbers (e.g., dividing by 0).
\item \( \text{Integer} \)

These are the negative and positive counting numbers (..., -3, -2, -1, 0, 1, 2, 3, ...)
\end{enumerate}

\textbf{General Comment:} First, you \textbf{NEED} to simplify the expression. This question simplifies to $144$. 
 
 Be sure you look at the simplified fraction and not just the decimal expansion. Numbers such as 13, 17, and 19 provide \textbf{long but repeating/terminating decimal expansions!} 
 
 The only ways to *not* be a Real number are: dividing by 0 or taking the square root of a negative number. 
 
 Irrational numbers are more than just square root of 3: adding or subtracting values from square root of 3 is also irrational.
}
\litem{
Simplify the expression below into the form $a+bi$. Then, choose the intervals that $a$ and $b$ belong to.
\[ \frac{45 + 88 i}{-1 + 4 i} \]
The solution is \( 18.06  - 15.76 i \), which is option A.\begin{enumerate}[label=\Alph*.]
\item \( a \in [17, 18.5] \text{ and } b \in [-16.5, -14.5] \)

* $18.06  - 15.76 i$, which is the correct option.
\item \( a \in [-24.5, -23] \text{ and } b \in [4.5, 6] \)

 $-23.35  + 5.41 i$, which corresponds to forgetting to multiply the conjugate by the numerator and not computing the conjugate correctly.
\item \( a \in [306, 308] \text{ and } b \in [-16.5, -14.5] \)

 $307.00  - 15.76 i$, which corresponds to forgetting to multiply the conjugate by the numerator and using a plus instead of a minus in the denominator.
\item \( a \in [-45.5, -44] \text{ and } b \in [21.5, 22.5] \)

 $-45.00  + 22.00 i$, which corresponds to just dividing the first term by the first term and the second by the second.
\item \( a \in [17, 18.5] \text{ and } b \in [-269, -267.5] \)

 $18.06  - 268.00 i$, which corresponds to forgetting to multiply the conjugate by the numerator.
\end{enumerate}

\textbf{General Comment:} Multiply the numerator and denominator by the *conjugate* of the denominator, then simplify. For example, if we have $2+3i$, the conjugate is $2-3i$.
}
\litem{
Choose the \textbf{smallest} set of Complex numbers that the number below belongs to.
\[ \frac{-9}{17}+\sqrt{198} i \]
The solution is \( \text{Nonreal Complex} \), which is option B.\begin{enumerate}[label=\Alph*.]
\item \( \text{Irrational} \)

These cannot be written as a fraction of Integers. Remember: $\pi$ is not an Integer!
\item \( \text{Nonreal Complex} \)

* This is the correct option!
\item \( \text{Rational} \)

These are numbers that can be written as fraction of Integers (e.g., -2/3 + 5)
\item \( \text{Pure Imaginary} \)

This is a Complex number $(a+bi)$ that \textbf{only} has an imaginary part like $2i$.
\item \( \text{Not a Complex Number} \)

This is not a number. The only non-Complex number we know is dividing by 0 as this is not a number!
\end{enumerate}

\textbf{General Comment:} Be sure to simplify $i^2 = -1$. This may remove the imaginary portion for your number. If you are having trouble, you may want to look at the \textit{Subgroups of the Real Numbers} section.
}
\litem{
Simplify the expression below into the form $a+bi$. Then, choose the intervals that $a$ and $b$ belong to.
\[ (-2 - 7 i)(4 - 8 i) \]
The solution is \( -64 - 12 i \), which is option A.\begin{enumerate}[label=\Alph*.]
\item \( a \in [-72, -63] \text{ and } b \in [-14, -11] \)

* $-64 - 12 i$, which is the correct option.
\item \( a \in [-12, -1] \text{ and } b \in [47, 57] \)

 $-8 + 56 i$, which corresponds to just multiplying the real terms to get the real part of the solution and the coefficients in the complex terms to get the complex part.
\item \( a \in [45, 52] \text{ and } b \in [-45, -43] \)

 $48 - 44 i$, which corresponds to adding a minus sign in the second term.
\item \( a \in [45, 52] \text{ and } b \in [39, 47] \)

 $48 + 44 i$, which corresponds to adding a minus sign in the first term.
\item \( a \in [-72, -63] \text{ and } b \in [10, 14] \)

 $-64 + 12 i$, which corresponds to adding a minus sign in both terms.
\end{enumerate}

\textbf{General Comment:} You can treat $i$ as a variable and distribute. Just remember that $i^2=-1$, so you can continue to reduce after you distribute.
}
\litem{
Simplify the expression below into the form $a+bi$. Then, choose the intervals that $a$ and $b$ belong to.
\[ \frac{-36 - 55 i}{-3 + 6 i} \]
The solution is \( -4.93  + 8.47 i \), which is option C.\begin{enumerate}[label=\Alph*.]
\item \( a \in [-5.5, -4.5] \text{ and } b \in [380.5, 382] \)

 $-4.93  + 381.00 i$, which corresponds to forgetting to multiply the conjugate by the numerator.
\item \( a \in [9, 10.5] \text{ and } b \in [-1.5, 0] \)

 $9.73  - 1.13 i$, which corresponds to forgetting to multiply the conjugate by the numerator and not computing the conjugate correctly.
\item \( a \in [-5.5, -4.5] \text{ and } b \in [8, 10] \)

* $-4.93  + 8.47 i$, which is the correct option.
\item \( a \in [11.5, 14] \text{ and } b \in [-9.5, -8] \)

 $12.00  - 9.17 i$, which corresponds to just dividing the first term by the first term and the second by the second.
\item \( a \in [-222.5, -221] \text{ and } b \in [8, 10] \)

 $-222.00  + 8.47 i$, which corresponds to forgetting to multiply the conjugate by the numerator and using a plus instead of a minus in the denominator.
\end{enumerate}

\textbf{General Comment:} Multiply the numerator and denominator by the *conjugate* of the denominator, then simplify. For example, if we have $2+3i$, the conjugate is $2-3i$.
}
\end{enumerate}

\end{document}