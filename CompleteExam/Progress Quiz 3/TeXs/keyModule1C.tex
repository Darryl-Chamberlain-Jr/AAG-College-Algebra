\documentclass{extbook}[14pt]
\usepackage{multicol, enumerate, enumitem, hyperref, color, soul, setspace, parskip, fancyhdr, amssymb, amsthm, amsmath, bbm, latexsym, units, mathtools}
\everymath{\displaystyle}
\usepackage[headsep=0.5cm,headheight=0cm, left=1 in,right= 1 in,top= 1 in,bottom= 1 in]{geometry}
\usepackage{dashrule}  % Package to use the command below to create lines between items
\newcommand{\litem}[1]{\item #1

\rule{\textwidth}{0.4pt}}
\pagestyle{fancy}
\lhead{}
\chead{Answer Key for Progress Quiz 3 Version C}
\rhead{}
\lfoot{3148-2249}
\cfoot{}
\rfoot{Spring 2021}
\begin{document}
\textbf{This key should allow you to understand why you choose the option you did (beyond just getting a question right or wrong). \href{https://xronos.clas.ufl.edu/mac1105spring2020/courseDescriptionAndMisc/Exams/LearningFromResults}{More instructions on how to use this key can be found here}.}

\textbf{If you have a suggestion to make the keys better, \href{https://forms.gle/CZkbZmPbC9XALEE88}{please fill out the short survey here}.}

\textit{Note: This key is auto-generated and may contain issues and/or errors. The keys are reviewed after each exam to ensure grading is done accurately. If there are issues (like duplicate options), they are noted in the offline gradebook. The keys are a work-in-progress to give students as many resources to improve as possible.}

\rule{\textwidth}{0.4pt}

\begin{enumerate}\litem{
Choose the \textbf{smallest} set of Complex numbers that the number below belongs to.
\[ \sqrt{\frac{-3094}{0}} i+\sqrt{165}i \]

The solution is \( \text{Not a Complex Number} \), which is option C.\begin{enumerate}[label=\Alph*.]
\item \( \text{Rational} \)

These are numbers that can be written as fraction of Integers (e.g., -2/3 + 5)
\item \( \text{Nonreal Complex} \)

This is a Complex number $(a+bi)$ that is not Real (has $i$ as part of the number).
\item \( \text{Not a Complex Number} \)

* This is the correct option!
\item \( \text{Irrational} \)

These cannot be written as a fraction of Integers. Remember: $\pi$ is not an Integer!
\item \( \text{Pure Imaginary} \)

This is a Complex number $(a+bi)$ that \textbf{only} has an imaginary part like $2i$.
\end{enumerate}

\textbf{General Comment:} Be sure to simplify $i^2 = -1$. This may remove the imaginary portion for your number. If you are having trouble, you may want to look at the \textit{Subgroups of the Real Numbers} section.
}
\litem{
Simplify the expression below and choose the interval the simplification is contained within.
\[ 12 - 18 \div 11 * 15 - (8 * 4) \]

The solution is \( -44.545 \), which is option A.\begin{enumerate}[label=\Alph*.]
\item \( [-45.55, -41.55] \)

* -44.545, which is the correct option.
\item \( [42.89, 48.89] \)

 43.891, which corresponds to not distributing addition and subtraction correctly.
\item \( [-85.18, -75.18] \)

 -82.182, which corresponds to not distributing a negative correctly.
\item \( [-20.11, -16.11] \)

 -20.109, which corresponds to an Order of Operations error: not reading left-to-right for multiplication/division.
\item \( \text{None of the above} \)

 You may have gotten this by making an unanticipated error. If you got a value that is not any of the others, please let the coordinator know so they can help you figure out what happened.
\end{enumerate}

\textbf{General Comment:} While you may remember (or were taught) PEMDAS is done in order, it is actually done as P/E/MD/AS. When we are at MD or AS, we read left to right.
}
\litem{
Simplify the expression below and choose the interval the simplification is contained within.
\[ 6 - 16 \div 20 * 10 - (19 * 5) \]

The solution is \( -97.000 \), which is option C.\begin{enumerate}[label=\Alph*.]
\item \( [-109, -104] \)

 -105.000, which corresponds to not distributing a negative correctly.
\item \( [-90.08, -83.08] \)

 -89.080, which corresponds to an Order of Operations error: not reading left-to-right for multiplication/division.
\item \( [-102, -92] \)

* -97.000, which is the correct option.
\item \( [98.92, 107.92] \)

 100.920, which corresponds to not distributing addition and subtraction correctly.
\item \( \text{None of the above} \)

 You may have gotten this by making an unanticipated error. If you got a value that is not any of the others, please let the coordinator know so they can help you figure out what happened.
\end{enumerate}

\textbf{General Comment:} While you may remember (or were taught) PEMDAS is done in order, it is actually done as P/E/MD/AS. When we are at MD or AS, we read left to right.
}
\litem{
Choose the \textbf{smallest} set of Real numbers that the number below belongs to.
\[ \sqrt{\frac{43681}{361}} \]

The solution is \( \text{Whole} \), which is option E.\begin{enumerate}[label=\Alph*.]
\item \( \text{Integer} \)

These are the negative and positive counting numbers (..., -3, -2, -1, 0, 1, 2, 3, ...)
\item \( \text{Rational} \)

These are numbers that can be written as fraction of Integers (e.g., -2/3)
\item \( \text{Not a Real number} \)

These are Nonreal Complex numbers \textbf{OR} things that are not numbers (e.g., dividing by 0).
\item \( \text{Irrational} \)

These cannot be written as a fraction of Integers.
\item \( \text{Whole} \)

* This is the correct option!
\end{enumerate}

\textbf{General Comment:} First, you \textbf{NEED} to simplify the expression. This question simplifies to $209$. 
 
 Be sure you look at the simplified fraction and not just the decimal expansion. Numbers such as 13, 17, and 19 provide \textbf{long but repeating/terminating decimal expansions!} 
 
 The only ways to *not* be a Real number are: dividing by 0 or taking the square root of a negative number. 
 
 Irrational numbers are more than just square root of 3: adding or subtracting values from square root of 3 is also irrational.
}
\litem{
Simplify the expression below into the form $a+bi$. Then, choose the intervals that $a$ and $b$ belong to.
\[ \frac{27 + 66 i}{5 + 7 i} \]

The solution is \( 8.07  + 1.91 i \), which is option E.\begin{enumerate}[label=\Alph*.]
\item \( a \in [-5, -3.5] \text{ and } b \in [6.5, 7.5] \)

 $-4.42  + 7.01 i$, which corresponds to forgetting to multiply the conjugate by the numerator and not computing the conjugate correctly.
\item \( a \in [596.5, 597.5] \text{ and } b \in [1, 4] \)

 $597.00  + 1.91 i$, which corresponds to forgetting to multiply the conjugate by the numerator and using a plus instead of a minus in the denominator.
\item \( a \in [7, 9] \text{ and } b \in [140.5, 142] \)

 $8.07  + 141.00 i$, which corresponds to forgetting to multiply the conjugate by the numerator.
\item \( a \in [4.5, 6] \text{ and } b \in [8, 10.5] \)

 $5.40  + 9.43 i$, which corresponds to just dividing the first term by the first term and the second by the second.
\item \( a \in [7, 9] \text{ and } b \in [1, 4] \)

* $8.07  + 1.91 i$, which is the correct option.
\end{enumerate}

\textbf{General Comment:} Multiply the numerator and denominator by the *conjugate* of the denominator, then simplify. For example, if we have $2+3i$, the conjugate is $2-3i$.
}
\litem{
Choose the \textbf{smallest} set of Complex numbers that the number below belongs to.
\[ \sqrt{\frac{0}{196}}+\sqrt{6}i \]

The solution is \( \text{Pure Imaginary} \), which is option B.\begin{enumerate}[label=\Alph*.]
\item \( \text{Not a Complex Number} \)

This is not a number. The only non-Complex number we know is dividing by 0 as this is not a number!
\item \( \text{Pure Imaginary} \)

* This is the correct option!
\item \( \text{Nonreal Complex} \)

This is a Complex number $(a+bi)$ that is not Real (has $i$ as part of the number).
\item \( \text{Rational} \)

These are numbers that can be written as fraction of Integers (e.g., -2/3 + 5)
\item \( \text{Irrational} \)

These cannot be written as a fraction of Integers. Remember: $\pi$ is not an Integer!
\end{enumerate}

\textbf{General Comment:} Be sure to simplify $i^2 = -1$. This may remove the imaginary portion for your number. If you are having trouble, you may want to look at the \textit{Subgroups of the Real Numbers} section.
}
\litem{
Simplify the expression below into the form $a+bi$. Then, choose the intervals that $a$ and $b$ belong to.
\[ (-3 + 6 i)(-2 - 10 i) \]

The solution is \( 66 + 18 i \), which is option D.\begin{enumerate}[label=\Alph*.]
\item \( a \in [6, 9] \text{ and } b \in [-62, -57] \)

 $6 - 60 i$, which corresponds to just multiplying the real terms to get the real part of the solution and the coefficients in the complex terms to get the complex part.
\item \( a \in [57, 70] \text{ and } b \in [-22, -11] \)

 $66 - 18 i$, which corresponds to adding a minus sign in both terms.
\item \( a \in [-55, -53] \text{ and } b \in [-46, -38] \)

 $-54 - 42 i$, which corresponds to adding a minus sign in the second term.
\item \( a \in [57, 70] \text{ and } b \in [15, 27] \)

* $66 + 18 i$, which is the correct option.
\item \( a \in [-55, -53] \text{ and } b \in [34, 49] \)

 $-54 + 42 i$, which corresponds to adding a minus sign in the first term.
\end{enumerate}

\textbf{General Comment:} You can treat $i$ as a variable and distribute. Just remember that $i^2=-1$, so you can continue to reduce after you distribute.
}
\litem{
Simplify the expression below into the form $a+bi$. Then, choose the intervals that $a$ and $b$ belong to.
\[ \frac{-27 - 55 i}{4 - 7 i} \]

The solution is \( 4.26  - 6.29 i \), which is option D.\begin{enumerate}[label=\Alph*.]
\item \( a \in [-8, -7] \text{ and } b \in [-2, 0] \)

 $-7.58  - 0.48 i$, which corresponds to forgetting to multiply the conjugate by the numerator and not computing the conjugate correctly.
\item \( a \in [3.5, 5.5] \text{ and } b \in [-409.5, -407.5] \)

 $4.26  - 409.00 i$, which corresponds to forgetting to multiply the conjugate by the numerator.
\item \( a \in [-7.5, -5.5] \text{ and } b \in [7, 8.5] \)

 $-6.75  + 7.86 i$, which corresponds to just dividing the first term by the first term and the second by the second.
\item \( a \in [3.5, 5.5] \text{ and } b \in [-8, -5.5] \)

* $4.26  - 6.29 i$, which is the correct option.
\item \( a \in [276.5, 278.5] \text{ and } b \in [-8, -5.5] \)

 $277.00  - 6.29 i$, which corresponds to forgetting to multiply the conjugate by the numerator and using a plus instead of a minus in the denominator.
\end{enumerate}

\textbf{General Comment:} Multiply the numerator and denominator by the *conjugate* of the denominator, then simplify. For example, if we have $2+3i$, the conjugate is $2-3i$.
}
\litem{
Simplify the expression below into the form $a+bi$. Then, choose the intervals that $a$ and $b$ belong to.
\[ (8 - 7 i)(-10 + 9 i) \]

The solution is \( -17 + 142 i \), which is option D.\begin{enumerate}[label=\Alph*.]
\item \( a \in [-145, -141] \text{ and } b \in [-2, 0] \)

 $-143 - 2 i$, which corresponds to adding a minus sign in the second term.
\item \( a \in [-145, -141] \text{ and } b \in [1, 3] \)

 $-143 + 2 i$, which corresponds to adding a minus sign in the first term.
\item \( a \in [-20, -8] \text{ and } b \in [-145, -139] \)

 $-17 - 142 i$, which corresponds to adding a minus sign in both terms.
\item \( a \in [-20, -8] \text{ and } b \in [141, 146] \)

* $-17 + 142 i$, which is the correct option.
\item \( a \in [-85, -77] \text{ and } b \in [-71, -58] \)

 $-80 - 63 i$, which corresponds to just multiplying the real terms to get the real part of the solution and the coefficients in the complex terms to get the complex part.
\end{enumerate}

\textbf{General Comment:} You can treat $i$ as a variable and distribute. Just remember that $i^2=-1$, so you can continue to reduce after you distribute.
}
\litem{
Choose the \textbf{smallest} set of Real numbers that the number below belongs to.
\[ -\sqrt{\frac{50625}{81}} \]

The solution is \( \text{Integer} \), which is option C.\begin{enumerate}[label=\Alph*.]
\item \( \text{Whole} \)

These are the counting numbers with 0 (0, 1, 2, 3, ...)
\item \( \text{Not a Real number} \)

These are Nonreal Complex numbers \textbf{OR} things that are not numbers (e.g., dividing by 0).
\item \( \text{Integer} \)

* This is the correct option!
\item \( \text{Irrational} \)

These cannot be written as a fraction of Integers.
\item \( \text{Rational} \)

These are numbers that can be written as fraction of Integers (e.g., -2/3)
\end{enumerate}

\textbf{General Comment:} First, you \textbf{NEED} to simplify the expression. This question simplifies to $-225$. 
 
 Be sure you look at the simplified fraction and not just the decimal expansion. Numbers such as 13, 17, and 19 provide \textbf{long but repeating/terminating decimal expansions!} 
 
 The only ways to *not* be a Real number are: dividing by 0 or taking the square root of a negative number. 
 
 Irrational numbers are more than just square root of 3: adding or subtracting values from square root of 3 is also irrational.
}
\end{enumerate}

\end{document}