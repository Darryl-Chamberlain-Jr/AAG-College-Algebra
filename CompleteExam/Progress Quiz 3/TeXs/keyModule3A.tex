\documentclass{extbook}[14pt]
\usepackage{multicol, enumerate, enumitem, hyperref, color, soul, setspace, parskip, fancyhdr, amssymb, amsthm, amsmath, bbm, latexsym, units, mathtools}
\everymath{\displaystyle}
\usepackage[headsep=0.5cm,headheight=0cm, left=1 in,right= 1 in,top= 1 in,bottom= 1 in]{geometry}
\usepackage{dashrule}  % Package to use the command below to create lines between items
\newcommand{\litem}[1]{\item #1

\rule{\textwidth}{0.4pt}}
\pagestyle{fancy}
\lhead{}
\chead{Answer Key for Progress Quiz 3 Version A}
\rhead{}
\lfoot{}
\cfoot{}
\rfoot{Fall 2020}
\begin{document}
\textbf{This key should allow you to understand why you choose the option you did (beyond just getting a question right or wrong). \href{https://xronos.clas.ufl.edu/mac1105spring2020/courseDescriptionAndMisc/Exams/LearningFromResults}{More instructions on how to use this key can be found here}.}

\textbf{If you have a suggestion to make the keys better, \href{https://forms.gle/CZkbZmPbC9XALEE88}{please fill out the short survey here}.}

\textit{Note: This key is auto-generated and may contain issues and/or errors. The keys are reviewed after each exam to ensure grading is done accurately. If there are issues (like duplicate options), they are noted in the offline gradebook. The keys are a work-in-progress to give students as many resources to improve as possible.}

\rule{\textwidth}{0.4pt}

\begin{enumerate}\litem{
Solve the linear inequality below. Then, choose the constant and interval combination that describes the solution set.
\[ -4 + 9 x > 10 x \text{ or } 9 + 4 x < 5 x \]
The solution is \( (-\infty, -4.0) \text{ or } (9.0, \infty) \), which is option D.\begin{enumerate}[label=\Alph*.]
\item \( (-\infty, a] \cup [b, \infty), \text{ where } a \in [-9, -6] \text{ and } b \in [0, 7] \)

Corresponds to including the endpoints AND negating.
\item \( (-\infty, a] \cup [b, \infty), \text{ where } a \in [-5, 1] \text{ and } b \in [9, 11] \)

Corresponds to including the endpoints (when they should be excluded).
\item \( (-\infty, a) \cup (b, \infty), \text{ where } a \in [-10, -5] \text{ and } b \in [3, 8] \)

Corresponds to inverting the inequality and negating the solution.
\item \( (-\infty, a) \cup (b, \infty), \text{ where } a \in [-8, 0] \text{ and } b \in [9, 15] \)

 * Correct option.
\item \( (-\infty, \infty) \)

Corresponds to the variable canceling, which does not happen in this instance.
\end{enumerate}

\textbf{General Comment:} When multiplying or dividing by a negative, flip the sign.
}
\litem{
Solve the linear inequality below. Then, choose the constant and interval combination that describes the solution set.
\[ 5x -10 > 9x + 9 \]
The solution is \( (-\infty, -4.75) \), which is option A.\begin{enumerate}[label=\Alph*.]
\item \( (-\infty, a), \text{ where } a \in [-6.75, 2.25] \)

* $(-\infty, -4.75)$, which is the correct option.
\item \( (a, \infty), \text{ where } a \in [-6.75, -3.75] \)

 $(-4.75, \infty)$, which corresponds to switching the direction of the interval. You likely did this if you did not flip the inequality when dividing by a negative!
\item \( (-\infty, a), \text{ where } a \in [3.75, 5.75] \)

 $(-\infty, 4.75)$, which corresponds to negating the endpoint of the solution.
\item \( (a, \infty), \text{ where } a \in [-0.25, 7.75] \)

 $(4.75, \infty)$, which corresponds to switching the direction of the interval AND negating the endpoint. You likely did this if you did not flip the inequality when dividing by a negative as well as not moving values over to a side properly.
\item \( \text{None of the above}. \)

You may have chosen this if you thought the inequality did not match the ends of the intervals.
\end{enumerate}

\textbf{General Comment:} Remember that less/greater than or equal to includes the endpoint, while less/greater do not. Also, remember that you need to flip the inequality when you multiply or divide by a negative.
}
\litem{
Solve the linear inequality below. Then, choose the constant and interval combination that describes the solution set.
\[ 4x -4 \leq 8x -8 \]
The solution is \( [1.0, \infty) \), which is option A.\begin{enumerate}[label=\Alph*.]
\item \( [a, \infty), \text{ where } a \in [0.4, 1.9] \)

* $[1.0, \infty)$, which is the correct option.
\item \( (-\infty, a], \text{ where } a \in [-1.23, 0.7] \)

 $(-\infty, -1.0]$, which corresponds to switching the direction of the interval AND negating the endpoint. You likely did this if you did not flip the inequality when dividing by a negative as well as not moving values over to a side properly.
\item \( (-\infty, a], \text{ where } a \in [0.2, 1.16] \)

 $(-\infty, 1.0]$, which corresponds to switching the direction of the interval. You likely did this if you did not flip the inequality when dividing by a negative!
\item \( [a, \infty), \text{ where } a \in [-2, -0.8] \)

 $[-1.0, \infty)$, which corresponds to negating the endpoint of the solution.
\item \( \text{None of the above}. \)

You may have chosen this if you thought the inequality did not match the ends of the intervals.
\end{enumerate}

\textbf{General Comment:} Remember that less/greater than or equal to includes the endpoint, while less/greater do not. Also, remember that you need to flip the inequality when you multiply or divide by a negative.
}
\litem{
Solve the linear inequality below. Then, choose the constant and interval combination that describes the solution set.
\[ -3 + 9 x < \frac{86 x - 5}{9} \leq -6 + 6 x \]
The solution is \( (-4.40, -1.53] \), which is option A.\begin{enumerate}[label=\Alph*.]
\item \( (a, b], \text{ where } a \in [-4.4, 3.6] \text{ and } b \in [-3.53, 1.47] \)

* $(-4.40, -1.53]$, which is the correct option.
\item \( [a, b), \text{ where } a \in [-4.4, -2.4] \text{ and } b \in [-3.53, 1.47] \)

$[-4.40, -1.53)$, which corresponds to flipping the inequality.
\item \( (-\infty, a] \cup (b, \infty), \text{ where } a \in [-12.4, -0.4] \text{ and } b \in [-8.53, 1.47] \)

$(-\infty, -4.40] \cup (-1.53, \infty)$, which corresponds to displaying the and-inequality as an or-inequality AND flipping the inequality.
\item \( (-\infty, a) \cup [b, \infty), \text{ where } a \in [-5.4, -1.4] \text{ and } b \in [-4.53, 0.47] \)

$(-\infty, -4.40) \cup [-1.53, \infty)$, which corresponds to displaying the and-inequality as an or-inequality.
\item \( \text{None of the above.} \)


\end{enumerate}

\textbf{General Comment:} To solve, you will need to break up the compound inequality into two inequalities. Be sure to keep track of the inequality! It may be best to draw a number line and graph your solution.
}
\litem{
Solve the linear inequality below. Then, choose the constant and interval combination that describes the solution set.
\[ 4 + 6 x \leq \frac{52 x - 8}{6} < 4 + 8 x \]
The solution is \( [2.00, 8.00) \), which is option B.\begin{enumerate}[label=\Alph*.]
\item \( (a, b], \text{ where } a \in [0, 7] \text{ and } b \in [3, 10] \)

$(2.00, 8.00]$, which corresponds to flipping the inequality.
\item \( [a, b), \text{ where } a \in [1, 7] \text{ and } b \in [8, 9] \)

$[2.00, 8.00)$, which is the correct option.
\item \( (-\infty, a] \cup (b, \infty), \text{ where } a \in [0, 3] \text{ and } b \in [4, 11] \)

$(-\infty, 2.00] \cup (8.00, \infty)$, which corresponds to displaying the and-inequality as an or-inequality.
\item \( (-\infty, a) \cup [b, \infty), \text{ where } a \in [0, 5] \text{ and } b \in [6, 11] \)

$(-\infty, 2.00) \cup [8.00, \infty)$, which corresponds to displaying the and-inequality as an or-inequality AND flipping the inequality.
\item \( \text{None of the above.} \)


\end{enumerate}

\textbf{General Comment:} To solve, you will need to break up the compound inequality into two inequalities. Be sure to keep track of the inequality! It may be best to draw a number line and graph your solution.
}
\litem{
Using an interval or intervals, describe all the $x$-values within or including a distance of the given values.
\[ \text{ More than } 10 \text{ units from the number } -10. \]
The solution is \( (-\infty, -20) \cup (0, \infty) \), which is option D.\begin{enumerate}[label=\Alph*.]
\item \( (-\infty, -20] \cup [0, \infty) \)

This describes the values no less than 10 from -10
\item \( (-20, 0) \)

This describes the values less than 10 from -10
\item \( [-20, 0] \)

This describes the values no more than 10 from -10
\item \( (-\infty, -20) \cup (0, \infty) \)

This describes the values more than 10 from -10
\item \( \text{None of the above} \)

You likely thought the values in the interval were not correct.
\end{enumerate}

\textbf{General Comment:} When thinking about this language, it helps to draw a number line and try points.
}
\litem{
Solve the linear inequality below. Then, choose the constant and interval combination that describes the solution set.
\[ \frac{-5}{3} - \frac{9}{9} x > \frac{5}{6} x + \frac{8}{7} \]
The solution is \( (-\infty, -1.532) \), which is option A.\begin{enumerate}[label=\Alph*.]
\item \( (-\infty, a), \text{ where } a \in [-4.53, -0.53] \)

* $(-\infty, -1.532)$, which is the correct option.
\item \( (a, \infty), \text{ where } a \in [1.53, 2.53] \)

 $(1.532, \infty)$, which corresponds to switching the direction of the interval AND negating the endpoint. You likely did this if you did not flip the inequality when dividing by a negative as well as not moving values over to a side properly.
\item \( (-\infty, a), \text{ where } a \in [0.53, 3.53] \)

 $(-\infty, 1.532)$, which corresponds to negating the endpoint of the solution.
\item \( (a, \infty), \text{ where } a \in [-2.53, 0.47] \)

 $(-1.532, \infty)$, which corresponds to switching the direction of the interval. You likely did this if you did not flip the inequality when dividing by a negative!
\item \( \text{None of the above}. \)

You may have chosen this if you thought the inequality did not match the ends of the intervals.
\end{enumerate}

\textbf{General Comment:} Remember that less/greater than or equal to includes the endpoint, while less/greater do not. Also, remember that you need to flip the inequality when you multiply or divide by a negative.
}
\litem{
Using an interval or intervals, describe all the $x$-values within or including a distance of the given values.
\[ \text{ More than } 8 \text{ units from the number } 2. \]
The solution is \( \text{None of the above} \), which is option E.\begin{enumerate}[label=\Alph*.]
\item \( (6, 10) \)

This describes the values less than 2 from 8
\item \( (-\infty, 6) \cup (10, \infty) \)

This describes the values more than 2 from 8
\item \( [6, 10] \)

This describes the values no more than 2 from 8
\item \( (-\infty, 6] \cup [10, \infty) \)

This describes the values no less than 2 from 8
\item \( \text{None of the above} \)

Options A-D described the values [more/less than] 2 units from 8, which is the reverse of what the question asked.
\end{enumerate}

\textbf{General Comment:} When thinking about this language, it helps to draw a number line and try points.
}
\litem{
Solve the linear inequality below. Then, choose the constant and interval combination that describes the solution set.
\[ -9 + 5 x > 6 x \text{ or } -8 + 7 x < 9 x \]
The solution is \( (-\infty, -9.0) \text{ or } (-4.0, \infty) \), which is option C.\begin{enumerate}[label=\Alph*.]
\item \( (-\infty, a] \cup [b, \infty), \text{ where } a \in [-15, -2] \text{ and } b \in [-6, -2] \)

Corresponds to including the endpoints (when they should be excluded).
\item \( (-\infty, a) \cup (b, \infty), \text{ where } a \in [-1, 5] \text{ and } b \in [7, 10] \)

Corresponds to inverting the inequality and negating the solution.
\item \( (-\infty, a) \cup (b, \infty), \text{ where } a \in [-9, -6] \text{ and } b \in [-7, -1] \)

 * Correct option.
\item \( (-\infty, a] \cup [b, \infty), \text{ where } a \in [0, 5] \text{ and } b \in [7, 14] \)

Corresponds to including the endpoints AND negating.
\item \( (-\infty, \infty) \)

Corresponds to the variable canceling, which does not happen in this instance.
\end{enumerate}

\textbf{General Comment:} When multiplying or dividing by a negative, flip the sign.
}
\litem{
Solve the linear inequality below. Then, choose the constant and interval combination that describes the solution set.
\[ \frac{-9}{5} - \frac{7}{7} x > \frac{7}{8} x + \frac{9}{6} \]
The solution is \( (-\infty, -1.76) \), which is option A.\begin{enumerate}[label=\Alph*.]
\item \( (-\infty, a), \text{ where } a \in [-3.76, -0.76] \)

* $(-\infty, -1.76)$, which is the correct option.
\item \( (-\infty, a), \text{ where } a \in [-1.24, 7.76] \)

 $(-\infty, 1.76)$, which corresponds to negating the endpoint of the solution.
\item \( (a, \infty), \text{ where } a \in [-0.24, 3.76] \)

 $(1.76, \infty)$, which corresponds to switching the direction of the interval AND negating the endpoint. You likely did this if you did not flip the inequality when dividing by a negative as well as not moving values over to a side properly.
\item \( (a, \infty), \text{ where } a \in [-2.76, 1.24] \)

 $(-1.76, \infty)$, which corresponds to switching the direction of the interval. You likely did this if you did not flip the inequality when dividing by a negative!
\item \( \text{None of the above}. \)

You may have chosen this if you thought the inequality did not match the ends of the intervals.
\end{enumerate}

\textbf{General Comment:} Remember that less/greater than or equal to includes the endpoint, while less/greater do not. Also, remember that you need to flip the inequality when you multiply or divide by a negative.
}
\end{enumerate}

\end{document}