\documentclass{extbook}[14pt]
\usepackage{multicol, enumerate, enumitem, hyperref, color, soul, setspace, parskip, fancyhdr, amssymb, amsthm, amsmath, bbm, latexsym, units, mathtools}
\everymath{\displaystyle}
\usepackage[headsep=0.5cm,headheight=0cm, left=1 in,right= 1 in,top= 1 in,bottom= 1 in]{geometry}
\usepackage{dashrule}  % Package to use the command below to create lines between items
\newcommand{\litem}[1]{\item #1

\rule{\textwidth}{0.4pt}}
\pagestyle{fancy}
\lhead{}
\chead{Answer Key for Progress Quiz 3 Version B}
\rhead{}
\lfoot{}
\cfoot{}
\rfoot{Fall 2020}
\begin{document}
\textbf{This key should allow you to understand why you choose the option you did (beyond just getting a question right or wrong). \href{https://xronos.clas.ufl.edu/mac1105spring2020/courseDescriptionAndMisc/Exams/LearningFromResults}{More instructions on how to use this key can be found here}.}

\textbf{If you have a suggestion to make the keys better, \href{https://forms.gle/CZkbZmPbC9XALEE88}{please fill out the short survey here}.}

\textit{Note: This key is auto-generated and may contain issues and/or errors. The keys are reviewed after each exam to ensure grading is done accurately. If there are issues (like duplicate options), they are noted in the offline gradebook. The keys are a work-in-progress to give students as many resources to improve as possible.}

\rule{\textwidth}{0.4pt}

\begin{enumerate}\litem{
Solve the linear inequality below. Then, choose the constant and interval combination that describes the solution set.
\[ \frac{-5}{5} - \frac{7}{3} x > \frac{4}{7} x - \frac{8}{2} \]
The solution is \( (-\infty, 1.033) \), which is option C.\begin{enumerate}[label=\Alph*.]
\item \( (-\infty, a), \text{ where } a \in [-1.2, 0.4] \)

 $(-\infty, -1.033)$, which corresponds to negating the endpoint of the solution.
\item \( (a, \infty), \text{ where } a \in [-3.03, -0.03] \)

 $(-1.033, \infty)$, which corresponds to switching the direction of the interval AND negating the endpoint. You likely did this if you did not flip the inequality when dividing by a negative as well as not moving values over to a side properly.
\item \( (-\infty, a), \text{ where } a \in [-0.8, 2.8] \)

* $(-\infty, 1.033)$, which is the correct option.
\item \( (a, \infty), \text{ where } a \in [-0.97, 2.03] \)

 $(1.033, \infty)$, which corresponds to switching the direction of the interval. You likely did this if you did not flip the inequality when dividing by a negative!
\item \( \text{None of the above}. \)

You may have chosen this if you thought the inequality did not match the ends of the intervals.
\end{enumerate}

\textbf{General Comment:} Remember that less/greater than or equal to includes the endpoint, while less/greater do not. Also, remember that you need to flip the inequality when you multiply or divide by a negative.
}
\litem{
Solve the linear inequality below. Then, choose the constant and interval combination that describes the solution set.
\[ -4 - 8 x \leq \frac{-27 x + 9}{8} < 4 - 4 x \]
The solution is \( \text{None of the above.} \), which is option E.\begin{enumerate}[label=\Alph*.]
\item \( (-\infty, a) \cup [b, \infty), \text{ where } a \in [-0.1, 1.3] \text{ and } b \in [-4.6, -0.6] \)

$(-\infty, 1.11) \cup [-4.60, \infty)$, which corresponds to displaying the and-inequality as an or-inequality AND flipping the inequality AND getting negatives of the actual endpoints.
\item \( [a, b), \text{ where } a \in [-0.7, 3.7] \text{ and } b \in [-4.6, -0.6] \)

$[1.11, -4.60)$, which is the correct interval but negatives of the actual endpoints.
\item \( (a, b], \text{ where } a \in [0.11, 6.11] \text{ and } b \in [-4.6, -2.6] \)

$(1.11, -4.60]$, which corresponds to flipping the inequality and getting negatives of the actual endpoints.
\item \( (-\infty, a] \cup (b, \infty), \text{ where } a \in [0.11, 7.11] \text{ and } b \in [-4.6, -3.6] \)

$(-\infty, 1.11] \cup (-4.60, \infty)$, which corresponds to displaying the and-inequality as an or-inequality and getting negatives of the actual endpoints.
\item \( \text{None of the above.} \)

* This is correct as the answer should be $[-1.11, 4.60)$.
\end{enumerate}

\textbf{General Comment:} To solve, you will need to break up the compound inequality into two inequalities. Be sure to keep track of the inequality! It may be best to draw a number line and graph your solution.
}
\litem{
Solve the linear inequality below. Then, choose the constant and interval combination that describes the solution set.
\[ 6 + 5 x \leq \frac{26 x + 4}{4} < 9 + 6 x \]
The solution is \( [3.33, 16.00) \), which is option C.\begin{enumerate}[label=\Alph*.]
\item \( (-\infty, a) \cup [b, \infty), \text{ where } a \in [3.33, 4.33] \text{ and } b \in [12, 17] \)

$(-\infty, 3.33) \cup [16.00, \infty)$, which corresponds to displaying the and-inequality as an or-inequality AND flipping the inequality.
\item \( (a, b], \text{ where } a \in [1.33, 4.33] \text{ and } b \in [16, 17] \)

$(3.33, 16.00]$, which corresponds to flipping the inequality.
\item \( [a, b), \text{ where } a \in [2.33, 6.33] \text{ and } b \in [15, 21] \)

$[3.33, 16.00)$, which is the correct option.
\item \( (-\infty, a] \cup (b, \infty), \text{ where } a \in [2.33, 9.33] \text{ and } b \in [15, 18] \)

$(-\infty, 3.33] \cup (16.00, \infty)$, which corresponds to displaying the and-inequality as an or-inequality.
\item \( \text{None of the above.} \)


\end{enumerate}

\textbf{General Comment:} To solve, you will need to break up the compound inequality into two inequalities. Be sure to keep track of the inequality! It may be best to draw a number line and graph your solution.
}
\litem{
Solve the linear inequality below. Then, choose the constant and interval combination that describes the solution set.
\[ \frac{10}{7} - \frac{7}{4} x \leq \frac{-3}{3} x - \frac{8}{5} \]
The solution is \( [4.038, \infty) \), which is option A.\begin{enumerate}[label=\Alph*.]
\item \( [a, \infty), \text{ where } a \in [0.04, 8.04] \)

* $[4.038, \infty)$, which is the correct option.
\item \( (-\infty, a], \text{ where } a \in [1.04, 5.04] \)

 $(-\infty, 4.038]$, which corresponds to switching the direction of the interval. You likely did this if you did not flip the inequality when dividing by a negative!
\item \( [a, \infty), \text{ where } a \in [-5.04, -3.04] \)

 $[-4.038, \infty)$, which corresponds to negating the endpoint of the solution.
\item \( (-\infty, a], \text{ where } a \in [-12.04, -2.04] \)

 $(-\infty, -4.038]$, which corresponds to switching the direction of the interval AND negating the endpoint. You likely did this if you did not flip the inequality when dividing by a negative as well as not moving values over to a side properly.
\item \( \text{None of the above}. \)

You may have chosen this if you thought the inequality did not match the ends of the intervals.
\end{enumerate}

\textbf{General Comment:} Remember that less/greater than or equal to includes the endpoint, while less/greater do not. Also, remember that you need to flip the inequality when you multiply or divide by a negative.
}
\litem{
Solve the linear inequality below. Then, choose the constant and interval combination that describes the solution set.
\[ 6x -9 < 10x + 6 \]
The solution is \( (-3.75, \infty) \), which is option C.\begin{enumerate}[label=\Alph*.]
\item \( (-\infty, a), \text{ where } a \in [-7.75, -0.75] \)

 $(-\infty, -3.75)$, which corresponds to switching the direction of the interval. You likely did this if you did not flip the inequality when dividing by a negative!
\item \( (a, \infty), \text{ where } a \in [1.75, 6.75] \)

 $(3.75, \infty)$, which corresponds to negating the endpoint of the solution.
\item \( (a, \infty), \text{ where } a \in [-6.75, -0.75] \)

* $(-3.75, \infty)$, which is the correct option.
\item \( (-\infty, a), \text{ where } a \in [-2.25, 7.75] \)

 $(-\infty, 3.75)$, which corresponds to switching the direction of the interval AND negating the endpoint. You likely did this if you did not flip the inequality when dividing by a negative as well as not moving values over to a side properly.
\item \( \text{None of the above}. \)

You may have chosen this if you thought the inequality did not match the ends of the intervals.
\end{enumerate}

\textbf{General Comment:} Remember that less/greater than or equal to includes the endpoint, while less/greater do not. Also, remember that you need to flip the inequality when you multiply or divide by a negative.
}
\litem{
Solve the linear inequality below. Then, choose the constant and interval combination that describes the solution set.
\[ -6 + 9 x > 10 x \text{ or } 3 + 7 x < 10 x \]
The solution is \( (-\infty, -6.0) \text{ or } (1.0, \infty) \), which is option C.\begin{enumerate}[label=\Alph*.]
\item \( (-\infty, a] \cup [b, \infty), \text{ where } a \in [-2, 0] \text{ and } b \in [4, 8] \)

Corresponds to including the endpoints AND negating.
\item \( (-\infty, a] \cup [b, \infty), \text{ where } a \in [-7, -2] \text{ and } b \in [1, 3] \)

Corresponds to including the endpoints (when they should be excluded).
\item \( (-\infty, a) \cup (b, \infty), \text{ where } a \in [-8, -4] \text{ and } b \in [-5, 4] \)

 * Correct option.
\item \( (-\infty, a) \cup (b, \infty), \text{ where } a \in [-3, 2] \text{ and } b \in [4, 9] \)

Corresponds to inverting the inequality and negating the solution.
\item \( (-\infty, \infty) \)

Corresponds to the variable canceling, which does not happen in this instance.
\end{enumerate}

\textbf{General Comment:} When multiplying or dividing by a negative, flip the sign.
}
\litem{
Using an interval or intervals, describe all the $x$-values within or including a distance of the given values.
\[ \text{ No more than } 9 \text{ units from the number } -9. \]
The solution is \( [-18, 0] \), which is option C.\begin{enumerate}[label=\Alph*.]
\item \( (-\infty, -18] \cup [0, \infty) \)

This describes the values no less than 9 from -9
\item \( (-18, 0) \)

This describes the values less than 9 from -9
\item \( [-18, 0] \)

This describes the values no more than 9 from -9
\item \( (-\infty, -18) \cup (0, \infty) \)

This describes the values more than 9 from -9
\item \( \text{None of the above} \)

You likely thought the values in the interval were not correct.
\end{enumerate}

\textbf{General Comment:} When thinking about this language, it helps to draw a number line and try points.
}
\litem{
Solve the linear inequality below. Then, choose the constant and interval combination that describes the solution set.
\[ -6 + 7 x > 8 x \text{ or } 9 + 6 x < 7 x \]
The solution is \( (-\infty, -6.0) \text{ or } (9.0, \infty) \), which is option C.\begin{enumerate}[label=\Alph*.]
\item \( (-\infty, a] \cup [b, \infty), \text{ where } a \in [-9, -7] \text{ and } b \in [5.7, 7.9] \)

Corresponds to including the endpoints AND negating.
\item \( (-\infty, a] \cup [b, \infty), \text{ where } a \in [-7, -4] \text{ and } b \in [8.7, 9.6] \)

Corresponds to including the endpoints (when they should be excluded).
\item \( (-\infty, a) \cup (b, \infty), \text{ where } a \in [-8, -3] \text{ and } b \in [9, 13] \)

 * Correct option.
\item \( (-\infty, a) \cup (b, \infty), \text{ where } a \in [-9, -7] \text{ and } b \in [6, 7] \)

Corresponds to inverting the inequality and negating the solution.
\item \( (-\infty, \infty) \)

Corresponds to the variable canceling, which does not happen in this instance.
\end{enumerate}

\textbf{General Comment:} When multiplying or dividing by a negative, flip the sign.
}
\litem{
Using an interval or intervals, describe all the $x$-values within or including a distance of the given values.
\[ \text{ No more than } 5 \text{ units from the number } 3. \]
The solution is \( [-2, 8] \), which is option B.\begin{enumerate}[label=\Alph*.]
\item \( (-2, 8) \)

This describes the values less than 5 from 3
\item \( [-2, 8] \)

This describes the values no more than 5 from 3
\item \( (-\infty, -2] \cup [8, \infty) \)

This describes the values no less than 5 from 3
\item \( (-\infty, -2) \cup (8, \infty) \)

This describes the values more than 5 from 3
\item \( \text{None of the above} \)

You likely thought the values in the interval were not correct.
\end{enumerate}

\textbf{General Comment:} When thinking about this language, it helps to draw a number line and try points.
}
\litem{
Solve the linear inequality below. Then, choose the constant and interval combination that describes the solution set.
\[ -10x -7 \geq -6x + 5 \]
The solution is \( (-\infty, -3.0] \), which is option B.\begin{enumerate}[label=\Alph*.]
\item \( (-\infty, a], \text{ where } a \in [2, 5] \)

 $(-\infty, 3.0]$, which corresponds to negating the endpoint of the solution.
\item \( (-\infty, a], \text{ where } a \in [-3, -1] \)

* $(-\infty, -3.0]$, which is the correct option.
\item \( [a, \infty), \text{ where } a \in [-4, 0] \)

 $[-3.0, \infty)$, which corresponds to switching the direction of the interval. You likely did this if you did not flip the inequality when dividing by a negative!
\item \( [a, \infty), \text{ where } a \in [1, 4] \)

 $[3.0, \infty)$, which corresponds to switching the direction of the interval AND negating the endpoint. You likely did this if you did not flip the inequality when dividing by a negative as well as not moving values over to a side properly.
\item \( \text{None of the above}. \)

You may have chosen this if you thought the inequality did not match the ends of the intervals.
\end{enumerate}

\textbf{General Comment:} Remember that less/greater than or equal to includes the endpoint, while less/greater do not. Also, remember that you need to flip the inequality when you multiply or divide by a negative.
}
\end{enumerate}

\end{document}