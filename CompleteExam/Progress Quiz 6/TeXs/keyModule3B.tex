\documentclass{extbook}[14pt]
\usepackage{multicol, enumerate, enumitem, hyperref, color, soul, setspace, parskip, fancyhdr, amssymb, amsthm, amsmath, latexsym, units, mathtools}
\everymath{\displaystyle}
\usepackage[headsep=0.5cm,headheight=0cm, left=1 in,right= 1 in,top= 1 in,bottom= 1 in]{geometry}
\usepackage{dashrule}  % Package to use the command below to create lines between items
\newcommand{\litem}[1]{\item #1

\rule{\textwidth}{0.4pt}}
\pagestyle{fancy}
\lhead{}
\chead{Answer Key for Progress Quiz 6 Version B}
\rhead{}
\lfoot{1430-1829}
\cfoot{}
\rfoot{test}
\begin{document}
\textbf{This key should allow you to understand why you choose the option you did (beyond just getting a question right or wrong). \href{https://xronos.clas.ufl.edu/mac1105spring2020/courseDescriptionAndMisc/Exams/LearningFromResults}{More instructions on how to use this key can be found here}.}

\textbf{If you have a suggestion to make the keys better, \href{https://forms.gle/CZkbZmPbC9XALEE88}{please fill out the short survey here}.}

\textit{Note: This key is auto-generated and may contain issues and/or errors. The keys are reviewed after each exam to ensure grading is done accurately. If there are issues (like duplicate options), they are noted in the offline gradebook. The keys are a work-in-progress to give students as many resources to improve as possible.}

\rule{\textwidth}{0.4pt}

\begin{enumerate}\litem{
Solve the linear inequality below. Then, choose the constant and interval combination that describes the solution set.
\[ \frac{-4}{3} - \frac{6}{4} x > \frac{7}{9} x + \frac{10}{6} \]The solution is \( (-\infty, -1.317) \), which is option D.\begin{enumerate}[label=\Alph*.]
\item \( (a, \infty), \text{ where } a \in [-3, 0] \)

 $(-1.317, \infty)$, which corresponds to switching the direction of the interval. You likely did this if you did not flip the inequality when dividing by a negative!
\item \( (-\infty, a), \text{ where } a \in [0, 3] \)

 $(-\infty, 1.317)$, which corresponds to negating the endpoint of the solution.
\item \( (a, \infty), \text{ where } a \in [0.75, 2.25] \)

 $(1.317, \infty)$, which corresponds to switching the direction of the interval AND negating the endpoint. You likely did this if you did not flip the inequality when dividing by a negative as well as not moving values over to a side properly.
\item \( (-\infty, a), \text{ where } a \in [-3, 0.75] \)

* $(-\infty, -1.317)$, which is the correct option.
\item \( \text{None of the above}. \)

You may have chosen this if you thought the inequality did not match the ends of the intervals.
\end{enumerate}

\textbf{General Comment:} Remember that less/greater than or equal to includes the endpoint, while less/greater do not. Also, remember that you need to flip the inequality when you multiply or divide by a negative.
}
\litem{
Solve the linear inequality below. Then, choose the constant and interval combination that describes the solution set.
\[ -8 + 9 x > 10 x \text{ or } -3 + 3 x < 4 x \]The solution is \( (-\infty, -8.0) \text{ or } (-3.0, \infty) \), which is option C.\begin{enumerate}[label=\Alph*.]
\item \( (-\infty, a] \cup [b, \infty), \text{ where } a \in [1.5, 10.5] \text{ and } b \in [7.5, 12.75] \)

Corresponds to including the endpoints AND negating.
\item \( (-\infty, a] \cup [b, \infty), \text{ where } a \in [-12.75, -3.75] \text{ and } b \in [-7.5, -2.25] \)

Corresponds to including the endpoints (when they should be excluded).
\item \( (-\infty, a) \cup (b, \infty), \text{ where } a \in [-12.75, -5.25] \text{ and } b \in [-7.5, -1.5] \)

 * Correct option.
\item \( (-\infty, a) \cup (b, \infty), \text{ where } a \in [0.75, 7.5] \text{ and } b \in [5.25, 13.5] \)

Corresponds to inverting the inequality and negating the solution.
\item \( (-\infty, \infty) \)

Corresponds to the variable canceling, which does not happen in this instance.
\end{enumerate}

\textbf{General Comment:} When multiplying or dividing by a negative, flip the sign.
}
\litem{
Solve the linear inequality below. Then, choose the constant and interval combination that describes the solution set.
\[ -7 + 6 x < \frac{51 x - 4}{8} \leq 7 + 3 x \]The solution is \( \text{None of the above.} \), which is option E.\begin{enumerate}[label=\Alph*.]
\item \( (a, b], \text{ where } a \in [12, 21] \text{ and } b \in [-6.75, 1.5] \)

$(17.33, -2.22]$, which is the correct interval but negatives of the actual endpoints.
\item \( (-\infty, a) \cup [b, \infty), \text{ where } a \in [15.75, 21] \text{ and } b \in [-3, -1.12] \)

$(-\infty, 17.33) \cup [-2.22, \infty)$, which corresponds to displaying the and-inequality as an or-inequality and getting negatives of the actual endpoints.
\item \( (-\infty, a] \cup (b, \infty), \text{ where } a \in [12.75, 18.75] \text{ and } b \in [-3.75, -1.5] \)

$(-\infty, 17.33] \cup (-2.22, \infty)$, which corresponds to displaying the and-inequality as an or-inequality AND flipping the inequality AND getting negatives of the actual endpoints.
\item \( [a, b), \text{ where } a \in [15.75, 20.25] \text{ and } b \in [-3.75, 0] \)

$[17.33, -2.22)$, which corresponds to flipping the inequality and getting negatives of the actual endpoints.
\item \( \text{None of the above.} \)

* This is correct as the answer should be $(-17.33, 2.22]$.
\end{enumerate}

\textbf{General Comment:} To solve, you will need to break up the compound inequality into two inequalities. Be sure to keep track of the inequality! It may be best to draw a number line and graph your solution.
}
\litem{
Using an interval or intervals, describe all the $x$-values within or including a distance of the given values.
\[ \text{ More than } 4 \text{ units from the number } 2. \]The solution is \( (-\infty, -2) \cup (6, \infty) \), which is option D.\begin{enumerate}[label=\Alph*.]
\item \( [-2, 6] \)

This describes the values no more than 4 from 2
\item \( (-\infty, -2] \cup [6, \infty) \)

This describes the values no less than 4 from 2
\item \( (-2, 6) \)

This describes the values less than 4 from 2
\item \( (-\infty, -2) \cup (6, \infty) \)

This describes the values more than 4 from 2
\item \( \text{None of the above} \)

You likely thought the values in the interval were not correct.
\end{enumerate}

\textbf{General Comment:} When thinking about this language, it helps to draw a number line and try points.
}
\litem{
Solve the linear inequality below. Then, choose the constant and interval combination that describes the solution set.
\[ -4x -3 < 9x + 6 \]The solution is \( (-0.692, \infty) \), which is option B.\begin{enumerate}[label=\Alph*.]
\item \( (-\infty, a), \text{ where } a \in [0.09, 1.14] \)

 $(-\infty, 0.692)$, which corresponds to switching the direction of the interval AND negating the endpoint. You likely did this if you did not flip the inequality when dividing by a negative as well as not moving values over to a side properly.
\item \( (a, \infty), \text{ where } a \in [-6.69, 0.31] \)

* $(-0.692, \infty)$, which is the correct option.
\item \( (-\infty, a), \text{ where } a \in [-2.02, -0.57] \)

 $(-\infty, -0.692)$, which corresponds to switching the direction of the interval. You likely did this if you did not flip the inequality when dividing by a negative!
\item \( (a, \infty), \text{ where } a \in [-0.31, 1.69] \)

 $(0.692, \infty)$, which corresponds to negating the endpoint of the solution.
\item \( \text{None of the above}. \)

You may have chosen this if you thought the inequality did not match the ends of the intervals.
\end{enumerate}

\textbf{General Comment:} Remember that less/greater than or equal to includes the endpoint, while less/greater do not. Also, remember that you need to flip the inequality when you multiply or divide by a negative.
}
\litem{
Solve the linear inequality below. Then, choose the constant and interval combination that describes the solution set.
\[ -9 + 5 x > 7 x \text{ or } 4 + 4 x < 7 x \]The solution is \( (-\infty, -4.5) \text{ or } (1.333, \infty) \), which is option A.\begin{enumerate}[label=\Alph*.]
\item \( (-\infty, a) \cup (b, \infty), \text{ where } a \in [-5.25, -3.75] \text{ and } b \in [-1.5, 2.25] \)

 * Correct option.
\item \( (-\infty, a] \cup [b, \infty), \text{ where } a \in [-3.75, 3] \text{ and } b \in [3.75, 6] \)

Corresponds to including the endpoints AND negating.
\item \( (-\infty, a) \cup (b, \infty), \text{ where } a \in [-3, 2.25] \text{ and } b \in [2.25, 9] \)

Corresponds to inverting the inequality and negating the solution.
\item \( (-\infty, a] \cup [b, \infty), \text{ where } a \in [-7.5, -3] \text{ and } b \in [0, 3.75] \)

Corresponds to including the endpoints (when they should be excluded).
\item \( (-\infty, \infty) \)

Corresponds to the variable canceling, which does not happen in this instance.
\end{enumerate}

\textbf{General Comment:} When multiplying or dividing by a negative, flip the sign.
}
\litem{
Using an interval or intervals, describe all the $x$-values within or including a distance of the given values.
\[ \text{ No more than } 6 \text{ units from the number } 4. \]The solution is \( [-2, 10] \), which is option D.\begin{enumerate}[label=\Alph*.]
\item \( (-2, 10) \)

This describes the values less than 6 from 4
\item \( (-\infty, -2) \cup (10, \infty) \)

This describes the values more than 6 from 4
\item \( (-\infty, -2] \cup [10, \infty) \)

This describes the values no less than 6 from 4
\item \( [-2, 10] \)

This describes the values no more than 6 from 4
\item \( \text{None of the above} \)

You likely thought the values in the interval were not correct.
\end{enumerate}

\textbf{General Comment:} When thinking about this language, it helps to draw a number line and try points.
}
\litem{
Solve the linear inequality below. Then, choose the constant and interval combination that describes the solution set.
\[ -9 + 3 x \leq \frac{25 x + 8}{7} < -9 - 4 x \]The solution is \( [-17.75, -1.34) \), which is option A.\begin{enumerate}[label=\Alph*.]
\item \( [a, b), \text{ where } a \in [-18.75, -13.5] \text{ and } b \in [-3, 0] \)

$[-17.75, -1.34)$, which is the correct option.
\item \( (-\infty, a] \cup (b, \infty), \text{ where } a \in [-21.75, -8.25] \text{ and } b \in [-3, 0] \)

$(-\infty, -17.75] \cup (-1.34, \infty)$, which corresponds to displaying the and-inequality as an or-inequality.
\item \( (a, b], \text{ where } a \in [-18.75, -15.75] \text{ and } b \in [-4.5, 0] \)

$(-17.75, -1.34]$, which corresponds to flipping the inequality.
\item \( (-\infty, a) \cup [b, \infty), \text{ where } a \in [-18.75, -15] \text{ and } b \in [-5.25, -0.75] \)

$(-\infty, -17.75) \cup [-1.34, \infty)$, which corresponds to displaying the and-inequality as an or-inequality AND flipping the inequality.
\item \( \text{None of the above.} \)


\end{enumerate}

\textbf{General Comment:} To solve, you will need to break up the compound inequality into two inequalities. Be sure to keep track of the inequality! It may be best to draw a number line and graph your solution.
}
\litem{
Solve the linear inequality below. Then, choose the constant and interval combination that describes the solution set.
\[ 3x + 9 \leq 6x + 8 \]The solution is \( [0.333, \infty) \), which is option D.\begin{enumerate}[label=\Alph*.]
\item \( (-\infty, a], \text{ where } a \in [-0.64, -0.27] \)

 $(-\infty, -0.333]$, which corresponds to switching the direction of the interval AND negating the endpoint. You likely did this if you did not flip the inequality when dividing by a negative as well as not moving values over to a side properly.
\item \( (-\infty, a], \text{ where } a \in [0.2, 1.37] \)

 $(-\infty, 0.333]$, which corresponds to switching the direction of the interval. You likely did this if you did not flip the inequality when dividing by a negative!
\item \( [a, \infty), \text{ where } a \in [-3.7, 0.3] \)

 $[-0.333, \infty)$, which corresponds to negating the endpoint of the solution.
\item \( [a, \infty), \text{ where } a \in [-0.3, 4.9] \)

* $[0.333, \infty)$, which is the correct option.
\item \( \text{None of the above}. \)

You may have chosen this if you thought the inequality did not match the ends of the intervals.
\end{enumerate}

\textbf{General Comment:} Remember that less/greater than or equal to includes the endpoint, while less/greater do not. Also, remember that you need to flip the inequality when you multiply or divide by a negative.
}
\litem{
Solve the linear inequality below. Then, choose the constant and interval combination that describes the solution set.
\[ \frac{-6}{2} + \frac{7}{8} x \geq \frac{9}{3} x + \frac{10}{7} \]The solution is \( (-\infty, -2.084] \), which is option D.\begin{enumerate}[label=\Alph*.]
\item \( (-\infty, a], \text{ where } a \in [1.5, 3.75] \)

 $(-\infty, 2.084]$, which corresponds to negating the endpoint of the solution.
\item \( [a, \infty), \text{ where } a \in [-0.75, 3.75] \)

 $[2.084, \infty)$, which corresponds to switching the direction of the interval AND negating the endpoint. You likely did this if you did not flip the inequality when dividing by a negative as well as not moving values over to a side properly.
\item \( [a, \infty), \text{ where } a \in [-2.25, 0.75] \)

 $[-2.084, \infty)$, which corresponds to switching the direction of the interval. You likely did this if you did not flip the inequality when dividing by a negative!
\item \( (-\infty, a], \text{ where } a \in [-3, -0.75] \)

* $(-\infty, -2.084]$, which is the correct option.
\item \( \text{None of the above}. \)

You may have chosen this if you thought the inequality did not match the ends of the intervals.
\end{enumerate}

\textbf{General Comment:} Remember that less/greater than or equal to includes the endpoint, while less/greater do not. Also, remember that you need to flip the inequality when you multiply or divide by a negative.
}
\end{enumerate}

\end{document}