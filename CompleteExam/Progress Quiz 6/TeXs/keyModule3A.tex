\documentclass{extbook}[14pt]
\usepackage{multicol, enumerate, enumitem, hyperref, color, soul, setspace, parskip, fancyhdr, amssymb, amsthm, amsmath, latexsym, units, mathtools}
\everymath{\displaystyle}
\usepackage[headsep=0.5cm,headheight=0cm, left=1 in,right= 1 in,top= 1 in,bottom= 1 in]{geometry}
\usepackage{dashrule}  % Package to use the command below to create lines between items
\newcommand{\litem}[1]{\item #1

\rule{\textwidth}{0.4pt}}
\pagestyle{fancy}
\lhead{}
\chead{Answer Key for Progress Quiz 6 Version A}
\rhead{}
\lfoot{1430-1829}
\cfoot{}
\rfoot{test}
\begin{document}
\textbf{This key should allow you to understand why you choose the option you did (beyond just getting a question right or wrong). \href{https://xronos.clas.ufl.edu/mac1105spring2020/courseDescriptionAndMisc/Exams/LearningFromResults}{More instructions on how to use this key can be found here}.}

\textbf{If you have a suggestion to make the keys better, \href{https://forms.gle/CZkbZmPbC9XALEE88}{please fill out the short survey here}.}

\textit{Note: This key is auto-generated and may contain issues and/or errors. The keys are reviewed after each exam to ensure grading is done accurately. If there are issues (like duplicate options), they are noted in the offline gradebook. The keys are a work-in-progress to give students as many resources to improve as possible.}

\rule{\textwidth}{0.4pt}

\begin{enumerate}\litem{
Solve the linear inequality below. Then, choose the constant and interval combination that describes the solution set.
\[ \frac{6}{8} + \frac{4}{5} x \geq \frac{9}{6} x - \frac{8}{9} \]The solution is \( (-\infty, 2.341] \), which is option C.\begin{enumerate}[label=\Alph*.]
\item \( [a, \infty), \text{ where } a \in [-1.5, 8.25] \)

 $[2.341, \infty)$, which corresponds to switching the direction of the interval. You likely did this if you did not flip the inequality when dividing by a negative!
\item \( (-\infty, a], \text{ where } a \in [-6, -0.75] \)

 $(-\infty, -2.341]$, which corresponds to negating the endpoint of the solution.
\item \( (-\infty, a], \text{ where } a \in [0.75, 6] \)

* $(-\infty, 2.341]$, which is the correct option.
\item \( [a, \infty), \text{ where } a \in [-4.5, 0.75] \)

 $[-2.341, \infty)$, which corresponds to switching the direction of the interval AND negating the endpoint. You likely did this if you did not flip the inequality when dividing by a negative as well as not moving values over to a side properly.
\item \( \text{None of the above}. \)

You may have chosen this if you thought the inequality did not match the ends of the intervals.
\end{enumerate}

\textbf{General Comment:} Remember that less/greater than or equal to includes the endpoint, while less/greater do not. Also, remember that you need to flip the inequality when you multiply or divide by a negative.
}
\litem{
Solve the linear inequality below. Then, choose the constant and interval combination that describes the solution set.
\[ -8 + 4 x > 5 x \text{ or } 3 + 6 x < 8 x \]The solution is \( (-\infty, -8.0) \text{ or } (1.5, \infty) \), which is option D.\begin{enumerate}[label=\Alph*.]
\item \( (-\infty, a] \cup [b, \infty), \text{ where } a \in [-11.25, -3.75] \text{ and } b \in [0.75, 6.75] \)

Corresponds to including the endpoints (when they should be excluded).
\item \( (-\infty, a] \cup [b, \infty), \text{ where } a \in [-4.5, 2.25] \text{ and } b \in [2.25, 9] \)

Corresponds to including the endpoints AND negating.
\item \( (-\infty, a) \cup (b, \infty), \text{ where } a \in [-3.75, 3] \text{ and } b \in [6.75, 9.75] \)

Corresponds to inverting the inequality and negating the solution.
\item \( (-\infty, a) \cup (b, \infty), \text{ where } a \in [-12, -3.75] \text{ and } b \in [-1.5, 7.5] \)

 * Correct option.
\item \( (-\infty, \infty) \)

Corresponds to the variable canceling, which does not happen in this instance.
\end{enumerate}

\textbf{General Comment:} When multiplying or dividing by a negative, flip the sign.
}
\litem{
Solve the linear inequality below. Then, choose the constant and interval combination that describes the solution set.
\[ 9 - 9 x < \frac{-49 x + 9}{9} \leq 6 - 6 x \]The solution is \( \text{None of the above.} \), which is option E.\begin{enumerate}[label=\Alph*.]
\item \( (-\infty, a) \cup [b, \infty), \text{ where } a \in [-3.75, -1.5] \text{ and } b \in [-11.25, -6.75] \)

$(-\infty, -2.25) \cup [-9.00, \infty)$, which corresponds to displaying the and-inequality as an or-inequality and getting negatives of the actual endpoints.
\item \( (a, b], \text{ where } a \in [-7.5, 0] \text{ and } b \in [-10.5, -7.5] \)

$(-2.25, -9.00]$, which is the correct interval but negatives of the actual endpoints.
\item \( (-\infty, a] \cup (b, \infty), \text{ where } a \in [-7.5, 1.5] \text{ and } b \in [-15.75, -4.5] \)

$(-\infty, -2.25] \cup (-9.00, \infty)$, which corresponds to displaying the and-inequality as an or-inequality AND flipping the inequality AND getting negatives of the actual endpoints.
\item \( [a, b), \text{ where } a \in [-3.75, 0.75] \text{ and } b \in [-9.75, -3] \)

$[-2.25, -9.00)$, which corresponds to flipping the inequality and getting negatives of the actual endpoints.
\item \( \text{None of the above.} \)

* This is correct as the answer should be $(2.25, 9.00]$.
\end{enumerate}

\textbf{General Comment:} To solve, you will need to break up the compound inequality into two inequalities. Be sure to keep track of the inequality! It may be best to draw a number line and graph your solution.
}
\litem{
Using an interval or intervals, describe all the $x$-values within or including a distance of the given values.
\[ \text{ More than } 10 \text{ units from the number } -4. \]The solution is \( (-\infty, -14) \cup (6, \infty) \), which is option A.\begin{enumerate}[label=\Alph*.]
\item \( (-\infty, -14) \cup (6, \infty) \)

This describes the values more than 10 from -4
\item \( [-14, 6] \)

This describes the values no more than 10 from -4
\item \( (-14, 6) \)

This describes the values less than 10 from -4
\item \( (-\infty, -14] \cup [6, \infty) \)

This describes the values no less than 10 from -4
\item \( \text{None of the above} \)

You likely thought the values in the interval were not correct.
\end{enumerate}

\textbf{General Comment:} When thinking about this language, it helps to draw a number line and try points.
}
\litem{
Solve the linear inequality below. Then, choose the constant and interval combination that describes the solution set.
\[ 8x -7 < 10x + 7 \]The solution is \( (-7.0, \infty) \), which is option C.\begin{enumerate}[label=\Alph*.]
\item \( (a, \infty), \text{ where } a \in [7, 11] \)

 $(7.0, \infty)$, which corresponds to negating the endpoint of the solution.
\item \( (-\infty, a), \text{ where } a \in [5, 8] \)

 $(-\infty, 7.0)$, which corresponds to switching the direction of the interval AND negating the endpoint. You likely did this if you did not flip the inequality when dividing by a negative as well as not moving values over to a side properly.
\item \( (a, \infty), \text{ where } a \in [-10, -6] \)

* $(-7.0, \infty)$, which is the correct option.
\item \( (-\infty, a), \text{ where } a \in [-11, -1] \)

 $(-\infty, -7.0)$, which corresponds to switching the direction of the interval. You likely did this if you did not flip the inequality when dividing by a negative!
\item \( \text{None of the above}. \)

You may have chosen this if you thought the inequality did not match the ends of the intervals.
\end{enumerate}

\textbf{General Comment:} Remember that less/greater than or equal to includes the endpoint, while less/greater do not. Also, remember that you need to flip the inequality when you multiply or divide by a negative.
}
\litem{
Solve the linear inequality below. Then, choose the constant and interval combination that describes the solution set.
\[ -5 + 4 x > 7 x \text{ or } 8 + 3 x < 4 x \]The solution is \( (-\infty, -1.667) \text{ or } (8.0, \infty) \), which is option C.\begin{enumerate}[label=\Alph*.]
\item \( (-\infty, a] \cup [b, \infty), \text{ where } a \in [-11.25, -7.5] \text{ and } b \in [-0.75, 5.25] \)

Corresponds to including the endpoints AND negating.
\item \( (-\infty, a] \cup [b, \infty), \text{ where } a \in [-3.75, 0.75] \text{ and } b \in [4.5, 9] \)

Corresponds to including the endpoints (when they should be excluded).
\item \( (-\infty, a) \cup (b, \infty), \text{ where } a \in [-4.5, 1.5] \text{ and } b \in [6.75, 15] \)

 * Correct option.
\item \( (-\infty, a) \cup (b, \infty), \text{ where } a \in [-11.25, -3] \text{ and } b \in [0.75, 4.5] \)

Corresponds to inverting the inequality and negating the solution.
\item \( (-\infty, \infty) \)

Corresponds to the variable canceling, which does not happen in this instance.
\end{enumerate}

\textbf{General Comment:} When multiplying or dividing by a negative, flip the sign.
}
\litem{
Using an interval or intervals, describe all the $x$-values within or including a distance of the given values.
\[ \text{ More than } 8 \text{ units from the number } 5. \]The solution is \( (-\infty, -3) \cup (13, \infty) \), which is option B.\begin{enumerate}[label=\Alph*.]
\item \( [-3, 13] \)

This describes the values no more than 8 from 5
\item \( (-\infty, -3) \cup (13, \infty) \)

This describes the values more than 8 from 5
\item \( (-3, 13) \)

This describes the values less than 8 from 5
\item \( (-\infty, -3] \cup [13, \infty) \)

This describes the values no less than 8 from 5
\item \( \text{None of the above} \)

You likely thought the values in the interval were not correct.
\end{enumerate}

\textbf{General Comment:} When thinking about this language, it helps to draw a number line and try points.
}
\litem{
Solve the linear inequality below. Then, choose the constant and interval combination that describes the solution set.
\[ -3 - 3 x < \frac{-7 x + 9}{4} \leq 9 - 3 x \]The solution is \( (-4.20, 5.40] \), which is option C.\begin{enumerate}[label=\Alph*.]
\item \( [a, b), \text{ where } a \in [-7.5, -0.75] \text{ and } b \in [2.25, 9] \)

$[-4.20, 5.40)$, which corresponds to flipping the inequality.
\item \( (-\infty, a] \cup (b, \infty), \text{ where } a \in [-5.25, 2.25] \text{ and } b \in [3, 10.5] \)

$(-\infty, -4.20] \cup (5.40, \infty)$, which corresponds to displaying the and-inequality as an or-inequality AND flipping the inequality.
\item \( (a, b], \text{ where } a \in [-5.25, -1.5] \text{ and } b \in [5.25, 6] \)

* $(-4.20, 5.40]$, which is the correct option.
\item \( (-\infty, a) \cup [b, \infty), \text{ where } a \in [-7.5, 0.75] \text{ and } b \in [2.25, 6] \)

$(-\infty, -4.20) \cup [5.40, \infty)$, which corresponds to displaying the and-inequality as an or-inequality.
\item \( \text{None of the above.} \)


\end{enumerate}

\textbf{General Comment:} To solve, you will need to break up the compound inequality into two inequalities. Be sure to keep track of the inequality! It may be best to draw a number line and graph your solution.
}
\litem{
Solve the linear inequality below. Then, choose the constant and interval combination that describes the solution set.
\[ -9x -7 > -6x + 8 \]The solution is \( (-\infty, -5.0) \), which is option B.\begin{enumerate}[label=\Alph*.]
\item \( (a, \infty), \text{ where } a \in [-7, 0] \)

 $(-5.0, \infty)$, which corresponds to switching the direction of the interval. You likely did this if you did not flip the inequality when dividing by a negative!
\item \( (-\infty, a), \text{ where } a \in [-6, 1] \)

* $(-\infty, -5.0)$, which is the correct option.
\item \( (-\infty, a), \text{ where } a \in [4, 8] \)

 $(-\infty, 5.0)$, which corresponds to negating the endpoint of the solution.
\item \( (a, \infty), \text{ where } a \in [5, 8] \)

 $(5.0, \infty)$, which corresponds to switching the direction of the interval AND negating the endpoint. You likely did this if you did not flip the inequality when dividing by a negative as well as not moving values over to a side properly.
\item \( \text{None of the above}. \)

You may have chosen this if you thought the inequality did not match the ends of the intervals.
\end{enumerate}

\textbf{General Comment:} Remember that less/greater than or equal to includes the endpoint, while less/greater do not. Also, remember that you need to flip the inequality when you multiply or divide by a negative.
}
\litem{
Solve the linear inequality below. Then, choose the constant and interval combination that describes the solution set.
\[ \frac{8}{3} - \frac{7}{6} x < \frac{-4}{4} x - \frac{10}{8} \]The solution is \( (23.5, \infty) \), which is option D.\begin{enumerate}[label=\Alph*.]
\item \( (-\infty, a), \text{ where } a \in [20.25, 25.5] \)

 $(-\infty, 23.5)$, which corresponds to switching the direction of the interval. You likely did this if you did not flip the inequality when dividing by a negative!
\item \( (a, \infty), \text{ where } a \in [-24.75, -20.25] \)

 $(-23.5, \infty)$, which corresponds to negating the endpoint of the solution.
\item \( (-\infty, a), \text{ where } a \in [-26.25, -20.25] \)

 $(-\infty, -23.5)$, which corresponds to switching the direction of the interval AND negating the endpoint. You likely did this if you did not flip the inequality when dividing by a negative as well as not moving values over to a side properly.
\item \( (a, \infty), \text{ where } a \in [21, 27.75] \)

* $(23.5, \infty)$, which is the correct option.
\item \( \text{None of the above}. \)

You may have chosen this if you thought the inequality did not match the ends of the intervals.
\end{enumerate}

\textbf{General Comment:} Remember that less/greater than or equal to includes the endpoint, while less/greater do not. Also, remember that you need to flip the inequality when you multiply or divide by a negative.
}
\end{enumerate}

\end{document}