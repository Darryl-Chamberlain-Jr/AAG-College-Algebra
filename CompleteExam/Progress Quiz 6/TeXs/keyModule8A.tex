\documentclass{extbook}[14pt]
\usepackage{multicol, enumerate, enumitem, hyperref, color, soul, setspace, parskip, fancyhdr, amssymb, amsthm, amsmath, latexsym, units, mathtools}
\everymath{\displaystyle}
\usepackage[headsep=0.5cm,headheight=0cm, left=1 in,right= 1 in,top= 1 in,bottom= 1 in]{geometry}
\usepackage{dashrule}  % Package to use the command below to create lines between items
\newcommand{\litem}[1]{\item #1

\rule{\textwidth}{0.4pt}}
\pagestyle{fancy}
\lhead{}
\chead{Answer Key for Progress Quiz 6 Version A}
\rhead{}
\lfoot{1430-1829}
\cfoot{}
\rfoot{test}
\begin{document}
\textbf{This key should allow you to understand why you choose the option you did (beyond just getting a question right or wrong). \href{https://xronos.clas.ufl.edu/mac1105spring2020/courseDescriptionAndMisc/Exams/LearningFromResults}{More instructions on how to use this key can be found here}.}

\textbf{If you have a suggestion to make the keys better, \href{https://forms.gle/CZkbZmPbC9XALEE88}{please fill out the short survey here}.}

\textit{Note: This key is auto-generated and may contain issues and/or errors. The keys are reviewed after each exam to ensure grading is done accurately. If there are issues (like duplicate options), they are noted in the offline gradebook. The keys are a work-in-progress to give students as many resources to improve as possible.}

\rule{\textwidth}{0.4pt}

\begin{enumerate}\litem{
Solve the equation for $x$ and choose the interval that contains the solution (if it exists).
\[ 5^{-4x+4} = \left(\frac{1}{64}\right)^{2x-5} \]The solution is \( x = 7.636 \), which is option A.\begin{enumerate}[label=\Alph*.]
\item \( x \in [5.5, 7.9] \)

* $x = 7.636$, which is the correct option.
\item \( x \in [1.1, 2.2] \)

$x = 1.500$, which corresponds to solving the numerators as equal while ignoring the bases are different.
\item \( x \in [-6.4, -4.1] \)

$x = -4.787$, which corresponds to distributing the $\ln(base)$ to the first term of the exponent only.
\item \( x \in [-4.4, -1.3] \)

$x = -2.393$, which corresponds to distributing the $\ln(base)$ to the second term of the exponent only.
\item \( \text{There is no Real solution to the equation.} \)

This corresponds to believing there is no solution since the bases are not powers of each other.
\end{enumerate}

\textbf{General Comment:} \textbf{General Comments:} This question was written so that the bases could not be written the same. You will need to take the log of both sides.
}
\litem{
Solve the equation for $x$ and choose the interval that contains the solution (if it exists).
\[ \log_{4}{(3x+8)}+4 = 3 \]The solution is \( x = -2.583 \), which is option A.\begin{enumerate}[label=\Alph*.]
\item \( x \in [-2.66, -2.43] \)

* $x = -2.583$, which is the correct option.
\item \( x \in [-2.53, -1.89] \)

$x = -2.333$, which corresponds to reversing the base and exponent when converting.
\item \( x \in [2.82, 3.22] \)

$x = 3.000$, which corresponds to reversing the base and exponent when converting and reversing the value with $x$.
\item \( x \in [18.37, 18.82] \)

$x = 18.667$, which corresponds to ignoring the vertical shift when converting to exponential form.
\item \( \text{There is no Real solution to the equation.} \)

Corresponds to believing a negative coefficient within the log equation means there is no Real solution.
\end{enumerate}

\textbf{General Comment:} \textbf{General Comments:} First, get the equation in the form $\log_b{(cx+d)} = a$. Then, convert to $b^a = cx+d$ and solve.
}
\litem{
Which of the following intervals describes the Domain of the function below?
\[ f(x) = -e^{x-7}+6 \]The solution is \( (-\infty, \infty) \), which is option E.\begin{enumerate}[label=\Alph*.]
\item \( (-\infty, a), a \in [3, 9] \)

$(-\infty, 6)$, which corresponds to using the correct vertical shift *if we wanted the Range*.
\item \( (a, \infty), a \in [-10, -3] \)

$(-6, \infty)$, which corresponds to using the negative vertical shift AND flipping the Range interval.
\item \( (-\infty, a], a \in [3, 9] \)

$(-\infty, 6]$, which corresponds to using the correct vertical shift *if we wanted the Range* AND including the endpoint.
\item \( [a, \infty), a \in [-10, -3] \)

$[-6, \infty)$, which corresponds to using the negative vertical shift AND flipping the Range interval AND including the endpoint.
\item \( (-\infty, \infty) \)

* This is the correct option.
\end{enumerate}

\textbf{General Comment:} \textbf{General Comments}: Domain of a basic exponential function is $(-\infty, \infty)$ while the Range is $(0, \infty)$. We can shift these intervals [and even flip when $a<0$!] to find the new Domain/Range.
}
\litem{
 Solve the equation for $x$ and choose the interval that contains $x$ (if it exists).
\[  14 = \sqrt[5]{\frac{8}{e^{6x}}} \]The solution is \( x = -1.853, \text{ which does not fit in any of the interval options.} \), which is option E.\begin{enumerate}[label=\Alph*.]
\item \( x \in [1.48, 2.38] \)

$x = 1.853$, which is the negative of the correct solution.
\item \( x \in [-0.61, 0.31] \)

$x = -0.533$, which corresponds to treating any root as a square root.
\item \( x \in [-12.35, -11.32] \)

$x = -12.013$, which corresponds to thinking you don't need to take the natural log of both sides before reducing, as if the right side already has a natural log.
\item \( \text{There is no Real solution to the equation.} \)

This corresponds to believing you cannot solve the equation.
\item \( \text{None of the above.} \)

* $x = -1.853$ is the correct solution and does not fit in any of the other intervals.
\end{enumerate}

\textbf{General Comment:} \textbf{General Comments}: After using the properties of logarithmic functions to break up the right-hand side, use $\ln(e) = 1$ to reduce the question to a linear function to solve. You can put $\ln(8)$ into a calculator if you are having trouble.
}
\litem{
Solve the equation for $x$ and choose the interval that contains the solution (if it exists).
\[ 3^{-4x-5} = 343^{-2x+3} \]The solution is \( x = 3.160 \), which is option B.\begin{enumerate}[label=\Alph*.]
\item \( x \in [0.1, 2.1] \)

$x = 1.099$, which corresponds to distributing the $\ln(base)$ to the first term of the exponent only.
\item \( x \in [1.16, 5.16] \)

* $x = 3.160$, which is the correct option.
\item \( x \in [-4, -1] \)

$x = -4.000$, which corresponds to solving the numerators as equal while ignoring the bases are different.
\item \( x \in [-13.5, -10.5] \)

$x = -11.503$, which corresponds to distributing the $\ln(base)$ to the second term of the exponent only.
\item \( \text{There is no Real solution to the equation.} \)

This corresponds to believing there is no solution since the bases are not powers of each other.
\end{enumerate}

\textbf{General Comment:} \textbf{General Comments:} This question was written so that the bases could not be written the same. You will need to take the log of both sides.
}
\litem{
Solve the equation for $x$ and choose the interval that contains the solution (if it exists).
\[ \log_{5}{(4x+6)}+4 = 3 \]The solution is \( x = -1.450 \), which is option D.\begin{enumerate}[label=\Alph*.]
\item \( x \in [29.65, 29.95] \)

$x = 29.750$, which corresponds to ignoring the vertical shift when converting to exponential form.
\item \( x \in [-1.75, -1.68] \)

$x = -1.750$, which corresponds to reversing the base and exponent when converting.
\item \( x \in [0.98, 1.35] \)

$x = 1.250$, which corresponds to reversing the base and exponent when converting and reversing the value with $x$.
\item \( x \in [-1.56, -1.39] \)

* $x = -1.450$, which is the correct option.
\item \( \text{There is no Real solution to the equation.} \)

Corresponds to believing a negative coefficient within the log equation means there is no Real solution.
\end{enumerate}

\textbf{General Comment:} \textbf{General Comments:} First, get the equation in the form $\log_b{(cx+d)} = a$. Then, convert to $b^a = cx+d$ and solve.
}
\litem{
Which of the following intervals describes the Domain of the function below?
\[ f(x) = \log_2{(x+3)}+8 \]The solution is \( (-3, \infty) \), which is option D.\begin{enumerate}[label=\Alph*.]
\item \( (-\infty, a), a \in [0.2, 5] \)

$(-\infty, 3)$, which corresponds to flipping the Domain. Remember: the general for is $a*\log(x-h)+k$, \textbf{where $a$ does not affect the domain}.
\item \( [a, \infty), a \in [5.4, 9.2] \)

$[8, \infty)$, which corresponds to using the vertical shift when shifting the Domain AND including the endpoint.
\item \( (-\infty, a], a \in [-9, -7.4] \)

$(-\infty, -8]$, which corresponds to using the negative vertical shift AND including the endpoint AND flipping the domain.
\item \( (a, \infty), a \in [-3.1, 0.4] \)

* $(-3, \infty)$, which is the correct option.
\item \( (-\infty, \infty) \)

This corresponds to thinking of the range of the log function (or the domain of the exponential function).
\end{enumerate}

\textbf{General Comment:} \textbf{General Comments}: The domain of a basic logarithmic function is $(0, \infty)$ and the Range is $(-\infty, \infty)$. We can use shifts when finding the Domain, but the Range will always be all Real numbers.
}
\litem{
Which of the following intervals describes the Domain of the function below?
\[ f(x) = -\log_2{(x-9)}+1 \]The solution is \( (9, \infty) \), which is option A.\begin{enumerate}[label=\Alph*.]
\item \( (a, \infty), a \in [7.2, 9.1] \)

* $(9, \infty)$, which is the correct option.
\item \( (-\infty, a), a \in [-9.1, -7.7] \)

$(-\infty, -9)$, which corresponds to flipping the Domain. Remember: the general for is $a*\log(x-h)+k$, \textbf{where $a$ does not affect the domain}.
\item \( [a, \infty), a \in [0.1, 1.4] \)

$[1, \infty)$, which corresponds to using the vertical shift when shifting the Domain AND including the endpoint.
\item \( (-\infty, a], a \in [-4.9, -0.1] \)

$(-\infty, -1]$, which corresponds to using the negative vertical shift AND including the endpoint AND flipping the domain.
\item \( (-\infty, \infty) \)

This corresponds to thinking of the range of the log function (or the domain of the exponential function).
\end{enumerate}

\textbf{General Comment:} \textbf{General Comments}: The domain of a basic logarithmic function is $(0, \infty)$ and the Range is $(-\infty, \infty)$. We can use shifts when finding the Domain, but the Range will always be all Real numbers.
}
\litem{
 Solve the equation for $x$ and choose the interval that contains $x$ (if it exists).
\[  20 = \sqrt[3]{\frac{15}{e^{9x}}} \]The solution is \( x = -0.698 \), which is option C.\begin{enumerate}[label=\Alph*.]
\item \( x \in [-7.1, -5.4] \)

$x = -6.968$, which corresponds to thinking you don't need to take the natural log of both sides before reducing, as if the equation already had a natural log on the right side.
\item \( x \in [-0.5, 1.6] \)

$x = -0.365$, which corresponds to treating any root as a square root.
\item \( x \in [-0.8, -0.4] \)

* $x = -0.698$, which is the correct option.
\item \( \text{There is no Real solution to the equation.} \)

This corresponds to believing you cannot solve the equation.
\item \( \text{None of the above.} \)

This corresponds to making an unexpected error.
\end{enumerate}

\textbf{General Comment:} \textbf{General Comments}: After using the properties of logarithmic functions to break up the right-hand side, use $\ln(e) = 1$ to reduce the question to a linear function to solve. You can put $\ln(15)$ into a calculator if you are having trouble.
}
\litem{
Which of the following intervals describes the Range of the function below?
\[ f(x) = -e^{x-6}+7 \]The solution is \( (-\infty, 7) \), which is option A.\begin{enumerate}[label=\Alph*.]
\item \( (-\infty, a), a \in [7, 9] \)

* $(-\infty, 7)$, which is the correct option.
\item \( (a, \infty), a \in [-7, -5] \)

$(-7, \infty)$, which corresponds to using the negative vertical shift AND flipping the Range interval.
\item \( [a, \infty), a \in [-7, -5] \)

$[-7, \infty)$, which corresponds to using the negative vertical shift AND flipping the Range interval AND including the endpoint.
\item \( (-\infty, a], a \in [7, 9] \)

$(-\infty, 7]$, which corresponds to including the endpoint.
\item \( (-\infty, \infty) \)

This corresponds to confusing range of an exponential function with the domain of an exponential function.
\end{enumerate}

\textbf{General Comment:} \textbf{General Comments}: Domain of a basic exponential function is $(-\infty, \infty)$ while the Range is $(0, \infty)$. We can shift these intervals [and even flip when $a<0$!] to find the new Domain/Range.
}
\end{enumerate}

\end{document}