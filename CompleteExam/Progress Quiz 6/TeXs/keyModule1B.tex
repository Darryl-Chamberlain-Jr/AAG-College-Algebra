\documentclass{extbook}[14pt]
\usepackage{multicol, enumerate, enumitem, hyperref, color, soul, setspace, parskip, fancyhdr, amssymb, amsthm, amsmath, latexsym, units, mathtools}
\everymath{\displaystyle}
\usepackage[headsep=0.5cm,headheight=0cm, left=1 in,right= 1 in,top= 1 in,bottom= 1 in]{geometry}
\usepackage{dashrule}  % Package to use the command below to create lines between items
\newcommand{\litem}[1]{\item #1

\rule{\textwidth}{0.4pt}}
\pagestyle{fancy}
\lhead{}
\chead{Answer Key for Progress Quiz 6 Version B}
\rhead{}
\lfoot{1430-1829}
\cfoot{}
\rfoot{test}
\begin{document}
\textbf{This key should allow you to understand why you choose the option you did (beyond just getting a question right or wrong). \href{https://xronos.clas.ufl.edu/mac1105spring2020/courseDescriptionAndMisc/Exams/LearningFromResults}{More instructions on how to use this key can be found here}.}

\textbf{If you have a suggestion to make the keys better, \href{https://forms.gle/CZkbZmPbC9XALEE88}{please fill out the short survey here}.}

\textit{Note: This key is auto-generated and may contain issues and/or errors. The keys are reviewed after each exam to ensure grading is done accurately. If there are issues (like duplicate options), they are noted in the offline gradebook. The keys are a work-in-progress to give students as many resources to improve as possible.}

\rule{\textwidth}{0.4pt}

\begin{enumerate}\litem{
Choose the \textbf{smallest} set of Complex numbers that the number below belongs to.
\[ \frac{0}{16 \pi}+\sqrt{9}i \]The solution is \( \text{Pure Imaginary} \), which is option D.\begin{enumerate}[label=\Alph*.]
\item \( \text{Nonreal Complex} \)

This is a Complex number $(a+bi)$ that is not Real (has $i$ as part of the number).
\item \( \text{Not a Complex Number} \)

This is not a number. The only non-Complex number we know is dividing by 0 as this is not a number!
\item \( \text{Irrational} \)

These cannot be written as a fraction of Integers. Remember: $\pi$ is not an Integer!
\item \( \text{Pure Imaginary} \)

* This is the correct option!
\item \( \text{Rational} \)

These are numbers that can be written as fraction of Integers (e.g., -2/3 + 5)
\end{enumerate}

\textbf{General Comment:} Be sure to simplify $i^2 = -1$. This may remove the imaginary portion for your number. If you are having trouble, you may want to look at the \textit{Subgroups of the Real Numbers} section.
}
\litem{
Simplify the expression below into the form $a+bi$. Then, choose the intervals that $a$ and $b$ belong to.
\[ \frac{54 + 55 i}{-4 + i} \]The solution is \( -9.47  - 16.12 i \), which is option C.\begin{enumerate}[label=\Alph*.]
\item \( a \in [-161.5, -159.5] \text{ and } b \in [-17, -15] \)

 $-161.00  - 16.12 i$, which corresponds to forgetting to multiply the conjugate by the numerator and using a plus instead of a minus in the denominator.
\item \( a \in [-17, -14.5] \text{ and } b \in [-10.5, -8.5] \)

 $-15.94  - 9.76 i$, which corresponds to forgetting to multiply the conjugate by the numerator and not computing the conjugate correctly.
\item \( a \in [-11, -8.5] \text{ and } b \in [-17, -15] \)

* $-9.47  - 16.12 i$, which is the correct option.
\item \( a \in [-11, -8.5] \text{ and } b \in [-276, -272.5] \)

 $-9.47  - 274.00 i$, which corresponds to forgetting to multiply the conjugate by the numerator.
\item \( a \in [-14.5, -12.5] \text{ and } b \in [54.5, 55.5] \)

 $-13.50  + 55.00 i$, which corresponds to just dividing the first term by the first term and the second by the second.
\end{enumerate}

\textbf{General Comment:} Multiply the numerator and denominator by the *conjugate* of the denominator, then simplify. For example, if we have $2+3i$, the conjugate is $2-3i$.
}
\litem{
Simplify the expression below into the form $a+bi$. Then, choose the intervals that $a$ and $b$ belong to.
\[ (-5 + 3 i)(-2 + 8 i) \]The solution is \( -14 - 46 i \), which is option C.\begin{enumerate}[label=\Alph*.]
\item \( a \in [34, 37] \text{ and } b \in [33, 37] \)

 $34 + 34 i$, which corresponds to adding a minus sign in the second term.
\item \( a \in [34, 37] \text{ and } b \in [-37, -33] \)

 $34 - 34 i$, which corresponds to adding a minus sign in the first term.
\item \( a \in [-16, -10] \text{ and } b \in [-50, -44] \)

* $-14 - 46 i$, which is the correct option.
\item \( a \in [5, 16] \text{ and } b \in [14, 30] \)

 $10 + 24 i$, which corresponds to just multiplying the real terms to get the real part of the solution and the coefficients in the complex terms to get the complex part.
\item \( a \in [-16, -10] \text{ and } b \in [46, 54] \)

 $-14 + 46 i$, which corresponds to adding a minus sign in both terms.
\end{enumerate}

\textbf{General Comment:} You can treat $i$ as a variable and distribute. Just remember that $i^2=-1$, so you can continue to reduce after you distribute.
}
\litem{
Simplify the expression below into the form $a+bi$. Then, choose the intervals that $a$ and $b$ belong to.
\[ \frac{-63 + 11 i}{-5 - 6 i} \]The solution is \( 4.08  - 7.10 i \), which is option D.\begin{enumerate}[label=\Alph*.]
\item \( a \in [248, 249.5] \text{ and } b \in [-8, -6] \)

 $249.00  - 7.10 i$, which corresponds to forgetting to multiply the conjugate by the numerator and using a plus instead of a minus in the denominator.
\item \( a \in [4, 4.5] \text{ and } b \in [-433.5, -432] \)

 $4.08  - 433.00 i$, which corresponds to forgetting to multiply the conjugate by the numerator.
\item \( a \in [12, 14] \text{ and } b \in [-2, -1] \)

 $12.60  - 1.83 i$, which corresponds to just dividing the first term by the first term and the second by the second.
\item \( a \in [4, 4.5] \text{ and } b \in [-8, -6] \)

* $4.08  - 7.10 i$, which is the correct option.
\item \( a \in [6, 7] \text{ and } b \in [3.5, 5.5] \)

 $6.25  + 5.30 i$, which corresponds to forgetting to multiply the conjugate by the numerator and not computing the conjugate correctly.
\end{enumerate}

\textbf{General Comment:} Multiply the numerator and denominator by the *conjugate* of the denominator, then simplify. For example, if we have $2+3i$, the conjugate is $2-3i$.
}
\litem{
Choose the \textbf{smallest} set of Real numbers that the number below belongs to.
\[ \sqrt{\frac{360000}{625}} \]The solution is \( \text{Whole} \), which is option C.\begin{enumerate}[label=\Alph*.]
\item \( \text{Rational} \)

These are numbers that can be written as fraction of Integers (e.g., -2/3)
\item \( \text{Irrational} \)

These cannot be written as a fraction of Integers.
\item \( \text{Whole} \)

* This is the correct option!
\item \( \text{Integer} \)

These are the negative and positive counting numbers (..., -3, -2, -1, 0, 1, 2, 3, ...)
\item \( \text{Not a Real number} \)

These are Nonreal Complex numbers \textbf{OR} things that are not numbers (e.g., dividing by 0).
\end{enumerate}

\textbf{General Comment:} First, you \textbf{NEED} to simplify the expression. This question simplifies to $600$. 
 
 Be sure you look at the simplified fraction and not just the decimal expansion. Numbers such as 13, 17, and 19 provide \textbf{long but repeating/terminating decimal expansions!} 
 
 The only ways to *not* be a Real number are: dividing by 0 or taking the square root of a negative number. 
 
 Irrational numbers are more than just square root of 3: adding or subtracting values from square root of 3 is also irrational.
}
\litem{
Choose the \textbf{smallest} set of Real numbers that the number below belongs to.
\[ \sqrt{\frac{-1386}{9}} \]The solution is \( \text{Not a Real number} \), which is option A.\begin{enumerate}[label=\Alph*.]
\item \( \text{Not a Real number} \)

* This is the correct option!
\item \( \text{Integer} \)

These are the negative and positive counting numbers (..., -3, -2, -1, 0, 1, 2, 3, ...)
\item \( \text{Irrational} \)

These cannot be written as a fraction of Integers.
\item \( \text{Whole} \)

These are the counting numbers with 0 (0, 1, 2, 3, ...)
\item \( \text{Rational} \)

These are numbers that can be written as fraction of Integers (e.g., -2/3)
\end{enumerate}

\textbf{General Comment:} First, you \textbf{NEED} to simplify the expression. This question simplifies to $\sqrt{154} i$. 
 
 Be sure you look at the simplified fraction and not just the decimal expansion. Numbers such as 13, 17, and 19 provide \textbf{long but repeating/terminating decimal expansions!} 
 
 The only ways to *not* be a Real number are: dividing by 0 or taking the square root of a negative number. 
 
 Irrational numbers are more than just square root of 3: adding or subtracting values from square root of 3 is also irrational.
}
\litem{
Simplify the expression below and choose the interval the simplification is contained within.
\[ 16 - 3 \div 1 * 15 - (2 * 8) \]The solution is \( -45.000 \), which is option B.\begin{enumerate}[label=\Alph*.]
\item \( [25.8, 32.8] \)

 31.800, which corresponds to not distributing addition and subtraction correctly.
\item \( [-48, -42] \)

* -45.000, which is the correct option.
\item \( [-2.2, 0.8] \)

 -0.200, which corresponds to an Order of Operations error: not reading left-to-right for multiplication/division.
\item \( [-253, -243] \)

 -248.000, which corresponds to not distributing a negative correctly.
\item \( \text{None of the above} \)

 You may have gotten this by making an unanticipated error. If you got a value that is not any of the others, please let the coordinator know so they can help you figure out what happened.
\end{enumerate}

\textbf{General Comment:} While you may remember (or were taught) PEMDAS is done in order, it is actually done as P/E/MD/AS. When we are at MD or AS, we read left to right.
}
\litem{
Choose the \textbf{smallest} set of Complex numbers that the number below belongs to.
\[ \sqrt{\frac{770}{11}}+\sqrt{143} i \]The solution is \( \text{Nonreal Complex} \), which is option C.\begin{enumerate}[label=\Alph*.]
\item \( \text{Not a Complex Number} \)

This is not a number. The only non-Complex number we know is dividing by 0 as this is not a number!
\item \( \text{Irrational} \)

These cannot be written as a fraction of Integers. Remember: $\pi$ is not an Integer!
\item \( \text{Nonreal Complex} \)

* This is the correct option!
\item \( \text{Rational} \)

These are numbers that can be written as fraction of Integers (e.g., -2/3 + 5)
\item \( \text{Pure Imaginary} \)

This is a Complex number $(a+bi)$ that \textbf{only} has an imaginary part like $2i$.
\end{enumerate}

\textbf{General Comment:} Be sure to simplify $i^2 = -1$. This may remove the imaginary portion for your number. If you are having trouble, you may want to look at the \textit{Subgroups of the Real Numbers} section.
}
\litem{
Simplify the expression below into the form $a+bi$. Then, choose the intervals that $a$ and $b$ belong to.
\[ (7 + 2 i)(-5 - 9 i) \]The solution is \( -17 - 73 i \), which is option C.\begin{enumerate}[label=\Alph*.]
\item \( a \in [-37, -31] \text{ and } b \in [-24, -13] \)

 $-35 - 18 i$, which corresponds to just multiplying the real terms to get the real part of the solution and the coefficients in the complex terms to get the complex part.
\item \( a \in [-53, -52] \text{ and } b \in [53, 56] \)

 $-53 + 53 i$, which corresponds to adding a minus sign in the second term.
\item \( a \in [-18, -13] \text{ and } b \in [-78, -68] \)

* $-17 - 73 i$, which is the correct option.
\item \( a \in [-18, -13] \text{ and } b \in [70, 75] \)

 $-17 + 73 i$, which corresponds to adding a minus sign in both terms.
\item \( a \in [-53, -52] \text{ and } b \in [-54, -52] \)

 $-53 - 53 i$, which corresponds to adding a minus sign in the first term.
\end{enumerate}

\textbf{General Comment:} You can treat $i$ as a variable and distribute. Just remember that $i^2=-1$, so you can continue to reduce after you distribute.
}
\litem{
Simplify the expression below and choose the interval the simplification is contained within.
\[ 14 - 3^2 + 9 \div 12 * 11 \div 17 \]The solution is \( 5.485 \), which is option D.\begin{enumerate}[label=\Alph*.]
\item \( [23.28, 23.96] \)

 23.485, which corresponds to an Order of Operations error: multiplying by negative before squaring. For example: $(-3)^2 \neq -3^2$
\item \( [22.06, 23.09] \)

 23.004, which corresponds to two Order of Operations errors.
\item \( [4.84, 5.01] \)

 5.004, which corresponds to an Order of Operations error: not reading left-to-right for multiplication/division.
\item \( [5.06, 5.61] \)

* 5.485, this is the correct option
\item \( \text{None of the above} \)

 You may have gotten this by making an unanticipated error. If you got a value that is not any of the others, please let the coordinator know so they can help you figure out what happened.
\end{enumerate}

\textbf{General Comment:} While you may remember (or were taught) PEMDAS is done in order, it is actually done as P/E/MD/AS. When we are at MD or AS, we read left to right.
}
\end{enumerate}

\end{document}