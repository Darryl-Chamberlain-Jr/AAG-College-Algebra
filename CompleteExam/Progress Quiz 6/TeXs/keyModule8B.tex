\documentclass{extbook}[14pt]
\usepackage{multicol, enumerate, enumitem, hyperref, color, soul, setspace, parskip, fancyhdr, amssymb, amsthm, amsmath, latexsym, units, mathtools}
\everymath{\displaystyle}
\usepackage[headsep=0.5cm,headheight=0cm, left=1 in,right= 1 in,top= 1 in,bottom= 1 in]{geometry}
\usepackage{dashrule}  % Package to use the command below to create lines between items
\newcommand{\litem}[1]{\item #1

\rule{\textwidth}{0.4pt}}
\pagestyle{fancy}
\lhead{}
\chead{Answer Key for Progress Quiz 6 Version B}
\rhead{}
\lfoot{1430-1829}
\cfoot{}
\rfoot{test}
\begin{document}
\textbf{This key should allow you to understand why you choose the option you did (beyond just getting a question right or wrong). \href{https://xronos.clas.ufl.edu/mac1105spring2020/courseDescriptionAndMisc/Exams/LearningFromResults}{More instructions on how to use this key can be found here}.}

\textbf{If you have a suggestion to make the keys better, \href{https://forms.gle/CZkbZmPbC9XALEE88}{please fill out the short survey here}.}

\textit{Note: This key is auto-generated and may contain issues and/or errors. The keys are reviewed after each exam to ensure grading is done accurately. If there are issues (like duplicate options), they are noted in the offline gradebook. The keys are a work-in-progress to give students as many resources to improve as possible.}

\rule{\textwidth}{0.4pt}

\begin{enumerate}\litem{
Solve the equation for $x$ and choose the interval that contains the solution (if it exists).
\[ 3^{4x+2} = 49^{3x-3} \]The solution is \( x = 1.905 \), which is option A.\begin{enumerate}[label=\Alph*.]
\item \( x \in [0.7, 3] \)

* $x = 1.905$, which is the correct option.
\item \( x \in [-5.7, -2.9] \)

$x = -5.000$, which corresponds to solving the numerators as equal while ignoring the bases are different.
\item \( x \in [-15.2, -12.1] \)

$x = -13.873$, which corresponds to distributing the $\ln(base)$ to the second term of the exponent only.
\item \( x \in [0.3, 1.8] \)

$x = 0.687$, which corresponds to distributing the $\ln(base)$ to the first term of the exponent only.
\item \( \text{There is no Real solution to the equation.} \)

This corresponds to believing there is no solution since the bases are not powers of each other.
\end{enumerate}

\textbf{General Comment:} \textbf{General Comments:} This question was written so that the bases could not be written the same. You will need to take the log of both sides.
}
\litem{
Solve the equation for $x$ and choose the interval that contains the solution (if it exists).
\[ \log_{4}{(-4x+8)}+5 = 3 \]The solution is \( x = 1.984 \), which is option D.\begin{enumerate}[label=\Alph*.]
\item \( x \in [-12, -5] \)

$x = -6.000$, which corresponds to reversing the base and exponent when converting and reversing the value with $x$.
\item \( x \in [-2, 1] \)

$x = -2.000$, which corresponds to reversing the base and exponent when converting.
\item \( x \in [-16, -13] \)

$x = -14.000$, which corresponds to ignoring the vertical shift when converting to exponential form.
\item \( x \in [-1.02, 4.98] \)

* $x = 1.984$, which is the correct option.
\item \( \text{There is no Real solution to the equation.} \)

Corresponds to believing a negative coefficient within the log equation means there is no Real solution.
\end{enumerate}

\textbf{General Comment:} \textbf{General Comments:} First, get the equation in the form $\log_b{(cx+d)} = a$. Then, convert to $b^a = cx+d$ and solve.
}
\litem{
Which of the following intervals describes the Domain of the function below?
\[ f(x) = -e^{x-8}-1 \]The solution is \( (-\infty, \infty) \), which is option E.\begin{enumerate}[label=\Alph*.]
\item \( (-\infty, a), a \in [-3.3, 0.5] \)

$(-\infty, -1)$, which corresponds to using the correct vertical shift *if we wanted the Range*.
\item \( (a, \infty), a \in [-0.2, 2.1] \)

$(1, \infty)$, which corresponds to using the negative vertical shift AND flipping the Range interval.
\item \( [a, \infty), a \in [-0.2, 2.1] \)

$[1, \infty)$, which corresponds to using the negative vertical shift AND flipping the Range interval AND including the endpoint.
\item \( (-\infty, a], a \in [-3.3, 0.5] \)

$(-\infty, -1]$, which corresponds to using the correct vertical shift *if we wanted the Range* AND including the endpoint.
\item \( (-\infty, \infty) \)

* This is the correct option.
\end{enumerate}

\textbf{General Comment:} \textbf{General Comments}: Domain of a basic exponential function is $(-\infty, \infty)$ while the Range is $(0, \infty)$. We can shift these intervals [and even flip when $a<0$!] to find the new Domain/Range.
}
\litem{
 Solve the equation for $x$ and choose the interval that contains $x$ (if it exists).
\[  25 = \sqrt[7]{\frac{26}{e^{8x}}} \]The solution is \( x = -2.409, \text{ which does not fit in any of the interval options.} \), which is option E.\begin{enumerate}[label=\Alph*.]
\item \( x \in [-22.8, -21.7] \)

$x = -22.282$, which corresponds to thinking you don't need to take the natural log of both sides before reducing, as if the right side already has a natural log.
\item \( x \in [-1.2, 1.3] \)

$x = -0.397$, which corresponds to treating any root as a square root.
\item \( x \in [2.4, 4.4] \)

$x = 2.409$, which is the negative of the correct solution.
\item \( \text{There is no Real solution to the equation.} \)

This corresponds to believing you cannot solve the equation.
\item \( \text{None of the above.} \)

* $x = -2.409$ is the correct solution and does not fit in any of the other intervals.
\end{enumerate}

\textbf{General Comment:} \textbf{General Comments}: After using the properties of logarithmic functions to break up the right-hand side, use $\ln(e) = 1$ to reduce the question to a linear function to solve. You can put $\ln(26)$ into a calculator if you are having trouble.
}
\litem{
Solve the equation for $x$ and choose the interval that contains the solution (if it exists).
\[ 5^{4x-2} = 216^{2x+5} \]The solution is \( x = -6.978 \), which is option A.\begin{enumerate}[label=\Alph*.]
\item \( x \in [-10.98, -4.98] \)

* $x = -6.978$, which is the correct option.
\item \( x \in [-3.62, -0.62] \)

$x = -1.623$, which corresponds to distributing the $\ln(base)$ to the first term of the exponent only.
\item \( x \in [0.5, 4.5] \)

$x = 3.500$, which corresponds to solving the numerators as equal while ignoring the bases are different.
\item \( x \in [14.05, 17.05] \)

$x = 15.048$, which corresponds to distributing the $\ln(base)$ to the second term of the exponent only.
\item \( \text{There is no Real solution to the equation.} \)

This corresponds to believing there is no solution since the bases are not powers of each other.
\end{enumerate}

\textbf{General Comment:} \textbf{General Comments:} This question was written so that the bases could not be written the same. You will need to take the log of both sides.
}
\litem{
Solve the equation for $x$ and choose the interval that contains the solution (if it exists).
\[ \log_{3}{(4x+6)}+4 = 2 \]The solution is \( x = -1.472 \), which is option C.\begin{enumerate}[label=\Alph*.]
\item \( x \in [-0.06, 0.9] \)

$x = 0.750$, which corresponds to ignoring the vertical shift when converting to exponential form.
\item \( x \in [-4.03, -2.7] \)

$x = -3.500$, which corresponds to reversing the base and exponent when converting.
\item \( x \in [-1.58, -1.14] \)

* $x = -1.472$, which is the correct option.
\item \( x \in [-0.64, -0.21] \)

$x = -0.500$, which corresponds to reversing the base and exponent when converting and reversing the value with $x$.
\item \( \text{There is no Real solution to the equation.} \)

Corresponds to believing a negative coefficient within the log equation means there is no Real solution.
\end{enumerate}

\textbf{General Comment:} \textbf{General Comments:} First, get the equation in the form $\log_b{(cx+d)} = a$. Then, convert to $b^a = cx+d$ and solve.
}
\litem{
Which of the following intervals describes the Domain of the function below?
\[ f(x) = \log_2{(x-1)}-2 \]The solution is \( (1, \infty) \), which is option A.\begin{enumerate}[label=\Alph*.]
\item \( (a, \infty), a \in [0.67, 1.01] \)

* $(1, \infty)$, which is the correct option.
\item \( (-\infty, a], a \in [1.38, 2.17] \)

$(-\infty, 2]$, which corresponds to using the negative vertical shift AND including the endpoint AND flipping the domain.
\item \( (-\infty, a), a \in [-1.19, -0.47] \)

$(-\infty, -1)$, which corresponds to flipping the Domain. Remember: the general for is $a*\log(x-h)+k$, \textbf{where $a$ does not affect the domain}.
\item \( [a, \infty), a \in [-2.33, -1.74] \)

$[-2, \infty)$, which corresponds to using the vertical shift when shifting the Domain AND including the endpoint.
\item \( (-\infty, \infty) \)

This corresponds to thinking of the range of the log function (or the domain of the exponential function).
\end{enumerate}

\textbf{General Comment:} \textbf{General Comments}: The domain of a basic logarithmic function is $(0, \infty)$ and the Range is $(-\infty, \infty)$. We can use shifts when finding the Domain, but the Range will always be all Real numbers.
}
\litem{
Which of the following intervals describes the Range of the function below?
\[ f(x) = \log_2{(x+8)}-7 \]The solution is \( (\infty, \infty) \), which is option E.\begin{enumerate}[label=\Alph*.]
\item \( [a, \infty), a \in [7.55, 8.98] \)

$[8, \infty)$, which corresponds to using the negative of the horizontal shift AND including the endpoint.
\item \( [a, \infty), a \in [-8.2, -7.71] \)

$[-7, \infty)$, which corresponds to using the flipped Domain AND including the endpoint.
\item \( (-\infty, a), a \in [-7.77, -6.22] \)

$(-\infty, -7)$, which corresponds to using the vertical shift while the Range is $(-\infty, \infty)$.
\item \( (-\infty, a), a \in [6.98, 7.66] \)

$(-\infty, 7)$, which corresponds to using the using the negative of vertical shift on $(0, \infty)$.
\item \( (-\infty, \infty) \)

*This is the correct option.
\end{enumerate}

\textbf{General Comment:} \textbf{General Comments}: The domain of a basic logarithmic function is $(0, \infty)$ and the Range is $(-\infty, \infty)$. We can use shifts when finding the Domain, but the Range will always be all Real numbers.
}
\litem{
 Solve the equation for $x$ and choose the interval that contains $x$ (if it exists).
\[  12 = \ln{\sqrt[4]{\frac{6}{e^{7x}}}} \]The solution is \( x = -6.601 \), which is option C.\begin{enumerate}[label=\Alph*.]
\item \( x \in [-2, -0.6] \)

$x = -1.676$, which corresponds to thinking you need to take the natural log of on the left before reducing.
\item \( x \in [-4.6, -2] \)

$x = -3.173$, which corresponds to treating any root as a square root.
\item \( x \in [-6.7, -5.7] \)

* $x = -6.601$, which is the correct option.
\item \( \text{There is no Real solution to the equation.} \)

This corresponds to believing you cannot solve the equation.
\item \( \text{None of the above.} \)

This corresponds to making an unexpected error.
\end{enumerate}

\textbf{General Comment:} \textbf{General Comments}: After using the properties of logarithmic functions to break up the right-hand side, use $\ln(e) = 1$ to reduce the question to a linear function to solve. You can put $\ln(6)$ into a calculator if you are having trouble.
}
\litem{
Which of the following intervals describes the Domain of the function below?
\[ f(x) = e^{x+3}+9 \]The solution is \( (-\infty, \infty) \), which is option E.\begin{enumerate}[label=\Alph*.]
\item \( (-\infty, a], a \in [7, 12] \)

$(-\infty, 9]$, which corresponds to using the correct vertical shift *if we wanted the Range* AND including the endpoint.
\item \( (-\infty, a), a \in [7, 12] \)

$(-\infty, 9)$, which corresponds to using the correct vertical shift *if we wanted the Range*.
\item \( [a, \infty), a \in [-9, 1] \)

$[-9, \infty)$, which corresponds to using the negative vertical shift AND flipping the Range interval AND including the endpoint.
\item \( (a, \infty), a \in [-9, 1] \)

$(-9, \infty)$, which corresponds to using the negative vertical shift AND flipping the Range interval.
\item \( (-\infty, \infty) \)

* This is the correct option.
\end{enumerate}

\textbf{General Comment:} \textbf{General Comments}: Domain of a basic exponential function is $(-\infty, \infty)$ while the Range is $(0, \infty)$. We can shift these intervals [and even flip when $a<0$!] to find the new Domain/Range.
}
\end{enumerate}

\end{document}