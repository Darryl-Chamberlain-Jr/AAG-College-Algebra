\documentclass{extbook}[14pt]
\usepackage{multicol, enumerate, enumitem, hyperref, color, soul, setspace, parskip, fancyhdr, amssymb, amsthm, amsmath, latexsym, units, mathtools}
\everymath{\displaystyle}
\usepackage[headsep=0.5cm,headheight=0cm, left=1 in,right= 1 in,top= 1 in,bottom= 1 in]{geometry}
\usepackage{dashrule}  % Package to use the command below to create lines between items
\newcommand{\litem}[1]{\item #1

\rule{\textwidth}{0.4pt}}
\pagestyle{fancy}
\lhead{}
\chead{Answer Key for Progress Quiz 6 Version C}
\rhead{}
\lfoot{1430-1829}
\cfoot{}
\rfoot{test}
\begin{document}
\textbf{This key should allow you to understand why you choose the option you did (beyond just getting a question right or wrong). \href{https://xronos.clas.ufl.edu/mac1105spring2020/courseDescriptionAndMisc/Exams/LearningFromResults}{More instructions on how to use this key can be found here}.}

\textbf{If you have a suggestion to make the keys better, \href{https://forms.gle/CZkbZmPbC9XALEE88}{please fill out the short survey here}.}

\textit{Note: This key is auto-generated and may contain issues and/or errors. The keys are reviewed after each exam to ensure grading is done accurately. If there are issues (like duplicate options), they are noted in the offline gradebook. The keys are a work-in-progress to give students as many resources to improve as possible.}

\rule{\textwidth}{0.4pt}

\begin{enumerate}\litem{
Solve the linear inequality below. Then, choose the constant and interval combination that describes the solution set.
\[ \frac{7}{6} - \frac{5}{8} x \leq \frac{4}{9} x + \frac{10}{2} \]The solution is \( [-3.584, \infty) \), which is option B.\begin{enumerate}[label=\Alph*.]
\item \( [a, \infty), \text{ where } a \in [0, 6] \)

 $[3.584, \infty)$, which corresponds to negating the endpoint of the solution.
\item \( [a, \infty), \text{ where } a \in [-4.5, 0] \)

* $[-3.584, \infty)$, which is the correct option.
\item \( (-\infty, a], \text{ where } a \in [-6.75, -2.25] \)

 $(-\infty, -3.584]$, which corresponds to switching the direction of the interval. You likely did this if you did not flip the inequality when dividing by a negative!
\item \( (-\infty, a], \text{ where } a \in [1.5, 6] \)

 $(-\infty, 3.584]$, which corresponds to switching the direction of the interval AND negating the endpoint. You likely did this if you did not flip the inequality when dividing by a negative as well as not moving values over to a side properly.
\item \( \text{None of the above}. \)

You may have chosen this if you thought the inequality did not match the ends of the intervals.
\end{enumerate}

\textbf{General Comment:} Remember that less/greater than or equal to includes the endpoint, while less/greater do not. Also, remember that you need to flip the inequality when you multiply or divide by a negative.
}
\litem{
Solve the linear inequality below. Then, choose the constant and interval combination that describes the solution set.
\[ -6 + 8 x > 11 x \text{ or } 6 + 7 x < 8 x \]The solution is \( (-\infty, -2.0) \text{ or } (6.0, \infty) \), which is option A.\begin{enumerate}[label=\Alph*.]
\item \( (-\infty, a) \cup (b, \infty), \text{ where } a \in [-3.75, 2.25] \text{ and } b \in [5.25, 9] \)

 * Correct option.
\item \( (-\infty, a] \cup [b, \infty), \text{ where } a \in [-4.5, 0.75] \text{ and } b \in [5.25, 9] \)

Corresponds to including the endpoints (when they should be excluded).
\item \( (-\infty, a) \cup (b, \infty), \text{ where } a \in [-8.25, -3.75] \text{ and } b \in [0.75, 4.5] \)

Corresponds to inverting the inequality and negating the solution.
\item \( (-\infty, a] \cup [b, \infty), \text{ where } a \in [-7.5, -4.5] \text{ and } b \in [1.5, 5.25] \)

Corresponds to including the endpoints AND negating.
\item \( (-\infty, \infty) \)

Corresponds to the variable canceling, which does not happen in this instance.
\end{enumerate}

\textbf{General Comment:} When multiplying or dividing by a negative, flip the sign.
}
\litem{
Solve the linear inequality below. Then, choose the constant and interval combination that describes the solution set.
\[ -4 - 5 x < \frac{-26 x - 4}{6} \leq 4 - 5 x \]The solution is \( \text{None of the above.} \), which is option E.\begin{enumerate}[label=\Alph*.]
\item \( (a, b], \text{ where } a \in [-1.5, 7.5] \text{ and } b \in [-8.25, -5.25] \)

$(5.00, -7.00]$, which is the correct interval but negatives of the actual endpoints.
\item \( [a, b), \text{ where } a \in [3.75, 8.25] \text{ and } b \in [-11.25, 0] \)

$[5.00, -7.00)$, which corresponds to flipping the inequality and getting negatives of the actual endpoints.
\item \( (-\infty, a] \cup (b, \infty), \text{ where } a \in [3.75, 6.75] \text{ and } b \in [-14.25, 0] \)

$(-\infty, 5.00] \cup (-7.00, \infty)$, which corresponds to displaying the and-inequality as an or-inequality AND flipping the inequality AND getting negatives of the actual endpoints.
\item \( (-\infty, a) \cup [b, \infty), \text{ where } a \in [4.5, 8.25] \text{ and } b \in [-11.25, 2.25] \)

$(-\infty, 5.00) \cup [-7.00, \infty)$, which corresponds to displaying the and-inequality as an or-inequality and getting negatives of the actual endpoints.
\item \( \text{None of the above.} \)

* This is correct as the answer should be $(-5.00, 7.00]$.
\end{enumerate}

\textbf{General Comment:} To solve, you will need to break up the compound inequality into two inequalities. Be sure to keep track of the inequality! It may be best to draw a number line and graph your solution.
}
\litem{
Using an interval or intervals, describe all the $x$-values within or including a distance of the given values.
\[ \text{ Less than } 2 \text{ units from the number } -4. \]The solution is \( (-6, -2) \), which is option C.\begin{enumerate}[label=\Alph*.]
\item \( (-\infty, -6] \cup [-2, \infty) \)

This describes the values no less than 2 from -4
\item \( [-6, -2] \)

This describes the values no more than 2 from -4
\item \( (-6, -2) \)

This describes the values less than 2 from -4
\item \( (-\infty, -6) \cup (-2, \infty) \)

This describes the values more than 2 from -4
\item \( \text{None of the above} \)

You likely thought the values in the interval were not correct.
\end{enumerate}

\textbf{General Comment:} When thinking about this language, it helps to draw a number line and try points.
}
\litem{
Solve the linear inequality below. Then, choose the constant and interval combination that describes the solution set.
\[ 4x + 9 > 5x -5 \]The solution is \( (-\infty, 14.0) \), which is option B.\begin{enumerate}[label=\Alph*.]
\item \( (a, \infty), \text{ where } a \in [-17, -4] \)

 $(-14.0, \infty)$, which corresponds to switching the direction of the interval AND negating the endpoint. You likely did this if you did not flip the inequality when dividing by a negative as well as not moving values over to a side properly.
\item \( (-\infty, a), \text{ where } a \in [13, 19] \)

* $(-\infty, 14.0)$, which is the correct option.
\item \( (a, \infty), \text{ where } a \in [9, 20] \)

 $(14.0, \infty)$, which corresponds to switching the direction of the interval. You likely did this if you did not flip the inequality when dividing by a negative!
\item \( (-\infty, a), \text{ where } a \in [-16, -3] \)

 $(-\infty, -14.0)$, which corresponds to negating the endpoint of the solution.
\item \( \text{None of the above}. \)

You may have chosen this if you thought the inequality did not match the ends of the intervals.
\end{enumerate}

\textbf{General Comment:} Remember that less/greater than or equal to includes the endpoint, while less/greater do not. Also, remember that you need to flip the inequality when you multiply or divide by a negative.
}
\litem{
Solve the linear inequality below. Then, choose the constant and interval combination that describes the solution set.
\[ -9 + 6 x > 8 x \text{ or } -5 + 4 x < 6 x \]The solution is \( (-\infty, -4.5) \text{ or } (-2.5, \infty) \), which is option D.\begin{enumerate}[label=\Alph*.]
\item \( (-\infty, a] \cup [b, \infty), \text{ where } a \in [-6, 0.75] \text{ and } b \in [-6, -1.5] \)

Corresponds to including the endpoints (when they should be excluded).
\item \( (-\infty, a) \cup (b, \infty), \text{ where } a \in [-3.75, 6] \text{ and } b \in [3.75, 6.75] \)

Corresponds to inverting the inequality and negating the solution.
\item \( (-\infty, a] \cup [b, \infty), \text{ where } a \in [-0.75, 6.75] \text{ and } b \in [-1.5, 6] \)

Corresponds to including the endpoints AND negating.
\item \( (-\infty, a) \cup (b, \infty), \text{ where } a \in [-5.25, -3] \text{ and } b \in [-8.25, 3] \)

 * Correct option.
\item \( (-\infty, \infty) \)

Corresponds to the variable canceling, which does not happen in this instance.
\end{enumerate}

\textbf{General Comment:} When multiplying or dividing by a negative, flip the sign.
}
\litem{
Using an interval or intervals, describe all the $x$-values within or including a distance of the given values.
\[ \text{ More than } 6 \text{ units from the number } 3. \]The solution is \( (-\infty, -3) \cup (9, \infty) \), which is option A.\begin{enumerate}[label=\Alph*.]
\item \( (-\infty, -3) \cup (9, \infty) \)

This describes the values more than 6 from 3
\item \( (-\infty, -3] \cup [9, \infty) \)

This describes the values no less than 6 from 3
\item \( (-3, 9) \)

This describes the values less than 6 from 3
\item \( [-3, 9] \)

This describes the values no more than 6 from 3
\item \( \text{None of the above} \)

You likely thought the values in the interval were not correct.
\end{enumerate}

\textbf{General Comment:} When thinking about this language, it helps to draw a number line and try points.
}
\litem{
Solve the linear inequality below. Then, choose the constant and interval combination that describes the solution set.
\[ -9 + 5 x \leq \frac{24 x - 3}{3} < 5 + 7 x \]The solution is \( [-2.67, 6.00) \), which is option A.\begin{enumerate}[label=\Alph*.]
\item \( [a, b), \text{ where } a \in [-3.75, 0.75] \text{ and } b \in [2.25, 8.25] \)

$[-2.67, 6.00)$, which is the correct option.
\item \( (-\infty, a) \cup [b, \infty), \text{ where } a \in [-5.25, 0] \text{ and } b \in [2.25, 6.75] \)

$(-\infty, -2.67) \cup [6.00, \infty)$, which corresponds to displaying the and-inequality as an or-inequality AND flipping the inequality.
\item \( (-\infty, a] \cup (b, \infty), \text{ where } a \in [-3, 0.75] \text{ and } b \in [4.5, 13.5] \)

$(-\infty, -2.67] \cup (6.00, \infty)$, which corresponds to displaying the and-inequality as an or-inequality.
\item \( (a, b], \text{ where } a \in [-7.5, -0.75] \text{ and } b \in [5.25, 6.75] \)

$(-2.67, 6.00]$, which corresponds to flipping the inequality.
\item \( \text{None of the above.} \)


\end{enumerate}

\textbf{General Comment:} To solve, you will need to break up the compound inequality into two inequalities. Be sure to keep track of the inequality! It may be best to draw a number line and graph your solution.
}
\litem{
Solve the linear inequality below. Then, choose the constant and interval combination that describes the solution set.
\[ -8x -3 \geq 9x -4 \]The solution is \( (-\infty, 0.059] \), which is option A.\begin{enumerate}[label=\Alph*.]
\item \( (-\infty, a], \text{ where } a \in [-0.04, 0.11] \)

* $(-\infty, 0.059]$, which is the correct option.
\item \( [a, \infty), \text{ where } a \in [-0.09, 0.02] \)

 $[-0.059, \infty)$, which corresponds to switching the direction of the interval AND negating the endpoint. You likely did this if you did not flip the inequality when dividing by a negative as well as not moving values over to a side properly.
\item \( (-\infty, a], \text{ where } a \in [-0.12, -0.02] \)

 $(-\infty, -0.059]$, which corresponds to negating the endpoint of the solution.
\item \( [a, \infty), \text{ where } a \in [0.02, 0.11] \)

 $[0.059, \infty)$, which corresponds to switching the direction of the interval. You likely did this if you did not flip the inequality when dividing by a negative!
\item \( \text{None of the above}. \)

You may have chosen this if you thought the inequality did not match the ends of the intervals.
\end{enumerate}

\textbf{General Comment:} Remember that less/greater than or equal to includes the endpoint, while less/greater do not. Also, remember that you need to flip the inequality when you multiply or divide by a negative.
}
\litem{
Solve the linear inequality below. Then, choose the constant and interval combination that describes the solution set.
\[ \frac{-7}{5} - \frac{4}{4} x < \frac{3}{3} x + \frac{6}{7} \]The solution is \( (-1.129, \infty) \), which is option D.\begin{enumerate}[label=\Alph*.]
\item \( (-\infty, a), \text{ where } a \in [0.75, 6] \)

 $(-\infty, 1.129)$, which corresponds to switching the direction of the interval AND negating the endpoint. You likely did this if you did not flip the inequality when dividing by a negative as well as not moving values over to a side properly.
\item \( (a, \infty), \text{ where } a \in [0.75, 2.25] \)

 $(1.129, \infty)$, which corresponds to negating the endpoint of the solution.
\item \( (-\infty, a), \text{ where } a \in [-6.75, 0.75] \)

 $(-\infty, -1.129)$, which corresponds to switching the direction of the interval. You likely did this if you did not flip the inequality when dividing by a negative!
\item \( (a, \infty), \text{ where } a \in [-2.25, 0] \)

* $(-1.129, \infty)$, which is the correct option.
\item \( \text{None of the above}. \)

You may have chosen this if you thought the inequality did not match the ends of the intervals.
\end{enumerate}

\textbf{General Comment:} Remember that less/greater than or equal to includes the endpoint, while less/greater do not. Also, remember that you need to flip the inequality when you multiply or divide by a negative.
}
\end{enumerate}

\end{document}