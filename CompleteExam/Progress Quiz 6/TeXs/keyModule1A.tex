\documentclass{extbook}[14pt]
\usepackage{multicol, enumerate, enumitem, hyperref, color, soul, setspace, parskip, fancyhdr, amssymb, amsthm, amsmath, latexsym, units, mathtools}
\everymath{\displaystyle}
\usepackage[headsep=0.5cm,headheight=0cm, left=1 in,right= 1 in,top= 1 in,bottom= 1 in]{geometry}
\usepackage{dashrule}  % Package to use the command below to create lines between items
\newcommand{\litem}[1]{\item #1

\rule{\textwidth}{0.4pt}}
\pagestyle{fancy}
\lhead{}
\chead{Answer Key for Progress Quiz 6 Version A}
\rhead{}
\lfoot{1430-1829}
\cfoot{}
\rfoot{test}
\begin{document}
\textbf{This key should allow you to understand why you choose the option you did (beyond just getting a question right or wrong). \href{https://xronos.clas.ufl.edu/mac1105spring2020/courseDescriptionAndMisc/Exams/LearningFromResults}{More instructions on how to use this key can be found here}.}

\textbf{If you have a suggestion to make the keys better, \href{https://forms.gle/CZkbZmPbC9XALEE88}{please fill out the short survey here}.}

\textit{Note: This key is auto-generated and may contain issues and/or errors. The keys are reviewed after each exam to ensure grading is done accurately. If there are issues (like duplicate options), they are noted in the offline gradebook. The keys are a work-in-progress to give students as many resources to improve as possible.}

\rule{\textwidth}{0.4pt}

\begin{enumerate}\litem{
Choose the \textbf{smallest} set of Complex numbers that the number below belongs to.
\[ \sqrt{\frac{-2640}{0}}+\sqrt{156} \]The solution is \( \text{Not a Complex Number} \), which is option A.\begin{enumerate}[label=\Alph*.]
\item \( \text{Not a Complex Number} \)

* This is the correct option!
\item \( \text{Pure Imaginary} \)

This is a Complex number $(a+bi)$ that \textbf{only} has an imaginary part like $2i$.
\item \( \text{Nonreal Complex} \)

This is a Complex number $(a+bi)$ that is not Real (has $i$ as part of the number).
\item \( \text{Rational} \)

These are numbers that can be written as fraction of Integers (e.g., -2/3 + 5)
\item \( \text{Irrational} \)

These cannot be written as a fraction of Integers. Remember: $\pi$ is not an Integer!
\end{enumerate}

\textbf{General Comment:} Be sure to simplify $i^2 = -1$. This may remove the imaginary portion for your number. If you are having trouble, you may want to look at the \textit{Subgroups of the Real Numbers} section.
}
\litem{
Simplify the expression below into the form $a+bi$. Then, choose the intervals that $a$ and $b$ belong to.
\[ \frac{36 - 22 i}{3 + 8 i} \]The solution is \( -0.93  - 4.85 i \), which is option D.\begin{enumerate}[label=\Alph*.]
\item \( a \in [-68.5, -66.5] \text{ and } b \in [-5.5, -4.5] \)

 $-68.00  - 4.85 i$, which corresponds to forgetting to multiply the conjugate by the numerator and using a plus instead of a minus in the denominator.
\item \( a \in [2.5, 4.5] \text{ and } b \in [2, 3.5] \)

 $3.89  + 3.04 i$, which corresponds to forgetting to multiply the conjugate by the numerator and not computing the conjugate correctly.
\item \( a \in [11, 13] \text{ and } b \in [-3.5, -2] \)

 $12.00  - 2.75 i$, which corresponds to just dividing the first term by the first term and the second by the second.
\item \( a \in [-1.5, -0.5] \text{ and } b \in [-5.5, -4.5] \)

* $-0.93  - 4.85 i$, which is the correct option.
\item \( a \in [-1.5, -0.5] \text{ and } b \in [-354.5, -353] \)

 $-0.93  - 354.00 i$, which corresponds to forgetting to multiply the conjugate by the numerator.
\end{enumerate}

\textbf{General Comment:} Multiply the numerator and denominator by the *conjugate* of the denominator, then simplify. For example, if we have $2+3i$, the conjugate is $2-3i$.
}
\litem{
Simplify the expression below into the form $a+bi$. Then, choose the intervals that $a$ and $b$ belong to.
\[ (-5 + 8 i)(-10 + 3 i) \]The solution is \( 26 - 95 i \), which is option E.\begin{enumerate}[label=\Alph*.]
\item \( a \in [23, 28] \text{ and } b \in [95, 97] \)

 $26 + 95 i$, which corresponds to adding a minus sign in both terms.
\item \( a \in [71, 76] \text{ and } b \in [-70, -62] \)

 $74 - 65 i$, which corresponds to adding a minus sign in the second term.
\item \( a \in [47, 52] \text{ and } b \in [19, 29] \)

 $50 + 24 i$, which corresponds to just multiplying the real terms to get the real part of the solution and the coefficients in the complex terms to get the complex part.
\item \( a \in [71, 76] \text{ and } b \in [64, 67] \)

 $74 + 65 i$, which corresponds to adding a minus sign in the first term.
\item \( a \in [23, 28] \text{ and } b \in [-96, -92] \)

* $26 - 95 i$, which is the correct option.
\end{enumerate}

\textbf{General Comment:} You can treat $i$ as a variable and distribute. Just remember that $i^2=-1$, so you can continue to reduce after you distribute.
}
\litem{
Simplify the expression below into the form $a+bi$. Then, choose the intervals that $a$ and $b$ belong to.
\[ \frac{9 - 44 i}{-2 - 6 i} \]The solution is \( 6.15  + 3.55 i \), which is option D.\begin{enumerate}[label=\Alph*.]
\item \( a \in [-7.5, -6] \text{ and } b \in [0, 1.5] \)

 $-7.05  + 0.85 i$, which corresponds to forgetting to multiply the conjugate by the numerator and not computing the conjugate correctly.
\item \( a \in [-5.5, -4] \text{ and } b \in [7, 8.5] \)

 $-4.50  + 7.33 i$, which corresponds to just dividing the first term by the first term and the second by the second.
\item \( a \in [244.5, 246.5] \text{ and } b \in [1.5, 5.5] \)

 $246.00  + 3.55 i$, which corresponds to forgetting to multiply the conjugate by the numerator and using a plus instead of a minus in the denominator.
\item \( a \in [5, 7] \text{ and } b \in [1.5, 5.5] \)

* $6.15  + 3.55 i$, which is the correct option.
\item \( a \in [5, 7] \text{ and } b \in [141, 142.5] \)

 $6.15  + 142.00 i$, which corresponds to forgetting to multiply the conjugate by the numerator.
\end{enumerate}

\textbf{General Comment:} Multiply the numerator and denominator by the *conjugate* of the denominator, then simplify. For example, if we have $2+3i$, the conjugate is $2-3i$.
}
\litem{
Choose the \textbf{smallest} set of Real numbers that the number below belongs to.
\[ -\sqrt{\frac{196}{529}} \]The solution is \( \text{Rational} \), which is option B.\begin{enumerate}[label=\Alph*.]
\item \( \text{Whole} \)

These are the counting numbers with 0 (0, 1, 2, 3, ...)
\item \( \text{Rational} \)

* This is the correct option!
\item \( \text{Integer} \)

These are the negative and positive counting numbers (..., -3, -2, -1, 0, 1, 2, 3, ...)
\item \( \text{Irrational} \)

These cannot be written as a fraction of Integers.
\item \( \text{Not a Real number} \)

These are Nonreal Complex numbers \textbf{OR} things that are not numbers (e.g., dividing by 0).
\end{enumerate}

\textbf{General Comment:} First, you \textbf{NEED} to simplify the expression. This question simplifies to $-\frac{14}{23}$. 
 
 Be sure you look at the simplified fraction and not just the decimal expansion. Numbers such as 13, 17, and 19 provide \textbf{long but repeating/terminating decimal expansions!} 
 
 The only ways to *not* be a Real number are: dividing by 0 or taking the square root of a negative number. 
 
 Irrational numbers are more than just square root of 3: adding or subtracting values from square root of 3 is also irrational.
}
\litem{
Choose the \textbf{smallest} set of Real numbers that the number below belongs to.
\[ \sqrt{\frac{1232}{7}} \]The solution is \( \text{Irrational} \), which is option E.\begin{enumerate}[label=\Alph*.]
\item \( \text{Integer} \)

These are the negative and positive counting numbers (..., -3, -2, -1, 0, 1, 2, 3, ...)
\item \( \text{Not a Real number} \)

These are Nonreal Complex numbers \textbf{OR} things that are not numbers (e.g., dividing by 0).
\item \( \text{Rational} \)

These are numbers that can be written as fraction of Integers (e.g., -2/3)
\item \( \text{Whole} \)

These are the counting numbers with 0 (0, 1, 2, 3, ...)
\item \( \text{Irrational} \)

* This is the correct option!
\end{enumerate}

\textbf{General Comment:} First, you \textbf{NEED} to simplify the expression. This question simplifies to $\sqrt{176}$. 
 
 Be sure you look at the simplified fraction and not just the decimal expansion. Numbers such as 13, 17, and 19 provide \textbf{long but repeating/terminating decimal expansions!} 
 
 The only ways to *not* be a Real number are: dividing by 0 or taking the square root of a negative number. 
 
 Irrational numbers are more than just square root of 3: adding or subtracting values from square root of 3 is also irrational.
}
\litem{
Simplify the expression below and choose the interval the simplification is contained within.
\[ 19 - 15^2 + 11 \div 16 * 14 \div 7 \]The solution is \( -204.625 \), which is option D.\begin{enumerate}[label=\Alph*.]
\item \( [244.3, 246.2] \)

 245.375, which corresponds to an Order of Operations error: multiplying by negative before squaring. For example: $(-3)^2 \neq -3^2$
\item \( [-206.9, -205.9] \)

 -205.993, which corresponds to an Order of Operations error: not reading left-to-right for multiplication/division.
\item \( [243.2, 244.4] \)

 244.007, which corresponds to two Order of Operations errors.
\item \( [-205.8, -203.4] \)

* -204.625, this is the correct option
\item \( \text{None of the above} \)

 You may have gotten this by making an unanticipated error. If you got a value that is not any of the others, please let the coordinator know so they can help you figure out what happened.
\end{enumerate}

\textbf{General Comment:} While you may remember (or were taught) PEMDAS is done in order, it is actually done as P/E/MD/AS. When we are at MD or AS, we read left to right.
}
\litem{
Choose the \textbf{smallest} set of Complex numbers that the number below belongs to.
\[ \sqrt{\frac{169}{324}} + 25i^2 \]The solution is \( \text{Rational} \), which is option C.\begin{enumerate}[label=\Alph*.]
\item \( \text{Pure Imaginary} \)

This is a Complex number $(a+bi)$ that \textbf{only} has an imaginary part like $2i$.
\item \( \text{Not a Complex Number} \)

This is not a number. The only non-Complex number we know is dividing by 0 as this is not a number!
\item \( \text{Rational} \)

* This is the correct option!
\item \( \text{Irrational} \)

These cannot be written as a fraction of Integers. Remember: $\pi$ is not an Integer!
\item \( \text{Nonreal Complex} \)

This is a Complex number $(a+bi)$ that is not Real (has $i$ as part of the number).
\end{enumerate}

\textbf{General Comment:} Be sure to simplify $i^2 = -1$. This may remove the imaginary portion for your number. If you are having trouble, you may want to look at the \textit{Subgroups of the Real Numbers} section.
}
\litem{
Simplify the expression below into the form $a+bi$. Then, choose the intervals that $a$ and $b$ belong to.
\[ (4 + 9 i)(-3 + 7 i) \]The solution is \( -75 + i \), which is option B.\begin{enumerate}[label=\Alph*.]
\item \( a \in [50, 56] \text{ and } b \in [54.9, 57.2] \)

 $51 + 55 i$, which corresponds to adding a minus sign in the first term.
\item \( a \in [-76, -69] \text{ and } b \in [-0.3, 1.8] \)

* $-75 + i$, which is the correct option.
\item \( a \in [-13, -11] \text{ and } b \in [59.7, 64.9] \)

 $-12 + 63 i$, which corresponds to just multiplying the real terms to get the real part of the solution and the coefficients in the complex terms to get the complex part.
\item \( a \in [50, 56] \text{ and } b \in [-55.5, -52.9] \)

 $51 - 55 i$, which corresponds to adding a minus sign in the second term.
\item \( a \in [-76, -69] \text{ and } b \in [-2.2, 0.4] \)

 $-75 - i$, which corresponds to adding a minus sign in both terms.
\end{enumerate}

\textbf{General Comment:} You can treat $i$ as a variable and distribute. Just remember that $i^2=-1$, so you can continue to reduce after you distribute.
}
\litem{
Simplify the expression below and choose the interval the simplification is contained within.
\[ 19 - 8 \div 10 * 5 - (6 * 12) \]The solution is \( -57.000 \), which is option C.\begin{enumerate}[label=\Alph*.]
\item \( [88.84, 91.84] \)

 90.840, which corresponds to not distributing addition and subtraction correctly.
\item \( [-53.16, -48.16] \)

 -53.160, which corresponds to an Order of Operations error: not reading left-to-right for multiplication/division.
\item \( [-60, -55] \)

* -57.000, which is the correct option.
\item \( [105, 113] \)

 108.000, which corresponds to not distributing a negative correctly.
\item \( \text{None of the above} \)

 You may have gotten this by making an unanticipated error. If you got a value that is not any of the others, please let the coordinator know so they can help you figure out what happened.
\end{enumerate}

\textbf{General Comment:} While you may remember (or were taught) PEMDAS is done in order, it is actually done as P/E/MD/AS. When we are at MD or AS, we read left to right.
}
\end{enumerate}

\end{document}