\documentclass{extbook}[14pt]
\usepackage{multicol, enumerate, enumitem, hyperref, color, soul, setspace, parskip, fancyhdr, amssymb, amsthm, amsmath, latexsym, units, mathtools}
\everymath{\displaystyle}
\usepackage[headsep=0.5cm,headheight=0cm, left=1 in,right= 1 in,top= 1 in,bottom= 1 in]{geometry}
\usepackage{dashrule}  % Package to use the command below to create lines between items
\newcommand{\litem}[1]{\item #1

\rule{\textwidth}{0.4pt}}
\pagestyle{fancy}
\lhead{}
\chead{Answer Key for Progress Quiz 6 Version C}
\rhead{}
\lfoot{1430-1829}
\cfoot{}
\rfoot{test}
\begin{document}
\textbf{This key should allow you to understand why you choose the option you did (beyond just getting a question right or wrong). \href{https://xronos.clas.ufl.edu/mac1105spring2020/courseDescriptionAndMisc/Exams/LearningFromResults}{More instructions on how to use this key can be found here}.}

\textbf{If you have a suggestion to make the keys better, \href{https://forms.gle/CZkbZmPbC9XALEE88}{please fill out the short survey here}.}

\textit{Note: This key is auto-generated and may contain issues and/or errors. The keys are reviewed after each exam to ensure grading is done accurately. If there are issues (like duplicate options), they are noted in the offline gradebook. The keys are a work-in-progress to give students as many resources to improve as possible.}

\rule{\textwidth}{0.4pt}

\begin{enumerate}\litem{
Choose the \textbf{smallest} set of Complex numbers that the number below belongs to.
\[ \frac{0}{10 \pi}+\sqrt{2}i \]The solution is \( \text{Pure Imaginary} \), which is option C.\begin{enumerate}[label=\Alph*.]
\item \( \text{Irrational} \)

These cannot be written as a fraction of Integers. Remember: $\pi$ is not an Integer!
\item \( \text{Not a Complex Number} \)

This is not a number. The only non-Complex number we know is dividing by 0 as this is not a number!
\item \( \text{Pure Imaginary} \)

* This is the correct option!
\item \( \text{Nonreal Complex} \)

This is a Complex number $(a+bi)$ that is not Real (has $i$ as part of the number).
\item \( \text{Rational} \)

These are numbers that can be written as fraction of Integers (e.g., -2/3 + 5)
\end{enumerate}

\textbf{General Comment:} Be sure to simplify $i^2 = -1$. This may remove the imaginary portion for your number. If you are having trouble, you may want to look at the \textit{Subgroups of the Real Numbers} section.
}
\litem{
Simplify the expression below into the form $a+bi$. Then, choose the intervals that $a$ and $b$ belong to.
\[ \frac{9 - 44 i}{-3 - 5 i} \]The solution is \( 5.68  + 5.21 i \), which is option C.\begin{enumerate}[label=\Alph*.]
\item \( a \in [-4, -2.5] \text{ and } b \in [7.5, 9.5] \)

 $-3.00  + 8.80 i$, which corresponds to just dividing the first term by the first term and the second by the second.
\item \( a \in [192.5, 194] \text{ and } b \in [4.5, 6.5] \)

 $193.00  + 5.21 i$, which corresponds to forgetting to multiply the conjugate by the numerator and using a plus instead of a minus in the denominator.
\item \( a \in [4.5, 6] \text{ and } b \in [4.5, 6.5] \)

* $5.68  + 5.21 i$, which is the correct option.
\item \( a \in [4.5, 6] \text{ and } b \in [176, 178] \)

 $5.68  + 177.00 i$, which corresponds to forgetting to multiply the conjugate by the numerator.
\item \( a \in [-8, -7] \text{ and } b \in [1.5, 3.5] \)

 $-7.26  + 2.56 i$, which corresponds to forgetting to multiply the conjugate by the numerator and not computing the conjugate correctly.
\end{enumerate}

\textbf{General Comment:} Multiply the numerator and denominator by the *conjugate* of the denominator, then simplify. For example, if we have $2+3i$, the conjugate is $2-3i$.
}
\litem{
Simplify the expression below into the form $a+bi$. Then, choose the intervals that $a$ and $b$ belong to.
\[ (4 - 9 i)(-7 - 3 i) \]The solution is \( -55 + 51 i \), which is option B.\begin{enumerate}[label=\Alph*.]
\item \( a \in [-4, 5] \text{ and } b \in [73, 81] \)

 $-1 + 75 i$, which corresponds to adding a minus sign in the second term.
\item \( a \in [-57, -47] \text{ and } b \in [45, 52] \)

* $-55 + 51 i$, which is the correct option.
\item \( a \in [-4, 5] \text{ and } b \in [-80, -67] \)

 $-1 - 75 i$, which corresponds to adding a minus sign in the first term.
\item \( a \in [-57, -47] \text{ and } b \in [-53, -50] \)

 $-55 - 51 i$, which corresponds to adding a minus sign in both terms.
\item \( a \in [-35, -26] \text{ and } b \in [27, 31] \)

 $-28 + 27 i$, which corresponds to just multiplying the real terms to get the real part of the solution and the coefficients in the complex terms to get the complex part.
\end{enumerate}

\textbf{General Comment:} You can treat $i$ as a variable and distribute. Just remember that $i^2=-1$, so you can continue to reduce after you distribute.
}
\litem{
Simplify the expression below into the form $a+bi$. Then, choose the intervals that $a$ and $b$ belong to.
\[ \frac{72 + 33 i}{-5 - 6 i} \]The solution is \( -9.15  + 4.38 i \), which is option C.\begin{enumerate}[label=\Alph*.]
\item \( a \in [-559, -557.5] \text{ and } b \in [3.5, 5] \)

 $-558.00  + 4.38 i$, which corresponds to forgetting to multiply the conjugate by the numerator and using a plus instead of a minus in the denominator.
\item \( a \in [-16, -13.5] \text{ and } b \in [-6.5, -5] \)

 $-14.40  - 5.50 i$, which corresponds to just dividing the first term by the first term and the second by the second.
\item \( a \in [-10.5, -8.5] \text{ and } b \in [3.5, 5] \)

* $-9.15  + 4.38 i$, which is the correct option.
\item \( a \in [-10.5, -8.5] \text{ and } b \in [265.5, 268] \)

 $-9.15  + 267.00 i$, which corresponds to forgetting to multiply the conjugate by the numerator.
\item \( a \in [-3.5, -1] \text{ and } b \in [-10, -9.5] \)

 $-2.66  - 9.79 i$, which corresponds to forgetting to multiply the conjugate by the numerator and not computing the conjugate correctly.
\end{enumerate}

\textbf{General Comment:} Multiply the numerator and denominator by the *conjugate* of the denominator, then simplify. For example, if we have $2+3i$, the conjugate is $2-3i$.
}
\litem{
Choose the \textbf{smallest} set of Real numbers that the number below belongs to.
\[ -\sqrt{\frac{36864}{64}} \]The solution is \( \text{Integer} \), which is option B.\begin{enumerate}[label=\Alph*.]
\item \( \text{Not a Real number} \)

These are Nonreal Complex numbers \textbf{OR} things that are not numbers (e.g., dividing by 0).
\item \( \text{Integer} \)

* This is the correct option!
\item \( \text{Rational} \)

These are numbers that can be written as fraction of Integers (e.g., -2/3)
\item \( \text{Irrational} \)

These cannot be written as a fraction of Integers.
\item \( \text{Whole} \)

These are the counting numbers with 0 (0, 1, 2, 3, ...)
\end{enumerate}

\textbf{General Comment:} First, you \textbf{NEED} to simplify the expression. This question simplifies to $-192$. 
 
 Be sure you look at the simplified fraction and not just the decimal expansion. Numbers such as 13, 17, and 19 provide \textbf{long but repeating/terminating decimal expansions!} 
 
 The only ways to *not* be a Real number are: dividing by 0 or taking the square root of a negative number. 
 
 Irrational numbers are more than just square root of 3: adding or subtracting values from square root of 3 is also irrational.
}
\litem{
Choose the \textbf{smallest} set of Real numbers that the number below belongs to.
\[ \sqrt{\frac{20736}{81}} \]The solution is \( \text{Whole} \), which is option B.\begin{enumerate}[label=\Alph*.]
\item \( \text{Irrational} \)

These cannot be written as a fraction of Integers.
\item \( \text{Whole} \)

* This is the correct option!
\item \( \text{Rational} \)

These are numbers that can be written as fraction of Integers (e.g., -2/3)
\item \( \text{Not a Real number} \)

These are Nonreal Complex numbers \textbf{OR} things that are not numbers (e.g., dividing by 0).
\item \( \text{Integer} \)

These are the negative and positive counting numbers (..., -3, -2, -1, 0, 1, 2, 3, ...)
\end{enumerate}

\textbf{General Comment:} First, you \textbf{NEED} to simplify the expression. This question simplifies to $144$. 
 
 Be sure you look at the simplified fraction and not just the decimal expansion. Numbers such as 13, 17, and 19 provide \textbf{long but repeating/terminating decimal expansions!} 
 
 The only ways to *not* be a Real number are: dividing by 0 or taking the square root of a negative number. 
 
 Irrational numbers are more than just square root of 3: adding or subtracting values from square root of 3 is also irrational.
}
\litem{
Simplify the expression below and choose the interval the simplification is contained within.
\[ 20 - 3^2 + 2 \div 4 * 15 \div 13 \]The solution is \( 11.577 \), which is option C.\begin{enumerate}[label=\Alph*.]
\item \( [10, 11.08] \)

 11.003, which corresponds to an Order of Operations error: not reading left-to-right for multiplication/division.
\item \( [29.28, 30.69] \)

 29.577, which corresponds to an Order of Operations error: multiplying by negative before squaring. For example: $(-3)^2 \neq -3^2$
\item \( [11.34, 11.71] \)

* 11.577, this is the correct option
\item \( [28.52, 29.53] \)

 29.003, which corresponds to two Order of Operations errors.
\item \( \text{None of the above} \)

 You may have gotten this by making an unanticipated error. If you got a value that is not any of the others, please let the coordinator know so they can help you figure out what happened.
\end{enumerate}

\textbf{General Comment:} While you may remember (or were taught) PEMDAS is done in order, it is actually done as P/E/MD/AS. When we are at MD or AS, we read left to right.
}
\litem{
Choose the \textbf{smallest} set of Complex numbers that the number below belongs to.
\[ \frac{\sqrt{210}}{7}+9i^2 \]The solution is \( \text{Irrational} \), which is option E.\begin{enumerate}[label=\Alph*.]
\item \( \text{Pure Imaginary} \)

This is a Complex number $(a+bi)$ that \textbf{only} has an imaginary part like $2i$.
\item \( \text{Rational} \)

These are numbers that can be written as fraction of Integers (e.g., -2/3 + 5)
\item \( \text{Not a Complex Number} \)

This is not a number. The only non-Complex number we know is dividing by 0 as this is not a number!
\item \( \text{Nonreal Complex} \)

This is a Complex number $(a+bi)$ that is not Real (has $i$ as part of the number).
\item \( \text{Irrational} \)

* This is the correct option!
\end{enumerate}

\textbf{General Comment:} Be sure to simplify $i^2 = -1$. This may remove the imaginary portion for your number. If you are having trouble, you may want to look at the \textit{Subgroups of the Real Numbers} section.
}
\litem{
Simplify the expression below into the form $a+bi$. Then, choose the intervals that $a$ and $b$ belong to.
\[ (4 + 2 i)(3 + 6 i) \]The solution is \( 0 + 30 i \), which is option E.\begin{enumerate}[label=\Alph*.]
\item \( a \in [22, 25] \text{ and } b \in [18, 20] \)

 $24 + 18 i$, which corresponds to adding a minus sign in the first term.
\item \( a \in [-4, 2] \text{ and } b \in [-32, -29] \)

 $0 - 30 i$, which corresponds to adding a minus sign in both terms.
\item \( a \in [10, 14] \text{ and } b \in [10, 14] \)

 $12 + 12 i$, which corresponds to just multiplying the real terms to get the real part of the solution and the coefficients in the complex terms to get the complex part.
\item \( a \in [22, 25] \text{ and } b \in [-18, -17] \)

 $24 - 18 i$, which corresponds to adding a minus sign in the second term.
\item \( a \in [-4, 2] \text{ and } b \in [26, 39] \)

* $0 + 30 i$, which is the correct option.
\end{enumerate}

\textbf{General Comment:} You can treat $i$ as a variable and distribute. Just remember that $i^2=-1$, so you can continue to reduce after you distribute.
}
\litem{
Simplify the expression below and choose the interval the simplification is contained within.
\[ 3 - 12^2 + 5 \div 19 * 18 \div 15 \]The solution is \( -140.684 \), which is option C.\begin{enumerate}[label=\Alph*.]
\item \( [-141.4, -140.76] \)

 -140.999, which corresponds to an Order of Operations error: not reading left-to-right for multiplication/division.
\item \( [146.4, 147.28] \)

 147.001, which corresponds to two Order of Operations errors.
\item \( [-140.88, -140.28] \)

* -140.684, this is the correct option
\item \( [147.01, 147.75] \)

 147.316, which corresponds to an Order of Operations error: multiplying by negative before squaring. For example: $(-3)^2 \neq -3^2$
\item \( \text{None of the above} \)

 You may have gotten this by making an unanticipated error. If you got a value that is not any of the others, please let the coordinator know so they can help you figure out what happened.
\end{enumerate}

\textbf{General Comment:} While you may remember (or were taught) PEMDAS is done in order, it is actually done as P/E/MD/AS. When we are at MD or AS, we read left to right.
}
\end{enumerate}

\end{document}