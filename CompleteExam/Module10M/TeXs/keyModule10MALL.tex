\documentclass{extbook}[14pt]
\usepackage{multicol, enumerate, enumitem, hyperref, color, soul, setspace, parskip, fancyhdr, amssymb, amsthm, amsmath, latexsym, units, mathtools}
\everymath{\displaystyle}
\usepackage[headsep=0.5cm,headheight=0cm, left=1 in,right= 1 in,top= 1 in,bottom= 1 in]{geometry}
\usepackage{dashrule}  % Package to use the command below to create lines between items
\newcommand{\litem}[1]{\item #1

\rule{\textwidth}{0.4pt}}
\pagestyle{fancy}
\lhead{}
\chead{Answer Key for Module10M Version ALL}
\rhead{}
\lfoot{9912-7625}
\cfoot{}
\rfoot{test}
\begin{document}
\textbf{This key should allow you to understand why you choose the option you did (beyond just getting a question right or wrong). \href{https://xronos.clas.ufl.edu/mac1105spring2020/courseDescriptionAndMisc/Exams/LearningFromResults}{More instructions on how to use this key can be found here}.}

\textbf{If you have a suggestion to make the keys better, \href{https://forms.gle/CZkbZmPbC9XALEE88}{please fill out the short survey here}.}

\textit{Note: This key is auto-generated and may contain issues and/or errors. The keys are reviewed after each exam to ensure grading is done accurately. If there are issues (like duplicate options), they are noted in the offline gradebook. The keys are a work-in-progress to give students as many resources to improve as possible.}

\rule{\textwidth}{0.4pt}

\begin{enumerate}\litem{
For the scenario below, use the model for the volume of a cylinder as $V = \pi r^2 h$ to find the coefficient for the model of the new volume $V_{	ext{new}} = k r^2 h$.

\begin{center}
    \textit{ Pepsi wants to increase the volume of soda in their cans. They've decided to increase the radius by 17 percent and increase the height by 10 percent. They want to model the new volume based on the radius and height of the original cans. }
\end{center}
The solution is \( k = 4.73058 \).\begin{enumerate}[label=\Alph*.]
\textbf{Plausible alternative answers include:}* This is the correct option and corresponds to the model: $V = \pi (1.17 r)^2 (1.10 h)$.
This corresponds to the model: $V = (1.17 r)^2 (1.10 h)$.
This corresponds to the model: $V = \pi (0.17 r)^2 (0.10 h)$.
This corresponds to the model: $V = (0.17 r)^2 (0.10 h)$.
If you chose this, please talk with the coordinator to discuss why you believe none of the options are correct.
\end{enumerate}

\textbf{General Comment:} When calculating the new dimensions, you need to add/subtract from 100\%. For example, a 10\% increase in height would result in 110\% of the original height: $1.1h_{old} = h_{new}$.
}
\litem{
A town has an initial population of 100000. The town's population for the next 9 years is provided below. Which type of function would be most appropriate to model the town's population?


\begin{tabular}{c|c|c|c|c|c|c|c|c|c}
\textbf{Year} &1 &2 &3 &4 &5 &6 &7 &8 &9\tabularnewline \hline
\textbf{Pop} &100000 &99986 &99978 &99972 &99967 &99964 &99961 &99958 &99956\end{tabular}The solution is \( \text{Logarithmic} \).\begin{enumerate}[label=\Alph*.]
\textbf{Plausible alternative answers include:}This suggests a growth faster than constant but slower than exponential.
This suggests the fastest of growths that we know.
This suggests a constant growth. You would be able to add or subtract the same amount year-to-year if this is the correct answer.
This suggests the slowest of growths that we know.
Please contact the coordinator to discuss why you believe none of the options model the population.
\end{enumerate}

\textbf{General Comment:} We are trying to compare the growth rate of the population. Growth rates can be characterized from slowest to fastest as: logarithmic, indirect, linear, direct, exponential. The best way to approach this is to first compare it to linear (is it linear, faster than linear, or slower than linear)? If faster, is it as fast as exponential? If slower, is it as slow as logarithmic?
}
\litem{
What model type would best describe the scenario below?

\begin{center}
    \textit{ Social distancing is a common tactic to counter potential epidemics. This is due to the exponential increase in number of people infected as the density of people living in an area increases. }
\end{center}
The solution is \( \text{None of the above} \).\begin{enumerate}[label=\Alph*.]
\textbf{Plausible alternative answers include:}



\end{enumerate}

\textbf{General Comment:} This is an exponential variation, which grows significantly faster than any power function.
}
\litem{
For the scenario below, model the rate of vibration (cm/s) of the string in terms of the length of the string. Then determine the variation constant $k$ of the model (if possible). The constant should be in terms of cm and s.

\begin{center}
    \textit{ The rate of vibration of a string under constant tension varies based on the type of string and the length of the string. The rate of vibration of string $\omega$ decreases as the square length of the string increases. For example, when string $\omega$ is 5 mm long, the rate of vibration is 24 cm/s. }
\end{center}
The solution is \( k = 6.00 \).\begin{enumerate}[label=\Alph*.]
\textbf{Plausible alternative answers include:}This option uses the model $R = kl^{2}$ as if this is a direct variation AND does not convert from mm to cm so that the units match.
This option uses the model $R = kl^{2}$ as if this is a direct variation.
This option uses the correct model, $R = \frac{k}{l^{2}}$, but does not convert from mm to cm so that the units match.
* This is the correct option, which corresponds to the model $R = \frac{k}{l^{2}}$ AND converts from mm to cm.
Talk with the coordinator if you chose this option.
\end{enumerate}

\textbf{General Comment:} The most common mistake on this question is to not convert mm to cm! When modeling, you need to make sure all of the units for your variables are compatible.
}
\litem{
For the scenario below, use the model for the volume of a cylinder as $V = \pi r^2 h$ to find the coefficient for the model of the new volume $V_{	ext{new}} = k r^2 h$.

\begin{center}
    \textit{ Pepsi wants to increase the volume of soda in their cans. They've decided to decrease the radius by 13 percent and increase the height by 16 percent. They want to model the new volume based on the radius and height of the original cans. }
\end{center}
The solution is \( k = 2.75833 \).\begin{enumerate}[label=\Alph*.]
\textbf{Plausible alternative answers include:}This corresponds to the model: $V = (0.13 r)^2 (0.16 h)$.
This corresponds to the model: $V = (0.87 r)^2 (1.16 h)$.
* This is the correct option and corresponds to the model: $V = \pi (0.87 r)^2 (1.16 h)$.
This corresponds to the model: $V = \pi (0.13 r)^2 (0.16 h)$.
If you chose this, please talk with the coordinator to discuss why you believe none of the options are correct.
\end{enumerate}

\textbf{General Comment:} When calculating the new dimensions, you need to add/subtract from 100\%. For example, a 10\% increase in height would result in 110\% of the original height: $1.1h_{old} = h_{new}$.
}
\litem{
A town has an initial population of 100000. The town's population for the next 9 years is provided below. Which type of function would be most appropriate to model the town's population?


\begin{tabular}{c|c|c|c|c|c|c|c|c|c}
\textbf{Year} &1 &2 &3 &4 &5 &6 &7 &8 &9\tabularnewline \hline
\textbf{Pop} &100018 &100046 &100058 &100086 &100098 &100126 &100138 &100166 &100178\end{tabular}The solution is \( \text{Non-Linear Power} \).\begin{enumerate}[label=\Alph*.]
\textbf{Plausible alternative answers include:}This suggests the fastest of growths that we know.
This suggests the slowest of growths that we know.
This suggests a constant growth. You would be able to add or subtract the same amount year-to-year if this is the correct answer.
This suggests a growth faster than constant but slower than exponential.
Please contact the coordinator to discuss why you believe none of the options model the population.
\end{enumerate}

\textbf{General Comment:} We are trying to compare the growth rate of the population. Growth rates can be characterized from slowest to fastest as: logarithmic, indirect, linear, direct, exponential. The best way to approach this is to first compare it to linear (is it linear, faster than linear, or slower than linear)? If faster, is it as fast as exponential? If slower, is it as slow as logarithmic?
}
\litem{
For the scenario below, model the rate of vibration (cm/s) of the string in terms of the length of the string. Then determine the variation constant $k$ of the model (if possible). The constant should be in terms of cm and s.

\begin{center}
    \textit{ The rate of vibration of a string under constant tension varies based on the type of string and the length of the string. The rate of vibration of string $\omega$ decreases as the cube length of the string increases. For example, when string $\omega$ is 3 mm long, the rate of vibration is 20 cm/s. }
\end{center}
The solution is \( k = 0.54 \).\begin{enumerate}[label=\Alph*.]
\textbf{Plausible alternative answers include:}This option uses the model $R = kl^{3}$ as if this is a direct variation AND does not convert from mm to cm so that the units match.
This option uses the model $R = kl^{3}$ as if this is a direct variation.
This option uses the correct model, $R = \frac{k}{l^{3}}$, but does not convert from mm to cm so that the units match.
* This is the correct option, which corresponds to the model $R = \frac{k}{l^{3}}$ AND converts from mm to cm.
Talk with the coordinator if you chose this option.
\end{enumerate}

\textbf{General Comment:} The most common mistake on this question is to not convert mm to cm! When modeling, you need to make sure all of the units for your variables are compatible.
}
\litem{
What model type would best describe the scenario below?

\begin{center}
    \textit{ Social distancing is a common tactic to counter potential epidemics. This is due to the exponential increase in number of people infected as the density of people living in an area increases. }
\end{center}
The solution is \( \text{None of the above} \).\begin{enumerate}[label=\Alph*.]
\textbf{Plausible alternative answers include:}



\end{enumerate}

\textbf{General Comment:} This is an exponential variation, which grows significantly faster than any power function.
}
\litem{
For the scenario below, find the variation constant $k$ of the model (if possible).

\begin{center}
    \textit{ In an alternative galaxy, the cube of the time, $T$ (Earth years), required for a planet to orbit Sun $\chi$ increases as the cube of the distance, $d$ (AUs), that the planet is from Sun $\chi$ increases. For example, when Ea's average distance from Sun $\chi$ is 2, it takes 82 Earth days to complete an orbit. }
\end{center}
The solution is \( k = 68921.000 \).\begin{enumerate}[label=\Alph*.]
\textbf{Plausible alternative answers include:}This corresponds to the model $T^{1/3} = k d^{1/3}$.
This copies the constant used in the homework.
* This is the correct option corresponding to the model $T^{3} = k d^{3}$.
This corresponds to the model $T^{3} = \frac{k}{d^{3}}$.
This corresponds to believing you cannot determine the type of model from the information given.
\end{enumerate}

\textbf{General Comment:} Since $T$ increases proportionally as $d$ increases, we know this is a direct variation model.
}
\litem{
For the scenario below, find the variation constant $k$ of the model (if possible).

\begin{center}
    \textit{ In an alternative galaxy, the cube of the time, $T$ (Earth years), required for a planet to orbit Sun $\chi$ decreases as the cube of the distance, $d$ (AUs), that the planet is from Sun $\chi$ decreases. For example, when Ea's average distance from Sun $\chi$ is 4, it takes 67 Earth days to complete an orbit. }
\end{center}
The solution is \( k = 4699.422 \).\begin{enumerate}[label=\Alph*.]
\textbf{Plausible alternative answers include:}This copies the constant used in the homework.
This corresponds to the model $T^{1/3} = k d^{1/3}$.
* This is the correct option corresponding to the model $T^{3} = k d^{3}$.
This corresponds to the model $T^{3} = \frac{k}{d^{3}}$.
This corresponds to believing you cannot determine the type of model from the information given.
\end{enumerate}

\textbf{General Comment:} Since $T$ decreases proportionally as $d$ decreases, we know this is a direct variation model.
}
\litem{
For the scenario below, use the model for the volume of a cylinder as $V = \pi r^2 h$ to find the coefficient for the model of the new volume $V_{	ext{new}} = k r^2 h$.

\begin{center}
    \textit{ Pepsi wants to increase the volume of soda in their cans. They've decided to increase the radius by 19 percent and increase the height by 12 percent. They want to model the new volume based on the radius and height of the original cans. }
\end{center}
The solution is \( k = 4.98267 \).\begin{enumerate}[label=\Alph*.]
\textbf{Plausible alternative answers include:}This corresponds to the model: $V = (0.19 r)^2 (0.12 h)$.
This corresponds to the model: $V = \pi (0.19 r)^2 (0.12 h)$.
* This is the correct option and corresponds to the model: $V = \pi (1.19 r)^2 (1.12 h)$.
This corresponds to the model: $V = (1.19 r)^2 (1.12 h)$.
If you chose this, please talk with the coordinator to discuss why you believe none of the options are correct.
\end{enumerate}

\textbf{General Comment:} When calculating the new dimensions, you need to add/subtract from 100\%. For example, a 10\% increase in height would result in 110\% of the original height: $1.1h_{old} = h_{new}$.
}
\litem{
A town has an initial population of 80000. The town's population for the next 9 years is provided below. Which type of function would be most appropriate to model the town's population?


\begin{tabular}{c|c|c|c|c|c|c|c|c|c}
\textbf{Year} &1 &2 &3 &4 &5 &6 &7 &8 &9\tabularnewline \hline
\textbf{Pop} &80017 &80045 &80057 &80085 &80097 &80125 &80137 &80165 &80177\end{tabular}The solution is \( \text{Non-Linear Power} \).\begin{enumerate}[label=\Alph*.]
\textbf{Plausible alternative answers include:}This suggests the fastest of growths that we know.
This suggests a growth faster than constant but slower than exponential.
This suggests a constant growth. You would be able to add or subtract the same amount year-to-year if this is the correct answer.
This suggests the slowest of growths that we know.
Please contact the coordinator to discuss why you believe none of the options model the population.
\end{enumerate}

\textbf{General Comment:} We are trying to compare the growth rate of the population. Growth rates can be characterized from slowest to fastest as: logarithmic, indirect, linear, direct, exponential. The best way to approach this is to first compare it to linear (is it linear, faster than linear, or slower than linear)? If faster, is it as fast as exponential? If slower, is it as slow as logarithmic?
}
\litem{
What model type would best describe the scenario below?

\begin{center}
    \textit{ Big O notation is common in computer science to describe how fast a program can solve a particular problem. Big O notation categorizes functions according to their growth rates, the same way we have categorized modeling real-world problems by certain types of functions. When analyzing a particular program, a student found the computer to need $x^x$ time to complete, where $x$ was the number of inputs into the program. }
\end{center}
The solution is \( \text{None of the above} \).\begin{enumerate}[label=\Alph*.]
\textbf{Plausible alternative answers include:}



\end{enumerate}

\textbf{General Comment:} We have been modeling real-world problems according to the growth rates of functions. So far, we've seen logarithmics to be the slowest, then power functions, then exponentials as the fastest. But, there are \textbf{far more types of functions than the ones we've looked at}! One such function is $x^x$, also known as a power tower. This function class grows significantly faster than exponentials. Remember for power variation, we need the exponent to be a constant.
}
\litem{
For the scenario below, model the rate of vibration (cm/s) of the string in terms of the length of the string. Then determine the variation constant $k$ of the model (if possible). The constant should be in terms of cm and s.

\begin{center}
    \textit{ The rate of vibration of a string under constant tension varies based on the type of string and the length of the string. The rate of vibration of string $\omega$ decreases as the quartic length of the string increases. For example, when string $\omega$ is 5 mm long, the rate of vibration is 21 cm/s. }
\end{center}
The solution is \( k = 1.31 \).\begin{enumerate}[label=\Alph*.]
\textbf{Plausible alternative answers include:}This option uses the model $R = kl^{4}$ as if this is a direct variation AND does not convert from mm to cm so that the units match.
This option uses the correct model, $R = \frac{k}{l^{4}}$, but does not convert from mm to cm so that the units match.
This option uses the model $R = kl^{4}$ as if this is a direct variation.
* This is the correct option, which corresponds to the model $R = \frac{k}{l^{4}}$ AND converts from mm to cm.
Talk with the coordinator if you chose this option.
\end{enumerate}

\textbf{General Comment:} The most common mistake on this question is to not convert mm to cm! When modeling, you need to make sure all of the units for your variables are compatible.
}
\litem{
For the scenario below, use the model for the volume of a cylinder as $V = \pi r^2 h$ to find the coefficient for the model of the new volume $V_{	ext{new}} = k r^2 h$.

\begin{center}
    \textit{ Pepsi wants to increase the volume of soda in their cans. They've decided to decrease the radius by 11 percent and increase the height by 18 percent. They want to model the new volume based on the radius and height of the original cans. }
\end{center}
The solution is \( k = 2.93638 \).\begin{enumerate}[label=\Alph*.]
\textbf{Plausible alternative answers include:}This corresponds to the model: $V = (0.89 r)^2 (1.18 h)$.
* This is the correct option and corresponds to the model: $V = \pi (0.89 r)^2 (1.18 h)$.
This corresponds to the model: $V = (0.11 r)^2 (0.18 h)$.
This corresponds to the model: $V = \pi (0.11 r)^2 (0.18 h)$.
If you chose this, please talk with the coordinator to discuss why you believe none of the options are correct.
\end{enumerate}

\textbf{General Comment:} When calculating the new dimensions, you need to add/subtract from 100\%. For example, a 10\% increase in height would result in 110\% of the original height: $1.1h_{old} = h_{new}$.
}
\litem{
A town has an initial population of 80000. The town's population for the next 9 years is provided below. Which type of function would be most appropriate to model the town's population?


\begin{tabular}{c|c|c|c|c|c|c|c|c|c}
\textbf{Year} &1 &2 &3 &4 &5 &6 &7 &8 &9\tabularnewline \hline
\textbf{Pop} &80000 &80013 &80021 &80027 &80032 &80035 &80038 &80041 &80043\end{tabular}The solution is \( \text{Logarithmic} \).\begin{enumerate}[label=\Alph*.]
\textbf{Plausible alternative answers include:}This suggests the fastest of growths that we know.
This suggests the slowest of growths that we know.
This suggests a constant growth. You would be able to add or subtract the same amount year-to-year if this is the correct answer.
This suggests a growth faster than constant but slower than exponential.
Please contact the coordinator to discuss why you believe none of the options model the population.
\end{enumerate}

\textbf{General Comment:} We are trying to compare the growth rate of the population. Growth rates can be characterized from slowest to fastest as: logarithmic, indirect, linear, direct, exponential. The best way to approach this is to first compare it to linear (is it linear, faster than linear, or slower than linear)? If faster, is it as fast as exponential? If slower, is it as slow as logarithmic?
}
\litem{
For the scenario below, model the rate of vibration (cm/s) of the string in terms of the length of the string. Then determine the variation constant $k$ of the model (if possible). The constant should be in terms of cm and s.

\begin{center}
    \textit{ The rate of vibration of a string under constant tension varies based on the type of string and the length of the string. The rate of vibration of string $\omega$ decreases as the quartic length of the string increases. For example, when string $\omega$ is 4 mm long, the rate of vibration is 30 cm/s. }
\end{center}
The solution is \( k = 0.77 \).\begin{enumerate}[label=\Alph*.]
\textbf{Plausible alternative answers include:}This option uses the model $R = kl^{4}$ as if this is a direct variation AND does not convert from mm to cm so that the units match.
This option uses the correct model, $R = \frac{k}{l^{4}}$, but does not convert from mm to cm so that the units match.
* This is the correct option, which corresponds to the model $R = \frac{k}{l^{4}}$ AND converts from mm to cm.
This option uses the model $R = kl^{4}$ as if this is a direct variation.
Talk with the coordinator if you chose this option.
\end{enumerate}

\textbf{General Comment:} The most common mistake on this question is to not convert mm to cm! When modeling, you need to make sure all of the units for your variables are compatible.
}
\litem{
What model type would best describe the scenario below?

\begin{center}
    \textit{ In economics, there are two common equations to model interest earned. The compound interest formula is $A = P (1 + \frac{r}{n})^{nt}$, where $A$ is the amount of money you end up with, $P$ is your starting money, $r$ is the interest rate, $n$ is the number of times compounded in a year, and $t$ is the total number of years. For example, if you were a parent and wanted to save \$10,000 in 3 years-time at 3.5\% interest compounded monthly, you would need to invest about \$9,000. }
\end{center}
The solution is \( \text{None of the above} \).\begin{enumerate}[label=\Alph*.]
\textbf{Plausible alternative answers include:}



\end{enumerate}

\textbf{General Comment:} When thinking about power functions, we want the exponent to be constant and the base to be a variable (or variables). In this case, we see variables in the exponent, which tips us off that this is not a power variation.
}
\litem{
For the scenario below, find the variation constant $k$ of the model (if possible).

\begin{center}
    \textit{ In an alternative galaxy, the cube of the time, $T$ (Earth years), required for a planet to orbit Sun $\chi$ increases as the square of the distance, $d$ (AUs), that the planet is from Sun $\chi$ increases. For example, when Ea's average distance from Sun $\chi$ is 7, it takes 56 Earth days to complete an orbit. }
\end{center}
The solution is \( k = 3584.000 \).\begin{enumerate}[label=\Alph*.]
\textbf{Plausible alternative answers include:}This corresponds to the model $T^{3} = \frac{k}{d^{2}}$.
This copies the constant used in the homework.
* This is the correct option corresponding to the model $T^{3} = k d^{2}$.
This corresponds to the model $T^{1/3} = k d^{1/2}$.
This corresponds to believing you cannot determine the type of model from the information given.
\end{enumerate}

\textbf{General Comment:} Since $T$ increases proportionally as $d$ increases, we know this is a direct variation model.
}
\litem{
For the scenario below, find the variation constant $k$ of the model (if possible).

\begin{center}
    \textit{ In an alternative galaxy, the quartic of the time, $T$ (Earth years), required for a planet to orbit Sun $\chi$ increases as the quartic of the distance, $d$ (AUs), that the planet is from Sun $\chi$ increases. For example, when Ea's average distance from Sun $\chi$ is 3, it takes 95 Earth days to complete an orbit. }
\end{center}
The solution is \( k = 1005563.272 \).\begin{enumerate}[label=\Alph*.]
\textbf{Plausible alternative answers include:}This copies the constant used in the homework.
This corresponds to the model $T^{4} = \frac{k}{d^{4}}$.
* This is the correct option corresponding to the model $T^{4} = k d^{4}$.
This corresponds to the model $T^{1/4} = k d^{1/4}$.
This corresponds to believing you cannot determine the type of model from the information given.
\end{enumerate}

\textbf{General Comment:} Since $T$ increases proportionally as $d$ increases, we know this is a direct variation model.
}
\litem{
For the scenario below, use the model for the volume of a cylinder as $V = \pi r^2 h$ to find the coefficient for the model of the new volume $V_{	ext{new}} = k r^2 h$.

\begin{center}
    \textit{ Pepsi wants to increase the volume of soda in their cans. They've decided to decrease the radius by 20 percent and decrease the height by 12 percent. They want to model the new volume based on the radius and height of the original cans. }
\end{center}
The solution is \( k = 1.76934 \).\begin{enumerate}[label=\Alph*.]
\textbf{Plausible alternative answers include:}This corresponds to the model: $V = (0.20 r)^2 (0.12 h)$.
This corresponds to the model: $V = \pi (0.20 r)^2 (0.12 h)$.
This corresponds to the model: $V = (0.80 r)^2 (0.88 h)$.
* This is the correct option and corresponds to the model: $V = \pi (0.80 r)^2 (0.88 h)$.
If you chose this, please talk with the coordinator to discuss why you believe none of the options are correct.
\end{enumerate}

\textbf{General Comment:} When calculating the new dimensions, you need to add/subtract from 100\%. For example, a 10\% increase in height would result in 110\% of the original height: $1.1h_{old} = h_{new}$.
}
\litem{
A town has an initial population of 90000. The town's population for the next 9 years is provided below. Which type of function would be most appropriate to model the town's population?


\begin{tabular}{c|c|c|c|c|c|c|c|c|c}
\textbf{Year} &1 &2 &3 &4 &5 &6 &7 &8 &9\tabularnewline \hline
\textbf{Pop} &90000 &90034 &90054 &90069 &90080 &90089 &90097 &90103 &90109\end{tabular}The solution is \( \text{Logarithmic} \).\begin{enumerate}[label=\Alph*.]
\textbf{Plausible alternative answers include:}This suggests the fastest of growths that we know.
This suggests the slowest of growths that we know.
This suggests a growth faster than constant but slower than exponential.
This suggests a constant growth. You would be able to add or subtract the same amount year-to-year if this is the correct answer.
Please contact the coordinator to discuss why you believe none of the options model the population.
\end{enumerate}

\textbf{General Comment:} We are trying to compare the growth rate of the population. Growth rates can be characterized from slowest to fastest as: logarithmic, indirect, linear, direct, exponential. The best way to approach this is to first compare it to linear (is it linear, faster than linear, or slower than linear)? If faster, is it as fast as exponential? If slower, is it as slow as logarithmic?
}
\litem{
What model type would best describe the scenario below?

\begin{center}
    \textit{ In economics, there are two common equations to model interest earned. The compound interest formula is $A = P (1 + \frac{r}{n})^{nt}$, where $A$ is the amount of money you end up with, $P$ is your starting money, $r$ is the interest rate, $n$ is the number of times compounded in a year, and $t$ is the total number of years. For example, if you were a parent and wanted to save \$10,000 in 3 years-time at 3.5\% interest compounded monthly, you would need to invest about \$9,000. }
\end{center}
The solution is \( \text{None of the above} \).\begin{enumerate}[label=\Alph*.]
\textbf{Plausible alternative answers include:}



\end{enumerate}

\textbf{General Comment:} When thinking about power functions, we want the exponent to be constant and the base to be a variable (or variables). In this case, we see variables in the exponent, which tips us off that this is not a power variation.
}
\litem{
For the scenario below, model the rate of vibration (cm/s) of the string in terms of the length of the string. Then determine the variation constant $k$ of the model (if possible). The constant should be in terms of cm and s.

\begin{center}
    \textit{ The rate of vibration of a string under constant tension varies based on the type of string and the length of the string. The rate of vibration of string $\omega$ increases as the cube length of the string decreases. For example, when string $\omega$ is 3 mm long, the rate of vibration is 31 cm/s. }
\end{center}
The solution is \( k = 0.84 \).\begin{enumerate}[label=\Alph*.]
\textbf{Plausible alternative answers include:}* This is the correct option, which corresponds to the model $R = \frac{k}{l^{3}}$ AND converts from mm to cm.
This option uses the model $R = kl^{3}$ as if this is a direct variation AND does not convert from mm to cm so that the units match.
This option uses the model $R = kl^{3}$ as if this is a direct variation.
This option uses the correct model, $R = \frac{k}{l^{3}}$, but does not convert from mm to cm so that the units match.
Talk with the coordinator if you chose this option.
\end{enumerate}

\textbf{General Comment:} The most common mistake on this question is to not convert mm to cm! When modeling, you need to make sure all of the units for your variables are compatible.
}
\litem{
For the scenario below, use the model for the volume of a cylinder as $V = \pi r^2 h$ to find the coefficient for the model of the new volume $V_{	ext{new}} = k r^2 h$.

\begin{center}
    \textit{ Pepsi wants to increase the volume of soda in their cans. They've decided to increase the radius by 16 percent and increase the height by 11 percent. They want to model the new volume based on the radius and height of the original cans. }
\end{center}
The solution is \( k = 4.69233 \).\begin{enumerate}[label=\Alph*.]
\textbf{Plausible alternative answers include:}* This is the correct option and corresponds to the model: $V = \pi (1.16 r)^2 (1.11 h)$.
This corresponds to the model: $V = (0.16 r)^2 (0.11 h)$.
This corresponds to the model: $V = (1.16 r)^2 (1.11 h)$.
This corresponds to the model: $V = \pi (0.16 r)^2 (0.11 h)$.
If you chose this, please talk with the coordinator to discuss why you believe none of the options are correct.
\end{enumerate}

\textbf{General Comment:} When calculating the new dimensions, you need to add/subtract from 100\%. For example, a 10\% increase in height would result in 110\% of the original height: $1.1h_{old} = h_{new}$.
}
\litem{
A town has an initial population of 20000. The town's population for the next 9 years is provided below. Which type of function would be most appropriate to model the town's population?


\begin{tabular}{c|c|c|c|c|c|c|c|c|c}
\textbf{Year} &1 &2 &3 &4 &5 &6 &7 &8 &9\tabularnewline \hline
\textbf{Pop} &20000 &19972 &19956 &19944 &19935 &19928 &19922 &19916 &19912\end{tabular}The solution is \( \text{Logarithmic} \).\begin{enumerate}[label=\Alph*.]
\textbf{Plausible alternative answers include:}This suggests a constant growth. You would be able to add or subtract the same amount year-to-year if this is the correct answer.
This suggests a growth faster than constant but slower than exponential.
This suggests the fastest of growths that we know.
This suggests the slowest of growths that we know.
Please contact the coordinator to discuss why you believe none of the options model the population.
\end{enumerate}

\textbf{General Comment:} We are trying to compare the growth rate of the population. Growth rates can be characterized from slowest to fastest as: logarithmic, indirect, linear, direct, exponential. The best way to approach this is to first compare it to linear (is it linear, faster than linear, or slower than linear)? If faster, is it as fast as exponential? If slower, is it as slow as logarithmic?
}
\litem{
For the scenario below, model the rate of vibration (cm/s) of the string in terms of the length of the string. Then determine the variation constant $k$ of the model (if possible). The constant should be in terms of cm and s.

\begin{center}
    \textit{ The rate of vibration of a string under constant tension varies based on the type of string and the length of the string. The rate of vibration of string $\omega$ increases as the square length of the string decreases. For example, when string $\omega$ is 5 mm long, the rate of vibration is 22 cm/s. }
\end{center}
The solution is \( k = 5.50 \).\begin{enumerate}[label=\Alph*.]
\textbf{Plausible alternative answers include:}* This is the correct option, which corresponds to the model $R = \frac{k}{l^{2}}$ AND converts from mm to cm.
This option uses the model $R = kl^{2}$ as if this is a direct variation AND does not convert from mm to cm so that the units match.
This option uses the correct model, $R = \frac{k}{l^{2}}$, but does not convert from mm to cm so that the units match.
This option uses the model $R = kl^{2}$ as if this is a direct variation.
Talk with the coordinator if you chose this option.
\end{enumerate}

\textbf{General Comment:} The most common mistake on this question is to not convert mm to cm! When modeling, you need to make sure all of the units for your variables are compatible.
}
\litem{
What model type would best describe the scenario below?

\begin{center}
    \textit{ Big O notation is common in computer science to describe how fast a program can solve a particular problem. Big O notation categorizes functions according to their growth rates, the same way we have categorized modeling real-world problems by certain types of functions. When analyzing a particular program, a student found the computer to need $x^x$ time to complete, where $x$ was the number of inputs into the program. }
\end{center}
The solution is \( \text{None of the above} \).\begin{enumerate}[label=\Alph*.]
\textbf{Plausible alternative answers include:}



\end{enumerate}

\textbf{General Comment:} We have been modeling real-world problems according to the growth rates of functions. So far, we've seen logarithmics to be the slowest, then power functions, then exponentials as the fastest. But, there are \textbf{far more types of functions than the ones we've looked at}! One such function is $x^x$, also known as a power tower. This function class grows significantly faster than exponentials. Remember for power variation, we need the exponent to be a constant.
}
\litem{
For the scenario below, find the variation constant $k$ of the model (if possible).

\begin{center}
    \textit{ In an alternative galaxy, the quartic of the time, $T$ (Earth years), required for a planet to orbit Sun $\chi$ increases as the square of the distance, $d$ (AUs), that the planet is from Sun $\chi$ increases. For example, when Ea's average distance from Sun $\chi$ is 2, it takes 91 Earth days to complete an orbit. }
\end{center}
The solution is \( k = 17143740.250 \).\begin{enumerate}[label=\Alph*.]
\textbf{Plausible alternative answers include:}This copies the constant used in the homework.
This corresponds to the model $T^{1/4} = k d^{1/2}$.
* This is the correct option corresponding to the model $T^{4} = k d^{2}$.
This corresponds to the model $T^{4} = \frac{k}{d^{2}}$.
This corresponds to believing you cannot determine the type of model from the information given.
\end{enumerate}

\textbf{General Comment:} Since $T$ increases proportionally as $d$ increases, we know this is a direct variation model.
}
\litem{
For the scenario below, find the variation constant $k$ of the model (if possible).

\begin{center}
    \textit{ In an alternative galaxy, the square of the time, $T$ (Earth years), required for a planet to orbit Sun $\chi$ increases as the quartic of the distance, $d$ (AUs), that the planet is from Sun $\chi$ increases. For example, when Ea's average distance from Sun $\chi$ is 2, it takes 97 Earth days to complete an orbit. }
\end{center}
The solution is \( k = 588.062 \).\begin{enumerate}[label=\Alph*.]
\textbf{Plausible alternative answers include:}This copies the constant used in the homework.
* This is the correct option corresponding to the model $T^{2} = k d^{4}$.
This corresponds to the model $T^{2} = \frac{k}{d^{4}}$.
This corresponds to the model $T^{1/2} = k d^{1/4}$.
This corresponds to believing you cannot determine the type of model from the information given.
\end{enumerate}

\textbf{General Comment:} Since $T$ increases proportionally as $d$ increases, we know this is a direct variation model.
}
\end{enumerate}

\end{document}