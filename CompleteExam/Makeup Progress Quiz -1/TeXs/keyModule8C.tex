\documentclass{extbook}[14pt]
\usepackage{multicol, enumerate, enumitem, hyperref, color, soul, setspace, parskip, fancyhdr, amssymb, amsthm, amsmath, bbm, latexsym, units, mathtools}
\everymath{\displaystyle}
\usepackage[headsep=0.5cm,headheight=0cm, left=1 in,right= 1 in,top= 1 in,bottom= 1 in]{geometry}
\usepackage{dashrule}  % Package to use the command below to create lines between items
\newcommand{\litem}[1]{\item #1

\rule{\textwidth}{0.4pt}}
\pagestyle{fancy}
\lhead{}
\chead{Answer Key for Makeup Progress Quiz -1 Version C}
\rhead{}
\lfoot{7547-2949}
\cfoot{}
\rfoot{Fall 2020}
\begin{document}
\textbf{This key should allow you to understand why you choose the option you did (beyond just getting a question right or wrong). \href{https://xronos.clas.ufl.edu/mac1105spring2020/courseDescriptionAndMisc/Exams/LearningFromResults}{More instructions on how to use this key can be found here}.}

\textbf{If you have a suggestion to make the keys better, \href{https://forms.gle/CZkbZmPbC9XALEE88}{please fill out the short survey here}.}

\textit{Note: This key is auto-generated and may contain issues and/or errors. The keys are reviewed after each exam to ensure grading is done accurately. If there are issues (like duplicate options), they are noted in the offline gradebook. The keys are a work-in-progress to give students as many resources to improve as possible.}

\rule{\textwidth}{0.4pt}

\begin{enumerate}\litem{
Solve the equation for $x$ and choose the interval that contains the solution (if it exists).
\[ \log_{2}{(-2x+6)}+4 = 2 \]

The solution is \( x = 2.875 \), which is option B.\begin{enumerate}[label=\Alph*.]
\item \( x \in [-2, 1.1] \)

$x = 1.000$, which corresponds to reversing the base and exponent when converting.
\item \( x \in [2.1, 4.2] \)

* $x = 2.875$, which is the correct option.
\item \( x \in [-2, 1.1] \)

$x = 1.000$, which corresponds to ignoring the vertical shift when converting to exponential form.
\item \( x \in [-9.2, -4.7] \)

$x = -5.000$, which corresponds to reversing the base and exponent when converting and reversing the value with $x$.
\item \( \text{There is no Real solution to the equation.} \)

Corresponds to believing a negative coefficient within the log equation means there is no Real solution.
\end{enumerate}

\textbf{General Comment:} \textbf{General Comments:} First, get the equation in the form $\log_b{(cx+d)} = a$. Then, convert to $b^a = cx+d$ and solve.
}
\litem{
 Solve the equation for $x$ and choose the interval that contains $x$ (if it exists).
\[  24 = \sqrt[6]{\frac{7}{e^{3x}}} \]

The solution is \( x = -5.707 \), which is option A.\begin{enumerate}[label=\Alph*.]
\item \( x \in [-7.71, -3.71] \)

* $x = -5.707$, which is the correct option.
\item \( x \in [-49.65, -47.65] \)

$x = -48.649$, which corresponds to thinking you don't need to take the natural log of both sides before reducing, as if the equation already had a natural log on the right side.
\item \( x \in [-4.47, -0.47] \)

$x = -1.470$, which corresponds to treating any root as a square root.
\item \( \text{There is no Real solution to the equation.} \)

This corresponds to believing you cannot solve the equation.
\item \( \text{None of the above.} \)

This corresponds to making an unexpected error.
\end{enumerate}

\textbf{General Comment:} \textbf{General Comments}: After using the properties of logarithmic functions to break up the right-hand side, use $\ln(e) = 1$ to reduce the question to a linear function to solve. You can put $\ln(7)$ into a calculator if you are having trouble.
}
\litem{
Which of the following intervals describes the Range of the function below?
\[ f(x) = -e^{x-5}-4 \]

The solution is \( (-\infty, -4) \), which is option A.\begin{enumerate}[label=\Alph*.]
\item \( (-\infty, a), a \in [-4, 0] \)

* $(-\infty, -4)$, which is the correct option.
\item \( (-\infty, a], a \in [-4, 0] \)

$(-\infty, -4]$, which corresponds to including the endpoint.
\item \( (a, \infty), a \in [0, 6] \)

$(4, \infty)$, which corresponds to using the negative vertical shift AND flipping the Range interval.
\item \( [a, \infty), a \in [0, 6] \)

$[4, \infty)$, which corresponds to using the negative vertical shift AND flipping the Range interval AND including the endpoint.
\item \( (-\infty, \infty) \)

This corresponds to confusing range of an exponential function with the domain of an exponential function.
\end{enumerate}

\textbf{General Comment:} \textbf{General Comments}: Domain of a basic exponential function is $(-\infty, \infty)$ while the Range is $(0, \infty)$. We can shift these intervals [and even flip when $a<0$!] to find the new Domain/Range.
}
\litem{
Which of the following intervals describes the Range of the function below?
\[ f(x) = -e^{x+4}+8 \]

The solution is \( (-\infty, 8) \), which is option B.\begin{enumerate}[label=\Alph*.]
\item \( (-\infty, a], a \in [7, 9] \)

$(-\infty, 8]$, which corresponds to including the endpoint.
\item \( (-\infty, a), a \in [7, 9] \)

* $(-\infty, 8)$, which is the correct option.
\item \( (a, \infty), a \in [-10, -4] \)

$(-8, \infty)$, which corresponds to using the negative vertical shift AND flipping the Range interval.
\item \( [a, \infty), a \in [-10, -4] \)

$[-8, \infty)$, which corresponds to using the negative vertical shift AND flipping the Range interval AND including the endpoint.
\item \( (-\infty, \infty) \)

This corresponds to confusing range of an exponential function with the domain of an exponential function.
\end{enumerate}

\textbf{General Comment:} \textbf{General Comments}: Domain of a basic exponential function is $(-\infty, \infty)$ while the Range is $(0, \infty)$. We can shift these intervals [and even flip when $a<0$!] to find the new Domain/Range.
}
\litem{
Which of the following intervals describes the Range of the function below?
\[ f(x) = \log_2{(x-4)}+6 \]

The solution is \( (\infty, \infty) \), which is option E.\begin{enumerate}[label=\Alph*.]
\item \( (-\infty, a), a \in [-6.73, -5.86] \)

$(-\infty, -6)$, which corresponds to using the using the negative of vertical shift on $(0, \infty)$.
\item \( [a, \infty), a \in [-5.19, -2.95] \)

$[-4, \infty)$, which corresponds to using the negative of the horizontal shift AND including the endpoint.
\item \( (-\infty, a), a \in [5.52, 7.4] \)

$(-\infty, 6)$, which corresponds to using the vertical shift while the Range is $(-\infty, \infty)$.
\item \( [a, \infty), a \in [3.35, 4.8] \)

$[6, \infty)$, which corresponds to using the flipped Domain AND including the endpoint.
\item \( (-\infty, \infty) \)

*This is the correct option.
\end{enumerate}

\textbf{General Comment:} \textbf{General Comments}: The domain of a basic logarithmic function is $(0, \infty)$ and the Range is $(-\infty, \infty)$. We can use shifts when finding the Domain, but the Range will always be all Real numbers.
}
\litem{
 Solve the equation for $x$ and choose the interval that contains $x$ (if it exists).
\[  9 = \ln{\sqrt[6]{\frac{12}{e^{5x}}}} \]

The solution is \( x = -10.303, \text{ which does not fit in any of the interval options.} \), which is option E.\begin{enumerate}[label=\Alph*.]
\item \( x \in [-3.13, -3.09] \)

$x = -3.103$, which corresponds to treating any root as a square root.
\item \( x \in [10.28, 10.33] \)

$x = 10.303$, which is the negative of the correct solution.
\item \( x \in [-3.15, -3.13] \)

$x = -3.134$, which corresponds to thinking you need to take the natural log of the left side before reducing.
\item \( \text{There is no Real solution to the equation.} \)

This corresponds to believing you cannot solve the equation.
\item \( \text{None of the above.} \)

*$x = -10.303$ is the correct solution and does not fit in any of the other intervals.
\end{enumerate}

\textbf{General Comment:} \textbf{General Comments}: After using the properties of logarithmic functions to break up the right-hand side, use $\ln(e) = 1$ to reduce the question to a linear function to solve. You can put $\ln(12)$ into a calculator if you are having trouble.
}
\litem{
Which of the following intervals describes the Range of the function below?
\[ f(x) = \log_2{(x-7)}-8 \]

The solution is \( (\infty, \infty) \), which is option E.\begin{enumerate}[label=\Alph*.]
\item \( [a, \infty), a \in [6.18, 7.17] \)

$[-8, \infty)$, which corresponds to using the flipped Domain AND including the endpoint.
\item \( (-\infty, a), a \in [7.63, 8.65] \)

$(-\infty, 8)$, which corresponds to using the using the negative of vertical shift on $(0, \infty)$.
\item \( [a, \infty), a \in [-7.88, -5.08] \)

$[-7, \infty)$, which corresponds to using the negative of the horizontal shift AND including the endpoint.
\item \( (-\infty, a), a \in [-9.22, -7.34] \)

$(-\infty, -8)$, which corresponds to using the vertical shift while the Range is $(-\infty, \infty)$.
\item \( (-\infty, \infty) \)

*This is the correct option.
\end{enumerate}

\textbf{General Comment:} \textbf{General Comments}: The domain of a basic logarithmic function is $(0, \infty)$ and the Range is $(-\infty, \infty)$. We can use shifts when finding the Domain, but the Range will always be all Real numbers.
}
\litem{
Solve the equation for $x$ and choose the interval that contains the solution (if it exists).
\[ 4^{2x-4} = \left(\frac{1}{25}\right)^{4x+5} \]

The solution is \( x = -0.674 \), which is option A.\begin{enumerate}[label=\Alph*.]
\item \( x \in [-3, -0.4] \)

* $x = -0.674$, which is the correct option.
\item \( x \in [3.1, 6.3] \)

$x = 5.275$, which corresponds to distributing the $\ln(base)$ to the second term of the exponent only.
\item \( x \in [-0.3, 1.9] \)

$x = 0.575$, which corresponds to distributing the $\ln(base)$ to the first term of the exponent only.
\item \( x \in [-6.2, -3.1] \)

$x = -4.500$, which corresponds to solving the numerators as equal while ignoring the bases are different.
\item \( \text{There is no Real solution to the equation.} \)

This corresponds to believing there is no solution since the bases are not powers of each other.
\end{enumerate}

\textbf{General Comment:} \textbf{General Comments:} This question was written so that the bases could not be written the same. You will need to take the log of both sides.
}
\litem{
Solve the equation for $x$ and choose the interval that contains the solution (if it exists).
\[ 2^{-3x+3} = \left(\frac{1}{25}\right)^{-4x-5} \]

The solution is \( x = -0.937 \), which is option D.\begin{enumerate}[label=\Alph*.]
\item \( x \in [-0.8, 1.5] \)

$x = 0.535$, which corresponds to distributing the $\ln(base)$ to the first term of the exponent only.
\item \( x \in [-10, -6.6] \)

$x = -8.000$, which corresponds to solving the numerators as equal while ignoring the bases are different.
\item \( x \in [13.6, 14.7] \)

$x = 14.015$, which corresponds to distributing the $\ln(base)$ to the second term of the exponent only.
\item \( x \in [-2.4, -0.5] \)

* $x = -0.937$, which is the correct option.
\item \( \text{There is no Real solution to the equation.} \)

This corresponds to believing there is no solution since the bases are not powers of each other.
\end{enumerate}

\textbf{General Comment:} \textbf{General Comments:} This question was written so that the bases could not be written the same. You will need to take the log of both sides.
}
\litem{
Solve the equation for $x$ and choose the interval that contains the solution (if it exists).
\[ \log_{5}{(-2x+5)}+5 = 2 \]

The solution is \( x = 2.496 \), which is option B.\begin{enumerate}[label=\Alph*.]
\item \( x \in [120, 126] \)

$x = 124.000$, which corresponds to reversing the base and exponent when converting.
\item \( x \in [-0.5, 5.5] \)

* $x = 2.496$, which is the correct option.
\item \( x \in [116, 122] \)

$x = 119.000$, which corresponds to reversing the base and exponent when converting and reversing the value with $x$.
\item \( x \in [-12, -9] \)

$x = -10.000$, which corresponds to ignoring the vertical shift when converting to exponential form.
\item \( \text{There is no Real solution to the equation.} \)

Corresponds to believing a negative coefficient within the log equation means there is no Real solution.
\end{enumerate}

\textbf{General Comment:} \textbf{General Comments:} First, get the equation in the form $\log_b{(cx+d)} = a$. Then, convert to $b^a = cx+d$ and solve.
}
\end{enumerate}

\end{document}