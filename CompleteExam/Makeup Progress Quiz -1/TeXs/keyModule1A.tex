\documentclass{extbook}[14pt]
\usepackage{multicol, enumerate, enumitem, hyperref, color, soul, setspace, parskip, fancyhdr, amssymb, amsthm, amsmath, bbm, latexsym, units, mathtools}
\everymath{\displaystyle}
\usepackage[headsep=0.5cm,headheight=0cm, left=1 in,right= 1 in,top= 1 in,bottom= 1 in]{geometry}
\usepackage{dashrule}  % Package to use the command below to create lines between items
\newcommand{\litem}[1]{\item #1

\rule{\textwidth}{0.4pt}}
\pagestyle{fancy}
\lhead{}
\chead{Answer Key for Makeup Progress Quiz -1 Version A}
\rhead{}
\lfoot{7547-2949}
\cfoot{}
\rfoot{Fall 2020}
\begin{document}
\textbf{This key should allow you to understand why you choose the option you did (beyond just getting a question right or wrong). \href{https://xronos.clas.ufl.edu/mac1105spring2020/courseDescriptionAndMisc/Exams/LearningFromResults}{More instructions on how to use this key can be found here}.}

\textbf{If you have a suggestion to make the keys better, \href{https://forms.gle/CZkbZmPbC9XALEE88}{please fill out the short survey here}.}

\textit{Note: This key is auto-generated and may contain issues and/or errors. The keys are reviewed after each exam to ensure grading is done accurately. If there are issues (like duplicate options), they are noted in the offline gradebook. The keys are a work-in-progress to give students as many resources to improve as possible.}

\rule{\textwidth}{0.4pt}

\begin{enumerate}\litem{
Choose the \textbf{smallest} set of Real numbers that the number below belongs to.
\[ -\sqrt{\frac{190969}{529}} \]

The solution is \( \text{Integer} \), which is option E.\begin{enumerate}[label=\Alph*.]
\item \( \text{Irrational} \)

These cannot be written as a fraction of Integers.
\item \( \text{Not a Real number} \)

These are Nonreal Complex numbers \textbf{OR} things that are not numbers (e.g., dividing by 0).
\item \( \text{Rational} \)

These are numbers that can be written as fraction of Integers (e.g., -2/3)
\item \( \text{Whole} \)

These are the counting numbers with 0 (0, 1, 2, 3, ...)
\item \( \text{Integer} \)

* This is the correct option!
\end{enumerate}

\textbf{General Comment:} First, you \textbf{NEED} to simplify the expression. This question simplifies to $-437$. 
 
 Be sure you look at the simplified fraction and not just the decimal expansion. Numbers such as 13, 17, and 19 provide \textbf{long but repeating/terminating decimal expansions!} 
 
 The only ways to *not* be a Real number are: dividing by 0 or taking the square root of a negative number. 
 
 Irrational numbers are more than just square root of 3: adding or subtracting values from square root of 3 is also irrational.
}
\litem{
Simplify the expression below into the form $a+bi$. Then, choose the intervals that $a$ and $b$ belong to.
\[ (7 + 8 i)(-10 - 5 i) \]

The solution is \( -30 - 115 i \), which is option C.\begin{enumerate}[label=\Alph*.]
\item \( a \in [-117, -107] \text{ and } b \in [43, 51] \)

 $-110 + 45 i$, which corresponds to adding a minus sign in the first term.
\item \( a \in [-117, -107] \text{ and } b \in [-46, -43] \)

 $-110 - 45 i$, which corresponds to adding a minus sign in the second term.
\item \( a \in [-32, -28] \text{ and } b \in [-117, -112] \)

* $-30 - 115 i$, which is the correct option.
\item \( a \in [-72, -67] \text{ and } b \in [-40, -37] \)

 $-70 - 40 i$, which corresponds to just multiplying the real terms to get the real part of the solution and the coefficients in the complex terms to get the complex part.
\item \( a \in [-32, -28] \text{ and } b \in [114, 117] \)

 $-30 + 115 i$, which corresponds to adding a minus sign in both terms.
\end{enumerate}

\textbf{General Comment:} You can treat $i$ as a variable and distribute. Just remember that $i^2=-1$, so you can continue to reduce after you distribute.
}
\litem{
Simplify the expression below into the form $a+bi$. Then, choose the intervals that $a$ and $b$ belong to.
\[ \frac{63 + 66 i}{-3 + i} \]

The solution is \( -12.30  - 26.10 i \), which is option E.\begin{enumerate}[label=\Alph*.]
\item \( a \in [-123.5, -122.5] \text{ and } b \in [-27, -25.5] \)

 $-123.00  - 26.10 i$, which corresponds to forgetting to multiply the conjugate by the numerator and using a plus instead of a minus in the denominator.
\item \( a \in [-26.5, -24] \text{ and } b \in [-14, -13] \)

 $-25.50  - 13.50 i$, which corresponds to forgetting to multiply the conjugate by the numerator and not computing the conjugate correctly.
\item \( a \in [-22.5, -20.5] \text{ and } b \in [65, 67] \)

 $-21.00  + 66.00 i$, which corresponds to just dividing the first term by the first term and the second by the second.
\item \( a \in [-14, -12] \text{ and } b \in [-262, -260.5] \)

 $-12.30  - 261.00 i$, which corresponds to forgetting to multiply the conjugate by the numerator.
\item \( a \in [-14, -12] \text{ and } b \in [-27, -25.5] \)

* $-12.30  - 26.10 i$, which is the correct option.
\end{enumerate}

\textbf{General Comment:} Multiply the numerator and denominator by the *conjugate* of the denominator, then simplify. For example, if we have $2+3i$, the conjugate is $2-3i$.
}
\litem{
Choose the \textbf{smallest} set of Complex numbers that the number below belongs to.
\[ \sqrt{\frac{-300}{5}} i+\sqrt{126}i \]

The solution is \( \text{Nonreal Complex} \), which is option E.\begin{enumerate}[label=\Alph*.]
\item \( \text{Not a Complex Number} \)

This is not a number. The only non-Complex number we know is dividing by 0 as this is not a number!
\item \( \text{Rational} \)

These are numbers that can be written as fraction of Integers (e.g., -2/3 + 5)
\item \( \text{Pure Imaginary} \)

This is a Complex number $(a+bi)$ that \textbf{only} has an imaginary part like $2i$.
\item \( \text{Irrational} \)

These cannot be written as a fraction of Integers. Remember: $\pi$ is not an Integer!
\item \( \text{Nonreal Complex} \)

* This is the correct option!
\end{enumerate}

\textbf{General Comment:} Be sure to simplify $i^2 = -1$. This may remove the imaginary portion for your number. If you are having trouble, you may want to look at the \textit{Subgroups of the Real Numbers} section.
}
\litem{
Simplify the expression below into the form $a+bi$. Then, choose the intervals that $a$ and $b$ belong to.
\[ \frac{9 + 33 i}{4 - 5 i} \]

The solution is \( -3.15  + 4.32 i \), which is option C.\begin{enumerate}[label=\Alph*.]
\item \( a \in [-130.5, -127.5] \text{ and } b \in [4, 6] \)

 $-129.00  + 4.32 i$, which corresponds to forgetting to multiply the conjugate by the numerator and using a plus instead of a minus in the denominator.
\item \( a \in [-3.5, -2.5] \text{ and } b \in [176.5, 177.5] \)

 $-3.15  + 177.00 i$, which corresponds to forgetting to multiply the conjugate by the numerator.
\item \( a \in [-3.5, -2.5] \text{ and } b \in [4, 6] \)

* $-3.15  + 4.32 i$, which is the correct option.
\item \( a \in [4.5, 5.5] \text{ and } b \in [1.5, 3.5] \)

 $4.90  + 2.12 i$, which corresponds to forgetting to multiply the conjugate by the numerator and not computing the conjugate correctly.
\item \( a \in [1.5, 3.5] \text{ and } b \in [-7.5, -6] \)

 $2.25  - 6.60 i$, which corresponds to just dividing the first term by the first term and the second by the second.
\end{enumerate}

\textbf{General Comment:} Multiply the numerator and denominator by the *conjugate* of the denominator, then simplify. For example, if we have $2+3i$, the conjugate is $2-3i$.
}
\litem{
Simplify the expression below and choose the interval the simplification is contained within.
\[ 3 - 8^2 + 1 \div 14 * 17 \div 6 \]

The solution is \( -60.798 \), which is option D.\begin{enumerate}[label=\Alph*.]
\item \( [-61.17, -60.85] \)

 -60.999, which corresponds to an Order of Operations error: not reading left-to-right for multiplication/division.
\item \( [67.12, 67.21] \)

 67.202, which corresponds to an Order of Operations error: multiplying by negative before squaring. For example: $(-3)^2 \neq -3^2$
\item \( [66.98, 67.09] \)

 67.001, which corresponds to two Order of Operations errors.
\item \( [-60.81, -60.7] \)

* -60.798, this is the correct option
\item \( \text{None of the above} \)

 You may have gotten this by making an unanticipated error. If you got a value that is not any of the others, please let the coordinator know so they can help you figure out what happened.
\end{enumerate}

\textbf{General Comment:} While you may remember (or were taught) PEMDAS is done in order, it is actually done as P/E/MD/AS. When we are at MD or AS, we read left to right.
}
\litem{
Choose the \textbf{smallest} set of Real numbers that the number below belongs to.
\[ -\sqrt{\frac{213444}{441}} \]

The solution is \( \text{Integer} \), which is option D.\begin{enumerate}[label=\Alph*.]
\item \( \text{Whole} \)

These are the counting numbers with 0 (0, 1, 2, 3, ...)
\item \( \text{Rational} \)

These are numbers that can be written as fraction of Integers (e.g., -2/3)
\item \( \text{Irrational} \)

These cannot be written as a fraction of Integers.
\item \( \text{Integer} \)

* This is the correct option!
\item \( \text{Not a Real number} \)

These are Nonreal Complex numbers \textbf{OR} things that are not numbers (e.g., dividing by 0).
\end{enumerate}

\textbf{General Comment:} First, you \textbf{NEED} to simplify the expression. This question simplifies to $-462$. 
 
 Be sure you look at the simplified fraction and not just the decimal expansion. Numbers such as 13, 17, and 19 provide \textbf{long but repeating/terminating decimal expansions!} 
 
 The only ways to *not* be a Real number are: dividing by 0 or taking the square root of a negative number. 
 
 Irrational numbers are more than just square root of 3: adding or subtracting values from square root of 3 is also irrational.
}
\litem{
Choose the \textbf{smallest} set of Complex numbers that the number below belongs to.
\[ \frac{-8}{-9}+64i^2 \]

The solution is \( \text{Rational} \), which is option B.\begin{enumerate}[label=\Alph*.]
\item \( \text{Pure Imaginary} \)

This is a Complex number $(a+bi)$ that \textbf{only} has an imaginary part like $2i$.
\item \( \text{Rational} \)

* This is the correct option!
\item \( \text{Not a Complex Number} \)

This is not a number. The only non-Complex number we know is dividing by 0 as this is not a number!
\item \( \text{Irrational} \)

These cannot be written as a fraction of Integers. Remember: $\pi$ is not an Integer!
\item \( \text{Nonreal Complex} \)

This is a Complex number $(a+bi)$ that is not Real (has $i$ as part of the number).
\end{enumerate}

\textbf{General Comment:} Be sure to simplify $i^2 = -1$. This may remove the imaginary portion for your number. If you are having trouble, you may want to look at the \textit{Subgroups of the Real Numbers} section.
}
\litem{
Simplify the expression below and choose the interval the simplification is contained within.
\[ 13 - 12^2 + 9 \div 3 * 20 \div 14 \]

The solution is \( -126.714 \), which is option B.\begin{enumerate}[label=\Alph*.]
\item \( [157.29, 165.29] \)

 161.286, which corresponds to an Order of Operations error: multiplying by negative before squaring. For example: $(-3)^2 \neq -3^2$
\item \( [-127.71, -125.71] \)

* -126.714, this is the correct option
\item \( [-137.99, -127.99] \)

 -130.989, which corresponds to an Order of Operations error: not reading left-to-right for multiplication/division.
\item \( [152.01, 159.01] \)

 157.011, which corresponds to two Order of Operations errors.
\item \( \text{None of the above} \)

 You may have gotten this by making an unanticipated error. If you got a value that is not any of the others, please let the coordinator know so they can help you figure out what happened.
\end{enumerate}

\textbf{General Comment:} While you may remember (or were taught) PEMDAS is done in order, it is actually done as P/E/MD/AS. When we are at MD or AS, we read left to right.
}
\litem{
Simplify the expression below into the form $a+bi$. Then, choose the intervals that $a$ and $b$ belong to.
\[ (-8 + 3 i)(-9 - 10 i) \]

The solution is \( 102 + 53 i \), which is option A.\begin{enumerate}[label=\Alph*.]
\item \( a \in [100, 107] \text{ and } b \in [50, 56] \)

* $102 + 53 i$, which is the correct option.
\item \( a \in [42, 47] \text{ and } b \in [104, 108] \)

 $42 + 107 i$, which corresponds to adding a minus sign in the first term.
\item \( a \in [100, 107] \text{ and } b \in [-53, -48] \)

 $102 - 53 i$, which corresponds to adding a minus sign in both terms.
\item \( a \in [42, 47] \text{ and } b \in [-111, -102] \)

 $42 - 107 i$, which corresponds to adding a minus sign in the second term.
\item \( a \in [72, 77] \text{ and } b \in [-34, -27] \)

 $72 - 30 i$, which corresponds to just multiplying the real terms to get the real part of the solution and the coefficients in the complex terms to get the complex part.
\end{enumerate}

\textbf{General Comment:} You can treat $i$ as a variable and distribute. Just remember that $i^2=-1$, so you can continue to reduce after you distribute.
}
\end{enumerate}

\end{document}