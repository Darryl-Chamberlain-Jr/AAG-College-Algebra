\documentclass{extbook}[14pt]
\usepackage{multicol, enumerate, enumitem, hyperref, color, soul, setspace, parskip, fancyhdr, amssymb, amsthm, amsmath, bbm, latexsym, units, mathtools}
\everymath{\displaystyle}
\usepackage[headsep=0.5cm,headheight=0cm, left=1 in,right= 1 in,top= 1 in,bottom= 1 in]{geometry}
\usepackage{dashrule}  % Package to use the command below to create lines between items
\newcommand{\litem}[1]{\item #1

\rule{\textwidth}{0.4pt}}
\pagestyle{fancy}
\lhead{}
\chead{Answer Key for Makeup Progress Quiz -1 Version C}
\rhead{}
\lfoot{7547-2949}
\cfoot{}
\rfoot{Fall 2020}
\begin{document}
\textbf{This key should allow you to understand why you choose the option you did (beyond just getting a question right or wrong). \href{https://xronos.clas.ufl.edu/mac1105spring2020/courseDescriptionAndMisc/Exams/LearningFromResults}{More instructions on how to use this key can be found here}.}

\textbf{If you have a suggestion to make the keys better, \href{https://forms.gle/CZkbZmPbC9XALEE88}{please fill out the short survey here}.}

\textit{Note: This key is auto-generated and may contain issues and/or errors. The keys are reviewed after each exam to ensure grading is done accurately. If there are issues (like duplicate options), they are noted in the offline gradebook. The keys are a work-in-progress to give students as many resources to improve as possible.}

\rule{\textwidth}{0.4pt}

\begin{enumerate}\litem{
What are the \textit{possible Integer} roots of the polynomial below?
\[ f(x) = 7x^{4} +7 x^{3} +4 x^{2} +5 x + 5 \]

The solution is \( \pm 1,\pm 5 \), which is option B.\begin{enumerate}[label=\Alph*.]
\item \( \text{ All combinations of: }\frac{\pm 1,\pm 7}{\pm 1,\pm 5} \)

 Distractor 3: Corresponds to the plus or minus of the inverse quotient (an/a0) of the factors. 
\item \( \pm 1,\pm 5 \)

* This is the solution \textbf{since we asked for the possible Integer roots}!
\item \( \pm 1,\pm 7 \)

 Distractor 1: Corresponds to the plus or minus factors of a1 only.
\item \( \text{ All combinations of: }\frac{\pm 1,\pm 5}{\pm 1,\pm 7} \)

This would have been the solution \textbf{if asked for the possible Rational roots}!
\item \( \text{There is no formula or theorem that tells us all possible Integer roots.} \)

 Distractor 4: Corresponds to not recognizing Integers as a subset of Rationals.
\end{enumerate}

\textbf{General Comment:} We have a way to find the possible Rational roots. The possible Integer roots are the Integers in this list.
}
\litem{
Factor the polynomial below completely. Then, choose the intervals the zeros of the polynomial belong to, where $z_1 \leq z_2 \leq z_3$. \textit{To make the problem easier, all zeros are between -5 and 5.}
\[ f(x) = 12x^{3} -91 x^{2} +175 x -100 \]

The solution is \( [1.25, 1.3333333333333333, 5] \), which is option B.\begin{enumerate}[label=\Alph*.]
\item \( z_1 \in [-5.2, -4.3], \text{   }  z_2 \in [-1.11, -0.74], \text{   and   } z_3 \in [-0.75, -0.75] \)

 Distractor 3: Corresponds to negatives of all zeros AND inversing rational roots.
\item \( z_1 \in [1.2, 1.7], \text{   }  z_2 \in [1, 1.43], \text{   and   } z_3 \in [4.9, 5.23] \)

* This is the solution!
\item \( z_1 \in [-5.2, -4.3], \text{   }  z_2 \in [-4.01, -3.75], \text{   and   } z_3 \in [-0.62, -0.18] \)

 Distractor 4: Corresponds to moving factors from one rational to another.
\item \( z_1 \in [-0.4, 0.9], \text{   }  z_2 \in [0.2, 1.17], \text{   and   } z_3 \in [4.9, 5.23] \)

 Distractor 2: Corresponds to inversing rational roots.
\item \( z_1 \in [-5.2, -4.3], \text{   }  z_2 \in [-1.61, -0.86], \text{   and   } z_3 \in [-1.25, -1.23] \)

 Distractor 1: Corresponds to negatives of all zeros.
\end{enumerate}

\textbf{General Comment:} Remember to try the middle-most integers first as these normally are the zeros. Also, once you get it to a quadratic, you can use your other factoring techniques to finish factoring.
}
\litem{
What are the \textit{possible Rational} roots of the polynomial below?
\[ f(x) = 5x^{2} +3 x + 3 \]

The solution is \( \text{ All combinations of: }\frac{\pm 1,\pm 3}{\pm 1,\pm 5} \), which is option C.\begin{enumerate}[label=\Alph*.]
\item \( \text{ All combinations of: }\frac{\pm 1,\pm 5}{\pm 1,\pm 3} \)

 Distractor 3: Corresponds to the plus or minus of the inverse quotient (an/a0) of the factors. 
\item \( \pm 1,\pm 5 \)

 Distractor 1: Corresponds to the plus or minus factors of a1 only.
\item \( \text{ All combinations of: }\frac{\pm 1,\pm 3}{\pm 1,\pm 5} \)

* This is the solution \textbf{since we asked for the possible Rational roots}!
\item \( \pm 1,\pm 3 \)

This would have been the solution \textbf{if asked for the possible Integer roots}!
\item \( \text{ There is no formula or theorem that tells us all possible Rational roots.} \)

 Distractor 4: Corresponds to not recalling the theorem for rational roots of a polynomial.
\end{enumerate}

\textbf{General Comment:} We have a way to find the possible Rational roots. The possible Integer roots are the Integers in this list.
}
\litem{
Perform the division below. Then, find the intervals that correspond to the quotient in the form $ax^2+bx+c$ and remainder $r$.
\[ \frac{10x^{3} +52 x^{2} +32 x -66}{x + 4} \]

The solution is \( 10x^{2} +12 x -16 + \frac{-2}{x + 4} \), which is option A.\begin{enumerate}[label=\Alph*.]
\item \( a \in [9, 14], \text{   } b \in [12, 16], \text{   } c \in [-18, -6], \text{   and   } r \in [-2, 3]. \)

* This is the solution!
\item \( a \in [9, 14], \text{   } b \in [92, 100], \text{   } c \in [398, 401], \text{   and   } r \in [1528, 1540]. \)

 You divided by the opposite of the factor.
\item \( a \in [-42, -38], \text{   } b \in [-108, -107], \text{   } c \in [-405, -395], \text{   and   } r \in [-1671, -1663]. \)

 You divided by the opposite of the factor AND multiplied the first factor rather than just bringing it down.
\item \( a \in [-42, -38], \text{   } b \in [209, 213], \text{   } c \in [-819, -815], \text{   and   } r \in [3192, 3199]. \)

 You multiplied by the synthetic number rather than bringing the first factor down.
\item \( a \in [9, 14], \text{   } b \in [-2, 8], \text{   } c \in [18, 27], \text{   and   } r \in [-179, -173]. \)

 You multiplied by the synthetic number and subtracted rather than adding during synthetic division.
\end{enumerate}

\textbf{General Comment:} Be sure to synthetically divide by the zero of the denominator!
}
\litem{
Perform the division below. Then, find the intervals that correspond to the quotient in the form $ax^2+bx+c$ and remainder $r$.
\[ \frac{10x^{3} +32 x^{2} -82 x + 35}{x + 5} \]

The solution is \( 10x^{2} -18 x + 8 + \frac{-5}{x + 5} \), which is option D.\begin{enumerate}[label=\Alph*.]
\item \( a \in [-55, -47], \text{   } b \in [-218, -211], \text{   } c \in [-1175, -1170], \text{   and   } r \in [-5827, -5821]. \)

 You divided by the opposite of the factor AND multiplied the first factor rather than just bringing it down.
\item \( a \in [-55, -47], \text{   } b \in [279, 286], \text{   } c \in [-1493, -1489], \text{   and   } r \in [7489, 7500]. \)

 You multiplied by the synthetic number rather than bringing the first factor down.
\item \( a \in [6, 15], \text{   } b \in [-30, -26], \text{   } c \in [86, 90], \text{   and   } r \in [-486, -477]. \)

 You multiplied by the synthetic number and subtracted rather than adding during synthetic division.
\item \( a \in [6, 15], \text{   } b \in [-24, -15], \text{   } c \in [3, 9], \text{   and   } r \in [-5, 0]. \)

* This is the solution!
\item \( a \in [6, 15], \text{   } b \in [76, 84], \text{   } c \in [324, 329], \text{   and   } r \in [1672, 1677]. \)

 You divided by the opposite of the factor.
\end{enumerate}

\textbf{General Comment:} Be sure to synthetically divide by the zero of the denominator!
}
\litem{
Perform the division below. Then, find the intervals that correspond to the quotient in the form $ax^2+bx+c$ and remainder $r$.
\[ \frac{9x^{3} -28 x -18}{x -2} \]

The solution is \( 9x^{2} +18 x + 8 + \frac{-2}{x -2} \), which is option E.\begin{enumerate}[label=\Alph*.]
\item \( a \in [9, 10], b \in [-18, -16], c \in [6, 11], \text{ and } r \in [-34.2, -29.9]. \)

 You divided by the opposite of the factor.
\item \( a \in [17, 24], b \in [-43, -33], c \in [38, 45], \text{ and } r \in [-107.2, -102.2]. \)

 You divided by the opposite of the factor AND multipled the first factor rather than just bringing it down.
\item \( a \in [9, 10], b \in [3, 15], c \in [-26, -13], \text{ and } r \in [-37.9, -35]. \)

 You multipled by the synthetic number and subtracted rather than adding during synthetic division.
\item \( a \in [17, 24], b \in [36, 38], c \in [38, 45], \text{ and } r \in [69.9, 71.4]. \)

 You multipled by the synthetic number rather than bringing the first factor down.
\item \( a \in [9, 10], b \in [17, 22], c \in [6, 11], \text{ and } r \in [-2.6, -1.3]. \)

* This is the solution!
\end{enumerate}

\textbf{General Comment:} Be sure to synthetically divide by the zero of the denominator! Also, make sure to include 0 placeholders for missing terms.
}
\litem{
Factor the polynomial below completely, knowing that $x+4$ is a factor. Then, choose the intervals the zeros of the polynomial belong to, where $z_1 \leq z_2 \leq z_3 \leq z_4$. \textit{To make the problem easier, all zeros are between -5 and 5.}
\[ f(x) = 8x^{4} +14 x^{3} -83 x^{2} -14 x + 120 \]

The solution is \( [-4, -1.25, 1.5, 2] \), which is option D.\begin{enumerate}[label=\Alph*.]
\item \( z_1 \in [-3, -0.6], \text{   }  z_2 \in [-1.62, -1.35], z_3 \in [1.21, 1.39], \text{   and   } z_4 \in [2.8, 4.7] \)

 Distractor 1: Corresponds to negatives of all zeros.
\item \( z_1 \in [-3, -0.6], \text{   }  z_2 \in [-0.73, -0.57], z_3 \in [0.78, 0.81], \text{   and   } z_4 \in [2.8, 4.7] \)

 Distractor 3: Corresponds to negatives of all zeros AND inversing rational roots.
\item \( z_1 \in [-4.6, -2.8], \text{   }  z_2 \in [-0.87, -0.74], z_3 \in [0.64, 0.74], \text{   and   } z_4 \in [1.8, 2.8] \)

 Distractor 2: Corresponds to inversing rational roots.
\item \( z_1 \in [-4.6, -2.8], \text{   }  z_2 \in [-1.28, -1.19], z_3 \in [1.48, 1.52], \text{   and   } z_4 \in [1.8, 2.8] \)

* This is the solution!
\item \( z_1 \in [-3, -0.6], \text{   }  z_2 \in [-0.39, -0.3], z_3 \in [3.73, 4.05], \text{   and   } z_4 \in [4.6, 6] \)

 Distractor 4: Corresponds to moving factors from one rational to another.
\end{enumerate}

\textbf{General Comment:} Remember to try the middle-most integers first as these normally are the zeros. Also, once you get it to a quadratic, you can use your other factoring techniques to finish factoring.
}
\litem{
Factor the polynomial below completely. Then, choose the intervals the zeros of the polynomial belong to, where $z_1 \leq z_2 \leq z_3$. \textit{To make the problem easier, all zeros are between -5 and 5.}
\[ f(x) = 8x^{3} -34 x^{2} +45 x -18 \]

The solution is \( [0.75, 1.5, 2] \), which is option A.\begin{enumerate}[label=\Alph*.]
\item \( z_1 \in [0.68, 0.8], \text{   }  z_2 \in [1.41, 1.55], \text{   and   } z_3 \in [1.94, 2.03] \)

* This is the solution!
\item \( z_1 \in [-3.08, -2.88], \text{   }  z_2 \in [-2.06, -1.89], \text{   and   } z_3 \in [-0.54, -0.29] \)

 Distractor 4: Corresponds to moving factors from one rational to another.
\item \( z_1 \in [-2.18, -1.77], \text{   }  z_2 \in [-1.57, -1.45], \text{   and   } z_3 \in [-0.85, -0.67] \)

 Distractor 1: Corresponds to negatives of all zeros.
\item \( z_1 \in [0.37, 0.71], \text{   }  z_2 \in [1.31, 1.47], \text{   and   } z_3 \in [1.94, 2.03] \)

 Distractor 2: Corresponds to inversing rational roots.
\item \( z_1 \in [-2.18, -1.77], \text{   }  z_2 \in [-1.42, -1.29], \text{   and   } z_3 \in [-0.7, -0.66] \)

 Distractor 3: Corresponds to negatives of all zeros AND inversing rational roots.
\end{enumerate}

\textbf{General Comment:} Remember to try the middle-most integers first as these normally are the zeros. Also, once you get it to a quadratic, you can use your other factoring techniques to finish factoring.
}
\litem{
Perform the division below. Then, find the intervals that correspond to the quotient in the form $ax^2+bx+c$ and remainder $r$.
\[ \frac{6x^{3} +21 x^{2} -31}{x + 3} \]

The solution is \( 6x^{2} +3 x -9 + \frac{-4}{x + 3} \), which is option C.\begin{enumerate}[label=\Alph*.]
\item \( a \in [4, 9], b \in [38, 48], c \in [110, 119], \text{ and } r \in [316, 323]. \)

 You divided by the opposite of the factor.
\item \( a \in [4, 9], b \in [-3, -1], c \in [11, 14], \text{ and } r \in [-84, -77]. \)

 You multipled by the synthetic number and subtracted rather than adding during synthetic division.
\item \( a \in [4, 9], b \in [-2, 5], c \in [-11, -7], \text{ and } r \in [-4, 1]. \)

* This is the solution!
\item \( a \in [-21, -16], b \in [-35, -27], c \in [-99, -97], \text{ and } r \in [-329, -318]. \)

 You divided by the opposite of the factor AND multipled the first factor rather than just bringing it down.
\item \( a \in [-21, -16], b \in [74, 80], c \in [-228, -223], \text{ and } r \in [641, 648]. \)

 You multipled by the synthetic number rather than bringing the first factor down.
\end{enumerate}

\textbf{General Comment:} Be sure to synthetically divide by the zero of the denominator! Also, make sure to include 0 placeholders for missing terms.
}
\litem{
Factor the polynomial below completely, knowing that $x+3$ is a factor. Then, choose the intervals the zeros of the polynomial belong to, where $z_1 \leq z_2 \leq z_3 \leq z_4$. \textit{To make the problem easier, all zeros are between -5 and 5.}
\[ f(x) = 8x^{4} +2 x^{3} -147 x^{2} -288 x -135 \]

The solution is \( [-3, -1.5, -0.75, 5] \), which is option B.\begin{enumerate}[label=\Alph*.]
\item \( z_1 \in [-3, -2], \text{   }  z_2 \in [-1.37, -1.31], z_3 \in [-0.72, -0.47], \text{   and   } z_4 \in [4.05, 5.38] \)

 Distractor 2: Corresponds to inversing rational roots.
\item \( z_1 \in [-3, -2], \text{   }  z_2 \in [-1.5, -1.47], z_3 \in [-0.87, -0.68], \text{   and   } z_4 \in [4.05, 5.38] \)

* This is the solution!
\item \( z_1 \in [-5, -4], \text{   }  z_2 \in [0.36, 0.4], z_3 \in [2.73, 3.27], \text{   and   } z_4 \in [2.23, 3.99] \)

 Distractor 4: Corresponds to moving factors from one rational to another.
\item \( z_1 \in [-5, -4], \text{   }  z_2 \in [0.65, 0.68], z_3 \in [1, 1.4], \text{   and   } z_4 \in [2.23, 3.99] \)

 Distractor 3: Corresponds to negatives of all zeros AND inversing rational roots.
\item \( z_1 \in [-5, -4], \text{   }  z_2 \in [0.75, 0.77], z_3 \in [1.42, 1.72], \text{   and   } z_4 \in [2.23, 3.99] \)

 Distractor 1: Corresponds to negatives of all zeros.
\end{enumerate}

\textbf{General Comment:} Remember to try the middle-most integers first as these normally are the zeros. Also, once you get it to a quadratic, you can use your other factoring techniques to finish factoring.
}
\end{enumerate}

\end{document}