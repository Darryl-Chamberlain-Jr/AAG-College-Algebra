\documentclass{extbook}[14pt]
\usepackage{multicol, enumerate, enumitem, hyperref, color, soul, setspace, parskip, fancyhdr, amssymb, amsthm, amsmath, bbm, latexsym, units, mathtools}
\everymath{\displaystyle}
\usepackage[headsep=0.5cm,headheight=0cm, left=1 in,right= 1 in,top= 1 in,bottom= 1 in]{geometry}
\usepackage{dashrule}  % Package to use the command below to create lines between items
\newcommand{\litem}[1]{\item #1

\rule{\textwidth}{0.4pt}}
\pagestyle{fancy}
\lhead{}
\chead{Answer Key for Makeup Progress Quiz -1 Version B}
\rhead{}
\lfoot{7547-2949}
\cfoot{}
\rfoot{Fall 2020}
\begin{document}
\textbf{This key should allow you to understand why you choose the option you did (beyond just getting a question right or wrong). \href{https://xronos.clas.ufl.edu/mac1105spring2020/courseDescriptionAndMisc/Exams/LearningFromResults}{More instructions on how to use this key can be found here}.}

\textbf{If you have a suggestion to make the keys better, \href{https://forms.gle/CZkbZmPbC9XALEE88}{please fill out the short survey here}.}

\textit{Note: This key is auto-generated and may contain issues and/or errors. The keys are reviewed after each exam to ensure grading is done accurately. If there are issues (like duplicate options), they are noted in the offline gradebook. The keys are a work-in-progress to give students as many resources to improve as possible.}

\rule{\textwidth}{0.4pt}

\begin{enumerate}\litem{
Solve the equation for $x$ and choose the interval that contains the solution (if it exists).
\[ \log_{5}{(-4x+7)}+4 = 3 \]

The solution is \( x = 1.700 \), which is option D.\begin{enumerate}[label=\Alph*.]
\item \( x \in [-29.86, -29.04] \)

$x = -29.500$, which corresponds to ignoring the vertical shift when converting to exponential form.
\item \( x \in [-2.02, -1.4] \)

$x = -1.500$, which corresponds to reversing the base and exponent when converting and reversing the value with $x$.
\item \( x \in [1.91, 2.25] \)

$x = 2.000$, which corresponds to reversing the base and exponent when converting.
\item \( x \in [1.33, 1.77] \)

* $x = 1.700$, which is the correct option.
\item \( \text{There is no Real solution to the equation.} \)

Corresponds to believing a negative coefficient within the log equation means there is no Real solution.
\end{enumerate}

\textbf{General Comment:} \textbf{General Comments:} First, get the equation in the form $\log_b{(cx+d)} = a$. Then, convert to $b^a = cx+d$ and solve.
}
\litem{
 Solve the equation for $x$ and choose the interval that contains $x$ (if it exists).
\[  22 = \ln{\sqrt[7]{\frac{5}{e^{4x}}}} \]

The solution is \( x = -38.098, \text{ which does not fit in any of the interval options.} \), which is option E.\begin{enumerate}[label=\Alph*.]
\item \( x \in [34.1, 40.1] \)

$x = 38.098$, which is the negative of the correct solution.
\item \( x \in [-7.81, -0.81] \)

$x = -5.812$, which corresponds to thinking you need to take the natural log of the left side before reducing.
\item \( x \in [-10.6, -8.6] \)

$x = -10.598$, which corresponds to treating any root as a square root.
\item \( \text{There is no Real solution to the equation.} \)

This corresponds to believing you cannot solve the equation.
\item \( \text{None of the above.} \)

*$x = -38.098$ is the correct solution and does not fit in any of the other intervals.
\end{enumerate}

\textbf{General Comment:} \textbf{General Comments}: After using the properties of logarithmic functions to break up the right-hand side, use $\ln(e) = 1$ to reduce the question to a linear function to solve. You can put $\ln(5)$ into a calculator if you are having trouble.
}
\litem{
Which of the following intervals describes the Domain of the function below?
\[ f(x) = -e^{x+5}-9 \]

The solution is \( (-\infty, \infty) \), which is option E.\begin{enumerate}[label=\Alph*.]
\item \( (a, \infty), a \in [7, 12] \)

$(9, \infty)$, which corresponds to using the negative vertical shift AND flipping the Range interval.
\item \( (-\infty, a], a \in [-13, -3] \)

$(-\infty, -9]$, which corresponds to using the correct vertical shift *if we wanted the Range* AND including the endpoint.
\item \( [a, \infty), a \in [7, 12] \)

$[9, \infty)$, which corresponds to using the negative vertical shift AND flipping the Range interval AND including the endpoint.
\item \( (-\infty, a), a \in [-13, -3] \)

$(-\infty, -9)$, which corresponds to using the correct vertical shift *if we wanted the Range*.
\item \( (-\infty, \infty) \)

* This is the correct option.
\end{enumerate}

\textbf{General Comment:} \textbf{General Comments}: Domain of a basic exponential function is $(-\infty, \infty)$ while the Range is $(0, \infty)$. We can shift these intervals [and even flip when $a<0$!] to find the new Domain/Range.
}
\litem{
Which of the following intervals describes the Domain of the function below?
\[ f(x) = -e^{x-5}-1 \]

The solution is \( (-\infty, \infty) \), which is option E.\begin{enumerate}[label=\Alph*.]
\item \( (-\infty, a), a \in [-1, 0] \)

$(-\infty, -1)$, which corresponds to using the correct vertical shift *if we wanted the Range*.
\item \( [a, \infty), a \in [1, 10] \)

$[1, \infty)$, which corresponds to using the negative vertical shift AND flipping the Range interval AND including the endpoint.
\item \( (-\infty, a], a \in [-1, 0] \)

$(-\infty, -1]$, which corresponds to using the correct vertical shift *if we wanted the Range* AND including the endpoint.
\item \( (a, \infty), a \in [1, 10] \)

$(1, \infty)$, which corresponds to using the negative vertical shift AND flipping the Range interval.
\item \( (-\infty, \infty) \)

* This is the correct option.
\end{enumerate}

\textbf{General Comment:} \textbf{General Comments}: Domain of a basic exponential function is $(-\infty, \infty)$ while the Range is $(0, \infty)$. We can shift these intervals [and even flip when $a<0$!] to find the new Domain/Range.
}
\litem{
Which of the following intervals describes the Range of the function below?
\[ f(x) = -\log_2{(x+7)}+5 \]

The solution is \( (\infty, \infty) \), which is option E.\begin{enumerate}[label=\Alph*.]
\item \( (-\infty, a), a \in [-6.1, -2.4] \)

$(-\infty, -5)$, which corresponds to using the using the negative of vertical shift on $(0, \infty)$.
\item \( [a, \infty), a \in [-9.5, -5.9] \)

$[5, \infty)$, which corresponds to using the flipped Domain AND including the endpoint.
\item \( [a, \infty), a \in [5.4, 7.1] \)

$[7, \infty)$, which corresponds to using the negative of the horizontal shift AND including the endpoint.
\item \( (-\infty, a), a \in [4.3, 6.9] \)

$(-\infty, 5)$, which corresponds to using the vertical shift while the Range is $(-\infty, \infty)$.
\item \( (-\infty, \infty) \)

*This is the correct option.
\end{enumerate}

\textbf{General Comment:} \textbf{General Comments}: The domain of a basic logarithmic function is $(0, \infty)$ and the Range is $(-\infty, \infty)$. We can use shifts when finding the Domain, but the Range will always be all Real numbers.
}
\litem{
 Solve the equation for $x$ and choose the interval that contains $x$ (if it exists).
\[  21 = \sqrt[5]{\frac{12}{e^{6x}}} \]

The solution is \( x = -2.123, \text{ which does not fit in any of the interval options.} \), which is option E.\begin{enumerate}[label=\Alph*.]
\item \( x \in [0.5, 3] \)

$x = 2.123$, which is the negative of the correct solution.
\item \( x \in [-1.2, -0.4] \)

$x = -0.601$, which corresponds to treating any root as a square root.
\item \( x \in [-18.5, -16.8] \)

$x = -17.914$, which corresponds to thinking you don't need to take the natural log of both sides before reducing, as if the right side already has a natural log.
\item \( \text{There is no Real solution to the equation.} \)

This corresponds to believing you cannot solve the equation.
\item \( \text{None of the above.} \)

* $x = -2.123$ is the correct solution and does not fit in any of the other intervals.
\end{enumerate}

\textbf{General Comment:} \textbf{General Comments}: After using the properties of logarithmic functions to break up the right-hand side, use $\ln(e) = 1$ to reduce the question to a linear function to solve. You can put $\ln(12)$ into a calculator if you are having trouble.
}
\litem{
Which of the following intervals describes the Range of the function below?
\[ f(x) = -\log_2{(x-9)}+1 \]

The solution is \( (\infty, \infty) \), which is option E.\begin{enumerate}[label=\Alph*.]
\item \( (-\infty, a), a \in [-0.6, 2.5] \)

$(-\infty, 1)$, which corresponds to using the vertical shift while the Range is $(-\infty, \infty)$.
\item \( [a, \infty), a \in [-9.6, -7.9] \)

$[-9, \infty)$, which corresponds to using the negative of the horizontal shift AND including the endpoint.
\item \( [a, \infty), a \in [7.5, 10.4] \)

$[1, \infty)$, which corresponds to using the flipped Domain AND including the endpoint.
\item \( (-\infty, a), a \in [-1.8, -0.6] \)

$(-\infty, -1)$, which corresponds to using the using the negative of vertical shift on $(0, \infty)$.
\item \( (-\infty, \infty) \)

*This is the correct option.
\end{enumerate}

\textbf{General Comment:} \textbf{General Comments}: The domain of a basic logarithmic function is $(0, \infty)$ and the Range is $(-\infty, \infty)$. We can use shifts when finding the Domain, but the Range will always be all Real numbers.
}
\litem{
Solve the equation for $x$ and choose the interval that contains the solution (if it exists).
\[ 5^{4x+2} = 49^{3x-2} \]

The solution is \( x = 2.101 \), which is option D.\begin{enumerate}[label=\Alph*.]
\item \( x \in [-12.6, -10] \)

$x = -11.003$, which corresponds to distributing the $\ln(base)$ to the second term of the exponent only.
\item \( x \in [-0.4, 1.4] \)

$x = 0.764$, which corresponds to distributing the $\ln(base)$ to the first term of the exponent only.
\item \( x \in [-5.7, -2] \)

$x = -4.000$, which corresponds to solving the numerators as equal while ignoring the bases are different.
\item \( x \in [1.4, 2.4] \)

* $x = 2.101$, which is the correct option.
\item \( \text{There is no Real solution to the equation.} \)

This corresponds to believing there is no solution since the bases are not powers of each other.
\end{enumerate}

\textbf{General Comment:} \textbf{General Comments:} This question was written so that the bases could not be written the same. You will need to take the log of both sides.
}
\litem{
Solve the equation for $x$ and choose the interval that contains the solution (if it exists).
\[ 2^{-4x-5} = \left(\frac{1}{9}\right)^{2x-3} \]

The solution is \( x = 6.201 \), which is option B.\begin{enumerate}[label=\Alph*.]
\item \( x \in [-1.8, -0.5] \)

$x = -1.676$, which corresponds to distributing the $\ln(base)$ to the second term of the exponent only.
\item \( x \in [4.7, 6.3] \)

* $x = 6.201$, which is the correct option.
\item \( x \in [0.3, 1.8] \)

$x = 1.233$, which corresponds to distributing the $\ln(base)$ to the first term of the exponent only.
\item \( x \in [-0.7, -0.2] \)

$x = -0.333$, which corresponds to solving the numerators as equal while ignoring the bases are different.
\item \( \text{There is no Real solution to the equation.} \)

This corresponds to believing there is no solution since the bases are not powers of each other.
\end{enumerate}

\textbf{General Comment:} \textbf{General Comments:} This question was written so that the bases could not be written the same. You will need to take the log of both sides.
}
\litem{
Solve the equation for $x$ and choose the interval that contains the solution (if it exists).
\[ \log_{4}{(4x+7)}+5 = 3 \]

The solution is \( x = -1.734 \), which is option C.\begin{enumerate}[label=\Alph*.]
\item \( x \in [2.7, 6.2] \)

$x = 5.750$, which corresponds to reversing the base and exponent when converting and reversing the value with $x$.
\item \( x \in [2, 3.9] \)

$x = 2.250$, which corresponds to reversing the base and exponent when converting.
\item \( x \in [-2.4, -1.2] \)

* $x = -1.734$, which is the correct option.
\item \( x \in [13.9, 14.7] \)

$x = 14.250$, which corresponds to ignoring the vertical shift when converting to exponential form.
\item \( \text{There is no Real solution to the equation.} \)

Corresponds to believing a negative coefficient within the log equation means there is no Real solution.
\end{enumerate}

\textbf{General Comment:} \textbf{General Comments:} First, get the equation in the form $\log_b{(cx+d)} = a$. Then, convert to $b^a = cx+d$ and solve.
}
\end{enumerate}

\end{document}