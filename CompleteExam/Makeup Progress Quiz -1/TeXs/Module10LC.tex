\documentclass[14pt]{extbook}
\usepackage{multicol, enumerate, enumitem, hyperref, color, soul, setspace, parskip, fancyhdr} %General Packages
\usepackage{amssymb, amsthm, amsmath, bbm, latexsym, units, mathtools} %Math Packages
\everymath{\displaystyle} %All math in Display Style
% Packages with additional options
\usepackage[headsep=0.5cm,headheight=12pt, left=1 in,right= 1 in,top= 1 in,bottom= 1 in]{geometry}
\usepackage[usenames,dvipsnames]{xcolor}
\usepackage{dashrule}  % Package to use the command below to create lines between items
\newcommand{\litem}[1]{\item#1\hspace*{-1cm}\rule{\textwidth}{0.4pt}}
\pagestyle{fancy}
\lhead{Makeup Progress Quiz -1}
\chead{}
\rhead{Version C}
\lfoot{7547-2949}
\cfoot{}
\rfoot{Fall 2020}
\begin{document}

\begin{enumerate}
\litem{
What are the \textit{possible Integer} roots of the polynomial below?\[ f(x) = 7x^{4} +7 x^{3} +4 x^{2} +5 x + 5 \]\begin{enumerate}[label=\Alph*.]
\item \( \text{ All combinations of: }\frac{\pm 1,\pm 7}{\pm 1,\pm 5} \)
\item \( \pm 1,\pm 5 \)
\item \( \pm 1,\pm 7 \)
\item \( \text{ All combinations of: }\frac{\pm 1,\pm 5}{\pm 1,\pm 7} \)
\item \( \text{There is no formula or theorem that tells us all possible Integer roots.} \)

\end{enumerate} }
\litem{
Factor the polynomial below completely. Then, choose the intervals the zeros of the polynomial belong to, where $z_1 \leq z_2 \leq z_3$. \textit{To make the problem easier, all zeros are between -5 and 5.}\[ f(x) = 12x^{3} -91 x^{2} +175 x -100 \]\begin{enumerate}[label=\Alph*.]
\item \( z_1 \in [-5.2, -4.3], \text{   }  z_2 \in [-1.11, -0.74], \text{   and   } z_3 \in [-0.75, -0.75] \)
\item \( z_1 \in [1.2, 1.7], \text{   }  z_2 \in [1, 1.43], \text{   and   } z_3 \in [4.9, 5.23] \)
\item \( z_1 \in [-5.2, -4.3], \text{   }  z_2 \in [-4.01, -3.75], \text{   and   } z_3 \in [-0.62, -0.18] \)
\item \( z_1 \in [-0.4, 0.9], \text{   }  z_2 \in [0.2, 1.17], \text{   and   } z_3 \in [4.9, 5.23] \)
\item \( z_1 \in [-5.2, -4.3], \text{   }  z_2 \in [-1.61, -0.86], \text{   and   } z_3 \in [-1.25, -1.23] \)

\end{enumerate} }
\litem{
What are the \textit{possible Rational} roots of the polynomial below?\[ f(x) = 5x^{2} +3 x + 3 \]\begin{enumerate}[label=\Alph*.]
\item \( \text{ All combinations of: }\frac{\pm 1,\pm 5}{\pm 1,\pm 3} \)
\item \( \pm 1,\pm 5 \)
\item \( \text{ All combinations of: }\frac{\pm 1,\pm 3}{\pm 1,\pm 5} \)
\item \( \pm 1,\pm 3 \)
\item \( \text{ There is no formula or theorem that tells us all possible Rational roots.} \)

\end{enumerate} }
\litem{
Perform the division below. Then, find the intervals that correspond to the quotient in the form $ax^2+bx+c$ and remainder $r$.\[ \frac{10x^{3} +52 x^{2} +32 x -66}{x + 4} \]\begin{enumerate}[label=\Alph*.]
\item \( a \in [9, 14], \text{   } b \in [12, 16], \text{   } c \in [-18, -6], \text{   and   } r \in [-2, 3]. \)
\item \( a \in [9, 14], \text{   } b \in [92, 100], \text{   } c \in [398, 401], \text{   and   } r \in [1528, 1540]. \)
\item \( a \in [-42, -38], \text{   } b \in [-108, -107], \text{   } c \in [-405, -395], \text{   and   } r \in [-1671, -1663]. \)
\item \( a \in [-42, -38], \text{   } b \in [209, 213], \text{   } c \in [-819, -815], \text{   and   } r \in [3192, 3199]. \)
\item \( a \in [9, 14], \text{   } b \in [-2, 8], \text{   } c \in [18, 27], \text{   and   } r \in [-179, -173]. \)

\end{enumerate} }
\litem{
Perform the division below. Then, find the intervals that correspond to the quotient in the form $ax^2+bx+c$ and remainder $r$.\[ \frac{10x^{3} +32 x^{2} -82 x + 35}{x + 5} \]\begin{enumerate}[label=\Alph*.]
\item \( a \in [-55, -47], \text{   } b \in [-218, -211], \text{   } c \in [-1175, -1170], \text{   and   } r \in [-5827, -5821]. \)
\item \( a \in [-55, -47], \text{   } b \in [279, 286], \text{   } c \in [-1493, -1489], \text{   and   } r \in [7489, 7500]. \)
\item \( a \in [6, 15], \text{   } b \in [-30, -26], \text{   } c \in [86, 90], \text{   and   } r \in [-486, -477]. \)
\item \( a \in [6, 15], \text{   } b \in [-24, -15], \text{   } c \in [3, 9], \text{   and   } r \in [-5, 0]. \)
\item \( a \in [6, 15], \text{   } b \in [76, 84], \text{   } c \in [324, 329], \text{   and   } r \in [1672, 1677]. \)

\end{enumerate} }
\litem{
Perform the division below. Then, find the intervals that correspond to the quotient in the form $ax^2+bx+c$ and remainder $r$.\[ \frac{9x^{3} -28 x -18}{x -2} \]\begin{enumerate}[label=\Alph*.]
\item \( a \in [9, 10], b \in [-18, -16], c \in [6, 11], \text{ and } r \in [-34.2, -29.9]. \)
\item \( a \in [17, 24], b \in [-43, -33], c \in [38, 45], \text{ and } r \in [-107.2, -102.2]. \)
\item \( a \in [9, 10], b \in [3, 15], c \in [-26, -13], \text{ and } r \in [-37.9, -35]. \)
\item \( a \in [17, 24], b \in [36, 38], c \in [38, 45], \text{ and } r \in [69.9, 71.4]. \)
\item \( a \in [9, 10], b \in [17, 22], c \in [6, 11], \text{ and } r \in [-2.6, -1.3]. \)

\end{enumerate} }
\litem{
Factor the polynomial below completely, knowing that $x+4$ is a factor. Then, choose the intervals the zeros of the polynomial belong to, where $z_1 \leq z_2 \leq z_3 \leq z_4$. \textit{To make the problem easier, all zeros are between -5 and 5.}\[ f(x) = 8x^{4} +14 x^{3} -83 x^{2} -14 x + 120 \]\begin{enumerate}[label=\Alph*.]
\item \( z_1 \in [-3, -0.6], \text{   }  z_2 \in [-1.62, -1.35], z_3 \in [1.21, 1.39], \text{   and   } z_4 \in [2.8, 4.7] \)
\item \( z_1 \in [-3, -0.6], \text{   }  z_2 \in [-0.73, -0.57], z_3 \in [0.78, 0.81], \text{   and   } z_4 \in [2.8, 4.7] \)
\item \( z_1 \in [-4.6, -2.8], \text{   }  z_2 \in [-0.87, -0.74], z_3 \in [0.64, 0.74], \text{   and   } z_4 \in [1.8, 2.8] \)
\item \( z_1 \in [-4.6, -2.8], \text{   }  z_2 \in [-1.28, -1.19], z_3 \in [1.48, 1.52], \text{   and   } z_4 \in [1.8, 2.8] \)
\item \( z_1 \in [-3, -0.6], \text{   }  z_2 \in [-0.39, -0.3], z_3 \in [3.73, 4.05], \text{   and   } z_4 \in [4.6, 6] \)

\end{enumerate} }
\litem{
Factor the polynomial below completely. Then, choose the intervals the zeros of the polynomial belong to, where $z_1 \leq z_2 \leq z_3$. \textit{To make the problem easier, all zeros are between -5 and 5.}\[ f(x) = 8x^{3} -34 x^{2} +45 x -18 \]\begin{enumerate}[label=\Alph*.]
\item \( z_1 \in [0.68, 0.8], \text{   }  z_2 \in [1.41, 1.55], \text{   and   } z_3 \in [1.94, 2.03] \)
\item \( z_1 \in [-3.08, -2.88], \text{   }  z_2 \in [-2.06, -1.89], \text{   and   } z_3 \in [-0.54, -0.29] \)
\item \( z_1 \in [-2.18, -1.77], \text{   }  z_2 \in [-1.57, -1.45], \text{   and   } z_3 \in [-0.85, -0.67] \)
\item \( z_1 \in [0.37, 0.71], \text{   }  z_2 \in [1.31, 1.47], \text{   and   } z_3 \in [1.94, 2.03] \)
\item \( z_1 \in [-2.18, -1.77], \text{   }  z_2 \in [-1.42, -1.29], \text{   and   } z_3 \in [-0.7, -0.66] \)

\end{enumerate} }
\litem{
Perform the division below. Then, find the intervals that correspond to the quotient in the form $ax^2+bx+c$ and remainder $r$.\[ \frac{6x^{3} +21 x^{2} -31}{x + 3} \]\begin{enumerate}[label=\Alph*.]
\item \( a \in [4, 9], b \in [38, 48], c \in [110, 119], \text{ and } r \in [316, 323]. \)
\item \( a \in [4, 9], b \in [-3, -1], c \in [11, 14], \text{ and } r \in [-84, -77]. \)
\item \( a \in [4, 9], b \in [-2, 5], c \in [-11, -7], \text{ and } r \in [-4, 1]. \)
\item \( a \in [-21, -16], b \in [-35, -27], c \in [-99, -97], \text{ and } r \in [-329, -318]. \)
\item \( a \in [-21, -16], b \in [74, 80], c \in [-228, -223], \text{ and } r \in [641, 648]. \)

\end{enumerate} }
\litem{
Factor the polynomial below completely, knowing that $x+3$ is a factor. Then, choose the intervals the zeros of the polynomial belong to, where $z_1 \leq z_2 \leq z_3 \leq z_4$. \textit{To make the problem easier, all zeros are between -5 and 5.}\[ f(x) = 8x^{4} +2 x^{3} -147 x^{2} -288 x -135 \]\begin{enumerate}[label=\Alph*.]
\item \( z_1 \in [-3, -2], \text{   }  z_2 \in [-1.37, -1.31], z_3 \in [-0.72, -0.47], \text{   and   } z_4 \in [4.05, 5.38] \)
\item \( z_1 \in [-3, -2], \text{   }  z_2 \in [-1.5, -1.47], z_3 \in [-0.87, -0.68], \text{   and   } z_4 \in [4.05, 5.38] \)
\item \( z_1 \in [-5, -4], \text{   }  z_2 \in [0.36, 0.4], z_3 \in [2.73, 3.27], \text{   and   } z_4 \in [2.23, 3.99] \)
\item \( z_1 \in [-5, -4], \text{   }  z_2 \in [0.65, 0.68], z_3 \in [1, 1.4], \text{   and   } z_4 \in [2.23, 3.99] \)
\item \( z_1 \in [-5, -4], \text{   }  z_2 \in [0.75, 0.77], z_3 \in [1.42, 1.72], \text{   and   } z_4 \in [2.23, 3.99] \)

\end{enumerate} }
\end{enumerate}

\end{document}