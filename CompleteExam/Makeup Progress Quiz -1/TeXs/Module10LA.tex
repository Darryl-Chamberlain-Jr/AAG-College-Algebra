\documentclass[14pt]{extbook}
\usepackage{multicol, enumerate, enumitem, hyperref, color, soul, setspace, parskip, fancyhdr} %General Packages
\usepackage{amssymb, amsthm, amsmath, bbm, latexsym, units, mathtools} %Math Packages
\everymath{\displaystyle} %All math in Display Style
% Packages with additional options
\usepackage[headsep=0.5cm,headheight=12pt, left=1 in,right= 1 in,top= 1 in,bottom= 1 in]{geometry}
\usepackage[usenames,dvipsnames]{xcolor}
\usepackage{dashrule}  % Package to use the command below to create lines between items
\newcommand{\litem}[1]{\item#1\hspace*{-1cm}\rule{\textwidth}{0.4pt}}
\pagestyle{fancy}
\lhead{Makeup Progress Quiz -1}
\chead{}
\rhead{Version A}
\lfoot{7547-2949}
\cfoot{}
\rfoot{Fall 2020}
\begin{document}

\begin{enumerate}
\litem{
What are the \textit{possible Integer} roots of the polynomial below?\[ f(x) = 2x^{3} +7 x^{2} +5 x + 3 \]\begin{enumerate}[label=\Alph*.]
\item \( \pm 1,\pm 3 \)
\item \( \text{ All combinations of: }\frac{\pm 1,\pm 3}{\pm 1,\pm 2} \)
\item \( \pm 1,\pm 2 \)
\item \( \text{ All combinations of: }\frac{\pm 1,\pm 2}{\pm 1,\pm 3} \)
\item \( \text{There is no formula or theorem that tells us all possible Integer roots.} \)

\end{enumerate} }
\litem{
Factor the polynomial below completely. Then, choose the intervals the zeros of the polynomial belong to, where $z_1 \leq z_2 \leq z_3$. \textit{To make the problem easier, all zeros are between -5 and 5.}\[ f(x) = 10x^{3} -11 x^{2} -106 x -40 \]\begin{enumerate}[label=\Alph*.]
\item \( z_1 \in [-2.5, -0.5], \text{   }  z_2 \in [-0.49, -0.31], \text{   and   } z_3 \in [2.6, 4.9] \)
\item \( z_1 \in [-6, -3], \text{   }  z_2 \in [0.21, 0.47], \text{   and   } z_3 \in [0.9, 2.8] \)
\item \( z_1 \in [-6, -3], \text{   }  z_2 \in [0.21, 0.47], \text{   and   } z_3 \in [0.9, 2.8] \)
\item \( z_1 \in [-2.5, -0.5], \text{   }  z_2 \in [-0.49, -0.31], \text{   and   } z_3 \in [2.6, 4.9] \)
\item \( z_1 \in [-6, -3], \text{   }  z_2 \in [0, 0.2], \text{   and   } z_3 \in [4.4, 5.1] \)

\end{enumerate} }
\litem{
What are the \textit{possible Rational} roots of the polynomial below?\[ f(x) = 4x^{3} +7 x^{2} +5 x + 7 \]\begin{enumerate}[label=\Alph*.]
\item \( \pm 1,\pm 2,\pm 4 \)
\item \( \text{ All combinations of: }\frac{\pm 1,\pm 2,\pm 4}{\pm 1,\pm 7} \)
\item \( \text{ All combinations of: }\frac{\pm 1,\pm 7}{\pm 1,\pm 2,\pm 4} \)
\item \( \pm 1,\pm 7 \)
\item \( \text{ There is no formula or theorem that tells us all possible Rational roots.} \)

\end{enumerate} }
\litem{
Perform the division below. Then, find the intervals that correspond to the quotient in the form $ax^2+bx+c$ and remainder $r$.\[ \frac{4x^{3} +10 x^{2} -18 x -41}{x + 3} \]\begin{enumerate}[label=\Alph*.]
\item \( a \in [-16, -10], \text{   } b \in [43, 47], \text{   } c \in [-158, -153], \text{   and   } r \in [424, 434]. \)
\item \( a \in [2, 5], \text{   } b \in [-4, 2], \text{   } c \in [-15, -7], \text{   and   } r \in [-8, -1]. \)
\item \( a \in [2, 5], \text{   } b \in [-13, -3], \text{   } c \in [5, 12], \text{   and   } r \in [-68, -64]. \)
\item \( a \in [-16, -10], \text{   } b \in [-27, -18], \text{   } c \in [-97, -90], \text{   and   } r \in [-331, -326]. \)
\item \( a \in [2, 5], \text{   } b \in [21, 25], \text{   } c \in [48, 51], \text{   and   } r \in [103, 109]. \)

\end{enumerate} }
\litem{
Perform the division below. Then, find the intervals that correspond to the quotient in the form $ax^2+bx+c$ and remainder $r$.\[ \frac{25x^{3} -25 x^{2} -125 x -72}{x -3} \]\begin{enumerate}[label=\Alph*.]
\item \( a \in [21, 30], \text{   } b \in [-103, -97], \text{   } c \in [170, 177], \text{   and   } r \in [-597, -595]. \)
\item \( a \in [72, 79], \text{   } b \in [-252, -246], \text{   } c \in [622, 630], \text{   and   } r \in [-1949, -1943]. \)
\item \( a \in [72, 79], \text{   } b \in [199, 204], \text{   } c \in [470, 477], \text{   and   } r \in [1347, 1356]. \)
\item \( a \in [21, 30], \text{   } b \in [19, 30], \text{   } c \in [-76, -74], \text{   and   } r \in [-224, -221]. \)
\item \( a \in [21, 30], \text{   } b \in [47, 55], \text{   } c \in [24, 27], \text{   and   } r \in [1, 7]. \)

\end{enumerate} }
\litem{
Perform the division below. Then, find the intervals that correspond to the quotient in the form $ax^2+bx+c$ and remainder $r$.\[ \frac{15x^{3} +65 x^{2} -82}{x + 4} \]\begin{enumerate}[label=\Alph*.]
\item \( a \in [-61, -54], b \in [305, 306], c \in [-1229, -1212], \text{ and } r \in [4794, 4805]. \)
\item \( a \in [10, 19], b \in [-10, -5], c \in [46, 54], \text{ and } r \in [-334, -325]. \)
\item \( a \in [10, 19], b \in [3, 7], c \in [-22, -17], \text{ and } r \in [-3, -1]. \)
\item \( a \in [-61, -54], b \in [-181, -171], c \in [-705, -698], \text{ and } r \in [-2885, -2879]. \)
\item \( a \in [10, 19], b \in [121, 130], c \in [500, 507], \text{ and } r \in [1913, 1921]. \)

\end{enumerate} }
\litem{
Factor the polynomial below completely, knowing that $x+3$ is a factor. Then, choose the intervals the zeros of the polynomial belong to, where $z_1 \leq z_2 \leq z_3 \leq z_4$. \textit{To make the problem easier, all zeros are between -5 and 5.}\[ f(x) = 10x^{4} +77 x^{3} +157 x^{2} -144 \]\begin{enumerate}[label=\Alph*.]
\item \( z_1 \in [-4.47, -3.95], \text{   }  z_2 \in [-3.43, -1.5], z_3 \in [-2.16, -1.33], \text{   and   } z_4 \in [0.55, 0.98] \)
\item \( z_1 \in [-0.43, -0.27], \text{   }  z_2 \in [2.29, 3.4], z_3 \in [2.32, 3.36], \text{   and   } z_4 \in [3.5, 4.38] \)
\item \( z_1 \in [-1.37, -1.06], \text{   }  z_2 \in [0.46, 0.91], z_3 \in [2.32, 3.36], \text{   and   } z_4 \in [3.5, 4.38] \)
\item \( z_1 \in [-0.85, -0.49], \text{   }  z_2 \in [1.26, 1.77], z_3 \in [2.32, 3.36], \text{   and   } z_4 \in [3.5, 4.38] \)
\item \( z_1 \in [-4.47, -3.95], \text{   }  z_2 \in [-3.43, -1.5], z_3 \in [-1.04, -0.04], \text{   and   } z_4 \in [1.01, 2.3] \)

\end{enumerate} }
\litem{
Factor the polynomial below completely. Then, choose the intervals the zeros of the polynomial belong to, where $z_1 \leq z_2 \leq z_3$. \textit{To make the problem easier, all zeros are between -5 and 5.}\[ f(x) = 25x^{3} -95 x^{2} -26 x + 24 \]\begin{enumerate}[label=\Alph*.]
\item \( z_1 \in [-5.8, -2.8], \text{   }  z_2 \in [-0.61, -0.32], \text{   and   } z_3 \in [0.3, 1.4] \)
\item \( z_1 \in [-5.8, -2.8], \text{   }  z_2 \in [-2.15, -1.87], \text{   and   } z_3 \in [0, 0.5] \)
\item \( z_1 \in [-1.9, -0.8], \text{   }  z_2 \in [2.17, 2.85], \text{   and   } z_3 \in [2.9, 5] \)
\item \( z_1 \in [-5.8, -2.8], \text{   }  z_2 \in [-2.88, -2.26], \text{   and   } z_3 \in [1.5, 2.3] \)
\item \( z_1 \in [-1, 1], \text{   }  z_2 \in [-0.28, 0.53], \text{   and   } z_3 \in [2.9, 5] \)

\end{enumerate} }
\litem{
Perform the division below. Then, find the intervals that correspond to the quotient in the form $ax^2+bx+c$ and remainder $r$.\[ \frac{16x^{3} -84 x^{2} + 102}{x -5} \]\begin{enumerate}[label=\Alph*.]
\item \( a \in [79, 82], b \in [-485, -478], c \in [2415, 2426], \text{ and } r \in [-11998, -11992]. \)
\item \( a \in [79, 82], b \in [312, 321], c \in [1576, 1582], \text{ and } r \in [8002, 8007]. \)
\item \( a \in [14, 20], b \in [-22, -19], c \in [-84, -79], \text{ and } r \in [-219, -214]. \)
\item \( a \in [14, 20], b \in [-166, -162], c \in [818, 822], \text{ and } r \in [-4008, -3996]. \)
\item \( a \in [14, 20], b \in [-7, -2], c \in [-23, -17], \text{ and } r \in [-6, 5]. \)

\end{enumerate} }
\litem{
Factor the polynomial below completely, knowing that $x+3$ is a factor. Then, choose the intervals the zeros of the polynomial belong to, where $z_1 \leq z_2 \leq z_3 \leq z_4$. \textit{To make the problem easier, all zeros are between -5 and 5.}\[ f(x) = 25x^{4} +5 x^{3} -231 x^{2} -45 x + 54 \]\begin{enumerate}[label=\Alph*.]
\item \( z_1 \in [-4, 1], \text{   }  z_2 \in [-0.51, -0.34], z_3 \in [0.48, 0.82], \text{   and   } z_4 \in [-2, 7] \)
\item \( z_1 \in [-4, 1], \text{   }  z_2 \in [-2.54, -2.36], z_3 \in [1.66, 1.68], \text{   and   } z_4 \in [-2, 7] \)
\item \( z_1 \in [-4, 1], \text{   }  z_2 \in [-0.38, 0.25], z_3 \in [2.84, 3.14], \text{   and   } z_4 \in [-2, 7] \)
\item \( z_1 \in [-4, 1], \text{   }  z_2 \in [-1.84, -1.58], z_3 \in [2.45, 2.54], \text{   and   } z_4 \in [-2, 7] \)
\item \( z_1 \in [-4, 1], \text{   }  z_2 \in [-0.72, -0.47], z_3 \in [0.36, 0.52], \text{   and   } z_4 \in [-2, 7] \)

\end{enumerate} }
\end{enumerate}

\end{document}