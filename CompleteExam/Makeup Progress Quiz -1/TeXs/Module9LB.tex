\documentclass[14pt]{extbook}
\usepackage{multicol, enumerate, enumitem, hyperref, color, soul, setspace, parskip, fancyhdr} %General Packages
\usepackage{amssymb, amsthm, amsmath, bbm, latexsym, units, mathtools} %Math Packages
\everymath{\displaystyle} %All math in Display Style
% Packages with additional options
\usepackage[headsep=0.5cm,headheight=12pt, left=1 in,right= 1 in,top= 1 in,bottom= 1 in]{geometry}
\usepackage[usenames,dvipsnames]{xcolor}
\usepackage{dashrule}  % Package to use the command below to create lines between items
\newcommand{\litem}[1]{\item#1\hspace*{-1cm}\rule{\textwidth}{0.4pt}}
\pagestyle{fancy}
\lhead{Makeup Progress Quiz -1}
\chead{}
\rhead{Version B}
\lfoot{7547-2949}
\cfoot{}
\rfoot{Fall 2020}
\begin{document}

\begin{enumerate}
\litem{
Find the inverse of the function below. Then, evaluate the inverse at $x = 9$ and choose the interval that $f^{-1}(9)$ belongs to.\[ f(x) = \ln{(x+2)}+4 \]\begin{enumerate}[label=\Alph*.]
\item \( f^{-1}(9) \in [135.41, 149.41] \)
\item \( f^{-1}(9) \in [59871.14, 59879.14] \)
\item \( f^{-1}(9) \in [442410.39, 442412.39] \)
\item \( f^{-1}(9) \in [148.41, 154.41] \)
\item \( f^{-1}(9) \in [1096.63, 1103.63] \)

\end{enumerate} }
\litem{
Add the following functions, then choose the domain of the resulting function from the list below.\[ f(x) = 4x^{4} +4 x^{3} +3 x^{2} +x + 5 \text{ and } g(x) = \frac{5}{4x+25} \]\begin{enumerate}[label=\Alph*.]
\item \( \text{ The domain is all Real numbers except } x = a, \text{ where } a \in [-8.25, -3.25] \)
\item \( \text{ The domain is all Real numbers less than or equal to } x = a, \text{ where } a \in [-9.4, -0.4] \)
\item \( \text{ The domain is all Real numbers greater than or equal to } x = a, \text{ where } a \in [-8.6, -4.6] \)
\item \( \text{ The domain is all Real numbers except } x = a \text{ and } x = b, \text{ where } a \in [-7.25, -0.25] \text{ and } b \in [-8.4, -5.4] \)
\item \( \text{ The domain is all Real numbers. } \)

\end{enumerate} }
\litem{
Choose the interval below that $f$ composed with $g$ at $x=1$ is in.\[ f(x) = -2x^{3} -2 x^{2} +4 x -1 \text{ and } g(x) = -4x^{3} -1 x^{2} +3 x \]\begin{enumerate}[label=\Alph*.]
\item \( (f \circ g)(1) \in [3.41, 4.26] \)
\item \( (f \circ g)(1) \in [-1.1, -0.45] \)
\item \( (f \circ g)(1) \in [-0.14, 0.31] \)
\item \( (f \circ g)(1) \in [7.8, 8.24] \)
\item \( \text{It is not possible to compose the two functions.} \)

\end{enumerate} }
\litem{
Find the inverse of the function below (if it exists). Then, evaluate the inverse at $x = -15$ and choose the interval that $f^{-1}(-15)$ belongs to.\[ f(x) = 2 x^2 - 3 \]\begin{enumerate}[label=\Alph*.]
\item \( f^{-1}(-15) \in [4.29, 4.73] \)
\item \( f^{-1}(-15) \in [1.98, 2.74] \)
\item \( f^{-1}(-15) \in [4.96, 5.57] \)
\item \( f^{-1}(-15) \in [2.51, 3.02] \)
\item \( \text{ The function is not invertible for all Real numbers. } \)

\end{enumerate} }
\litem{
Multiply the following functions, then choose the domain of the resulting function from the list below.\[ f(x) = 2x^{3} +8 x^{2} +x + 3 \text{ and } g(x) = x^{2} + 1 \]\begin{enumerate}[label=\Alph*.]
\item \( \text{ The domain is all Real numbers less than or equal to } x = a, \text{ where } a \in [-3.67, -1.67] \)
\item \( \text{ The domain is all Real numbers except } x = a, \text{ where } a \in [-6.67, -0.67] \)
\item \( \text{ The domain is all Real numbers greater than or equal to } x = a, \text{ where } a \in [-10, -3] \)
\item \( \text{ The domain is all Real numbers except } x = a \text{ and } x = b, \text{ where } a \in [-6.2, -5.2] \text{ and } b \in [-8.8, -4.8] \)
\item \( \text{ The domain is all Real numbers. } \)

\end{enumerate} }
\litem{
Determine whether the function below is 1-1.\[ f(x) = (6 x + 40)^3 \]\begin{enumerate}[label=\Alph*.]
\item \( \text{No, because the range of the function is not $(-\infty, \infty)$.} \)
\item \( \text{No, because there is a $y$-value that goes to 2 different $x$-values.} \)
\item \( \text{Yes, the function is 1-1.} \)
\item \( \text{No, because the domain of the function is not $(-\infty, \infty)$.} \)
\item \( \text{No, because there is an $x$-value that goes to 2 different $y$-values.} \)

\end{enumerate} }
\litem{
Choose the interval below that $f$ composed with $g$ at $x=-2$ is in.\[ f(x) = -3x^{3} -4 x^{2} +3 x \text{ and } g(x) = -2x^{3} -2 x^{2} +3 x \]\begin{enumerate}[label=\Alph*.]
\item \( (f \circ g)(-2) \in [-39, -31] \)
\item \( (f \circ g)(-2) \in [-29, -24] \)
\item \( (f \circ g)(-2) \in [-44, -39] \)
\item \( (f \circ g)(-2) \in [-20, -16] \)
\item \( \text{It is not possible to compose the two functions.} \)

\end{enumerate} }
\litem{
Find the inverse of the function below (if it exists). Then, evaluate the inverse at $x = 12$ and choose the interval the $f^{-1}(12)$ belongs to.\[ f(x) = \sqrt[3]{4 x + 5} \]\begin{enumerate}[label=\Alph*.]
\item \( f^{-1}(12) \in [-432.1, -430.2] \)
\item \( f^{-1}(12) \in [433, 433.6] \)
\item \( f^{-1}(12) \in [429.5, 432.6] \)
\item \( f^{-1}(12) \in [-435.8, -432.9] \)
\item \( \text{ The function is not invertible for all Real numbers. } \)

\end{enumerate} }
\litem{
Determine whether the function below is 1-1.\[ f(x) = (5 x + 30)^3 \]\begin{enumerate}[label=\Alph*.]
\item \( \text{No, because the range of the function is not $(-\infty, \infty)$.} \)
\item \( \text{No, because the domain of the function is not $(-\infty, \infty)$.} \)
\item \( \text{No, because there is an $x$-value that goes to 2 different $y$-values.} \)
\item \( \text{Yes, the function is 1-1.} \)
\item \( \text{No, because there is a $y$-value that goes to 2 different $x$-values.} \)

\end{enumerate} }
\litem{
Find the inverse of the function below. Then, evaluate the inverse at $x = 7$ and choose the interval that $f^{-1}(7)$ belongs to.\[ f(x) = \ln{(x-3)}+2 \]\begin{enumerate}[label=\Alph*.]
\item \( f^{-1}(7) \in [145.2, 146.4] \)
\item \( f^{-1}(7) \in [22028.1, 22028.9] \)
\item \( f^{-1}(7) \in [52.3, 58.4] \)
\item \( f^{-1}(7) \in [8104.6, 8107.1] \)
\item \( f^{-1}(7) \in [148.5, 152.3] \)

\end{enumerate} }
\end{enumerate}

\end{document}