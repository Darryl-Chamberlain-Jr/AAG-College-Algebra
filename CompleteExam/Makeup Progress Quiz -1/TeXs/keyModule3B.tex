\documentclass{extbook}[14pt]
\usepackage{multicol, enumerate, enumitem, hyperref, color, soul, setspace, parskip, fancyhdr, amssymb, amsthm, amsmath, bbm, latexsym, units, mathtools}
\everymath{\displaystyle}
\usepackage[headsep=0.5cm,headheight=0cm, left=1 in,right= 1 in,top= 1 in,bottom= 1 in]{geometry}
\usepackage{dashrule}  % Package to use the command below to create lines between items
\newcommand{\litem}[1]{\item #1

\rule{\textwidth}{0.4pt}}
\pagestyle{fancy}
\lhead{}
\chead{Answer Key for Makeup Progress Quiz -1 Version B}
\rhead{}
\lfoot{7547-2949}
\cfoot{}
\rfoot{Fall 2020}
\begin{document}
\textbf{This key should allow you to understand why you choose the option you did (beyond just getting a question right or wrong). \href{https://xronos.clas.ufl.edu/mac1105spring2020/courseDescriptionAndMisc/Exams/LearningFromResults}{More instructions on how to use this key can be found here}.}

\textbf{If you have a suggestion to make the keys better, \href{https://forms.gle/CZkbZmPbC9XALEE88}{please fill out the short survey here}.}

\textit{Note: This key is auto-generated and may contain issues and/or errors. The keys are reviewed after each exam to ensure grading is done accurately. If there are issues (like duplicate options), they are noted in the offline gradebook. The keys are a work-in-progress to give students as many resources to improve as possible.}

\rule{\textwidth}{0.4pt}

\begin{enumerate}\litem{
Solve the linear inequality below. Then, choose the constant and interval combination that describes the solution set.
\[ 5 - 8 x \leq \frac{-16 x - 9}{4} < 9 - 5 x \]

The solution is \( [1.81, 11.25) \), which is option A.\begin{enumerate}[label=\Alph*.]
\item \( [a, b), \text{ where } a \in [-1.19, 3.81] \text{ and } b \in [11.25, 16.25] \)

$[1.81, 11.25)$, which is the correct option.
\item \( (-\infty, a] \cup (b, \infty), \text{ where } a \in [1.81, 2.81] \text{ and } b \in [10.25, 12.25] \)

$(-\infty, 1.81] \cup (11.25, \infty)$, which corresponds to displaying the and-inequality as an or-inequality.
\item \( (-\infty, a) \cup [b, \infty), \text{ where } a \in [0.81, 3.81] \text{ and } b \in [11.25, 15.25] \)

$(-\infty, 1.81) \cup [11.25, \infty)$, which corresponds to displaying the and-inequality as an or-inequality AND flipping the inequality.
\item \( (a, b], \text{ where } a \in [0.81, 4.81] \text{ and } b \in [11.25, 16.25] \)

$(1.81, 11.25]$, which corresponds to flipping the inequality.
\item \( \text{None of the above.} \)


\end{enumerate}

\textbf{General Comment:} To solve, you will need to break up the compound inequality into two inequalities. Be sure to keep track of the inequality! It may be best to draw a number line and graph your solution.
}
\litem{
Solve the linear inequality below. Then, choose the constant and interval combination that describes the solution set.
\[ -9x -8 \leq 10x -9 \]

The solution is \( [0.053, \infty) \), which is option A.\begin{enumerate}[label=\Alph*.]
\item \( [a, \infty), \text{ where } a \in [-0.05, 0.23] \)

* $[0.053, \infty)$, which is the correct option.
\item \( (-\infty, a], \text{ where } a \in [-0.05, 0.22] \)

 $(-\infty, 0.053]$, which corresponds to switching the direction of the interval. You likely did this if you did not flip the inequality when dividing by a negative!
\item \( (-\infty, a], \text{ where } a \in [-0.14, -0.03] \)

 $(-\infty, -0.053]$, which corresponds to switching the direction of the interval AND negating the endpoint. You likely did this if you did not flip the inequality when dividing by a negative as well as not moving values over to a side properly.
\item \( [a, \infty), \text{ where } a \in [-0.1, 0.05] \)

 $[-0.053, \infty)$, which corresponds to negating the endpoint of the solution.
\item \( \text{None of the above}. \)

You may have chosen this if you thought the inequality did not match the ends of the intervals.
\end{enumerate}

\textbf{General Comment:} Remember that less/greater than or equal to includes the endpoint, while less/greater do not. Also, remember that you need to flip the inequality when you multiply or divide by a negative.
}
\litem{
Solve the linear inequality below. Then, choose the constant and interval combination that describes the solution set.
\[ \frac{3}{9} + \frac{3}{7} x \geq \frac{10}{8} x - \frac{8}{6} \]

The solution is \( (-\infty, 2.029] \), which is option C.\begin{enumerate}[label=\Alph*.]
\item \( (-\infty, a], \text{ where } a \in [-2.03, -0.03] \)

 $(-\infty, -2.029]$, which corresponds to negating the endpoint of the solution.
\item \( [a, \infty), \text{ where } a \in [-0.97, 3.03] \)

 $[2.029, \infty)$, which corresponds to switching the direction of the interval. You likely did this if you did not flip the inequality when dividing by a negative!
\item \( (-\infty, a], \text{ where } a \in [1.03, 4.03] \)

* $(-\infty, 2.029]$, which is the correct option.
\item \( [a, \infty), \text{ where } a \in [-3.03, -0.03] \)

 $[-2.029, \infty)$, which corresponds to switching the direction of the interval AND negating the endpoint. You likely did this if you did not flip the inequality when dividing by a negative as well as not moving values over to a side properly.
\item \( \text{None of the above}. \)

You may have chosen this if you thought the inequality did not match the ends of the intervals.
\end{enumerate}

\textbf{General Comment:} Remember that less/greater than or equal to includes the endpoint, while less/greater do not. Also, remember that you need to flip the inequality when you multiply or divide by a negative.
}
\litem{
Solve the linear inequality below. Then, choose the constant and interval combination that describes the solution set.
\[ 3 + 6 x > 8 x \text{ or } 8 + 4 x < 6 x \]

The solution is \( (-\infty, 1.5) \text{ or } (4.0, \infty) \), which is option C.\begin{enumerate}[label=\Alph*.]
\item \( (-\infty, a) \cup (b, \infty), \text{ where } a \in [-5, 1] \text{ and } b \in [-2.5, -0.5] \)

Corresponds to inverting the inequality and negating the solution.
\item \( (-\infty, a] \cup [b, \infty), \text{ where } a \in [-5, -2] \text{ and } b \in [-2.5, 1.5] \)

Corresponds to including the endpoints AND negating.
\item \( (-\infty, a) \cup (b, \infty), \text{ where } a \in [0.5, 2.5] \text{ and } b \in [3, 8] \)

 * Correct option.
\item \( (-\infty, a] \cup [b, \infty), \text{ where } a \in [-0.5, 3.5] \text{ and } b \in [2, 5] \)

Corresponds to including the endpoints (when they should be excluded).
\item \( (-\infty, \infty) \)

Corresponds to the variable canceling, which does not happen in this instance.
\end{enumerate}

\textbf{General Comment:} When multiplying or dividing by a negative, flip the sign.
}
\litem{
Using an interval or intervals, describe all the $x$-values within or including a distance of the given values.
\[ \text{ Less than } 9 \text{ units from the number } 7. \]

The solution is \( (-2, 16) \), which is option A.\begin{enumerate}[label=\Alph*.]
\item \( (-2, 16) \)

This describes the values less than 9 from 7
\item \( (-\infty, -2] \cup [16, \infty) \)

This describes the values no less than 9 from 7
\item \( [-2, 16] \)

This describes the values no more than 9 from 7
\item \( (-\infty, -2) \cup (16, \infty) \)

This describes the values more than 9 from 7
\item \( \text{None of the above} \)

You likely thought the values in the interval were not correct.
\end{enumerate}

\textbf{General Comment:} When thinking about this language, it helps to draw a number line and try points.
}
\litem{
Solve the linear inequality below. Then, choose the constant and interval combination that describes the solution set.
\[ -9 + 6 x > 8 x \text{ or } 4 + 5 x < 7 x \]

The solution is \( (-\infty, -4.5) \text{ or } (2.0, \infty) \), which is option D.\begin{enumerate}[label=\Alph*.]
\item \( (-\infty, a) \cup (b, \infty), \text{ where } a \in [-2, 3] \text{ and } b \in [2.5, 8.5] \)

Corresponds to inverting the inequality and negating the solution.
\item \( (-\infty, a] \cup [b, \infty), \text{ where } a \in [-5.5, -2.5] \text{ and } b \in [0.5, 3.7] \)

Corresponds to including the endpoints (when they should be excluded).
\item \( (-\infty, a] \cup [b, \infty), \text{ where } a \in [-2.8, 1.5] \text{ and } b \in [3, 4.7] \)

Corresponds to including the endpoints AND negating.
\item \( (-\infty, a) \cup (b, \infty), \text{ where } a \in [-8.5, -2.5] \text{ and } b \in [-5, 4] \)

 * Correct option.
\item \( (-\infty, \infty) \)

Corresponds to the variable canceling, which does not happen in this instance.
\end{enumerate}

\textbf{General Comment:} When multiplying or dividing by a negative, flip the sign.
}
\litem{
Solve the linear inequality below. Then, choose the constant and interval combination that describes the solution set.
\[ \frac{7}{3} - \frac{3}{5} x < \frac{10}{7} x - \frac{5}{4} \]

The solution is \( (1.766, \infty) \), which is option C.\begin{enumerate}[label=\Alph*.]
\item \( (a, \infty), \text{ where } a \in [-2.77, 0.23] \)

 $(-1.766, \infty)$, which corresponds to negating the endpoint of the solution.
\item \( (-\infty, a), \text{ where } a \in [-0.23, 4.77] \)

 $(-\infty, 1.766)$, which corresponds to switching the direction of the interval. You likely did this if you did not flip the inequality when dividing by a negative!
\item \( (a, \infty), \text{ where } a \in [0.77, 3.77] \)

* $(1.766, \infty)$, which is the correct option.
\item \( (-\infty, a), \text{ where } a \in [-2.77, 1.23] \)

 $(-\infty, -1.766)$, which corresponds to switching the direction of the interval AND negating the endpoint. You likely did this if you did not flip the inequality when dividing by a negative as well as not moving values over to a side properly.
\item \( \text{None of the above}. \)

You may have chosen this if you thought the inequality did not match the ends of the intervals.
\end{enumerate}

\textbf{General Comment:} Remember that less/greater than or equal to includes the endpoint, while less/greater do not. Also, remember that you need to flip the inequality when you multiply or divide by a negative.
}
\litem{
Solve the linear inequality below. Then, choose the constant and interval combination that describes the solution set.
\[ -6 + 8 x \leq \frac{68 x - 3}{8} < -5 + 6 x \]

The solution is \( \text{None of the above.} \), which is option E.\begin{enumerate}[label=\Alph*.]
\item \( (a, b], \text{ where } a \in [10.25, 12.25] \text{ and } b \in [0.85, 3.85] \)

$(11.25, 1.85]$, which corresponds to flipping the inequality and getting negatives of the actual endpoints.
\item \( (-\infty, a) \cup [b, \infty), \text{ where } a \in [10.25, 13.25] \text{ and } b \in [1, 3.6] \)

$(-\infty, 11.25) \cup [1.85, \infty)$, which corresponds to displaying the and-inequality as an or-inequality AND flipping the inequality AND getting negatives of the actual endpoints.
\item \( (-\infty, a] \cup (b, \infty), \text{ where } a \in [10.25, 13.25] \text{ and } b \in [-1.15, 4.85] \)

$(-\infty, 11.25] \cup (1.85, \infty)$, which corresponds to displaying the and-inequality as an or-inequality and getting negatives of the actual endpoints.
\item \( [a, b), \text{ where } a \in [10.25, 14.25] \text{ and } b \in [-0.15, 5.85] \)

$[11.25, 1.85)$, which is the correct interval but negatives of the actual endpoints.
\item \( \text{None of the above.} \)

* This is correct as the answer should be $[-11.25, -1.85)$.
\end{enumerate}

\textbf{General Comment:} To solve, you will need to break up the compound inequality into two inequalities. Be sure to keep track of the inequality! It may be best to draw a number line and graph your solution.
}
\litem{
Solve the linear inequality below. Then, choose the constant and interval combination that describes the solution set.
\[ -10x + 10 < -8x -8 \]

The solution is \( (9.0, \infty) \), which is option A.\begin{enumerate}[label=\Alph*.]
\item \( (a, \infty), \text{ where } a \in [8, 16] \)

* $(9.0, \infty)$, which is the correct option.
\item \( (-\infty, a), \text{ where } a \in [-15, -7] \)

 $(-\infty, -9.0)$, which corresponds to switching the direction of the interval AND negating the endpoint. You likely did this if you did not flip the inequality when dividing by a negative as well as not moving values over to a side properly.
\item \( (a, \infty), \text{ where } a \in [-11, -2] \)

 $(-9.0, \infty)$, which corresponds to negating the endpoint of the solution.
\item \( (-\infty, a), \text{ where } a \in [7, 14] \)

 $(-\infty, 9.0)$, which corresponds to switching the direction of the interval. You likely did this if you did not flip the inequality when dividing by a negative!
\item \( \text{None of the above}. \)

You may have chosen this if you thought the inequality did not match the ends of the intervals.
\end{enumerate}

\textbf{General Comment:} Remember that less/greater than or equal to includes the endpoint, while less/greater do not. Also, remember that you need to flip the inequality when you multiply or divide by a negative.
}
\litem{
Using an interval or intervals, describe all the $x$-values within or including a distance of the given values.
\[ \text{ Less than } 2 \text{ units from the number } -1. \]

The solution is \( (-3, 1) \), which is option C.\begin{enumerate}[label=\Alph*.]
\item \( (-\infty, -3) \cup (1, \infty) \)

This describes the values more than 2 from -1
\item \( [-3, 1] \)

This describes the values no more than 2 from -1
\item \( (-3, 1) \)

This describes the values less than 2 from -1
\item \( (-\infty, -3] \cup [1, \infty) \)

This describes the values no less than 2 from -1
\item \( \text{None of the above} \)

You likely thought the values in the interval were not correct.
\end{enumerate}

\textbf{General Comment:} When thinking about this language, it helps to draw a number line and try points.
}
\end{enumerate}

\end{document}