\documentclass[14pt]{extbook}
\usepackage{multicol, enumerate, enumitem, hyperref, color, soul, setspace, parskip, fancyhdr} %General Packages
\usepackage{amssymb, amsthm, amsmath, bbm, latexsym, units, mathtools} %Math Packages
\everymath{\displaystyle} %All math in Display Style
% Packages with additional options
\usepackage[headsep=0.5cm,headheight=12pt, left=1 in,right= 1 in,top= 1 in,bottom= 1 in]{geometry}
\usepackage[usenames,dvipsnames]{xcolor}
\usepackage{dashrule}  % Package to use the command below to create lines between items
\newcommand{\litem}[1]{\item#1\hspace*{-1cm}\rule{\textwidth}{0.4pt}}
\pagestyle{fancy}
\lhead{Makeup Progress Quiz -1}
\chead{}
\rhead{Version B}
\lfoot{7547-2949}
\cfoot{}
\rfoot{Fall 2020}
\begin{document}

\begin{enumerate}
\litem{
What are the \textit{possible Rational} roots of the polynomial below?\[ f(x) = 7x^{4} +6 x^{3} +4 x^{2} +4 x + 5 \]\begin{enumerate}[label=\Alph*.]
\item \( \pm 1,\pm 5 \)
\item \( \text{ All combinations of: }\frac{\pm 1,\pm 7}{\pm 1,\pm 5} \)
\item \( \text{ All combinations of: }\frac{\pm 1,\pm 5}{\pm 1,\pm 7} \)
\item \( \pm 1,\pm 7 \)
\item \( \text{ There is no formula or theorem that tells us all possible Rational roots.} \)

\end{enumerate} }
\litem{
Factor the polynomial below completely. Then, choose the intervals the zeros of the polynomial belong to, where $z_1 \leq z_2 \leq z_3$. \textit{To make the problem easier, all zeros are between -5 and 5.}\[ f(x) = 15x^{3} -49 x^{2} +44 x -12 \]\begin{enumerate}[label=\Alph*.]
\item \( z_1 \in [-2.1, -1], \text{   }  z_2 \in [-1.14, -0.19], \text{   and   } z_3 \in [-1.08, -0.56] \)
\item \( z_1 \in [-0.1, 0.9], \text{   }  z_2 \in [0.07, 0.97], \text{   and   } z_3 \in [1.54, 2.39] \)
\item \( z_1 \in [-2.1, -1], \text{   }  z_2 \in [-2.02, -1.81], \text{   and   } z_3 \in [-0.39, -0.08] \)
\item \( z_1 \in [-2.1, -1], \text{   }  z_2 \in [-1.73, -1.51], \text{   and   } z_3 \in [-1.6, -1.14] \)
\item \( z_1 \in [1, 2.5], \text{   }  z_2 \in [1.49, 1.7], \text{   and   } z_3 \in [1.54, 2.39] \)

\end{enumerate} }
\litem{
What are the \textit{possible Rational} roots of the polynomial below?\[ f(x) = 3x^{4} +7 x^{3} +2 x^{2} +5 x + 2 \]\begin{enumerate}[label=\Alph*.]
\item \( \pm 1,\pm 2 \)
\item \( \text{ All combinations of: }\frac{\pm 1,\pm 3}{\pm 1,\pm 2} \)
\item \( \text{ All combinations of: }\frac{\pm 1,\pm 2}{\pm 1,\pm 3} \)
\item \( \pm 1,\pm 3 \)
\item \( \text{ There is no formula or theorem that tells us all possible Rational roots.} \)

\end{enumerate} }
\litem{
Perform the division below. Then, find the intervals that correspond to the quotient in the form $ax^2+bx+c$ and remainder $r$.\[ \frac{16x^{3} +32 x^{2} -4 x -12}{x + 2} \]\begin{enumerate}[label=\Alph*.]
\item \( a \in [-32, -31], \text{   } b \in [-32, -26], \text{   } c \in [-72, -63], \text{   and   } r \in [-151, -145]. \)
\item \( a \in [-32, -31], \text{   } b \in [95, 101], \text{   } c \in [-200, -193], \text{   and   } r \in [376, 387]. \)
\item \( a \in [16, 18], \text{   } b \in [-17, -15], \text{   } c \in [43, 45], \text{   and   } r \in [-147, -143]. \)
\item \( a \in [16, 18], \text{   } b \in [60, 66], \text{   } c \in [124, 131], \text{   and   } r \in [235, 239]. \)
\item \( a \in [16, 18], \text{   } b \in [-1, 3], \text{   } c \in [-15, 2], \text{   and   } r \in [-6, -1]. \)

\end{enumerate} }
\litem{
Perform the division below. Then, find the intervals that correspond to the quotient in the form $ax^2+bx+c$ and remainder $r$.\[ \frac{4x^{3} -10 x^{2} -16 x + 38}{x -2} \]\begin{enumerate}[label=\Alph*.]
\item \( a \in [2.9, 4.7], \text{   } b \in [-5.6, 0.9], \text{   } c \in [-21.9, -18.1], \text{   and   } r \in [-3, 0]. \)
\item \( a \in [2.9, 4.7], \text{   } b \in [-7, -4.7], \text{   } c \in [-22.3, -20.4], \text{   and   } r \in [12, 26]. \)
\item \( a \in [4.3, 9.4], \text{   } b \in [5.5, 7.2], \text{   } c \in [-4.9, -1.2], \text{   and   } r \in [25, 34]. \)
\item \( a \in [4.3, 9.4], \text{   } b \in [-29, -25.9], \text{   } c \in [35.5, 41], \text{   and   } r \in [-35, -32]. \)
\item \( a \in [2.9, 4.7], \text{   } b \in [-18.6, -17.5], \text{   } c \in [18.1, 23.6], \text{   and   } r \in [-3, 0]. \)

\end{enumerate} }
\litem{
Perform the division below. Then, find the intervals that correspond to the quotient in the form $ax^2+bx+c$ and remainder $r$.\[ \frac{10x^{3} -70 x -63}{x -3} \]\begin{enumerate}[label=\Alph*.]
\item \( a \in [9, 11], b \in [12, 24], c \in [-32, -28], \text{ and } r \in [-125, -117]. \)
\item \( a \in [9, 11], b \in [-31, -28], c \in [18, 24], \text{ and } r \in [-125, -117]. \)
\item \( a \in [30, 35], b \in [88, 93], c \in [198, 202], \text{ and } r \in [537, 540]. \)
\item \( a \in [9, 11], b \in [28, 32], c \in [18, 24], \text{ and } r \in [-7, -2]. \)
\item \( a \in [30, 35], b \in [-93, -88], c \in [198, 202], \text{ and } r \in [-665, -660]. \)

\end{enumerate} }
\litem{
Factor the polynomial below completely, knowing that $x-2$ is a factor. Then, choose the intervals the zeros of the polynomial belong to, where $z_1 \leq z_2 \leq z_3 \leq z_4$. \textit{To make the problem easier, all zeros are between -5 and 5.}\[ f(x) = 12x^{4} +25 x^{3} -114 x^{2} -48 x + 160 \]\begin{enumerate}[label=\Alph*.]
\item \( z_1 \in [-3.5, -1.7], \text{   }  z_2 \in [-0.43, -0.4], z_3 \in [3.99, 4.05], \text{   and   } z_4 \in [4, 9] \)
\item \( z_1 \in [-5.9, -2.3], \text{   }  z_2 \in [-0.78, -0.74], z_3 \in [0.79, 0.85], \text{   and   } z_4 \in [1, 3] \)
\item \( z_1 \in [-3.5, -1.7], \text{   }  z_2 \in [-0.82, -0.79], z_3 \in [0.72, 0.76], \text{   and   } z_4 \in [4, 9] \)
\item \( z_1 \in [-5.9, -2.3], \text{   }  z_2 \in [-1.38, -1.3], z_3 \in [1.23, 1.25], \text{   and   } z_4 \in [1, 3] \)
\item \( z_1 \in [-3.5, -1.7], \text{   }  z_2 \in [-1.32, -1.23], z_3 \in [1.33, 1.38], \text{   and   } z_4 \in [4, 9] \)

\end{enumerate} }
\litem{
Factor the polynomial below completely. Then, choose the intervals the zeros of the polynomial belong to, where $z_1 \leq z_2 \leq z_3$. \textit{To make the problem easier, all zeros are between -5 and 5.}\[ f(x) = 6x^{3} +41 x^{2} +45 x -50 \]\begin{enumerate}[label=\Alph*.]
\item \( z_1 \in [-5.23, -4.91], \text{   }  z_2 \in [-0.6, -0.1], \text{   and   } z_3 \in [1.2, 1.6] \)
\item \( z_1 \in [-0.68, -0.37], \text{   }  z_2 \in [1.92, 2.62], \text{   and   } z_3 \in [3.9, 6.2] \)
\item \( z_1 \in [-5.23, -4.91], \text{   }  z_2 \in [-3.38, -2.35], \text{   and   } z_3 \in [0.4, 1.4] \)
\item \( z_1 \in [-1.59, -1.34], \text{   }  z_2 \in [0.25, 0.49], \text{   and   } z_3 \in [3.9, 6.2] \)
\item \( z_1 \in [-0.49, -0.12], \text{   }  z_2 \in [4.57, 5.85], \text{   and   } z_3 \in [3.9, 6.2] \)

\end{enumerate} }
\litem{
Perform the division below. Then, find the intervals that correspond to the quotient in the form $ax^2+bx+c$ and remainder $r$.\[ \frac{20x^{3} -84 x^{2} + 62}{x -4} \]\begin{enumerate}[label=\Alph*.]
\item \( a \in [79, 83], b \in [-406, -397], c \in [1615, 1618], \text{ and } r \in [-6403, -6399]. \)
\item \( a \in [79, 83], b \in [225, 245], c \in [942, 956], \text{ and } r \in [3838, 3846]. \)
\item \( a \in [20, 27], b \in [-11, 3], c \in [-20, -12], \text{ and } r \in [-2, 1]. \)
\item \( a \in [20, 27], b \in [-28, -17], c \in [-75, -71], \text{ and } r \in [-155, -152]. \)
\item \( a \in [20, 27], b \in [-167, -161], c \in [651, 660], \text{ and } r \in [-2562, -2555]. \)

\end{enumerate} }
\litem{
Factor the polynomial below completely, knowing that $x-3$ is a factor. Then, choose the intervals the zeros of the polynomial belong to, where $z_1 \leq z_2 \leq z_3 \leq z_4$. \textit{To make the problem easier, all zeros are between -5 and 5.}\[ f(x) = 10x^{4} -89 x^{3} +238 x^{2} -123 x -180 \]\begin{enumerate}[label=\Alph*.]
\item \( z_1 \in [-0.6, 0.4], \text{   }  z_2 \in [1.1, 3.5], z_3 \in [2.77, 3.1], \text{   and   } z_4 \in [3.12, 4.1] \)
\item \( z_1 \in [-1.67, -0.67], \text{   }  z_2 \in [-0.8, 0.9], z_3 \in [2.77, 3.1], \text{   and   } z_4 \in [3.12, 4.1] \)
\item \( z_1 \in [-6, -2], \text{   }  z_2 \in [-4.3, -1.2], z_3 \in [-0.41, -0.36], \text{   and   } z_4 \in [1.43, 2.5] \)
\item \( z_1 \in [-6, -2], \text{   }  z_2 \in [-4.3, -1.2], z_3 \in [-3.11, -2.35], \text{   and   } z_4 \in [-0.23, 1.2] \)
\item \( z_1 \in [-6, -2], \text{   }  z_2 \in [-4.3, -1.2], z_3 \in [-0.85, -0.44], \text{   and   } z_4 \in [2.73, 3.26] \)

\end{enumerate} }
\end{enumerate}

\end{document}