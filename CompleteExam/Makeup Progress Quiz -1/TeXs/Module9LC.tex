\documentclass[14pt]{extbook}
\usepackage{multicol, enumerate, enumitem, hyperref, color, soul, setspace, parskip, fancyhdr} %General Packages
\usepackage{amssymb, amsthm, amsmath, bbm, latexsym, units, mathtools} %Math Packages
\everymath{\displaystyle} %All math in Display Style
% Packages with additional options
\usepackage[headsep=0.5cm,headheight=12pt, left=1 in,right= 1 in,top= 1 in,bottom= 1 in]{geometry}
\usepackage[usenames,dvipsnames]{xcolor}
\usepackage{dashrule}  % Package to use the command below to create lines between items
\newcommand{\litem}[1]{\item#1\hspace*{-1cm}\rule{\textwidth}{0.4pt}}
\pagestyle{fancy}
\lhead{Makeup Progress Quiz -1}
\chead{}
\rhead{Version C}
\lfoot{7547-2949}
\cfoot{}
\rfoot{Fall 2020}
\begin{document}

\begin{enumerate}
\litem{
Find the inverse of the function below. Then, evaluate the inverse at $x = 9$ and choose the interval that $f^{-1}(9)$ belongs to.\[ f(x) = \ln{(x-4)}+3 \]\begin{enumerate}[label=\Alph*.]
\item \( f^{-1}(9) \in [162758.79, 162766.79] \)
\item \( f^{-1}(9) \in [147.41, 155.41] \)
\item \( f^{-1}(9) \in [442414.39, 442422.39] \)
\item \( f^{-1}(9) \in [404.43, 409.43] \)
\item \( f^{-1}(9) \in [398.43, 402.43] \)

\end{enumerate} }
\litem{
Multiply the following functions, then choose the domain of the resulting function from the list below.\[ f(x) = 5x + 3 \text{ and } g(x) = 9x^{2} +3 x + 8 \]\begin{enumerate}[label=\Alph*.]
\item \( \text{ The domain is all Real numbers less than or equal to } x = a, \text{ where } a \in [-4.5, -0.5] \)
\item \( \text{ The domain is all Real numbers except } x = a, \text{ where } a \in [2.25, 13.25] \)
\item \( \text{ The domain is all Real numbers greater than or equal to } x = a, \text{ where } a \in [5.33, 10.33] \)
\item \( \text{ The domain is all Real numbers except } x = a \text{ and } x = b, \text{ where } a \in [-10.25, 7.75] \text{ and } b \in [0.8, 7.8] \)
\item \( \text{ The domain is all Real numbers. } \)

\end{enumerate} }
\litem{
Choose the interval below that $f$ composed with $g$ at $x=1$ is in.\[ f(x) = -x^{3} -2 x^{2} +x \text{ and } g(x) = -4x^{3} +2 x^{2} +x \]\begin{enumerate}[label=\Alph*.]
\item \( (f \circ g)(1) \in [38, 39] \)
\item \( (f \circ g)(1) \in [29, 32] \)
\item \( (f \circ g)(1) \in [5, 11] \)
\item \( (f \circ g)(1) \in [-7, -1] \)
\item \( \text{It is not possible to compose the two functions.} \)

\end{enumerate} }
\litem{
Find the inverse of the function below (if it exists). Then, evaluate the inverse at $x = 10$ and choose the interval that $f^{-1}(10)$ belongs to.\[ f(x) = 5 x^2 + 3 \]\begin{enumerate}[label=\Alph*.]
\item \( f^{-1}(10) \in [1.61, 1.68] \)
\item \( f^{-1}(10) \in [7.17, 7.38] \)
\item \( f^{-1}(10) \in [0.86, 1.23] \)
\item \( f^{-1}(10) \in [3.94, 4.3] \)
\item \( \text{ The function is not invertible for all Real numbers. } \)

\end{enumerate} }
\litem{
Multiply the following functions, then choose the domain of the resulting function from the list below.\[ f(x) = x^{3} +7 x^{2} +3 x + 4 \text{ and } g(x) = \sqrt{-3x-15}  \]\begin{enumerate}[label=\Alph*.]
\item \( \text{ The domain is all Real numbers less than or equal to } x = a, \text{ where } a \in [-10, 2] \)
\item \( \text{ The domain is all Real numbers greater than or equal to } x = a, \text{ where } a \in [-6, -3] \)
\item \( \text{ The domain is all Real numbers except } x = a, \text{ where } a \in [-10.25, -3.25] \)
\item \( \text{ The domain is all Real numbers except } x = a \text{ and } x = b, \text{ where } a \in [3.33, 8.33] \text{ and } b \in [-10.2, -2.2] \)
\item \( \text{ The domain is all Real numbers. } \)

\end{enumerate} }
\litem{
Determine whether the function below is 1-1.\[ f(x) = (3 x - 20)^3 \]\begin{enumerate}[label=\Alph*.]
\item \( \text{No, because the range of the function is not $(-\infty, \infty)$.} \)
\item \( \text{No, because the domain of the function is not $(-\infty, \infty)$.} \)
\item \( \text{No, because there is an $x$-value that goes to 2 different $y$-values.} \)
\item \( \text{Yes, the function is 1-1.} \)
\item \( \text{No, because there is a $y$-value that goes to 2 different $x$-values.} \)

\end{enumerate} }
\litem{
Choose the interval below that $f$ composed with $g$ at $x=1$ is in.\[ f(x) = 2x^{3} +2 x^{2} -4 x + 1 \text{ and } g(x) = -2x^{3} +4 x^{2} -x + 1 \]\begin{enumerate}[label=\Alph*.]
\item \( (f \circ g)(1) \in [-12, -5] \)
\item \( (f \circ g)(1) \in [-2, 7] \)
\item \( (f \circ g)(1) \in [16, 22] \)
\item \( (f \circ g)(1) \in [11, 13] \)
\item \( \text{It is not possible to compose the two functions.} \)

\end{enumerate} }
\litem{
Find the inverse of the function below (if it exists). Then, evaluate the inverse at $x = 11$ and choose the interval that $f^{-1}(11)$ belongs to.\[ f(x) = 3 x^2 + 2 \]\begin{enumerate}[label=\Alph*.]
\item \( f^{-1}(11) \in [6.7, 6.99] \)
\item \( f^{-1}(11) \in [1.79, 2.1] \)
\item \( f^{-1}(11) \in [4.6, 4.92] \)
\item \( f^{-1}(11) \in [1.71, 1.91] \)
\item \( \text{ The function is not invertible for all Real numbers. } \)

\end{enumerate} }
\litem{
Determine whether the function below is 1-1.\[ f(x) = 36 x^2 + 168 x + 196 \]\begin{enumerate}[label=\Alph*.]
\item \( \text{No, because there is a $y$-value that goes to 2 different $x$-values.} \)
\item \( \text{No, because there is an $x$-value that goes to 2 different $y$-values.} \)
\item \( \text{Yes, the function is 1-1.} \)
\item \( \text{No, because the range of the function is not $(-\infty, \infty)$.} \)
\item \( \text{No, because the domain of the function is not $(-\infty, \infty)$.} \)

\end{enumerate} }
\litem{
Find the inverse of the function below. Then, evaluate the inverse at $x = 7$ and choose the interval that $f^{-1}(7)$ belongs to.\[ f(x) = \ln{(x+5)}-4 \]\begin{enumerate}[label=\Alph*.]
\item \( f^{-1}(7) \in [59877.14, 59886.14] \)
\item \( f^{-1}(7) \in [59865.14, 59874.14] \)
\item \( f^{-1}(7) \in [162747.79, 162755.79] \)
\item \( f^{-1}(7) \in [14.09, 21.09] \)
\item \( f^{-1}(7) \in [-0.61, 4.39] \)

\end{enumerate} }
\end{enumerate}

\end{document}