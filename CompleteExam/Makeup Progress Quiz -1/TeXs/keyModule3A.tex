\documentclass{extbook}[14pt]
\usepackage{multicol, enumerate, enumitem, hyperref, color, soul, setspace, parskip, fancyhdr, amssymb, amsthm, amsmath, bbm, latexsym, units, mathtools}
\everymath{\displaystyle}
\usepackage[headsep=0.5cm,headheight=0cm, left=1 in,right= 1 in,top= 1 in,bottom= 1 in]{geometry}
\usepackage{dashrule}  % Package to use the command below to create lines between items
\newcommand{\litem}[1]{\item #1

\rule{\textwidth}{0.4pt}}
\pagestyle{fancy}
\lhead{}
\chead{Answer Key for Makeup Progress Quiz -1 Version A}
\rhead{}
\lfoot{7547-2949}
\cfoot{}
\rfoot{Fall 2020}
\begin{document}
\textbf{This key should allow you to understand why you choose the option you did (beyond just getting a question right or wrong). \href{https://xronos.clas.ufl.edu/mac1105spring2020/courseDescriptionAndMisc/Exams/LearningFromResults}{More instructions on how to use this key can be found here}.}

\textbf{If you have a suggestion to make the keys better, \href{https://forms.gle/CZkbZmPbC9XALEE88}{please fill out the short survey here}.}

\textit{Note: This key is auto-generated and may contain issues and/or errors. The keys are reviewed after each exam to ensure grading is done accurately. If there are issues (like duplicate options), they are noted in the offline gradebook. The keys are a work-in-progress to give students as many resources to improve as possible.}

\rule{\textwidth}{0.4pt}

\begin{enumerate}\litem{
Solve the linear inequality below. Then, choose the constant and interval combination that describes the solution set.
\[ 4 - 3 x < \frac{-20 x - 6}{8} \leq 5 - 3 x \]

The solution is \( \text{None of the above.} \), which is option E.\begin{enumerate}[label=\Alph*.]
\item \( (-\infty, a] \cup (b, \infty), \text{ where } a \in [-11.5, -7.5] \text{ and } b \in [-13.5, -10.5] \)

$(-\infty, -9.50] \cup (-11.50, \infty)$, which corresponds to displaying the and-inequality as an or-inequality AND flipping the inequality AND getting negatives of the actual endpoints.
\item \( (a, b], \text{ where } a \in [-10.5, -4.5] \text{ and } b \in [-12.5, -3.5] \)

$(-9.50, -11.50]$, which is the correct interval but negatives of the actual endpoints.
\item \( [a, b), \text{ where } a \in [-9.5, -6.5] \text{ and } b \in [-11.5, -9.5] \)

$[-9.50, -11.50)$, which corresponds to flipping the inequality and getting negatives of the actual endpoints.
\item \( (-\infty, a) \cup [b, \infty), \text{ where } a \in [-9.5, -8.5] \text{ and } b \in [-12.5, -6.5] \)

$(-\infty, -9.50) \cup [-11.50, \infty)$, which corresponds to displaying the and-inequality as an or-inequality and getting negatives of the actual endpoints.
\item \( \text{None of the above.} \)

* This is correct as the answer should be $(9.50, 11.50]$.
\end{enumerate}

\textbf{General Comment:} To solve, you will need to break up the compound inequality into two inequalities. Be sure to keep track of the inequality! It may be best to draw a number line and graph your solution.
}
\litem{
Solve the linear inequality below. Then, choose the constant and interval combination that describes the solution set.
\[ 5x -8 \geq 10x + 5 \]

The solution is \( (-\infty, -2.6] \), which is option D.\begin{enumerate}[label=\Alph*.]
\item \( [a, \infty), \text{ where } a \in [-0.4, 9.6] \)

 $[2.6, \infty)$, which corresponds to switching the direction of the interval AND negating the endpoint. You likely did this if you did not flip the inequality when dividing by a negative as well as not moving values over to a side properly.
\item \( [a, \infty), \text{ where } a \in [-5.6, 1.4] \)

 $[-2.6, \infty)$, which corresponds to switching the direction of the interval. You likely did this if you did not flip the inequality when dividing by a negative!
\item \( (-\infty, a], \text{ where } a \in [2.6, 9.6] \)

 $(-\infty, 2.6]$, which corresponds to negating the endpoint of the solution.
\item \( (-\infty, a], \text{ where } a \in [-10.6, 1.4] \)

* $(-\infty, -2.6]$, which is the correct option.
\item \( \text{None of the above}. \)

You may have chosen this if you thought the inequality did not match the ends of the intervals.
\end{enumerate}

\textbf{General Comment:} Remember that less/greater than or equal to includes the endpoint, while less/greater do not. Also, remember that you need to flip the inequality when you multiply or divide by a negative.
}
\litem{
Solve the linear inequality below. Then, choose the constant and interval combination that describes the solution set.
\[ \frac{-7}{3} - \frac{9}{8} x \geq \frac{6}{7} x + \frac{9}{4} \]

The solution is \( (-\infty, -2.312] \), which is option A.\begin{enumerate}[label=\Alph*.]
\item \( (-\infty, a], \text{ where } a \in [-4.31, 0.69] \)

* $(-\infty, -2.312]$, which is the correct option.
\item \( (-\infty, a], \text{ where } a \in [2.31, 5.31] \)

 $(-\infty, 2.312]$, which corresponds to negating the endpoint of the solution.
\item \( [a, \infty), \text{ where } a \in [-5.31, -0.31] \)

 $[-2.312, \infty)$, which corresponds to switching the direction of the interval. You likely did this if you did not flip the inequality when dividing by a negative!
\item \( [a, \infty), \text{ where } a \in [1.31, 6.31] \)

 $[2.312, \infty)$, which corresponds to switching the direction of the interval AND negating the endpoint. You likely did this if you did not flip the inequality when dividing by a negative as well as not moving values over to a side properly.
\item \( \text{None of the above}. \)

You may have chosen this if you thought the inequality did not match the ends of the intervals.
\end{enumerate}

\textbf{General Comment:} Remember that less/greater than or equal to includes the endpoint, while less/greater do not. Also, remember that you need to flip the inequality when you multiply or divide by a negative.
}
\litem{
Solve the linear inequality below. Then, choose the constant and interval combination that describes the solution set.
\[ -5 + 9 x > 10 x \text{ or } 9 + 3 x < 4 x \]

The solution is \( (-\infty, -5.0) \text{ or } (9.0, \infty) \), which is option C.\begin{enumerate}[label=\Alph*.]
\item \( (-\infty, a] \cup [b, \infty), \text{ where } a \in [-6, 1] \text{ and } b \in [9, 15] \)

Corresponds to including the endpoints (when they should be excluded).
\item \( (-\infty, a] \cup [b, \infty), \text{ where } a \in [-14, -6] \text{ and } b \in [4, 7] \)

Corresponds to including the endpoints AND negating.
\item \( (-\infty, a) \cup (b, \infty), \text{ where } a \in [-5, -3] \text{ and } b \in [8, 13] \)

 * Correct option.
\item \( (-\infty, a) \cup (b, \infty), \text{ where } a \in [-10, -7] \text{ and } b \in [3, 7] \)

Corresponds to inverting the inequality and negating the solution.
\item \( (-\infty, \infty) \)

Corresponds to the variable canceling, which does not happen in this instance.
\end{enumerate}

\textbf{General Comment:} When multiplying or dividing by a negative, flip the sign.
}
\litem{
Using an interval or intervals, describe all the $x$-values within or including a distance of the given values.
\[ \text{ No less than } 7 \text{ units from the number } 1. \]

The solution is \( \text{None of the above} \), which is option E.\begin{enumerate}[label=\Alph*.]
\item \( [6, 8] \)

This describes the values no more than 1 from 7
\item \( (6, 8) \)

This describes the values less than 1 from 7
\item \( (-\infty, 6] \cup [8, \infty) \)

This describes the values no less than 1 from 7
\item \( (-\infty, 6) \cup (8, \infty) \)

This describes the values more than 1 from 7
\item \( \text{None of the above} \)

Options A-D described the values [more/less than] 1 units from 7, which is the reverse of what the question asked.
\end{enumerate}

\textbf{General Comment:} When thinking about this language, it helps to draw a number line and try points.
}
\litem{
Solve the linear inequality below. Then, choose the constant and interval combination that describes the solution set.
\[ -5 + 3 x > 4 x \text{ or } -6 + 7 x < 9 x \]

The solution is \( (-\infty, -5.0) \text{ or } (-3.0, \infty) \), which is option B.\begin{enumerate}[label=\Alph*.]
\item \( (-\infty, a) \cup (b, \infty), \text{ where } a \in [3, 4] \text{ and } b \in [-1, 8] \)

Corresponds to inverting the inequality and negating the solution.
\item \( (-\infty, a) \cup (b, \infty), \text{ where } a \in [-6, -3] \text{ and } b \in [-5, -1] \)

 * Correct option.
\item \( (-\infty, a] \cup [b, \infty), \text{ where } a \in [-6, -4] \text{ and } b \in [-4, 1] \)

Corresponds to including the endpoints (when they should be excluded).
\item \( (-\infty, a] \cup [b, \infty), \text{ where } a \in [3, 5] \text{ and } b \in [2, 9] \)

Corresponds to including the endpoints AND negating.
\item \( (-\infty, \infty) \)

Corresponds to the variable canceling, which does not happen in this instance.
\end{enumerate}

\textbf{General Comment:} When multiplying or dividing by a negative, flip the sign.
}
\litem{
Solve the linear inequality below. Then, choose the constant and interval combination that describes the solution set.
\[ \frac{4}{5} - \frac{4}{6} x \leq \frac{3}{3} x + \frac{5}{2} \]

The solution is \( [-1.02, \infty) \), which is option C.\begin{enumerate}[label=\Alph*.]
\item \( (-\infty, a], \text{ where } a \in [-1.02, -0.02] \)

 $(-\infty, -1.02]$, which corresponds to switching the direction of the interval. You likely did this if you did not flip the inequality when dividing by a negative!
\item \( (-\infty, a], \text{ where } a \in [0.02, 3.02] \)

 $(-\infty, 1.02]$, which corresponds to switching the direction of the interval AND negating the endpoint. You likely did this if you did not flip the inequality when dividing by a negative as well as not moving values over to a side properly.
\item \( [a, \infty), \text{ where } a \in [-2.02, -0.02] \)

* $[-1.02, \infty)$, which is the correct option.
\item \( [a, \infty), \text{ where } a \in [-0.98, 2.02] \)

 $[1.02, \infty)$, which corresponds to negating the endpoint of the solution.
\item \( \text{None of the above}. \)

You may have chosen this if you thought the inequality did not match the ends of the intervals.
\end{enumerate}

\textbf{General Comment:} Remember that less/greater than or equal to includes the endpoint, while less/greater do not. Also, remember that you need to flip the inequality when you multiply or divide by a negative.
}
\litem{
Solve the linear inequality below. Then, choose the constant and interval combination that describes the solution set.
\[ -5 - 6 x \leq \frac{-28 x + 7}{5} < -4 - 6 x \]

The solution is \( [-16.00, -13.50) \), which is option C.\begin{enumerate}[label=\Alph*.]
\item \( (-\infty, a) \cup [b, \infty), \text{ where } a \in [-16, -12] \text{ and } b \in [-16.5, -11.5] \)

$(-\infty, -16.00) \cup [-13.50, \infty)$, which corresponds to displaying the and-inequality as an or-inequality AND flipping the inequality.
\item \( (a, b], \text{ where } a \in [-18, -14] \text{ and } b \in [-16.5, -11.5] \)

$(-16.00, -13.50]$, which corresponds to flipping the inequality.
\item \( [a, b), \text{ where } a \in [-22, -15] \text{ and } b \in [-16.5, -12.5] \)

$[-16.00, -13.50)$, which is the correct option.
\item \( (-\infty, a] \cup (b, \infty), \text{ where } a \in [-16, -12] \text{ and } b \in [-16.5, -10.5] \)

$(-\infty, -16.00] \cup (-13.50, \infty)$, which corresponds to displaying the and-inequality as an or-inequality.
\item \( \text{None of the above.} \)


\end{enumerate}

\textbf{General Comment:} To solve, you will need to break up the compound inequality into two inequalities. Be sure to keep track of the inequality! It may be best to draw a number line and graph your solution.
}
\litem{
Solve the linear inequality below. Then, choose the constant and interval combination that describes the solution set.
\[ -8x -7 \leq 8x -4 \]

The solution is \( [-0.188, \infty) \), which is option D.\begin{enumerate}[label=\Alph*.]
\item \( [a, \infty), \text{ where } a \in [0.08, 0.31] \)

 $[0.188, \infty)$, which corresponds to negating the endpoint of the solution.
\item \( (-\infty, a], \text{ where } a \in [-0.05, 0.4] \)

 $(-\infty, 0.188]$, which corresponds to switching the direction of the interval AND negating the endpoint. You likely did this if you did not flip the inequality when dividing by a negative as well as not moving values over to a side properly.
\item \( (-\infty, a], \text{ where } a \in [-1.19, -0.05] \)

 $(-\infty, -0.188]$, which corresponds to switching the direction of the interval. You likely did this if you did not flip the inequality when dividing by a negative!
\item \( [a, \infty), \text{ where } a \in [-0.33, -0.03] \)

* $[-0.188, \infty)$, which is the correct option.
\item \( \text{None of the above}. \)

You may have chosen this if you thought the inequality did not match the ends of the intervals.
\end{enumerate}

\textbf{General Comment:} Remember that less/greater than or equal to includes the endpoint, while less/greater do not. Also, remember that you need to flip the inequality when you multiply or divide by a negative.
}
\litem{
Using an interval or intervals, describe all the $x$-values within or including a distance of the given values.
\[ \text{ More than } 5 \text{ units from the number } -9. \]

The solution is \( (-\infty, -14) \cup (-4, \infty) \), which is option D.\begin{enumerate}[label=\Alph*.]
\item \( (-14, -4) \)

This describes the values less than 5 from -9
\item \( [-14, -4] \)

This describes the values no more than 5 from -9
\item \( (-\infty, -14] \cup [-4, \infty) \)

This describes the values no less than 5 from -9
\item \( (-\infty, -14) \cup (-4, \infty) \)

This describes the values more than 5 from -9
\item \( \text{None of the above} \)

You likely thought the values in the interval were not correct.
\end{enumerate}

\textbf{General Comment:} When thinking about this language, it helps to draw a number line and try points.
}
\end{enumerate}

\end{document}