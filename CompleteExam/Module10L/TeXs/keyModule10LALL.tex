\documentclass{extbook}[14pt]
\usepackage{multicol, enumerate, enumitem, hyperref, color, soul, setspace, parskip, fancyhdr, amssymb, amsthm, amsmath, latexsym, units, mathtools}
\everymath{\displaystyle}
\usepackage[headsep=0.5cm,headheight=0cm, left=1 in,right= 1 in,top= 1 in,bottom= 1 in]{geometry}
\usepackage{dashrule}  % Package to use the command below to create lines between items
\newcommand{\litem}[1]{\item #1

\rule{\textwidth}{0.4pt}}
\pagestyle{fancy}
\lhead{}
\chead{Answer Key for Module10L Version ALL}
\rhead{}
\lfoot{4044-9918}
\cfoot{}
\rfoot{test}
\begin{document}
\textbf{This key should allow you to understand why you choose the option you did (beyond just getting a question right or wrong). \href{https://xronos.clas.ufl.edu/mac1105spring2020/courseDescriptionAndMisc/Exams/LearningFromResults}{More instructions on how to use this key can be found here}.}

\textbf{If you have a suggestion to make the keys better, \href{https://forms.gle/CZkbZmPbC9XALEE88}{please fill out the short survey here}.}

\textit{Note: This key is auto-generated and may contain issues and/or errors. The keys are reviewed after each exam to ensure grading is done accurately. If there are issues (like duplicate options), they are noted in the offline gradebook. The keys are a work-in-progress to give students as many resources to improve as possible.}

\rule{\textwidth}{0.4pt}

\begin{enumerate}\litem{
Factor the polynomial below completely, knowing that $x -3$ is a factor. \textit{To make the problem easier, all zeros are between -5 and 5.}
\[ f(x) = 4x^{4} -45 x^{2} + 81 \]The solution is \( [-3, -1.5, 1.5, 3] \).\begin{enumerate}[label=\Alph*.]
\textbf{Plausible alternative answers include:} Distractor 2: Corresponds to inversing rational roots.
* This is the solution!
 Distractor 4: Corresponds to moving factors from one rational to another.
 Distractor 1: Corresponds to negatives of all zeros.
 Distractor 3: Corresponds to negatives of all zeros AND inversing rational roots.
\end{enumerate}

\textbf{General Comment:} Remember to try the middle-most integers first as these normally are the zeros. Also, once you get it to a quadratic, you can use your other factoring techniques to finish factoring.
}
\litem{
Factor the polynomial below completely. \textit{To make the problem easier, all zeros are between -5 and 5.}
\[ f(x) = 9x^{3} +9 x^{2} -46 x + 24 \]The solution is \( [-3, 0.67, 1.33] \).\begin{enumerate}[label=\Alph*.]
\textbf{Plausible alternative answers include:} Distractor 3: Corresponds to negatives of all zeros AND inversing rational roots.
 Distractor 1: Corresponds to negatives of all zeros.
 Distractor 2: Corresponds to inversing rational roots.
 Distractor 4: Corresponds to moving factors from one rational to another.
* This is the solution!
\end{enumerate}

\textbf{General Comment:} Remember to try the middle-most integers first as these normally are the zeros. Also, once you get it to a quadratic, you can use your other factoring techniques to finish factoring.
}
\litem{
Factor the polynomial below completely, knowing that $x -3$ is a factor. \textit{To make the problem easier, all zeros are between -5 and 5.}
\[ f(x) = 10x^{4} -59 x^{3} +37 x^{2} +174 x -72 \]The solution is \( [-1.5, 0.4, 3, 4] \).\begin{enumerate}[label=\Alph*.]
\textbf{Plausible alternative answers include:} Distractor 2: Corresponds to inversing rational roots.
 Distractor 1: Corresponds to negatives of all zeros.
 Distractor 3: Corresponds to negatives of all zeros AND inversing rational roots.
* This is the solution!
 Distractor 4: Corresponds to moving factors from one rational to another.
\end{enumerate}

\textbf{General Comment:} Remember to try the middle-most integers first as these normally are the zeros. Also, once you get it to a quadratic, you can use your other factoring techniques to finish factoring.
}
\litem{
Perform the division below. Write the resulting quotient in the form $ax^2+bx+c$ and remainder as $r$.
\[ \frac{12x^{3} -39 x^{2} + 29}{x -3} \]The solution is \( 12x^{2} -3 x -9 + \frac{2}{x -3} \).\begin{enumerate}[label=\Alph*.]
\textbf{Plausible alternative answers include:}* This is the solution!
 You multipled by the synthetic number and subtracted rather than adding during synthetic division.
 You divided by the opposite of the factor.
 You divided by the opposite of the factor AND multipled the first factor rather than just bringing it down.
 You multipled by the synthetic number rather than bringing the first factor down.
\end{enumerate}

\textbf{General Comment:} Be sure to synthetically divide by the zero of the denominator! Also, make sure to include 0 placeholders for missing terms.
}
\litem{
Perform the division below. Write the resulting quotient in the form $ax^2+bx+c$ and remainder as $r$.
\[ \frac{9x^{3} -24 x^{2} -95 x -45}{x -5} \]The solution is \( 9x^{2} +21 x + 10 + \frac{5}{x -5} \).\begin{enumerate}[label=\Alph*.]
\textbf{Plausible alternative answers include:} You multiplied by the synthetic number rather than bringing the first factor down.
 You multiplied by the synthetic number and subtracted rather than adding during synthetic division.
 You divided by the opposite of the factor AND multiplied the first factor rather than just bringing it down.
* This is the solution!
 You divided by the opposite of the factor.
\end{enumerate}

\textbf{General Comment:} Be sure to synthetically divide by the zero of the denominator!
}
\litem{
Perform the division below. Write the resulting quotient in the form $ax^2+bx+c$ and remainder as $r$.
\[ \frac{8x^{3} -62 x + 27}{x + 3} \]The solution is \( 8x^{2} -24 x + 10 + \frac{-3}{x + 3} \).\begin{enumerate}[label=\Alph*.]
\textbf{Plausible alternative answers include:} You divided by the opposite of the factor.
 You divided by the opposite of the factor AND multipled the first factor rather than just bringing it down.
 You multipled by the synthetic number rather than bringing the first factor down.
 You multipled by the synthetic number and subtracted rather than adding during synthetic division.
* This is the solution!
\end{enumerate}

\textbf{General Comment:} Be sure to synthetically divide by the zero of the denominator! Also, make sure to include 0 placeholders for missing terms.
}
\litem{
What are the \textit{possible Rational} roots of the polynomial below?
\[ f(x) = 5x^{2} +7 x + 2 \]The solution is \( \text{ All combinations of: }\frac{\pm 1,\pm 2}{\pm 1,\pm 5} \).\begin{enumerate}[label=\Alph*.]
\textbf{Plausible alternative answers include:} Distractor 3: Corresponds to the plus or minus of the inverse quotient (an/a0) of the factors. 
This would have been the solution \textbf{if asked for the possible Integer roots}!
 Distractor 1: Corresponds to the plus or minus factors of a1 only.
* This is the solution \textbf{since we asked for the possible Rational roots}!
 Distractor 4: Corresponds to not recalling the theorem for rational roots of a polynomial.
\end{enumerate}

\textbf{General Comment:} We have a way to find the possible Rational roots. The possible Integer roots are the Integers in this list.
}
\litem{
What are the \textit{possible Integer} roots of the polynomial below?
\[ f(x) = 2x^{2} +6 x + 3 \]The solution is \( \pm 1,\pm 3 \).\begin{enumerate}[label=\Alph*.]
\textbf{Plausible alternative answers include:} Distractor 3: Corresponds to the plus or minus of the inverse quotient (an/a0) of the factors. 
 Distractor 1: Corresponds to the plus or minus factors of a1 only.
This would have been the solution \textbf{if asked for the possible Rational roots}!
* This is the solution \textbf{since we asked for the possible Integer roots}!
 Distractor 4: Corresponds to not recognizing Integers as a subset of Rationals.
\end{enumerate}

\textbf{General Comment:} We have a way to find the possible Rational roots. The possible Integer roots are the Integers in this list.
}
\litem{
Perform the division below. Write the resulting quotient in the form $ax^2+bx+c$ and remainder as $r$.
\[ \frac{12x^{3} +40 x^{2} -92 x + 45}{x + 5} \]The solution is \( 12x^{2} -20 x + 8 + \frac{5}{x + 5} \).\begin{enumerate}[label=\Alph*.]
\textbf{Plausible alternative answers include:} You multiplied by the synthetic number rather than bringing the first factor down.
 You divided by the opposite of the factor.
 You multiplied by the synthetic number and subtracted rather than adding during synthetic division.
* This is the solution!
 You divided by the opposite of the factor AND multiplied the first factor rather than just bringing it down.
\end{enumerate}

\textbf{General Comment:} Be sure to synthetically divide by the zero of the denominator!
}
\litem{
Factor the polynomial below completely. \textit{To make the problem easier, all zeros are between -5 and 5.}
\[ f(x) = 9x^{3} -27 x^{2} -4 x + 12 \]The solution is \( [-0.67, 0.67, 3] \).\begin{enumerate}[label=\Alph*.]
\textbf{Plausible alternative answers include:} Distractor 4: Corresponds to moving factors from one rational to another.
 Distractor 3: Corresponds to negatives of all zeros AND inversing rational roots.
 Distractor 1: Corresponds to negatives of all zeros.
* This is the solution!
 Distractor 2: Corresponds to inversing rational roots.
\end{enumerate}

\textbf{General Comment:} Remember to try the middle-most integers first as these normally are the zeros. Also, once you get it to a quadratic, you can use your other factoring techniques to finish factoring.
}
\litem{
Factor the polynomial below completely, knowing that $x -3$ is a factor. \textit{To make the problem easier, all zeros are between -5 and 5.}
\[ f(x) = 20x^{4} -133 x^{3} +93 x^{2} +333 x + 135 \]The solution is \( [-0.75, -0.6, 3, 5] \).\begin{enumerate}[label=\Alph*.]
\textbf{Plausible alternative answers include:} Distractor 3: Corresponds to negatives of all zeros AND inversing rational roots.
 Distractor 2: Corresponds to inversing rational roots.
 Distractor 4: Corresponds to moving factors from one rational to another.
* This is the solution!
 Distractor 1: Corresponds to negatives of all zeros.
\end{enumerate}

\textbf{General Comment:} Remember to try the middle-most integers first as these normally are the zeros. Also, once you get it to a quadratic, you can use your other factoring techniques to finish factoring.
}
\litem{
Factor the polynomial below completely. \textit{To make the problem easier, all zeros are between -5 and 5.}
\[ f(x) = 4x^{3} -24 x^{2} +5 x + 75 \]The solution is \( [-1.5, 2.5, 5] \).\begin{enumerate}[label=\Alph*.]
\textbf{Plausible alternative answers include:} Distractor 3: Corresponds to negatives of all zeros AND inversing rational roots.
* This is the solution!
 Distractor 2: Corresponds to inversing rational roots.
 Distractor 4: Corresponds to moving factors from one rational to another.
 Distractor 1: Corresponds to negatives of all zeros.
\end{enumerate}

\textbf{General Comment:} Remember to try the middle-most integers first as these normally are the zeros. Also, once you get it to a quadratic, you can use your other factoring techniques to finish factoring.
}
\litem{
Factor the polynomial below completely, knowing that $x -4$ is a factor. \textit{To make the problem easier, all zeros are between -5 and 5.}
\[ f(x) = 8x^{4} +30 x^{3} -123 x^{2} -425 x -300 \]The solution is \( [-5, -1.5, -1.25, 4] \).\begin{enumerate}[label=\Alph*.]
\textbf{Plausible alternative answers include:} Distractor 2: Corresponds to inversing rational roots.
* This is the solution!
 Distractor 3: Corresponds to negatives of all zeros AND inversing rational roots.
 Distractor 4: Corresponds to moving factors from one rational to another.
 Distractor 1: Corresponds to negatives of all zeros.
\end{enumerate}

\textbf{General Comment:} Remember to try the middle-most integers first as these normally are the zeros. Also, once you get it to a quadratic, you can use your other factoring techniques to finish factoring.
}
\litem{
Perform the division below. Write the resulting quotient in the form $ax^2+bx+c$ and remainder as $r$.
\[ \frac{6x^{3} -18 x -10}{x -2} \]The solution is \( 6x^{2} +12 x + 6 + \frac{2}{x -2} \).\begin{enumerate}[label=\Alph*.]
\textbf{Plausible alternative answers include:} You divided by the opposite of the factor AND multipled the first factor rather than just bringing it down.
 You multipled by the synthetic number and subtracted rather than adding during synthetic division.
 You divided by the opposite of the factor.
 You multipled by the synthetic number rather than bringing the first factor down.
* This is the solution!
\end{enumerate}

\textbf{General Comment:} Be sure to synthetically divide by the zero of the denominator! Also, make sure to include 0 placeholders for missing terms.
}
\litem{
Perform the division below. Write the resulting quotient in the form $ax^2+bx+c$ and remainder as $r$.
\[ \frac{25x^{3} -85 x^{2} -184 x -82}{x -5} \]The solution is \( 25x^{2} +40 x + 16 + \frac{-2}{x -5} \).\begin{enumerate}[label=\Alph*.]
\textbf{Plausible alternative answers include:} You multiplied by the synthetic number rather than bringing the first factor down.
 You divided by the opposite of the factor AND multiplied the first factor rather than just bringing it down.
 You multiplied by the synthetic number and subtracted rather than adding during synthetic division.
* This is the solution!
 You divided by the opposite of the factor.
\end{enumerate}

\textbf{General Comment:} Be sure to synthetically divide by the zero of the denominator!
}
\litem{
Perform the division below. Write the resulting quotient in the form $ax^2+bx+c$ and remainder as $r$.
\[ \frac{12x^{3} -65 x^{2} + 127}{x -5} \]The solution is \( 12x^{2} -5 x -25 + \frac{2}{x -5} \).\begin{enumerate}[label=\Alph*.]
\textbf{Plausible alternative answers include:} You divided by the opposite of the factor AND multipled the first factor rather than just bringing it down.
 You multipled by the synthetic number rather than bringing the first factor down.
 You divided by the opposite of the factor.
* This is the solution!
 You multipled by the synthetic number and subtracted rather than adding during synthetic division.
\end{enumerate}

\textbf{General Comment:} Be sure to synthetically divide by the zero of the denominator! Also, make sure to include 0 placeholders for missing terms.
}
\litem{
What are the \textit{possible Integer} roots of the polynomial below?
\[ f(x) = 4x^{4} +6 x^{3} +6 x^{2} +7 x + 6 \]The solution is \( \pm 1,\pm 2,\pm 3,\pm 6 \).\begin{enumerate}[label=\Alph*.]
\textbf{Plausible alternative answers include:}This would have been the solution \textbf{if asked for the possible Rational roots}!
 Distractor 1: Corresponds to the plus or minus factors of a1 only.
 Distractor 3: Corresponds to the plus or minus of the inverse quotient (an/a0) of the factors. 
* This is the solution \textbf{since we asked for the possible Integer roots}!
 Distractor 4: Corresponds to not recognizing Integers as a subset of Rationals.
\end{enumerate}

\textbf{General Comment:} We have a way to find the possible Rational roots. The possible Integer roots are the Integers in this list.
}
\litem{
What are the \textit{possible Rational} roots of the polynomial below?
\[ f(x) = 2x^{2} +2 x + 3 \]The solution is \( \text{ All combinations of: }\frac{\pm 1,\pm 3}{\pm 1,\pm 2} \).\begin{enumerate}[label=\Alph*.]
\textbf{Plausible alternative answers include:} Distractor 1: Corresponds to the plus or minus factors of a1 only.
* This is the solution \textbf{since we asked for the possible Rational roots}!
 Distractor 3: Corresponds to the plus or minus of the inverse quotient (an/a0) of the factors. 
This would have been the solution \textbf{if asked for the possible Integer roots}!
 Distractor 4: Corresponds to not recalling the theorem for rational roots of a polynomial.
\end{enumerate}

\textbf{General Comment:} We have a way to find the possible Rational roots. The possible Integer roots are the Integers in this list.
}
\litem{
Perform the division below. Write the resulting quotient in the form $ax^2+bx+c$ and remainder as $r$.
\[ \frac{15x^{3} +21 x^{2} -24 x -16}{x + 2} \]The solution is \( 15x^{2} -9 x -6 + \frac{-4}{x + 2} \).\begin{enumerate}[label=\Alph*.]
\textbf{Plausible alternative answers include:} You divided by the opposite of the factor AND multiplied the first factor rather than just bringing it down.
* This is the solution!
 You multiplied by the synthetic number and subtracted rather than adding during synthetic division.
 You multiplied by the synthetic number rather than bringing the first factor down.
 You divided by the opposite of the factor.
\end{enumerate}

\textbf{General Comment:} Be sure to synthetically divide by the zero of the denominator!
}
\litem{
Factor the polynomial below completely. \textit{To make the problem easier, all zeros are between -5 and 5.}
\[ f(x) = 16x^{3} +40 x^{2} -39 x -45 \]The solution is \( [-3, -0.75, 1.25] \).\begin{enumerate}[label=\Alph*.]
\textbf{Plausible alternative answers include:}* This is the solution!
 Distractor 3: Corresponds to negatives of all zeros AND inversing rational roots.
 Distractor 1: Corresponds to negatives of all zeros.
 Distractor 2: Corresponds to inversing rational roots.
 Distractor 4: Corresponds to moving factors from one rational to another.
\end{enumerate}

\textbf{General Comment:} Remember to try the middle-most integers first as these normally are the zeros. Also, once you get it to a quadratic, you can use your other factoring techniques to finish factoring.
}
\litem{
Factor the polynomial below completely, knowing that $x -5$ is a factor. \textit{To make the problem easier, all zeros are between -5 and 5.}
\[ f(x) = 10x^{4} -69 x^{3} + x^{2} +510 x -200 \]The solution is \( [-2.5, 0.4, 4, 5] \).\begin{enumerate}[label=\Alph*.]
\textbf{Plausible alternative answers include:} Distractor 2: Corresponds to inversing rational roots.
 Distractor 1: Corresponds to negatives of all zeros.
* This is the solution!
 Distractor 3: Corresponds to negatives of all zeros AND inversing rational roots.
 Distractor 4: Corresponds to moving factors from one rational to another.
\end{enumerate}

\textbf{General Comment:} Remember to try the middle-most integers first as these normally are the zeros. Also, once you get it to a quadratic, you can use your other factoring techniques to finish factoring.
}
\litem{
Factor the polynomial below completely. \textit{To make the problem easier, all zeros are between -5 and 5.}
\[ f(x) = 8x^{3} -38 x^{2} +15 x + 36 \]The solution is \( [-0.75, 1.5, 4] \).\begin{enumerate}[label=\Alph*.]
\textbf{Plausible alternative answers include:} Distractor 2: Corresponds to inversing rational roots.
 Distractor 1: Corresponds to negatives of all zeros.
 Distractor 4: Corresponds to moving factors from one rational to another.
* This is the solution!
 Distractor 3: Corresponds to negatives of all zeros AND inversing rational roots.
\end{enumerate}

\textbf{General Comment:} Remember to try the middle-most integers first as these normally are the zeros. Also, once you get it to a quadratic, you can use your other factoring techniques to finish factoring.
}
\litem{
Factor the polynomial below completely, knowing that $x -5$ is a factor. \textit{To make the problem easier, all zeros are between -5 and 5.}
\[ f(x) = 25x^{4} -50 x^{3} -379 x^{2} +8 x + 60 \]The solution is \( [-3, -0.4, 0.4, 5] \).\begin{enumerate}[label=\Alph*.]
\textbf{Plausible alternative answers include:} Distractor 3: Corresponds to negatives of all zeros AND inversing rational roots.
* This is the solution!
 Distractor 4: Corresponds to moving factors from one rational to another.
 Distractor 1: Corresponds to negatives of all zeros.
 Distractor 2: Corresponds to inversing rational roots.
\end{enumerate}

\textbf{General Comment:} Remember to try the middle-most integers first as these normally are the zeros. Also, once you get it to a quadratic, you can use your other factoring techniques to finish factoring.
}
\litem{
Perform the division below. Write the resulting quotient in the form $ax^2+bx+c$ and remainder as $r$.
\[ \frac{10x^{3} +35 x^{2} -49}{x + 3} \]The solution is \( 10x^{2} +5 x -15 + \frac{-4}{x + 3} \).\begin{enumerate}[label=\Alph*.]
\textbf{Plausible alternative answers include:} You multipled by the synthetic number and subtracted rather than adding during synthetic division.
* This is the solution!
 You multipled by the synthetic number rather than bringing the first factor down.
 You divided by the opposite of the factor.
 You divided by the opposite of the factor AND multipled the first factor rather than just bringing it down.
\end{enumerate}

\textbf{General Comment:} Be sure to synthetically divide by the zero of the denominator! Also, make sure to include 0 placeholders for missing terms.
}
\litem{
Perform the division below. Write the resulting quotient in the form $ax^2+bx+c$ and remainder as $r$.
\[ \frac{6x^{3} -24 x^{2} +30 x -8}{x -2} \]The solution is \( 6x^{2} -12 x + 6 + \frac{4}{x -2} \).\begin{enumerate}[label=\Alph*.]
\textbf{Plausible alternative answers include:} You divided by the opposite of the factor.
* This is the solution!
 You multiplied by the synthetic number rather than bringing the first factor down.
 You multiplied by the synthetic number and subtracted rather than adding during synthetic division.
 You divided by the opposite of the factor AND multiplied the first factor rather than just bringing it down.
\end{enumerate}

\textbf{General Comment:} Be sure to synthetically divide by the zero of the denominator!
}
\litem{
Perform the division below. Write the resulting quotient in the form $ax^2+bx+c$ and remainder as $r$.
\[ \frac{15x^{3} -35 x^{2} + 17}{x -2} \]The solution is \( 15x^{2} -5 x -10 + \frac{-3}{x -2} \).\begin{enumerate}[label=\Alph*.]
\textbf{Plausible alternative answers include:} You multipled by the synthetic number rather than bringing the first factor down.
 You multipled by the synthetic number and subtracted rather than adding during synthetic division.
* This is the solution!
 You divided by the opposite of the factor.
 You divided by the opposite of the factor AND multipled the first factor rather than just bringing it down.
\end{enumerate}

\textbf{General Comment:} Be sure to synthetically divide by the zero of the denominator! Also, make sure to include 0 placeholders for missing terms.
}
\litem{
What are the \textit{possible Rational} roots of the polynomial below?
\[ f(x) = 7x^{3} +7 x^{2} +4 x + 6 \]The solution is \( \text{ All combinations of: }\frac{\pm 1,\pm 2,\pm 3,\pm 6}{\pm 1,\pm 7} \).\begin{enumerate}[label=\Alph*.]
\textbf{Plausible alternative answers include:} Distractor 1: Corresponds to the plus or minus factors of a1 only.
 Distractor 3: Corresponds to the plus or minus of the inverse quotient (an/a0) of the factors. 
* This is the solution \textbf{since we asked for the possible Rational roots}!
This would have been the solution \textbf{if asked for the possible Integer roots}!
 Distractor 4: Corresponds to not recalling the theorem for rational roots of a polynomial.
\end{enumerate}

\textbf{General Comment:} We have a way to find the possible Rational roots. The possible Integer roots are the Integers in this list.
}
\litem{
What are the \textit{possible Rational} roots of the polynomial below?
\[ f(x) = 2x^{4} +3 x^{3} +3 x^{2} +7 x + 4 \]The solution is \( \text{ All combinations of: }\frac{\pm 1,\pm 2,\pm 4}{\pm 1,\pm 2} \).\begin{enumerate}[label=\Alph*.]
\textbf{Plausible alternative answers include:}This would have been the solution \textbf{if asked for the possible Integer roots}!
 Distractor 3: Corresponds to the plus or minus of the inverse quotient (an/a0) of the factors. 
 Distractor 1: Corresponds to the plus or minus factors of a1 only.
* This is the solution \textbf{since we asked for the possible Rational roots}!
 Distractor 4: Corresponds to not recalling the theorem for rational roots of a polynomial.
\end{enumerate}

\textbf{General Comment:} We have a way to find the possible Rational roots. The possible Integer roots are the Integers in this list.
}
\litem{
Perform the division below. Write the resulting quotient in the form $ax^2+bx+c$ and remainder as $r$.
\[ \frac{12x^{3} -81 x^{2} +114 x -43}{x -5} \]The solution is \( 12x^{2} -21 x + 9 + \frac{2}{x -5} \).\begin{enumerate}[label=\Alph*.]
\textbf{Plausible alternative answers include:} You divided by the opposite of the factor.
 You multiplied by the synthetic number rather than bringing the first factor down.
 You divided by the opposite of the factor AND multiplied the first factor rather than just bringing it down.
 You multiplied by the synthetic number and subtracted rather than adding during synthetic division.
* This is the solution!
\end{enumerate}

\textbf{General Comment:} Be sure to synthetically divide by the zero of the denominator!
}
\litem{
Factor the polynomial below completely. \textit{To make the problem easier, all zeros are between -5 and 5.}
\[ f(x) = 9x^{3} -33 x^{2} +10 x + 24 \]The solution is \( [-0.67, 1.33, 3] \).\begin{enumerate}[label=\Alph*.]
\textbf{Plausible alternative answers include:} Distractor 2: Corresponds to inversing rational roots.
 Distractor 4: Corresponds to moving factors from one rational to another.
* This is the solution!
 Distractor 1: Corresponds to negatives of all zeros.
 Distractor 3: Corresponds to negatives of all zeros AND inversing rational roots.
\end{enumerate}

\textbf{General Comment:} Remember to try the middle-most integers first as these normally are the zeros. Also, once you get it to a quadratic, you can use your other factoring techniques to finish factoring.
}
\end{enumerate}

\end{document}