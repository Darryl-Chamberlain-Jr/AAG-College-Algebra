\documentclass{extbook}[14pt]
\usepackage{multicol, enumerate, enumitem, hyperref, color, soul, setspace, parskip, fancyhdr, amssymb, amsthm, amsmath, latexsym, units, mathtools}
\everymath{\displaystyle}
\usepackage[headsep=0.5cm,headheight=0cm, left=1 in,right= 1 in,top= 1 in,bottom= 1 in]{geometry}
\usepackage{dashrule}  % Package to use the command below to create lines between items
\newcommand{\litem}[1]{\item #1

\rule{\textwidth}{0.4pt}}
\pagestyle{fancy}
\lhead{}
\chead{Answer Key for Progress Quiz 7 Version A}
\rhead{}
\lfoot{4173-5738}
\cfoot{}
\rfoot{Spring 2021}
\begin{document}
\textbf{This key should allow you to understand why you choose the option you did (beyond just getting a question right or wrong). \href{https://xronos.clas.ufl.edu/mac1105spring2020/courseDescriptionAndMisc/Exams/LearningFromResults}{More instructions on how to use this key can be found here}.}

\textbf{If you have a suggestion to make the keys better, \href{https://forms.gle/CZkbZmPbC9XALEE88}{please fill out the short survey here}.}

\textit{Note: This key is auto-generated and may contain issues and/or errors. The keys are reviewed after each exam to ensure grading is done accurately. If there are issues (like duplicate options), they are noted in the offline gradebook. The keys are a work-in-progress to give students as many resources to improve as possible.}

\rule{\textwidth}{0.4pt}

\begin{enumerate}\litem{
Which of the following intervals describes the Range of the function below?
\[ f(x) = -\log_2{(x-6)}-8 \]The solution is \( (\infty, \infty) \), which is option E.\begin{enumerate}[label=\Alph*.]
\item \( (-\infty, a), a \in [7.6, 9.57] \)

$(-\infty, 8)$, which corresponds to using the using the negative of vertical shift on $(0, \infty)$.
\item \( (-\infty, a), a \in [-8.9, -7.09] \)

$(-\infty, -8)$, which corresponds to using the vertical shift while the Range is $(-\infty, \infty)$.
\item \( [a, \infty), a \in [5.58, 6.46] \)

$[-8, \infty)$, which corresponds to using the flipped Domain AND including the endpoint.
\item \( [a, \infty), a \in [-6.16, -5.57] \)

$[-6, \infty)$, which corresponds to using the negative of the horizontal shift AND including the endpoint.
\item \( (-\infty, \infty) \)

*This is the correct option.
\end{enumerate}

\textbf{General Comment:} \textbf{General Comments}: The domain of a basic logarithmic function is $(0, \infty)$ and the Range is $(-\infty, \infty)$. We can use shifts when finding the Domain, but the Range will always be all Real numbers.
}
\litem{
Solve the equation for $x$ and choose the interval that contains the solution (if it exists).
\[ \log_{4}{(4x+7)}+6 = 2 \]The solution is \( x = -1.749 \), which is option D.\begin{enumerate}[label=\Alph*.]
\item \( x \in [62.25, 64.25] \)

$x = 62.250$, which corresponds to reversing the base and exponent when converting.
\item \( x \in [0.25, 3.25] \)

$x = 2.250$, which corresponds to ignoring the vertical shift when converting to exponential form.
\item \( x \in [65.75, 68.75] \)

$x = 65.750$, which corresponds to reversing the base and exponent when converting and reversing the value with $x$.
\item \( x \in [-2.75, -0.75] \)

* $x = -1.749$, which is the correct option.
\item \( \text{There is no Real solution to the equation.} \)

Corresponds to believing a negative coefficient within the log equation means there is no Real solution.
\end{enumerate}

\textbf{General Comment:} \textbf{General Comments:} First, get the equation in the form $\log_b{(cx+d)} = a$. Then, convert to $b^a = cx+d$ and solve.
}
\litem{
Which of the following intervals describes the Domain of the function below?
\[ f(x) = e^{x-1}-7 \]The solution is \( (-\infty, \infty) \), which is option E.\begin{enumerate}[label=\Alph*.]
\item \( [a, \infty), a \in [5, 10] \)

$[7, \infty)$, which corresponds to using the negative vertical shift AND flipping the Range interval AND including the endpoint.
\item \( (-\infty, a], a \in [-7, -5] \)

$(-\infty, -7]$, which corresponds to using the correct vertical shift *if we wanted the Range* AND including the endpoint.
\item \( (-\infty, a), a \in [-7, -5] \)

$(-\infty, -7)$, which corresponds to using the correct vertical shift *if we wanted the Range*.
\item \( (a, \infty), a \in [5, 10] \)

$(7, \infty)$, which corresponds to using the negative vertical shift AND flipping the Range interval.
\item \( (-\infty, \infty) \)

* This is the correct option.
\end{enumerate}

\textbf{General Comment:} \textbf{General Comments}: Domain of a basic exponential function is $(-\infty, \infty)$ while the Range is $(0, \infty)$. We can shift these intervals [and even flip when $a<0$!] to find the new Domain/Range.
}
\litem{
Which of the following intervals describes the Domain of the function below?
\[ f(x) = \log_2{(x-9)}-3 \]The solution is \( (9, \infty) \), which is option C.\begin{enumerate}[label=\Alph*.]
\item \( [a, \infty), a \in [-5.9, 1] \)

$[-3, \infty)$, which corresponds to using the vertical shift when shifting the Domain AND including the endpoint.
\item \( (-\infty, a], a \in [1.9, 4.4] \)

$(-\infty, 3]$, which corresponds to using the negative vertical shift AND including the endpoint AND flipping the domain.
\item \( (a, \infty), a \in [7.9, 13] \)

* $(9, \infty)$, which is the correct option.
\item \( (-\infty, a), a \in [-13.6, -8.9] \)

$(-\infty, -9)$, which corresponds to flipping the Domain. Remember: the general for is $a*\log(x-h)+k$, \textbf{where $a$ does not affect the domain}.
\item \( (-\infty, \infty) \)

This corresponds to thinking of the range of the log function (or the domain of the exponential function).
\end{enumerate}

\textbf{General Comment:} \textbf{General Comments}: The domain of a basic logarithmic function is $(0, \infty)$ and the Range is $(-\infty, \infty)$. We can use shifts when finding the Domain, but the Range will always be all Real numbers.
}
\litem{
 Solve the equation for $x$ and choose the interval that contains $x$ (if it exists).
\[  21 = \sqrt[5]{\frac{26}{e^{4x}}} \]The solution is \( x = -2.991 \), which is option C.\begin{enumerate}[label=\Alph*.]
\item \( x \in [-1.9, 1.3] \)

$x = -0.708$, which corresponds to treating any root as a square root.
\item \( x \in [-28.7, -26.2] \)

$x = -27.065$, which corresponds to thinking you don't need to take the natural log of both sides before reducing, as if the equation already had a natural log on the right side.
\item \( x \in [-4.2, -1.1] \)

* $x = -2.991$, which is the correct option.
\item \( \text{There is no Real solution to the equation.} \)

This corresponds to believing you cannot solve the equation.
\item \( \text{None of the above.} \)

This corresponds to making an unexpected error.
\end{enumerate}

\textbf{General Comment:} \textbf{General Comments}: After using the properties of logarithmic functions to break up the right-hand side, use $\ln(e) = 1$ to reduce the question to a linear function to solve. You can put $\ln(26)$ into a calculator if you are having trouble.
}
\litem{
Solve the equation for $x$ and choose the interval that contains the solution (if it exists).
\[ \log_{5}{(4x+8)}+5 = 2 \]The solution is \( x = -1.998 \), which is option B.\begin{enumerate}[label=\Alph*.]
\item \( x \in [0.25, 6.25] \)

$x = 4.250$, which corresponds to ignoring the vertical shift when converting to exponential form.
\item \( x \in [-4, 3] \)

* $x = -1.998$, which is the correct option.
\item \( x \in [-58.75, -54.75] \)

$x = -58.750$, which corresponds to reversing the base and exponent when converting and reversing the value with $x$.
\item \( x \in [-62.75, -59.75] \)

$x = -62.750$, which corresponds to reversing the base and exponent when converting.
\item \( \text{There is no Real solution to the equation.} \)

Corresponds to believing a negative coefficient within the log equation means there is no Real solution.
\end{enumerate}

\textbf{General Comment:} \textbf{General Comments:} First, get the equation in the form $\log_b{(cx+d)} = a$. Then, convert to $b^a = cx+d$ and solve.
}
\litem{
 Solve the equation for $x$ and choose the interval that contains $x$ (if it exists).
\[  24 = \sqrt[4]{\frac{28}{e^{9x}}} \]The solution is \( x = -1.042 \), which is option B.\begin{enumerate}[label=\Alph*.]
\item \( x \in [-11.4, -9] \)

$x = -11.037$, which corresponds to thinking you don't need to take the natural log of both sides before reducing, as if the equation already had a natural log on the right side.
\item \( x \in [-1.3, -0.7] \)

* $x = -1.042$, which is the correct option.
\item \( x \in [-0.4, 0] \)

$x = -0.336$, which corresponds to treating any root as a square root.
\item \( \text{There is no Real solution to the equation.} \)

This corresponds to believing you cannot solve the equation.
\item \( \text{None of the above.} \)

This corresponds to making an unexpected error.
\end{enumerate}

\textbf{General Comment:} \textbf{General Comments}: After using the properties of logarithmic functions to break up the right-hand side, use $\ln(e) = 1$ to reduce the question to a linear function to solve. You can put $\ln(28)$ into a calculator if you are having trouble.
}
\litem{
Solve the equation for $x$ and choose the interval that contains the solution (if it exists).
\[ 4^{5x+3} = 25^{4x-3} \]The solution is \( x = 2.324 \), which is option D.\begin{enumerate}[label=\Alph*.]
\item \( x \in [-1.99, 2.01] \)

$x = 1.009$, which corresponds to distributing the $\ln(base)$ to the first term of the exponent only.
\item \( x \in [-15.82, -11.82] \)

$x = -13.816$, which corresponds to distributing the $\ln(base)$ to the second term of the exponent only.
\item \( x \in [-8, -2] \)

$x = -6.000$, which corresponds to solving the numerators as equal while ignoring the bases are different.
\item \( x \in [1.32, 3.32] \)

* $x = 2.324$, which is the correct option.
\item \( \text{There is no Real solution to the equation.} \)

This corresponds to believing there is no solution since the bases are not powers of each other.
\end{enumerate}

\textbf{General Comment:} \textbf{General Comments:} This question was written so that the bases could not be written the same. You will need to take the log of both sides.
}
\litem{
Which of the following intervals describes the Range of the function below?
\[ f(x) = e^{x-3}+4 \]The solution is \( (4, \infty) \), which is option D.\begin{enumerate}[label=\Alph*.]
\item \( [a, \infty), a \in [-3, 11] \)

$[4, \infty)$, which corresponds to including the endpoint.
\item \( (-\infty, a), a \in [-9, -2] \)

$(-\infty, -4)$, which corresponds to using the negative vertical shift AND flipping the Range interval.
\item \( (-\infty, a], a \in [-9, -2] \)

$(-\infty, -4]$, which corresponds to using the negative vertical shift AND flipping the Range interval AND including the endpoint.
\item \( (a, \infty), a \in [-3, 11] \)

* $(4, \infty)$, which is the correct option.
\item \( (-\infty, \infty) \)

This corresponds to confusing range of an exponential function with the domain of an exponential function.
\end{enumerate}

\textbf{General Comment:} \textbf{General Comments}: Domain of a basic exponential function is $(-\infty, \infty)$ while the Range is $(0, \infty)$. We can shift these intervals [and even flip when $a<0$!] to find the new Domain/Range.
}
\litem{
Solve the equation for $x$ and choose the interval that contains the solution (if it exists).
\[ 2^{5x+4} = 125^{3x-5} \]The solution is \( x = 2.442 \), which is option D.\begin{enumerate}[label=\Alph*.]
\item \( x \in [-6.4, -3.1] \)

$x = -4.500$, which corresponds to solving the numerators as equal while ignoring the bases are different.
\item \( x \in [-14.3, -13.3] \)

$x = -13.457$, which corresponds to distributing the $\ln(base)$ to the second term of the exponent only.
\item \( x \in [0.2, 1] \)

$x = 0.817$, which corresponds to distributing the $\ln(base)$ to the first term of the exponent only.
\item \( x \in [1.8, 2.7] \)

* $x = 2.442$, which is the correct option.
\item \( \text{There is no Real solution to the equation.} \)

This corresponds to believing there is no solution since the bases are not powers of each other.
\end{enumerate}

\textbf{General Comment:} \textbf{General Comments:} This question was written so that the bases could not be written the same. You will need to take the log of both sides.
}
\end{enumerate}

\end{document}