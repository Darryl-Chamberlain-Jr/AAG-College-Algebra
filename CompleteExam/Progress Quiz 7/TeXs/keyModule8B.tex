\documentclass{extbook}[14pt]
\usepackage{multicol, enumerate, enumitem, hyperref, color, soul, setspace, parskip, fancyhdr, amssymb, amsthm, amsmath, latexsym, units, mathtools}
\everymath{\displaystyle}
\usepackage[headsep=0.5cm,headheight=0cm, left=1 in,right= 1 in,top= 1 in,bottom= 1 in]{geometry}
\usepackage{dashrule}  % Package to use the command below to create lines between items
\newcommand{\litem}[1]{\item #1

\rule{\textwidth}{0.4pt}}
\pagestyle{fancy}
\lhead{}
\chead{Answer Key for Progress Quiz 7 Version B}
\rhead{}
\lfoot{4173-5738}
\cfoot{}
\rfoot{Spring 2021}
\begin{document}
\textbf{This key should allow you to understand why you choose the option you did (beyond just getting a question right or wrong). \href{https://xronos.clas.ufl.edu/mac1105spring2020/courseDescriptionAndMisc/Exams/LearningFromResults}{More instructions on how to use this key can be found here}.}

\textbf{If you have a suggestion to make the keys better, \href{https://forms.gle/CZkbZmPbC9XALEE88}{please fill out the short survey here}.}

\textit{Note: This key is auto-generated and may contain issues and/or errors. The keys are reviewed after each exam to ensure grading is done accurately. If there are issues (like duplicate options), they are noted in the offline gradebook. The keys are a work-in-progress to give students as many resources to improve as possible.}

\rule{\textwidth}{0.4pt}

\begin{enumerate}\litem{
Which of the following intervals describes the Domain of the function below?
\[ f(x) = \log_2{(x-4)}+8 \]The solution is \( (4, \infty) \), which is option D.\begin{enumerate}[label=\Alph*.]
\item \( (-\infty, a), a \in [-4, 0] \)

$(-\infty, -4)$, which corresponds to flipping the Domain. Remember: the general for is $a*\log(x-h)+k$, \textbf{where $a$ does not affect the domain}.
\item \( [a, \infty), a \in [5, 12] \)

$[8, \infty)$, which corresponds to using the vertical shift when shifting the Domain AND including the endpoint.
\item \( (-\infty, a], a \in [-10, -7] \)

$(-\infty, -8]$, which corresponds to using the negative vertical shift AND including the endpoint AND flipping the domain.
\item \( (a, \infty), a \in [3, 6] \)

* $(4, \infty)$, which is the correct option.
\item \( (-\infty, \infty) \)

This corresponds to thinking of the range of the log function (or the domain of the exponential function).
\end{enumerate}

\textbf{General Comment:} \textbf{General Comments}: The domain of a basic logarithmic function is $(0, \infty)$ and the Range is $(-\infty, \infty)$. We can use shifts when finding the Domain, but the Range will always be all Real numbers.
}
\litem{
Solve the equation for $x$ and choose the interval that contains the solution (if it exists).
\[ \log_{5}{(4x+5)}+6 = 2 \]The solution is \( x = -1.250 \), which is option A.\begin{enumerate}[label=\Alph*.]
\item \( x \in [-1.25, 4.75] \)

* $x = -1.250$, which is the correct option.
\item \( x \in [-262.25, -255.25] \)

$x = -257.250$, which corresponds to reversing the base and exponent when converting.
\item \( x \in [-255.75, -249.75] \)

$x = -254.750$, which corresponds to reversing the base and exponent when converting and reversing the value with $x$.
\item \( x \in [1, 6] \)

$x = 5.000$, which corresponds to ignoring the vertical shift when converting to exponential form.
\item \( \text{There is no Real solution to the equation.} \)

Corresponds to believing a negative coefficient within the log equation means there is no Real solution.
\end{enumerate}

\textbf{General Comment:} \textbf{General Comments:} First, get the equation in the form $\log_b{(cx+d)} = a$. Then, convert to $b^a = cx+d$ and solve.
}
\litem{
Which of the following intervals describes the Domain of the function below?
\[ f(x) = -e^{x+3}+8 \]The solution is \( (-\infty, \infty) \), which is option E.\begin{enumerate}[label=\Alph*.]
\item \( (-\infty, a), a \in [2, 10] \)

$(-\infty, 8)$, which corresponds to using the correct vertical shift *if we wanted the Range*.
\item \( [a, \infty), a \in [-10, -6] \)

$[-8, \infty)$, which corresponds to using the negative vertical shift AND flipping the Range interval AND including the endpoint.
\item \( (-\infty, a], a \in [2, 10] \)

$(-\infty, 8]$, which corresponds to using the correct vertical shift *if we wanted the Range* AND including the endpoint.
\item \( (a, \infty), a \in [-10, -6] \)

$(-8, \infty)$, which corresponds to using the negative vertical shift AND flipping the Range interval.
\item \( (-\infty, \infty) \)

* This is the correct option.
\end{enumerate}

\textbf{General Comment:} \textbf{General Comments}: Domain of a basic exponential function is $(-\infty, \infty)$ while the Range is $(0, \infty)$. We can shift these intervals [and even flip when $a<0$!] to find the new Domain/Range.
}
\litem{
Which of the following intervals describes the Range of the function below?
\[ f(x) = -\log_2{(x-4)}-4 \]The solution is \( (\infty, \infty) \), which is option E.\begin{enumerate}[label=\Alph*.]
\item \( [a, \infty), a \in [-6, -2] \)

$[-4, \infty)$, which corresponds to using the negative of the horizontal shift AND including the endpoint.
\item \( (-\infty, a), a \in [0, 9] \)

$(-\infty, 4)$, which corresponds to using the using the negative of vertical shift on $(0, \infty)$.
\item \( [a, \infty), a \in [0, 9] \)

$[-4, \infty)$, which corresponds to using the flipped Domain AND including the endpoint.
\item \( (-\infty, a), a \in [-6, -2] \)

$(-\infty, -4)$, which corresponds to using the vertical shift while the Range is $(-\infty, \infty)$.
\item \( (-\infty, \infty) \)

*This is the correct option.
\end{enumerate}

\textbf{General Comment:} \textbf{General Comments}: The domain of a basic logarithmic function is $(0, \infty)$ and the Range is $(-\infty, \infty)$. We can use shifts when finding the Domain, but the Range will always be all Real numbers.
}
\litem{
 Solve the equation for $x$ and choose the interval that contains $x$ (if it exists).
\[  11 = \ln{\sqrt[5]{\frac{24}{e^{6x}}}} \]The solution is \( x = -8.637, \text{ which does not fit in any of the interval options.} \), which is option E.\begin{enumerate}[label=\Alph*.]
\item \( x \in [-3.89, -2.83] \)

$x = -3.137$, which corresponds to treating any root as a square root.
\item \( x \in [8.25, 8.96] \)

$x = 8.637$, which is the negative of the correct solution.
\item \( x \in [-3.11, -2.24] \)

$x = -2.528$, which corresponds to thinking you need to take the natural log of the left side before reducing.
\item \( \text{There is no Real solution to the equation.} \)

This corresponds to believing you cannot solve the equation.
\item \( \text{None of the above.} \)

*$x = -8.637$ is the correct solution and does not fit in any of the other intervals.
\end{enumerate}

\textbf{General Comment:} \textbf{General Comments}: After using the properties of logarithmic functions to break up the right-hand side, use $\ln(e) = 1$ to reduce the question to a linear function to solve. You can put $\ln(24)$ into a calculator if you are having trouble.
}
\litem{
Solve the equation for $x$ and choose the interval that contains the solution (if it exists).
\[ \log_{4}{(-4x+8)}+6 = 3 \]The solution is \( x = 1.996 \), which is option A.\begin{enumerate}[label=\Alph*.]
\item \( x \in [-2.5, 2.5] \)

* $x = 1.996$, which is the correct option.
\item \( x \in [-22.9, -18.7] \)

$x = -22.250$, which corresponds to reversing the base and exponent when converting and reversing the value with $x$.
\item \( x \in [-16.5, -11.1] \)

$x = -14.000$, which corresponds to ignoring the vertical shift when converting to exponential form.
\item \( x \in [-20.5, -14.9] \)

$x = -18.250$, which corresponds to reversing the base and exponent when converting.
\item \( \text{There is no Real solution to the equation.} \)

Corresponds to believing a negative coefficient within the log equation means there is no Real solution.
\end{enumerate}

\textbf{General Comment:} \textbf{General Comments:} First, get the equation in the form $\log_b{(cx+d)} = a$. Then, convert to $b^a = cx+d$ and solve.
}
\litem{
 Solve the equation for $x$ and choose the interval that contains $x$ (if it exists).
\[  14 = \sqrt[7]{\frac{25}{e^{3x}}} \]The solution is \( x = -5.085 \), which is option B.\begin{enumerate}[label=\Alph*.]
\item \( x \in [-34.74, -31.74] \)

$x = -33.740$, which corresponds to thinking you don't need to take the natural log of both sides before reducing, as if the equation already had a natural log on the right side.
\item \( x \in [-6.08, -2.08] \)

* $x = -5.085$, which is the correct option.
\item \( x \in [-3.69, 4.31] \)

$x = -0.686$, which corresponds to treating any root as a square root.
\item \( \text{There is no Real solution to the equation.} \)

This corresponds to believing you cannot solve the equation.
\item \( \text{None of the above.} \)

This corresponds to making an unexpected error.
\end{enumerate}

\textbf{General Comment:} \textbf{General Comments}: After using the properties of logarithmic functions to break up the right-hand side, use $\ln(e) = 1$ to reduce the question to a linear function to solve. You can put $\ln(25)$ into a calculator if you are having trouble.
}
\litem{
Solve the equation for $x$ and choose the interval that contains the solution (if it exists).
\[ 3^{5x+4} = \left(\frac{1}{25}\right)^{-2x-4} \]The solution is \( x = -8.978 \), which is option A.\begin{enumerate}[label=\Alph*.]
\item \( x \in [-10, -5.4] \)

* $x = -8.978$, which is the correct option.
\item \( x \in [8.3, 9.3] \)

$x = 8.468$, which corresponds to distributing the $\ln(base)$ to the first term of the exponent only.
\item \( x \in [0.3, 2.1] \)

$x = 1.212$, which corresponds to distributing the $\ln(base)$ to the second term of the exponent only.
\item \( x \in [-3.4, -1] \)

$x = -1.143$, which corresponds to solving the numerators as equal while ignoring the bases are different.
\item \( \text{There is no Real solution to the equation.} \)

This corresponds to believing there is no solution since the bases are not powers of each other.
\end{enumerate}

\textbf{General Comment:} \textbf{General Comments:} This question was written so that the bases could not be written the same. You will need to take the log of both sides.
}
\litem{
Which of the following intervals describes the Range of the function below?
\[ f(x) = e^{x-7}-1 \]The solution is \( (-1, \infty) \), which is option A.\begin{enumerate}[label=\Alph*.]
\item \( (a, \infty), a \in [-1.47, -0.87] \)

* $(-1, \infty)$, which is the correct option.
\item \( (-\infty, a], a \in [0.55, 1.58] \)

$(-\infty, 1]$, which corresponds to using the negative vertical shift AND flipping the Range interval AND including the endpoint.
\item \( (-\infty, a), a \in [0.55, 1.58] \)

$(-\infty, 1)$, which corresponds to using the negative vertical shift AND flipping the Range interval.
\item \( [a, \infty), a \in [-1.47, -0.87] \)

$[-1, \infty)$, which corresponds to including the endpoint.
\item \( (-\infty, \infty) \)

This corresponds to confusing range of an exponential function with the domain of an exponential function.
\end{enumerate}

\textbf{General Comment:} \textbf{General Comments}: Domain of a basic exponential function is $(-\infty, \infty)$ while the Range is $(0, \infty)$. We can shift these intervals [and even flip when $a<0$!] to find the new Domain/Range.
}
\litem{
Solve the equation for $x$ and choose the interval that contains the solution (if it exists).
\[ 5^{3x+3} = 36^{4x-4} \]The solution is \( x = 2.016 \), which is option D.\begin{enumerate}[label=\Alph*.]
\item \( x \in [17.16, 21.16] \)

$x = 19.162$, which corresponds to distributing the $\ln(base)$ to the second term of the exponent only.
\item \( x \in [6, 8] \)

$x = 7.000$, which corresponds to solving the numerators as equal while ignoring the bases are different.
\item \( x \in [-0.26, 1.74] \)

$x = 0.736$, which corresponds to distributing the $\ln(base)$ to the first term of the exponent only.
\item \( x \in [1.02, 3.02] \)

* $x = 2.016$, which is the correct option.
\item \( \text{There is no Real solution to the equation.} \)

This corresponds to believing there is no solution since the bases are not powers of each other.
\end{enumerate}

\textbf{General Comment:} \textbf{General Comments:} This question was written so that the bases could not be written the same. You will need to take the log of both sides.
}
\end{enumerate}

\end{document}