\documentclass{extbook}[14pt]
\usepackage{multicol, enumerate, enumitem, hyperref, color, soul, setspace, parskip, fancyhdr, amssymb, amsthm, amsmath, latexsym, units, mathtools}
\everymath{\displaystyle}
\usepackage[headsep=0.5cm,headheight=0cm, left=1 in,right= 1 in,top= 1 in,bottom= 1 in]{geometry}
\usepackage{dashrule}  % Package to use the command below to create lines between items
\newcommand{\litem}[1]{\item #1

\rule{\textwidth}{0.4pt}}
\pagestyle{fancy}
\lhead{}
\chead{Answer Key for Progress Quiz 7 Version C}
\rhead{}
\lfoot{6523-2736}
\cfoot{}
\rfoot{test}
\begin{document}
\textbf{This key should allow you to understand why you choose the option you did (beyond just getting a question right or wrong). \href{https://xronos.clas.ufl.edu/mac1105spring2020/courseDescriptionAndMisc/Exams/LearningFromResults}{More instructions on how to use this key can be found here}.}

\textbf{If you have a suggestion to make the keys better, \href{https://forms.gle/CZkbZmPbC9XALEE88}{please fill out the short survey here}.}

\textit{Note: This key is auto-generated and may contain issues and/or errors. The keys are reviewed after each exam to ensure grading is done accurately. If there are issues (like duplicate options), they are noted in the offline gradebook. The keys are a work-in-progress to give students as many resources to improve as possible.}

\rule{\textwidth}{0.4pt}

\begin{enumerate}\litem{
Using an interval or intervals, describe all the $x$-values within or including a distance of the given values.
\[ \text{ More than } 9 \text{ units from the number } 3. \]The solution is \( \text{None of the above} \), which is option E.\begin{enumerate}[label=\Alph*.]
\item \( (6, 12) \)

This describes the values less than 3 from 9
\item \( [6, 12] \)

This describes the values no more than 3 from 9
\item \( (-\infty, 6] \cup [12, \infty) \)

This describes the values no less than 3 from 9
\item \( (-\infty, 6) \cup (12, \infty) \)

This describes the values more than 3 from 9
\item \( \text{None of the above} \)

Options A-D described the values [more/less than] 3 units from 9, which is the reverse of what the question asked.
\end{enumerate}

\textbf{General Comment:} When thinking about this language, it helps to draw a number line and try points.
}
\litem{
Solve the linear inequality below. Then, choose the constant and interval combination that describes the solution set.
\[ -10x -9 < -3x -6 \]The solution is \( (-0.429, \infty) \), which is option B.\begin{enumerate}[label=\Alph*.]
\item \( (a, \infty), \text{ where } a \in [0.11, 0.63] \)

 $(0.429, \infty)$, which corresponds to negating the endpoint of the solution.
\item \( (a, \infty), \text{ where } a \in [-0.99, -0.39] \)

* $(-0.429, \infty)$, which is the correct option.
\item \( (-\infty, a), \text{ where } a \in [-0.14, 1.05] \)

 $(-\infty, 0.429)$, which corresponds to switching the direction of the interval AND negating the endpoint. You likely did this if you did not flip the inequality when dividing by a negative as well as not moving values over to a side properly.
\item \( (-\infty, a), \text{ where } a \in [-1.36, -0.07] \)

 $(-\infty, -0.429)$, which corresponds to switching the direction of the interval. You likely did this if you did not flip the inequality when dividing by a negative!
\item \( \text{None of the above}. \)

You may have chosen this if you thought the inequality did not match the ends of the intervals.
\end{enumerate}

\textbf{General Comment:} Remember that less/greater than or equal to includes the endpoint, while less/greater do not. Also, remember that you need to flip the inequality when you multiply or divide by a negative.
}
\litem{
Solve the linear inequality below. Then, choose the constant and interval combination that describes the solution set.
\[ -6 + 6 x < \frac{26 x - 7}{3} \leq 4 + 7 x \]The solution is \( \text{None of the above.} \), which is option E.\begin{enumerate}[label=\Alph*.]
\item \( (-\infty, a) \cup [b, \infty), \text{ where } a \in [0.82, 3.75] \text{ and } b \in [-13.5, -1.5] \)

$(-\infty, 1.38) \cup [-3.80, \infty)$, which corresponds to displaying the and-inequality as an or-inequality and getting negatives of the actual endpoints.
\item \( (a, b], \text{ where } a \in [0.75, 4.5] \text{ and } b \in [-8.25, -2.25] \)

$(1.38, -3.80]$, which is the correct interval but negatives of the actual endpoints.
\item \( [a, b), \text{ where } a \in [-0.45, 1.88] \text{ and } b \in [-7.5, 0.75] \)

$[1.38, -3.80)$, which corresponds to flipping the inequality and getting negatives of the actual endpoints.
\item \( (-\infty, a] \cup (b, \infty), \text{ where } a \in [-0.75, 5.25] \text{ and } b \in [-6, -3] \)

$(-\infty, 1.38] \cup (-3.80, \infty)$, which corresponds to displaying the and-inequality as an or-inequality AND flipping the inequality AND getting negatives of the actual endpoints.
\item \( \text{None of the above.} \)

* This is correct as the answer should be $(-1.38, 3.80]$.
\end{enumerate}

\textbf{General Comment:} To solve, you will need to break up the compound inequality into two inequalities. Be sure to keep track of the inequality! It may be best to draw a number line and graph your solution.
}
\litem{
Solve the linear inequality below. Then, choose the constant and interval combination that describes the solution set.
\[ -8x + 8 < 9x + 9 \]The solution is \( (-0.059, \infty) \), which is option B.\begin{enumerate}[label=\Alph*.]
\item \( (-\infty, a), \text{ where } a \in [-0.19, 0.05] \)

 $(-\infty, -0.059)$, which corresponds to switching the direction of the interval. You likely did this if you did not flip the inequality when dividing by a negative!
\item \( (a, \infty), \text{ where } a \in [-0.2, 0.03] \)

* $(-0.059, \infty)$, which is the correct option.
\item \( (-\infty, a), \text{ where } a \in [0.02, 0.19] \)

 $(-\infty, 0.059)$, which corresponds to switching the direction of the interval AND negating the endpoint. You likely did this if you did not flip the inequality when dividing by a negative as well as not moving values over to a side properly.
\item \( (a, \infty), \text{ where } a \in [-0.01, 0.17] \)

 $(0.059, \infty)$, which corresponds to negating the endpoint of the solution.
\item \( \text{None of the above}. \)

You may have chosen this if you thought the inequality did not match the ends of the intervals.
\end{enumerate}

\textbf{General Comment:} Remember that less/greater than or equal to includes the endpoint, while less/greater do not. Also, remember that you need to flip the inequality when you multiply or divide by a negative.
}
\litem{
Solve the linear inequality below. Then, choose the constant and interval combination that describes the solution set.
\[ -7 + 7 x > 8 x \text{ or } 9 + 7 x < 10 x \]The solution is \( (-\infty, -7.0) \text{ or } (3.0, \infty) \), which is option C.\begin{enumerate}[label=\Alph*.]
\item \( (-\infty, a) \cup (b, \infty), \text{ where } a \in [-6, -1.5] \text{ and } b \in [6, 8.25] \)

Corresponds to inverting the inequality and negating the solution.
\item \( (-\infty, a] \cup [b, \infty), \text{ where } a \in [-9, -3.75] \text{ and } b \in [2.25, 5.25] \)

Corresponds to including the endpoints (when they should be excluded).
\item \( (-\infty, a) \cup (b, \infty), \text{ where } a \in [-7.5, -6.75] \text{ and } b \in [-2.25, 3.75] \)

 * Correct option.
\item \( (-\infty, a] \cup [b, \infty), \text{ where } a \in [-6, -0.75] \text{ and } b \in [6.75, 8.25] \)

Corresponds to including the endpoints AND negating.
\item \( (-\infty, \infty) \)

Corresponds to the variable canceling, which does not happen in this instance.
\end{enumerate}

\textbf{General Comment:} When multiplying or dividing by a negative, flip the sign.
}
\litem{
Solve the linear inequality below. Then, choose the constant and interval combination that describes the solution set.
\[ -4 + 9 x < \frac{58 x - 4}{6} \leq -6 + 7 x \]The solution is \( \text{None of the above.} \), which is option E.\begin{enumerate}[label=\Alph*.]
\item \( [a, b), \text{ where } a \in [3.75, 7.5] \text{ and } b \in [0, 6.75] \)

$[5.00, 2.00)$, which corresponds to flipping the inequality and getting negatives of the actual endpoints.
\item \( (-\infty, a) \cup [b, \infty), \text{ where } a \in [3, 7.5] \text{ and } b \in [1.5, 3] \)

$(-\infty, 5.00) \cup [2.00, \infty)$, which corresponds to displaying the and-inequality as an or-inequality and getting negatives of the actual endpoints.
\item \( (-\infty, a] \cup (b, \infty), \text{ where } a \in [2.25, 11.25] \text{ and } b \in [-1.5, 3.75] \)

$(-\infty, 5.00] \cup (2.00, \infty)$, which corresponds to displaying the and-inequality as an or-inequality AND flipping the inequality AND getting negatives of the actual endpoints.
\item \( (a, b], \text{ where } a \in [1.5, 9.75] \text{ and } b \in [0.75, 4.5] \)

$(5.00, 2.00]$, which is the correct interval but negatives of the actual endpoints.
\item \( \text{None of the above.} \)

* This is correct as the answer should be $(-5.00, -2.00]$.
\end{enumerate}

\textbf{General Comment:} To solve, you will need to break up the compound inequality into two inequalities. Be sure to keep track of the inequality! It may be best to draw a number line and graph your solution.
}
\litem{
Solve the linear inequality below. Then, choose the constant and interval combination that describes the solution set.
\[ -7 + 7 x > 10 x \text{ or } 8 + 4 x < 5 x \]The solution is \( (-\infty, -2.333) \text{ or } (8.0, \infty) \), which is option B.\begin{enumerate}[label=\Alph*.]
\item \( (-\infty, a) \cup (b, \infty), \text{ where } a \in [-14.25, -6.75] \text{ and } b \in [2.25, 3.75] \)

Corresponds to inverting the inequality and negating the solution.
\item \( (-\infty, a) \cup (b, \infty), \text{ where } a \in [-3, 0] \text{ and } b \in [5.25, 9] \)

 * Correct option.
\item \( (-\infty, a] \cup [b, \infty), \text{ where } a \in [-8.25, -3.75] \text{ and } b \in [-0.75, 6] \)

Corresponds to including the endpoints AND negating.
\item \( (-\infty, a] \cup [b, \infty), \text{ where } a \in [-5.25, 1.5] \text{ and } b \in [5.25, 9] \)

Corresponds to including the endpoints (when they should be excluded).
\item \( (-\infty, \infty) \)

Corresponds to the variable canceling, which does not happen in this instance.
\end{enumerate}

\textbf{General Comment:} When multiplying or dividing by a negative, flip the sign.
}
\litem{
Solve the linear inequality below. Then, choose the constant and interval combination that describes the solution set.
\[ \frac{-5}{2} - \frac{5}{6} x \geq \frac{-3}{8} x + \frac{10}{3} \]The solution is \( (-\infty, -12.727] \), which is option B.\begin{enumerate}[label=\Alph*.]
\item \( [a, \infty), \text{ where } a \in [11.25, 13.5] \)

 $[12.727, \infty)$, which corresponds to switching the direction of the interval AND negating the endpoint. You likely did this if you did not flip the inequality when dividing by a negative as well as not moving values over to a side properly.
\item \( (-\infty, a], \text{ where } a \in [-17.25, -10.5] \)

* $(-\infty, -12.727]$, which is the correct option.
\item \( (-\infty, a], \text{ where } a \in [9, 14.25] \)

 $(-\infty, 12.727]$, which corresponds to negating the endpoint of the solution.
\item \( [a, \infty), \text{ where } a \in [-17.25, -10.5] \)

 $[-12.727, \infty)$, which corresponds to switching the direction of the interval. You likely did this if you did not flip the inequality when dividing by a negative!
\item \( \text{None of the above}. \)

You may have chosen this if you thought the inequality did not match the ends of the intervals.
\end{enumerate}

\textbf{General Comment:} Remember that less/greater than or equal to includes the endpoint, while less/greater do not. Also, remember that you need to flip the inequality when you multiply or divide by a negative.
}
\litem{
Solve the linear inequality below. Then, choose the constant and interval combination that describes the solution set.
\[ \frac{7}{4} + \frac{3}{5} x > \frac{5}{8} x - \frac{4}{3} \]The solution is \( (-\infty, 123.333) \), which is option B.\begin{enumerate}[label=\Alph*.]
\item \( (a, \infty), \text{ where } a \in [121.5, 126.75] \)

 $(123.333, \infty)$, which corresponds to switching the direction of the interval. You likely did this if you did not flip the inequality when dividing by a negative!
\item \( (-\infty, a), \text{ where } a \in [120, 124.5] \)

* $(-\infty, 123.333)$, which is the correct option.
\item \( (a, \infty), \text{ where } a \in [-125.25, -122.25] \)

 $(-123.333, \infty)$, which corresponds to switching the direction of the interval AND negating the endpoint. You likely did this if you did not flip the inequality when dividing by a negative as well as not moving values over to a side properly.
\item \( (-\infty, a), \text{ where } a \in [-124.5, -120] \)

 $(-\infty, -123.333)$, which corresponds to negating the endpoint of the solution.
\item \( \text{None of the above}. \)

You may have chosen this if you thought the inequality did not match the ends of the intervals.
\end{enumerate}

\textbf{General Comment:} Remember that less/greater than or equal to includes the endpoint, while less/greater do not. Also, remember that you need to flip the inequality when you multiply or divide by a negative.
}
\litem{
Using an interval or intervals, describe all the $x$-values within or including a distance of the given values.
\[ \text{ More than } 9 \text{ units from the number } 3. \]The solution is \( \text{None of the above} \), which is option E.\begin{enumerate}[label=\Alph*.]
\item \( [6, 12] \)

This describes the values no more than 3 from 9
\item \( (-\infty, 6] \cup [12, \infty) \)

This describes the values no less than 3 from 9
\item \( (6, 12) \)

This describes the values less than 3 from 9
\item \( (-\infty, 6) \cup (12, \infty) \)

This describes the values more than 3 from 9
\item \( \text{None of the above} \)

Options A-D described the values [more/less than] 3 units from 9, which is the reverse of what the question asked.
\end{enumerate}

\textbf{General Comment:} When thinking about this language, it helps to draw a number line and try points.
}
\end{enumerate}

\end{document}