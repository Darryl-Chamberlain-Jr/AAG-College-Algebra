\documentclass{extbook}[14pt]
\usepackage{multicol, enumerate, enumitem, hyperref, color, soul, setspace, parskip, fancyhdr, amssymb, amsthm, amsmath, latexsym, units, mathtools}
\everymath{\displaystyle}
\usepackage[headsep=0.5cm,headheight=0cm, left=1 in,right= 1 in,top= 1 in,bottom= 1 in]{geometry}
\usepackage{dashrule}  % Package to use the command below to create lines between items
\newcommand{\litem}[1]{\item #1

\rule{\textwidth}{0.4pt}}
\pagestyle{fancy}
\lhead{}
\chead{Answer Key for Progress Quiz 7 Version C}
\rhead{}
\lfoot{4173-5738}
\cfoot{}
\rfoot{Spring 2021}
\begin{document}
\textbf{This key should allow you to understand why you choose the option you did (beyond just getting a question right or wrong). \href{https://xronos.clas.ufl.edu/mac1105spring2020/courseDescriptionAndMisc/Exams/LearningFromResults}{More instructions on how to use this key can be found here}.}

\textbf{If you have a suggestion to make the keys better, \href{https://forms.gle/CZkbZmPbC9XALEE88}{please fill out the short survey here}.}

\textit{Note: This key is auto-generated and may contain issues and/or errors. The keys are reviewed after each exam to ensure grading is done accurately. If there are issues (like duplicate options), they are noted in the offline gradebook. The keys are a work-in-progress to give students as many resources to improve as possible.}

\rule{\textwidth}{0.4pt}

\begin{enumerate}\litem{
Find the inverse of the function below (if it exists). Then, evaluate the inverse at $x = 14$ and choose the interval that $f^{-1}(14)$ belongs to.
\[ f(x) = 2 x^2 + 3 \]The solution is \( \text{ The function is not invertible for all Real numbers. } \), which is option E.\begin{enumerate}[label=\Alph*.]
\item \( f^{-1}(14) \in [2.4, 3.69] \)

 Distractor 2: This corresponds to finding the (nonexistent) inverse and not subtracting by the vertical shift.
\item \( f^{-1}(14) \in [2.3, 2.6] \)

 Distractor 1: This corresponds to trying to find the inverse even though the function is not 1-1. 
\item \( f^{-1}(14) \in [3.69, 4.71] \)

 Distractor 3: This corresponds to finding the (nonexistent) inverse and dividing by a negative.
\item \( f^{-1}(14) \in [5.04, 5.78] \)

 Distractor 4: This corresponds to both distractors 2 and 3.
\item \( \text{ The function is not invertible for all Real numbers. } \)

* This is the correct option.
\end{enumerate}

\textbf{General Comment:} Be sure you check that the function is 1-1 before trying to find the inverse!
}
\litem{
Determine whether the function below is 1-1.
\[ f(x) = \sqrt{-6 x + 33} \]The solution is \( \text{yes} \), which is option B.\begin{enumerate}[label=\Alph*.]
\item \( \text{No, because there is an $x$-value that goes to 2 different $y$-values.} \)

Corresponds to the Vertical Line test, which checks if an expression is a function.
\item \( \text{Yes, the function is 1-1.} \)

* This is the solution.
\item \( \text{No, because the range of the function is not $(-\infty, \infty)$.} \)

Corresponds to believing 1-1 means the range is all Real numbers.
\item \( \text{No, because there is a $y$-value that goes to 2 different $x$-values.} \)

Corresponds to the Horizontal Line test, which this function passes.
\item \( \text{No, because the domain of the function is not $(-\infty, \infty)$.} \)

Corresponds to believing 1-1 means the domain is all Real numbers.
\end{enumerate}

\textbf{General Comment:} There are only two valid options: The function is 1-1 OR No because there is a $y$-value that goes to 2 different $x$-values.
}
\litem{
Choose the interval below that $f$ composed with $g$ at $x=1$ is in.
\[ f(x) = -2x^{3} +2 x^{2} +x -3 \text{ and } g(x) = -x^{3} +2 x^{2} -2 x -2 \]The solution is \( 66.0 \), which is option D.\begin{enumerate}[label=\Alph*.]
\item \( (f \circ g)(1) \in [9, 12] \)

 Distractor 3: Corresponds to being slightly off from the solution.
\item \( (f \circ g)(1) \in [17, 25] \)

 Distractor 1: Corresponds to reversing the composition.
\item \( (f \circ g)(1) \in [71, 76] \)

 Distractor 2: Corresponds to being slightly off from the solution.
\item \( (f \circ g)(1) \in [63, 68] \)

* This is the correct solution
\item \( \text{It is not possible to compose the two functions.} \)


\end{enumerate}

\textbf{General Comment:} $f$ composed with $g$ at $x$ means $f(g(x))$. The order matters!
}
\litem{
Add the following functions, then choose the domain of the resulting function from the list below.
\[ f(x) = \sqrt{3x-21}  \text{ and } g(x) = 3x^{3} +8 x^{2} +6 x \]The solution is \( \text{ The domain is all Real numbers greater than or equal to} x = 7.0. \), which is option C.\begin{enumerate}[label=\Alph*.]
\item \( \text{ The domain is all Real numbers less than or equal to } x = a, \text{ where } a \in [-0.5, 9.5] \)


\item \( \text{ The domain is all Real numbers except } x = a, \text{ where } a \in [3.67, 14.67] \)


\item \( \text{ The domain is all Real numbers greater than or equal to } x = a, \text{ where } a \in [2, 9] \)


\item \( \text{ The domain is all Real numbers except } x = a \text{ and } x = b, \text{ where } a \in [-12.67, -2.67] \text{ and } b \in [1.17, 12.17] \)


\item \( \text{ The domain is all Real numbers. } \)


\end{enumerate}

\textbf{General Comment:} The new domain is the intersection of the previous domains.
}
\litem{
Find the inverse of the function below. Then, evaluate the inverse at $x = 8$ and choose the interval that $f^{-1}(8)$ belongs to.
\[ f(x) = \ln{(x-4)}-5 \]The solution is \( f^{-1}(8) = 442417.392 \), which is option C.\begin{enumerate}[label=\Alph*.]
\item \( f^{-1}(8) \in [442407.39, 442415.39] \)

 This solution corresponds to distractor 3.
\item \( f^{-1}(8) \in [162742.79, 162756.79] \)

 This solution corresponds to distractor 2.
\item \( f^{-1}(8) \in [442417.39, 442419.39] \)

 This is the solution.
\item \( f^{-1}(8) \in [20.09, 28.09] \)

 This solution corresponds to distractor 1.
\item \( f^{-1}(8) \in [47.6, 50.6] \)

 This solution corresponds to distractor 4.
\end{enumerate}

\textbf{General Comment:} Natural log and exponential functions always have an inverse. Once you switch the $x$ and $y$, use the conversion $ e^y = x \leftrightarrow y=\ln(x)$.
}
\litem{
Subtract the following functions, then choose the domain of the resulting function from the list below.
\[ f(x) = \frac{4}{5x-34} \text{ and } g(x) = \frac{1}{5x-21} \]The solution is \( \text{ The domain is all Real numbers except } x = 6.8 \text{ and } x = 4.2 \), which is option D.\begin{enumerate}[label=\Alph*.]
\item \( \text{ The domain is all Real numbers greater than or equal to } x = a, \text{ where } a \in [-8, 2] \)


\item \( \text{ The domain is all Real numbers less than or equal to } x = a, \text{ where } a \in [-5.2, -4.2] \)


\item \( \text{ The domain is all Real numbers except } x = a, \text{ where } a \in [-9.2, 2.8] \)


\item \( \text{ The domain is all Real numbers except } x = a \text{ and } x = b, \text{ where } a \in [5.8, 7.8] \text{ and } b \in [-1.8, 14.2] \)


\item \( \text{ The domain is all Real numbers. } \)


\end{enumerate}

\textbf{General Comment:} The new domain is the intersection of the previous domains.
}
\litem{
Determine whether the function below is 1-1.
\[ f(x) = 16 x^2 + 136 x + 289 \]The solution is \( \text{no} \), which is option B.\begin{enumerate}[label=\Alph*.]
\item \( \text{Yes, the function is 1-1.} \)

Corresponds to believing the function passes the Horizontal Line test.
\item \( \text{No, because there is a $y$-value that goes to 2 different $x$-values.} \)

* This is the solution.
\item \( \text{No, because the range of the function is not $(-\infty, \infty)$.} \)

Corresponds to believing 1-1 means the range is all Real numbers.
\item \( \text{No, because the domain of the function is not $(-\infty, \infty)$.} \)

Corresponds to believing 1-1 means the domain is all Real numbers.
\item \( \text{No, because there is an $x$-value that goes to 2 different $y$-values.} \)

Corresponds to the Vertical Line test, which checks if an expression is a function.
\end{enumerate}

\textbf{General Comment:} There are only two valid options: The function is 1-1 OR No because there is a $y$-value that goes to 2 different $x$-values.
}
\litem{
Find the inverse of the function below (if it exists). Then, evaluate the inverse at $x = 13$ and choose the interval the $f^{-1}(13)$ belongs to.
\[ f(x) = \sqrt[3]{3 x - 4} \]The solution is \( 733.6666666666666 \), which is option C.\begin{enumerate}[label=\Alph*.]
\item \( f^{-1}(13) \in [-734, -733.1] \)

 This solution corresponds to distractor 2.
\item \( f^{-1}(13) \in [-732.5, -730] \)

 This solution corresponds to distractor 3.
\item \( f^{-1}(13) \in [732.9, 735.1] \)

* This is the correct solution.
\item \( f^{-1}(13) \in [729.8, 732.7] \)

 Distractor 1: This corresponds to 
\item \( \text{ The function is not invertible for all Real numbers. } \)

 This solution corresponds to distractor 4.
\end{enumerate}

\textbf{General Comment:} Be sure you check that the function is 1-1 before trying to find the inverse!
}
\litem{
Find the inverse of the function below. Then, evaluate the inverse at $x = 7$ and choose the interval that $f^{-1}(7)$ belongs to.
\[ f(x) = \ln{(x+5)}+4 \]The solution is \( f^{-1}(7) = 15.086 \), which is option A.\begin{enumerate}[label=\Alph*.]
\item \( f^{-1}(7) \in [12.5, 18.2] \)

 This is the solution.
\item \( f^{-1}(7) \in [22.8, 27] \)

 This solution corresponds to distractor 3.
\item \( f^{-1}(7) \in [59867.2, 59870.4] \)

 This solution corresponds to distractor 1.
\item \( f^{-1}(7) \in [162756.3, 162759.2] \)

 This solution corresponds to distractor 4.
\item \( f^{-1}(7) \in [10, 15] \)

 This solution corresponds to distractor 2.
\end{enumerate}

\textbf{General Comment:} Natural log and exponential functions always have an inverse. Once you switch the $x$ and $y$, use the conversion $ e^y = x \leftrightarrow y=\ln(x)$.
}
\litem{
Choose the interval below that $f$ composed with $g$ at $x=-1$ is in.
\[ f(x) = 4x^{3} +2 x^{2} +x \text{ and } g(x) = -2x^{3} -2 x^{2} -4 x -4 \]The solution is \( 0.0 \), which is option B.\begin{enumerate}[label=\Alph*.]
\item \( (f \circ g)(-1) \in [32, 36] \)

 Distractor 3: Corresponds to being slightly off from the solution.
\item \( (f \circ g)(-1) \in [-2, 2] \)

* This is the correct solution
\item \( (f \circ g)(-1) \in [44, 45] \)

 Distractor 1: Corresponds to reversing the composition.
\item \( (f \circ g)(-1) \in [9, 16] \)

 Distractor 2: Corresponds to being slightly off from the solution.
\item \( \text{It is not possible to compose the two functions.} \)


\end{enumerate}

\textbf{General Comment:} $f$ composed with $g$ at $x$ means $f(g(x))$. The order matters!
}
\end{enumerate}

\end{document}