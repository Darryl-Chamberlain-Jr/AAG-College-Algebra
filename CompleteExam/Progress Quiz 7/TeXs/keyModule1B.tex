\documentclass{extbook}[14pt]
\usepackage{multicol, enumerate, enumitem, hyperref, color, soul, setspace, parskip, fancyhdr, amssymb, amsthm, amsmath, latexsym, units, mathtools}
\everymath{\displaystyle}
\usepackage[headsep=0.5cm,headheight=0cm, left=1 in,right= 1 in,top= 1 in,bottom= 1 in]{geometry}
\usepackage{dashrule}  % Package to use the command below to create lines between items
\newcommand{\litem}[1]{\item #1

\rule{\textwidth}{0.4pt}}
\pagestyle{fancy}
\lhead{}
\chead{Answer Key for Progress Quiz 7 Version B}
\rhead{}
\lfoot{4173-5738}
\cfoot{}
\rfoot{Spring 2021}
\begin{document}
\textbf{This key should allow you to understand why you choose the option you did (beyond just getting a question right or wrong). \href{https://xronos.clas.ufl.edu/mac1105spring2020/courseDescriptionAndMisc/Exams/LearningFromResults}{More instructions on how to use this key can be found here}.}

\textbf{If you have a suggestion to make the keys better, \href{https://forms.gle/CZkbZmPbC9XALEE88}{please fill out the short survey here}.}

\textit{Note: This key is auto-generated and may contain issues and/or errors. The keys are reviewed after each exam to ensure grading is done accurately. If there are issues (like duplicate options), they are noted in the offline gradebook. The keys are a work-in-progress to give students as many resources to improve as possible.}

\rule{\textwidth}{0.4pt}

\begin{enumerate}\litem{
Simplify the expression below into the form $a+bi$. Then, choose the intervals that $a$ and $b$ belong to.
\[ \frac{72 + 44 i}{-6 - 7 i} \]The solution is \( -8.71  + 2.82 i \), which is option E.\begin{enumerate}[label=\Alph*.]
\item \( a \in [-740.5, -739.5] \text{ and } b \in [2.5, 4.5] \)

 $-740.00  + 2.82 i$, which corresponds to forgetting to multiply the conjugate by the numerator and using a plus instead of a minus in the denominator.
\item \( a \in [-3, -1] \text{ and } b \in [-11.5, -8.5] \)

 $-1.46  - 9.04 i$, which corresponds to forgetting to multiply the conjugate by the numerator and not computing the conjugate correctly.
\item \( a \in [-10, -8] \text{ and } b \in [239.5, 241] \)

 $-8.71  + 240.00 i$, which corresponds to forgetting to multiply the conjugate by the numerator.
\item \( a \in [-13, -10.5] \text{ and } b \in [-6.5, -5] \)

 $-12.00  - 6.29 i$, which corresponds to just dividing the first term by the first term and the second by the second.
\item \( a \in [-10, -8] \text{ and } b \in [2.5, 4.5] \)

* $-8.71  + 2.82 i$, which is the correct option.
\end{enumerate}

\textbf{General Comment:} Multiply the numerator and denominator by the *conjugate* of the denominator, then simplify. For example, if we have $2+3i$, the conjugate is $2-3i$.
}
\litem{
Choose the \textbf{smallest} set of Real numbers that the number below belongs to.
\[ -\sqrt{\frac{10000}{25}} \]The solution is \( \text{Integer} \), which is option D.\begin{enumerate}[label=\Alph*.]
\item \( \text{Not a Real number} \)

These are Nonreal Complex numbers \textbf{OR} things that are not numbers (e.g., dividing by 0).
\item \( \text{Whole} \)

These are the counting numbers with 0 (0, 1, 2, 3, ...)
\item \( \text{Irrational} \)

These cannot be written as a fraction of Integers.
\item \( \text{Integer} \)

* This is the correct option!
\item \( \text{Rational} \)

These are numbers that can be written as fraction of Integers (e.g., -2/3)
\end{enumerate}

\textbf{General Comment:} First, you \textbf{NEED} to simplify the expression. This question simplifies to $-100$. 
 
 Be sure you look at the simplified fraction and not just the decimal expansion. Numbers such as 13, 17, and 19 provide \textbf{long but repeating/terminating decimal expansions!} 
 
 The only ways to *not* be a Real number are: dividing by 0 or taking the square root of a negative number. 
 
 Irrational numbers are more than just square root of 3: adding or subtracting values from square root of 3 is also irrational.
}
\litem{
Simplify the expression below into the form $a+bi$. Then, choose the intervals that $a$ and $b$ belong to.
\[ (5 + 6 i)(-4 + 9 i) \]The solution is \( -74 + 21 i \), which is option B.\begin{enumerate}[label=\Alph*.]
\item \( a \in [34, 40] \text{ and } b \in [-75, -67] \)

 $34 - 69 i$, which corresponds to adding a minus sign in the second term.
\item \( a \in [-76, -72] \text{ and } b \in [18, 23] \)

* $-74 + 21 i$, which is the correct option.
\item \( a \in [-76, -72] \text{ and } b \in [-29, -20] \)

 $-74 - 21 i$, which corresponds to adding a minus sign in both terms.
\item \( a \in [-23, -14] \text{ and } b \in [50, 59] \)

 $-20 + 54 i$, which corresponds to just multiplying the real terms to get the real part of the solution and the coefficients in the complex terms to get the complex part.
\item \( a \in [34, 40] \text{ and } b \in [65, 71] \)

 $34 + 69 i$, which corresponds to adding a minus sign in the first term.
\end{enumerate}

\textbf{General Comment:} You can treat $i$ as a variable and distribute. Just remember that $i^2=-1$, so you can continue to reduce after you distribute.
}
\litem{
Choose the \textbf{smallest} set of Complex numbers that the number below belongs to.
\[ \frac{-9}{10}+\sqrt{-9}i \]The solution is \( \text{Rational} \), which is option C.\begin{enumerate}[label=\Alph*.]
\item \( \text{Nonreal Complex} \)

This is a Complex number $(a+bi)$ that is not Real (has $i$ as part of the number).
\item \( \text{Irrational} \)

These cannot be written as a fraction of Integers. Remember: $\pi$ is not an Integer!
\item \( \text{Rational} \)

* This is the correct option!
\item \( \text{Not a Complex Number} \)

This is not a number. The only non-Complex number we know is dividing by 0 as this is not a number!
\item \( \text{Pure Imaginary} \)

This is a Complex number $(a+bi)$ that \textbf{only} has an imaginary part like $2i$.
\end{enumerate}

\textbf{General Comment:} Be sure to simplify $i^2 = -1$. This may remove the imaginary portion for your number. If you are having trouble, you may want to look at the \textit{Subgroups of the Real Numbers} section.
}
\litem{
Simplify the expression below into the form $a+bi$. Then, choose the intervals that $a$ and $b$ belong to.
\[ \frac{-63 - 55 i}{2 + 4 i} \]The solution is \( -17.30  + 7.10 i \), which is option E.\begin{enumerate}[label=\Alph*.]
\item \( a \in [-32, -30.5] \text{ and } b \in [-14.5, -12.5] \)

 $-31.50  - 13.75 i$, which corresponds to just dividing the first term by the first term and the second by the second.
\item \( a \in [-347, -344.5] \text{ and } b \in [6.5, 8.5] \)

 $-346.00  + 7.10 i$, which corresponds to forgetting to multiply the conjugate by the numerator and using a plus instead of a minus in the denominator.
\item \( a \in [2.5, 5.5] \text{ and } b \in [-19, -17.5] \)

 $4.70  - 18.10 i$, which corresponds to forgetting to multiply the conjugate by the numerator and not computing the conjugate correctly.
\item \( a \in [-18, -17] \text{ and } b \in [141, 142.5] \)

 $-17.30  + 142.00 i$, which corresponds to forgetting to multiply the conjugate by the numerator.
\item \( a \in [-18, -17] \text{ and } b \in [6.5, 8.5] \)

* $-17.30  + 7.10 i$, which is the correct option.
\end{enumerate}

\textbf{General Comment:} Multiply the numerator and denominator by the *conjugate* of the denominator, then simplify. For example, if we have $2+3i$, the conjugate is $2-3i$.
}
\litem{
Simplify the expression below and choose the interval the simplification is contained within.
\[ 6 - 14 \div 19 * 18 - (5 * 2) \]The solution is \( -17.263 \), which is option B.\begin{enumerate}[label=\Alph*.]
\item \( [-27.53, -20.53] \)

 -24.526, which corresponds to not distributing a negative correctly.
\item \( [-18.26, -15.26] \)

* -17.263, which is the correct option.
\item \( [10.96, 16.96] \)

 15.959, which corresponds to not distributing addition and subtraction correctly.
\item \( [-10.04, -0.04] \)

 -4.041, which corresponds to an Order of Operations error: not reading left-to-right for multiplication/division.
\item \( \text{None of the above} \)

 You may have gotten this by making an unanticipated error. If you got a value that is not any of the others, please let the coordinator know so they can help you figure out what happened.
\end{enumerate}

\textbf{General Comment:} While you may remember (or were taught) PEMDAS is done in order, it is actually done as P/E/MD/AS. When we are at MD or AS, we read left to right.
}
\litem{
Simplify the expression below into the form $a+bi$. Then, choose the intervals that $a$ and $b$ belong to.
\[ (7 + 3 i)(-4 - 8 i) \]The solution is \( -4 - 68 i \), which is option A.\begin{enumerate}[label=\Alph*.]
\item \( a \in [-7, -3] \text{ and } b \in [-68, -67] \)

* $-4 - 68 i$, which is the correct option.
\item \( a \in [-7, -3] \text{ and } b \in [68, 72] \)

 $-4 + 68 i$, which corresponds to adding a minus sign in both terms.
\item \( a \in [-31, -26] \text{ and } b \in [-24, -21] \)

 $-28 - 24 i$, which corresponds to just multiplying the real terms to get the real part of the solution and the coefficients in the complex terms to get the complex part.
\item \( a \in [-55, -51] \text{ and } b \in [43, 49] \)

 $-52 + 44 i$, which corresponds to adding a minus sign in the second term.
\item \( a \in [-55, -51] \text{ and } b \in [-45, -37] \)

 $-52 - 44 i$, which corresponds to adding a minus sign in the first term.
\end{enumerate}

\textbf{General Comment:} You can treat $i$ as a variable and distribute. Just remember that $i^2=-1$, so you can continue to reduce after you distribute.
}
\litem{
Choose the \textbf{smallest} set of Real numbers that the number below belongs to.
\[ \sqrt{\frac{990}{9}} \]The solution is \( \text{Irrational} \), which is option B.\begin{enumerate}[label=\Alph*.]
\item \( \text{Integer} \)

These are the negative and positive counting numbers (..., -3, -2, -1, 0, 1, 2, 3, ...)
\item \( \text{Irrational} \)

* This is the correct option!
\item \( \text{Not a Real number} \)

These are Nonreal Complex numbers \textbf{OR} things that are not numbers (e.g., dividing by 0).
\item \( \text{Whole} \)

These are the counting numbers with 0 (0, 1, 2, 3, ...)
\item \( \text{Rational} \)

These are numbers that can be written as fraction of Integers (e.g., -2/3)
\end{enumerate}

\textbf{General Comment:} First, you \textbf{NEED} to simplify the expression. This question simplifies to $\sqrt{110}$. 
 
 Be sure you look at the simplified fraction and not just the decimal expansion. Numbers such as 13, 17, and 19 provide \textbf{long but repeating/terminating decimal expansions!} 
 
 The only ways to *not* be a Real number are: dividing by 0 or taking the square root of a negative number. 
 
 Irrational numbers are more than just square root of 3: adding or subtracting values from square root of 3 is also irrational.
}
\litem{
Choose the \textbf{smallest} set of Complex numbers that the number below belongs to.
\[ \frac{6}{-19}+36i^2 \]The solution is \( \text{Rational} \), which is option D.\begin{enumerate}[label=\Alph*.]
\item \( \text{Irrational} \)

These cannot be written as a fraction of Integers. Remember: $\pi$ is not an Integer!
\item \( \text{Nonreal Complex} \)

This is a Complex number $(a+bi)$ that is not Real (has $i$ as part of the number).
\item \( \text{Pure Imaginary} \)

This is a Complex number $(a+bi)$ that \textbf{only} has an imaginary part like $2i$.
\item \( \text{Rational} \)

* This is the correct option!
\item \( \text{Not a Complex Number} \)

This is not a number. The only non-Complex number we know is dividing by 0 as this is not a number!
\end{enumerate}

\textbf{General Comment:} Be sure to simplify $i^2 = -1$. This may remove the imaginary portion for your number. If you are having trouble, you may want to look at the \textit{Subgroups of the Real Numbers} section.
}
\litem{
Simplify the expression below and choose the interval the simplification is contained within.
\[ 19 - 20^2 + 12 \div 17 * 3 \div 16 \]The solution is \( -380.868 \), which is option A.\begin{enumerate}[label=\Alph*.]
\item \( [-380.98, -380.74] \)

* -380.868, this is the correct option
\item \( [-381.23, -380.87] \)

 -380.985, which corresponds to an Order of Operations error: not reading left-to-right for multiplication/division.
\item \( [418.93, 419.11] \)

 419.015, which corresponds to two Order of Operations errors.
\item \( [419.09, 419.4] \)

 419.132, which corresponds to an Order of Operations error: multiplying by negative before squaring. For example: $(-3)^2 \neq -3^2$
\item \( \text{None of the above} \)

 You may have gotten this by making an unanticipated error. If you got a value that is not any of the others, please let the coordinator know so they can help you figure out what happened.
\end{enumerate}

\textbf{General Comment:} While you may remember (or were taught) PEMDAS is done in order, it is actually done as P/E/MD/AS. When we are at MD or AS, we read left to right.
}
\end{enumerate}

\end{document}