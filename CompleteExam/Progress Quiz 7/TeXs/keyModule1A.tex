\documentclass{extbook}[14pt]
\usepackage{multicol, enumerate, enumitem, hyperref, color, soul, setspace, parskip, fancyhdr, amssymb, amsthm, amsmath, latexsym, units, mathtools}
\everymath{\displaystyle}
\usepackage[headsep=0.5cm,headheight=0cm, left=1 in,right= 1 in,top= 1 in,bottom= 1 in]{geometry}
\usepackage{dashrule}  % Package to use the command below to create lines between items
\newcommand{\litem}[1]{\item #1

\rule{\textwidth}{0.4pt}}
\pagestyle{fancy}
\lhead{}
\chead{Answer Key for Progress Quiz 7 Version A}
\rhead{}
\lfoot{4173-5738}
\cfoot{}
\rfoot{Spring 2021}
\begin{document}
\textbf{This key should allow you to understand why you choose the option you did (beyond just getting a question right or wrong). \href{https://xronos.clas.ufl.edu/mac1105spring2020/courseDescriptionAndMisc/Exams/LearningFromResults}{More instructions on how to use this key can be found here}.}

\textbf{If you have a suggestion to make the keys better, \href{https://forms.gle/CZkbZmPbC9XALEE88}{please fill out the short survey here}.}

\textit{Note: This key is auto-generated and may contain issues and/or errors. The keys are reviewed after each exam to ensure grading is done accurately. If there are issues (like duplicate options), they are noted in the offline gradebook. The keys are a work-in-progress to give students as many resources to improve as possible.}

\rule{\textwidth}{0.4pt}

\begin{enumerate}\litem{
Simplify the expression below into the form $a+bi$. Then, choose the intervals that $a$ and $b$ belong to.
\[ \frac{27 - 22 i}{-6 - 8 i} \]The solution is \( 0.14  + 3.48 i \), which is option A.\begin{enumerate}[label=\Alph*.]
\item \( a \in [-0.5, 0.5] \text{ and } b \in [3.35, 3.9] \)

* $0.14  + 3.48 i$, which is the correct option.
\item \( a \in [13, 15] \text{ and } b \in [3.35, 3.9] \)

 $14.00  + 3.48 i$, which corresponds to forgetting to multiply the conjugate by the numerator and using a plus instead of a minus in the denominator.
\item \( a \in [-0.5, 0.5] \text{ and } b \in [347.85, 348.05] \)

 $0.14  + 348.00 i$, which corresponds to forgetting to multiply the conjugate by the numerator.
\item \( a \in [-4, -2.5] \text{ and } b \in [-1.6, -0.15] \)

 $-3.38  - 0.84 i$, which corresponds to forgetting to multiply the conjugate by the numerator and not computing the conjugate correctly.
\item \( a \in [-5.5, -3.5] \text{ and } b \in [2.35, 2.85] \)

 $-4.50  + 2.75 i$, which corresponds to just dividing the first term by the first term and the second by the second.
\end{enumerate}

\textbf{General Comment:} Multiply the numerator and denominator by the *conjugate* of the denominator, then simplify. For example, if we have $2+3i$, the conjugate is $2-3i$.
}
\litem{
Choose the \textbf{smallest} set of Real numbers that the number below belongs to.
\[ \sqrt{\frac{12996}{36}} \]The solution is \( \text{Whole} \), which is option E.\begin{enumerate}[label=\Alph*.]
\item \( \text{Not a Real number} \)

These are Nonreal Complex numbers \textbf{OR} things that are not numbers (e.g., dividing by 0).
\item \( \text{Rational} \)

These are numbers that can be written as fraction of Integers (e.g., -2/3)
\item \( \text{Irrational} \)

These cannot be written as a fraction of Integers.
\item \( \text{Integer} \)

These are the negative and positive counting numbers (..., -3, -2, -1, 0, 1, 2, 3, ...)
\item \( \text{Whole} \)

* This is the correct option!
\end{enumerate}

\textbf{General Comment:} First, you \textbf{NEED} to simplify the expression. This question simplifies to $114$. 
 
 Be sure you look at the simplified fraction and not just the decimal expansion. Numbers such as 13, 17, and 19 provide \textbf{long but repeating/terminating decimal expansions!} 
 
 The only ways to *not* be a Real number are: dividing by 0 or taking the square root of a negative number. 
 
 Irrational numbers are more than just square root of 3: adding or subtracting values from square root of 3 is also irrational.
}
\litem{
Simplify the expression below into the form $a+bi$. Then, choose the intervals that $a$ and $b$ belong to.
\[ (-2 + 8 i)(4 + 10 i) \]The solution is \( -88 + 12 i \), which is option B.\begin{enumerate}[label=\Alph*.]
\item \( a \in [72, 78] \text{ and } b \in [49, 59] \)

 $72 + 52 i$, which corresponds to adding a minus sign in the second term.
\item \( a \in [-94, -86] \text{ and } b \in [11, 15] \)

* $-88 + 12 i$, which is the correct option.
\item \( a \in [72, 78] \text{ and } b \in [-55, -50] \)

 $72 - 52 i$, which corresponds to adding a minus sign in the first term.
\item \( a \in [-94, -86] \text{ and } b \in [-14, -10] \)

 $-88 - 12 i$, which corresponds to adding a minus sign in both terms.
\item \( a \in [-15, -5] \text{ and } b \in [79, 85] \)

 $-8 + 80 i$, which corresponds to just multiplying the real terms to get the real part of the solution and the coefficients in the complex terms to get the complex part.
\end{enumerate}

\textbf{General Comment:} You can treat $i$ as a variable and distribute. Just remember that $i^2=-1$, so you can continue to reduce after you distribute.
}
\litem{
Choose the \textbf{smallest} set of Complex numbers that the number below belongs to.
\[ \frac{-2}{-13}+81i^2 \]The solution is \( \text{Rational} \), which is option E.\begin{enumerate}[label=\Alph*.]
\item \( \text{Pure Imaginary} \)

This is a Complex number $(a+bi)$ that \textbf{only} has an imaginary part like $2i$.
\item \( \text{Irrational} \)

These cannot be written as a fraction of Integers. Remember: $\pi$ is not an Integer!
\item \( \text{Not a Complex Number} \)

This is not a number. The only non-Complex number we know is dividing by 0 as this is not a number!
\item \( \text{Nonreal Complex} \)

This is a Complex number $(a+bi)$ that is not Real (has $i$ as part of the number).
\item \( \text{Rational} \)

* This is the correct option!
\end{enumerate}

\textbf{General Comment:} Be sure to simplify $i^2 = -1$. This may remove the imaginary portion for your number. If you are having trouble, you may want to look at the \textit{Subgroups of the Real Numbers} section.
}
\litem{
Simplify the expression below into the form $a+bi$. Then, choose the intervals that $a$ and $b$ belong to.
\[ \frac{18 + 77 i}{6 - 5 i} \]The solution is \( -4.54  + 9.05 i \), which is option A.\begin{enumerate}[label=\Alph*.]
\item \( a \in [-5, -3.5] \text{ and } b \in [8.5, 9.5] \)

* $-4.54  + 9.05 i$, which is the correct option.
\item \( a \in [-5, -3.5] \text{ and } b \in [551.5, 553.5] \)

 $-4.54  + 552.00 i$, which corresponds to forgetting to multiply the conjugate by the numerator.
\item \( a \in [2, 3.5] \text{ and } b \in [-16.5, -13.5] \)

 $3.00  - 15.40 i$, which corresponds to just dividing the first term by the first term and the second by the second.
\item \( a \in [7.5, 9.5] \text{ and } b \in [5.5, 6.5] \)

 $8.08  + 6.10 i$, which corresponds to forgetting to multiply the conjugate by the numerator and not computing the conjugate correctly.
\item \( a \in [-278, -276] \text{ and } b \in [8.5, 9.5] \)

 $-277.00  + 9.05 i$, which corresponds to forgetting to multiply the conjugate by the numerator and using a plus instead of a minus in the denominator.
\end{enumerate}

\textbf{General Comment:} Multiply the numerator and denominator by the *conjugate* of the denominator, then simplify. For example, if we have $2+3i$, the conjugate is $2-3i$.
}
\litem{
Simplify the expression below and choose the interval the simplification is contained within.
\[ 17 - 19 \div 3 * 7 - (8 * 5) \]The solution is \( -67.333 \), which is option A.\begin{enumerate}[label=\Alph*.]
\item \( [-68.33, -60.33] \)

* -67.333, which is the correct option.
\item \( [-177.67, -172.67] \)

 -176.667, which corresponds to not distributing a negative correctly.
\item \( [55.1, 60.1] \)

 56.095, which corresponds to not distributing addition and subtraction correctly.
\item \( [-24.9, -21.9] \)

 -23.905, which corresponds to an Order of Operations error: not reading left-to-right for multiplication/division.
\item \( \text{None of the above} \)

 You may have gotten this by making an unanticipated error. If you got a value that is not any of the others, please let the coordinator know so they can help you figure out what happened.
\end{enumerate}

\textbf{General Comment:} While you may remember (or were taught) PEMDAS is done in order, it is actually done as P/E/MD/AS. When we are at MD or AS, we read left to right.
}
\litem{
Simplify the expression below into the form $a+bi$. Then, choose the intervals that $a$ and $b$ belong to.
\[ (-7 - 4 i)(-3 + 9 i) \]The solution is \( 57 - 51 i \), which is option E.\begin{enumerate}[label=\Alph*.]
\item \( a \in [56, 59] \text{ and } b \in [46, 52] \)

 $57 + 51 i$, which corresponds to adding a minus sign in both terms.
\item \( a \in [-16, -10] \text{ and } b \in [-83, -70] \)

 $-15 - 75 i$, which corresponds to adding a minus sign in the first term.
\item \( a \in [-16, -10] \text{ and } b \in [75, 77] \)

 $-15 + 75 i$, which corresponds to adding a minus sign in the second term.
\item \( a \in [17, 26] \text{ and } b \in [-36, -32] \)

 $21 - 36 i$, which corresponds to just multiplying the real terms to get the real part of the solution and the coefficients in the complex terms to get the complex part.
\item \( a \in [56, 59] \text{ and } b \in [-51, -45] \)

* $57 - 51 i$, which is the correct option.
\end{enumerate}

\textbf{General Comment:} You can treat $i$ as a variable and distribute. Just remember that $i^2=-1$, so you can continue to reduce after you distribute.
}
\litem{
Choose the \textbf{smallest} set of Real numbers that the number below belongs to.
\[ -\sqrt{\frac{-765}{9}} \]The solution is \( \text{Not a Real number} \), which is option A.\begin{enumerate}[label=\Alph*.]
\item \( \text{Not a Real number} \)

* This is the correct option!
\item \( \text{Rational} \)

These are numbers that can be written as fraction of Integers (e.g., -2/3)
\item \( \text{Integer} \)

These are the negative and positive counting numbers (..., -3, -2, -1, 0, 1, 2, 3, ...)
\item \( \text{Irrational} \)

These cannot be written as a fraction of Integers.
\item \( \text{Whole} \)

These are the counting numbers with 0 (0, 1, 2, 3, ...)
\end{enumerate}

\textbf{General Comment:} First, you \textbf{NEED} to simplify the expression. This question simplifies to $-\sqrt{85} i$. 
 
 Be sure you look at the simplified fraction and not just the decimal expansion. Numbers such as 13, 17, and 19 provide \textbf{long but repeating/terminating decimal expansions!} 
 
 The only ways to *not* be a Real number are: dividing by 0 or taking the square root of a negative number. 
 
 Irrational numbers are more than just square root of 3: adding or subtracting values from square root of 3 is also irrational.
}
\litem{
Choose the \textbf{smallest} set of Complex numbers that the number below belongs to.
\[ \frac{-12}{2}+\sqrt{-49}i \]The solution is \( \text{Rational} \), which is option A.\begin{enumerate}[label=\Alph*.]
\item \( \text{Rational} \)

* This is the correct option!
\item \( \text{Nonreal Complex} \)

This is a Complex number $(a+bi)$ that is not Real (has $i$ as part of the number).
\item \( \text{Not a Complex Number} \)

This is not a number. The only non-Complex number we know is dividing by 0 as this is not a number!
\item \( \text{Pure Imaginary} \)

This is a Complex number $(a+bi)$ that \textbf{only} has an imaginary part like $2i$.
\item \( \text{Irrational} \)

These cannot be written as a fraction of Integers. Remember: $\pi$ is not an Integer!
\end{enumerate}

\textbf{General Comment:} Be sure to simplify $i^2 = -1$. This may remove the imaginary portion for your number. If you are having trouble, you may want to look at the \textit{Subgroups of the Real Numbers} section.
}
\litem{
Simplify the expression below and choose the interval the simplification is contained within.
\[ 2 - 3^2 + 8 \div 9 * 16 \div 17 \]The solution is \( -6.163 \), which is option D.\begin{enumerate}[label=\Alph*.]
\item \( [-7.25, -6.31] \)

 -6.997, which corresponds to an Order of Operations error: not reading left-to-right for multiplication/division.
\item \( [10.84, 11.05] \)

 11.003, which corresponds to two Order of Operations errors.
\item \( [11.19, 12.39] \)

 11.837, which corresponds to an Order of Operations error: multiplying by negative before squaring. For example: $(-3)^2 \neq -3^2$
\item \( [-6.65, -5.89] \)

* -6.163, this is the correct option
\item \( \text{None of the above} \)

 You may have gotten this by making an unanticipated error. If you got a value that is not any of the others, please let the coordinator know so they can help you figure out what happened.
\end{enumerate}

\textbf{General Comment:} While you may remember (or were taught) PEMDAS is done in order, it is actually done as P/E/MD/AS. When we are at MD or AS, we read left to right.
}
\end{enumerate}

\end{document}