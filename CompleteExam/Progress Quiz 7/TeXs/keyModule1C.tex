\documentclass{extbook}[14pt]
\usepackage{multicol, enumerate, enumitem, hyperref, color, soul, setspace, parskip, fancyhdr, amssymb, amsthm, amsmath, latexsym, units, mathtools}
\everymath{\displaystyle}
\usepackage[headsep=0.5cm,headheight=0cm, left=1 in,right= 1 in,top= 1 in,bottom= 1 in]{geometry}
\usepackage{dashrule}  % Package to use the command below to create lines between items
\newcommand{\litem}[1]{\item #1

\rule{\textwidth}{0.4pt}}
\pagestyle{fancy}
\lhead{}
\chead{Answer Key for Progress Quiz 7 Version C}
\rhead{}
\lfoot{6523-2736}
\cfoot{}
\rfoot{test}
\begin{document}
\textbf{This key should allow you to understand why you choose the option you did (beyond just getting a question right or wrong). \href{https://xronos.clas.ufl.edu/mac1105spring2020/courseDescriptionAndMisc/Exams/LearningFromResults}{More instructions on how to use this key can be found here}.}

\textbf{If you have a suggestion to make the keys better, \href{https://forms.gle/CZkbZmPbC9XALEE88}{please fill out the short survey here}.}

\textit{Note: This key is auto-generated and may contain issues and/or errors. The keys are reviewed after each exam to ensure grading is done accurately. If there are issues (like duplicate options), they are noted in the offline gradebook. The keys are a work-in-progress to give students as many resources to improve as possible.}

\rule{\textwidth}{0.4pt}

\begin{enumerate}\litem{
Choose the \textbf{smallest} set of Real numbers that the number below belongs to.
\[ -\sqrt{\frac{3276}{14}} \]The solution is \( \text{Irrational} \), which is option E.\begin{enumerate}[label=\Alph*.]
\item \( \text{Whole} \)

These are the counting numbers with 0 (0, 1, 2, 3, ...)
\item \( \text{Integer} \)

These are the negative and positive counting numbers (..., -3, -2, -1, 0, 1, 2, 3, ...)
\item \( \text{Not a Real number} \)

These are Nonreal Complex numbers \textbf{OR} things that are not numbers (e.g., dividing by 0).
\item \( \text{Rational} \)

These are numbers that can be written as fraction of Integers (e.g., -2/3)
\item \( \text{Irrational} \)

* This is the correct option!
\end{enumerate}

\textbf{General Comment:} First, you \textbf{NEED} to simplify the expression. This question simplifies to $-\sqrt{234}$. 
 
 Be sure you look at the simplified fraction and not just the decimal expansion. Numbers such as 13, 17, and 19 provide \textbf{long but repeating/terminating decimal expansions!} 
 
 The only ways to *not* be a Real number are: dividing by 0 or taking the square root of a negative number. 
 
 Irrational numbers are more than just square root of 3: adding or subtracting values from square root of 3 is also irrational.
}
\litem{
Choose the \textbf{smallest} set of Real numbers that the number below belongs to.
\[ -\sqrt{\frac{140625}{225}} \]The solution is \( \text{Integer} \), which is option E.\begin{enumerate}[label=\Alph*.]
\item \( \text{Whole} \)

These are the counting numbers with 0 (0, 1, 2, 3, ...)
\item \( \text{Not a Real number} \)

These are Nonreal Complex numbers \textbf{OR} things that are not numbers (e.g., dividing by 0).
\item \( \text{Irrational} \)

These cannot be written as a fraction of Integers.
\item \( \text{Rational} \)

These are numbers that can be written as fraction of Integers (e.g., -2/3)
\item \( \text{Integer} \)

* This is the correct option!
\end{enumerate}

\textbf{General Comment:} First, you \textbf{NEED} to simplify the expression. This question simplifies to $-375$. 
 
 Be sure you look at the simplified fraction and not just the decimal expansion. Numbers such as 13, 17, and 19 provide \textbf{long but repeating/terminating decimal expansions!} 
 
 The only ways to *not* be a Real number are: dividing by 0 or taking the square root of a negative number. 
 
 Irrational numbers are more than just square root of 3: adding or subtracting values from square root of 3 is also irrational.
}
\litem{
Simplify the expression below and choose the interval the simplification is contained within.
\[ 10 - 9^2 + 8 \div 13 * 14 \div 4 \]The solution is \( -68.846 \), which is option B.\begin{enumerate}[label=\Alph*.]
\item \( [91.5, 94.1] \)

 93.154, which corresponds to an Order of Operations error: multiplying by negative before squaring. For example: $(-3)^2 \neq -3^2$
\item \( [-70.3, -67] \)

* -68.846, this is the correct option
\item \( [89.1, 92.6] \)

 91.011, which corresponds to two Order of Operations errors.
\item \( [-72.7, -69.9] \)

 -70.989, which corresponds to an Order of Operations error: not reading left-to-right for multiplication/division.
\item \( \text{None of the above} \)

 You may have gotten this by making an unanticipated error. If you got a value that is not any of the others, please let the coordinator know so they can help you figure out what happened.
\end{enumerate}

\textbf{General Comment:} While you may remember (or were taught) PEMDAS is done in order, it is actually done as P/E/MD/AS. When we are at MD or AS, we read left to right.
}
\litem{
Choose the \textbf{smallest} set of Complex numbers that the number below belongs to.
\[ \sqrt{\frac{-1008}{12}}+\sqrt{63} \]The solution is \( \text{Nonreal Complex} \), which is option E.\begin{enumerate}[label=\Alph*.]
\item \( \text{Not a Complex Number} \)

This is not a number. The only non-Complex number we know is dividing by 0 as this is not a number!
\item \( \text{Pure Imaginary} \)

This is a Complex number $(a+bi)$ that \textbf{only} has an imaginary part like $2i$.
\item \( \text{Rational} \)

These are numbers that can be written as fraction of Integers (e.g., -2/3 + 5)
\item \( \text{Irrational} \)

These cannot be written as a fraction of Integers. Remember: $\pi$ is not an Integer!
\item \( \text{Nonreal Complex} \)

* This is the correct option!
\end{enumerate}

\textbf{General Comment:} Be sure to simplify $i^2 = -1$. This may remove the imaginary portion for your number. If you are having trouble, you may want to look at the \textit{Subgroups of the Real Numbers} section.
}
\litem{
Simplify the expression below into the form $a+bi$. Then, choose the intervals that $a$ and $b$ belong to.
\[ \frac{-27 - 11 i}{-8 + 4 i} \]The solution is \( 2.15  + 2.45 i \), which is option B.\begin{enumerate}[label=\Alph*.]
\item \( a \in [1.8, 2.25] \text{ and } b \in [194, 196.5] \)

 $2.15  + 196.00 i$, which corresponds to forgetting to multiply the conjugate by the numerator.
\item \( a \in [1.8, 2.25] \text{ and } b \in [1.5, 3] \)

* $2.15  + 2.45 i$, which is the correct option.
\item \( a \in [3.35, 3.45] \text{ and } b \in [-4, -1.5] \)

 $3.38  - 2.75 i$, which corresponds to just dividing the first term by the first term and the second by the second.
\item \( a \in [171.65, 172.3] \text{ and } b \in [1.5, 3] \)

 $172.00  + 2.45 i$, which corresponds to forgetting to multiply the conjugate by the numerator and using a plus instead of a minus in the denominator.
\item \( a \in [2.8, 3.3] \text{ and } b \in [-1, 0] \)

 $3.25  - 0.25 i$, which corresponds to forgetting to multiply the conjugate by the numerator and not computing the conjugate correctly.
\end{enumerate}

\textbf{General Comment:} Multiply the numerator and denominator by the *conjugate* of the denominator, then simplify. For example, if we have $2+3i$, the conjugate is $2-3i$.
}
\litem{
Simplify the expression below into the form $a+bi$. Then, choose the intervals that $a$ and $b$ belong to.
\[ (-7 + 9 i)(10 + 4 i) \]The solution is \( -106 + 62 i \), which is option E.\begin{enumerate}[label=\Alph*.]
\item \( a \in [-106, -98] \text{ and } b \in [-62, -60] \)

 $-106 - 62 i$, which corresponds to adding a minus sign in both terms.
\item \( a \in [-35, -24] \text{ and } b \in [117, 121] \)

 $-34 + 118 i$, which corresponds to adding a minus sign in the second term.
\item \( a \in [-35, -24] \text{ and } b \in [-121, -117] \)

 $-34 - 118 i$, which corresponds to adding a minus sign in the first term.
\item \( a \in [-73, -60] \text{ and } b \in [35, 42] \)

 $-70 + 36 i$, which corresponds to just multiplying the real terms to get the real part of the solution and the coefficients in the complex terms to get the complex part.
\item \( a \in [-106, -98] \text{ and } b \in [62, 66] \)

* $-106 + 62 i$, which is the correct option.
\end{enumerate}

\textbf{General Comment:} You can treat $i$ as a variable and distribute. Just remember that $i^2=-1$, so you can continue to reduce after you distribute.
}
\litem{
Simplify the expression below into the form $a+bi$. Then, choose the intervals that $a$ and $b$ belong to.
\[ \frac{-27 - 11 i}{5 + 6 i} \]The solution is \( -3.30  + 1.75 i \), which is option E.\begin{enumerate}[label=\Alph*.]
\item \( a \in [-202.5, -200.5] \text{ and } b \in [1.5, 2.5] \)

 $-201.00  + 1.75 i$, which corresponds to forgetting to multiply the conjugate by the numerator and using a plus instead of a minus in the denominator.
\item \( a \in [-6, -4] \text{ and } b \in [-3, -1.5] \)

 $-5.40  - 1.83 i$, which corresponds to just dividing the first term by the first term and the second by the second.
\item \( a \in [-3.5, -2] \text{ and } b \in [106, 107.5] \)

 $-3.30  + 107.00 i$, which corresponds to forgetting to multiply the conjugate by the numerator.
\item \( a \in [-1.5, 0] \text{ and } b \in [-5, -3] \)

 $-1.13  - 3.56 i$, which corresponds to forgetting to multiply the conjugate by the numerator and not computing the conjugate correctly.
\item \( a \in [-3.5, -2] \text{ and } b \in [1.5, 2.5] \)

* $-3.30  + 1.75 i$, which is the correct option.
\end{enumerate}

\textbf{General Comment:} Multiply the numerator and denominator by the *conjugate* of the denominator, then simplify. For example, if we have $2+3i$, the conjugate is $2-3i$.
}
\litem{
Simplify the expression below and choose the interval the simplification is contained within.
\[ 18 - 19^2 + 7 \div 11 * 4 \div 5 \]The solution is \( -342.491 \), which is option D.\begin{enumerate}[label=\Alph*.]
\item \( [-343.57, -342.68] \)

 -342.968, which corresponds to an Order of Operations error: not reading left-to-right for multiplication/division.
\item \( [378.53, 379.29] \)

 379.032, which corresponds to two Order of Operations errors.
\item \( [379.25, 380.16] \)

 379.509, which corresponds to an Order of Operations error: multiplying by negative before squaring. For example: $(-3)^2 \neq -3^2$
\item \( [-342.59, -342.13] \)

* -342.491, this is the correct option
\item \( \text{None of the above} \)

 You may have gotten this by making an unanticipated error. If you got a value that is not any of the others, please let the coordinator know so they can help you figure out what happened.
\end{enumerate}

\textbf{General Comment:} While you may remember (or were taught) PEMDAS is done in order, it is actually done as P/E/MD/AS. When we are at MD or AS, we read left to right.
}
\litem{
Choose the \textbf{smallest} set of Complex numbers that the number below belongs to.
\[ \sqrt{\frac{0}{576}}+\sqrt{4}i \]The solution is \( \text{Pure Imaginary} \), which is option D.\begin{enumerate}[label=\Alph*.]
\item \( \text{Rational} \)

These are numbers that can be written as fraction of Integers (e.g., -2/3 + 5)
\item \( \text{Irrational} \)

These cannot be written as a fraction of Integers. Remember: $\pi$ is not an Integer!
\item \( \text{Nonreal Complex} \)

This is a Complex number $(a+bi)$ that is not Real (has $i$ as part of the number).
\item \( \text{Pure Imaginary} \)

* This is the correct option!
\item \( \text{Not a Complex Number} \)

This is not a number. The only non-Complex number we know is dividing by 0 as this is not a number!
\end{enumerate}

\textbf{General Comment:} Be sure to simplify $i^2 = -1$. This may remove the imaginary portion for your number. If you are having trouble, you may want to look at the \textit{Subgroups of the Real Numbers} section.
}
\litem{
Simplify the expression below into the form $a+bi$. Then, choose the intervals that $a$ and $b$ belong to.
\[ (7 - 4 i)(-5 + 3 i) \]The solution is \( -23 + 41 i \), which is option A.\begin{enumerate}[label=\Alph*.]
\item \( a \in [-25, -19] \text{ and } b \in [40.51, 41.33] \)

* $-23 + 41 i$, which is the correct option.
\item \( a \in [-25, -19] \text{ and } b \in [-41.91, -40.51] \)

 $-23 - 41 i$, which corresponds to adding a minus sign in both terms.
\item \( a \in [-36, -32] \text{ and } b \in [-12.77, -11.81] \)

 $-35 - 12 i$, which corresponds to just multiplying the real terms to get the real part of the solution and the coefficients in the complex terms to get the complex part.
\item \( a \in [-49, -45] \text{ and } b \in [-1.59, -0.39] \)

 $-47 - i$, which corresponds to adding a minus sign in the second term.
\item \( a \in [-49, -45] \text{ and } b \in [0.87, 1.51] \)

 $-47 + i$, which corresponds to adding a minus sign in the first term.
\end{enumerate}

\textbf{General Comment:} You can treat $i$ as a variable and distribute. Just remember that $i^2=-1$, so you can continue to reduce after you distribute.
}
\end{enumerate}

\end{document}