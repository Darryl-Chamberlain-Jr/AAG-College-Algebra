\documentclass{extbook}[14pt]
\usepackage{multicol, enumerate, enumitem, hyperref, color, soul, setspace, parskip, fancyhdr, amssymb, amsthm, amsmath, latexsym, units, mathtools}
\everymath{\displaystyle}
\usepackage[headsep=0.5cm,headheight=0cm, left=1 in,right= 1 in,top= 1 in,bottom= 1 in]{geometry}
\usepackage{dashrule}  % Package to use the command below to create lines between items
\newcommand{\litem}[1]{\item #1

\rule{\textwidth}{0.4pt}}
\pagestyle{fancy}
\lhead{}
\chead{Answer Key for Progress Quiz 7 Version C}
\rhead{}
\lfoot{4173-5738}
\cfoot{}
\rfoot{Spring 2021}
\begin{document}
\textbf{This key should allow you to understand why you choose the option you did (beyond just getting a question right or wrong). \href{https://xronos.clas.ufl.edu/mac1105spring2020/courseDescriptionAndMisc/Exams/LearningFromResults}{More instructions on how to use this key can be found here}.}

\textbf{If you have a suggestion to make the keys better, \href{https://forms.gle/CZkbZmPbC9XALEE88}{please fill out the short survey here}.}

\textit{Note: This key is auto-generated and may contain issues and/or errors. The keys are reviewed after each exam to ensure grading is done accurately. If there are issues (like duplicate options), they are noted in the offline gradebook. The keys are a work-in-progress to give students as many resources to improve as possible.}

\rule{\textwidth}{0.4pt}

\begin{enumerate}\litem{
Simplify the expression below into the form $a+bi$. Then, choose the intervals that $a$ and $b$ belong to.
\[ \frac{-27 - 22 i}{7 - 8 i} \]The solution is \( -0.12  - 3.27 i \), which is option D.\begin{enumerate}[label=\Alph*.]
\item \( a \in [-13.5, -12.95] \text{ and } b \in [-5, -3] \)

 $-13.00  - 3.27 i$, which corresponds to forgetting to multiply the conjugate by the numerator and using a plus instead of a minus in the denominator.
\item \( a \in [-0.4, -0.05] \text{ and } b \in [-372, -369.5] \)

 $-0.12  - 370.00 i$, which corresponds to forgetting to multiply the conjugate by the numerator.
\item \( a \in [-3.65, -2.55] \text{ and } b \in [0, 2] \)

 $-3.23  + 0.55 i$, which corresponds to forgetting to multiply the conjugate by the numerator and not computing the conjugate correctly.
\item \( a \in [-0.4, -0.05] \text{ and } b \in [-5, -3] \)

* $-0.12  - 3.27 i$, which is the correct option.
\item \( a \in [-4.15, -3.3] \text{ and } b \in [2.5, 3.5] \)

 $-3.86  + 2.75 i$, which corresponds to just dividing the first term by the first term and the second by the second.
\end{enumerate}

\textbf{General Comment:} Multiply the numerator and denominator by the *conjugate* of the denominator, then simplify. For example, if we have $2+3i$, the conjugate is $2-3i$.
}
\litem{
Choose the \textbf{smallest} set of Real numbers that the number below belongs to.
\[ \sqrt{\frac{324}{121}} \]The solution is \( \text{Rational} \), which is option B.\begin{enumerate}[label=\Alph*.]
\item \( \text{Integer} \)

These are the negative and positive counting numbers (..., -3, -2, -1, 0, 1, 2, 3, ...)
\item \( \text{Rational} \)

* This is the correct option!
\item \( \text{Irrational} \)

These cannot be written as a fraction of Integers.
\item \( \text{Not a Real number} \)

These are Nonreal Complex numbers \textbf{OR} things that are not numbers (e.g., dividing by 0).
\item \( \text{Whole} \)

These are the counting numbers with 0 (0, 1, 2, 3, ...)
\end{enumerate}

\textbf{General Comment:} First, you \textbf{NEED} to simplify the expression. This question simplifies to $\frac{18}{11}$. 
 
 Be sure you look at the simplified fraction and not just the decimal expansion. Numbers such as 13, 17, and 19 provide \textbf{long but repeating/terminating decimal expansions!} 
 
 The only ways to *not* be a Real number are: dividing by 0 or taking the square root of a negative number. 
 
 Irrational numbers are more than just square root of 3: adding or subtracting values from square root of 3 is also irrational.
}
\litem{
Simplify the expression below into the form $a+bi$. Then, choose the intervals that $a$ and $b$ belong to.
\[ (-8 + 9 i)(-2 + 5 i) \]The solution is \( -29 - 58 i \), which is option E.\begin{enumerate}[label=\Alph*.]
\item \( a \in [14, 24] \text{ and } b \in [45, 48] \)

 $16 + 45 i$, which corresponds to just multiplying the real terms to get the real part of the solution and the coefficients in the complex terms to get the complex part.
\item \( a \in [59, 69] \text{ and } b \in [-22, -19] \)

 $61 - 22 i$, which corresponds to adding a minus sign in the first term.
\item \( a \in [59, 69] \text{ and } b \in [21, 28] \)

 $61 + 22 i$, which corresponds to adding a minus sign in the second term.
\item \( a \in [-32, -22] \text{ and } b \in [56, 61] \)

 $-29 + 58 i$, which corresponds to adding a minus sign in both terms.
\item \( a \in [-32, -22] \text{ and } b \in [-58, -52] \)

* $-29 - 58 i$, which is the correct option.
\end{enumerate}

\textbf{General Comment:} You can treat $i$ as a variable and distribute. Just remember that $i^2=-1$, so you can continue to reduce after you distribute.
}
\litem{
Choose the \textbf{smallest} set of Complex numbers that the number below belongs to.
\[ \frac{12}{-20}+\sqrt{-100}i \]The solution is \( \text{Rational} \), which is option E.\begin{enumerate}[label=\Alph*.]
\item \( \text{Not a Complex Number} \)

This is not a number. The only non-Complex number we know is dividing by 0 as this is not a number!
\item \( \text{Nonreal Complex} \)

This is a Complex number $(a+bi)$ that is not Real (has $i$ as part of the number).
\item \( \text{Irrational} \)

These cannot be written as a fraction of Integers. Remember: $\pi$ is not an Integer!
\item \( \text{Pure Imaginary} \)

This is a Complex number $(a+bi)$ that \textbf{only} has an imaginary part like $2i$.
\item \( \text{Rational} \)

* This is the correct option!
\end{enumerate}

\textbf{General Comment:} Be sure to simplify $i^2 = -1$. This may remove the imaginary portion for your number. If you are having trouble, you may want to look at the \textit{Subgroups of the Real Numbers} section.
}
\litem{
Simplify the expression below into the form $a+bi$. Then, choose the intervals that $a$ and $b$ belong to.
\[ \frac{72 + 33 i}{1 + 7 i} \]The solution is \( 6.06  - 9.42 i \), which is option A.\begin{enumerate}[label=\Alph*.]
\item \( a \in [5, 7.5] \text{ and } b \in [-10, -7.5] \)

* $6.06  - 9.42 i$, which is the correct option.
\item \( a \in [70, 72.5] \text{ and } b \in [3, 5] \)

 $72.00  + 4.71 i$, which corresponds to just dividing the first term by the first term and the second by the second.
\item \( a \in [-4, -1.5] \text{ and } b \in [10, 11.5] \)

 $-3.18  + 10.74 i$, which corresponds to forgetting to multiply the conjugate by the numerator and not computing the conjugate correctly.
\item \( a \in [5, 7.5] \text{ and } b \in [-472.5, -470] \)

 $6.06  - 471.00 i$, which corresponds to forgetting to multiply the conjugate by the numerator.
\item \( a \in [302.5, 303.5] \text{ and } b \in [-10, -7.5] \)

 $303.00  - 9.42 i$, which corresponds to forgetting to multiply the conjugate by the numerator and using a plus instead of a minus in the denominator.
\end{enumerate}

\textbf{General Comment:} Multiply the numerator and denominator by the *conjugate* of the denominator, then simplify. For example, if we have $2+3i$, the conjugate is $2-3i$.
}
\litem{
Simplify the expression below and choose the interval the simplification is contained within.
\[ 3 - 20 \div 18 * 14 - (4 * 10) \]The solution is \( -52.556 \), which is option D.\begin{enumerate}[label=\Alph*.]
\item \( [37.92, 46.92] \)

 42.921, which corresponds to not distributing addition and subtraction correctly.
\item \( [-166.56, -161.56] \)

 -165.556, which corresponds to not distributing a negative correctly.
\item \( [-42.08, -35.08] \)

 -37.079, which corresponds to an Order of Operations error: not reading left-to-right for multiplication/division.
\item \( [-54.56, -49.56] \)

* -52.556, which is the correct option.
\item \( \text{None of the above} \)

 You may have gotten this by making an unanticipated error. If you got a value that is not any of the others, please let the coordinator know so they can help you figure out what happened.
\end{enumerate}

\textbf{General Comment:} While you may remember (or were taught) PEMDAS is done in order, it is actually done as P/E/MD/AS. When we are at MD or AS, we read left to right.
}
\litem{
Simplify the expression below into the form $a+bi$. Then, choose the intervals that $a$ and $b$ belong to.
\[ (5 - 10 i)(9 - 6 i) \]The solution is \( -15 - 120 i \), which is option D.\begin{enumerate}[label=\Alph*.]
\item \( a \in [104, 109] \text{ and } b \in [54, 68] \)

 $105 + 60 i$, which corresponds to adding a minus sign in the first term.
\item \( a \in [104, 109] \text{ and } b \in [-67, -58] \)

 $105 - 60 i$, which corresponds to adding a minus sign in the second term.
\item \( a \in [-18, -12] \text{ and } b \in [119, 126] \)

 $-15 + 120 i$, which corresponds to adding a minus sign in both terms.
\item \( a \in [-18, -12] \text{ and } b \in [-120, -119] \)

* $-15 - 120 i$, which is the correct option.
\item \( a \in [40, 48] \text{ and } b \in [54, 68] \)

 $45 + 60 i$, which corresponds to just multiplying the real terms to get the real part of the solution and the coefficients in the complex terms to get the complex part.
\end{enumerate}

\textbf{General Comment:} You can treat $i$ as a variable and distribute. Just remember that $i^2=-1$, so you can continue to reduce after you distribute.
}
\litem{
Choose the \textbf{smallest} set of Real numbers that the number below belongs to.
\[ \sqrt{\frac{1210}{11}} \]The solution is \( \text{Irrational} \), which is option C.\begin{enumerate}[label=\Alph*.]
\item \( \text{Integer} \)

These are the negative and positive counting numbers (..., -3, -2, -1, 0, 1, 2, 3, ...)
\item \( \text{Rational} \)

These are numbers that can be written as fraction of Integers (e.g., -2/3)
\item \( \text{Irrational} \)

* This is the correct option!
\item \( \text{Whole} \)

These are the counting numbers with 0 (0, 1, 2, 3, ...)
\item \( \text{Not a Real number} \)

These are Nonreal Complex numbers \textbf{OR} things that are not numbers (e.g., dividing by 0).
\end{enumerate}

\textbf{General Comment:} First, you \textbf{NEED} to simplify the expression. This question simplifies to $\sqrt{110}$. 
 
 Be sure you look at the simplified fraction and not just the decimal expansion. Numbers such as 13, 17, and 19 provide \textbf{long but repeating/terminating decimal expansions!} 
 
 The only ways to *not* be a Real number are: dividing by 0 or taking the square root of a negative number. 
 
 Irrational numbers are more than just square root of 3: adding or subtracting values from square root of 3 is also irrational.
}
\litem{
Choose the \textbf{smallest} set of Complex numbers that the number below belongs to.
\[ \sqrt{\frac{-2178}{11}}+\sqrt{0}i \]The solution is \( \text{Pure Imaginary} \), which is option A.\begin{enumerate}[label=\Alph*.]
\item \( \text{Pure Imaginary} \)

* This is the correct option!
\item \( \text{Nonreal Complex} \)

This is a Complex number $(a+bi)$ that is not Real (has $i$ as part of the number).
\item \( \text{Not a Complex Number} \)

This is not a number. The only non-Complex number we know is dividing by 0 as this is not a number!
\item \( \text{Rational} \)

These are numbers that can be written as fraction of Integers (e.g., -2/3 + 5)
\item \( \text{Irrational} \)

These cannot be written as a fraction of Integers. Remember: $\pi$ is not an Integer!
\end{enumerate}

\textbf{General Comment:} Be sure to simplify $i^2 = -1$. This may remove the imaginary portion for your number. If you are having trouble, you may want to look at the \textit{Subgroups of the Real Numbers} section.
}
\litem{
Simplify the expression below and choose the interval the simplification is contained within.
\[ 12 - 4^2 + 19 \div 1 * 8 \div 13 \]The solution is \( 7.692 \), which is option A.\begin{enumerate}[label=\Alph*.]
\item \( [0.69, 14.69] \)

* 7.692, this is the correct option
\item \( [-5.82, -2.82] \)

 -3.817, which corresponds to an Order of Operations error: not reading left-to-right for multiplication/division.
\item \( [26.18, 31.18] \)

 28.183, which corresponds to two Order of Operations errors.
\item \( [37.69, 40.69] \)

 39.692, which corresponds to an Order of Operations error: multiplying by negative before squaring. For example: $(-3)^2 \neq -3^2$
\item \( \text{None of the above} \)

 You may have gotten this by making an unanticipated error. If you got a value that is not any of the others, please let the coordinator know so they can help you figure out what happened.
\end{enumerate}

\textbf{General Comment:} While you may remember (or were taught) PEMDAS is done in order, it is actually done as P/E/MD/AS. When we are at MD or AS, we read left to right.
}
\end{enumerate}

\end{document}