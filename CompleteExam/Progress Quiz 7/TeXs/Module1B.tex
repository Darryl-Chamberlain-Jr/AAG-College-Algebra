\documentclass[14pt]{extbook}
\usepackage{multicol, enumerate, enumitem, hyperref, color, soul, setspace, parskip, fancyhdr} %General Packages
\usepackage{amssymb, amsthm, amsmath, latexsym, units, mathtools} %Math Packages
\everymath{\displaystyle} %All math in Display Style
% Packages with additional options
\usepackage[headsep=0.5cm,headheight=12pt, left=1 in,right= 1 in,top= 1 in,bottom= 1 in]{geometry}
\usepackage[usenames,dvipsnames]{xcolor}
\usepackage{dashrule}  % Package to use the command below to create lines between items
\newcommand{\litem}[1]{\item#1\hspace*{-1cm}\rule{\textwidth}{0.4pt}}
\pagestyle{fancy}
\lhead{Progress Quiz 7}
\chead{}
\rhead{Version B}
\lfoot{4173-5738}
\cfoot{}
\rfoot{Spring 2021}
\begin{document}

\begin{enumerate}
\litem{
Simplify the expression below into the form $a+bi$. Then, choose the intervals that $a$ and $b$ belong to.\[ \frac{72 + 44 i}{-6 - 7 i} \]\begin{enumerate}[label=\Alph*.]
\item \( a \in [-740.5, -739.5] \text{ and } b \in [2.5, 4.5] \)
\item \( a \in [-3, -1] \text{ and } b \in [-11.5, -8.5] \)
\item \( a \in [-10, -8] \text{ and } b \in [239.5, 241] \)
\item \( a \in [-13, -10.5] \text{ and } b \in [-6.5, -5] \)
\item \( a \in [-10, -8] \text{ and } b \in [2.5, 4.5] \)

\end{enumerate} }
\litem{
Choose the \textbf{smallest} set of Real numbers that the number below belongs to.\[ -\sqrt{\frac{10000}{25}} \]\begin{enumerate}[label=\Alph*.]
\item \( \text{Not a Real number} \)
\item \( \text{Whole} \)
\item \( \text{Irrational} \)
\item \( \text{Integer} \)
\item \( \text{Rational} \)

\end{enumerate} }
\litem{
Simplify the expression below into the form $a+bi$. Then, choose the intervals that $a$ and $b$ belong to.\[ (5 + 6 i)(-4 + 9 i) \]\begin{enumerate}[label=\Alph*.]
\item \( a \in [34, 40] \text{ and } b \in [-75, -67] \)
\item \( a \in [-76, -72] \text{ and } b \in [18, 23] \)
\item \( a \in [-76, -72] \text{ and } b \in [-29, -20] \)
\item \( a \in [-23, -14] \text{ and } b \in [50, 59] \)
\item \( a \in [34, 40] \text{ and } b \in [65, 71] \)

\end{enumerate} }
\litem{
Choose the \textbf{smallest} set of Complex numbers that the number below belongs to.\[ \frac{-9}{10}+\sqrt{-9}i \]\begin{enumerate}[label=\Alph*.]
\item \( \text{Nonreal Complex} \)
\item \( \text{Irrational} \)
\item \( \text{Rational} \)
\item \( \text{Not a Complex Number} \)
\item \( \text{Pure Imaginary} \)

\end{enumerate} }
\litem{
Simplify the expression below into the form $a+bi$. Then, choose the intervals that $a$ and $b$ belong to.\[ \frac{-63 - 55 i}{2 + 4 i} \]\begin{enumerate}[label=\Alph*.]
\item \( a \in [-32, -30.5] \text{ and } b \in [-14.5, -12.5] \)
\item \( a \in [-347, -344.5] \text{ and } b \in [6.5, 8.5] \)
\item \( a \in [2.5, 5.5] \text{ and } b \in [-19, -17.5] \)
\item \( a \in [-18, -17] \text{ and } b \in [141, 142.5] \)
\item \( a \in [-18, -17] \text{ and } b \in [6.5, 8.5] \)

\end{enumerate} }
\litem{
Simplify the expression below and choose the interval the simplification is contained within.\[ 6 - 14 \div 19 * 18 - (5 * 2) \]\begin{enumerate}[label=\Alph*.]
\item \( [-27.53, -20.53] \)
\item \( [-18.26, -15.26] \)
\item \( [10.96, 16.96] \)
\item \( [-10.04, -0.04] \)
\item \( \text{None of the above} \)

\end{enumerate} }
\litem{
Simplify the expression below into the form $a+bi$. Then, choose the intervals that $a$ and $b$ belong to.\[ (7 + 3 i)(-4 - 8 i) \]\begin{enumerate}[label=\Alph*.]
\item \( a \in [-7, -3] \text{ and } b \in [-68, -67] \)
\item \( a \in [-7, -3] \text{ and } b \in [68, 72] \)
\item \( a \in [-31, -26] \text{ and } b \in [-24, -21] \)
\item \( a \in [-55, -51] \text{ and } b \in [43, 49] \)
\item \( a \in [-55, -51] \text{ and } b \in [-45, -37] \)

\end{enumerate} }
\litem{
Choose the \textbf{smallest} set of Real numbers that the number below belongs to.\[ \sqrt{\frac{990}{9}} \]\begin{enumerate}[label=\Alph*.]
\item \( \text{Integer} \)
\item \( \text{Irrational} \)
\item \( \text{Not a Real number} \)
\item \( \text{Whole} \)
\item \( \text{Rational} \)

\end{enumerate} }
\litem{
Choose the \textbf{smallest} set of Complex numbers that the number below belongs to.\[ \frac{6}{-19}+36i^2 \]\begin{enumerate}[label=\Alph*.]
\item \( \text{Irrational} \)
\item \( \text{Nonreal Complex} \)
\item \( \text{Pure Imaginary} \)
\item \( \text{Rational} \)
\item \( \text{Not a Complex Number} \)

\end{enumerate} }
\litem{
Simplify the expression below and choose the interval the simplification is contained within.\[ 19 - 20^2 + 12 \div 17 * 3 \div 16 \]\begin{enumerate}[label=\Alph*.]
\item \( [-380.98, -380.74] \)
\item \( [-381.23, -380.87] \)
\item \( [418.93, 419.11] \)
\item \( [419.09, 419.4] \)
\item \( \text{None of the above} \)

\end{enumerate} }
\end{enumerate}

\end{document}