\documentclass{extbook}[14pt]
\usepackage{multicol, enumerate, enumitem, hyperref, color, soul, setspace, parskip, fancyhdr, amssymb, amsthm, amsmath, latexsym, units, mathtools}
\everymath{\displaystyle}
\usepackage[headsep=0.5cm,headheight=0cm, left=1 in,right= 1 in,top= 1 in,bottom= 1 in]{geometry}
\usepackage{dashrule}  % Package to use the command below to create lines between items
\newcommand{\litem}[1]{\item #1

\rule{\textwidth}{0.4pt}}
\pagestyle{fancy}
\lhead{}
\chead{Answer Key for Progress Quiz 7 Version A}
\rhead{}
\lfoot{4173-5738}
\cfoot{}
\rfoot{Spring 2021}
\begin{document}
\textbf{This key should allow you to understand why you choose the option you did (beyond just getting a question right or wrong). \href{https://xronos.clas.ufl.edu/mac1105spring2020/courseDescriptionAndMisc/Exams/LearningFromResults}{More instructions on how to use this key can be found here}.}

\textbf{If you have a suggestion to make the keys better, \href{https://forms.gle/CZkbZmPbC9XALEE88}{please fill out the short survey here}.}

\textit{Note: This key is auto-generated and may contain issues and/or errors. The keys are reviewed after each exam to ensure grading is done accurately. If there are issues (like duplicate options), they are noted in the offline gradebook. The keys are a work-in-progress to give students as many resources to improve as possible.}

\rule{\textwidth}{0.4pt}

\begin{enumerate}\litem{
Find the inverse of the function below (if it exists). Then, evaluate the inverse at $x = 11$ and choose the interval that $f^{-1}(11)$ belongs to.
\[ f(x) = 2 x^2 + 4 \]The solution is \( \text{ The function is not invertible for all Real numbers. } \), which is option E.\begin{enumerate}[label=\Alph*.]
\item \( f^{-1}(11) \in [1.65, 1.89] \)

 Distractor 1: This corresponds to trying to find the inverse even though the function is not 1-1. 
\item \( f^{-1}(11) \in [4.83, 5.13] \)

 Distractor 4: This corresponds to both distractors 2 and 3.
\item \( f^{-1}(11) \in [2.82, 3.03] \)

 Distractor 3: This corresponds to finding the (nonexistent) inverse and dividing by a negative.
\item \( f^{-1}(11) \in [2.59, 2.81] \)

 Distractor 2: This corresponds to finding the (nonexistent) inverse and not subtracting by the vertical shift.
\item \( \text{ The function is not invertible for all Real numbers. } \)

* This is the correct option.
\end{enumerate}

\textbf{General Comment:} Be sure you check that the function is 1-1 before trying to find the inverse!
}
\litem{
Determine whether the function below is 1-1.
\[ f(x) = 36 x^2 + 420 x + 1225 \]The solution is \( \text{no} \), which is option D.\begin{enumerate}[label=\Alph*.]
\item \( \text{Yes, the function is 1-1.} \)

Corresponds to believing the function passes the Horizontal Line test.
\item \( \text{No, because the domain of the function is not $(-\infty, \infty)$.} \)

Corresponds to believing 1-1 means the domain is all Real numbers.
\item \( \text{No, because there is an $x$-value that goes to 2 different $y$-values.} \)

Corresponds to the Vertical Line test, which checks if an expression is a function.
\item \( \text{No, because there is a $y$-value that goes to 2 different $x$-values.} \)

* This is the solution.
\item \( \text{No, because the range of the function is not $(-\infty, \infty)$.} \)

Corresponds to believing 1-1 means the range is all Real numbers.
\end{enumerate}

\textbf{General Comment:} There are only two valid options: The function is 1-1 OR No because there is a $y$-value that goes to 2 different $x$-values.
}
\litem{
Choose the interval below that $f$ composed with $g$ at $x=-1$ is in.
\[ f(x) = 3x^{3} -3 x^{2} -2 x + 3 \text{ and } g(x) = -x^{3} -1 x^{2} -3 x \]The solution is \( 51.0 \), which is option D.\begin{enumerate}[label=\Alph*.]
\item \( (f \circ g)(-1) \in [-11, -1] \)

 Distractor 3: Corresponds to being slightly off from the solution.
\item \( (f \circ g)(-1) \in [2, 4] \)

 Distractor 1: Corresponds to reversing the composition.
\item \( (f \circ g)(-1) \in [42, 45] \)

 Distractor 2: Corresponds to being slightly off from the solution.
\item \( (f \circ g)(-1) \in [48, 58] \)

* This is the correct solution
\item \( \text{It is not possible to compose the two functions.} \)


\end{enumerate}

\textbf{General Comment:} $f$ composed with $g$ at $x$ means $f(g(x))$. The order matters!
}
\litem{
Multiply the following functions, then choose the domain of the resulting function from the list below.
\[ f(x) = \frac{1}{6x+29} \text{ and } g(x) = \frac{4}{5x-27} \]The solution is \( \text{ The domain is all Real numbers except } x = -4.833333333333333 \text{ and } x = 5.4 \), which is option D.\begin{enumerate}[label=\Alph*.]
\item \( \text{ The domain is all Real numbers less than or equal to } x = a, \text{ where } a \in [-5.6, 3.4] \)


\item \( \text{ The domain is all Real numbers greater than or equal to } x = a, \text{ where } a \in [-8.67, -1.67] \)


\item \( \text{ The domain is all Real numbers except } x = a, \text{ where } a \in [-8.25, 0.75] \)


\item \( \text{ The domain is all Real numbers except } x = a \text{ and } x = b, \text{ where } a \in [-6.83, -3.83] \text{ and } b \in [3.4, 8.4] \)


\item \( \text{ The domain is all Real numbers. } \)


\end{enumerate}

\textbf{General Comment:} The new domain is the intersection of the previous domains.
}
\litem{
Find the inverse of the function below. Then, evaluate the inverse at $x = 7$ and choose the interval that $f^{-1}(7)$ belongs to.
\[ f(x) = e^{x+3}+2 \]The solution is \( f^{-1}(7) = -1.391 \), which is option B.\begin{enumerate}[label=\Alph*.]
\item \( f^{-1}(7) \in [4.38, 4.67] \)

 This solution corresponds to distractor 1.
\item \( f^{-1}(7) \in [-1.58, -1.34] \)

 This is the solution.
\item \( f^{-1}(7) \in [4, 4.2] \)

 This solution corresponds to distractor 2.
\item \( f^{-1}(7) \in [4.21, 4.54] \)

 This solution corresponds to distractor 4.
\item \( f^{-1}(7) \in [3.37, 3.41] \)

 This solution corresponds to distractor 3.
\end{enumerate}

\textbf{General Comment:} Natural log and exponential functions always have an inverse. Once you switch the $x$ and $y$, use the conversion $ e^y = x \leftrightarrow y=\ln(x)$.
}
\litem{
Add the following functions, then choose the domain of the resulting function from the list below.
\[ f(x) = \frac{1}{4x+25} \text{ and } g(x) = \frac{3}{3x-20} \]The solution is \( \text{ The domain is all Real numbers except } x = -6.25 \text{ and } x = 6.666666666666667 \), which is option D.\begin{enumerate}[label=\Alph*.]
\item \( \text{ The domain is all Real numbers except } x = a, \text{ where } a \in [-5.17, -2.17] \)


\item \( \text{ The domain is all Real numbers greater than or equal to } x = a, \text{ where } a \in [-15.2, -4.2] \)


\item \( \text{ The domain is all Real numbers less than or equal to } x = a, \text{ where } a \in [0.33, 3.33] \)


\item \( \text{ The domain is all Real numbers except } x = a \text{ and } x = b, \text{ where } a \in [-9.25, -2.25] \text{ and } b \in [3.67, 16.67] \)


\item \( \text{ The domain is all Real numbers. } \)


\end{enumerate}

\textbf{General Comment:} The new domain is the intersection of the previous domains.
}
\litem{
Determine whether the function below is 1-1.
\[ f(x) = (5 x - 31)^3 \]The solution is \( \text{yes} \), which is option C.\begin{enumerate}[label=\Alph*.]
\item \( \text{No, because there is an $x$-value that goes to 2 different $y$-values.} \)

Corresponds to the Vertical Line test, which checks if an expression is a function.
\item \( \text{No, because the domain of the function is not $(-\infty, \infty)$.} \)

Corresponds to believing 1-1 means the domain is all Real numbers.
\item \( \text{Yes, the function is 1-1.} \)

* This is the solution.
\item \( \text{No, because there is a $y$-value that goes to 2 different $x$-values.} \)

Corresponds to the Horizontal Line test, which this function passes.
\item \( \text{No, because the range of the function is not $(-\infty, \infty)$.} \)

Corresponds to believing 1-1 means the range is all Real numbers.
\end{enumerate}

\textbf{General Comment:} There are only two valid options: The function is 1-1 OR No because there is a $y$-value that goes to 2 different $x$-values.
}
\litem{
Find the inverse of the function below (if it exists). Then, evaluate the inverse at $x = 11$ and choose the interval the $f^{-1}(11)$ belongs to.
\[ f(x) = \sqrt[3]{4 x + 2} \]The solution is \( 332.25 \), which is option A.\begin{enumerate}[label=\Alph*.]
\item \( f^{-1}(11) \in [332.1, 332.6] \)

* This is the correct solution.
\item \( f^{-1}(11) \in [332.7, 334.5] \)

 Distractor 1: This corresponds to 
\item \( f^{-1}(11) \in [-333, -332] \)

 This solution corresponds to distractor 2.
\item \( f^{-1}(11) \in [-335.1, -332.4] \)

 This solution corresponds to distractor 3.
\item \( \text{ The function is not invertible for all Real numbers. } \)

 This solution corresponds to distractor 4.
\end{enumerate}

\textbf{General Comment:} Be sure you check that the function is 1-1 before trying to find the inverse!
}
\litem{
Find the inverse of the function below. Then, evaluate the inverse at $x = 9$ and choose the interval that $f^{-1}(9)$ belongs to.
\[ f(x) = \ln{(x+4)}-2 \]The solution is \( f^{-1}(9) = 59870.142 \), which is option D.\begin{enumerate}[label=\Alph*.]
\item \( f^{-1}(9) \in [1089.63, 1096.63] \)

 This solution corresponds to distractor 1.
\item \( f^{-1}(9) \in [59877.14, 59887.14] \)

 This solution corresponds to distractor 3.
\item \( f^{-1}(9) \in [442404.39, 442424.39] \)

 This solution corresponds to distractor 4.
\item \( f^{-1}(9) \in [59870.14, 59872.14] \)

 This is the solution.
\item \( f^{-1}(9) \in [144.41, 150.41] \)

 This solution corresponds to distractor 2.
\end{enumerate}

\textbf{General Comment:} Natural log and exponential functions always have an inverse. Once you switch the $x$ and $y$, use the conversion $ e^y = x \leftrightarrow y=\ln(x)$.
}
\litem{
Choose the interval below that $f$ composed with $g$ at $x=-1$ is in.
\[ f(x) = 2x^{3} +3 x^{2} -x \text{ and } g(x) = 4x^{3} +3 x^{2} +x + 4 \]The solution is \( 26.0 \), which is option A.\begin{enumerate}[label=\Alph*.]
\item \( (f \circ g)(-1) \in [25, 29] \)

* This is the correct solution
\item \( (f \circ g)(-1) \in [42, 43] \)

 Distractor 3: Corresponds to being slightly off from the solution.
\item \( (f \circ g)(-1) \in [46, 51] \)

 Distractor 1: Corresponds to reversing the composition.
\item \( (f \circ g)(-1) \in [32, 38] \)

 Distractor 2: Corresponds to being slightly off from the solution.
\item \( \text{It is not possible to compose the two functions.} \)


\end{enumerate}

\textbf{General Comment:} $f$ composed with $g$ at $x$ means $f(g(x))$. The order matters!
}
\end{enumerate}

\end{document}