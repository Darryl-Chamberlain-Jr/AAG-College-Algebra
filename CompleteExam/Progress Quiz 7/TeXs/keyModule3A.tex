\documentclass{extbook}[14pt]
\usepackage{multicol, enumerate, enumitem, hyperref, color, soul, setspace, parskip, fancyhdr, amssymb, amsthm, amsmath, latexsym, units, mathtools}
\everymath{\displaystyle}
\usepackage[headsep=0.5cm,headheight=0cm, left=1 in,right= 1 in,top= 1 in,bottom= 1 in]{geometry}
\usepackage{dashrule}  % Package to use the command below to create lines between items
\newcommand{\litem}[1]{\item #1

\rule{\textwidth}{0.4pt}}
\pagestyle{fancy}
\lhead{}
\chead{Answer Key for Progress Quiz 7 Version A}
\rhead{}
\lfoot{6523-2736}
\cfoot{}
\rfoot{test}
\begin{document}
\textbf{This key should allow you to understand why you choose the option you did (beyond just getting a question right or wrong). \href{https://xronos.clas.ufl.edu/mac1105spring2020/courseDescriptionAndMisc/Exams/LearningFromResults}{More instructions on how to use this key can be found here}.}

\textbf{If you have a suggestion to make the keys better, \href{https://forms.gle/CZkbZmPbC9XALEE88}{please fill out the short survey here}.}

\textit{Note: This key is auto-generated and may contain issues and/or errors. The keys are reviewed after each exam to ensure grading is done accurately. If there are issues (like duplicate options), they are noted in the offline gradebook. The keys are a work-in-progress to give students as many resources to improve as possible.}

\rule{\textwidth}{0.4pt}

\begin{enumerate}\litem{
Using an interval or intervals, describe all the $x$-values within or including a distance of the given values.
\[ \text{ Less than } 9 \text{ units from the number } 6. \]The solution is \( (-3, 15) \), which is option B.\begin{enumerate}[label=\Alph*.]
\item \( (-\infty, -3) \cup (15, \infty) \)

This describes the values more than 9 from 6
\item \( (-3, 15) \)

This describes the values less than 9 from 6
\item \( [-3, 15] \)

This describes the values no more than 9 from 6
\item \( (-\infty, -3] \cup [15, \infty) \)

This describes the values no less than 9 from 6
\item \( \text{None of the above} \)

You likely thought the values in the interval were not correct.
\end{enumerate}

\textbf{General Comment:} When thinking about this language, it helps to draw a number line and try points.
}
\litem{
Solve the linear inequality below. Then, choose the constant and interval combination that describes the solution set.
\[ -5x + 6 \leq 6x + 10 \]The solution is \( [-0.364, \infty) \), which is option D.\begin{enumerate}[label=\Alph*.]
\item \( (-\infty, a], \text{ where } a \in [0.06, 1.68] \)

 $(-\infty, 0.364]$, which corresponds to switching the direction of the interval AND negating the endpoint. You likely did this if you did not flip the inequality when dividing by a negative as well as not moving values over to a side properly.
\item \( (-\infty, a], \text{ where } a \in [-0.51, -0.32] \)

 $(-\infty, -0.364]$, which corresponds to switching the direction of the interval. You likely did this if you did not flip the inequality when dividing by a negative!
\item \( [a, \infty), \text{ where } a \in [0.14, 0.87] \)

 $[0.364, \infty)$, which corresponds to negating the endpoint of the solution.
\item \( [a, \infty), \text{ where } a \in [-0.57, 0.31] \)

* $[-0.364, \infty)$, which is the correct option.
\item \( \text{None of the above}. \)

You may have chosen this if you thought the inequality did not match the ends of the intervals.
\end{enumerate}

\textbf{General Comment:} Remember that less/greater than or equal to includes the endpoint, while less/greater do not. Also, remember that you need to flip the inequality when you multiply or divide by a negative.
}
\litem{
Solve the linear inequality below. Then, choose the constant and interval combination that describes the solution set.
\[ -7 - 5 x < \frac{-12 x - 5}{3} \leq -9 - 9 x \]The solution is \( (-5.33, -1.47] \), which is option D.\begin{enumerate}[label=\Alph*.]
\item \( (-\infty, a] \cup (b, \infty), \text{ where } a \in [-6, -4.5] \text{ and } b \in [-2.92, -0.6] \)

$(-\infty, -5.33] \cup (-1.47, \infty)$, which corresponds to displaying the and-inequality as an or-inequality AND flipping the inequality.
\item \( [a, b), \text{ where } a \in [-6.75, -3.75] \text{ and } b \in [-1.8, 0] \)

$[-5.33, -1.47)$, which corresponds to flipping the inequality.
\item \( (-\infty, a) \cup [b, \infty), \text{ where } a \in [-8.25, -3.75] \text{ and } b \in [-3.75, 0] \)

$(-\infty, -5.33) \cup [-1.47, \infty)$, which corresponds to displaying the and-inequality as an or-inequality.
\item \( (a, b], \text{ where } a \in [-6.75, -4.5] \text{ and } b \in [-2.25, 0] \)

* $(-5.33, -1.47]$, which is the correct option.
\item \( \text{None of the above.} \)


\end{enumerate}

\textbf{General Comment:} To solve, you will need to break up the compound inequality into two inequalities. Be sure to keep track of the inequality! It may be best to draw a number line and graph your solution.
}
\litem{
Solve the linear inequality below. Then, choose the constant and interval combination that describes the solution set.
\[ -4x -6 < 9x + 3 \]The solution is \( (-0.692, \infty) \), which is option D.\begin{enumerate}[label=\Alph*.]
\item \( (-\infty, a), \text{ where } a \in [-1.27, -0.48] \)

 $(-\infty, -0.692)$, which corresponds to switching the direction of the interval. You likely did this if you did not flip the inequality when dividing by a negative!
\item \( (-\infty, a), \text{ where } a \in [0.09, 1.26] \)

 $(-\infty, 0.692)$, which corresponds to switching the direction of the interval AND negating the endpoint. You likely did this if you did not flip the inequality when dividing by a negative as well as not moving values over to a side properly.
\item \( (a, \infty), \text{ where } a \in [-0.05, 2.51] \)

 $(0.692, \infty)$, which corresponds to negating the endpoint of the solution.
\item \( (a, \infty), \text{ where } a \in [-1.55, -0.11] \)

* $(-0.692, \infty)$, which is the correct option.
\item \( \text{None of the above}. \)

You may have chosen this if you thought the inequality did not match the ends of the intervals.
\end{enumerate}

\textbf{General Comment:} Remember that less/greater than or equal to includes the endpoint, while less/greater do not. Also, remember that you need to flip the inequality when you multiply or divide by a negative.
}
\litem{
Solve the linear inequality below. Then, choose the constant and interval combination that describes the solution set.
\[ -7 + 5 x > 8 x \text{ or } 5 + 5 x < 6 x \]The solution is \( (-\infty, -2.333) \text{ or } (5.0, \infty) \), which is option A.\begin{enumerate}[label=\Alph*.]
\item \( (-\infty, a) \cup (b, \infty), \text{ where } a \in [-2.85, -0.07] \text{ and } b \in [3, 6] \)

 * Correct option.
\item \( (-\infty, a) \cup (b, \infty), \text{ where } a \in [-8.17, -3.52] \text{ and } b \in [-5.25, 4.5] \)

Corresponds to inverting the inequality and negating the solution.
\item \( (-\infty, a] \cup [b, \infty), \text{ where } a \in [-5.7, -4.65] \text{ and } b \in [-1.65, 4.35] \)

Corresponds to including the endpoints AND negating.
\item \( (-\infty, a] \cup [b, \infty), \text{ where } a \in [-3.67, -2.02] \text{ and } b \in [3.52, 6.3] \)

Corresponds to including the endpoints (when they should be excluded).
\item \( (-\infty, \infty) \)

Corresponds to the variable canceling, which does not happen in this instance.
\end{enumerate}

\textbf{General Comment:} When multiplying or dividing by a negative, flip the sign.
}
\litem{
Solve the linear inequality below. Then, choose the constant and interval combination that describes the solution set.
\[ -8 + 5 x \leq \frac{25 x - 9}{4} < 9 + 3 x \]The solution is \( [-4.60, 3.46) \), which is option D.\begin{enumerate}[label=\Alph*.]
\item \( (-\infty, a] \cup (b, \infty), \text{ where } a \in [-8.25, 2.25] \text{ and } b \in [-0.75, 5.25] \)

$(-\infty, -4.60] \cup (3.46, \infty)$, which corresponds to displaying the and-inequality as an or-inequality.
\item \( (-\infty, a) \cup [b, \infty), \text{ where } a \in [-5.25, -3.75] \text{ and } b \in [-1.5, 4.5] \)

$(-\infty, -4.60) \cup [3.46, \infty)$, which corresponds to displaying the and-inequality as an or-inequality AND flipping the inequality.
\item \( (a, b], \text{ where } a \in [-8.25, -3.75] \text{ and } b \in [3, 7.5] \)

$(-4.60, 3.46]$, which corresponds to flipping the inequality.
\item \( [a, b), \text{ where } a \in [-5.25, 3.75] \text{ and } b \in [1.5, 7.5] \)

$[-4.60, 3.46)$, which is the correct option.
\item \( \text{None of the above.} \)


\end{enumerate}

\textbf{General Comment:} To solve, you will need to break up the compound inequality into two inequalities. Be sure to keep track of the inequality! It may be best to draw a number line and graph your solution.
}
\litem{
Solve the linear inequality below. Then, choose the constant and interval combination that describes the solution set.
\[ -7 + 8 x > 11 x \text{ or } 6 + 4 x < 5 x \]The solution is \( (-\infty, -2.333) \text{ or } (6.0, \infty) \), which is option A.\begin{enumerate}[label=\Alph*.]
\item \( (-\infty, a) \cup (b, \infty), \text{ where } a \in [-3.75, 0.75] \text{ and } b \in [4.5, 6.75] \)

 * Correct option.
\item \( (-\infty, a] \cup [b, \infty), \text{ where } a \in [-5.25, 0.15] \text{ and } b \in [4.5, 15] \)

Corresponds to including the endpoints (when they should be excluded).
\item \( (-\infty, a] \cup [b, \infty), \text{ where } a \in [-6.6, -4.65] \text{ and } b \in [-1.5, 3] \)

Corresponds to including the endpoints AND negating.
\item \( (-\infty, a) \cup (b, \infty), \text{ where } a \in [-9.75, -5.25] \text{ and } b \in [2.25, 5.25] \)

Corresponds to inverting the inequality and negating the solution.
\item \( (-\infty, \infty) \)

Corresponds to the variable canceling, which does not happen in this instance.
\end{enumerate}

\textbf{General Comment:} When multiplying or dividing by a negative, flip the sign.
}
\litem{
Solve the linear inequality below. Then, choose the constant and interval combination that describes the solution set.
\[ \frac{10}{3} - \frac{4}{9} x \geq \frac{4}{7} x + \frac{3}{5} \]The solution is \( (-\infty, 2.691] \), which is option A.\begin{enumerate}[label=\Alph*.]
\item \( (-\infty, a], \text{ where } a \in [0, 3.75] \)

* $(-\infty, 2.691]$, which is the correct option.
\item \( [a, \infty), \text{ where } a \in [2.25, 6] \)

 $[2.691, \infty)$, which corresponds to switching the direction of the interval. You likely did this if you did not flip the inequality when dividing by a negative!
\item \( (-\infty, a], \text{ where } a \in [-6, -0.75] \)

 $(-\infty, -2.691]$, which corresponds to negating the endpoint of the solution.
\item \( [a, \infty), \text{ where } a \in [-5.25, 0] \)

 $[-2.691, \infty)$, which corresponds to switching the direction of the interval AND negating the endpoint. You likely did this if you did not flip the inequality when dividing by a negative as well as not moving values over to a side properly.
\item \( \text{None of the above}. \)

You may have chosen this if you thought the inequality did not match the ends of the intervals.
\end{enumerate}

\textbf{General Comment:} Remember that less/greater than or equal to includes the endpoint, while less/greater do not. Also, remember that you need to flip the inequality when you multiply or divide by a negative.
}
\litem{
Solve the linear inequality below. Then, choose the constant and interval combination that describes the solution set.
\[ \frac{-4}{3} - \frac{5}{5} x \geq \frac{5}{4} x + \frac{7}{6} \]The solution is \( (-\infty, -1.111] \), which is option D.\begin{enumerate}[label=\Alph*.]
\item \( (-\infty, a], \text{ where } a \in [0.75, 4.5] \)

 $(-\infty, 1.111]$, which corresponds to negating the endpoint of the solution.
\item \( [a, \infty), \text{ where } a \in [-2.32, -0.9] \)

 $[-1.111, \infty)$, which corresponds to switching the direction of the interval. You likely did this if you did not flip the inequality when dividing by a negative!
\item \( [a, \infty), \text{ where } a \in [-0.22, 1.57] \)

 $[1.111, \infty)$, which corresponds to switching the direction of the interval AND negating the endpoint. You likely did this if you did not flip the inequality when dividing by a negative as well as not moving values over to a side properly.
\item \( (-\infty, a], \text{ where } a \in [-6, 0.75] \)

* $(-\infty, -1.111]$, which is the correct option.
\item \( \text{None of the above}. \)

You may have chosen this if you thought the inequality did not match the ends of the intervals.
\end{enumerate}

\textbf{General Comment:} Remember that less/greater than or equal to includes the endpoint, while less/greater do not. Also, remember that you need to flip the inequality when you multiply or divide by a negative.
}
\litem{
Using an interval or intervals, describe all the $x$-values within or including a distance of the given values.
\[ \text{ No less than } 6 \text{ units from the number } 2. \]The solution is \( \text{None of the above} \), which is option E.\begin{enumerate}[label=\Alph*.]
\item \( (4, 8) \)

This describes the values less than 2 from 6
\item \( (-\infty, 4) \cup (8, \infty) \)

This describes the values more than 2 from 6
\item \( (-\infty, 4] \cup [8, \infty) \)

This describes the values no less than 2 from 6
\item \( [4, 8] \)

This describes the values no more than 2 from 6
\item \( \text{None of the above} \)

Options A-D described the values [more/less than] 2 units from 6, which is the reverse of what the question asked.
\end{enumerate}

\textbf{General Comment:} When thinking about this language, it helps to draw a number line and try points.
}
\end{enumerate}

\end{document}