\documentclass{extbook}[14pt]
\usepackage{multicol, enumerate, enumitem, hyperref, color, soul, setspace, parskip, fancyhdr, amssymb, amsthm, amsmath, latexsym, units, mathtools}
\everymath{\displaystyle}
\usepackage[headsep=0.5cm,headheight=0cm, left=1 in,right= 1 in,top= 1 in,bottom= 1 in]{geometry}
\usepackage{dashrule}  % Package to use the command below to create lines between items
\newcommand{\litem}[1]{\item #1

\rule{\textwidth}{0.4pt}}
\pagestyle{fancy}
\lhead{}
\chead{Answer Key for Progress Quiz 7 Version B}
\rhead{}
\lfoot{6523-2736}
\cfoot{}
\rfoot{test}
\begin{document}
\textbf{This key should allow you to understand why you choose the option you did (beyond just getting a question right or wrong). \href{https://xronos.clas.ufl.edu/mac1105spring2020/courseDescriptionAndMisc/Exams/LearningFromResults}{More instructions on how to use this key can be found here}.}

\textbf{If you have a suggestion to make the keys better, \href{https://forms.gle/CZkbZmPbC9XALEE88}{please fill out the short survey here}.}

\textit{Note: This key is auto-generated and may contain issues and/or errors. The keys are reviewed after each exam to ensure grading is done accurately. If there are issues (like duplicate options), they are noted in the offline gradebook. The keys are a work-in-progress to give students as many resources to improve as possible.}

\rule{\textwidth}{0.4pt}

\begin{enumerate}\litem{
Using an interval or intervals, describe all the $x$-values within or including a distance of the given values.
\[ \text{ No more than } 3 \text{ units from the number } 4. \]The solution is \( \text{None of the above} \), which is option E.\begin{enumerate}[label=\Alph*.]
\item \( [-1, 7] \)

This describes the values no more than 4 from 3
\item \( (-1, 7) \)

This describes the values less than 4 from 3
\item \( (-\infty, -1] \cup [7, \infty) \)

This describes the values no less than 4 from 3
\item \( (-\infty, -1) \cup (7, \infty) \)

This describes the values more than 4 from 3
\item \( \text{None of the above} \)

Options A-D described the values [more/less than] 4 units from 3, which is the reverse of what the question asked.
\end{enumerate}

\textbf{General Comment:} When thinking about this language, it helps to draw a number line and try points.
}
\litem{
Solve the linear inequality below. Then, choose the constant and interval combination that describes the solution set.
\[ -6x + 4 \geq 3x -5 \]The solution is \( (-\infty, 1.0] \), which is option D.\begin{enumerate}[label=\Alph*.]
\item \( [a, \infty), \text{ where } a \in [1, 2] \)

 $[1.0, \infty)$, which corresponds to switching the direction of the interval. You likely did this if you did not flip the inequality when dividing by a negative!
\item \( [a, \infty), \text{ where } a \in [-4, 0] \)

 $[-1.0, \infty)$, which corresponds to switching the direction of the interval AND negating the endpoint. You likely did this if you did not flip the inequality when dividing by a negative as well as not moving values over to a side properly.
\item \( (-\infty, a], \text{ where } a \in [-1.5, 0] \)

 $(-\infty, -1.0]$, which corresponds to negating the endpoint of the solution.
\item \( (-\infty, a], \text{ where } a \in [-0.6, 2.3] \)

* $(-\infty, 1.0]$, which is the correct option.
\item \( \text{None of the above}. \)

You may have chosen this if you thought the inequality did not match the ends of the intervals.
\end{enumerate}

\textbf{General Comment:} Remember that less/greater than or equal to includes the endpoint, while less/greater do not. Also, remember that you need to flip the inequality when you multiply or divide by a negative.
}
\litem{
Solve the linear inequality below. Then, choose the constant and interval combination that describes the solution set.
\[ -6 - 5 x < \frac{-30 x + 3}{7} \leq 6 - 7 x \]The solution is \( (-9.00, 2.05] \), which is option D.\begin{enumerate}[label=\Alph*.]
\item \( [a, b), \text{ where } a \in [-10.5, -6.75] \text{ and } b \in [0, 3] \)

$[-9.00, 2.05)$, which corresponds to flipping the inequality.
\item \( (-\infty, a] \cup (b, \infty), \text{ where } a \in [-9.75, -5.25] \text{ and } b \in [-1.5, 13.5] \)

$(-\infty, -9.00] \cup (2.05, \infty)$, which corresponds to displaying the and-inequality as an or-inequality AND flipping the inequality.
\item \( (-\infty, a) \cup [b, \infty), \text{ where } a \in [-15.75, -8.25] \text{ and } b \in [0.75, 3] \)

$(-\infty, -9.00) \cup [2.05, \infty)$, which corresponds to displaying the and-inequality as an or-inequality.
\item \( (a, b], \text{ where } a \in [-11.25, -6.75] \text{ and } b \in [0.75, 8.25] \)

* $(-9.00, 2.05]$, which is the correct option.
\item \( \text{None of the above.} \)


\end{enumerate}

\textbf{General Comment:} To solve, you will need to break up the compound inequality into two inequalities. Be sure to keep track of the inequality! It may be best to draw a number line and graph your solution.
}
\litem{
Solve the linear inequality below. Then, choose the constant and interval combination that describes the solution set.
\[ -9x + 9 \leq -7x + 6 \]The solution is \( [1.5, \infty) \), which is option C.\begin{enumerate}[label=\Alph*.]
\item \( (-\infty, a], \text{ where } a \in [0.5, 3.5] \)

 $(-\infty, 1.5]$, which corresponds to switching the direction of the interval. You likely did this if you did not flip the inequality when dividing by a negative!
\item \( [a, \infty), \text{ where } a \in [-2.8, -1] \)

 $[-1.5, \infty)$, which corresponds to negating the endpoint of the solution.
\item \( [a, \infty), \text{ where } a \in [0.9, 1.9] \)

* $[1.5, \infty)$, which is the correct option.
\item \( (-\infty, a], \text{ where } a \in [-5.5, 0.5] \)

 $(-\infty, -1.5]$, which corresponds to switching the direction of the interval AND negating the endpoint. You likely did this if you did not flip the inequality when dividing by a negative as well as not moving values over to a side properly.
\item \( \text{None of the above}. \)

You may have chosen this if you thought the inequality did not match the ends of the intervals.
\end{enumerate}

\textbf{General Comment:} Remember that less/greater than or equal to includes the endpoint, while less/greater do not. Also, remember that you need to flip the inequality when you multiply or divide by a negative.
}
\litem{
Solve the linear inequality below. Then, choose the constant and interval combination that describes the solution set.
\[ 6 + 6 x > 9 x \text{ or } 9 + 7 x < 9 x \]The solution is \( (-\infty, 2.0) \text{ or } (4.5, \infty) \), which is option A.\begin{enumerate}[label=\Alph*.]
\item \( (-\infty, a) \cup (b, \infty), \text{ where } a \in [0.75, 7.5] \text{ and } b \in [3, 9.75] \)

 * Correct option.
\item \( (-\infty, a] \cup [b, \infty), \text{ where } a \in [-1.5, 3] \text{ and } b \in [1.5, 6] \)

Corresponds to including the endpoints (when they should be excluded).
\item \( (-\infty, a) \cup (b, \infty), \text{ where } a \in [-9, -3.75] \text{ and } b \in [-5.25, 0] \)

Corresponds to inverting the inequality and negating the solution.
\item \( (-\infty, a] \cup [b, \infty), \text{ where } a \in [-8.25, -3.75] \text{ and } b \in [-4.5, 0.75] \)

Corresponds to including the endpoints AND negating.
\item \( (-\infty, \infty) \)

Corresponds to the variable canceling, which does not happen in this instance.
\end{enumerate}

\textbf{General Comment:} When multiplying or dividing by a negative, flip the sign.
}
\litem{
Solve the linear inequality below. Then, choose the constant and interval combination that describes the solution set.
\[ -9 + 5 x \leq \frac{66 x - 4}{8} < 8 + 8 x \]The solution is \( [-2.62, 34.00) \), which is option B.\begin{enumerate}[label=\Alph*.]
\item \( (a, b], \text{ where } a \in [-3.75, 1.5] \text{ and } b \in [30.75, 36] \)

$(-2.62, 34.00]$, which corresponds to flipping the inequality.
\item \( [a, b), \text{ where } a \in [-6, 2.25] \text{ and } b \in [28.5, 36.75] \)

$[-2.62, 34.00)$, which is the correct option.
\item \( (-\infty, a] \cup (b, \infty), \text{ where } a \in [-3, -0.75] \text{ and } b \in [32.25, 39] \)

$(-\infty, -2.62] \cup (34.00, \infty)$, which corresponds to displaying the and-inequality as an or-inequality.
\item \( (-\infty, a) \cup [b, \infty), \text{ where } a \in [-6.75, -0.75] \text{ and } b \in [33.75, 35.25] \)

$(-\infty, -2.62) \cup [34.00, \infty)$, which corresponds to displaying the and-inequality as an or-inequality AND flipping the inequality.
\item \( \text{None of the above.} \)


\end{enumerate}

\textbf{General Comment:} To solve, you will need to break up the compound inequality into two inequalities. Be sure to keep track of the inequality! It may be best to draw a number line and graph your solution.
}
\litem{
Solve the linear inequality below. Then, choose the constant and interval combination that describes the solution set.
\[ -6 + 5 x > 8 x \text{ or } 6 + 3 x < 5 x \]The solution is \( (-\infty, -2.0) \text{ or } (3.0, \infty) \), which is option A.\begin{enumerate}[label=\Alph*.]
\item \( (-\infty, a) \cup (b, \infty), \text{ where } a \in [-2.79, -1.49] \text{ and } b \in [2.1, 4.8] \)

 * Correct option.
\item \( (-\infty, a) \cup (b, \infty), \text{ where } a \in [-3.39, -2.93] \text{ and } b \in [-0.38, 2.32] \)

Corresponds to inverting the inequality and negating the solution.
\item \( (-\infty, a] \cup [b, \infty), \text{ where } a \in [-2.48, -0.3] \text{ and } b \in [2.4, 4.88] \)

Corresponds to including the endpoints (when they should be excluded).
\item \( (-\infty, a] \cup [b, \infty), \text{ where } a \in [-4.95, -2.62] \text{ and } b \in [0.82, 2.32] \)

Corresponds to including the endpoints AND negating.
\item \( (-\infty, \infty) \)

Corresponds to the variable canceling, which does not happen in this instance.
\end{enumerate}

\textbf{General Comment:} When multiplying or dividing by a negative, flip the sign.
}
\litem{
Solve the linear inequality below. Then, choose the constant and interval combination that describes the solution set.
\[ \frac{5}{9} + \frac{7}{6} x \leq \frac{10}{7} x - \frac{3}{3} \]The solution is \( [5.939, \infty) \), which is option A.\begin{enumerate}[label=\Alph*.]
\item \( [a, \infty), \text{ where } a \in [4.5, 8.25] \)

* $[5.939, \infty)$, which is the correct option.
\item \( (-\infty, a], \text{ where } a \in [3.75, 8.25] \)

 $(-\infty, 5.939]$, which corresponds to switching the direction of the interval. You likely did this if you did not flip the inequality when dividing by a negative!
\item \( [a, \infty), \text{ where } a \in [-9, -5.25] \)

 $[-5.939, \infty)$, which corresponds to negating the endpoint of the solution.
\item \( (-\infty, a], \text{ where } a \in [-8.25, -3.75] \)

 $(-\infty, -5.939]$, which corresponds to switching the direction of the interval AND negating the endpoint. You likely did this if you did not flip the inequality when dividing by a negative as well as not moving values over to a side properly.
\item \( \text{None of the above}. \)

You may have chosen this if you thought the inequality did not match the ends of the intervals.
\end{enumerate}

\textbf{General Comment:} Remember that less/greater than or equal to includes the endpoint, while less/greater do not. Also, remember that you need to flip the inequality when you multiply or divide by a negative.
}
\litem{
Solve the linear inequality below. Then, choose the constant and interval combination that describes the solution set.
\[ \frac{6}{2} + \frac{3}{5} x > \frac{8}{8} x + \frac{5}{4} \]The solution is \( (-\infty, 4.375) \), which is option D.\begin{enumerate}[label=\Alph*.]
\item \( (a, \infty), \text{ where } a \in [3.75, 4.5] \)

 $(4.375, \infty)$, which corresponds to switching the direction of the interval. You likely did this if you did not flip the inequality when dividing by a negative!
\item \( (-\infty, a), \text{ where } a \in [-5.25, -2.25] \)

 $(-\infty, -4.375)$, which corresponds to negating the endpoint of the solution.
\item \( (a, \infty), \text{ where } a \in [-5.25, -1.5] \)

 $(-4.375, \infty)$, which corresponds to switching the direction of the interval AND negating the endpoint. You likely did this if you did not flip the inequality when dividing by a negative as well as not moving values over to a side properly.
\item \( (-\infty, a), \text{ where } a \in [0.75, 6] \)

* $(-\infty, 4.375)$, which is the correct option.
\item \( \text{None of the above}. \)

You may have chosen this if you thought the inequality did not match the ends of the intervals.
\end{enumerate}

\textbf{General Comment:} Remember that less/greater than or equal to includes the endpoint, while less/greater do not. Also, remember that you need to flip the inequality when you multiply or divide by a negative.
}
\litem{
Using an interval or intervals, describe all the $x$-values within or including a distance of the given values.
\[ \text{ Less than } 7 \text{ units from the number } -3. \]The solution is \( (-10, 4) \), which is option A.\begin{enumerate}[label=\Alph*.]
\item \( (-10, 4) \)

This describes the values less than 7 from -3
\item \( [-10, 4] \)

This describes the values no more than 7 from -3
\item \( (-\infty, -10) \cup (4, \infty) \)

This describes the values more than 7 from -3
\item \( (-\infty, -10] \cup [4, \infty) \)

This describes the values no less than 7 from -3
\item \( \text{None of the above} \)

You likely thought the values in the interval were not correct.
\end{enumerate}

\textbf{General Comment:} When thinking about this language, it helps to draw a number line and try points.
}
\end{enumerate}

\end{document}