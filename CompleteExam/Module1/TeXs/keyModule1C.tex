\documentclass{extbook}[14pt]
\usepackage{multicol, enumerate, enumitem, hyperref, color, soul, setspace, parskip, fancyhdr, amssymb, amsthm, amsmath, bbm, latexsym, units, mathtools}
\everymath{\displaystyle}
\usepackage[headsep=0.5cm,headheight=0cm, left=1 in,right= 1 in,top= 1 in,bottom= 1 in]{geometry}
\usepackage{dashrule}  % Package to use the command below to create lines between items
\newcommand{\litem}[1]{\item #1

\rule{\textwidth}{0.4pt}}
\pagestyle{fancy}
\lhead{}
\chead{Answer Key for Module1 Version C}
\rhead{}
\lfoot{4213-4786}
\cfoot{}
\rfoot{test}
\begin{document}
\textbf{This key should allow you to understand why you choose the option you did (beyond just getting a question right or wrong). \href{https://xronos.clas.ufl.edu/mac1105spring2020/courseDescriptionAndMisc/Exams/LearningFromResults}{More instructions on how to use this key can be found here}.}

\textbf{If you have a suggestion to make the keys better, \href{https://forms.gle/CZkbZmPbC9XALEE88}{please fill out the short survey here}.}

\textit{Note: This key is auto-generated and may contain issues and/or errors. The keys are reviewed after each exam to ensure grading is done accurately. If there are issues (like duplicate options), they are noted in the offline gradebook. The keys are a work-in-progress to give students as many resources to improve as possible.}

\rule{\textwidth}{0.4pt}

\begin{enumerate}\litem{
Simplify the expression below and choose the interval the simplification is contained within.
\[ 12 - 7^2 + 9 \div 8 * 4 \div 13 \]

The solution is \( -36.654 \), which is option D.\begin{enumerate}[label=\Alph*.]
\item \( [60.92, 61.3] \)

 61.022, which corresponds to two Order of Operations errors.
\item \( [-37.11, -36.86] \)

 -36.978, which corresponds to an Order of Operations error: not reading left-to-right for multiplication/division.
\item \( [61.1, 61.59] \)

 61.346, which corresponds to an Order of Operations error: multiplying by negative before squaring. For example: $(-3)^2 \neq -3^2$
\item \( [-36.73, -36.27] \)

* -36.654, this is the correct option
\item \( \text{None of the above} \)

 You may have gotten this by making an unanticipated error. If you got a value that is not any of the others, please let the coordinator know so they can help you figure out what happened.
\end{enumerate}

\textbf{General Comment:} While you may remember (or were taught) PEMDAS is done in order, it is actually done as P/E/MD/AS. When we are at MD or AS, we read left to right.
}
\litem{
Simplify the expression below into the form $a+bi$. Then, choose the intervals that $a$ and $b$ belong to.
\[ (5 + 9 i)(10 - 2 i) \]

The solution is \( 68 + 80 i \), which is option E.\begin{enumerate}[label=\Alph*.]
\item \( a \in [29, 34] \text{ and } b \in [-104, -98] \)

 $32 - 100 i$, which corresponds to adding a minus sign in the first term.
\item \( a \in [29, 34] \text{ and } b \in [98, 107] \)

 $32 + 100 i$, which corresponds to adding a minus sign in the second term.
\item \( a \in [66, 69] \text{ and } b \in [-81, -75] \)

 $68 - 80 i$, which corresponds to adding a minus sign in both terms.
\item \( a \in [46, 52] \text{ and } b \in [-25, -13] \)

 $50 - 18 i$, which corresponds to just multiplying the real terms to get the real part of the solution and the coefficients in the complex terms to get the complex part.
\item \( a \in [66, 69] \text{ and } b \in [77, 83] \)

* $68 + 80 i$, which is the correct option.
\end{enumerate}

\textbf{General Comment:} You can treat $i$ as a variable and distribute. Just remember that $i^2=-1$, so you can continue to reduce after you distribute.
}
\litem{
Choose the \textbf{smallest} set of Complex numbers that the number below belongs to.
\[ \sqrt{\frac{324}{529}}+\sqrt{176} i \]

The solution is \( \text{Nonreal Complex} \), which is option C.\begin{enumerate}[label=\Alph*.]
\item \( \text{Pure Imaginary} \)

This is a Complex number $(a+bi)$ that \textbf{only} has an imaginary part like $2i$.
\item \( \text{Not a Complex Number} \)

This is not a number. The only non-Complex number we know is dividing by 0 as this is not a number!
\item \( \text{Nonreal Complex} \)

* This is the correct option!
\item \( \text{Irrational} \)

These cannot be written as a fraction of Integers. Remember: $\pi$ is not an Integer!
\item \( \text{Rational} \)

These are numbers that can be written as fraction of Integers (e.g., -2/3 + 5)
\end{enumerate}

\textbf{General Comment:} Be sure to simplify $i^2 = -1$. This may remove the imaginary portion for your number. If you are having trouble, you may want to look at the \textit{Subgroups of the Real Numbers} section.
}
\litem{
Choose the \textbf{smallest} set of Real numbers that the number below belongs to.
\[ \sqrt{\frac{576}{289}} \]

The solution is \( \text{Rational} \), which is option C.\begin{enumerate}[label=\Alph*.]
\item \( \text{Integer} \)

These are the negative and positive counting numbers (..., -3, -2, -1, 0, 1, 2, 3, ...)
\item \( \text{Irrational} \)

These cannot be written as a fraction of Integers.
\item \( \text{Rational} \)

* This is the correct option!
\item \( \text{Whole} \)

These are the counting numbers with 0 (0, 1, 2, 3, ...)
\item \( \text{Not a Real number} \)

These are Nonreal Complex numbers \textbf{OR} things that are not numbers (e.g., dividing by 0).
\end{enumerate}

\textbf{General Comment:} First, you \textbf{NEED} to simplify the expression. This question simplifies to $\frac{24}{17}$. 
 
 Be sure you look at the simplified fraction and not just the decimal expansion. Numbers such as 13, 17, and 19 provide \textbf{long but repeating/terminating decimal expansions!} 
 
 The only ways to *not* be a Real number are: dividing by 0 or taking the square root of a negative number. 
 
 Irrational numbers are more than just square root of 3: adding or subtracting values from square root of 3 is also irrational.
}
\litem{
Choose the \textbf{smallest} set of Complex numbers that the number below belongs to.
\[ \sqrt{\frac{0}{64}}+\sqrt{2}i \]

The solution is \( \text{Pure Imaginary} \), which is option C.\begin{enumerate}[label=\Alph*.]
\item \( \text{Irrational} \)

These cannot be written as a fraction of Integers. Remember: $\pi$ is not an Integer!
\item \( \text{Nonreal Complex} \)

This is a Complex number $(a+bi)$ that is not Real (has $i$ as part of the number).
\item \( \text{Pure Imaginary} \)

* This is the correct option!
\item \( \text{Rational} \)

These are numbers that can be written as fraction of Integers (e.g., -2/3 + 5)
\item \( \text{Not a Complex Number} \)

This is not a number. The only non-Complex number we know is dividing by 0 as this is not a number!
\end{enumerate}

\textbf{General Comment:} Be sure to simplify $i^2 = -1$. This may remove the imaginary portion for your number. If you are having trouble, you may want to look at the \textit{Subgroups of the Real Numbers} section.
}
\litem{
Simplify the expression below and choose the interval the simplification is contained within.
\[ 6 - 10^2 + 16 \div 14 * 11 \div 19 \]

The solution is \( -93.338 \), which is option D.\begin{enumerate}[label=\Alph*.]
\item \( [-94.12, -93.52] \)

 -93.995, which corresponds to an Order of Operations error: not reading left-to-right for multiplication/division.
\item \( [105.71, 106.28] \)

 106.005, which corresponds to two Order of Operations errors.
\item \( [106.56, 106.71] \)

 106.662, which corresponds to an Order of Operations error: multiplying by negative before squaring. For example: $(-3)^2 \neq -3^2$
\item \( [-93.86, -93.09] \)

* -93.338, this is the correct option
\item \( \text{None of the above} \)

 You may have gotten this by making an unanticipated error. If you got a value that is not any of the others, please let the coordinator know so they can help you figure out what happened.
\end{enumerate}

\textbf{General Comment:} While you may remember (or were taught) PEMDAS is done in order, it is actually done as P/E/MD/AS. When we are at MD or AS, we read left to right.
}
\litem{
Simplify the expression below into the form $a+bi$. Then, choose the intervals that $a$ and $b$ belong to.
\[ (4 - 9 i)(7 + 2 i) \]

The solution is \( 46 - 55 i \), which is option C.\begin{enumerate}[label=\Alph*.]
\item \( a \in [4, 11] \text{ and } b \in [65, 75] \)

 $10 + 71 i$, which corresponds to adding a minus sign in the first term.
\item \( a \in [27, 29] \text{ and } b \in [-19, -14] \)

 $28 - 18 i$, which corresponds to just multiplying the real terms to get the real part of the solution and the coefficients in the complex terms to get the complex part.
\item \( a \in [43, 48] \text{ and } b \in [-56, -52] \)

* $46 - 55 i$, which is the correct option.
\item \( a \in [43, 48] \text{ and } b \in [53, 56] \)

 $46 + 55 i$, which corresponds to adding a minus sign in both terms.
\item \( a \in [4, 11] \text{ and } b \in [-78, -66] \)

 $10 - 71 i$, which corresponds to adding a minus sign in the second term.
\end{enumerate}

\textbf{General Comment:} You can treat $i$ as a variable and distribute. Just remember that $i^2=-1$, so you can continue to reduce after you distribute.
}
\litem{
Choose the \textbf{smallest} set of Real numbers that the number below belongs to.
\[ -\sqrt{\frac{104976}{324}} \]

The solution is \( \text{Integer} \), which is option D.\begin{enumerate}[label=\Alph*.]
\item \( \text{Rational} \)

These are numbers that can be written as fraction of Integers (e.g., -2/3)
\item \( \text{Whole} \)

These are the counting numbers with 0 (0, 1, 2, 3, ...)
\item \( \text{Irrational} \)

These cannot be written as a fraction of Integers.
\item \( \text{Integer} \)

* This is the correct option!
\item \( \text{Not a Real number} \)

These are Nonreal Complex numbers \textbf{OR} things that are not numbers (e.g., dividing by 0).
\end{enumerate}

\textbf{General Comment:} First, you \textbf{NEED} to simplify the expression. This question simplifies to $-324$. 
 
 Be sure you look at the simplified fraction and not just the decimal expansion. Numbers such as 13, 17, and 19 provide \textbf{long but repeating/terminating decimal expansions!} 
 
 The only ways to *not* be a Real number are: dividing by 0 or taking the square root of a negative number. 
 
 Irrational numbers are more than just square root of 3: adding or subtracting values from square root of 3 is also irrational.
}
\litem{
Simplify the expression below into the form $a+bi$. Then, choose the intervals that $a$ and $b$ belong to.
\[ \frac{45 - 11 i}{-8 + 7 i} \]

The solution is \( -3.87  - 2.01 i \), which is option B.\begin{enumerate}[label=\Alph*.]
\item \( a \in [-438.5, -435.5] \text{ and } b \in [-2.06, -1.77] \)

 $-437.00  - 2.01 i$, which corresponds to forgetting to multiply the conjugate by the numerator and using a plus instead of a minus in the denominator.
\item \( a \in [-5, -3.5] \text{ and } b \in [-2.06, -1.77] \)

* $-3.87  - 2.01 i$, which is the correct option.
\item \( a \in [-3, -1.5] \text{ and } b \in [3.27, 3.7] \)

 $-2.50  + 3.57 i$, which corresponds to forgetting to multiply the conjugate by the numerator and not computing the conjugate correctly.
\item \( a \in [-6, -5.5] \text{ and } b \in [-1.78, -1.54] \)

 $-5.62  - 1.57 i$, which corresponds to just dividing the first term by the first term and the second by the second.
\item \( a \in [-5, -3.5] \text{ and } b \in [-227.19, -226.88] \)

 $-3.87  - 227.00 i$, which corresponds to forgetting to multiply the conjugate by the numerator.
\end{enumerate}

\textbf{General Comment:} Multiply the numerator and denominator by the *conjugate* of the denominator, then simplify. For example, if we have $2+3i$, the conjugate is $2-3i$.
}
\litem{
Simplify the expression below into the form $a+bi$. Then, choose the intervals that $a$ and $b$ belong to.
\[ \frac{36 + 33 i}{6 - 7 i} \]

The solution is \( -0.18  + 5.29 i \), which is option B.\begin{enumerate}[label=\Alph*.]
\item \( a \in [-15.5, -14.5] \text{ and } b \in [4.5, 6] \)

 $-15.00  + 5.29 i$, which corresponds to forgetting to multiply the conjugate by the numerator and using a plus instead of a minus in the denominator.
\item \( a \in [-0.5, 1] \text{ and } b \in [4.5, 6] \)

* $-0.18  + 5.29 i$, which is the correct option.
\item \( a \in [-0.5, 1] \text{ and } b \in [449.5, 451] \)

 $-0.18  + 450.00 i$, which corresponds to forgetting to multiply the conjugate by the numerator.
\item \( a \in [5.5, 7] \text{ and } b \in [-5.5, -4] \)

 $6.00  - 4.71 i$, which corresponds to just dividing the first term by the first term and the second by the second.
\item \( a \in [4, 5.5] \text{ and } b \in [-2, 1] \)

 $5.26  - 0.64 i$, which corresponds to forgetting to multiply the conjugate by the numerator and not computing the conjugate correctly.
\end{enumerate}

\textbf{General Comment:} Multiply the numerator and denominator by the *conjugate* of the denominator, then simplify. For example, if we have $2+3i$, the conjugate is $2-3i$.
}
\end{enumerate}

\end{document}