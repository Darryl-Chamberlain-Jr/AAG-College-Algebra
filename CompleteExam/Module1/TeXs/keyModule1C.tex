\documentclass{extbook}[14pt]
\usepackage{multicol, enumerate, enumitem, hyperref, color, soul, setspace, parskip, fancyhdr, amssymb, amsthm, amsmath, latexsym, units, mathtools}
\everymath{\displaystyle}
\usepackage[headsep=0.5cm,headheight=0cm, left=1 in,right= 1 in,top= 1 in,bottom= 1 in]{geometry}
\usepackage{dashrule}  % Package to use the command below to create lines between items
\newcommand{\litem}[1]{\item #1

\rule{\textwidth}{0.4pt}}
\pagestyle{fancy}
\lhead{}
\chead{Answer Key for Module1 Version C}
\rhead{}
\lfoot{8000-1344}
\cfoot{}
\rfoot{testing}
\begin{document}
\textbf{This key should allow you to understand why you choose the option you did (beyond just getting a question right or wrong). \href{https://xronos.clas.ufl.edu/mac1105spring2020/courseDescriptionAndMisc/Exams/LearningFromResults}{More instructions on how to use this key can be found here}.}

\textbf{If you have a suggestion to make the keys better, \href{https://forms.gle/CZkbZmPbC9XALEE88}{please fill out the short survey here}.}

\textit{Note: This key is auto-generated and may contain issues and/or errors. The keys are reviewed after each exam to ensure grading is done accurately. If there are issues (like duplicate options), they are noted in the offline gradebook. The keys are a work-in-progress to give students as many resources to improve as possible.}

\rule{\textwidth}{0.4pt}

\begin{enumerate}\litem{
Choose the \textbf{smallest} set of Real numbers that the number below belongs to.
\[ \sqrt{\frac{78400}{400}} \]The solution is \( \text{Whole} \), which is option E.\begin{enumerate}[label=\Alph*.]
\item \( \text{Rational} \)

These are numbers that can be written as fraction of Integers (e.g., -2/3)
\item \( \text{Integer} \)

These are the negative and positive counting numbers (..., -3, -2, -1, 0, 1, 2, 3, ...)
\item \( \text{Irrational} \)

These cannot be written as a fraction of Integers.
\item \( \text{Not a Real number} \)

These are Nonreal Complex numbers \textbf{OR} things that are not numbers (e.g., dividing by 0).
\item \( \text{Whole} \)

* This is the correct option!
\end{enumerate}

\textbf{General Comment:} First, you \textbf{NEED} to simplify the expression. This question simplifies to $280$. 
 
 Be sure you look at the simplified fraction and not just the decimal expansion. Numbers such as 13, 17, and 19 provide \textbf{long but repeating/terminating decimal expansions!} 
 
 The only ways to *not* be a Real number are: dividing by 0 or taking the square root of a negative number. 
 
 Irrational numbers are more than just square root of 3: adding or subtracting values from square root of 3 is also irrational.
}
\litem{
Simplify the expression below into the form $a+bi$. Then, choose the intervals that $a$ and $b$ belong to.
\[ (7 - 2 i)(-10 - 3 i) \]The solution is \( -76 - i \), which is option E.\begin{enumerate}[label=\Alph*.]
\item \( a \in [-74, -69] \text{ and } b \in [5.7, 6.6] \)

 $-70 + 6 i$, which corresponds to just multiplying the real terms to get the real part of the solution and the coefficients in the complex terms to get the complex part.
\item \( a \in [-80, -75] \text{ and } b \in [0.8, 1.2] \)

 $-76 + i$, which corresponds to adding a minus sign in both terms.
\item \( a \in [-67, -60] \text{ and } b \in [39.9, 41.6] \)

 $-64 + 41 i$, which corresponds to adding a minus sign in the second term.
\item \( a \in [-67, -60] \text{ and } b \in [-41.8, -39.7] \)

 $-64 - 41 i$, which corresponds to adding a minus sign in the first term.
\item \( a \in [-80, -75] \text{ and } b \in [-1.1, 0.5] \)

* $-76 - i$, which is the correct option.
\end{enumerate}

\textbf{General Comment:} You can treat $i$ as a variable and distribute. Just remember that $i^2=-1$, so you can continue to reduce after you distribute.
}
\litem{
Simplify the expression below into the form $a+bi$. Then, choose the intervals that $a$ and $b$ belong to.
\[ \frac{-54 + 11 i}{-7 - 3 i} \]The solution is \( 5.95  - 4.12 i \), which is option C.\begin{enumerate}[label=\Alph*.]
\item \( a \in [344.6, 345.3] \text{ and } b \in [-4.27, -4.02] \)

 $345.00  - 4.12 i$, which corresponds to forgetting to multiply the conjugate by the numerator and using a plus instead of a minus in the denominator.
\item \( a \in [7.55, 8.5] \text{ and } b \in [-3.81, -3.6] \)

 $7.71  - 3.67 i$, which corresponds to just dividing the first term by the first term and the second by the second.
\item \( a \in [5.45, 6.45] \text{ and } b \in [-4.27, -4.02] \)

* $5.95  - 4.12 i$, which is the correct option.
\item \( a \in [5.45, 6.45] \text{ and } b \in [-239.1, -238.97] \)

 $5.95  - 239.00 i$, which corresponds to forgetting to multiply the conjugate by the numerator.
\item \( a \in [6.65, 7.15] \text{ and } b \in [1.43, 1.6] \)

 $7.09  + 1.47 i$, which corresponds to forgetting to multiply the conjugate by the numerator and not computing the conjugate correctly.
\end{enumerate}

\textbf{General Comment:} Multiply the numerator and denominator by the *conjugate* of the denominator, then simplify. For example, if we have $2+3i$, the conjugate is $2-3i$.
}
\litem{
Simplify the expression below and choose the interval the simplification is contained within.
\[ 12 - 3 \div 8 * 20 - (4 * 13) \]The solution is \( -47.500 \), which is option A.\begin{enumerate}[label=\Alph*.]
\item \( [-48.5, -45.5] \)

* -47.500, which is the correct option.
\item \( [3.5, 8.5] \)

 6.500, which corresponds to not distributing a negative correctly.
\item \( [-41.02, -35.02] \)

 -40.019, which corresponds to an Order of Operations error: not reading left-to-right for multiplication/division.
\item \( [60.98, 64.98] \)

 63.981, which corresponds to not distributing addition and subtraction correctly.
\item \( \text{None of the above} \)

 You may have gotten this by making an unanticipated error. If you got a value that is not any of the others, please let the coordinator know so they can help you figure out what happened.
\end{enumerate}

\textbf{General Comment:} While you may remember (or were taught) PEMDAS is done in order, it is actually done as P/E/MD/AS. When we are at MD or AS, we read left to right.
}
\litem{
Choose the \textbf{smallest} set of Complex numbers that the number below belongs to.
\[ -\sqrt{\frac{400}{289}} + 25i^2 \]The solution is \( \text{Rational} \), which is option E.\begin{enumerate}[label=\Alph*.]
\item \( \text{Irrational} \)

These cannot be written as a fraction of Integers. Remember: $\pi$ is not an Integer!
\item \( \text{Nonreal Complex} \)

This is a Complex number $(a+bi)$ that is not Real (has $i$ as part of the number).
\item \( \text{Not a Complex Number} \)

This is not a number. The only non-Complex number we know is dividing by 0 as this is not a number!
\item \( \text{Pure Imaginary} \)

This is a Complex number $(a+bi)$ that \textbf{only} has an imaginary part like $2i$.
\item \( \text{Rational} \)

* This is the correct option!
\end{enumerate}

\textbf{General Comment:} Be sure to simplify $i^2 = -1$. This may remove the imaginary portion for your number. If you are having trouble, you may want to look at the \textit{Subgroups of the Real Numbers} section.
}
\litem{
Simplify the expression below into the form $a+bi$. Then, choose the intervals that $a$ and $b$ belong to.
\[ \frac{18 - 77 i}{-5 - i} \]The solution is \( -0.50  + 15.50 i \), which is option E.\begin{enumerate}[label=\Alph*.]
\item \( a \in [-6.5, -5.5] \text{ and } b \in [13.5, 15] \)

 $-6.42  + 14.12 i$, which corresponds to forgetting to multiply the conjugate by the numerator and not computing the conjugate correctly.
\item \( a \in [-5.5, -2] \text{ and } b \in [76, 78.5] \)

 $-3.60  + 77.00 i$, which corresponds to just dividing the first term by the first term and the second by the second.
\item \( a \in [-1.5, 0.5] \text{ and } b \in [402, 403.5] \)

 $-0.50  + 403.00 i$, which corresponds to forgetting to multiply the conjugate by the numerator.
\item \( a \in [-13.5, -12.5] \text{ and } b \in [15, 17] \)

 $-13.00  + 15.50 i$, which corresponds to forgetting to multiply the conjugate by the numerator and using a plus instead of a minus in the denominator.
\item \( a \in [-1.5, 0.5] \text{ and } b \in [15, 17] \)

* $-0.50  + 15.50 i$, which is the correct option.
\end{enumerate}

\textbf{General Comment:} Multiply the numerator and denominator by the *conjugate* of the denominator, then simplify. For example, if we have $2+3i$, the conjugate is $2-3i$.
}
\litem{
Choose the \textbf{smallest} set of Real numbers that the number below belongs to.
\[ \sqrt{\frac{32400}{81}} \]The solution is \( \text{Whole} \), which is option A.\begin{enumerate}[label=\Alph*.]
\item \( \text{Whole} \)

* This is the correct option!
\item \( \text{Rational} \)

These are numbers that can be written as fraction of Integers (e.g., -2/3)
\item \( \text{Integer} \)

These are the negative and positive counting numbers (..., -3, -2, -1, 0, 1, 2, 3, ...)
\item \( \text{Irrational} \)

These cannot be written as a fraction of Integers.
\item \( \text{Not a Real number} \)

These are Nonreal Complex numbers \textbf{OR} things that are not numbers (e.g., dividing by 0).
\end{enumerate}

\textbf{General Comment:} First, you \textbf{NEED} to simplify the expression. This question simplifies to $180$. 
 
 Be sure you look at the simplified fraction and not just the decimal expansion. Numbers such as 13, 17, and 19 provide \textbf{long but repeating/terminating decimal expansions!} 
 
 The only ways to *not* be a Real number are: dividing by 0 or taking the square root of a negative number. 
 
 Irrational numbers are more than just square root of 3: adding or subtracting values from square root of 3 is also irrational.
}
\litem{
Simplify the expression below into the form $a+bi$. Then, choose the intervals that $a$ and $b$ belong to.
\[ (-4 - 5 i)(-7 - 6 i) \]The solution is \( -2 + 59 i \), which is option A.\begin{enumerate}[label=\Alph*.]
\item \( a \in [-2, -1] \text{ and } b \in [57, 62] \)

* $-2 + 59 i$, which is the correct option.
\item \( a \in [-2, -1] \text{ and } b \in [-60, -57] \)

 $-2 - 59 i$, which corresponds to adding a minus sign in both terms.
\item \( a \in [57, 63] \text{ and } b \in [-12, -1] \)

 $58 - 11 i$, which corresponds to adding a minus sign in the first term.
\item \( a \in [57, 63] \text{ and } b \in [8, 16] \)

 $58 + 11 i$, which corresponds to adding a minus sign in the second term.
\item \( a \in [25, 34] \text{ and } b \in [29, 35] \)

 $28 + 30 i$, which corresponds to just multiplying the real terms to get the real part of the solution and the coefficients in the complex terms to get the complex part.
\end{enumerate}

\textbf{General Comment:} You can treat $i$ as a variable and distribute. Just remember that $i^2=-1$, so you can continue to reduce after you distribute.
}
\litem{
Choose the \textbf{smallest} set of Complex numbers that the number below belongs to.
\[ \sqrt{\frac{121}{0}}+\sqrt{117} i \]The solution is \( \text{Not a Complex Number} \), which is option E.\begin{enumerate}[label=\Alph*.]
\item \( \text{Pure Imaginary} \)

This is a Complex number $(a+bi)$ that \textbf{only} has an imaginary part like $2i$.
\item \( \text{Nonreal Complex} \)

This is a Complex number $(a+bi)$ that is not Real (has $i$ as part of the number).
\item \( \text{Rational} \)

These are numbers that can be written as fraction of Integers (e.g., -2/3 + 5)
\item \( \text{Irrational} \)

These cannot be written as a fraction of Integers. Remember: $\pi$ is not an Integer!
\item \( \text{Not a Complex Number} \)

* This is the correct option!
\end{enumerate}

\textbf{General Comment:} Be sure to simplify $i^2 = -1$. This may remove the imaginary portion for your number. If you are having trouble, you may want to look at the \textit{Subgroups of the Real Numbers} section.
}
\litem{
Simplify the expression below and choose the interval the simplification is contained within.
\[ 4 - 8^2 + 14 \div 13 * 19 \div 6 \]The solution is \( -56.590 \), which is option D.\begin{enumerate}[label=\Alph*.]
\item \( [71.41, 76.41] \)

 71.410, which corresponds to an Order of Operations error: multiplying by negative before squaring. For example: $(-3)^2 \neq -3^2$
\item \( [-61.99, -56.99] \)

 -59.991, which corresponds to an Order of Operations error: not reading left-to-right for multiplication/division.
\item \( [64.01, 70.01] \)

 68.009, which corresponds to two Order of Operations errors.
\item \( [-58.59, -53.59] \)

* -56.590, this is the correct option
\item \( \text{None of the above} \)

 You may have gotten this by making an unanticipated error. If you got a value that is not any of the others, please let the coordinator know so they can help you figure out what happened.
\end{enumerate}

\textbf{General Comment:} While you may remember (or were taught) PEMDAS is done in order, it is actually done as P/E/MD/AS. When we are at MD or AS, we read left to right.
}
\end{enumerate}

\end{document}