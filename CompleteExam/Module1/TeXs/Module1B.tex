\documentclass[14pt]{extbook}
\usepackage{multicol, enumerate, enumitem, hyperref, color, soul, setspace, parskip, fancyhdr} %General Packages
\usepackage{amssymb, amsthm, amsmath, latexsym, units, mathtools} %Math Packages
\everymath{\displaystyle} %All math in Display Style
% Packages with additional options
\usepackage[headsep=0.5cm,headheight=12pt, left=1 in,right= 1 in,top= 1 in,bottom= 1 in]{geometry}
\usepackage[usenames,dvipsnames]{xcolor}
\usepackage{dashrule}  % Package to use the command below to create lines between items
\newcommand{\litem}[1]{\item#1\hspace*{-1cm}\rule{\textwidth}{0.4pt}}
\pagestyle{fancy}
\lhead{Module1}
\chead{}
\rhead{Version B}
\lfoot{8000-1344}
\cfoot{}
\rfoot{testing}
\begin{document}

\begin{enumerate}
\litem{
Choose the \textbf{smallest} set of Real numbers that the number below belongs to.\[ \sqrt{\frac{3969}{81}} \]\begin{enumerate}[label=\Alph*.]
\item \( \text{Irrational} \)
\item \( \text{Integer} \)
\item \( \text{Whole} \)
\item \( \text{Rational} \)
\item \( \text{Not a Real number} \)

\end{enumerate} }
\litem{
Simplify the expression below into the form $a+bi$. Then, choose the intervals that $a$ and $b$ belong to.\[ (2 - 7 i)(5 - 9 i) \]\begin{enumerate}[label=\Alph*.]
\item \( a \in [-56, -48] \text{ and } b \in [-57, -52] \)
\item \( a \in [4, 17] \text{ and } b \in [58, 66] \)
\item \( a \in [69, 75] \text{ and } b \in [-20, -12] \)
\item \( a \in [-56, -48] \text{ and } b \in [49, 57] \)
\item \( a \in [69, 75] \text{ and } b \in [16, 18] \)

\end{enumerate} }
\litem{
Simplify the expression below into the form $a+bi$. Then, choose the intervals that $a$ and $b$ belong to.\[ \frac{54 + 44 i}{-7 - 5 i} \]\begin{enumerate}[label=\Alph*.]
\item \( a \in [-8.2, -7.75] \text{ and } b \in [-38.5, -37] \)
\item \( a \in [-598.35, -597.5] \text{ and } b \in [-1.5, 0] \)
\item \( a \in [-8.2, -7.75] \text{ and } b \in [-1.5, 0] \)
\item \( a \in [-2.55, -1.6] \text{ and } b \in [-8.5, -7] \)
\item \( a \in [-7.8, -7.65] \text{ and } b \in [-10, -8.5] \)

\end{enumerate} }
\litem{
Simplify the expression below and choose the interval the simplification is contained within.\[ 17 - 6^2 + 11 \div 10 * 16 \div 3 \]\begin{enumerate}[label=\Alph*.]
\item \( [-19.98, -17.98] \)
\item \( [-17.13, -12.13] \)
\item \( [53.02, 56.02] \)
\item \( [55.87, 60.87] \)
\item \( \text{None of the above} \)

\end{enumerate} }
\litem{
Choose the \textbf{smallest} set of Complex numbers that the number below belongs to.\[ \frac{5}{-9}+36i^2 \]\begin{enumerate}[label=\Alph*.]
\item \( \text{Rational} \)
\item \( \text{Not a Complex Number} \)
\item \( \text{Pure Imaginary} \)
\item \( \text{Nonreal Complex} \)
\item \( \text{Irrational} \)

\end{enumerate} }
\litem{
Simplify the expression below into the form $a+bi$. Then, choose the intervals that $a$ and $b$ belong to.\[ \frac{9 - 44 i}{2 + 7 i} \]\begin{enumerate}[label=\Alph*.]
\item \( a \in [-5.5, -4.5] \text{ and } b \in [-3.5, -1] \)
\item \( a \in [-290.5, -289.5] \text{ and } b \in [-3.5, -1] \)
\item \( a \in [5.5, 7] \text{ and } b \in [-1, 1] \)
\item \( a \in [3.5, 5.5] \text{ and } b \in [-7.5, -6] \)
\item \( a \in [-5.5, -4.5] \text{ and } b \in [-151.5, -149.5] \)

\end{enumerate} }
\litem{
Choose the \textbf{smallest} set of Real numbers that the number below belongs to.\[ -\sqrt{\frac{1176}{14}} \]\begin{enumerate}[label=\Alph*.]
\item \( \text{Not a Real number} \)
\item \( \text{Whole} \)
\item \( \text{Rational} \)
\item \( \text{Integer} \)
\item \( \text{Irrational} \)

\end{enumerate} }
\litem{
Simplify the expression below into the form $a+bi$. Then, choose the intervals that $a$ and $b$ belong to.\[ (-7 + 5 i)(-4 + 2 i) \]\begin{enumerate}[label=\Alph*.]
\item \( a \in [16, 20] \text{ and } b \in [-37, -30] \)
\item \( a \in [37, 42] \text{ and } b \in [6, 8] \)
\item \( a \in [37, 42] \text{ and } b \in [-6, 1] \)
\item \( a \in [16, 20] \text{ and } b \in [29, 37] \)
\item \( a \in [26, 33] \text{ and } b \in [8, 16] \)

\end{enumerate} }
\litem{
Choose the \textbf{smallest} set of Complex numbers that the number below belongs to.\[ \sqrt{\frac{1078}{0}}+\sqrt{182} i \]\begin{enumerate}[label=\Alph*.]
\item \( \text{Not a Complex Number} \)
\item \( \text{Nonreal Complex} \)
\item \( \text{Rational} \)
\item \( \text{Pure Imaginary} \)
\item \( \text{Irrational} \)

\end{enumerate} }
\litem{
Simplify the expression below and choose the interval the simplification is contained within.\[ 18 - 1 \div 8 * 20 - (15 * 6) \]\begin{enumerate}[label=\Alph*.]
\item \( [3, 8] \)
\item \( [-72.01, -64.01] \)
\item \( [-76.5, -72.5] \)
\item \( [105.99, 111.99] \)
\item \( \text{None of the above} \)

\end{enumerate} }
\end{enumerate}

\end{document}