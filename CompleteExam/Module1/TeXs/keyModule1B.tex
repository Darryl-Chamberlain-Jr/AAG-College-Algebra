\documentclass{extbook}[14pt]
\usepackage{multicol, enumerate, enumitem, hyperref, color, soul, setspace, parskip, fancyhdr, amssymb, amsthm, amsmath, latexsym, units, mathtools}
\everymath{\displaystyle}
\usepackage[headsep=0.5cm,headheight=0cm, left=1 in,right= 1 in,top= 1 in,bottom= 1 in]{geometry}
\usepackage{dashrule}  % Package to use the command below to create lines between items
\newcommand{\litem}[1]{\item #1

\rule{\textwidth}{0.4pt}}
\pagestyle{fancy}
\lhead{}
\chead{Answer Key for Module1 Version B}
\rhead{}
\lfoot{4877-7341}
\cfoot{}
\rfoot{test}
\begin{document}
\textbf{This key should allow you to understand why you choose the option you did (beyond just getting a question right or wrong). \href{https://xronos.clas.ufl.edu/mac1105spring2020/courseDescriptionAndMisc/Exams/LearningFromResults}{More instructions on how to use this key can be found here}.}

\textbf{If you have a suggestion to make the keys better, \href{https://forms.gle/CZkbZmPbC9XALEE88}{please fill out the short survey here}.}

\textit{Note: This key is auto-generated and may contain issues and/or errors. The keys are reviewed after each exam to ensure grading is done accurately. If there are issues (like duplicate options), they are noted in the offline gradebook. The keys are a work-in-progress to give students as many resources to improve as possible.}

\rule{\textwidth}{0.4pt}

\begin{enumerate}\litem{
Choose the \textbf{smallest} set of Complex numbers that the number below belongs to.
\[ \sqrt{\frac{-450}{0}}+\sqrt{130} \]The solution is \( \text{Not a Complex Number} \), which is option D.\begin{enumerate}[label=\Alph*.]
\item \( \text{Rational} \)

These are numbers that can be written as fraction of Integers (e.g., -2/3 + 5)
\item \( \text{Irrational} \)

These cannot be written as a fraction of Integers. Remember: $\pi$ is not an Integer!
\item \( \text{Pure Imaginary} \)

This is a Complex number $(a+bi)$ that \textbf{only} has an imaginary part like $2i$.
\item \( \text{Not a Complex Number} \)

* This is the correct option!
\item \( \text{Nonreal Complex} \)

This is a Complex number $(a+bi)$ that is not Real (has $i$ as part of the number).
\end{enumerate}

\textbf{General Comment:} Be sure to simplify $i^2 = -1$. This may remove the imaginary portion for your number. If you are having trouble, you may want to look at the \textit{Subgroups of the Real Numbers} section.
}
\litem{
Simplify the expression below and choose the interval the simplification is contained within.
\[ 7 - 2 \div 20 * 6 - (18 * 14) \]The solution is \( -245.600 \), which is option D.\begin{enumerate}[label=\Alph*.]
\item \( [-245.31, -244.11] \)

 -245.017, which corresponds to an Order of Operations error: not reading left-to-right for multiplication/division.
\item \( [-162.58, -161.38] \)

 -162.400, which corresponds to not distributing a negative correctly.
\item \( [258.34, 259.56] \)

 258.983, which corresponds to not distributing addition and subtraction correctly.
\item \( [-245.62, -245.55] \)

* -245.600, which is the correct option.
\item \( \text{None of the above} \)

 You may have gotten this by making an unanticipated error. If you got a value that is not any of the others, please let the coordinator know so they can help you figure out what happened.
\end{enumerate}

\textbf{General Comment:} While you may remember (or were taught) PEMDAS is done in order, it is actually done as P/E/MD/AS. When we are at MD or AS, we read left to right.
}
\litem{
Choose the \textbf{smallest} set of Complex numbers that the number below belongs to.
\[ \frac{18}{5}+\sqrt{65} i \]The solution is \( \text{Nonreal Complex} \), which is option A.\begin{enumerate}[label=\Alph*.]
\item \( \text{Nonreal Complex} \)

* This is the correct option!
\item \( \text{Not a Complex Number} \)

This is not a number. The only non-Complex number we know is dividing by 0 as this is not a number!
\item \( \text{Rational} \)

These are numbers that can be written as fraction of Integers (e.g., -2/3 + 5)
\item \( \text{Irrational} \)

These cannot be written as a fraction of Integers. Remember: $\pi$ is not an Integer!
\item \( \text{Pure Imaginary} \)

This is a Complex number $(a+bi)$ that \textbf{only} has an imaginary part like $2i$.
\end{enumerate}

\textbf{General Comment:} Be sure to simplify $i^2 = -1$. This may remove the imaginary portion for your number. If you are having trouble, you may want to look at the \textit{Subgroups of the Real Numbers} section.
}
\litem{
Simplify the expression below into the form $a+bi$. Then, choose the intervals that $a$ and $b$ belong to.
\[ \frac{9 + 55 i}{-8 + 6 i} \]The solution is \( 2.58  - 4.94 i \), which is option C.\begin{enumerate}[label=\Alph*.]
\item \( a \in [-2, 0] \text{ and } b \in [8.95, 9.8] \)

 $-1.12  + 9.17 i$, which corresponds to just dividing the first term by the first term and the second by the second.
\item \( a \in [-6, -4] \text{ and } b \in [-4.25, -3.4] \)

 $-4.02  - 3.86 i$, which corresponds to forgetting to multiply the conjugate by the numerator and not computing the conjugate correctly.
\item \( a \in [2, 4] \text{ and } b \in [-5.3, -4.35] \)

* $2.58  - 4.94 i$, which is the correct option.
\item \( a \in [257.5, 259] \text{ and } b \in [-5.3, -4.35] \)

 $258.00  - 4.94 i$, which corresponds to forgetting to multiply the conjugate by the numerator and using a plus instead of a minus in the denominator.
\item \( a \in [2, 4] \text{ and } b \in [-494.15, -493.75] \)

 $2.58  - 494.00 i$, which corresponds to forgetting to multiply the conjugate by the numerator.
\end{enumerate}

\textbf{General Comment:} Multiply the numerator and denominator by the *conjugate* of the denominator, then simplify. For example, if we have $2+3i$, the conjugate is $2-3i$.
}
\litem{
Simplify the expression below into the form $a+bi$. Then, choose the intervals that $a$ and $b$ belong to.
\[ \frac{54 - 77 i}{-1 + 4 i} \]The solution is \( -21.29  - 8.18 i \), which is option A.\begin{enumerate}[label=\Alph*.]
\item \( a \in [-22, -21] \text{ and } b \in [-9, -7.5] \)

* $-21.29  - 8.18 i$, which is the correct option.
\item \( a \in [-55, -53] \text{ and } b \in [-21, -18.5] \)

 $-54.00  - 19.25 i$, which corresponds to just dividing the first term by the first term and the second by the second.
\item \( a \in [14.5, 15.5] \text{ and } b \in [16, 18] \)

 $14.94  + 17.24 i$, which corresponds to forgetting to multiply the conjugate by the numerator and not computing the conjugate correctly.
\item \( a \in [-22, -21] \text{ and } b \in [-140.5, -138] \)

 $-21.29  - 139.00 i$, which corresponds to forgetting to multiply the conjugate by the numerator.
\item \( a \in [-363, -361.5] \text{ and } b \in [-9, -7.5] \)

 $-362.00  - 8.18 i$, which corresponds to forgetting to multiply the conjugate by the numerator and using a plus instead of a minus in the denominator.
\end{enumerate}

\textbf{General Comment:} Multiply the numerator and denominator by the *conjugate* of the denominator, then simplify. For example, if we have $2+3i$, the conjugate is $2-3i$.
}
\litem{
Simplify the expression below and choose the interval the simplification is contained within.
\[ 11 - 5^2 + 4 \div 16 * 14 \div 8 \]The solution is \( -13.562 \), which is option D.\begin{enumerate}[label=\Alph*.]
\item \( [35.89, 36.11] \)

 36.002, which corresponds to two Order of Operations errors.
\item \( [-14.25, -13.84] \)

 -13.998, which corresponds to an Order of Operations error: not reading left-to-right for multiplication/division.
\item \( [36.09, 36.83] \)

 36.438, which corresponds to an Order of Operations error: multiplying by negative before squaring. For example: $(-3)^2 \neq -3^2$
\item \( [-13.66, -13.37] \)

* -13.562, this is the correct option
\item \( \text{None of the above} \)

 You may have gotten this by making an unanticipated error. If you got a value that is not any of the others, please let the coordinator know so they can help you figure out what happened.
\end{enumerate}

\textbf{General Comment:} While you may remember (or were taught) PEMDAS is done in order, it is actually done as P/E/MD/AS. When we are at MD or AS, we read left to right.
}
\litem{
Simplify the expression below into the form $a+bi$. Then, choose the intervals that $a$ and $b$ belong to.
\[ (7 - 9 i)(-10 - 8 i) \]The solution is \( -142 + 34 i \), which is option C.\begin{enumerate}[label=\Alph*.]
\item \( a \in [-144, -140] \text{ and } b \in [-34, -33] \)

 $-142 - 34 i$, which corresponds to adding a minus sign in both terms.
\item \( a \in [-75, -65] \text{ and } b \in [72, 76] \)

 $-70 + 72 i$, which corresponds to just multiplying the real terms to get the real part of the solution and the coefficients in the complex terms to get the complex part.
\item \( a \in [-144, -140] \text{ and } b \in [30, 36] \)

* $-142 + 34 i$, which is the correct option.
\item \( a \in [-4, 5] \text{ and } b \in [-152, -145] \)

 $2 - 146 i$, which corresponds to adding a minus sign in the first term.
\item \( a \in [-4, 5] \text{ and } b \in [141, 150] \)

 $2 + 146 i$, which corresponds to adding a minus sign in the second term.
\end{enumerate}

\textbf{General Comment:} You can treat $i$ as a variable and distribute. Just remember that $i^2=-1$, so you can continue to reduce after you distribute.
}
\litem{
Choose the \textbf{smallest} set of Real numbers that the number below belongs to.
\[ \sqrt{\frac{30625}{625}} \]The solution is \( \text{Whole} \), which is option D.\begin{enumerate}[label=\Alph*.]
\item \( \text{Irrational} \)

These cannot be written as a fraction of Integers.
\item \( \text{Not a Real number} \)

These are Nonreal Complex numbers \textbf{OR} things that are not numbers (e.g., dividing by 0).
\item \( \text{Rational} \)

These are numbers that can be written as fraction of Integers (e.g., -2/3)
\item \( \text{Whole} \)

* This is the correct option!
\item \( \text{Integer} \)

These are the negative and positive counting numbers (..., -3, -2, -1, 0, 1, 2, 3, ...)
\end{enumerate}

\textbf{General Comment:} First, you \textbf{NEED} to simplify the expression. This question simplifies to $175$. 
 
 Be sure you look at the simplified fraction and not just the decimal expansion. Numbers such as 13, 17, and 19 provide \textbf{long but repeating/terminating decimal expansions!} 
 
 The only ways to *not* be a Real number are: dividing by 0 or taking the square root of a negative number. 
 
 Irrational numbers are more than just square root of 3: adding or subtracting values from square root of 3 is also irrational.
}
\litem{
Simplify the expression below into the form $a+bi$. Then, choose the intervals that $a$ and $b$ belong to.
\[ (-3 - 2 i)(-5 - 8 i) \]The solution is \( -1 + 34 i \), which is option B.\begin{enumerate}[label=\Alph*.]
\item \( a \in [-3, 0] \text{ and } b \in [-39, -28] \)

 $-1 - 34 i$, which corresponds to adding a minus sign in both terms.
\item \( a \in [-3, 0] \text{ and } b \in [31, 38] \)

* $-1 + 34 i$, which is the correct option.
\item \( a \in [27, 36] \text{ and } b \in [-14, -10] \)

 $31 - 14 i$, which corresponds to adding a minus sign in the second term.
\item \( a \in [27, 36] \text{ and } b \in [10, 15] \)

 $31 + 14 i$, which corresponds to adding a minus sign in the first term.
\item \( a \in [14, 21] \text{ and } b \in [16, 21] \)

 $15 + 16 i$, which corresponds to just multiplying the real terms to get the real part of the solution and the coefficients in the complex terms to get the complex part.
\end{enumerate}

\textbf{General Comment:} You can treat $i$ as a variable and distribute. Just remember that $i^2=-1$, so you can continue to reduce after you distribute.
}
\litem{
Choose the \textbf{smallest} set of Real numbers that the number below belongs to.
\[ -\sqrt{\frac{18}{0}} \]The solution is \( \text{Not a Real number} \), which is option B.\begin{enumerate}[label=\Alph*.]
\item \( \text{Integer} \)

These are the negative and positive counting numbers (..., -3, -2, -1, 0, 1, 2, 3, ...)
\item \( \text{Not a Real number} \)

* This is the correct option!
\item \( \text{Whole} \)

These are the counting numbers with 0 (0, 1, 2, 3, ...)
\item \( \text{Rational} \)

These are numbers that can be written as fraction of Integers (e.g., -2/3)
\item \( \text{Irrational} \)

These cannot be written as a fraction of Integers.
\end{enumerate}

\textbf{General Comment:} First, you \textbf{NEED} to simplify the expression. This question simplifies to $-\sqrt{\frac{18}{0}}$. 
 
 Be sure you look at the simplified fraction and not just the decimal expansion. Numbers such as 13, 17, and 19 provide \textbf{long but repeating/terminating decimal expansions!} 
 
 The only ways to *not* be a Real number are: dividing by 0 or taking the square root of a negative number. 
 
 Irrational numbers are more than just square root of 3: adding or subtracting values from square root of 3 is also irrational.
}
\end{enumerate}

\end{document}