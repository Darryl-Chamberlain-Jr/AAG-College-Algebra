\documentclass{extbook}[14pt]
\usepackage{multicol, enumerate, enumitem, hyperref, color, soul, setspace, parskip, fancyhdr, amssymb, amsthm, amsmath, bbm, latexsym, units, mathtools}
\everymath{\displaystyle}
\usepackage[headsep=0.5cm,headheight=0cm, left=1 in,right= 1 in,top= 1 in,bottom= 1 in]{geometry}
\usepackage{dashrule}  % Package to use the command below to create lines between items
\newcommand{\litem}[1]{\item #1

\rule{\textwidth}{0.4pt}}
\pagestyle{fancy}
\lhead{}
\chead{Answer Key for Module1 Version B}
\rhead{}
\lfoot{4213-4786}
\cfoot{}
\rfoot{test}
\begin{document}
\textbf{This key should allow you to understand why you choose the option you did (beyond just getting a question right or wrong). \href{https://xronos.clas.ufl.edu/mac1105spring2020/courseDescriptionAndMisc/Exams/LearningFromResults}{More instructions on how to use this key can be found here}.}

\textbf{If you have a suggestion to make the keys better, \href{https://forms.gle/CZkbZmPbC9XALEE88}{please fill out the short survey here}.}

\textit{Note: This key is auto-generated and may contain issues and/or errors. The keys are reviewed after each exam to ensure grading is done accurately. If there are issues (like duplicate options), they are noted in the offline gradebook. The keys are a work-in-progress to give students as many resources to improve as possible.}

\rule{\textwidth}{0.4pt}

\begin{enumerate}\litem{
Simplify the expression below and choose the interval the simplification is contained within.
\[ 4 - 12^2 + 14 \div 10 * 20 \div 7 \]

The solution is \( -136.000 \), which is option B.\begin{enumerate}[label=\Alph*.]
\item \( [-139.99, -137.99] \)

 -139.990, which corresponds to an Order of Operations error: not reading left-to-right for multiplication/division.
\item \( [-138, -133] \)

* -136.000, this is the correct option
\item \( [151, 154] \)

 152.000, which corresponds to an Order of Operations error: multiplying by negative before squaring. For example: $(-3)^2 \neq -3^2$
\item \( [147.01, 150.01] \)

 148.010, which corresponds to two Order of Operations errors.
\item \( \text{None of the above} \)

 You may have gotten this by making an unanticipated error. If you got a value that is not any of the others, please let the coordinator know so they can help you figure out what happened.
\end{enumerate}

\textbf{General Comment:} While you may remember (or were taught) PEMDAS is done in order, it is actually done as P/E/MD/AS. When we are at MD or AS, we read left to right.
}
\litem{
Simplify the expression below into the form $a+bi$. Then, choose the intervals that $a$ and $b$ belong to.
\[ (2 + 3 i)(-4 - 10 i) \]

The solution is \( 22 - 32 i \), which is option B.\begin{enumerate}[label=\Alph*.]
\item \( a \in [-10, -6] \text{ and } b \in [-31.9, -29.2] \)

 $-8 - 30 i$, which corresponds to just multiplying the real terms to get the real part of the solution and the coefficients in the complex terms to get the complex part.
\item \( a \in [19, 25] \text{ and } b \in [-33.9, -31.7] \)

* $22 - 32 i$, which is the correct option.
\item \( a \in [19, 25] \text{ and } b \in [31.7, 34.7] \)

 $22 + 32 i$, which corresponds to adding a minus sign in both terms.
\item \( a \in [-44, -36] \text{ and } b \in [7.1, 10] \)

 $-38 + 8 i$, which corresponds to adding a minus sign in the second term.
\item \( a \in [-44, -36] \text{ and } b \in [-11.7, -6.1] \)

 $-38 - 8 i$, which corresponds to adding a minus sign in the first term.
\end{enumerate}

\textbf{General Comment:} You can treat $i$ as a variable and distribute. Just remember that $i^2=-1$, so you can continue to reduce after you distribute.
}
\litem{
Choose the \textbf{smallest} set of Complex numbers that the number below belongs to.
\[ \frac{\sqrt{154}}{14}+10i^2 \]

The solution is \( \text{Irrational} \), which is option B.\begin{enumerate}[label=\Alph*.]
\item \( \text{Pure Imaginary} \)

This is a Complex number $(a+bi)$ that \textbf{only} has an imaginary part like $2i$.
\item \( \text{Irrational} \)

* This is the correct option!
\item \( \text{Nonreal Complex} \)

This is a Complex number $(a+bi)$ that is not Real (has $i$ as part of the number).
\item \( \text{Rational} \)

These are numbers that can be written as fraction of Integers (e.g., -2/3 + 5)
\item \( \text{Not a Complex Number} \)

This is not a number. The only non-Complex number we know is dividing by 0 as this is not a number!
\end{enumerate}

\textbf{General Comment:} Be sure to simplify $i^2 = -1$. This may remove the imaginary portion for your number. If you are having trouble, you may want to look at the \textit{Subgroups of the Real Numbers} section.
}
\litem{
Choose the \textbf{smallest} set of Real numbers that the number below belongs to.
\[ \sqrt{\frac{32400}{400}} \]

The solution is \( \text{Whole} \), which is option E.\begin{enumerate}[label=\Alph*.]
\item \( \text{Irrational} \)

These cannot be written as a fraction of Integers.
\item \( \text{Not a Real number} \)

These are Nonreal Complex numbers \textbf{OR} things that are not numbers (e.g., dividing by 0).
\item \( \text{Integer} \)

These are the negative and positive counting numbers (..., -3, -2, -1, 0, 1, 2, 3, ...)
\item \( \text{Rational} \)

These are numbers that can be written as fraction of Integers (e.g., -2/3)
\item \( \text{Whole} \)

* This is the correct option!
\end{enumerate}

\textbf{General Comment:} First, you \textbf{NEED} to simplify the expression. This question simplifies to $180$. 
 
 Be sure you look at the simplified fraction and not just the decimal expansion. Numbers such as 13, 17, and 19 provide \textbf{long but repeating/terminating decimal expansions!} 
 
 The only ways to *not* be a Real number are: dividing by 0 or taking the square root of a negative number. 
 
 Irrational numbers are more than just square root of 3: adding or subtracting values from square root of 3 is also irrational.
}
\litem{
Choose the \textbf{smallest} set of Complex numbers that the number below belongs to.
\[ -\sqrt{\frac{825}{5}}+5i^2 \]

The solution is \( \text{Irrational} \), which is option D.\begin{enumerate}[label=\Alph*.]
\item \( \text{Pure Imaginary} \)

This is a Complex number $(a+bi)$ that \textbf{only} has an imaginary part like $2i$.
\item \( \text{Not a Complex Number} \)

This is not a number. The only non-Complex number we know is dividing by 0 as this is not a number!
\item \( \text{Rational} \)

These are numbers that can be written as fraction of Integers (e.g., -2/3 + 5)
\item \( \text{Irrational} \)

* This is the correct option!
\item \( \text{Nonreal Complex} \)

This is a Complex number $(a+bi)$ that is not Real (has $i$ as part of the number).
\end{enumerate}

\textbf{General Comment:} Be sure to simplify $i^2 = -1$. This may remove the imaginary portion for your number. If you are having trouble, you may want to look at the \textit{Subgroups of the Real Numbers} section.
}
\litem{
Simplify the expression below and choose the interval the simplification is contained within.
\[ 20 - 13^2 + 15 \div 1 * 18 \div 10 \]

The solution is \( -122.000 \), which is option C.\begin{enumerate}[label=\Alph*.]
\item \( [189.08, 196.08] \)

 189.083, which corresponds to two Order of Operations errors.
\item \( [-151.92, -141.92] \)

 -148.917, which corresponds to an Order of Operations error: not reading left-to-right for multiplication/division.
\item \( [-126, -115] \)

* -122.000, this is the correct option
\item \( [216, 220] \)

 216.000, which corresponds to an Order of Operations error: multiplying by negative before squaring. For example: $(-3)^2 \neq -3^2$
\item \( \text{None of the above} \)

 You may have gotten this by making an unanticipated error. If you got a value that is not any of the others, please let the coordinator know so they can help you figure out what happened.
\end{enumerate}

\textbf{General Comment:} While you may remember (or were taught) PEMDAS is done in order, it is actually done as P/E/MD/AS. When we are at MD or AS, we read left to right.
}
\litem{
Simplify the expression below into the form $a+bi$. Then, choose the intervals that $a$ and $b$ belong to.
\[ (-7 - 2 i)(-9 - 4 i) \]

The solution is \( 55 + 46 i \), which is option A.\begin{enumerate}[label=\Alph*.]
\item \( a \in [52, 60] \text{ and } b \in [45.4, 49.8] \)

* $55 + 46 i$, which is the correct option.
\item \( a \in [68, 74] \text{ and } b \in [8.5, 11.6] \)

 $71 + 10 i$, which corresponds to adding a minus sign in the first term.
\item \( a \in [61, 69] \text{ and } b \in [7.9, 9.8] \)

 $63 + 8 i$, which corresponds to just multiplying the real terms to get the real part of the solution and the coefficients in the complex terms to get the complex part.
\item \( a \in [52, 60] \text{ and } b \in [-48.3, -45.9] \)

 $55 - 46 i$, which corresponds to adding a minus sign in both terms.
\item \( a \in [68, 74] \text{ and } b \in [-10.2, -7.7] \)

 $71 - 10 i$, which corresponds to adding a minus sign in the second term.
\end{enumerate}

\textbf{General Comment:} You can treat $i$ as a variable and distribute. Just remember that $i^2=-1$, so you can continue to reduce after you distribute.
}
\litem{
Choose the \textbf{smallest} set of Real numbers that the number below belongs to.
\[ \sqrt{\frac{24}{0}} \]

The solution is \( \text{Not a Real number} \), which is option D.\begin{enumerate}[label=\Alph*.]
\item \( \text{Integer} \)

These are the negative and positive counting numbers (..., -3, -2, -1, 0, 1, 2, 3, ...)
\item \( \text{Rational} \)

These are numbers that can be written as fraction of Integers (e.g., -2/3)
\item \( \text{Whole} \)

These are the counting numbers with 0 (0, 1, 2, 3, ...)
\item \( \text{Not a Real number} \)

* This is the correct option!
\item \( \text{Irrational} \)

These cannot be written as a fraction of Integers.
\end{enumerate}

\textbf{General Comment:} First, you \textbf{NEED} to simplify the expression. This question simplifies to $\sqrt{\frac{24}{0}}$. 
 
 Be sure you look at the simplified fraction and not just the decimal expansion. Numbers such as 13, 17, and 19 provide \textbf{long but repeating/terminating decimal expansions!} 
 
 The only ways to *not* be a Real number are: dividing by 0 or taking the square root of a negative number. 
 
 Irrational numbers are more than just square root of 3: adding or subtracting values from square root of 3 is also irrational.
}
\litem{
Simplify the expression below into the form $a+bi$. Then, choose the intervals that $a$ and $b$ belong to.
\[ \frac{36 + 77 i}{3 - 8 i} \]

The solution is \( -6.96  + 7.11 i \), which is option A.\begin{enumerate}[label=\Alph*.]
\item \( a \in [-7.5, -6.5] \text{ and } b \in [6, 8] \)

* $-6.96  + 7.11 i$, which is the correct option.
\item \( a \in [-7.5, -6.5] \text{ and } b \in [518.5, 519.5] \)

 $-6.96  + 519.00 i$, which corresponds to forgetting to multiply the conjugate by the numerator.
\item \( a \in [11.5, 12.5] \text{ and } b \in [-10, -8.5] \)

 $12.00  - 9.62 i$, which corresponds to just dividing the first term by the first term and the second by the second.
\item \( a \in [9, 11.5] \text{ and } b \in [-2, 1] \)

 $9.92  - 0.78 i$, which corresponds to forgetting to multiply the conjugate by the numerator and not computing the conjugate correctly.
\item \( a \in [-508.5, -507.5] \text{ and } b \in [6, 8] \)

 $-508.00  + 7.11 i$, which corresponds to forgetting to multiply the conjugate by the numerator and using a plus instead of a minus in the denominator.
\end{enumerate}

\textbf{General Comment:} Multiply the numerator and denominator by the *conjugate* of the denominator, then simplify. For example, if we have $2+3i$, the conjugate is $2-3i$.
}
\litem{
Simplify the expression below into the form $a+bi$. Then, choose the intervals that $a$ and $b$ belong to.
\[ \frac{9 + 88 i}{-5 + 4 i} \]

The solution is \( 7.49  - 11.61 i \), which is option D.\begin{enumerate}[label=\Alph*.]
\item \( a \in [306.5, 308.5] \text{ and } b \in [-12, -11] \)

 $307.00  - 11.61 i$, which corresponds to forgetting to multiply the conjugate by the numerator and using a plus instead of a minus in the denominator.
\item \( a \in [6.5, 9] \text{ and } b \in [-477, -475] \)

 $7.49  - 476.00 i$, which corresponds to forgetting to multiply the conjugate by the numerator.
\item \( a \in [-3, -1] \text{ and } b \in [21.5, 22.5] \)

 $-1.80  + 22.00 i$, which corresponds to just dividing the first term by the first term and the second by the second.
\item \( a \in [6.5, 9] \text{ and } b \in [-12, -11] \)

* $7.49  - 11.61 i$, which is the correct option.
\item \( a \in [-10.5, -8.5] \text{ and } b \in [-11, -8.5] \)

 $-9.68  - 9.85 i$, which corresponds to forgetting to multiply the conjugate by the numerator and not computing the conjugate correctly.
\end{enumerate}

\textbf{General Comment:} Multiply the numerator and denominator by the *conjugate* of the denominator, then simplify. For example, if we have $2+3i$, the conjugate is $2-3i$.
}
\end{enumerate}

\end{document}