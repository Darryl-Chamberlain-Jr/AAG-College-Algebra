\documentclass{extbook}[14pt]
\usepackage{multicol, enumerate, enumitem, hyperref, color, soul, setspace, parskip, fancyhdr, amssymb, amsthm, amsmath, latexsym, units, mathtools}
\everymath{\displaystyle}
\usepackage[headsep=0.5cm,headheight=0cm, left=1 in,right= 1 in,top= 1 in,bottom= 1 in]{geometry}
\usepackage{dashrule}  % Package to use the command below to create lines between items
\newcommand{\litem}[1]{\item #1

\rule{\textwidth}{0.4pt}}
\pagestyle{fancy}
\lhead{}
\chead{Answer Key for Module1 Version B}
\rhead{}
\lfoot{6227-9062}
\cfoot{}
\rfoot{testing}
\begin{document}
\textbf{This key should allow you to understand why you choose the option you did (beyond just getting a question right or wrong). \href{https://xronos.clas.ufl.edu/mac1105spring2020/courseDescriptionAndMisc/Exams/LearningFromResults}{More instructions on how to use this key can be found here}.}

\textbf{If you have a suggestion to make the keys better, \href{https://forms.gle/CZkbZmPbC9XALEE88}{please fill out the short survey here}.}

\textit{Note: This key is auto-generated and may contain issues and/or errors. The keys are reviewed after each exam to ensure grading is done accurately. If there are issues (like duplicate options), they are noted in the offline gradebook. The keys are a work-in-progress to give students as many resources to improve as possible.}

\rule{\textwidth}{0.4pt}

\begin{enumerate}\litem{
Choose the \textbf{smallest} set of Complex numbers that the number below belongs to.
\[ \frac{-15}{0}+\sqrt{154} i \]The solution is \( \text{Not a Complex Number} \), which is option E.\begin{enumerate}[label=\Alph*.]
\item \( \text{Rational} \)

These are numbers that can be written as fraction of Integers (e.g., -2/3 + 5)
\item \( \text{Pure Imaginary} \)

This is a Complex number $(a+bi)$ that \textbf{only} has an imaginary part like $2i$.
\item \( \text{Irrational} \)

These cannot be written as a fraction of Integers. Remember: $\pi$ is not an Integer!
\item \( \text{Nonreal Complex} \)

This is a Complex number $(a+bi)$ that is not Real (has $i$ as part of the number).
\item \( \text{Not a Complex Number} \)

* This is the correct option!
\end{enumerate}

\textbf{General Comment:} Be sure to simplify $i^2 = -1$. This may remove the imaginary portion for your number. If you are having trouble, you may want to look at the \textit{Subgroups of the Real Numbers} section.
}
\litem{
Simplify the expression below into the form $a+bi$. Then, choose the intervals that $a$ and $b$ belong to.
\[ \frac{-45 - 11 i}{6 - 3 i} \]The solution is \( -5.27  - 4.47 i \), which is option E.\begin{enumerate}[label=\Alph*.]
\item \( a \in [-7.55, -7.05] \text{ and } b \in [2, 4.5] \)

 $-7.50  + 3.67 i$, which corresponds to just dividing the first term by the first term and the second by the second.
\item \( a \in [-237.2, -236.95] \text{ and } b \in [-5, -3.5] \)

 $-237.00  - 4.47 i$, which corresponds to forgetting to multiply the conjugate by the numerator and using a plus instead of a minus in the denominator.
\item \( a \in [-6.1, -4.8] \text{ and } b \in [-202, -200.5] \)

 $-5.27  - 201.00 i$, which corresponds to forgetting to multiply the conjugate by the numerator.
\item \( a \in [-6.75, -5.6] \text{ and } b \in [1, 2] \)

 $-6.73  + 1.53 i$, which corresponds to forgetting to multiply the conjugate by the numerator and not computing the conjugate correctly.
\item \( a \in [-6.1, -4.8] \text{ and } b \in [-5, -3.5] \)

* $-5.27  - 4.47 i$, which is the correct option.
\end{enumerate}

\textbf{General Comment:} Multiply the numerator and denominator by the *conjugate* of the denominator, then simplify. For example, if we have $2+3i$, the conjugate is $2-3i$.
}
\litem{
Simplify the expression below into the form $a+bi$. Then, choose the intervals that $a$ and $b$ belong to.
\[ (-3 - 2 i)(10 + 4 i) \]The solution is \( -22 - 32 i \), which is option D.\begin{enumerate}[label=\Alph*.]
\item \( a \in [-25, -19] \text{ and } b \in [29, 38] \)

 $-22 + 32 i$, which corresponds to adding a minus sign in both terms.
\item \( a \in [-40, -36] \text{ and } b \in [-9, -7] \)

 $-38 - 8 i$, which corresponds to adding a minus sign in the second term.
\item \( a \in [-33, -25] \text{ and } b \in [-9, -7] \)

 $-30 - 8 i$, which corresponds to just multiplying the real terms to get the real part of the solution and the coefficients in the complex terms to get the complex part.
\item \( a \in [-25, -19] \text{ and } b \in [-37, -26] \)

* $-22 - 32 i$, which is the correct option.
\item \( a \in [-40, -36] \text{ and } b \in [7, 16] \)

 $-38 + 8 i$, which corresponds to adding a minus sign in the first term.
\end{enumerate}

\textbf{General Comment:} You can treat $i$ as a variable and distribute. Just remember that $i^2=-1$, so you can continue to reduce after you distribute.
}
\litem{
Simplify the expression below into the form $a+bi$. Then, choose the intervals that $a$ and $b$ belong to.
\[ \frac{-45 + 88 i}{6 + 7 i} \]The solution is \( 4.07  + 9.92 i \), which is option D.\begin{enumerate}[label=\Alph*.]
\item \( a \in [-11.5, -9.5] \text{ and } b \in [1.5, 4] \)

 $-10.42  + 2.51 i$, which corresponds to forgetting to multiply the conjugate by the numerator and not computing the conjugate correctly.
\item \( a \in [345, 346.5] \text{ and } b \in [8.5, 10.5] \)

 $346.00  + 9.92 i$, which corresponds to forgetting to multiply the conjugate by the numerator and using a plus instead of a minus in the denominator.
\item \( a \in [3.5, 5.5] \text{ and } b \in [842.5, 844] \)

 $4.07  + 843.00 i$, which corresponds to forgetting to multiply the conjugate by the numerator.
\item \( a \in [3.5, 5.5] \text{ and } b \in [8.5, 10.5] \)

* $4.07  + 9.92 i$, which is the correct option.
\item \( a \in [-9, -7] \text{ and } b \in [11.5, 13] \)

 $-7.50  + 12.57 i$, which corresponds to just dividing the first term by the first term and the second by the second.
\end{enumerate}

\textbf{General Comment:} Multiply the numerator and denominator by the *conjugate* of the denominator, then simplify. For example, if we have $2+3i$, the conjugate is $2-3i$.
}
\litem{
Simplify the expression below and choose the interval the simplification is contained within.
\[ 17 - 18 \div 4 * 11 - (14 * 2) \]The solution is \( -60.500 \), which is option A.\begin{enumerate}[label=\Alph*.]
\item \( [-60.5, -55.5] \)

* -60.500, which is the correct option.
\item \( [-96, -92] \)

 -93.000, which corresponds to not distributing a negative correctly.
\item \( [41.59, 45.59] \)

 44.591, which corresponds to not distributing addition and subtraction correctly.
\item \( [-11.41, -7.41] \)

 -11.409, which corresponds to an Order of Operations error: not reading left-to-right for multiplication/division.
\item \( \text{None of the above} \)

 You may have gotten this by making an unanticipated error. If you got a value that is not any of the others, please let the coordinator know so they can help you figure out what happened.
\end{enumerate}

\textbf{General Comment:} While you may remember (or were taught) PEMDAS is done in order, it is actually done as P/E/MD/AS. When we are at MD or AS, we read left to right.
}
\litem{
Choose the \textbf{smallest} set of Real numbers that the number below belongs to.
\[ \sqrt{\frac{20736}{144}} \]The solution is \( \text{Whole} \), which is option B.\begin{enumerate}[label=\Alph*.]
\item \( \text{Integer} \)

These are the negative and positive counting numbers (..., -3, -2, -1, 0, 1, 2, 3, ...)
\item \( \text{Whole} \)

* This is the correct option!
\item \( \text{Irrational} \)

These cannot be written as a fraction of Integers.
\item \( \text{Not a Real number} \)

These are Nonreal Complex numbers \textbf{OR} things that are not numbers (e.g., dividing by 0).
\item \( \text{Rational} \)

These are numbers that can be written as fraction of Integers (e.g., -2/3)
\end{enumerate}

\textbf{General Comment:} First, you \textbf{NEED} to simplify the expression. This question simplifies to $144$. 
 
 Be sure you look at the simplified fraction and not just the decimal expansion. Numbers such as 13, 17, and 19 provide \textbf{long but repeating/terminating decimal expansions!} 
 
 The only ways to *not* be a Real number are: dividing by 0 or taking the square root of a negative number. 
 
 Irrational numbers are more than just square root of 3: adding or subtracting values from square root of 3 is also irrational.
}
\litem{
Choose the \textbf{smallest} set of Complex numbers that the number below belongs to.
\[ \sqrt{\frac{-990}{10}} i+\sqrt{165}i \]The solution is \( \text{Nonreal Complex} \), which is option B.\begin{enumerate}[label=\Alph*.]
\item \( \text{Rational} \)

These are numbers that can be written as fraction of Integers (e.g., -2/3 + 5)
\item \( \text{Nonreal Complex} \)

* This is the correct option!
\item \( \text{Irrational} \)

These cannot be written as a fraction of Integers. Remember: $\pi$ is not an Integer!
\item \( \text{Pure Imaginary} \)

This is a Complex number $(a+bi)$ that \textbf{only} has an imaginary part like $2i$.
\item \( \text{Not a Complex Number} \)

This is not a number. The only non-Complex number we know is dividing by 0 as this is not a number!
\end{enumerate}

\textbf{General Comment:} Be sure to simplify $i^2 = -1$. This may remove the imaginary portion for your number. If you are having trouble, you may want to look at the \textit{Subgroups of the Real Numbers} section.
}
\litem{
Simplify the expression below into the form $a+bi$. Then, choose the intervals that $a$ and $b$ belong to.
\[ (-2 - 4 i)(-10 + 8 i) \]The solution is \( 52 + 24 i \), which is option E.\begin{enumerate}[label=\Alph*.]
\item \( a \in [-18, -6] \text{ and } b \in [53, 60] \)

 $-12 + 56 i$, which corresponds to adding a minus sign in the second term.
\item \( a \in [46, 58] \text{ and } b \in [-24, -23] \)

 $52 - 24 i$, which corresponds to adding a minus sign in both terms.
\item \( a \in [-18, -6] \text{ and } b \in [-59, -53] \)

 $-12 - 56 i$, which corresponds to adding a minus sign in the first term.
\item \( a \in [15, 24] \text{ and } b \in [-36, -29] \)

 $20 - 32 i$, which corresponds to just multiplying the real terms to get the real part of the solution and the coefficients in the complex terms to get the complex part.
\item \( a \in [46, 58] \text{ and } b \in [17, 29] \)

* $52 + 24 i$, which is the correct option.
\end{enumerate}

\textbf{General Comment:} You can treat $i$ as a variable and distribute. Just remember that $i^2=-1$, so you can continue to reduce after you distribute.
}
\litem{
Choose the \textbf{smallest} set of Real numbers that the number below belongs to.
\[ -\sqrt{\frac{330625}{625}} \]The solution is \( \text{Integer} \), which is option E.\begin{enumerate}[label=\Alph*.]
\item \( \text{Whole} \)

These are the counting numbers with 0 (0, 1, 2, 3, ...)
\item \( \text{Rational} \)

These are numbers that can be written as fraction of Integers (e.g., -2/3)
\item \( \text{Not a Real number} \)

These are Nonreal Complex numbers \textbf{OR} things that are not numbers (e.g., dividing by 0).
\item \( \text{Irrational} \)

These cannot be written as a fraction of Integers.
\item \( \text{Integer} \)

* This is the correct option!
\end{enumerate}

\textbf{General Comment:} First, you \textbf{NEED} to simplify the expression. This question simplifies to $-575$. 
 
 Be sure you look at the simplified fraction and not just the decimal expansion. Numbers such as 13, 17, and 19 provide \textbf{long but repeating/terminating decimal expansions!} 
 
 The only ways to *not* be a Real number are: dividing by 0 or taking the square root of a negative number. 
 
 Irrational numbers are more than just square root of 3: adding or subtracting values from square root of 3 is also irrational.
}
\litem{
Simplify the expression below and choose the interval the simplification is contained within.
\[ 18 - 14 \div 15 * 19 - (12 * 6) \]The solution is \( -71.733 \), which is option B.\begin{enumerate}[label=\Alph*.]
\item \( [89.23, 90.02] \)

 89.951, which corresponds to not distributing addition and subtraction correctly.
\item \( [-72.93, -70.46] \)

* -71.733, which is the correct option.
\item \( [-71.48, -69.28] \)

 -70.400, which corresponds to not distributing a negative correctly.
\item \( [-54.18, -52.78] \)

 -54.049, which corresponds to an Order of Operations error: not reading left-to-right for multiplication/division.
\item \( \text{None of the above} \)

 You may have gotten this by making an unanticipated error. If you got a value that is not any of the others, please let the coordinator know so they can help you figure out what happened.
\end{enumerate}

\textbf{General Comment:} While you may remember (or were taught) PEMDAS is done in order, it is actually done as P/E/MD/AS. When we are at MD or AS, we read left to right.
}
\end{enumerate}

\end{document}