\documentclass{extbook}[14pt]
\usepackage{multicol, enumerate, enumitem, hyperref, color, soul, setspace, parskip, fancyhdr, amssymb, amsthm, amsmath, latexsym, units, mathtools}
\everymath{\displaystyle}
\usepackage[headsep=0.5cm,headheight=0cm, left=1 in,right= 1 in,top= 1 in,bottom= 1 in]{geometry}
\usepackage{dashrule}  % Package to use the command below to create lines between items
\newcommand{\litem}[1]{\item #1

\rule{\textwidth}{0.4pt}}
\pagestyle{fancy}
\lhead{}
\chead{Answer Key for Module1 Version A}
\rhead{}
\lfoot{6227-9062}
\cfoot{}
\rfoot{testing}
\begin{document}
\textbf{This key should allow you to understand why you choose the option you did (beyond just getting a question right or wrong). \href{https://xronos.clas.ufl.edu/mac1105spring2020/courseDescriptionAndMisc/Exams/LearningFromResults}{More instructions on how to use this key can be found here}.}

\textbf{If you have a suggestion to make the keys better, \href{https://forms.gle/CZkbZmPbC9XALEE88}{please fill out the short survey here}.}

\textit{Note: This key is auto-generated and may contain issues and/or errors. The keys are reviewed after each exam to ensure grading is done accurately. If there are issues (like duplicate options), they are noted in the offline gradebook. The keys are a work-in-progress to give students as many resources to improve as possible.}

\rule{\textwidth}{0.4pt}

\begin{enumerate}\litem{
Choose the \textbf{smallest} set of Complex numbers that the number below belongs to.
\[ \frac{5}{-20}+\sqrt{-4}i \]The solution is \( \text{Rational} \), which is option E.\begin{enumerate}[label=\Alph*.]
\item \( \text{Not a Complex Number} \)

This is not a number. The only non-Complex number we know is dividing by 0 as this is not a number!
\item \( \text{Nonreal Complex} \)

This is a Complex number $(a+bi)$ that is not Real (has $i$ as part of the number).
\item \( \text{Irrational} \)

These cannot be written as a fraction of Integers. Remember: $\pi$ is not an Integer!
\item \( \text{Pure Imaginary} \)

This is a Complex number $(a+bi)$ that \textbf{only} has an imaginary part like $2i$.
\item \( \text{Rational} \)

* This is the correct option!
\end{enumerate}

\textbf{General Comment:} Be sure to simplify $i^2 = -1$. This may remove the imaginary portion for your number. If you are having trouble, you may want to look at the \textit{Subgroups of the Real Numbers} section.
}
\litem{
Simplify the expression below into the form $a+bi$. Then, choose the intervals that $a$ and $b$ belong to.
\[ \frac{9 + 55 i}{6 - 8 i} \]The solution is \( -3.86  + 4.02 i \), which is option C.\begin{enumerate}[label=\Alph*.]
\item \( a \in [0, 3] \text{ and } b \in [-8, -6.5] \)

 $1.50  - 6.88 i$, which corresponds to just dividing the first term by the first term and the second by the second.
\item \( a \in [4.5, 6] \text{ and } b \in [2, 4] \)

 $4.94  + 2.58 i$, which corresponds to forgetting to multiply the conjugate by the numerator and not computing the conjugate correctly.
\item \( a \in [-4.5, -2.5] \text{ and } b \in [3.5, 5.5] \)

* $-3.86  + 4.02 i$, which is the correct option.
\item \( a \in [-386.5, -384.5] \text{ and } b \in [3.5, 5.5] \)

 $-386.00  + 4.02 i$, which corresponds to forgetting to multiply the conjugate by the numerator and using a plus instead of a minus in the denominator.
\item \( a \in [-4.5, -2.5] \text{ and } b \in [401.5, 403.5] \)

 $-3.86  + 402.00 i$, which corresponds to forgetting to multiply the conjugate by the numerator.
\end{enumerate}

\textbf{General Comment:} Multiply the numerator and denominator by the *conjugate* of the denominator, then simplify. For example, if we have $2+3i$, the conjugate is $2-3i$.
}
\litem{
Simplify the expression below into the form $a+bi$. Then, choose the intervals that $a$ and $b$ belong to.
\[ (-2 - 6 i)(10 + 3 i) \]The solution is \( -2 - 66 i \), which is option E.\begin{enumerate}[label=\Alph*.]
\item \( a \in [-5, 0] \text{ and } b \in [65, 68] \)

 $-2 + 66 i$, which corresponds to adding a minus sign in both terms.
\item \( a \in [-21, -15] \text{ and } b \in [-19, -14] \)

 $-20 - 18 i$, which corresponds to just multiplying the real terms to get the real part of the solution and the coefficients in the complex terms to get the complex part.
\item \( a \in [-40, -36] \text{ and } b \in [53, 55] \)

 $-38 + 54 i$, which corresponds to adding a minus sign in the first term.
\item \( a \in [-40, -36] \text{ and } b \in [-56, -50] \)

 $-38 - 54 i$, which corresponds to adding a minus sign in the second term.
\item \( a \in [-5, 0] \text{ and } b \in [-66, -60] \)

* $-2 - 66 i$, which is the correct option.
\end{enumerate}

\textbf{General Comment:} You can treat $i$ as a variable and distribute. Just remember that $i^2=-1$, so you can continue to reduce after you distribute.
}
\litem{
Simplify the expression below into the form $a+bi$. Then, choose the intervals that $a$ and $b$ belong to.
\[ \frac{36 - 88 i}{1 - 6 i} \]The solution is \( 15.24  + 3.46 i \), which is option B.\begin{enumerate}[label=\Alph*.]
\item \( a \in [35, 36.5] \text{ and } b \in [14.5, 15] \)

 $36.00  + 14.67 i$, which corresponds to just dividing the first term by the first term and the second by the second.
\item \( a \in [13.5, 15.5] \text{ and } b \in [2.5, 4] \)

* $15.24  + 3.46 i$, which is the correct option.
\item \( a \in [-13.5, -12] \text{ and } b \in [-8.5, -8] \)

 $-13.30  - 8.22 i$, which corresponds to forgetting to multiply the conjugate by the numerator and not computing the conjugate correctly.
\item \( a \in [563.5, 565] \text{ and } b \in [2.5, 4] \)

 $564.00  + 3.46 i$, which corresponds to forgetting to multiply the conjugate by the numerator and using a plus instead of a minus in the denominator.
\item \( a \in [13.5, 15.5] \text{ and } b \in [127.5, 129] \)

 $15.24  + 128.00 i$, which corresponds to forgetting to multiply the conjugate by the numerator.
\end{enumerate}

\textbf{General Comment:} Multiply the numerator and denominator by the *conjugate* of the denominator, then simplify. For example, if we have $2+3i$, the conjugate is $2-3i$.
}
\litem{
Simplify the expression below and choose the interval the simplification is contained within.
\[ 9 - 2 \div 15 * 19 - (13 * 12) \]The solution is \( -149.533 \), which is option B.\begin{enumerate}[label=\Alph*.]
\item \( [-148.01, -144.01] \)

 -147.007, which corresponds to an Order of Operations error: not reading left-to-right for multiplication/division.
\item \( [-154.53, -148.53] \)

* -149.533, which is the correct option.
\item \( [-82.4, -75.4] \)

 -78.400, which corresponds to not distributing a negative correctly.
\item \( [164.99, 169.99] \)

 164.993, which corresponds to not distributing addition and subtraction correctly.
\item \( \text{None of the above} \)

 You may have gotten this by making an unanticipated error. If you got a value that is not any of the others, please let the coordinator know so they can help you figure out what happened.
\end{enumerate}

\textbf{General Comment:} While you may remember (or were taught) PEMDAS is done in order, it is actually done as P/E/MD/AS. When we are at MD or AS, we read left to right.
}
\litem{
Choose the \textbf{smallest} set of Real numbers that the number below belongs to.
\[ -\sqrt{\frac{8100}{25}} \]The solution is \( \text{Integer} \), which is option A.\begin{enumerate}[label=\Alph*.]
\item \( \text{Integer} \)

* This is the correct option!
\item \( \text{Not a Real number} \)

These are Nonreal Complex numbers \textbf{OR} things that are not numbers (e.g., dividing by 0).
\item \( \text{Whole} \)

These are the counting numbers with 0 (0, 1, 2, 3, ...)
\item \( \text{Irrational} \)

These cannot be written as a fraction of Integers.
\item \( \text{Rational} \)

These are numbers that can be written as fraction of Integers (e.g., -2/3)
\end{enumerate}

\textbf{General Comment:} First, you \textbf{NEED} to simplify the expression. This question simplifies to $-90$. 
 
 Be sure you look at the simplified fraction and not just the decimal expansion. Numbers such as 13, 17, and 19 provide \textbf{long but repeating/terminating decimal expansions!} 
 
 The only ways to *not* be a Real number are: dividing by 0 or taking the square root of a negative number. 
 
 Irrational numbers are more than just square root of 3: adding or subtracting values from square root of 3 is also irrational.
}
\litem{
Choose the \textbf{smallest} set of Complex numbers that the number below belongs to.
\[ -\sqrt{\frac{625}{36}} + 36i^2 \]The solution is \( \text{Rational} \), which is option C.\begin{enumerate}[label=\Alph*.]
\item \( \text{Irrational} \)

These cannot be written as a fraction of Integers. Remember: $\pi$ is not an Integer!
\item \( \text{Pure Imaginary} \)

This is a Complex number $(a+bi)$ that \textbf{only} has an imaginary part like $2i$.
\item \( \text{Rational} \)

* This is the correct option!
\item \( \text{Not a Complex Number} \)

This is not a number. The only non-Complex number we know is dividing by 0 as this is not a number!
\item \( \text{Nonreal Complex} \)

This is a Complex number $(a+bi)$ that is not Real (has $i$ as part of the number).
\end{enumerate}

\textbf{General Comment:} Be sure to simplify $i^2 = -1$. This may remove the imaginary portion for your number. If you are having trouble, you may want to look at the \textit{Subgroups of the Real Numbers} section.
}
\litem{
Simplify the expression below into the form $a+bi$. Then, choose the intervals that $a$ and $b$ belong to.
\[ (-2 - 10 i)(8 + 6 i) \]The solution is \( 44 - 92 i \), which is option B.\begin{enumerate}[label=\Alph*.]
\item \( a \in [-82, -71] \text{ and } b \in [59, 75] \)

 $-76 + 68 i$, which corresponds to adding a minus sign in the first term.
\item \( a \in [36, 48] \text{ and } b \in [-94, -86] \)

* $44 - 92 i$, which is the correct option.
\item \( a \in [-82, -71] \text{ and } b \in [-70, -61] \)

 $-76 - 68 i$, which corresponds to adding a minus sign in the second term.
\item \( a \in [36, 48] \text{ and } b \in [84, 98] \)

 $44 + 92 i$, which corresponds to adding a minus sign in both terms.
\item \( a \in [-20, -15] \text{ and } b \in [-65, -57] \)

 $-16 - 60 i$, which corresponds to just multiplying the real terms to get the real part of the solution and the coefficients in the complex terms to get the complex part.
\end{enumerate}

\textbf{General Comment:} You can treat $i$ as a variable and distribute. Just remember that $i^2=-1$, so you can continue to reduce after you distribute.
}
\litem{
Choose the \textbf{smallest} set of Real numbers that the number below belongs to.
\[ -\sqrt{\frac{11664}{324}} \]The solution is \( \text{Integer} \), which is option E.\begin{enumerate}[label=\Alph*.]
\item \( \text{Whole} \)

These are the counting numbers with 0 (0, 1, 2, 3, ...)
\item \( \text{Not a Real number} \)

These are Nonreal Complex numbers \textbf{OR} things that are not numbers (e.g., dividing by 0).
\item \( \text{Rational} \)

These are numbers that can be written as fraction of Integers (e.g., -2/3)
\item \( \text{Irrational} \)

These cannot be written as a fraction of Integers.
\item \( \text{Integer} \)

* This is the correct option!
\end{enumerate}

\textbf{General Comment:} First, you \textbf{NEED} to simplify the expression. This question simplifies to $-108$. 
 
 Be sure you look at the simplified fraction and not just the decimal expansion. Numbers such as 13, 17, and 19 provide \textbf{long but repeating/terminating decimal expansions!} 
 
 The only ways to *not* be a Real number are: dividing by 0 or taking the square root of a negative number. 
 
 Irrational numbers are more than just square root of 3: adding or subtracting values from square root of 3 is also irrational.
}
\litem{
Simplify the expression below and choose the interval the simplification is contained within.
\[ 12 - 14 \div 1 * 19 - (3 * 9) \]The solution is \( -281.000 \), which is option D.\begin{enumerate}[label=\Alph*.]
\item \( [33.26, 42.26] \)

 38.263, which corresponds to not distributing addition and subtraction correctly.
\item \( [-22.74, -13.74] \)

 -15.737, which corresponds to an Order of Operations error: not reading left-to-right for multiplication/division.
\item \( [-2316, -2310] \)

 -2313.000, which corresponds to not distributing a negative correctly.
\item \( [-283, -278] \)

* -281.000, which is the correct option.
\item \( \text{None of the above} \)

 You may have gotten this by making an unanticipated error. If you got a value that is not any of the others, please let the coordinator know so they can help you figure out what happened.
\end{enumerate}

\textbf{General Comment:} While you may remember (or were taught) PEMDAS is done in order, it is actually done as P/E/MD/AS. When we are at MD or AS, we read left to right.
}
\end{enumerate}

\end{document}