\documentclass{extbook}[14pt]
\usepackage{multicol, enumerate, enumitem, hyperref, color, soul, setspace, parskip, fancyhdr, amssymb, amsthm, amsmath, bbm, latexsym, units, mathtools}
\everymath{\displaystyle}
\usepackage[headsep=0.5cm,headheight=0cm, left=1 in,right= 1 in,top= 1 in,bottom= 1 in]{geometry}
\usepackage{dashrule}  % Package to use the command below to create lines between items
\newcommand{\litem}[1]{\item #1

\rule{\textwidth}{0.4pt}}
\pagestyle{fancy}
\lhead{}
\chead{Answer Key for Module1 Version A}
\rhead{}
\lfoot{4565-2610}
\cfoot{}
\rfoot{Fall 2020}
\begin{document}
\textbf{This key should allow you to understand why you choose the option you did (beyond just getting a question right or wrong). \href{https://xronos.clas.ufl.edu/mac1105spring2020/courseDescriptionAndMisc/Exams/LearningFromResults}{More instructions on how to use this key can be found here}.}

\textbf{If you have a suggestion to make the keys better, \href{https://forms.gle/CZkbZmPbC9XALEE88}{please fill out the short survey here}.}

\textit{Note: This key is auto-generated and may contain issues and/or errors. The keys are reviewed after each exam to ensure grading is done accurately. If there are issues (like duplicate options), they are noted in the offline gradebook. The keys are a work-in-progress to give students as many resources to improve as possible.}

\rule{\textwidth}{0.4pt}

\begin{enumerate}\litem{
Choose the \textbf{smallest} set of Complex numbers that the number below belongs to.
\[ \sqrt{\frac{990}{11}}+\sqrt{90} i \]
The solution is \( \text{Nonreal Complex} \), which is option C.\begin{enumerate}[label=\Alph*.]
\item \( \text{Pure Imaginary} \)

This is a Complex number $(a+bi)$ that \textbf{only} has an imaginary part like $2i$.
\item \( \text{Irrational} \)

These cannot be written as a fraction of Integers. Remember: $\pi$ is not an Integer!
\item \( \text{Nonreal Complex} \)

* This is the correct option!
\item \( \text{Rational} \)

These are numbers that can be written as fraction of Integers (e.g., -2/3 + 5)
\item \( \text{Not a Complex Number} \)

This is not a number. The only non-Complex number we know is dividing by 0 as this is not a number!
\end{enumerate}

\textbf{General Comment:} Be sure to simplify $i^2 = -1$. This may remove the imaginary portion for your number. If you are having trouble, you may want to look at the \textit{Subgroups of the Real Numbers} section.
}
\litem{
Simplify the expression below into the form $a+bi$. Then, choose the intervals that $a$ and $b$ belong to.
\[ \frac{-36 + 77 i}{3 - 6 i} \]
The solution is \( -12.67  + 0.33 i \), which is option D.\begin{enumerate}[label=\Alph*.]
\item \( a \in [6.9, 8.7] \text{ and } b \in [9, 10.5] \)

 $7.87  + 9.93 i$, which corresponds to forgetting to multiply the conjugate by the numerator and not computing the conjugate correctly.
\item \( a \in [-12.3, -11.45] \text{ and } b \in [-14, -12] \)

 $-12.00  - 12.83 i$, which corresponds to just dividing the first term by the first term and the second by the second.
\item \( a \in [-12.9, -12.35] \text{ and } b \in [14.5, 15.5] \)

 $-12.67  + 15.00 i$, which corresponds to forgetting to multiply the conjugate by the numerator.
\item \( a \in [-12.9, -12.35] \text{ and } b \in [0, 1] \)

* $-12.67  + 0.33 i$, which is the correct option.
\item \( a \in [-570.35, -569.5] \text{ and } b \in [0, 1] \)

 $-570.00  + 0.33 i$, which corresponds to forgetting to multiply the conjugate by the numerator and using a plus instead of a minus in the denominator.
\end{enumerate}

\textbf{General Comment:} Multiply the numerator and denominator by the *conjugate* of the denominator, then simplify. For example, if we have $2+3i$, the conjugate is $2-3i$.
}
\litem{
Simplify the expression below into the form $a+bi$. Then, choose the intervals that $a$ and $b$ belong to.
\[ (-5 - 3 i)(-7 - 10 i) \]
The solution is \( 5 + 71 i \), which is option D.\begin{enumerate}[label=\Alph*.]
\item \( a \in [30, 37] \text{ and } b \in [29.92, 30.33] \)

 $35 + 30 i$, which corresponds to just multiplying the real terms to get the real part of the solution and the coefficients in the complex terms to get the complex part.
\item \( a \in [65, 66] \text{ and } b \in [28.37, 29.97] \)

 $65 + 29 i$, which corresponds to adding a minus sign in the first term.
\item \( a \in [0, 15] \text{ and } b \in [-71.23, -70.52] \)

 $5 - 71 i$, which corresponds to adding a minus sign in both terms.
\item \( a \in [0, 15] \text{ and } b \in [70.75, 71.19] \)

* $5 + 71 i$, which is the correct option.
\item \( a \in [65, 66] \text{ and } b \in [-30.01, -27.46] \)

 $65 - 29 i$, which corresponds to adding a minus sign in the second term.
\end{enumerate}

\textbf{General Comment:} You can treat $i$ as a variable and distribute. Just remember that $i^2=-1$, so you can continue to reduce after you distribute.
}
\litem{
Choose the \textbf{smallest} set of Complex numbers that the number below belongs to.
\[ \frac{3}{-8}+\sqrt{-16}i \]
The solution is \( \text{Rational} \), which is option C.\begin{enumerate}[label=\Alph*.]
\item \( \text{Irrational} \)

These cannot be written as a fraction of Integers. Remember: $\pi$ is not an Integer!
\item \( \text{Not a Complex Number} \)

This is not a number. The only non-Complex number we know is dividing by 0 as this is not a number!
\item \( \text{Rational} \)

* This is the correct option!
\item \( \text{Pure Imaginary} \)

This is a Complex number $(a+bi)$ that \textbf{only} has an imaginary part like $2i$.
\item \( \text{Nonreal Complex} \)

This is a Complex number $(a+bi)$ that is not Real (has $i$ as part of the number).
\end{enumerate}

\textbf{General Comment:} Be sure to simplify $i^2 = -1$. This may remove the imaginary portion for your number. If you are having trouble, you may want to look at the \textit{Subgroups of the Real Numbers} section.
}
\litem{
Simplify the expression below into the form $a+bi$. Then, choose the intervals that $a$ and $b$ belong to.
\[ (5 - 7 i)(-4 - 3 i) \]
The solution is \( -41 + 13 i \), which is option A.\begin{enumerate}[label=\Alph*.]
\item \( a \in [-46, -39] \text{ and } b \in [10, 15] \)

* $-41 + 13 i$, which is the correct option.
\item \( a \in [1, 2] \text{ and } b \in [35, 49] \)

 $1 + 43 i$, which corresponds to adding a minus sign in the second term.
\item \( a \in [1, 2] \text{ and } b \in [-45, -41] \)

 $1 - 43 i$, which corresponds to adding a minus sign in the first term.
\item \( a \in [-46, -39] \text{ and } b \in [-17, -10] \)

 $-41 - 13 i$, which corresponds to adding a minus sign in both terms.
\item \( a \in [-26, -15] \text{ and } b \in [21, 24] \)

 $-20 + 21 i$, which corresponds to just multiplying the real terms to get the real part of the solution and the coefficients in the complex terms to get the complex part.
\end{enumerate}

\textbf{General Comment:} You can treat $i$ as a variable and distribute. Just remember that $i^2=-1$, so you can continue to reduce after you distribute.
}
\litem{
Simplify the expression below into the form $a+bi$. Then, choose the intervals that $a$ and $b$ belong to.
\[ \frac{63 - 44 i}{8 + 5 i} \]
The solution is \( 3.19  - 7.49 i \), which is option D.\begin{enumerate}[label=\Alph*.]
\item \( a \in [2.92, 3.33] \text{ and } b \in [-667.5, -666.5] \)

 $3.19  - 667.00 i$, which corresponds to forgetting to multiply the conjugate by the numerator.
\item \( a \in [7.94, 8.27] \text{ and } b \in [-1.5, 0] \)

 $8.13  - 0.42 i$, which corresponds to forgetting to multiply the conjugate by the numerator and not computing the conjugate correctly.
\item \( a \in [283.9, 284.21] \text{ and } b \in [-8, -6.5] \)

 $284.00  - 7.49 i$, which corresponds to forgetting to multiply the conjugate by the numerator and using a plus instead of a minus in the denominator.
\item \( a \in [2.92, 3.33] \text{ and } b \in [-8, -6.5] \)

* $3.19  - 7.49 i$, which is the correct option.
\item \( a \in [7.72, 7.88] \text{ and } b \in [-9, -8] \)

 $7.88  - 8.80 i$, which corresponds to just dividing the first term by the first term and the second by the second.
\end{enumerate}

\textbf{General Comment:} Multiply the numerator and denominator by the *conjugate* of the denominator, then simplify. For example, if we have $2+3i$, the conjugate is $2-3i$.
}
\litem{
Simplify the expression below and choose the interval the simplification is contained within.
\[ 7 - 19^2 + 9 \div 12 * 14 \div 1 \]
The solution is \( -343.500 \), which is option C.\begin{enumerate}[label=\Alph*.]
\item \( [375.5, 379.5] \)

 378.500, which corresponds to an Order of Operations error: multiplying by negative before squaring. For example: $(-3)^2 \neq -3^2$
\item \( [364.05, 371.05] \)

 368.054, which corresponds to two Order of Operations errors.
\item \( [-345.5, -341.5] \)

* -343.500, this is the correct option
\item \( [-355.95, -352.95] \)

 -353.946, which corresponds to an Order of Operations error: not reading left-to-right for multiplication/division.
\item \( \text{None of the above} \)

 You may have gotten this by making an unanticipated error. If you got a value that is not any of the others, please let the coordinator know so they can help you figure out what happened.
\end{enumerate}

\textbf{General Comment:} While you may remember (or were taught) PEMDAS is done in order, it is actually done as P/E/MD/AS. When we are at MD or AS, we read left to right.
}
\litem{
Choose the \textbf{smallest} set of Real numbers that the number below belongs to.
\[ -\sqrt{\frac{-1232}{8}} \]
The solution is \( \text{Not a Real number} \), which is option C.\begin{enumerate}[label=\Alph*.]
\item \( \text{Rational} \)

These are numbers that can be written as fraction of Integers (e.g., -2/3)
\item \( \text{Whole} \)

These are the counting numbers with 0 (0, 1, 2, 3, ...)
\item \( \text{Not a Real number} \)

* This is the correct option!
\item \( \text{Integer} \)

These are the negative and positive counting numbers (..., -3, -2, -1, 0, 1, 2, 3, ...)
\item \( \text{Irrational} \)

These cannot be written as a fraction of Integers.
\end{enumerate}

\textbf{General Comment:} First, you \textbf{NEED} to simplify the expression. This question simplifies to $-\sqrt{154} i$. 
 
 Be sure you look at the simplified fraction and not just the decimal expansion. Numbers such as 13, 17, and 19 provide \textbf{long but repeating/terminating decimal expansions!} 
 
 The only ways to *not* be a Real number are: dividing by 0 or taking the square root of a negative number. 
 
 Irrational numbers are more than just square root of 3: adding or subtracting values from square root of 3 is also irrational.
}
\litem{
Simplify the expression below and choose the interval the simplification is contained within.
\[ 14 - 11^2 + 7 \div 1 * 2 \div 17 \]
The solution is \( -106.176 \), which is option B.\begin{enumerate}[label=\Alph*.]
\item \( [135.13, 135.44] \)

 135.206, which corresponds to two Order of Operations errors.
\item \( [-106.21, -105.91] \)

* -106.176, this is the correct option
\item \( [-107.39, -106.46] \)

 -106.794, which corresponds to an Order of Operations error: not reading left-to-right for multiplication/division.
\item \( [135.74, 136.1] \)

 135.824, which corresponds to an Order of Operations error: multiplying by negative before squaring. For example: $(-3)^2 \neq -3^2$
\item \( \text{None of the above} \)

 You may have gotten this by making an unanticipated error. If you got a value that is not any of the others, please let the coordinator know so they can help you figure out what happened.
\end{enumerate}

\textbf{General Comment:} While you may remember (or were taught) PEMDAS is done in order, it is actually done as P/E/MD/AS. When we are at MD or AS, we read left to right.
}
\litem{
Choose the \textbf{smallest} set of Real numbers that the number below belongs to.
\[ \sqrt{\frac{194481}{441}} \]
The solution is \( \text{Whole} \), which is option A.\begin{enumerate}[label=\Alph*.]
\item \( \text{Whole} \)

* This is the correct option!
\item \( \text{Not a Real number} \)

These are Nonreal Complex numbers \textbf{OR} things that are not numbers (e.g., dividing by 0).
\item \( \text{Integer} \)

These are the negative and positive counting numbers (..., -3, -2, -1, 0, 1, 2, 3, ...)
\item \( \text{Rational} \)

These are numbers that can be written as fraction of Integers (e.g., -2/3)
\item \( \text{Irrational} \)

These cannot be written as a fraction of Integers.
\end{enumerate}

\textbf{General Comment:} First, you \textbf{NEED} to simplify the expression. This question simplifies to $441$. 
 
 Be sure you look at the simplified fraction and not just the decimal expansion. Numbers such as 13, 17, and 19 provide \textbf{long but repeating/terminating decimal expansions!} 
 
 The only ways to *not* be a Real number are: dividing by 0 or taking the square root of a negative number. 
 
 Irrational numbers are more than just square root of 3: adding or subtracting values from square root of 3 is also irrational.
}
\end{enumerate}

\end{document}