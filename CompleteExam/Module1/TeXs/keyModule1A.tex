\documentclass{extbook}[14pt]
\usepackage{multicol, enumerate, enumitem, hyperref, color, soul, setspace, parskip, fancyhdr, amssymb, amsthm, amsmath, bbm, latexsym, units, mathtools}
\everymath{\displaystyle}
\usepackage[headsep=0.5cm,headheight=0cm, left=1 in,right= 1 in,top= 1 in,bottom= 1 in]{geometry}
\usepackage{dashrule}  % Package to use the command below to create lines between items
\newcommand{\litem}[1]{\item #1

\rule{\textwidth}{0.4pt}}
\pagestyle{fancy}
\lhead{}
\chead{Answer Key for Module1 Version A}
\rhead{}
\lfoot{4213-4786}
\cfoot{}
\rfoot{test}
\begin{document}
\textbf{This key should allow you to understand why you choose the option you did (beyond just getting a question right or wrong). \href{https://xronos.clas.ufl.edu/mac1105spring2020/courseDescriptionAndMisc/Exams/LearningFromResults}{More instructions on how to use this key can be found here}.}

\textbf{If you have a suggestion to make the keys better, \href{https://forms.gle/CZkbZmPbC9XALEE88}{please fill out the short survey here}.}

\textit{Note: This key is auto-generated and may contain issues and/or errors. The keys are reviewed after each exam to ensure grading is done accurately. If there are issues (like duplicate options), they are noted in the offline gradebook. The keys are a work-in-progress to give students as many resources to improve as possible.}

\rule{\textwidth}{0.4pt}

\begin{enumerate}\litem{
Simplify the expression below and choose the interval the simplification is contained within.
\[ 20 - 2 \div 12 * 13 - (19 * 10) \]

The solution is \( -172.167 \), which is option D.\begin{enumerate}[label=\Alph*.]
\item \( [-12.8, -9.3] \)

 -11.667, which corresponds to not distributing a negative correctly.
\item \( [207.5, 212.6] \)

 209.987, which corresponds to not distributing addition and subtraction correctly.
\item \( [-171.7, -168.3] \)

 -170.013, which corresponds to an Order of Operations error: not reading left-to-right for multiplication/division.
\item \( [-173.4, -171.5] \)

* -172.167, which is the correct option.
\item \( \text{None of the above} \)

 You may have gotten this by making an unanticipated error. If you got a value that is not any of the others, please let the coordinator know so they can help you figure out what happened.
\end{enumerate}

\textbf{General Comment:} While you may remember (or were taught) PEMDAS is done in order, it is actually done as P/E/MD/AS. When we are at MD or AS, we read left to right.
}
\litem{
Simplify the expression below into the form $a+bi$. Then, choose the intervals that $a$ and $b$ belong to.
\[ (8 + 5 i)(6 + 4 i) \]

The solution is \( 28 + 62 i \), which is option B.\begin{enumerate}[label=\Alph*.]
\item \( a \in [65, 75] \text{ and } b \in [-7, 1] \)

 $68 - 2 i$, which corresponds to adding a minus sign in the second term.
\item \( a \in [24, 32] \text{ and } b \in [61, 66] \)

* $28 + 62 i$, which is the correct option.
\item \( a \in [65, 75] \text{ and } b \in [-1, 9] \)

 $68 + 2 i$, which corresponds to adding a minus sign in the first term.
\item \( a \in [24, 32] \text{ and } b \in [-69, -58] \)

 $28 - 62 i$, which corresponds to adding a minus sign in both terms.
\item \( a \in [47, 49] \text{ and } b \in [19, 21] \)

 $48 + 20 i$, which corresponds to just multiplying the real terms to get the real part of the solution and the coefficients in the complex terms to get the complex part.
\end{enumerate}

\textbf{General Comment:} You can treat $i$ as a variable and distribute. Just remember that $i^2=-1$, so you can continue to reduce after you distribute.
}
\litem{
Choose the \textbf{smallest} set of Complex numbers that the number below belongs to.
\[ \sqrt{\frac{1053}{9}}+\sqrt{143} i \]

The solution is \( \text{Nonreal Complex} \), which is option E.\begin{enumerate}[label=\Alph*.]
\item \( \text{Irrational} \)

These cannot be written as a fraction of Integers. Remember: $\pi$ is not an Integer!
\item \( \text{Not a Complex Number} \)

This is not a number. The only non-Complex number we know is dividing by 0 as this is not a number!
\item \( \text{Pure Imaginary} \)

This is a Complex number $(a+bi)$ that \textbf{only} has an imaginary part like $2i$.
\item \( \text{Rational} \)

These are numbers that can be written as fraction of Integers (e.g., -2/3 + 5)
\item \( \text{Nonreal Complex} \)

* This is the correct option!
\end{enumerate}

\textbf{General Comment:} Be sure to simplify $i^2 = -1$. This may remove the imaginary portion for your number. If you are having trouble, you may want to look at the \textit{Subgroups of the Real Numbers} section.
}
\litem{
Choose the \textbf{smallest} set of Real numbers that the number below belongs to.
\[ \sqrt{\frac{19600}{196}} \]

The solution is \( \text{Whole} \), which is option A.\begin{enumerate}[label=\Alph*.]
\item \( \text{Whole} \)

* This is the correct option!
\item \( \text{Integer} \)

These are the negative and positive counting numbers (..., -3, -2, -1, 0, 1, 2, 3, ...)
\item \( \text{Rational} \)

These are numbers that can be written as fraction of Integers (e.g., -2/3)
\item \( \text{Irrational} \)

These cannot be written as a fraction of Integers.
\item \( \text{Not a Real number} \)

These are Nonreal Complex numbers \textbf{OR} things that are not numbers (e.g., dividing by 0).
\end{enumerate}

\textbf{General Comment:} First, you \textbf{NEED} to simplify the expression. This question simplifies to $140$. 
 
 Be sure you look at the simplified fraction and not just the decimal expansion. Numbers such as 13, 17, and 19 provide \textbf{long but repeating/terminating decimal expansions!} 
 
 The only ways to *not* be a Real number are: dividing by 0 or taking the square root of a negative number. 
 
 Irrational numbers are more than just square root of 3: adding or subtracting values from square root of 3 is also irrational.
}
\litem{
Choose the \textbf{smallest} set of Complex numbers that the number below belongs to.
\[ \sqrt{\frac{0}{144}}+\sqrt{10}i \]

The solution is \( \text{Pure Imaginary} \), which is option B.\begin{enumerate}[label=\Alph*.]
\item \( \text{Rational} \)

These are numbers that can be written as fraction of Integers (e.g., -2/3 + 5)
\item \( \text{Pure Imaginary} \)

* This is the correct option!
\item \( \text{Irrational} \)

These cannot be written as a fraction of Integers. Remember: $\pi$ is not an Integer!
\item \( \text{Nonreal Complex} \)

This is a Complex number $(a+bi)$ that is not Real (has $i$ as part of the number).
\item \( \text{Not a Complex Number} \)

This is not a number. The only non-Complex number we know is dividing by 0 as this is not a number!
\end{enumerate}

\textbf{General Comment:} Be sure to simplify $i^2 = -1$. This may remove the imaginary portion for your number. If you are having trouble, you may want to look at the \textit{Subgroups of the Real Numbers} section.
}
\litem{
Simplify the expression below and choose the interval the simplification is contained within.
\[ 1 - 18 \div 16 * 4 - (9 * 8) \]

The solution is \( -75.500 \), which is option B.\begin{enumerate}[label=\Alph*.]
\item \( [71.72, 73.72] \)

 72.719, which corresponds to not distributing addition and subtraction correctly.
\item \( [-82.5, -71.5] \)

* -75.500, which is the correct option.
\item \( [-75.28, -70.28] \)

 -71.281, which corresponds to an Order of Operations error: not reading left-to-right for multiplication/division.
\item \( [-101, -96] \)

 -100.000, which corresponds to not distributing a negative correctly.
\item \( \text{None of the above} \)

 You may have gotten this by making an unanticipated error. If you got a value that is not any of the others, please let the coordinator know so they can help you figure out what happened.
\end{enumerate}

\textbf{General Comment:} While you may remember (or were taught) PEMDAS is done in order, it is actually done as P/E/MD/AS. When we are at MD or AS, we read left to right.
}
\litem{
Simplify the expression below into the form $a+bi$. Then, choose the intervals that $a$ and $b$ belong to.
\[ (5 + 2 i)(8 - 9 i) \]

The solution is \( 58 - 29 i \), which is option E.\begin{enumerate}[label=\Alph*.]
\item \( a \in [20, 28] \text{ and } b \in [59, 62] \)

 $22 + 61 i$, which corresponds to adding a minus sign in the second term.
\item \( a \in [40, 43] \text{ and } b \in [-21, -7] \)

 $40 - 18 i$, which corresponds to just multiplying the real terms to get the real part of the solution and the coefficients in the complex terms to get the complex part.
\item \( a \in [20, 28] \text{ and } b \in [-67, -59] \)

 $22 - 61 i$, which corresponds to adding a minus sign in the first term.
\item \( a \in [57, 60] \text{ and } b \in [28, 30] \)

 $58 + 29 i$, which corresponds to adding a minus sign in both terms.
\item \( a \in [57, 60] \text{ and } b \in [-30, -23] \)

* $58 - 29 i$, which is the correct option.
\end{enumerate}

\textbf{General Comment:} You can treat $i$ as a variable and distribute. Just remember that $i^2=-1$, so you can continue to reduce after you distribute.
}
\litem{
Choose the \textbf{smallest} set of Real numbers that the number below belongs to.
\[ \sqrt{\frac{1980}{10}} \]

The solution is \( \text{Irrational} \), which is option B.\begin{enumerate}[label=\Alph*.]
\item \( \text{Whole} \)

These are the counting numbers with 0 (0, 1, 2, 3, ...)
\item \( \text{Irrational} \)

* This is the correct option!
\item \( \text{Integer} \)

These are the negative and positive counting numbers (..., -3, -2, -1, 0, 1, 2, 3, ...)
\item \( \text{Rational} \)

These are numbers that can be written as fraction of Integers (e.g., -2/3)
\item \( \text{Not a Real number} \)

These are Nonreal Complex numbers \textbf{OR} things that are not numbers (e.g., dividing by 0).
\end{enumerate}

\textbf{General Comment:} First, you \textbf{NEED} to simplify the expression. This question simplifies to $\sqrt{198}$. 
 
 Be sure you look at the simplified fraction and not just the decimal expansion. Numbers such as 13, 17, and 19 provide \textbf{long but repeating/terminating decimal expansions!} 
 
 The only ways to *not* be a Real number are: dividing by 0 or taking the square root of a negative number. 
 
 Irrational numbers are more than just square root of 3: adding or subtracting values from square root of 3 is also irrational.
}
\litem{
Simplify the expression below into the form $a+bi$. Then, choose the intervals that $a$ and $b$ belong to.
\[ \frac{-18 - 11 i}{6 - 5 i} \]

The solution is \( -0.87  - 2.56 i \), which is option C.\begin{enumerate}[label=\Alph*.]
\item \( a \in [-53.25, -52.45] \text{ and } b \in [-4, -1] \)

 $-53.00  - 2.56 i$, which corresponds to forgetting to multiply the conjugate by the numerator and using a plus instead of a minus in the denominator.
\item \( a \in [-3.1, -2.7] \text{ and } b \in [1, 3.5] \)

 $-3.00  + 2.20 i$, which corresponds to just dividing the first term by the first term and the second by the second.
\item \( a \in [-1.6, -0.45] \text{ and } b \in [-4, -1] \)

* $-0.87  - 2.56 i$, which is the correct option.
\item \( a \in [-1.6, -0.45] \text{ and } b \in [-156.5, -155] \)

 $-0.87  - 156.00 i$, which corresponds to forgetting to multiply the conjugate by the numerator.
\item \( a \in [-2.85, -2.6] \text{ and } b \in [-0.5, 0.5] \)

 $-2.67  + 0.39 i$, which corresponds to forgetting to multiply the conjugate by the numerator and not computing the conjugate correctly.
\end{enumerate}

\textbf{General Comment:} Multiply the numerator and denominator by the *conjugate* of the denominator, then simplify. For example, if we have $2+3i$, the conjugate is $2-3i$.
}
\litem{
Simplify the expression below into the form $a+bi$. Then, choose the intervals that $a$ and $b$ belong to.
\[ \frac{-9 + 22 i}{6 - 7 i} \]

The solution is \( -2.45  + 0.81 i \), which is option C.\begin{enumerate}[label=\Alph*.]
\item \( a \in [-2.6, -2.3] \text{ and } b \in [68.5, 70] \)

 $-2.45  + 69.00 i$, which corresponds to forgetting to multiply the conjugate by the numerator.
\item \( a \in [0.65, 1.3] \text{ and } b \in [2, 2.5] \)

 $1.18  + 2.29 i$, which corresponds to forgetting to multiply the conjugate by the numerator and not computing the conjugate correctly.
\item \( a \in [-2.6, -2.3] \text{ and } b \in [0, 2] \)

* $-2.45  + 0.81 i$, which is the correct option.
\item \( a \in [-1.9, -0.75] \text{ and } b \in [-3.5, -2] \)

 $-1.50  - 3.14 i$, which corresponds to just dividing the first term by the first term and the second by the second.
\item \( a \in [-208.2, -207.95] \text{ and } b \in [0, 2] \)

 $-208.00  + 0.81 i$, which corresponds to forgetting to multiply the conjugate by the numerator and using a plus instead of a minus in the denominator.
\end{enumerate}

\textbf{General Comment:} Multiply the numerator and denominator by the *conjugate* of the denominator, then simplify. For example, if we have $2+3i$, the conjugate is $2-3i$.
}
\end{enumerate}

\end{document}