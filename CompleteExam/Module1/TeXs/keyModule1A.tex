\documentclass{extbook}[14pt]
\usepackage{multicol, enumerate, enumitem, hyperref, color, soul, setspace, parskip, fancyhdr, amssymb, amsthm, amsmath, latexsym, units, mathtools}
\everymath{\displaystyle}
\usepackage[headsep=0.5cm,headheight=0cm, left=1 in,right= 1 in,top= 1 in,bottom= 1 in]{geometry}
\usepackage{dashrule}  % Package to use the command below to create lines between items
\newcommand{\litem}[1]{\item #1

\rule{\textwidth}{0.4pt}}
\pagestyle{fancy}
\lhead{}
\chead{Answer Key for Module1 Version A}
\rhead{}
\lfoot{4877-7341}
\cfoot{}
\rfoot{test}
\begin{document}
\textbf{This key should allow you to understand why you choose the option you did (beyond just getting a question right or wrong). \href{https://xronos.clas.ufl.edu/mac1105spring2020/courseDescriptionAndMisc/Exams/LearningFromResults}{More instructions on how to use this key can be found here}.}

\textbf{If you have a suggestion to make the keys better, \href{https://forms.gle/CZkbZmPbC9XALEE88}{please fill out the short survey here}.}

\textit{Note: This key is auto-generated and may contain issues and/or errors. The keys are reviewed after each exam to ensure grading is done accurately. If there are issues (like duplicate options), they are noted in the offline gradebook. The keys are a work-in-progress to give students as many resources to improve as possible.}

\rule{\textwidth}{0.4pt}

\begin{enumerate}\litem{
Choose the \textbf{smallest} set of Complex numbers that the number below belongs to.
\[ \frac{-11}{3}+81i^2 \]The solution is \( \text{Rational} \), which is option B.\begin{enumerate}[label=\Alph*.]
\item \( \text{Irrational} \)

These cannot be written as a fraction of Integers. Remember: $\pi$ is not an Integer!
\item \( \text{Rational} \)

* This is the correct option!
\item \( \text{Nonreal Complex} \)

This is a Complex number $(a+bi)$ that is not Real (has $i$ as part of the number).
\item \( \text{Not a Complex Number} \)

This is not a number. The only non-Complex number we know is dividing by 0 as this is not a number!
\item \( \text{Pure Imaginary} \)

This is a Complex number $(a+bi)$ that \textbf{only} has an imaginary part like $2i$.
\end{enumerate}

\textbf{General Comment:} Be sure to simplify $i^2 = -1$. This may remove the imaginary portion for your number. If you are having trouble, you may want to look at the \textit{Subgroups of the Real Numbers} section.
}
\litem{
Simplify the expression below and choose the interval the simplification is contained within.
\[ 5 - 14^2 + 19 \div 16 * 4 \div 9 \]The solution is \( -190.472 \), which is option A.\begin{enumerate}[label=\Alph*.]
\item \( [-190.87, -189.7] \)

* -190.472, this is the correct option
\item \( [-191.22, -190.73] \)

 -190.967, which corresponds to an Order of Operations error: not reading left-to-right for multiplication/division.
\item \( [201.3, 201.59] \)

 201.528, which corresponds to an Order of Operations error: multiplying by negative before squaring. For example: $(-3)^2 \neq -3^2$
\item \( [200.56, 201.24] \)

 201.033, which corresponds to two Order of Operations errors.
\item \( \text{None of the above} \)

 You may have gotten this by making an unanticipated error. If you got a value that is not any of the others, please let the coordinator know so they can help you figure out what happened.
\end{enumerate}

\textbf{General Comment:} While you may remember (or were taught) PEMDAS is done in order, it is actually done as P/E/MD/AS. When we are at MD or AS, we read left to right.
}
\litem{
Choose the \textbf{smallest} set of Complex numbers that the number below belongs to.
\[ \sqrt{\frac{100}{361}} + 4i^2 \]The solution is \( \text{Rational} \), which is option E.\begin{enumerate}[label=\Alph*.]
\item \( \text{Irrational} \)

These cannot be written as a fraction of Integers. Remember: $\pi$ is not an Integer!
\item \( \text{Nonreal Complex} \)

This is a Complex number $(a+bi)$ that is not Real (has $i$ as part of the number).
\item \( \text{Pure Imaginary} \)

This is a Complex number $(a+bi)$ that \textbf{only} has an imaginary part like $2i$.
\item \( \text{Not a Complex Number} \)

This is not a number. The only non-Complex number we know is dividing by 0 as this is not a number!
\item \( \text{Rational} \)

* This is the correct option!
\end{enumerate}

\textbf{General Comment:} Be sure to simplify $i^2 = -1$. This may remove the imaginary portion for your number. If you are having trouble, you may want to look at the \textit{Subgroups of the Real Numbers} section.
}
\litem{
Simplify the expression below into the form $a+bi$. Then, choose the intervals that $a$ and $b$ belong to.
\[ \frac{-9 - 77 i}{3 + 4 i} \]The solution is \( -13.40  - 7.80 i \), which is option C.\begin{enumerate}[label=\Alph*.]
\item \( a \in [-13.5, -13] \text{ and } b \in [-196, -194.5] \)

 $-13.40  - 195.00 i$, which corresponds to forgetting to multiply the conjugate by the numerator.
\item \( a \in [-3.5, -2.5] \text{ and } b \in [-20, -18] \)

 $-3.00  - 19.25 i$, which corresponds to just dividing the first term by the first term and the second by the second.
\item \( a \in [-13.5, -13] \text{ and } b \in [-8, -7] \)

* $-13.40  - 7.80 i$, which is the correct option.
\item \( a \in [-336, -334.5] \text{ and } b \in [-8, -7] \)

 $-335.00  - 7.80 i$, which corresponds to forgetting to multiply the conjugate by the numerator and using a plus instead of a minus in the denominator.
\item \( a \in [11, 12.5] \text{ and } b \in [-11, -9.5] \)

 $11.24  - 10.68 i$, which corresponds to forgetting to multiply the conjugate by the numerator and not computing the conjugate correctly.
\end{enumerate}

\textbf{General Comment:} Multiply the numerator and denominator by the *conjugate* of the denominator, then simplify. For example, if we have $2+3i$, the conjugate is $2-3i$.
}
\litem{
Simplify the expression below into the form $a+bi$. Then, choose the intervals that $a$ and $b$ belong to.
\[ \frac{9 + 77 i}{-4 - 6 i} \]The solution is \( -9.58  - 4.88 i \), which is option D.\begin{enumerate}[label=\Alph*.]
\item \( a \in [-499.5, -497.5] \text{ and } b \in [-5, -4] \)

 $-498.00  - 4.88 i$, which corresponds to forgetting to multiply the conjugate by the numerator and using a plus instead of a minus in the denominator.
\item \( a \in [-10, -9.5] \text{ and } b \in [-255, -253.5] \)

 $-9.58  - 254.00 i$, which corresponds to forgetting to multiply the conjugate by the numerator.
\item \( a \in [-2.5, -1] \text{ and } b \in [-14, -11.5] \)

 $-2.25  - 12.83 i$, which corresponds to just dividing the first term by the first term and the second by the second.
\item \( a \in [-10, -9.5] \text{ and } b \in [-5, -4] \)

* $-9.58  - 4.88 i$, which is the correct option.
\item \( a \in [7.5, 9.5] \text{ and } b \in [-7.5, -6] \)

 $8.19  - 6.96 i$, which corresponds to forgetting to multiply the conjugate by the numerator and not computing the conjugate correctly.
\end{enumerate}

\textbf{General Comment:} Multiply the numerator and denominator by the *conjugate* of the denominator, then simplify. For example, if we have $2+3i$, the conjugate is $2-3i$.
}
\litem{
Simplify the expression below and choose the interval the simplification is contained within.
\[ 1 - 16^2 + 4 \div 3 * 15 \div 5 \]The solution is \( -251.000 \), which is option D.\begin{enumerate}[label=\Alph*.]
\item \( [-256.7, -254.9] \)

 -254.982, which corresponds to an Order of Operations error: not reading left-to-right for multiplication/division.
\item \( [257.3, 261.1] \)

 261.000, which corresponds to an Order of Operations error: multiplying by negative before squaring. For example: $(-3)^2 \neq -3^2$
\item \( [254.2, 257.8] \)

 257.018, which corresponds to two Order of Operations errors.
\item \( [-253.8, -249.7] \)

* -251.000, this is the correct option
\item \( \text{None of the above} \)

 You may have gotten this by making an unanticipated error. If you got a value that is not any of the others, please let the coordinator know so they can help you figure out what happened.
\end{enumerate}

\textbf{General Comment:} While you may remember (or were taught) PEMDAS is done in order, it is actually done as P/E/MD/AS. When we are at MD or AS, we read left to right.
}
\litem{
Simplify the expression below into the form $a+bi$. Then, choose the intervals that $a$ and $b$ belong to.
\[ (-6 + 9 i)(-2 + 5 i) \]The solution is \( -33 - 48 i \), which is option B.\begin{enumerate}[label=\Alph*.]
\item \( a \in [50, 62] \text{ and } b \in [7, 18] \)

 $57 + 12 i$, which corresponds to adding a minus sign in the second term.
\item \( a \in [-38, -30] \text{ and } b \in [-48, -45] \)

* $-33 - 48 i$, which is the correct option.
\item \( a \in [7, 15] \text{ and } b \in [44, 47] \)

 $12 + 45 i$, which corresponds to just multiplying the real terms to get the real part of the solution and the coefficients in the complex terms to get the complex part.
\item \( a \in [-38, -30] \text{ and } b \in [46, 54] \)

 $-33 + 48 i$, which corresponds to adding a minus sign in both terms.
\item \( a \in [50, 62] \text{ and } b \in [-14, -11] \)

 $57 - 12 i$, which corresponds to adding a minus sign in the first term.
\end{enumerate}

\textbf{General Comment:} You can treat $i$ as a variable and distribute. Just remember that $i^2=-1$, so you can continue to reduce after you distribute.
}
\litem{
Choose the \textbf{smallest} set of Real numbers that the number below belongs to.
\[ \sqrt{\frac{12996}{361}} \]The solution is \( \text{Whole} \), which is option C.\begin{enumerate}[label=\Alph*.]
\item \( \text{Integer} \)

These are the negative and positive counting numbers (..., -3, -2, -1, 0, 1, 2, 3, ...)
\item \( \text{Irrational} \)

These cannot be written as a fraction of Integers.
\item \( \text{Whole} \)

* This is the correct option!
\item \( \text{Rational} \)

These are numbers that can be written as fraction of Integers (e.g., -2/3)
\item \( \text{Not a Real number} \)

These are Nonreal Complex numbers \textbf{OR} things that are not numbers (e.g., dividing by 0).
\end{enumerate}

\textbf{General Comment:} First, you \textbf{NEED} to simplify the expression. This question simplifies to $114$. 
 
 Be sure you look at the simplified fraction and not just the decimal expansion. Numbers such as 13, 17, and 19 provide \textbf{long but repeating/terminating decimal expansions!} 
 
 The only ways to *not* be a Real number are: dividing by 0 or taking the square root of a negative number. 
 
 Irrational numbers are more than just square root of 3: adding or subtracting values from square root of 3 is also irrational.
}
\litem{
Simplify the expression below into the form $a+bi$. Then, choose the intervals that $a$ and $b$ belong to.
\[ (7 + 3 i)(-4 + 6 i) \]The solution is \( -46 + 30 i \), which is option A.\begin{enumerate}[label=\Alph*.]
\item \( a \in [-50, -41] \text{ and } b \in [29, 31] \)

* $-46 + 30 i$, which is the correct option.
\item \( a \in [-50, -41] \text{ and } b \in [-33, -29] \)

 $-46 - 30 i$, which corresponds to adding a minus sign in both terms.
\item \( a \in [-10, -6] \text{ and } b \in [49, 57] \)

 $-10 + 54 i$, which corresponds to adding a minus sign in the first term.
\item \( a \in [-34, -24] \text{ and } b \in [15, 20] \)

 $-28 + 18 i$, which corresponds to just multiplying the real terms to get the real part of the solution and the coefficients in the complex terms to get the complex part.
\item \( a \in [-10, -6] \text{ and } b \in [-56, -50] \)

 $-10 - 54 i$, which corresponds to adding a minus sign in the second term.
\end{enumerate}

\textbf{General Comment:} You can treat $i$ as a variable and distribute. Just remember that $i^2=-1$, so you can continue to reduce after you distribute.
}
\litem{
Choose the \textbf{smallest} set of Real numbers that the number below belongs to.
\[ \sqrt{\frac{324}{361}} \]The solution is \( \text{Rational} \), which is option A.\begin{enumerate}[label=\Alph*.]
\item \( \text{Rational} \)

* This is the correct option!
\item \( \text{Not a Real number} \)

These are Nonreal Complex numbers \textbf{OR} things that are not numbers (e.g., dividing by 0).
\item \( \text{Integer} \)

These are the negative and positive counting numbers (..., -3, -2, -1, 0, 1, 2, 3, ...)
\item \( \text{Irrational} \)

These cannot be written as a fraction of Integers.
\item \( \text{Whole} \)

These are the counting numbers with 0 (0, 1, 2, 3, ...)
\end{enumerate}

\textbf{General Comment:} First, you \textbf{NEED} to simplify the expression. This question simplifies to $\frac{18}{19}$. 
 
 Be sure you look at the simplified fraction and not just the decimal expansion. Numbers such as 13, 17, and 19 provide \textbf{long but repeating/terminating decimal expansions!} 
 
 The only ways to *not* be a Real number are: dividing by 0 or taking the square root of a negative number. 
 
 Irrational numbers are more than just square root of 3: adding or subtracting values from square root of 3 is also irrational.
}
\end{enumerate}

\end{document}