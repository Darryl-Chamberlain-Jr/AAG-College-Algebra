\documentclass[14pt]{extbook}
\usepackage{multicol, enumerate, enumitem, hyperref, color, soul, setspace, parskip, fancyhdr} %General Packages
\usepackage{amssymb, amsthm, amsmath, latexsym, units, mathtools} %Math Packages
\everymath{\displaystyle} %All math in Display Style
% Packages with additional options
\usepackage[headsep=0.5cm,headheight=12pt, left=1 in,right= 1 in,top= 1 in,bottom= 1 in]{geometry}
\usepackage[usenames,dvipsnames]{xcolor}
\usepackage{dashrule}  % Package to use the command below to create lines between items
\newcommand{\litem}[1]{\item#1\hspace*{-1cm}\rule{\textwidth}{0.4pt}}
\pagestyle{fancy}
\lhead{Module1}
\chead{}
\rhead{Version C}
\lfoot{6227-9062}
\cfoot{}
\rfoot{testing}
\begin{document}

\begin{enumerate}
\litem{
Choose the \textbf{smallest} set of Complex numbers that the number below belongs to.\[ \sqrt{\frac{-1404}{0}}+\sqrt{126} \]\begin{enumerate}[label=\Alph*.]
\item \( \text{Rational} \)
\item \( \text{Irrational} \)
\item \( \text{Pure Imaginary} \)
\item \( \text{Nonreal Complex} \)
\item \( \text{Not a Complex Number} \)

\end{enumerate} }
\litem{
Simplify the expression below into the form $a+bi$. Then, choose the intervals that $a$ and $b$ belong to.\[ \frac{63 - 33 i}{-5 - i} \]\begin{enumerate}[label=\Alph*.]
\item \( a \in [-13.45, -13.3] \text{ and } b \in [3.5, 5] \)
\item \( a \in [-13.15, -11.9] \text{ and } b \in [32.5, 33.5] \)
\item \( a \in [-11.45, -10.5] \text{ and } b \in [8.5, 10.5] \)
\item \( a \in [-11.45, -10.5] \text{ and } b \in [227.5, 229.5] \)
\item \( a \in [-282.1, -281.3] \text{ and } b \in [8.5, 10.5] \)

\end{enumerate} }
\litem{
Simplify the expression below into the form $a+bi$. Then, choose the intervals that $a$ and $b$ belong to.\[ (-5 - 2 i)(-6 - 8 i) \]\begin{enumerate}[label=\Alph*.]
\item \( a \in [44, 52] \text{ and } b \in [-32, -21] \)
\item \( a \in [44, 52] \text{ and } b \in [28, 30] \)
\item \( a \in [13, 20] \text{ and } b \in [48, 57] \)
\item \( a \in [13, 20] \text{ and } b \in [-52, -50] \)
\item \( a \in [27, 33] \text{ and } b \in [16, 18] \)

\end{enumerate} }
\litem{
Simplify the expression below into the form $a+bi$. Then, choose the intervals that $a$ and $b$ belong to.\[ \frac{9 + 55 i}{6 + 2 i} \]\begin{enumerate}[label=\Alph*.]
\item \( a \in [163.5, 164.5] \text{ and } b \in [6.5, 8] \)
\item \( a \in [3.5, 4.5] \text{ and } b \in [6.5, 8] \)
\item \( a \in [1, 2.5] \text{ and } b \in [26, 28.5] \)
\item \( a \in [-2.5, 0] \text{ and } b \in [8.5, 9] \)
\item \( a \in [3.5, 4.5] \text{ and } b \in [311.5, 312.5] \)

\end{enumerate} }
\litem{
Simplify the expression below and choose the interval the simplification is contained within.\[ 18 - 20 \div 1 * 17 - (3 * 4) \]\begin{enumerate}[label=\Alph*.]
\item \( [26.82, 31.82] \)
\item \( [-339, -333] \)
\item \( [-1302, -1298] \)
\item \( [2.82, 5.82] \)
\item \( \text{None of the above} \)

\end{enumerate} }
\litem{
Choose the \textbf{smallest} set of Real numbers that the number below belongs to.\[ \sqrt{\frac{540}{12}} \]\begin{enumerate}[label=\Alph*.]
\item \( \text{Whole} \)
\item \( \text{Irrational} \)
\item \( \text{Not a Real number} \)
\item \( \text{Rational} \)
\item \( \text{Integer} \)

\end{enumerate} }
\litem{
Choose the \textbf{smallest} set of Complex numbers that the number below belongs to.\[ \sqrt{\frac{-693}{7}}+\sqrt{0}i \]\begin{enumerate}[label=\Alph*.]
\item \( \text{Pure Imaginary} \)
\item \( \text{Irrational} \)
\item \( \text{Not a Complex Number} \)
\item \( \text{Rational} \)
\item \( \text{Nonreal Complex} \)

\end{enumerate} }
\litem{
Simplify the expression below into the form $a+bi$. Then, choose the intervals that $a$ and $b$ belong to.\[ (-7 - 3 i)(4 - 2 i) \]\begin{enumerate}[label=\Alph*.]
\item \( a \in [-32, -26] \text{ and } b \in [4, 8.1] \)
\item \( a \in [-36, -29] \text{ and } b \in [-3.7, -1.2] \)
\item \( a \in [-27, -20] \text{ and } b \in [-26.7, -24.5] \)
\item \( a \in [-36, -29] \text{ and } b \in [-1.3, 3.1] \)
\item \( a \in [-27, -20] \text{ and } b \in [23.8, 27] \)

\end{enumerate} }
\litem{
Choose the \textbf{smallest} set of Real numbers that the number below belongs to.\[ \sqrt{\frac{196}{169}} \]\begin{enumerate}[label=\Alph*.]
\item \( \text{Irrational} \)
\item \( \text{Integer} \)
\item \( \text{Whole} \)
\item \( \text{Rational} \)
\item \( \text{Not a Real number} \)

\end{enumerate} }
\litem{
Simplify the expression below and choose the interval the simplification is contained within.\[ 1 - 17 \div 4 * 16 - (10 * 12) \]\begin{enumerate}[label=\Alph*.]
\item \( [-121.27, -118.27] \)
\item \( [114.73, 123.73] \)
\item \( [-926, -920] \)
\item \( [-188, -182] \)
\item \( \text{None of the above} \)

\end{enumerate} }
\end{enumerate}

\end{document}