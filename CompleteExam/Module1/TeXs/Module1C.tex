\documentclass[14pt]{extbook}
\usepackage{multicol, enumerate, enumitem, hyperref, color, soul, setspace, parskip, fancyhdr} %General Packages
\usepackage{amssymb, amsthm, amsmath, latexsym, units, mathtools} %Math Packages
\everymath{\displaystyle} %All math in Display Style
% Packages with additional options
\usepackage[headsep=0.5cm,headheight=12pt, left=1 in,right= 1 in,top= 1 in,bottom= 1 in]{geometry}
\usepackage[usenames,dvipsnames]{xcolor}
\usepackage{dashrule}  % Package to use the command below to create lines between items
\newcommand{\litem}[1]{\item#1\hspace*{-1cm}\rule{\textwidth}{0.4pt}}
\pagestyle{fancy}
\lhead{Module1}
\chead{}
\rhead{Version C}
\lfoot{8000-1344}
\cfoot{}
\rfoot{testing}
\begin{document}

\begin{enumerate}
\litem{
Choose the \textbf{smallest} set of Real numbers that the number below belongs to.\[ \sqrt{\frac{78400}{400}} \]\begin{enumerate}[label=\Alph*.]
\item \( \text{Rational} \)
\item \( \text{Integer} \)
\item \( \text{Irrational} \)
\item \( \text{Not a Real number} \)
\item \( \text{Whole} \)

\end{enumerate} }
\litem{
Simplify the expression below into the form $a+bi$. Then, choose the intervals that $a$ and $b$ belong to.\[ (7 - 2 i)(-10 - 3 i) \]\begin{enumerate}[label=\Alph*.]
\item \( a \in [-74, -69] \text{ and } b \in [5.7, 6.6] \)
\item \( a \in [-80, -75] \text{ and } b \in [0.8, 1.2] \)
\item \( a \in [-67, -60] \text{ and } b \in [39.9, 41.6] \)
\item \( a \in [-67, -60] \text{ and } b \in [-41.8, -39.7] \)
\item \( a \in [-80, -75] \text{ and } b \in [-1.1, 0.5] \)

\end{enumerate} }
\litem{
Simplify the expression below into the form $a+bi$. Then, choose the intervals that $a$ and $b$ belong to.\[ \frac{-54 + 11 i}{-7 - 3 i} \]\begin{enumerate}[label=\Alph*.]
\item \( a \in [344.6, 345.3] \text{ and } b \in [-4.27, -4.02] \)
\item \( a \in [7.55, 8.5] \text{ and } b \in [-3.81, -3.6] \)
\item \( a \in [5.45, 6.45] \text{ and } b \in [-4.27, -4.02] \)
\item \( a \in [5.45, 6.45] \text{ and } b \in [-239.1, -238.97] \)
\item \( a \in [6.65, 7.15] \text{ and } b \in [1.43, 1.6] \)

\end{enumerate} }
\litem{
Simplify the expression below and choose the interval the simplification is contained within.\[ 12 - 3 \div 8 * 20 - (4 * 13) \]\begin{enumerate}[label=\Alph*.]
\item \( [-48.5, -45.5] \)
\item \( [3.5, 8.5] \)
\item \( [-41.02, -35.02] \)
\item \( [60.98, 64.98] \)
\item \( \text{None of the above} \)

\end{enumerate} }
\litem{
Choose the \textbf{smallest} set of Complex numbers that the number below belongs to.\[ -\sqrt{\frac{400}{289}} + 25i^2 \]\begin{enumerate}[label=\Alph*.]
\item \( \text{Irrational} \)
\item \( \text{Nonreal Complex} \)
\item \( \text{Not a Complex Number} \)
\item \( \text{Pure Imaginary} \)
\item \( \text{Rational} \)

\end{enumerate} }
\litem{
Simplify the expression below into the form $a+bi$. Then, choose the intervals that $a$ and $b$ belong to.\[ \frac{18 - 77 i}{-5 - i} \]\begin{enumerate}[label=\Alph*.]
\item \( a \in [-6.5, -5.5] \text{ and } b \in [13.5, 15] \)
\item \( a \in [-5.5, -2] \text{ and } b \in [76, 78.5] \)
\item \( a \in [-1.5, 0.5] \text{ and } b \in [402, 403.5] \)
\item \( a \in [-13.5, -12.5] \text{ and } b \in [15, 17] \)
\item \( a \in [-1.5, 0.5] \text{ and } b \in [15, 17] \)

\end{enumerate} }
\litem{
Choose the \textbf{smallest} set of Real numbers that the number below belongs to.\[ \sqrt{\frac{32400}{81}} \]\begin{enumerate}[label=\Alph*.]
\item \( \text{Whole} \)
\item \( \text{Rational} \)
\item \( \text{Integer} \)
\item \( \text{Irrational} \)
\item \( \text{Not a Real number} \)

\end{enumerate} }
\litem{
Simplify the expression below into the form $a+bi$. Then, choose the intervals that $a$ and $b$ belong to.\[ (-4 - 5 i)(-7 - 6 i) \]\begin{enumerate}[label=\Alph*.]
\item \( a \in [-2, -1] \text{ and } b \in [57, 62] \)
\item \( a \in [-2, -1] \text{ and } b \in [-60, -57] \)
\item \( a \in [57, 63] \text{ and } b \in [-12, -1] \)
\item \( a \in [57, 63] \text{ and } b \in [8, 16] \)
\item \( a \in [25, 34] \text{ and } b \in [29, 35] \)

\end{enumerate} }
\litem{
Choose the \textbf{smallest} set of Complex numbers that the number below belongs to.\[ \sqrt{\frac{121}{0}}+\sqrt{117} i \]\begin{enumerate}[label=\Alph*.]
\item \( \text{Pure Imaginary} \)
\item \( \text{Nonreal Complex} \)
\item \( \text{Rational} \)
\item \( \text{Irrational} \)
\item \( \text{Not a Complex Number} \)

\end{enumerate} }
\litem{
Simplify the expression below and choose the interval the simplification is contained within.\[ 4 - 8^2 + 14 \div 13 * 19 \div 6 \]\begin{enumerate}[label=\Alph*.]
\item \( [71.41, 76.41] \)
\item \( [-61.99, -56.99] \)
\item \( [64.01, 70.01] \)
\item \( [-58.59, -53.59] \)
\item \( \text{None of the above} \)

\end{enumerate} }
\end{enumerate}

\end{document}