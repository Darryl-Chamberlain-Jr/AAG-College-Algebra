\documentclass{extbook}[14pt]
\usepackage{multicol, enumerate, enumitem, hyperref, color, soul, setspace, parskip, fancyhdr, amssymb, amsthm, amsmath, latexsym, units, mathtools}
\everymath{\displaystyle}
\usepackage[headsep=0.5cm,headheight=0cm, left=1 in,right= 1 in,top= 1 in,bottom= 1 in]{geometry}
\usepackage{dashrule}  % Package to use the command below to create lines between items
\newcommand{\litem}[1]{\item #1

\rule{\textwidth}{0.4pt}}
\pagestyle{fancy}
\lhead{}
\chead{Answer Key for Progress Quiz 6 Version B}
\rhead{}
\lfoot{9689-6866}
\cfoot{}
\rfoot{Spring 2021}
\begin{document}
\textbf{This key should allow you to understand why you choose the option you did (beyond just getting a question right or wrong). \href{https://xronos.clas.ufl.edu/mac1105spring2020/courseDescriptionAndMisc/Exams/LearningFromResults}{More instructions on how to use this key can be found here}.}

\textbf{If you have a suggestion to make the keys better, \href{https://forms.gle/CZkbZmPbC9XALEE88}{please fill out the short survey here}.}

\textit{Note: This key is auto-generated and may contain issues and/or errors. The keys are reviewed after each exam to ensure grading is done accurately. If there are issues (like duplicate options), they are noted in the offline gradebook. The keys are a work-in-progress to give students as many resources to improve as possible.}

\rule{\textwidth}{0.4pt}

\begin{enumerate}\litem{
Perform the division below. Then, find the intervals that correspond to the quotient in the form $ax^2+bx+c$ and remainder $r$.
\[ \frac{9x^{3} -72 x^{2} +155 x -104}{x -5} \]The solution is \( 9x^{2} -27 x + 20 + \frac{-4}{x -5} \), which is option B.\begin{enumerate}[label=\Alph*.]
\item \( a \in [40, 48], \text{   } b \in [150, 154], \text{   } c \in [918, 926], \text{   and   } r \in [4488, 4500]. \)

 You multiplied by the synthetic number rather than bringing the first factor down.
\item \( a \in [7, 12], \text{   } b \in [-30, -19], \text{   } c \in [17, 23], \text{   and   } r \in [-6, -1]. \)

* This is the solution!
\item \( a \in [7, 12], \text{   } b \in [-117, -112], \text{   } c \in [735, 743], \text{   and   } r \in [-3804, -3797]. \)

 You divided by the opposite of the factor.
\item \( a \in [7, 12], \text{   } b \in [-37, -33], \text{   } c \in [8, 16], \text{   and   } r \in [-64, -57]. \)

 You multiplied by the synthetic number and subtracted rather than adding during synthetic division.
\item \( a \in [40, 48], \text{   } b \in [-299, -294], \text{   } c \in [1635, 1650], \text{   and   } r \in [-8308, -8300]. \)

 You divided by the opposite of the factor AND multiplied the first factor rather than just bringing it down.
\end{enumerate}

\textbf{General Comment:} Be sure to synthetically divide by the zero of the denominator!
}
\litem{
Factor the polynomial below completely. Then, choose the intervals the zeros of the polynomial belong to, where $z_1 \leq z_2 \leq z_3$. \textit{To make the problem easier, all zeros are between -5 and 5.}
\[ f(x) = 9x^{3} -39 x^{2} -38 x + 40 \]The solution is \( [-1.3333333333333333, 0.6666666666666666, 5] \), which is option C.\begin{enumerate}[label=\Alph*.]
\item \( z_1 \in [-5.47, -4.95], \text{   }  z_2 \in [-1.65, -0.97], \text{   and   } z_3 \in [0.61, 1.1] \)

 Distractor 3: Corresponds to negatives of all zeros AND inversing rational roots.
\item \( z_1 \in [-1.17, -0.63], \text{   }  z_2 \in [1.41, 1.74], \text{   and   } z_3 \in [4.72, 5.11] \)

 Distractor 2: Corresponds to inversing rational roots.
\item \( z_1 \in [-1.67, -0.77], \text{   }  z_2 \in [0.66, 1.21], \text{   and   } z_3 \in [4.72, 5.11] \)

* This is the solution!
\item \( z_1 \in [-5.47, -4.95], \text{   }  z_2 \in [-2.01, -1.92], \text{   and   } z_3 \in [0.16, 0.7] \)

 Distractor 4: Corresponds to moving factors from one rational to another.
\item \( z_1 \in [-5.47, -4.95], \text{   }  z_2 \in [-0.89, -0.18], \text{   and   } z_3 \in [1.32, 1.46] \)

 Distractor 1: Corresponds to negatives of all zeros.
\end{enumerate}

\textbf{General Comment:} Remember to try the middle-most integers first as these normally are the zeros. Also, once you get it to a quadratic, you can use your other factoring techniques to finish factoring.
}
\litem{
Factor the polynomial below completely, knowing that $x-4$ is a factor. Then, choose the intervals the zeros of the polynomial belong to, where $z_1 \leq z_2 \leq z_3 \leq z_4$. \textit{To make the problem easier, all zeros are between -5 and 5.}
\[ f(x) = 25x^{4} -150 x^{3} +191 x^{2} +54 x -72 \]The solution is \( [-0.6, 0.6, 2, 4] \), which is option D.\begin{enumerate}[label=\Alph*.]
\item \( z_1 \in [-2.34, -0.86], \text{   }  z_2 \in [0.67, 4.67], z_3 \in [1.66, 2.34], \text{   and   } z_4 \in [3.84, 4.71] \)

 Distractor 2: Corresponds to inversing rational roots.
\item \( z_1 \in [-4.09, -3.29], \text{   }  z_2 \in [-4, 0], z_3 \in [-1.03, -0.55], \text{   and   } z_4 \in [0.3, 1.01] \)

 Distractor 1: Corresponds to negatives of all zeros.
\item \( z_1 \in [-4.09, -3.29], \text{   }  z_2 \in [-4, 0], z_3 \in [-0.23, -0.09], \text{   and   } z_4 \in [2.81, 3.08] \)

 Distractor 4: Corresponds to moving factors from one rational to another.
\item \( z_1 \in [-0.88, -0.58], \text{   }  z_2 \in [0.6, 1.6], z_3 \in [1.66, 2.34], \text{   and   } z_4 \in [3.84, 4.71] \)

* This is the solution!
\item \( z_1 \in [-4.09, -3.29], \text{   }  z_2 \in [-4, 0], z_3 \in [-1.68, -1.55], \text{   and   } z_4 \in [1.61, 2.09] \)

 Distractor 3: Corresponds to negatives of all zeros AND inversing rational roots.
\end{enumerate}

\textbf{General Comment:} Remember to try the middle-most integers first as these normally are the zeros. Also, once you get it to a quadratic, you can use your other factoring techniques to finish factoring.
}
\litem{
Factor the polynomial below completely, knowing that $x+4$ is a factor. Then, choose the intervals the zeros of the polynomial belong to, where $z_1 \leq z_2 \leq z_3 \leq z_4$. \textit{To make the problem easier, all zeros are between -5 and 5.}
\[ f(x) = 20x^{4} +63 x^{3} -108 x^{2} -172 x -48 \]The solution is \( [-4, -0.75, -0.4, 2] \), which is option A.\begin{enumerate}[label=\Alph*.]
\item \( z_1 \in [-5.3, -3.9], \text{   }  z_2 \in [-0.9, -0.58], z_3 \in [-0.54, -0.22], \text{   and   } z_4 \in [1.4, 2.8] \)

* This is the solution!
\item \( z_1 \in [-2.1, -1.8], \text{   }  z_2 \in [1.27, 1.54], z_3 \in [2.28, 2.52], \text{   and   } z_4 \in [3.9, 4.6] \)

 Distractor 3: Corresponds to negatives of all zeros AND inversing rational roots.
\item \( z_1 \in [-5.3, -3.9], \text{   }  z_2 \in [-2.97, -2.47], z_3 \in [-1.51, -1.26], \text{   and   } z_4 \in [1.4, 2.8] \)

 Distractor 2: Corresponds to inversing rational roots.
\item \( z_1 \in [-2.1, -1.8], \text{   }  z_2 \in [-0.44, 0.14], z_3 \in [2.54, 3.41], \text{   and   } z_4 \in [3.9, 4.6] \)

 Distractor 4: Corresponds to moving factors from one rational to another.
\item \( z_1 \in [-2.1, -1.8], \text{   }  z_2 \in [0.31, 0.94], z_3 \in [0, 0.8], \text{   and   } z_4 \in [3.9, 4.6] \)

 Distractor 1: Corresponds to negatives of all zeros.
\end{enumerate}

\textbf{General Comment:} Remember to try the middle-most integers first as these normally are the zeros. Also, once you get it to a quadratic, you can use your other factoring techniques to finish factoring.
}
\litem{
What are the \textit{possible Rational} roots of the polynomial below?
\[ f(x) = 4x^{2} +6 x + 6 \]The solution is \( \text{ All combinations of: }\frac{\pm 1,\pm 2,\pm 3,\pm 6}{\pm 1,\pm 2,\pm 4} \), which is option D.\begin{enumerate}[label=\Alph*.]
\item \( \pm 1,\pm 2,\pm 4 \)

 Distractor 1: Corresponds to the plus or minus factors of a1 only.
\item \( \text{ All combinations of: }\frac{\pm 1,\pm 2,\pm 4}{\pm 1,\pm 2,\pm 3,\pm 6} \)

 Distractor 3: Corresponds to the plus or minus of the inverse quotient (an/a0) of the factors. 
\item \( \pm 1,\pm 2,\pm 3,\pm 6 \)

This would have been the solution \textbf{if asked for the possible Integer roots}!
\item \( \text{ All combinations of: }\frac{\pm 1,\pm 2,\pm 3,\pm 6}{\pm 1,\pm 2,\pm 4} \)

* This is the solution \textbf{since we asked for the possible Rational roots}!
\item \( \text{ There is no formula or theorem that tells us all possible Rational roots.} \)

 Distractor 4: Corresponds to not recalling the theorem for rational roots of a polynomial.
\end{enumerate}

\textbf{General Comment:} We have a way to find the possible Rational roots. The possible Integer roots are the Integers in this list.
}
\litem{
Factor the polynomial below completely. Then, choose the intervals the zeros of the polynomial belong to, where $z_1 \leq z_2 \leq z_3$. \textit{To make the problem easier, all zeros are between -5 and 5.}
\[ f(x) = 16x^{3} +32 x^{2} -25 x -50 \]The solution is \( [-2, -1.25, 1.25] \), which is option B.\begin{enumerate}[label=\Alph*.]
\item \( z_1 \in [-2.09, -1.86], \text{   }  z_2 \in [-1.06, -0.54], \text{   and   } z_3 \in [0.67, 1.19] \)

 Distractor 2: Corresponds to inversing rational roots.
\item \( z_1 \in [-2.09, -1.86], \text{   }  z_2 \in [-1.28, -1.11], \text{   and   } z_3 \in [1.23, 1.9] \)

* This is the solution!
\item \( z_1 \in [-1.27, -0.82], \text{   }  z_2 \in [1.14, 1.32], \text{   and   } z_3 \in [1.73, 3.34] \)

 Distractor 1: Corresponds to negatives of all zeros.
\item \( z_1 \in [-5.26, -4.5], \text{   }  z_2 \in [0.09, 0.34], \text{   and   } z_3 \in [1.73, 3.34] \)

 Distractor 4: Corresponds to moving factors from one rational to another.
\item \( z_1 \in [-0.87, 0.04], \text{   }  z_2 \in [0.5, 0.94], \text{   and   } z_3 \in [1.73, 3.34] \)

 Distractor 3: Corresponds to negatives of all zeros AND inversing rational roots.
\end{enumerate}

\textbf{General Comment:} Remember to try the middle-most integers first as these normally are the zeros. Also, once you get it to a quadratic, you can use your other factoring techniques to finish factoring.
}
\litem{
Perform the division below. Then, find the intervals that correspond to the quotient in the form $ax^2+bx+c$ and remainder $r$.
\[ \frac{8x^{3} +20 x^{2} -56 x -27}{x + 4} \]The solution is \( 8x^{2} -12 x -8 + \frac{5}{x + 4} \), which is option C.\begin{enumerate}[label=\Alph*.]
\item \( a \in [5, 14], \text{   } b \in [43, 58], \text{   } c \in [152, 156], \text{   and   } r \in [579, 585]. \)

 You divided by the opposite of the factor.
\item \( a \in [-35, -28], \text{   } b \in [148, 153], \text{   } c \in [-652, -645], \text{   and   } r \in [2558, 2573]. \)

 You multiplied by the synthetic number rather than bringing the first factor down.
\item \( a \in [5, 14], \text{   } b \in [-18, -7], \text{   } c \in [-12, 0], \text{   and   } r \in [0, 7]. \)

* This is the solution!
\item \( a \in [5, 14], \text{   } b \in [-24, -18], \text{   } c \in [43, 45], \text{   and   } r \in [-249, -246]. \)

 You multiplied by the synthetic number and subtracted rather than adding during synthetic division.
\item \( a \in [-35, -28], \text{   } b \in [-109, -105], \text{   } c \in [-495, -485], \text{   and   } r \in [-1984, -1978]. \)

 You divided by the opposite of the factor AND multiplied the first factor rather than just bringing it down.
\end{enumerate}

\textbf{General Comment:} Be sure to synthetically divide by the zero of the denominator!
}
\litem{
What are the \textit{possible Rational} roots of the polynomial below?
\[ f(x) = 7x^{3} +2 x^{2} +2 x + 6 \]The solution is \( \text{ All combinations of: }\frac{\pm 1,\pm 2,\pm 3,\pm 6}{\pm 1,\pm 7} \), which is option C.\begin{enumerate}[label=\Alph*.]
\item \( \pm 1,\pm 2,\pm 3,\pm 6 \)

This would have been the solution \textbf{if asked for the possible Integer roots}!
\item \( \text{ All combinations of: }\frac{\pm 1,\pm 7}{\pm 1,\pm 2,\pm 3,\pm 6} \)

 Distractor 3: Corresponds to the plus or minus of the inverse quotient (an/a0) of the factors. 
\item \( \text{ All combinations of: }\frac{\pm 1,\pm 2,\pm 3,\pm 6}{\pm 1,\pm 7} \)

* This is the solution \textbf{since we asked for the possible Rational roots}!
\item \( \pm 1,\pm 7 \)

 Distractor 1: Corresponds to the plus or minus factors of a1 only.
\item \( \text{ There is no formula or theorem that tells us all possible Rational roots.} \)

 Distractor 4: Corresponds to not recalling the theorem for rational roots of a polynomial.
\end{enumerate}

\textbf{General Comment:} We have a way to find the possible Rational roots. The possible Integer roots are the Integers in this list.
}
\litem{
Perform the division below. Then, find the intervals that correspond to the quotient in the form $ax^2+bx+c$ and remainder $r$.
\[ \frac{4x^{3} +12 x^{2} -20}{x + 2} \]The solution is \( 4x^{2} +4 x -8 + \frac{-4}{x + 2} \), which is option E.\begin{enumerate}[label=\Alph*.]
\item \( a \in [-10, -7], b \in [-5.6, -3.6], c \in [-15, -4], \text{ and } r \in [-36, -31]. \)

 You divided by the opposite of the factor AND multipled the first factor rather than just bringing it down.
\item \( a \in [3, 13], b \in [-1.5, 0.5], c \in [-2, 3], \text{ and } r \in [-20, -16]. \)

 You multipled by the synthetic number and subtracted rather than adding during synthetic division.
\item \( a \in [3, 13], b \in [18.9, 21.4], c \in [39, 43], \text{ and } r \in [57, 61]. \)

 You divided by the opposite of the factor.
\item \( a \in [-10, -7], b \in [27.9, 29.1], c \in [-60, -55], \text{ and } r \in [90, 97]. \)

 You multipled by the synthetic number rather than bringing the first factor down.
\item \( a \in [3, 13], b \in [3.1, 4.1], c \in [-15, -4], \text{ and } r \in [-6, -1]. \)

* This is the solution!
\end{enumerate}

\textbf{General Comment:} Be sure to synthetically divide by the zero of the denominator! Also, make sure to include 0 placeholders for missing terms.
}
\litem{
Perform the division below. Then, find the intervals that correspond to the quotient in the form $ax^2+bx+c$ and remainder $r$.
\[ \frac{20x^{3} +63 x^{2} -31}{x + 3} \]The solution is \( 20x^{2} +3 x -9 + \frac{-4}{x + 3} \), which is option A.\begin{enumerate}[label=\Alph*.]
\item \( a \in [18, 23], b \in [2, 9], c \in [-9, -8], \text{ and } r \in [-5, -2]. \)

* This is the solution!
\item \( a \in [-61, -54], b \in [239, 251], c \in [-733, -721], \text{ and } r \in [2154, 2161]. \)

 You multipled by the synthetic number rather than bringing the first factor down.
\item \( a \in [-61, -54], b \in [-118, -115], c \in [-351, -349], \text{ and } r \in [-1084, -1082]. \)

 You divided by the opposite of the factor AND multipled the first factor rather than just bringing it down.
\item \( a \in [18, 23], b \in [122, 127], c \in [366, 371], \text{ and } r \in [1073, 1078]. \)

 You divided by the opposite of the factor.
\item \( a \in [18, 23], b \in [-22, -16], c \in [68, 70], \text{ and } r \in [-306, -300]. \)

 You multipled by the synthetic number and subtracted rather than adding during synthetic division.
\end{enumerate}

\textbf{General Comment:} Be sure to synthetically divide by the zero of the denominator! Also, make sure to include 0 placeholders for missing terms.
}
\end{enumerate}

\end{document}