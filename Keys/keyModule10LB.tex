\documentclass{extbook}[14pt]
\usepackage{multicol, enumerate, enumitem, hyperref, color, soul, setspace, parskip, fancyhdr, amssymb, amsthm, amsmath, bbm, latexsym, units, mathtools}
\everymath{\displaystyle}
\usepackage[headsep=0.5cm,headheight=0cm, left=1 in,right= 1 in,top= 1 in,bottom= 1 in]{geometry}
\pagestyle{fancy}
\lhead{}
\chead{Answer Key for Module\,10L\,-\,Synthetic\,Division Version B}
\rhead{}
\lfoot{Summer\,C\,2020}
\cfoot{}
\rfoot{}
\begin{document}
\textbf{This key should allow you to understand why you choose the option you did (beyond just getting a question right or wrong). \href{https://xronos.clas.ufl.edu/mac1105spring2020/courseDescriptionAndMisc/Exams/LearningFromResults}{More instructions on how to use this key can be found here}.}

\textbf{If you have a suggestion to make the keys better, \href{https://forms.gle/CZkbZmPbC9XALEE88}{please fill out the short survey here}.}

\textit{Note: This key is auto-generated and may contain issues and/or errors. The keys are reviewed after each exam to ensure grading is done accurately. If there are issues (like duplicate options), they are noted in the offline gradebook. The keys are a work-in-progress to give students as many resources to improve as possible.}

\rule{\textwidth}{0.4pt}

66. Perform the division below. Then, find the intervals that correspond to the quotient in the form $ax^2+bx+c$ and remainder $r$.
\[ \frac{4x^{3} -12 x + 5}{x + 2} \] 
The solution is $ 4x^{2} -8 x + 4 + \frac{-3}{x + 2} $ 

\begin{enumerate}[label=\Alph*.] 
\item $ a \in [0, 5], b \in [3, 9], c \in [1, 7], \text{ and } r \in [11, 17]. $ 

  You divided by the opposite of the factor. 
\item $ a \in [-14, -6], b \in [15, 20], c \in [-48, -39], \text{ and } r \in [91, 94]. $ 

  You multipled by the synthetic number rather than bringing the first factor down. 
\item $ a \in [0, 5], b \in [-10, -2], c \in [1, 7], \text{ and } r \in [-7, -1]. $ 

 * This is the solution! 
\item $ a \in [-14, -6], b \in [-18, -13], c \in [-48, -39], \text{ and } r \in [-84, -78]. $ 

  You divided by the opposite of the factor AND multipled the first factor rather than just bringing it down. 
\item $ a \in [0, 5], b \in [-13, -10], c \in [23, 26], \text{ and } r \in [-69, -66]. $ 

  You multipled by the synthetic number and subtracted rather than adding during synthetic division. 
\end{enumerate} 
 
General Comments: Be sure to synthetically divide by the zero of the denominator! Also, make sure to include 0 placeholders for missing terms.

-----------------------------------------------

67. Factor the polynomial below completely. Then, choose the intervals the zeros of the polynomial belong to, where $z_1 \leq z_2 \leq z_3$. \textit{To make the problem easier, all zeros are between -5 and 5.}
\[ f(x) = 6x^{3} -5 x^{2} -22 x + 24 \] 
The solution is $ [-2, 1.3333333333333333, 1.5] $ 

\begin{enumerate}[label=\Alph*.] 
\item $ z_1 \in [-2.82, -1.65], \text{   }  z_2 \in [-0.5, 0.9], \text{   and   } z_3 \in [0.25, 1.03] $ 

  Distractor 2: Corresponds to inversing rational roots. 
\item $ z_1 \in [-3.34, -2.54], \text{   }  z_2 \in [-1, -0.3], \text{   and   } z_3 \in [1.94, 2.41] $ 

  Distractor 4: Corresponds to moving factors from one rational to another. 
\item $ z_1 \in [-2.82, -1.65], \text{   }  z_2 \in [1.2, 2.2], \text{   and   } z_3 \in [1.45, 1.52] $ 

 * This is the solution! 
\item $ z_1 \in [-1.57, -1.04], \text{   }  z_2 \in [-1.9, -1.2], \text{   and   } z_3 \in [1.94, 2.41] $ 

  Distractor 1: Corresponds to negatives of all zeros. 
\item $ z_1 \in [-0.93, -0.6], \text{   }  z_2 \in [-1, -0.3], \text{   and   } z_3 \in [1.94, 2.41] $ 

  Distractor 3: Corresponds to negatives of all zeros AND inversing rational roots. 
\end{enumerate} 
 
General Comments: Remember to try the middle-most integers first as these normally are the zeros. Also, once you get it to a quadratic, you can use your other factoring techniques to finish factoring.

-----------------------------------------------

68. Perform the division below. Then, find the intervals that correspond to the quotient in the form $ax^2+bx+c$ and remainder $r$.
\[ \frac{6x^{3} +43 x^{2} +86 x + 38}{x + 4} \] 
The solution is $ 6x^{2} +19 x + 10 + \frac{-2}{x + 4} $ 

\begin{enumerate}[label=\Alph*.] 
\item $ a \in [2, 10], \text{   } b \in [17, 21], \text{   } c \in [8, 17], \text{   and   } r \in [-3, 0]. $ 

 * This is the solution! 
\item $ a \in [-33, -18], \text{   } b \in [134, 145], \text{   } c \in [-473, -467], \text{   and   } r \in [1917, 1919]. $ 

  You multiplied by the synthetic number rather than bringing the first factor down. 
\item $ a \in [2, 10], \text{   } b \in [65, 70], \text{   } c \in [349, 360], \text{   and   } r \in [1451, 1462]. $ 

  You divided by the opposite of the factor. 
\item $ a \in [2, 10], \text{   } b \in [11, 15], \text{   } c \in [19, 25], \text{   and   } r \in [-75, -65]. $ 

  You multiplied by the synthetic number and subtracted rather than adding during synthetic division. 
\item $ a \in [-33, -18], \text{   } b \in [-57, -51], \text{   } c \in [-131, -118], \text{   and   } r \in [-469, -465]. $ 

  You divided by the opposite of the factor AND multiplied the first factor rather than just bringing it down. 
\end{enumerate} 
 
General Comments: Be sure to synthetically divide by the zero of the denominator!

-----------------------------------------------

69. What are the \textit{possible Integer} roots of the polynomial below?
\[ f(x) = 2x^{4} +4 x^{3} +3 x^{2} +7 x + 7 \] 
The solution is $ \pm 1,\pm 7 $ 

\begin{enumerate}[label=\Alph*.] 
\item $ \text{ All combinations of: }\frac{\pm 1,\pm 7}{\pm 1,\pm 2} $ 

 This would have been the solution \textbf{if asked for the possible Rational roots}! 
\item $ \pm 1,\pm 2 $ 

  Distractor 1: Corresponds to the plus or minus factors of a1 only. 
\item $ \pm 1,\pm 7 $ 

 * This is the solution \textbf{since we asked for the possible Integer roots}! 
\item $ \text{ All combinations of: }\frac{\pm 1,\pm 2}{\pm 1,\pm 7} $ 

  Distractor 3: Corresponds to the plus or minus of the inverse quotient (an/a0) of the factors.  
\item $ \text{There is no formula or theorem that tells us all possible Integer roots.} $ 

  Distractor 4: Corresponds to not recognizing Integers as a subset of Rationals. 
\end{enumerate} 
 
General Comments: We have a way to find the possible Rational roots. The possible Integer roots are the Integers in this list.

-----------------------------------------------

70. Factor the polynomial below completely, knowing that $x+5$ is a factor. Then, choose the intervals the zeros of the polynomial belong to, where $z_1 \leq z_2 \leq z_3 \leq z_4$. \textit{To make the problem easier, all zeros are between -5 and 5.}
\[ f(x) = 10x^{4} +33 x^{3} -165 x^{2} -448 x -240 \] 
The solution is $ [-5, -1.5, -0.8, 4] $ 

\begin{enumerate}[label=\Alph*.] 
\item $ z_1 \in [-4.2, -1.2], \text{   }  z_2 \in [0.22, 0.43], z_3 \in [2.96, 3.09], \text{   and   } z_4 \in [4.04, 5.73] $ 

  Distractor 4: Corresponds to moving factors from one rational to another. 
\item $ z_1 \in [-7.3, -4.2], \text{   }  z_2 \in [-1.53, -1.46], z_3 \in [-0.94, -0.78], \text{   and   } z_4 \in [3.87, 4.67] $ 

 * This is the solution! 
\item $ z_1 \in [-4.2, -1.2], \text{   }  z_2 \in [0.47, 0.77], z_3 \in [0.9, 1.27], \text{   and   } z_4 \in [4.04, 5.73] $ 

  Distractor 3: Corresponds to negatives of all zeros AND inversing rational roots. 
\item $ z_1 \in [-7.3, -4.2], \text{   }  z_2 \in [-1.39, -1.22], z_3 \in [-0.72, -0.65], \text{   and   } z_4 \in [3.87, 4.67] $ 

  Distractor 2: Corresponds to inversing rational roots. 
\item $ z_1 \in [-4.2, -1.2], \text{   }  z_2 \in [0.74, 1], z_3 \in [1.36, 1.51], \text{   and   } z_4 \in [4.04, 5.73] $ 

  Distractor 1: Corresponds to negatives of all zeros. 
\end{enumerate} 
 
General Comments: Remember to try the middle-most integers first as these normally are the zeros. Also, once you get it to a quadratic, you can use your other factoring techniques to finish factoring.

-----------------------------------------------


\end{document}

