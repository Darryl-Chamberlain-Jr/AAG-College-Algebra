\documentclass{extbook}[14pt]
\usepackage{multicol, enumerate, enumitem, hyperref, color, soul, setspace, parskip, fancyhdr, amssymb, amsthm, amsmath, bbm, latexsym, units, mathtools}
\everymath{\displaystyle}
\usepackage[headsep=0.5cm,headheight=0cm, left=1 in,right= 1 in,top= 1 in,bottom= 1 in]{geometry}
\usepackage{dashrule}  % Package to use the command below to create lines between items
\newcommand{\litem}[1]{\item #1

\rule{\textwidth}{0.4pt}}
\pagestyle{fancy}
\lhead{}
\chead{Answer Key for Progress Quiz 9 Version A}
\rhead{}
\lfoot{8590-6105}
\cfoot{}
\rfoot{Fall 2020}
\begin{document}
\textbf{This key should allow you to understand why you choose the option you did (beyond just getting a question right or wrong). \href{https://xronos.clas.ufl.edu/mac1105spring2020/courseDescriptionAndMisc/Exams/LearningFromResults}{More instructions on how to use this key can be found here}.}

\textbf{If you have a suggestion to make the keys better, \href{https://forms.gle/CZkbZmPbC9XALEE88}{please fill out the short survey here}.}

\textit{Note: This key is auto-generated and may contain issues and/or errors. The keys are reviewed after each exam to ensure grading is done accurately. If there are issues (like duplicate options), they are noted in the offline gradebook. The keys are a work-in-progress to give students as many resources to improve as possible.}

\rule{\textwidth}{0.4pt}

\begin{enumerate}\litem{
Solve the linear inequality below. Then, choose the constant and interval combination that describes the solution set.
\[ -10x + 3 \leq -4x -9 \]

The solution is \( [2.0, \infty) \), which is option A.\begin{enumerate}[label=\Alph*.]
\item \( [a, \infty), \text{ where } a \in [0, 7] \)

* $[2.0, \infty)$, which is the correct option.
\item \( (-\infty, a], \text{ where } a \in [-7.8, 1.4] \)

 $(-\infty, -2.0]$, which corresponds to switching the direction of the interval AND negating the endpoint. You likely did this if you did not flip the inequality when dividing by a negative as well as not moving values over to a side properly.
\item \( (-\infty, a], \text{ where } a \in [1.2, 5.3] \)

 $(-\infty, 2.0]$, which corresponds to switching the direction of the interval. You likely did this if you did not flip the inequality when dividing by a negative!
\item \( [a, \infty), \text{ where } a \in [-5, -1] \)

 $[-2.0, \infty)$, which corresponds to negating the endpoint of the solution.
\item \( \text{None of the above}. \)

You may have chosen this if you thought the inequality did not match the ends of the intervals.
\end{enumerate}

\textbf{General Comment:} Remember that less/greater than or equal to includes the endpoint, while less/greater do not. Also, remember that you need to flip the inequality when you multiply or divide by a negative.
}
\litem{
Solve the linear inequality below. Then, choose the constant and interval combination that describes the solution set.
\[ -9x + 4 < 5x -8 \]

The solution is \( (0.857, \infty) \), which is option C.\begin{enumerate}[label=\Alph*.]
\item \( (a, \infty), \text{ where } a \in [-2.01, -0.06] \)

 $(-0.857, \infty)$, which corresponds to negating the endpoint of the solution.
\item \( (-\infty, a), \text{ where } a \in [0, 0.9] \)

 $(-\infty, 0.857)$, which corresponds to switching the direction of the interval. You likely did this if you did not flip the inequality when dividing by a negative!
\item \( (a, \infty), \text{ where } a \in [0.27, 1.63] \)

* $(0.857, \infty)$, which is the correct option.
\item \( (-\infty, a), \text{ where } a \in [-4.5, -0.5] \)

 $(-\infty, -0.857)$, which corresponds to switching the direction of the interval AND negating the endpoint. You likely did this if you did not flip the inequality when dividing by a negative as well as not moving values over to a side properly.
\item \( \text{None of the above}. \)

You may have chosen this if you thought the inequality did not match the ends of the intervals.
\end{enumerate}

\textbf{General Comment:} Remember that less/greater than or equal to includes the endpoint, while less/greater do not. Also, remember that you need to flip the inequality when you multiply or divide by a negative.
}
\litem{
Solve the linear inequality below. Then, choose the constant and interval combination that describes the solution set.
\[ -6 + 7 x > 8 x \text{ or } -5 + 3 x < 6 x \]

The solution is \( (-\infty, -6.0) \text{ or } (-1.667, \infty) \), which is option B.\begin{enumerate}[label=\Alph*.]
\item \( (-\infty, a) \cup (b, \infty), \text{ where } a \in [-2.33, 4.67] \text{ and } b \in [6, 7] \)

Corresponds to inverting the inequality and negating the solution.
\item \( (-\infty, a) \cup (b, \infty), \text{ where } a \in [-6, -1] \text{ and } b \in [-1.67, -0.67] \)

 * Correct option.
\item \( (-\infty, a] \cup [b, \infty), \text{ where } a \in [-9, -3] \text{ and } b \in [-8.67, 4.33] \)

Corresponds to including the endpoints (when they should be excluded).
\item \( (-\infty, a] \cup [b, \infty), \text{ where } a \in [0.67, 3.67] \text{ and } b \in [3, 9] \)

Corresponds to including the endpoints AND negating.
\item \( (-\infty, \infty) \)

Corresponds to the variable canceling, which does not happen in this instance.
\end{enumerate}

\textbf{General Comment:} When multiplying or dividing by a negative, flip the sign.
}
\litem{
Solve the linear inequality below. Then, choose the constant and interval combination that describes the solution set.
\[ 7 - 3 x < \frac{-9 x + 4}{4} \leq 8 - 3 x \]

The solution is \( (8.00, 9.33] \), which is option A.\begin{enumerate}[label=\Alph*.]
\item \( (a, b], \text{ where } a \in [7, 14] \text{ and } b \in [6.33, 11.33] \)

* $(8.00, 9.33]$, which is the correct option.
\item \( (-\infty, a) \cup [b, \infty), \text{ where } a \in [8, 13] \text{ and } b \in [7.33, 13.33] \)

$(-\infty, 8.00) \cup [9.33, \infty)$, which corresponds to displaying the and-inequality as an or-inequality.
\item \( [a, b), \text{ where } a \in [4, 13] \text{ and } b \in [5.33, 10.33] \)

$[8.00, 9.33)$, which corresponds to flipping the inequality.
\item \( (-\infty, a] \cup (b, \infty), \text{ where } a \in [-1, 12] \text{ and } b \in [2.33, 12.33] \)

$(-\infty, 8.00] \cup (9.33, \infty)$, which corresponds to displaying the and-inequality as an or-inequality AND flipping the inequality.
\item \( \text{None of the above.} \)


\end{enumerate}

\textbf{General Comment:} To solve, you will need to break up the compound inequality into two inequalities. Be sure to keep track of the inequality! It may be best to draw a number line and graph your solution.
}
\litem{
Using an interval or intervals, describe all the $x$-values within or including a distance of the given values.
\[ \text{ Less than } 7 \text{ units from the number } -2. \]

The solution is \( (-9, 5) \), which is option C.\begin{enumerate}[label=\Alph*.]
\item \( (-\infty, -9] \cup [5, \infty) \)

This describes the values no less than 7 from -2
\item \( [-9, 5] \)

This describes the values no more than 7 from -2
\item \( (-9, 5) \)

This describes the values less than 7 from -2
\item \( (-\infty, -9) \cup (5, \infty) \)

This describes the values more than 7 from -2
\item \( \text{None of the above} \)

You likely thought the values in the interval were not correct.
\end{enumerate}

\textbf{General Comment:} When thinking about this language, it helps to draw a number line and try points.
}
\litem{
Using an interval or intervals, describe all the $x$-values within or including a distance of the given values.
\[ \text{ Less than } 6 \text{ units from the number } -1. \]

The solution is \( (-7, 5) \), which is option A.\begin{enumerate}[label=\Alph*.]
\item \( (-7, 5) \)

This describes the values less than 6 from -1
\item \( (-\infty, -7] \cup [5, \infty) \)

This describes the values no less than 6 from -1
\item \( (-\infty, -7) \cup (5, \infty) \)

This describes the values more than 6 from -1
\item \( [-7, 5] \)

This describes the values no more than 6 from -1
\item \( \text{None of the above} \)

You likely thought the values in the interval were not correct.
\end{enumerate}

\textbf{General Comment:} When thinking about this language, it helps to draw a number line and try points.
}
\litem{
Solve the linear inequality below. Then, choose the constant and interval combination that describes the solution set.
\[ -7 + 4 x > 6 x \text{ or } -7 + 6 x < 9 x \]

The solution is \( (-\infty, -3.5) \text{ or } (-2.333, \infty) \), which is option D.\begin{enumerate}[label=\Alph*.]
\item \( (-\infty, a] \cup [b, \infty), \text{ where } a \in [0.33, 7.33] \text{ and } b \in [2.5, 6.5] \)

Corresponds to including the endpoints AND negating.
\item \( (-\infty, a] \cup [b, \infty), \text{ where } a \in [-5.5, 1.5] \text{ and } b \in [-6.33, -1.33] \)

Corresponds to including the endpoints (when they should be excluded).
\item \( (-\infty, a) \cup (b, \infty), \text{ where } a \in [-2.67, 8.33] \text{ and } b \in [2.5, 6.5] \)

Corresponds to inverting the inequality and negating the solution.
\item \( (-\infty, a) \cup (b, \infty), \text{ where } a \in [-3.5, 1.5] \text{ and } b \in [-3.33, 0.67] \)

 * Correct option.
\item \( (-\infty, \infty) \)

Corresponds to the variable canceling, which does not happen in this instance.
\end{enumerate}

\textbf{General Comment:} When multiplying or dividing by a negative, flip the sign.
}
\litem{
Solve the linear inequality below. Then, choose the constant and interval combination that describes the solution set.
\[ \frac{8}{4} - \frac{6}{8} x \geq \frac{5}{7} x + \frac{3}{6} \]

The solution is \( (-\infty, 1.024] \), which is option C.\begin{enumerate}[label=\Alph*.]
\item \( [a, \infty), \text{ where } a \in [-3.02, 0.98] \)

 $[-1.024, \infty)$, which corresponds to switching the direction of the interval AND negating the endpoint. You likely did this if you did not flip the inequality when dividing by a negative as well as not moving values over to a side properly.
\item \( (-\infty, a], \text{ where } a \in [-3.02, -0.02] \)

 $(-\infty, -1.024]$, which corresponds to negating the endpoint of the solution.
\item \( (-\infty, a], \text{ where } a \in [-0.98, 3.02] \)

* $(-\infty, 1.024]$, which is the correct option.
\item \( [a, \infty), \text{ where } a \in [-0.98, 2.02] \)

 $[1.024, \infty)$, which corresponds to switching the direction of the interval. You likely did this if you did not flip the inequality when dividing by a negative!
\item \( \text{None of the above}. \)

You may have chosen this if you thought the inequality did not match the ends of the intervals.
\end{enumerate}

\textbf{General Comment:} Remember that less/greater than or equal to includes the endpoint, while less/greater do not. Also, remember that you need to flip the inequality when you multiply or divide by a negative.
}
\litem{
Solve the linear inequality below. Then, choose the constant and interval combination that describes the solution set.
\[ -8 + 4 x < \frac{44 x - 7}{6} \leq 8 + 7 x \]

The solution is \( \text{None of the above.} \), which is option E.\begin{enumerate}[label=\Alph*.]
\item \( (-\infty, a) \cup [b, \infty), \text{ where } a \in [-0.95, 5.05] \text{ and } b \in [-32.5, -23.5] \)

$(-\infty, 2.05) \cup [-27.50, \infty)$, which corresponds to displaying the and-inequality as an or-inequality and getting negatives of the actual endpoints.
\item \( (a, b], \text{ where } a \in [2.05, 9.05] \text{ and } b \in [-30.5, -21.5] \)

$(2.05, -27.50]$, which is the correct interval but negatives of the actual endpoints.
\item \( [a, b), \text{ where } a \in [-1.95, 4.05] \text{ and } b \in [-29.5, -21.5] \)

$[2.05, -27.50)$, which corresponds to flipping the inequality and getting negatives of the actual endpoints.
\item \( (-\infty, a] \cup (b, \infty), \text{ where } a \in [-1.95, 3.05] \text{ and } b \in [-28.5, -25.5] \)

$(-\infty, 2.05] \cup (-27.50, \infty)$, which corresponds to displaying the and-inequality as an or-inequality AND flipping the inequality AND getting negatives of the actual endpoints.
\item \( \text{None of the above.} \)

* This is correct as the answer should be $(-2.05, 27.50]$.
\end{enumerate}

\textbf{General Comment:} To solve, you will need to break up the compound inequality into two inequalities. Be sure to keep track of the inequality! It may be best to draw a number line and graph your solution.
}
\litem{
Solve the linear inequality below. Then, choose the constant and interval combination that describes the solution set.
\[ \frac{8}{3} - \frac{4}{4} x < \frac{3}{6} x + \frac{3}{7} \]

The solution is \( (1.492, \infty) \), which is option A.\begin{enumerate}[label=\Alph*.]
\item \( (a, \infty), \text{ where } a \in [0.9, 3] \)

* $(1.492, \infty)$, which is the correct option.
\item \( (-\infty, a), \text{ where } a \in [0.49, 2.49] \)

 $(-\infty, 1.492)$, which corresponds to switching the direction of the interval. You likely did this if you did not flip the inequality when dividing by a negative!
\item \( (a, \infty), \text{ where } a \in [-1.6, -0.5] \)

 $(-1.492, \infty)$, which corresponds to negating the endpoint of the solution.
\item \( (-\infty, a), \text{ where } a \in [-2.49, 0.51] \)

 $(-\infty, -1.492)$, which corresponds to switching the direction of the interval AND negating the endpoint. You likely did this if you did not flip the inequality when dividing by a negative as well as not moving values over to a side properly.
\item \( \text{None of the above}. \)

You may have chosen this if you thought the inequality did not match the ends of the intervals.
\end{enumerate}

\textbf{General Comment:} Remember that less/greater than or equal to includes the endpoint, while less/greater do not. Also, remember that you need to flip the inequality when you multiply or divide by a negative.
}
\end{enumerate}

\end{document}