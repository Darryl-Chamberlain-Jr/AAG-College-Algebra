\documentclass{extbook}[14pt]
\usepackage{multicol, enumerate, enumitem, hyperref, color, soul, setspace, parskip, fancyhdr, amssymb, amsthm, amsmath, latexsym, units, mathtools}
\everymath{\displaystyle}
\usepackage[headsep=0.5cm,headheight=0cm, left=1 in,right= 1 in,top= 1 in,bottom= 1 in]{geometry}
\usepackage{dashrule}  % Package to use the command below to create lines between items
\newcommand{\litem}[1]{\item #1

\rule{\textwidth}{0.4pt}}
\pagestyle{fancy}
\lhead{}
\chead{Answer Key for Progress Quiz 7 Version A}
\rhead{}
\lfoot{4173-5738}
\cfoot{}
\rfoot{Spring 2021}
\begin{document}
\textbf{This key should allow you to understand why you choose the option you did (beyond just getting a question right or wrong). \href{https://xronos.clas.ufl.edu/mac1105spring2020/courseDescriptionAndMisc/Exams/LearningFromResults}{More instructions on how to use this key can be found here}.}

\textbf{If you have a suggestion to make the keys better, \href{https://forms.gle/CZkbZmPbC9XALEE88}{please fill out the short survey here}.}

\textit{Note: This key is auto-generated and may contain issues and/or errors. The keys are reviewed after each exam to ensure grading is done accurately. If there are issues (like duplicate options), they are noted in the offline gradebook. The keys are a work-in-progress to give students as many resources to improve as possible.}

\rule{\textwidth}{0.4pt}

\begin{enumerate}\litem{
Solve the linear inequality below. Then, choose the constant and interval combination that describes the solution set.
\[ -7 + 5 x < \frac{17 x - 4}{3} \leq -8 + 3 x \]The solution is \( (-8.50, -2.50] \), which is option B.\begin{enumerate}[label=\Alph*.]
\item \( (-\infty, a) \cup [b, \infty), \text{ where } a \in [-9.75, -2.25] \text{ and } b \in [-4.5, -1.5] \)

$(-\infty, -8.50) \cup [-2.50, \infty)$, which corresponds to displaying the and-inequality as an or-inequality.
\item \( (a, b], \text{ where } a \in [-12, -8.25] \text{ and } b \in [-6, 2.25] \)

* $(-8.50, -2.50]$, which is the correct option.
\item \( [a, b), \text{ where } a \in [-9, -1.5] \text{ and } b \in [-8.25, -2.25] \)

$[-8.50, -2.50)$, which corresponds to flipping the inequality.
\item \( (-\infty, a] \cup (b, \infty), \text{ where } a \in [-11.25, -5.25] \text{ and } b \in [-7.5, 0] \)

$(-\infty, -8.50] \cup (-2.50, \infty)$, which corresponds to displaying the and-inequality as an or-inequality AND flipping the inequality.
\item \( \text{None of the above.} \)


\end{enumerate}

\textbf{General Comment:} To solve, you will need to break up the compound inequality into two inequalities. Be sure to keep track of the inequality! It may be best to draw a number line and graph your solution.
}
\litem{
Using an interval or intervals, describe all the $x$-values within or including a distance of the given values.
\[ \text{ No more than } 5 \text{ units from the number } 1. \]The solution is \( [-4, 6] \), which is option A.\begin{enumerate}[label=\Alph*.]
\item \( [-4, 6] \)

This describes the values no more than 5 from 1
\item \( (-4, 6) \)

This describes the values less than 5 from 1
\item \( (-\infty, -4) \cup (6, \infty) \)

This describes the values more than 5 from 1
\item \( (-\infty, -4] \cup [6, \infty) \)

This describes the values no less than 5 from 1
\item \( \text{None of the above} \)

You likely thought the values in the interval were not correct.
\end{enumerate}

\textbf{General Comment:} When thinking about this language, it helps to draw a number line and try points.
}
\litem{
Solve the linear inequality below. Then, choose the constant and interval combination that describes the solution set.
\[ \frac{4}{8} - \frac{6}{9} x \geq \frac{6}{4} x - \frac{10}{2} \]The solution is \( (-\infty, 2.538] \), which is option B.\begin{enumerate}[label=\Alph*.]
\item \( (-\infty, a], \text{ where } a \in [-3, 0.75] \)

 $(-\infty, -2.538]$, which corresponds to negating the endpoint of the solution.
\item \( (-\infty, a], \text{ where } a \in [-0.75, 5.25] \)

* $(-\infty, 2.538]$, which is the correct option.
\item \( [a, \infty), \text{ where } a \in [-0.75, 5.25] \)

 $[2.538, \infty)$, which corresponds to switching the direction of the interval. You likely did this if you did not flip the inequality when dividing by a negative!
\item \( [a, \infty), \text{ where } a \in [-3, 0] \)

 $[-2.538, \infty)$, which corresponds to switching the direction of the interval AND negating the endpoint. You likely did this if you did not flip the inequality when dividing by a negative as well as not moving values over to a side properly.
\item \( \text{None of the above}. \)

You may have chosen this if you thought the inequality did not match the ends of the intervals.
\end{enumerate}

\textbf{General Comment:} Remember that less/greater than or equal to includes the endpoint, while less/greater do not. Also, remember that you need to flip the inequality when you multiply or divide by a negative.
}
\litem{
Solve the linear inequality below. Then, choose the constant and interval combination that describes the solution set.
\[ -6 + 8 x \leq \frac{39 x + 9}{4} < 8 + 9 x \]The solution is \( \text{None of the above.} \), which is option E.\begin{enumerate}[label=\Alph*.]
\item \( (-\infty, a] \cup (b, \infty), \text{ where } a \in [3, 9] \text{ and } b \in [-11.25, -2.25] \)

$(-\infty, 4.71] \cup (-7.67, \infty)$, which corresponds to displaying the and-inequality as an or-inequality and getting negatives of the actual endpoints.
\item \( (-\infty, a) \cup [b, \infty), \text{ where } a \in [4.5, 8.25] \text{ and } b \in [-8.25, -6] \)

$(-\infty, 4.71) \cup [-7.67, \infty)$, which corresponds to displaying the and-inequality as an or-inequality AND flipping the inequality AND getting negatives of the actual endpoints.
\item \( [a, b), \text{ where } a \in [3.75, 5.25] \text{ and } b \in [-9.75, -1.5] \)

$[4.71, -7.67)$, which is the correct interval but negatives of the actual endpoints.
\item \( (a, b], \text{ where } a \in [2.25, 6.75] \text{ and } b \in [-9.75, -4.5] \)

$(4.71, -7.67]$, which corresponds to flipping the inequality and getting negatives of the actual endpoints.
\item \( \text{None of the above.} \)

* This is correct as the answer should be $[-4.71, 7.67)$.
\end{enumerate}

\textbf{General Comment:} To solve, you will need to break up the compound inequality into two inequalities. Be sure to keep track of the inequality! It may be best to draw a number line and graph your solution.
}
\litem{
Using an interval or intervals, describe all the $x$-values within or including a distance of the given values.
\[ \text{ No more than } 4 \text{ units from the number } 2. \]The solution is \( [-2, 6] \), which is option C.\begin{enumerate}[label=\Alph*.]
\item \( (-\infty, -2] \cup [6, \infty) \)

This describes the values no less than 4 from 2
\item \( (-\infty, -2) \cup (6, \infty) \)

This describes the values more than 4 from 2
\item \( [-2, 6] \)

This describes the values no more than 4 from 2
\item \( (-2, 6) \)

This describes the values less than 4 from 2
\item \( \text{None of the above} \)

You likely thought the values in the interval were not correct.
\end{enumerate}

\textbf{General Comment:} When thinking about this language, it helps to draw a number line and try points.
}
\litem{
Solve the linear inequality below. Then, choose the constant and interval combination that describes the solution set.
\[ -3 + 6 x > 8 x \text{ or } 8 + 9 x < 11 x \]The solution is \( (-\infty, -1.5) \text{ or } (4.0, \infty) \), which is option A.\begin{enumerate}[label=\Alph*.]
\item \( (-\infty, a) \cup (b, \infty), \text{ where } a \in [-2.25, 0.75] \text{ and } b \in [3.52, 6] \)

 * Correct option.
\item \( (-\infty, a] \cup [b, \infty), \text{ where } a \in [-2.02, 0.67] \text{ and } b \in [3, 7.5] \)

Corresponds to including the endpoints (when they should be excluded).
\item \( (-\infty, a) \cup (b, \infty), \text{ where } a \in [-6, -3.75] \text{ and } b \in [0.9, 3.9] \)

Corresponds to inverting the inequality and negating the solution.
\item \( (-\infty, a] \cup [b, \infty), \text{ where } a \in [-4.65, -3.67] \text{ and } b \in [0.75, 3.75] \)

Corresponds to including the endpoints AND negating.
\item \( (-\infty, \infty) \)

Corresponds to the variable canceling, which does not happen in this instance.
\end{enumerate}

\textbf{General Comment:} When multiplying or dividing by a negative, flip the sign.
}
\litem{
Solve the linear inequality below. Then, choose the constant and interval combination that describes the solution set.
\[ -7 + 4 x > 5 x \text{ or } 3 + 3 x < 5 x \]The solution is \( (-\infty, -7.0) \text{ or } (1.5, \infty) \), which is option A.\begin{enumerate}[label=\Alph*.]
\item \( (-\infty, a) \cup (b, \infty), \text{ where } a \in [-8.25, -3] \text{ and } b \in [0, 2.25] \)

 * Correct option.
\item \( (-\infty, a) \cup (b, \infty), \text{ where } a \in [-2.25, 1.5] \text{ and } b \in [4.5, 10.5] \)

Corresponds to inverting the inequality and negating the solution.
\item \( (-\infty, a] \cup [b, \infty), \text{ where } a \in [-9.75, -6.75] \text{ and } b \in [-4.5, 3] \)

Corresponds to including the endpoints (when they should be excluded).
\item \( (-\infty, a] \cup [b, \infty), \text{ where } a \in [-3, 1.5] \text{ and } b \in [6, 12] \)

Corresponds to including the endpoints AND negating.
\item \( (-\infty, \infty) \)

Corresponds to the variable canceling, which does not happen in this instance.
\end{enumerate}

\textbf{General Comment:} When multiplying or dividing by a negative, flip the sign.
}
\litem{
Solve the linear inequality below. Then, choose the constant and interval combination that describes the solution set.
\[ -7x + 10 > -4x -4 \]The solution is \( (-\infty, 4.667) \), which is option C.\begin{enumerate}[label=\Alph*.]
\item \( (a, \infty), \text{ where } a \in [2.67, 10.67] \)

 $(4.667, \infty)$, which corresponds to switching the direction of the interval. You likely did this if you did not flip the inequality when dividing by a negative!
\item \( (a, \infty), \text{ where } a \in [-6.67, -0.67] \)

 $(-4.667, \infty)$, which corresponds to switching the direction of the interval AND negating the endpoint. You likely did this if you did not flip the inequality when dividing by a negative as well as not moving values over to a side properly.
\item \( (-\infty, a), \text{ where } a \in [1.67, 5.67] \)

* $(-\infty, 4.667)$, which is the correct option.
\item \( (-\infty, a), \text{ where } a \in [-4.67, 1.33] \)

 $(-\infty, -4.667)$, which corresponds to negating the endpoint of the solution.
\item \( \text{None of the above}. \)

You may have chosen this if you thought the inequality did not match the ends of the intervals.
\end{enumerate}

\textbf{General Comment:} Remember that less/greater than or equal to includes the endpoint, while less/greater do not. Also, remember that you need to flip the inequality when you multiply or divide by a negative.
}
\litem{
Solve the linear inequality below. Then, choose the constant and interval combination that describes the solution set.
\[ \frac{4}{8} - \frac{4}{3} x < \frac{-3}{9} x - \frac{6}{7} \]The solution is \( (1.357, \infty) \), which is option B.\begin{enumerate}[label=\Alph*.]
\item \( (-\infty, a), \text{ where } a \in [-3.75, 0] \)

 $(-\infty, -1.357)$, which corresponds to switching the direction of the interval AND negating the endpoint. You likely did this if you did not flip the inequality when dividing by a negative as well as not moving values over to a side properly.
\item \( (a, \infty), \text{ where } a \in [0, 4.5] \)

* $(1.357, \infty)$, which is the correct option.
\item \( (-\infty, a), \text{ where } a \in [0.75, 4.5] \)

 $(-\infty, 1.357)$, which corresponds to switching the direction of the interval. You likely did this if you did not flip the inequality when dividing by a negative!
\item \( (a, \infty), \text{ where } a \in [-3, -0.75] \)

 $(-1.357, \infty)$, which corresponds to negating the endpoint of the solution.
\item \( \text{None of the above}. \)

You may have chosen this if you thought the inequality did not match the ends of the intervals.
\end{enumerate}

\textbf{General Comment:} Remember that less/greater than or equal to includes the endpoint, while less/greater do not. Also, remember that you need to flip the inequality when you multiply or divide by a negative.
}
\litem{
Solve the linear inequality below. Then, choose the constant and interval combination that describes the solution set.
\[ -10x + 7 \leq 7x + 6 \]The solution is \( [0.059, \infty) \), which is option C.\begin{enumerate}[label=\Alph*.]
\item \( (-\infty, a], \text{ where } a \in [-0.04, 0.12] \)

 $(-\infty, 0.059]$, which corresponds to switching the direction of the interval. You likely did this if you did not flip the inequality when dividing by a negative!
\item \( (-\infty, a], \text{ where } a \in [-0.07, -0.01] \)

 $(-\infty, -0.059]$, which corresponds to switching the direction of the interval AND negating the endpoint. You likely did this if you did not flip the inequality when dividing by a negative as well as not moving values over to a side properly.
\item \( [a, \infty), \text{ where } a \in [0.04, 0.16] \)

* $[0.059, \infty)$, which is the correct option.
\item \( [a, \infty), \text{ where } a \in [-0.09, -0.05] \)

 $[-0.059, \infty)$, which corresponds to negating the endpoint of the solution.
\item \( \text{None of the above}. \)

You may have chosen this if you thought the inequality did not match the ends of the intervals.
\end{enumerate}

\textbf{General Comment:} Remember that less/greater than or equal to includes the endpoint, while less/greater do not. Also, remember that you need to flip the inequality when you multiply or divide by a negative.
}
\end{enumerate}

\end{document}