\documentclass{extbook}[14pt]
\usepackage{multicol, enumerate, enumitem, hyperref, color, soul, setspace, parskip, fancyhdr, amssymb, amsthm, amsmath, bbm, latexsym, units, mathtools}
\everymath{\displaystyle}
\usepackage[headsep=0.5cm,headheight=0cm, left=1 in,right= 1 in,top= 1 in,bottom= 1 in]{geometry}
\usepackage{dashrule}  % Package to use the command below to create lines between items
\newcommand{\litem}[1]{\item #1

\rule{\textwidth}{0.4pt}}
\pagestyle{fancy}
\lhead{}
\chead{Answer Key for Progress Quiz 8 Version A}
\rhead{}
\lfoot{4553-3922}
\cfoot{}
\rfoot{Fall 2020}
\begin{document}
\textbf{This key should allow you to understand why you choose the option you did (beyond just getting a question right or wrong). \href{https://xronos.clas.ufl.edu/mac1105spring2020/courseDescriptionAndMisc/Exams/LearningFromResults}{More instructions on how to use this key can be found here}.}

\textbf{If you have a suggestion to make the keys better, \href{https://forms.gle/CZkbZmPbC9XALEE88}{please fill out the short survey here}.}

\textit{Note: This key is auto-generated and may contain issues and/or errors. The keys are reviewed after each exam to ensure grading is done accurately. If there are issues (like duplicate options), they are noted in the offline gradebook. The keys are a work-in-progress to give students as many resources to improve as possible.}

\rule{\textwidth}{0.4pt}

\begin{enumerate}\litem{
Solve the linear inequality below. Then, choose the constant and interval combination that describes the solution set.
\[ 3x -4 > 10x + 8 \]

The solution is \( (-\infty, -1.714) \), which is option B.\begin{enumerate}[label=\Alph*.]
\item \( (a, \infty), \text{ where } a \in [1.71, 2.71] \)

 $(1.714, \infty)$, which corresponds to switching the direction of the interval AND negating the endpoint. You likely did this if you did not flip the inequality when dividing by a negative as well as not moving values over to a side properly.
\item \( (-\infty, a), \text{ where } a \in [-6.71, 1.29] \)

* $(-\infty, -1.714)$, which is the correct option.
\item \( (a, \infty), \text{ where } a \in [-5.71, 0.29] \)

 $(-1.714, \infty)$, which corresponds to switching the direction of the interval. You likely did this if you did not flip the inequality when dividing by a negative!
\item \( (-\infty, a), \text{ where } a \in [0.71, 5.71] \)

 $(-\infty, 1.714)$, which corresponds to negating the endpoint of the solution.
\item \( \text{None of the above}. \)

You may have chosen this if you thought the inequality did not match the ends of the intervals.
\end{enumerate}

\textbf{General Comment:} Remember that less/greater than or equal to includes the endpoint, while less/greater do not. Also, remember that you need to flip the inequality when you multiply or divide by a negative.
}
\litem{
Using an interval or intervals, describe all the $x$-values within or including a distance of the given values.
\[ \text{ Less than } 10 \text{ units from the number } 7. \]

The solution is \( (-3, 17) \), which is option C.\begin{enumerate}[label=\Alph*.]
\item \( (-\infty, -3] \cup [17, \infty) \)

This describes the values no less than 10 from 7
\item \( [-3, 17] \)

This describes the values no more than 10 from 7
\item \( (-3, 17) \)

This describes the values less than 10 from 7
\item \( (-\infty, -3) \cup (17, \infty) \)

This describes the values more than 10 from 7
\item \( \text{None of the above} \)

You likely thought the values in the interval were not correct.
\end{enumerate}

\textbf{General Comment:} When thinking about this language, it helps to draw a number line and try points.
}
\litem{
Solve the linear inequality below. Then, choose the constant and interval combination that describes the solution set.
\[ -6 - 3 x \leq \frac{-22 x - 3}{9} < 4 - 3 x \]

The solution is \( \text{None of the above.} \), which is option E.\begin{enumerate}[label=\Alph*.]
\item \( (-\infty, a) \cup [b, \infty), \text{ where } a \in [10.2, 11.2] \text{ and } b \in [-11.8, -3.8] \)

$(-\infty, 10.20) \cup [-7.80, \infty)$, which corresponds to displaying the and-inequality as an or-inequality AND flipping the inequality AND getting negatives of the actual endpoints.
\item \( [a, b), \text{ where } a \in [7.2, 13.2] \text{ and } b \in [-8.8, -3.8] \)

$[10.20, -7.80)$, which is the correct interval but negatives of the actual endpoints.
\item \( (-\infty, a] \cup (b, \infty), \text{ where } a \in [5.2, 12.2] \text{ and } b \in [-9.8, -6.8] \)

$(-\infty, 10.20] \cup (-7.80, \infty)$, which corresponds to displaying the and-inequality as an or-inequality and getting negatives of the actual endpoints.
\item \( (a, b], \text{ where } a \in [6.2, 14.2] \text{ and } b \in [-7.8, -4.8] \)

$(10.20, -7.80]$, which corresponds to flipping the inequality and getting negatives of the actual endpoints.
\item \( \text{None of the above.} \)

* This is correct as the answer should be $[-10.20, 7.80)$.
\end{enumerate}

\textbf{General Comment:} To solve, you will need to break up the compound inequality into two inequalities. Be sure to keep track of the inequality! It may be best to draw a number line and graph your solution.
}
\litem{
Solve the linear inequality below. Then, choose the constant and interval combination that describes the solution set.
\[ \frac{-5}{3} - \frac{10}{5} x \leq \frac{-3}{4} x - \frac{3}{9} \]

The solution is \( [-1.067, \infty) \), which is option B.\begin{enumerate}[label=\Alph*.]
\item \( (-\infty, a], \text{ where } a \in [-2.9, -0.9] \)

 $(-\infty, -1.067]$, which corresponds to switching the direction of the interval. You likely did this if you did not flip the inequality when dividing by a negative!
\item \( [a, \infty), \text{ where } a \in [-3.07, 0.93] \)

* $[-1.067, \infty)$, which is the correct option.
\item \( [a, \infty), \text{ where } a \in [0.07, 2.07] \)

 $[1.067, \infty)$, which corresponds to negating the endpoint of the solution.
\item \( (-\infty, a], \text{ where } a \in [0.4, 3.1] \)

 $(-\infty, 1.067]$, which corresponds to switching the direction of the interval AND negating the endpoint. You likely did this if you did not flip the inequality when dividing by a negative as well as not moving values over to a side properly.
\item \( \text{None of the above}. \)

You may have chosen this if you thought the inequality did not match the ends of the intervals.
\end{enumerate}

\textbf{General Comment:} Remember that less/greater than or equal to includes the endpoint, while less/greater do not. Also, remember that you need to flip the inequality when you multiply or divide by a negative.
}
\litem{
Solve the linear inequality below. Then, choose the constant and interval combination that describes the solution set.
\[ -3 + 8 x > 9 x \text{ or } -6 + 7 x < 10 x \]

The solution is \( (-\infty, -3.0) \text{ or } (-2.0, \infty) \), which is option B.\begin{enumerate}[label=\Alph*.]
\item \( (-\infty, a] \cup [b, \infty), \text{ where } a \in [-3, 0] \text{ and } b \in [-2, 2] \)

Corresponds to including the endpoints (when they should be excluded).
\item \( (-\infty, a) \cup (b, \infty), \text{ where } a \in [-3, 1] \text{ and } b \in [-4, -1] \)

 * Correct option.
\item \( (-\infty, a) \cup (b, \infty), \text{ where } a \in [2, 9] \text{ and } b \in [-1, 4] \)

Corresponds to inverting the inequality and negating the solution.
\item \( (-\infty, a] \cup [b, \infty), \text{ where } a \in [1, 4] \text{ and } b \in [2, 4] \)

Corresponds to including the endpoints AND negating.
\item \( (-\infty, \infty) \)

Corresponds to the variable canceling, which does not happen in this instance.
\end{enumerate}

\textbf{General Comment:} When multiplying or dividing by a negative, flip the sign.
}
\litem{
Solve the linear inequality below. Then, choose the constant and interval combination that describes the solution set.
\[ -7 + 7 x > 8 x \text{ or } 3 + 3 x < 6 x \]

The solution is \( (-\infty, -7.0) \text{ or } (1.0, \infty) \), which is option D.\begin{enumerate}[label=\Alph*.]
\item \( (-\infty, a] \cup [b, \infty), \text{ where } a \in [-4, 1] \text{ and } b \in [7, 10] \)

Corresponds to including the endpoints AND negating.
\item \( (-\infty, a) \cup (b, \infty), \text{ where } a \in [-2, 1] \text{ and } b \in [7, 12] \)

Corresponds to inverting the inequality and negating the solution.
\item \( (-\infty, a] \cup [b, \infty), \text{ where } a \in [-9, -6] \text{ and } b \in [-2, 4] \)

Corresponds to including the endpoints (when they should be excluded).
\item \( (-\infty, a) \cup (b, \infty), \text{ where } a \in [-7, -6] \text{ and } b \in [-2, 3] \)

 * Correct option.
\item \( (-\infty, \infty) \)

Corresponds to the variable canceling, which does not happen in this instance.
\end{enumerate}

\textbf{General Comment:} When multiplying or dividing by a negative, flip the sign.
}
\litem{
Solve the linear inequality below. Then, choose the constant and interval combination that describes the solution set.
\[ -5 + 6 x \leq \frac{40 x + 9}{6} < -7 + 4 x \]

The solution is \( [-9.75, -3.19) \), which is option A.\begin{enumerate}[label=\Alph*.]
\item \( [a, b), \text{ where } a \in [-9.75, -7.75] \text{ and } b \in [-7.19, -2.19] \)

$[-9.75, -3.19)$, which is the correct option.
\item \( (-\infty, a) \cup [b, \infty), \text{ where } a \in [-12.75, -6.75] \text{ and } b \in [-6.19, 0.81] \)

$(-\infty, -9.75) \cup [-3.19, \infty)$, which corresponds to displaying the and-inequality as an or-inequality AND flipping the inequality.
\item \( (a, b], \text{ where } a \in [-10.75, -7.75] \text{ and } b \in [-7.19, -1.19] \)

$(-9.75, -3.19]$, which corresponds to flipping the inequality.
\item \( (-\infty, a] \cup (b, \infty), \text{ where } a \in [-9.75, -8.75] \text{ and } b \in [-7.19, 1.81] \)

$(-\infty, -9.75] \cup (-3.19, \infty)$, which corresponds to displaying the and-inequality as an or-inequality.
\item \( \text{None of the above.} \)


\end{enumerate}

\textbf{General Comment:} To solve, you will need to break up the compound inequality into two inequalities. Be sure to keep track of the inequality! It may be best to draw a number line and graph your solution.
}
\litem{
Using an interval or intervals, describe all the $x$-values within or including a distance of the given values.
\[ \text{ No more than } 5 \text{ units from the number } 7. \]

The solution is \( [2, 12] \), which is option A.\begin{enumerate}[label=\Alph*.]
\item \( [2, 12] \)

This describes the values no more than 5 from 7
\item \( (-\infty, 2] \cup [12, \infty) \)

This describes the values no less than 5 from 7
\item \( (-\infty, 2) \cup (12, \infty) \)

This describes the values more than 5 from 7
\item \( (2, 12) \)

This describes the values less than 5 from 7
\item \( \text{None of the above} \)

You likely thought the values in the interval were not correct.
\end{enumerate}

\textbf{General Comment:} When thinking about this language, it helps to draw a number line and try points.
}
\litem{
Solve the linear inequality below. Then, choose the constant and interval combination that describes the solution set.
\[ \frac{6}{5} - \frac{6}{8} x \leq \frac{-5}{6} x - \frac{9}{4} \]

The solution is \( (-\infty, -41.4] \), which is option B.\begin{enumerate}[label=\Alph*.]
\item \( (-\infty, a], \text{ where } a \in [40.4, 43.4] \)

 $(-\infty, 41.4]$, which corresponds to negating the endpoint of the solution.
\item \( (-\infty, a], \text{ where } a \in [-42.4, -39.4] \)

* $(-\infty, -41.4]$, which is the correct option.
\item \( [a, \infty), \text{ where } a \in [40.4, 43.4] \)

 $[41.4, \infty)$, which corresponds to switching the direction of the interval AND negating the endpoint. You likely did this if you did not flip the inequality when dividing by a negative as well as not moving values over to a side properly.
\item \( [a, \infty), \text{ where } a \in [-43.4, -39.4] \)

 $[-41.4, \infty)$, which corresponds to switching the direction of the interval. You likely did this if you did not flip the inequality when dividing by a negative!
\item \( \text{None of the above}. \)

You may have chosen this if you thought the inequality did not match the ends of the intervals.
\end{enumerate}

\textbf{General Comment:} Remember that less/greater than or equal to includes the endpoint, while less/greater do not. Also, remember that you need to flip the inequality when you multiply or divide by a negative.
}
\litem{
Solve the linear inequality below. Then, choose the constant and interval combination that describes the solution set.
\[ 3x -5 < 6x -3 \]

The solution is \( (-0.667, \infty) \), which is option D.\begin{enumerate}[label=\Alph*.]
\item \( (-\infty, a), \text{ where } a \in [-4.5, -0.5] \)

 $(-\infty, -0.667)$, which corresponds to switching the direction of the interval. You likely did this if you did not flip the inequality when dividing by a negative!
\item \( (-\infty, a), \text{ where } a \in [0.3, 3.3] \)

 $(-\infty, 0.667)$, which corresponds to switching the direction of the interval AND negating the endpoint. You likely did this if you did not flip the inequality when dividing by a negative as well as not moving values over to a side properly.
\item \( (a, \infty), \text{ where } a \in [-0.24, 1.18] \)

 $(0.667, \infty)$, which corresponds to negating the endpoint of the solution.
\item \( (a, \infty), \text{ where } a \in [-2.11, -0.11] \)

* $(-0.667, \infty)$, which is the correct option.
\item \( \text{None of the above}. \)

You may have chosen this if you thought the inequality did not match the ends of the intervals.
\end{enumerate}

\textbf{General Comment:} Remember that less/greater than or equal to includes the endpoint, while less/greater do not. Also, remember that you need to flip the inequality when you multiply or divide by a negative.
}
\end{enumerate}

\end{document}