\documentclass{extbook}[14pt]
\usepackage{multicol, enumerate, enumitem, hyperref, color, soul, setspace, parskip, fancyhdr, amssymb, amsthm, amsmath, bbm, latexsym, units, mathtools}
\everymath{\displaystyle}
\usepackage[headsep=0.5cm,headheight=0cm, left=1 in,right= 1 in,top= 1 in,bottom= 1 in]{geometry}
\usepackage{dashrule}  % Package to use the command below to create lines between items
\newcommand{\litem}[1]{\item #1

\rule{\textwidth}{0.4pt}}
\pagestyle{fancy}
\lhead{}
\chead{Answer Key for Progress Quiz 4 Version A}
\rhead{}
\lfoot{6286-1986}
\cfoot{}
\rfoot{Fall 2020}
\begin{document}
\textbf{This key should allow you to understand why you choose the option you did (beyond just getting a question right or wrong). \href{https://xronos.clas.ufl.edu/mac1105spring2020/courseDescriptionAndMisc/Exams/LearningFromResults}{More instructions on how to use this key can be found here}.}

\textbf{If you have a suggestion to make the keys better, \href{https://forms.gle/CZkbZmPbC9XALEE88}{please fill out the short survey here}.}

\textit{Note: This key is auto-generated and may contain issues and/or errors. The keys are reviewed after each exam to ensure grading is done accurately. If there are issues (like duplicate options), they are noted in the offline gradebook. The keys are a work-in-progress to give students as many resources to improve as possible.}

\rule{\textwidth}{0.4pt}

\begin{enumerate}\litem{
Solve the linear inequality below. Then, choose the constant and interval combination that describes the solution set.
\[ -6 + 7 x > 9 x \text{ or } 9 + 9 x < 11 x \]
The solution is \( (-\infty, -3.0) \text{ or } (4.5, \infty) \), which is option C.\begin{enumerate}[label=\Alph*.]
\item \( (-\infty, a] \cup [b, \infty), \text{ where } a \in [-3, -1] \text{ and } b \in [4, 8.5] \)

Corresponds to including the endpoints (when they should be excluded).
\item \( (-\infty, a) \cup (b, \infty), \text{ where } a \in [-8.5, -3.5] \text{ and } b \in [-5, 4] \)

Corresponds to inverting the inequality and negating the solution.
\item \( (-\infty, a) \cup (b, \infty), \text{ where } a \in [-4, -1] \text{ and } b \in [3.5, 6.5] \)

 * Correct option.
\item \( (-\infty, a] \cup [b, \infty), \text{ where } a \in [-7.5, -3.5] \text{ and } b \in [1.5, 3.6] \)

Corresponds to including the endpoints AND negating.
\item \( (-\infty, \infty) \)

Corresponds to the variable canceling, which does not happen in this instance.
\end{enumerate}

\textbf{General Comment:} When multiplying or dividing by a negative, flip the sign.
}
\litem{
Using an interval or intervals, describe all the $x$-values within or including a distance of the given values.
\[ \text{ Less than } 6 \text{ units from the number } 9. \]
The solution is \( \text{None of the above} \), which is option E.\begin{enumerate}[label=\Alph*.]
\item \( (-\infty, -3) \cup (15, \infty) \)

This describes the values more than 9 from 6
\item \( (-\infty, -3] \cup [15, \infty) \)

This describes the values no less than 9 from 6
\item \( [-3, 15] \)

This describes the values no more than 9 from 6
\item \( (-3, 15) \)

This describes the values less than 9 from 6
\item \( \text{None of the above} \)

Options A-D described the values [more/less than] 9 units from 6, which is the reverse of what the question asked.
\end{enumerate}

\textbf{General Comment:} When thinking about this language, it helps to draw a number line and try points.
}
\litem{
Solve the linear inequality below. Then, choose the constant and interval combination that describes the solution set.
\[ 6x -9 \geq 7x + 6 \]
The solution is \( (-\infty, -15.0] \), which is option B.\begin{enumerate}[label=\Alph*.]
\item \( [a, \infty), \text{ where } a \in [13, 16] \)

 $[15.0, \infty)$, which corresponds to switching the direction of the interval AND negating the endpoint. You likely did this if you did not flip the inequality when dividing by a negative as well as not moving values over to a side properly.
\item \( (-\infty, a], \text{ where } a \in [-21, -14] \)

* $(-\infty, -15.0]$, which is the correct option.
\item \( (-\infty, a], \text{ where } a \in [12, 17] \)

 $(-\infty, 15.0]$, which corresponds to negating the endpoint of the solution.
\item \( [a, \infty), \text{ where } a \in [-15, -8] \)

 $[-15.0, \infty)$, which corresponds to switching the direction of the interval. You likely did this if you did not flip the inequality when dividing by a negative!
\item \( \text{None of the above}. \)

You may have chosen this if you thought the inequality did not match the ends of the intervals.
\end{enumerate}

\textbf{General Comment:} Remember that less/greater than or equal to includes the endpoint, while less/greater do not. Also, remember that you need to flip the inequality when you multiply or divide by a negative.
}
\litem{
Using an interval or intervals, describe all the $x$-values within or including a distance of the given values.
\[ \text{ No less than } 3 \text{ units from the number } 7. \]
The solution is \( (-\infty, 4] \cup [10, \infty) \), which is option A.\begin{enumerate}[label=\Alph*.]
\item \( (-\infty, 4] \cup [10, \infty) \)

This describes the values no less than 3 from 7
\item \( (-\infty, 4) \cup (10, \infty) \)

This describes the values more than 3 from 7
\item \( [4, 10] \)

This describes the values no more than 3 from 7
\item \( (4, 10) \)

This describes the values less than 3 from 7
\item \( \text{None of the above} \)

You likely thought the values in the interval were not correct.
\end{enumerate}

\textbf{General Comment:} When thinking about this language, it helps to draw a number line and try points.
}
\litem{
Solve the linear inequality below. Then, choose the constant and interval combination that describes the solution set.
\[ \frac{-4}{7} - \frac{6}{3} x < \frac{3}{4} x + \frac{8}{2} \]
The solution is \( (-1.662, \infty) \), which is option D.\begin{enumerate}[label=\Alph*.]
\item \( (-\infty, a), \text{ where } a \in [-2.66, -0.66] \)

 $(-\infty, -1.662)$, which corresponds to switching the direction of the interval. You likely did this if you did not flip the inequality when dividing by a negative!
\item \( (-\infty, a), \text{ where } a \in [-1.34, 3.66] \)

 $(-\infty, 1.662)$, which corresponds to switching the direction of the interval AND negating the endpoint. You likely did this if you did not flip the inequality when dividing by a negative as well as not moving values over to a side properly.
\item \( (a, \infty), \text{ where } a \in [1.66, 2.66] \)

 $(1.662, \infty)$, which corresponds to negating the endpoint of the solution.
\item \( (a, \infty), \text{ where } a \in [-2.66, -0.66] \)

* $(-1.662, \infty)$, which is the correct option.
\item \( \text{None of the above}. \)

You may have chosen this if you thought the inequality did not match the ends of the intervals.
\end{enumerate}

\textbf{General Comment:} Remember that less/greater than or equal to includes the endpoint, while less/greater do not. Also, remember that you need to flip the inequality when you multiply or divide by a negative.
}
\litem{
Solve the linear inequality below. Then, choose the constant and interval combination that describes the solution set.
\[ -7 + 4 x > 5 x \text{ or } -8 + 6 x < 9 x \]
The solution is \( (-\infty, -7.0) \text{ or } (-2.667, \infty) \), which is option D.\begin{enumerate}[label=\Alph*.]
\item \( (-\infty, a] \cup [b, \infty), \text{ where } a \in [-0.33, 8.67] \text{ and } b \in [4, 8] \)

Corresponds to including the endpoints AND negating.
\item \( (-\infty, a] \cup [b, \infty), \text{ where } a \in [-7, -2] \text{ and } b \in [-7.67, 0.33] \)

Corresponds to including the endpoints (when they should be excluded).
\item \( (-\infty, a) \cup (b, \infty), \text{ where } a \in [-0.33, 4.67] \text{ and } b \in [6, 9] \)

Corresponds to inverting the inequality and negating the solution.
\item \( (-\infty, a) \cup (b, \infty), \text{ where } a \in [-10, -6] \text{ and } b \in [-6.67, 2.33] \)

 * Correct option.
\item \( (-\infty, \infty) \)

Corresponds to the variable canceling, which does not happen in this instance.
\end{enumerate}

\textbf{General Comment:} When multiplying or dividing by a negative, flip the sign.
}
\litem{
Solve the linear inequality below. Then, choose the constant and interval combination that describes the solution set.
\[ \frac{7}{9} - \frac{4}{3} x \geq \frac{5}{5} x - \frac{8}{4} \]
The solution is \( (-\infty, 1.19] \), which is option C.\begin{enumerate}[label=\Alph*.]
\item \( (-\infty, a], \text{ where } a \in [-1.19, -0.19] \)

 $(-\infty, -1.19]$, which corresponds to negating the endpoint of the solution.
\item \( [a, \infty), \text{ where } a \in [-4.19, 0.81] \)

 $[-1.19, \infty)$, which corresponds to switching the direction of the interval AND negating the endpoint. You likely did this if you did not flip the inequality when dividing by a negative as well as not moving values over to a side properly.
\item \( (-\infty, a], \text{ where } a \in [1.19, 3.19] \)

* $(-\infty, 1.19]$, which is the correct option.
\item \( [a, \infty), \text{ where } a \in [0.19, 6.19] \)

 $[1.19, \infty)$, which corresponds to switching the direction of the interval. You likely did this if you did not flip the inequality when dividing by a negative!
\item \( \text{None of the above}. \)

You may have chosen this if you thought the inequality did not match the ends of the intervals.
\end{enumerate}

\textbf{General Comment:} Remember that less/greater than or equal to includes the endpoint, while less/greater do not. Also, remember that you need to flip the inequality when you multiply or divide by a negative.
}
\litem{
Solve the linear inequality below. Then, choose the constant and interval combination that describes the solution set.
\[ 4x + 9 < 7x + 3 \]
The solution is \( (2.0, \infty) \), which is option A.\begin{enumerate}[label=\Alph*.]
\item \( (a, \infty), \text{ where } a \in [-1, 2.3] \)

* $(2.0, \infty)$, which is the correct option.
\item \( (a, \infty), \text{ where } a \in [-3.6, 0] \)

 $(-2.0, \infty)$, which corresponds to negating the endpoint of the solution.
\item \( (-\infty, a), \text{ where } a \in [-5, 0] \)

 $(-\infty, -2.0)$, which corresponds to switching the direction of the interval AND negating the endpoint. You likely did this if you did not flip the inequality when dividing by a negative as well as not moving values over to a side properly.
\item \( (-\infty, a), \text{ where } a \in [1, 8] \)

 $(-\infty, 2.0)$, which corresponds to switching the direction of the interval. You likely did this if you did not flip the inequality when dividing by a negative!
\item \( \text{None of the above}. \)

You may have chosen this if you thought the inequality did not match the ends of the intervals.
\end{enumerate}

\textbf{General Comment:} Remember that less/greater than or equal to includes the endpoint, while less/greater do not. Also, remember that you need to flip the inequality when you multiply or divide by a negative.
}
\litem{
Solve the linear inequality below. Then, choose the constant and interval combination that describes the solution set.
\[ 3 - 8 x < \frac{-52 x - 3}{7} \leq 7 - 8 x \]
The solution is \( \text{None of the above.} \), which is option E.\begin{enumerate}[label=\Alph*.]
\item \( (a, b], \text{ where } a \in [-8, -5] \text{ and } b \in [-17, -11] \)

$(-6.00, -13.00]$, which is the correct interval but negatives of the actual endpoints.
\item \( [a, b), \text{ where } a \in [-6, -3] \text{ and } b \in [-16, -10] \)

$[-6.00, -13.00)$, which corresponds to flipping the inequality and getting negatives of the actual endpoints.
\item \( (-\infty, a] \cup (b, \infty), \text{ where } a \in [-9, -5] \text{ and } b \in [-17, -7] \)

$(-\infty, -6.00] \cup (-13.00, \infty)$, which corresponds to displaying the and-inequality as an or-inequality AND flipping the inequality AND getting negatives of the actual endpoints.
\item \( (-\infty, a) \cup [b, \infty), \text{ where } a \in [-7, -3] \text{ and } b \in [-13, -10] \)

$(-\infty, -6.00) \cup [-13.00, \infty)$, which corresponds to displaying the and-inequality as an or-inequality and getting negatives of the actual endpoints.
\item \( \text{None of the above.} \)

* This is correct as the answer should be $(6.00, 13.00]$.
\end{enumerate}

\textbf{General Comment:} To solve, you will need to break up the compound inequality into two inequalities. Be sure to keep track of the inequality! It may be best to draw a number line and graph your solution.
}
\litem{
Solve the linear inequality below. Then, choose the constant and interval combination that describes the solution set.
\[ -6 + 5 x < \frac{45 x - 9}{5} \leq 6 + 8 x \]
The solution is \( (-1.05, 7.80] \), which is option C.\begin{enumerate}[label=\Alph*.]
\item \( (-\infty, a] \cup (b, \infty), \text{ where } a \in [-7.05, -0.05] \text{ and } b \in [7.8, 9.8] \)

$(-\infty, -1.05] \cup (7.80, \infty)$, which corresponds to displaying the and-inequality as an or-inequality AND flipping the inequality.
\item \( [a, b), \text{ where } a \in [-2.8, -0.6] \text{ and } b \in [6.8, 10.8] \)

$[-1.05, 7.80)$, which corresponds to flipping the inequality.
\item \( (a, b], \text{ where } a \in [-2.2, 0.1] \text{ and } b \in [7.8, 9.8] \)

* $(-1.05, 7.80]$, which is the correct option.
\item \( (-\infty, a) \cup [b, \infty), \text{ where } a \in [-1.3, 0.4] \text{ and } b \in [7.8, 11.8] \)

$(-\infty, -1.05) \cup [7.80, \infty)$, which corresponds to displaying the and-inequality as an or-inequality.
\item \( \text{None of the above.} \)


\end{enumerate}

\textbf{General Comment:} To solve, you will need to break up the compound inequality into two inequalities. Be sure to keep track of the inequality! It may be best to draw a number line and graph your solution.
}
\end{enumerate}

\end{document}