\documentclass{extbook}[14pt]
\usepackage{multicol, enumerate, enumitem, hyperref, color, soul, setspace, parskip, fancyhdr, amssymb, amsthm, amsmath, latexsym, units, mathtools}
\everymath{\displaystyle}
\usepackage[headsep=0.5cm,headheight=0cm, left=1 in,right= 1 in,top= 1 in,bottom= 1 in]{geometry}
\usepackage{dashrule}  % Package to use the command below to create lines between items
\newcommand{\litem}[1]{\item #1

\rule{\textwidth}{0.4pt}}
\pagestyle{fancy}
\lhead{}
\chead{Answer Key for Progress Quiz 6 Version C}
\rhead{}
\lfoot{9689-6866}
\cfoot{}
\rfoot{Spring 2021}
\begin{document}
\textbf{This key should allow you to understand why you choose the option you did (beyond just getting a question right or wrong). \href{https://xronos.clas.ufl.edu/mac1105spring2020/courseDescriptionAndMisc/Exams/LearningFromResults}{More instructions on how to use this key can be found here}.}

\textbf{If you have a suggestion to make the keys better, \href{https://forms.gle/CZkbZmPbC9XALEE88}{please fill out the short survey here}.}

\textit{Note: This key is auto-generated and may contain issues and/or errors. The keys are reviewed after each exam to ensure grading is done accurately. If there are issues (like duplicate options), they are noted in the offline gradebook. The keys are a work-in-progress to give students as many resources to improve as possible.}

\rule{\textwidth}{0.4pt}

\begin{enumerate}\litem{
Choose the interval below that $f$ composed with $g$ at $x=1$ is in.
\[ f(x) = 4x^{3} -2 x^{2} -4 x \text{ and } g(x) = 4x^{3} -1 x^{2} -3 x \]The solution is \( 0.0 \), which is option C.\begin{enumerate}[label=\Alph*.]
\item \( (f \circ g)(1) \in [-26, -23] \)

 Distractor 3: Corresponds to being slightly off from the solution.
\item \( (f \circ g)(1) \in [10, 16] \)

 Distractor 2: Corresponds to being slightly off from the solution.
\item \( (f \circ g)(1) \in [0, 2] \)

* This is the correct solution
\item \( (f \circ g)(1) \in [-31, -27] \)

 Distractor 1: Corresponds to reversing the composition.
\item \( \text{It is not possible to compose the two functions.} \)


\end{enumerate}

\textbf{General Comment:} $f$ composed with $g$ at $x$ means $f(g(x))$. The order matters!
}
\litem{
Choose the interval below that $f$ composed with $g$ at $x=1$ is in.
\[ f(x) = -3x^{3} -3 x^{2} +4 x \text{ and } g(x) = -x^{3} +2 x^{2} +2 x -2 \]The solution is \( -2.0 \), which is option A.\begin{enumerate}[label=\Alph*.]
\item \( (f \circ g)(1) \in [-2, 4] \)

* This is the correct solution
\item \( (f \circ g)(1) \in [4, 13] \)

 Distractor 1: Corresponds to reversing the composition.
\item \( (f \circ g)(1) \in [20, 21] \)

 Distractor 3: Corresponds to being slightly off from the solution.
\item \( (f \circ g)(1) \in [-12, -10] \)

 Distractor 2: Corresponds to being slightly off from the solution.
\item \( \text{It is not possible to compose the two functions.} \)


\end{enumerate}

\textbf{General Comment:} $f$ composed with $g$ at $x$ means $f(g(x))$. The order matters!
}
\litem{
Multiply the following functions, then choose the domain of the resulting function from the list below.
\[ f(x) = 3x^{3} +7 x^{2} +3 x + 3 \text{ and } g(x) = 6x + 8 \]The solution is \( (-\infty, \infty) \), which is option E.\begin{enumerate}[label=\Alph*.]
\item \( \text{ The domain is all Real numbers less than or equal to } x = a, \text{ where } a \in [4, 7] \)


\item \( \text{ The domain is all Real numbers except } x = a, \text{ where } a \in [4.83, 6.83] \)


\item \( \text{ The domain is all Real numbers greater than or equal to } x = a, \text{ where } a \in [7.33, 11.33] \)


\item \( \text{ The domain is all Real numbers except } x = a \text{ and } x = b, \text{ where } a \in [-3.2, 5.8] \text{ and } b \in [5.4, 7.4] \)


\item \( \text{ The domain is all Real numbers. } \)


\end{enumerate}

\textbf{General Comment:} The new domain is the intersection of the previous domains.
}
\litem{
Determine whether the function below is 1-1.
\[ f(x) = (6 x + 39)^3 \]The solution is \( \text{yes} \), which is option B.\begin{enumerate}[label=\Alph*.]
\item \( \text{No, because the domain of the function is not $(-\infty, \infty)$.} \)

Corresponds to believing 1-1 means the domain is all Real numbers.
\item \( \text{Yes, the function is 1-1.} \)

* This is the solution.
\item \( \text{No, because the range of the function is not $(-\infty, \infty)$.} \)

Corresponds to believing 1-1 means the range is all Real numbers.
\item \( \text{No, because there is a $y$-value that goes to 2 different $x$-values.} \)

Corresponds to the Horizontal Line test, which this function passes.
\item \( \text{No, because there is an $x$-value that goes to 2 different $y$-values.} \)

Corresponds to the Vertical Line test, which checks if an expression is a function.
\end{enumerate}

\textbf{General Comment:} There are only two valid options: The function is 1-1 OR No because there is a $y$-value that goes to 2 different $x$-values.
}
\litem{
Find the inverse of the function below (if it exists). Then, evaluate the inverse at $x = -13$ and choose the interval that $f^{-1}(-13)$ belongs to.
\[ f(x) = 2 x^2 + 5 \]The solution is \( \text{ The function is not invertible for all Real numbers. } \), which is option E.\begin{enumerate}[label=\Alph*.]
\item \( f^{-1}(-13) \in [1.9, 2.35] \)

 Distractor 2: This corresponds to finding the (nonexistent) inverse and not subtracting by the vertical shift.
\item \( f^{-1}(-13) \in [5.61, 6.01] \)

 Distractor 3: This corresponds to finding the (nonexistent) inverse and dividing by a negative.
\item \( f^{-1}(-13) \in [2.8, 3.48] \)

 Distractor 1: This corresponds to trying to find the inverse even though the function is not 1-1. 
\item \( f^{-1}(-13) \in [7.96, 8.38] \)

 Distractor 4: This corresponds to both distractors 2 and 3.
\item \( \text{ The function is not invertible for all Real numbers. } \)

* This is the correct option.
\end{enumerate}

\textbf{General Comment:} Be sure you check that the function is 1-1 before trying to find the inverse!
}
\litem{
Find the inverse of the function below (if it exists). Then, evaluate the inverse at $x = -12$ and choose the interval that $f^{-1}(-12)$ belongs to.
\[ f(x) = 5 x^2 + 4 \]The solution is \( \text{ The function is not invertible for all Real numbers. } \), which is option E.\begin{enumerate}[label=\Alph*.]
\item \( f^{-1}(-12) \in [2.64, 2.84] \)

 Distractor 3: This corresponds to finding the (nonexistent) inverse and dividing by a negative.
\item \( f^{-1}(-12) \in [4.45, 5.76] \)

 Distractor 4: This corresponds to both distractors 2 and 3.
\item \( f^{-1}(-12) \in [1.77, 2.56] \)

 Distractor 1: This corresponds to trying to find the inverse even though the function is not 1-1. 
\item \( f^{-1}(-12) \in [0.92, 1.55] \)

 Distractor 2: This corresponds to finding the (nonexistent) inverse and not subtracting by the vertical shift.
\item \( \text{ The function is not invertible for all Real numbers. } \)

* This is the correct option.
\end{enumerate}

\textbf{General Comment:} Be sure you check that the function is 1-1 before trying to find the inverse!
}
\litem{
Determine whether the function below is 1-1.
\[ f(x) = (6 x + 34)^3 \]The solution is \( \text{yes} \), which is option A.\begin{enumerate}[label=\Alph*.]
\item \( \text{Yes, the function is 1-1.} \)

* This is the solution.
\item \( \text{No, because there is an $x$-value that goes to 2 different $y$-values.} \)

Corresponds to the Vertical Line test, which checks if an expression is a function.
\item \( \text{No, because there is a $y$-value that goes to 2 different $x$-values.} \)

Corresponds to the Horizontal Line test, which this function passes.
\item \( \text{No, because the domain of the function is not $(-\infty, \infty)$.} \)

Corresponds to believing 1-1 means the domain is all Real numbers.
\item \( \text{No, because the range of the function is not $(-\infty, \infty)$.} \)

Corresponds to believing 1-1 means the range is all Real numbers.
\end{enumerate}

\textbf{General Comment:} There are only two valid options: The function is 1-1 OR No because there is a $y$-value that goes to 2 different $x$-values.
}
\litem{
Find the inverse of the function below. Then, evaluate the inverse at $x = 9$ and choose the interval that $f^{-1}(9)$ belongs to.
\[ f(x) = e^{x-4}-3 \]The solution is \( f^{-1}(9) = 6.485 \), which is option E.\begin{enumerate}[label=\Alph*.]
\item \( f^{-1}(9) \in [-1.24, -1.19] \)

 This solution corresponds to distractor 2.
\item \( f^{-1}(9) \in [-1.55, -1.48] \)

 This solution corresponds to distractor 1.
\item \( f^{-1}(9) \in [-0.48, -0.33] \)

 This solution corresponds to distractor 3.
\item \( f^{-1}(9) \in [-1.48, -1.36] \)

 This solution corresponds to distractor 4.
\item \( f^{-1}(9) \in [6.48, 6.58] \)

 This is the solution.
\end{enumerate}

\textbf{General Comment:} Natural log and exponential functions always have an inverse. Once you switch the $x$ and $y$, use the conversion $ e^y = x \leftrightarrow y=\ln(x)$.
}
\litem{
Multiply the following functions, then choose the domain of the resulting function from the list below.
\[ f(x) = 8x + 9 \text{ and } g(x) = 3x^{3} +4 x^{2} +6 x + 4 \]The solution is \( (-\infty, \infty) \), which is option E.\begin{enumerate}[label=\Alph*.]
\item \( \text{ The domain is all Real numbers less than or equal to } x = a, \text{ where } a \in [-2.4, 10.6] \)


\item \( \text{ The domain is all Real numbers except } x = a, \text{ where } a \in [-7.83, 0.17] \)


\item \( \text{ The domain is all Real numbers greater than or equal to } x = a, \text{ where } a \in [2.5, 8.5] \)


\item \( \text{ The domain is all Real numbers except } x = a \text{ and } x = b, \text{ where } a \in [1.67, 10.67] \text{ and } b \in [-14.2, -2.2] \)


\item \( \text{ The domain is all Real numbers. } \)


\end{enumerate}

\textbf{General Comment:} The new domain is the intersection of the previous domains.
}
\litem{
Find the inverse of the function below. Then, evaluate the inverse at $x = 8$ and choose the interval that $f^{-1}(8)$ belongs to.
\[ f(x) = e^{x+3}+5 \]The solution is \( f^{-1}(8) = -1.901 \), which is option E.\begin{enumerate}[label=\Alph*.]
\item \( f^{-1}(8) \in [7.46, 7.6] \)

 This solution corresponds to distractor 2.
\item \( f^{-1}(8) \in [4.07, 4.19] \)

 This solution corresponds to distractor 1.
\item \( f^{-1}(8) \in [7.3, 7.44] \)

 This solution corresponds to distractor 4.
\item \( f^{-1}(8) \in [6.44, 6.65] \)

 This solution corresponds to distractor 3.
\item \( f^{-1}(8) \in [-2.12, -1.87] \)

 This is the solution.
\end{enumerate}

\textbf{General Comment:} Natural log and exponential functions always have an inverse. Once you switch the $x$ and $y$, use the conversion $ e^y = x \leftrightarrow y=\ln(x)$.
}
\end{enumerate}

\end{document}