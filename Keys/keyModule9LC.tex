\documentclass{extbook}[14pt]
\usepackage{multicol, enumerate, enumitem, hyperref, color, soul, setspace, parskip, fancyhdr, amssymb, amsthm, amsmath, bbm, latexsym, units, mathtools}
\everymath{\displaystyle}
\usepackage[headsep=0.5cm,headheight=0cm, left=1 in,right= 1 in,top= 1 in,bottom= 1 in]{geometry}
\pagestyle{fancy}
\lhead{}
\chead{Answer Key for Module\,9L\,-\,Operations\,on\,Functions Version C}
\rhead{}
\lfoot{Summer\,C\,2020}
\cfoot{}
\rfoot{}
\begin{document}
\textbf{This key should allow you to understand why you choose the option you did (beyond just getting a question right or wrong). \href{https://xronos.clas.ufl.edu/mac1105spring2020/courseDescriptionAndMisc/Exams/LearningFromResults}{More instructions on how to use this key can be found here}.}

\textbf{If you have a suggestion to make the keys better, \href{https://forms.gle/CZkbZmPbC9XALEE88}{please fill out the short survey here}.}

\textit{Note: This key is auto-generated and may contain issues and/or errors. The keys are reviewed after each exam to ensure grading is done accurately. If there are issues (like duplicate options), they are noted in the offline gradebook. The keys are a work-in-progress to give students as many resources to improve as possible.}

\rule{\textwidth}{0.4pt}

61. Find the inverse of the function below. Then, evaluate the inverse at $x = 9$ and choose the interval that $f^{-1}(9)$ belongs to.
\[ f(x) = e^{x+5}+3 \] 
The solution is $ f^{-1}(9) = -3.208 $ 

\begin{enumerate}[label=\Alph*.] 
\item $ f^{-1}(9) \in [6.47, 6.96] $ 

  This solution corresponds to distractor 1. 
\item $ f^{-1}(9) \in [5.45, 5.55] $ 

  This solution corresponds to distractor 2. 
\item $ f^{-1}(9) \in [-3.54, -3.01] $ 

  This is the solution. 
\item $ f^{-1}(9) \in [4.28, 4.77] $ 

  This solution corresponds to distractor 3. 
\item $ f^{-1}(9) \in [5.59, 5.99] $ 

  This solution corresponds to distractor 4. 
\end{enumerate} 
 
Natural log and exponential functions always have an inverse. Once you switch the $x$ and $y$, use the conversion $ e^y = x \leftrightarrow y=\ln(x)$.

-----------------------------------------------

62. Multiply the following functions, then choose the domain of the resulting function from the list below.
\[ f(x) = \frac{4}{4x+23} \text{ and } g(x) = \frac{2}{3x+11} \] 
The solution is $ \text{ The domain is all Real numbers except } x = -5.75 \text{ and } x = -3.66666666667 $ 

\begin{enumerate}[label=\Alph*.] 
\item $ \text{ The domain is all Real numbers except } x = a, \text{ where } a \in [-8, -3] $ 

  
\item $ \text{ The domain is all Real numbers less than or equal to } x = a, \text{ where } a \in [-8, 0] $ 

  
\item $ \text{ The domain is all Real numbers greater than or equal to } x = a, \text{ where } a \in [1, 7] $ 

  
\item $ \text{ The domain is all Real numbers except } x = a \text{ and } x = b, \text{ where } a \in [-7, -5] \text{ and } b \in [-4, 5] $ 

  
\item $ \text{ The domain is all Real numbers. } $ 

  
\end{enumerate} 
 
General Comments: The new domain is the intersection of the previous domains.

-----------------------------------------------

63. Find the inverse of the function below (if it exists). Then, evaluate the inverse at $x = 10$ and choose the interval that $f^{-1}(10)$ belongs to.
\[ f(x) = 5 x^2 + 2 \] 
The solution is $ \text{ The function is not invertible for all Real numbers. } $ 

\begin{enumerate}[label=\Alph*.] 
\item $ f^{-1}(10) \in [1.41, 1.58] $ 

  Distractor 2: This corresponds to finding the (nonexistent) inverse and not subtracting by the vertical shift. 
\item $ f^{-1}(10) \in [0.84, 1.45] $ 

  Distractor 1: This corresponds to trying to find the inverse even though the function is not 1-1.  
\item $ f^{-1}(10) \in [5.08, 5.58] $ 

  Distractor 4: This corresponds to both distractors 2 and 3. 
\item $ f^{-1}(10) \in [4.12, 4.6] $ 

  Distractor 3: This corresponds to finding the (nonexistent) inverse and dividing by a negative. 
\item $ \text{ The function is not invertible for all Real numbers. } $ 

 * This is the correct option. 
\end{enumerate} 
 
General Comments: Be sure you check that the function is 1-1 before trying to find the inverse!

-----------------------------------------------

64. Choose the interval below that $f$ composed with $g$ at $x=1$ is in.
\[ f(x) = 2x^{3} +3 x^{2} -3 x \text{ and } g(x) = 3x^{3} -4 x^{2} +4 x \] 
The solution is $ 72.0 $ 

\begin{enumerate}[label=\Alph*.] 
\item $ (f \circ g)(1) \in [55, 69] $ 

  Distractor 2: Corresponds to being slightly off from the solution. 
\item $ (f \circ g)(1) \in [15, 19] $ 

  Distractor 1: Corresponds to reversing the composition. 
\item $ (f \circ g)(1) \in [3, 14] $ 

  Distractor 3: Corresponds to being slightly off from the solution. 
\item $ (f \circ g)(1) \in [71, 76] $ 

 * This is the correct solution 
\item $ \text{It is not possible to compose the two functions.} $ 

  
\end{enumerate} 
 
General Comments: $f$ composed with $g$ at $x$ means $f(g(x))$. The order matters!

-----------------------------------------------

65. Determine whether the function below is 1-1.
\[ f(x) = 36 x^2 - 264 x + 484 \] 
The solution is $ \text{no} $ 

\begin{enumerate}[label=\Alph*.] 
\item $ \text{Yes, the function is 1-1.} $ 

 Corresponds to believing the function passes the Horizontal Line test. 
\item $ \text{No, because there is a $y$-value that goes to 2 different $x$-values.} $ 

 * This is the solution. 
\item $ \text{No, because there is an $x$-value that goes to 2 different $y$-values.} $ 

 Corresponds to the Vertical Line test, which checks if an expression is a function. 
\item $ \text{No, because the range of the function is not $(-\infty, \infty)$.} $ 

 Corresponds to believing 1-1 means the range is all Real numbers. 
\item $ \text{No, because the domain of the function is not $(-\infty, \infty)$.} $ 

 Corresponds to believing 1-1 means the domain is all Real numbers. 
\end{enumerate} 
 
\textbf{General Comments:} There are only two valid options: The function is 1-1 OR No because there is a $y$-value that goes to 2 different $x$-values.

-----------------------------------------------


\end{document}

