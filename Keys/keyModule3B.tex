\documentclass{extbook}[14pt]
\usepackage{multicol, enumerate, enumitem, hyperref, color, soul, setspace, parskip, fancyhdr, amssymb, amsthm, amsmath, latexsym, units, mathtools}
\everymath{\displaystyle}
\usepackage[headsep=0.5cm,headheight=0cm, left=1 in,right= 1 in,top= 1 in,bottom= 1 in]{geometry}
\usepackage{dashrule}  % Package to use the command below to create lines between items
\newcommand{\litem}[1]{\item #1

\rule{\textwidth}{0.4pt}}
\pagestyle{fancy}
\lhead{}
\chead{Answer Key for Progress Quiz 7 Version B}
\rhead{}
\lfoot{4173-5738}
\cfoot{}
\rfoot{Spring 2021}
\begin{document}
\textbf{This key should allow you to understand why you choose the option you did (beyond just getting a question right or wrong). \href{https://xronos.clas.ufl.edu/mac1105spring2020/courseDescriptionAndMisc/Exams/LearningFromResults}{More instructions on how to use this key can be found here}.}

\textbf{If you have a suggestion to make the keys better, \href{https://forms.gle/CZkbZmPbC9XALEE88}{please fill out the short survey here}.}

\textit{Note: This key is auto-generated and may contain issues and/or errors. The keys are reviewed after each exam to ensure grading is done accurately. If there are issues (like duplicate options), they are noted in the offline gradebook. The keys are a work-in-progress to give students as many resources to improve as possible.}

\rule{\textwidth}{0.4pt}

\begin{enumerate}\litem{
Solve the linear inequality below. Then, choose the constant and interval combination that describes the solution set.
\[ -6 + 7 x < \frac{61 x + 8}{8} \leq -7 + 5 x \]The solution is \( (-11.20, -3.05] \), which is option D.\begin{enumerate}[label=\Alph*.]
\item \( (-\infty, a] \cup (b, \infty), \text{ where } a \in [-13.5, -6.75] \text{ and } b \in [-4.5, -2.25] \)

$(-\infty, -11.20] \cup (-3.05, \infty)$, which corresponds to displaying the and-inequality as an or-inequality AND flipping the inequality.
\item \( (-\infty, a) \cup [b, \infty), \text{ where } a \in [-12.75, -9] \text{ and } b \in [-5.25, -0.75] \)

$(-\infty, -11.20) \cup [-3.05, \infty)$, which corresponds to displaying the and-inequality as an or-inequality.
\item \( [a, b), \text{ where } a \in [-16.5, -9] \text{ and } b \in [-3.75, 2.25] \)

$[-11.20, -3.05)$, which corresponds to flipping the inequality.
\item \( (a, b], \text{ where } a \in [-15.75, -7.5] \text{ and } b \in [-9.75, -1.5] \)

* $(-11.20, -3.05]$, which is the correct option.
\item \( \text{None of the above.} \)


\end{enumerate}

\textbf{General Comment:} To solve, you will need to break up the compound inequality into two inequalities. Be sure to keep track of the inequality! It may be best to draw a number line and graph your solution.
}
\litem{
Using an interval or intervals, describe all the $x$-values within or including a distance of the given values.
\[ \text{ No more than } 7 \text{ units from the number } -4. \]The solution is \( [-11, 3] \), which is option D.\begin{enumerate}[label=\Alph*.]
\item \( (-\infty, -11] \cup [3, \infty) \)

This describes the values no less than 7 from -4
\item \( (-\infty, -11) \cup (3, \infty) \)

This describes the values more than 7 from -4
\item \( (-11, 3) \)

This describes the values less than 7 from -4
\item \( [-11, 3] \)

This describes the values no more than 7 from -4
\item \( \text{None of the above} \)

You likely thought the values in the interval were not correct.
\end{enumerate}

\textbf{General Comment:} When thinking about this language, it helps to draw a number line and try points.
}
\litem{
Solve the linear inequality below. Then, choose the constant and interval combination that describes the solution set.
\[ \frac{-9}{9} - \frac{10}{8} x < \frac{6}{6} x + \frac{9}{5} \]The solution is \( (-1.244, \infty) \), which is option C.\begin{enumerate}[label=\Alph*.]
\item \( (-\infty, a), \text{ where } a \in [-0.3, 2.17] \)

 $(-\infty, 1.244)$, which corresponds to switching the direction of the interval AND negating the endpoint. You likely did this if you did not flip the inequality when dividing by a negative as well as not moving values over to a side properly.
\item \( (a, \infty), \text{ where } a \in [0, 6] \)

 $(1.244, \infty)$, which corresponds to negating the endpoint of the solution.
\item \( (a, \infty), \text{ where } a \in [-2.25, -0.75] \)

* $(-1.244, \infty)$, which is the correct option.
\item \( (-\infty, a), \text{ where } a \in [-1.65, -0.82] \)

 $(-\infty, -1.244)$, which corresponds to switching the direction of the interval. You likely did this if you did not flip the inequality when dividing by a negative!
\item \( \text{None of the above}. \)

You may have chosen this if you thought the inequality did not match the ends of the intervals.
\end{enumerate}

\textbf{General Comment:} Remember that less/greater than or equal to includes the endpoint, while less/greater do not. Also, remember that you need to flip the inequality when you multiply or divide by a negative.
}
\litem{
Solve the linear inequality below. Then, choose the constant and interval combination that describes the solution set.
\[ 3 - 5 x < \frac{-17 x - 3}{5} \leq 9 - 4 x \]The solution is \( (2.25, 16.00] \), which is option D.\begin{enumerate}[label=\Alph*.]
\item \( (-\infty, a) \cup [b, \infty), \text{ where } a \in [1.5, 6.75] \text{ and } b \in [12.75, 20.25] \)

$(-\infty, 2.25) \cup [16.00, \infty)$, which corresponds to displaying the and-inequality as an or-inequality.
\item \( (-\infty, a] \cup (b, \infty), \text{ where } a \in [0.75, 9.75] \text{ and } b \in [11.25, 16.5] \)

$(-\infty, 2.25] \cup (16.00, \infty)$, which corresponds to displaying the and-inequality as an or-inequality AND flipping the inequality.
\item \( [a, b), \text{ where } a \in [-1.5, 7.5] \text{ and } b \in [15, 20.25] \)

$[2.25, 16.00)$, which corresponds to flipping the inequality.
\item \( (a, b], \text{ where } a \in [1.5, 9.75] \text{ and } b \in [13.5, 18] \)

* $(2.25, 16.00]$, which is the correct option.
\item \( \text{None of the above.} \)


\end{enumerate}

\textbf{General Comment:} To solve, you will need to break up the compound inequality into two inequalities. Be sure to keep track of the inequality! It may be best to draw a number line and graph your solution.
}
\litem{
Using an interval or intervals, describe all the $x$-values within or including a distance of the given values.
\[ \text{ No more than } 6 \text{ units from the number } -3. \]The solution is \( [-9, 3] \), which is option B.\begin{enumerate}[label=\Alph*.]
\item \( (-\infty, -9] \cup [3, \infty) \)

This describes the values no less than 6 from -3
\item \( [-9, 3] \)

This describes the values no more than 6 from -3
\item \( (-9, 3) \)

This describes the values less than 6 from -3
\item \( (-\infty, -9) \cup (3, \infty) \)

This describes the values more than 6 from -3
\item \( \text{None of the above} \)

You likely thought the values in the interval were not correct.
\end{enumerate}

\textbf{General Comment:} When thinking about this language, it helps to draw a number line and try points.
}
\litem{
Solve the linear inequality below. Then, choose the constant and interval combination that describes the solution set.
\[ -9 + 5 x > 8 x \text{ or } -6 + 4 x < 7 x \]The solution is \( (-\infty, -3.0) \text{ or } (-2.0, \infty) \), which is option D.\begin{enumerate}[label=\Alph*.]
\item \( (-\infty, a] \cup [b, \infty), \text{ where } a \in [0, 3.75] \text{ and } b \in [1.5, 8.25] \)

Corresponds to including the endpoints AND negating.
\item \( (-\infty, a) \cup (b, \infty), \text{ where } a \in [0.75, 2.25] \text{ and } b \in [0.75, 3.75] \)

Corresponds to inverting the inequality and negating the solution.
\item \( (-\infty, a] \cup [b, \infty), \text{ where } a \in [-3.75, -2.25] \text{ and } b \in [-3.75, -0.75] \)

Corresponds to including the endpoints (when they should be excluded).
\item \( (-\infty, a) \cup (b, \infty), \text{ where } a \in [-3.75, -1.5] \text{ and } b \in [-6, 0] \)

 * Correct option.
\item \( (-\infty, \infty) \)

Corresponds to the variable canceling, which does not happen in this instance.
\end{enumerate}

\textbf{General Comment:} When multiplying or dividing by a negative, flip the sign.
}
\litem{
Solve the linear inequality below. Then, choose the constant and interval combination that describes the solution set.
\[ -5 + 5 x > 7 x \text{ or } -3 + 6 x < 9 x \]The solution is \( (-\infty, -2.5) \text{ or } (-1.0, \infty) \), which is option A.\begin{enumerate}[label=\Alph*.]
\item \( (-\infty, a) \cup (b, \infty), \text{ where } a \in [-4.5, -0.75] \text{ and } b \in [-5.25, 1.5] \)

 * Correct option.
\item \( (-\infty, a] \cup [b, \infty), \text{ where } a \in [-8.25, 0.75] \text{ and } b \in [-2.17, -0.3] \)

Corresponds to including the endpoints (when they should be excluded).
\item \( (-\infty, a) \cup (b, \infty), \text{ where } a \in [-2.25, 6] \text{ and } b \in [-0.75, 5.25] \)

Corresponds to inverting the inequality and negating the solution.
\item \( (-\infty, a] \cup [b, \infty), \text{ where } a \in [-0.75, 5.25] \text{ and } b \in [2.17, 2.55] \)

Corresponds to including the endpoints AND negating.
\item \( (-\infty, \infty) \)

Corresponds to the variable canceling, which does not happen in this instance.
\end{enumerate}

\textbf{General Comment:} When multiplying or dividing by a negative, flip the sign.
}
\litem{
Solve the linear inequality below. Then, choose the constant and interval combination that describes the solution set.
\[ -7x -8 > 5x + 5 \]The solution is \( (-\infty, -1.083) \), which is option B.\begin{enumerate}[label=\Alph*.]
\item \( (a, \infty), \text{ where } a \in [1.08, 8.08] \)

 $(1.083, \infty)$, which corresponds to switching the direction of the interval AND negating the endpoint. You likely did this if you did not flip the inequality when dividing by a negative as well as not moving values over to a side properly.
\item \( (-\infty, a), \text{ where } a \in [-2.08, -0.08] \)

* $(-\infty, -1.083)$, which is the correct option.
\item \( (-\infty, a), \text{ where } a \in [0.08, 5.08] \)

 $(-\infty, 1.083)$, which corresponds to negating the endpoint of the solution.
\item \( (a, \infty), \text{ where } a \in [-7.08, 0.92] \)

 $(-1.083, \infty)$, which corresponds to switching the direction of the interval. You likely did this if you did not flip the inequality when dividing by a negative!
\item \( \text{None of the above}. \)

You may have chosen this if you thought the inequality did not match the ends of the intervals.
\end{enumerate}

\textbf{General Comment:} Remember that less/greater than or equal to includes the endpoint, while less/greater do not. Also, remember that you need to flip the inequality when you multiply or divide by a negative.
}
\litem{
Solve the linear inequality below. Then, choose the constant and interval combination that describes the solution set.
\[ \frac{5}{3} + \frac{5}{8} x > \frac{10}{6} x - \frac{7}{2} \]The solution is \( (-\infty, 4.96) \), which is option D.\begin{enumerate}[label=\Alph*.]
\item \( (a, \infty), \text{ where } a \in [-6.75, -1.5] \)

 $(-4.96, \infty)$, which corresponds to switching the direction of the interval AND negating the endpoint. You likely did this if you did not flip the inequality when dividing by a negative as well as not moving values over to a side properly.
\item \( (-\infty, a), \text{ where } a \in [-8.25, -3] \)

 $(-\infty, -4.96)$, which corresponds to negating the endpoint of the solution.
\item \( (a, \infty), \text{ where } a \in [2.25, 6.75] \)

 $(4.96, \infty)$, which corresponds to switching the direction of the interval. You likely did this if you did not flip the inequality when dividing by a negative!
\item \( (-\infty, a), \text{ where } a \in [3, 6.75] \)

* $(-\infty, 4.96)$, which is the correct option.
\item \( \text{None of the above}. \)

You may have chosen this if you thought the inequality did not match the ends of the intervals.
\end{enumerate}

\textbf{General Comment:} Remember that less/greater than or equal to includes the endpoint, while less/greater do not. Also, remember that you need to flip the inequality when you multiply or divide by a negative.
}
\litem{
Solve the linear inequality below. Then, choose the constant and interval combination that describes the solution set.
\[ -8x -8 \leq 10x -9 \]The solution is \( [0.056, \infty) \), which is option C.\begin{enumerate}[label=\Alph*.]
\item \( (-\infty, a], \text{ where } a \in [-0.76, 0.04] \)

 $(-\infty, -0.056]$, which corresponds to switching the direction of the interval AND negating the endpoint. You likely did this if you did not flip the inequality when dividing by a negative as well as not moving values over to a side properly.
\item \( (-\infty, a], \text{ where } a \in [-0.02, 0.2] \)

 $(-\infty, 0.056]$, which corresponds to switching the direction of the interval. You likely did this if you did not flip the inequality when dividing by a negative!
\item \( [a, \infty), \text{ where } a \in [-0, 0.07] \)

* $[0.056, \infty)$, which is the correct option.
\item \( [a, \infty), \text{ where } a \in [-0.17, 0.02] \)

 $[-0.056, \infty)$, which corresponds to negating the endpoint of the solution.
\item \( \text{None of the above}. \)

You may have chosen this if you thought the inequality did not match the ends of the intervals.
\end{enumerate}

\textbf{General Comment:} Remember that less/greater than or equal to includes the endpoint, while less/greater do not. Also, remember that you need to flip the inequality when you multiply or divide by a negative.
}
\end{enumerate}

\end{document}