\documentclass{extbook}[14pt]
\usepackage{multicol, enumerate, enumitem, hyperref, color, soul, setspace, parskip, fancyhdr, amssymb, amsthm, amsmath, bbm, latexsym, units, mathtools}
\everymath{\displaystyle}
\usepackage[headsep=0.5cm,headheight=0cm, left=1 in,right= 1 in,top= 1 in,bottom= 1 in]{geometry}
\pagestyle{fancy}
\lhead{}
\chead{Answer Key for Module\,3\,-\,Inequalities Version B}
\rhead{}
\lfoot{Summer\,C\,2020}
\cfoot{}
\rfoot{}
\begin{document}
\textbf{This key should allow you to understand why you choose the option you did (beyond just getting a question right or wrong). \href{https://xronos.clas.ufl.edu/mac1105spring2020/courseDescriptionAndMisc/Exams/LearningFromResults}{More instructions on how to use this key can be found here}.}

\textbf{If you have a suggestion to make the keys better, \href{https://forms.gle/CZkbZmPbC9XALEE88}{please fill out the short survey here}.}

\textit{Note: This key is auto-generated and may contain issues and/or errors. The keys are reviewed after each exam to ensure grading is done accurately. If there are issues (like duplicate options), they are noted in the offline gradebook. The keys are a work-in-progress to give students as many resources to improve as possible.}

\rule{\textwidth}{0.4pt}

1. Solve the linear inequality below. Then, choose the constant and interval combination that describes the solution set.
\[ \frac{-6}{9} - \frac{5}{6} x < \frac{-4}{2} x + \frac{3}{3} \] 
The solution is $ (-\infty, 1.429) $ 

\begin{enumerate}[label=\Alph*.] 
\item $ (a, \infty), \text{ where } a \in [-1.9, -0.2] $ 

  $(-1.429, \infty)$, which corresponds to switching the direction of the interval AND negating the endpoint. You likely did this if you did not flip the inequality when dividing by a negative as well as not moving values over to a side properly. 
\item $ (-\infty, a), \text{ where } a \in [-4, 1] $ 

  $(-\infty, -1.429)$, which corresponds to negating the endpoint of the solution. 
\item $ (-\infty, a), \text{ where } a \in [0, 2] $ 

 * $(-\infty, 1.429)$, which is the correct option. 
\item $ (a, \infty), \text{ where } a \in [0.5, 2.4] $ 

  $(1.429, \infty)$, which corresponds to switching the direction of the interval. You likely did this if you did not flip the inequality when dividing by a negative! 
\item $ \text{None of the above}. $ 

 You may have chosen this if you thought the inequality did not match the ends of the intervals. 
\end{enumerate} 
 
\textbf{General Comment:} General Comments: Remember that less/greater than or equal to includes the endpoint, while less/greater do not. Also, remember that you need to flip the inequality when you multiply or divide by a negative. 

-----------------------------------------------

2. Using an interval or intervals, describe all the $x$-values within or including a distance of the given values.
\[ \text{ No less than } 6 \text{ units from the number } -6. \] 
The solution is $ (-\infty, -12] \cup [0, \infty) $ 

\begin{enumerate}[label=\Alph*.] 
\item $ (-\infty, -12] \cup [0, \infty) $ 

 This describes the values no less than 6 from -6 
\item $ (-12, 0) $ 

 This describes the values less than 6 from -6 
\item $ (-\infty, -12) \cup (0, \infty) $ 

 This describes the values more than 6 from -6 
\item $ [-12, 0] $ 

 This describes the values no more than 6 from -6 
\item $ \text{None of the above} $ 

 You likely thought the values in the interval were not correct. 
\end{enumerate} 
 
\textbf{General Comment:} General Comments: When thinking about this language, it helps to draw a number line and try points. 

-----------------------------------------------

3. Solve the linear inequality below. Then, choose the constant and interval combination that describes the solution set.
\[ 9 + 7 x < \frac{75 x + 4}{9} \leq 5 + 8 x \] 
The solution is $ \text{None of the above.} $ 

\begin{enumerate}[label=\Alph*.] 
\item $ [a, b), \text{ where } a \in [-10, -1] \text{ and } b \in [-18, -8] $ 

 $[-6.42, -13.67)$, which corresponds to flipping the inequality and getting negatives of the actual endpoints. 
\item $ (-\infty, a] \cup (b, \infty), \text{ where } a \in [-8, -4] \text{ and } b \in [-15, -11] $ 

 $(-\infty, -6.42] \cup (-13.67, \infty)$, which corresponds to displaying the and-inequality as an or-inequality AND flipping the inequality AND getting negatives of the actual endpoints. 
\item $ (a, b], \text{ where } a \in [-10, -1] \text{ and } b \in [-17, -10] $ 

 $(-6.42, -13.67]$, which is the correct interval but negatives of the actual endpoints. 
\item $ (-\infty, a) \cup [b, \infty), \text{ where } a \in [-8, -1] \text{ and } b \in [-17, -12] $ 

 $(-\infty, -6.42) \cup [-13.67, \infty)$, which corresponds to displaying the and-inequality as an or-inequality and getting negatives of the actual endpoints. 
\item $ \text{None of the above.} $ 

 * This is correct as the answer should be $(6.42, 13.67]$. 
\end{enumerate} 
 
\textbf{General Comment:} To solve, you will need to break up the compound inequality into two inequalities. Be sure to keep track of the inequality! It may be best to draw a number line and graph your solution. 

-----------------------------------------------

4. Solve the linear inequality below. Then, choose the constant and interval combination that describes the solution set.
\[ -6 + 6 x > 8 x \text{ or } 4 + 8 x < 9 x \] 
The solution is $ (-\infty, -3.0) \text{ or } (4.0, \infty) $ 

\begin{enumerate}[label=\Alph*.] 
\item $ (-\infty, a) \cup (b, \infty), \text{ where } a \in [-3.74, -2.16] \text{ and } b \in [3.6, 6.3] $ 

  * Correct option. 
\item $ (-\infty, a] \cup [b, \infty), \text{ where } a \in [-4.65, -3.44] \text{ and } b \in [1.3, 3.1] $ 

 Corresponds to including the endpoints AND negating. 
\item $ (-\infty, a) \cup (b, \infty), \text{ where } a \in [-5.04, -3.94] \text{ and } b \in [2.7, 3.4] $ 

 Corresponds to inverting the inequality and negating the solution. 
\item $ (-\infty, a] \cup [b, \infty), \text{ where } a \in [-3.56, -2.77] \text{ and } b \in [3.5, 5.1] $ 

 Corresponds to including the endpoints (when they should be excluded). 
\item $ (-\infty, \infty) $ 

 Corresponds to the variable canceling, which does not happen in this instance. 
\end{enumerate} 
 
\textbf{General Comment:} General Comments: When multiplying or dividing by a negative, flip the sign. 

-----------------------------------------------

0. Solve the linear inequality below. Then, choose the constant and interval combination that describes the solution set.
\[ -3x + 3 < 5x + 4 \] 
The solution is $ (-0.125, \infty) $ 

\begin{enumerate}[label=\Alph*.] 
\item $ (a, \infty), \text{ where } a \in [-0.55, 0.07] $ 

 * $(-0.125, \infty)$, which is the correct option. 
\item $ (-\infty, a), \text{ where } a \in [0.06, 1.29] $ 

  $(-\infty, 0.125)$, which corresponds to switching the direction of the interval AND negating the endpoint. You likely did this if you did not flip the inequality when dividing by a negative as well as not moving values over to a side properly. 
\item $ (a, \infty), \text{ where } a \in [0.1, 0.31] $ 

  $(0.125, \infty)$, which corresponds to negating the endpoint of the solution. 
\item $ (-\infty, a), \text{ where } a \in [-0.49, 0.02] $ 

  $(-\infty, -0.125)$, which corresponds to switching the direction of the interval. You likely did this if you did not flip the inequality when dividing by a negative! 
\item $ \text{None of the above}. $ 

 You may have chosen this if you thought the inequality did not match the ends of the intervals. 
\end{enumerate} 
 
\textbf{General Comment:} General Comments: Remember that less/greater than or equal to includes the endpoint, while less/greater do not. Also, remember that you need to flip the inequality when you multiply or divide by a negative. 

-----------------------------------------------


\end{document}

