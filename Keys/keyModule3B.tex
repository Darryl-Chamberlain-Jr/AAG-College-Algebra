\documentclass{extbook}[14pt]
\usepackage{multicol, enumerate, enumitem, hyperref, color, soul, setspace, parskip, fancyhdr, amssymb, amsthm, amsmath, bbm, latexsym, units, mathtools}
\everymath{\displaystyle}
\usepackage[headsep=0.5cm,headheight=0cm, left=1 in,right= 1 in,top= 1 in,bottom= 1 in]{geometry}
\pagestyle{fancy}
\lhead{}
\chead{Answer Key for Module\,3\,-\,Inequalities Version B}
\rhead{}
\lfoot{Summer\,C\,2020}
\cfoot{}
\rfoot{}
\begin{document}
\textbf{This key should allow you to understand why you choose the option you did (beyond just getting a question right or wrong). \href{https://xronos.clas.ufl.edu/mac1105spring2020/courseDescriptionAndMisc/Exams/LearningFromResults}{More instructions on how to use this key can be found here}.}

\textbf{If you have a suggestion to make the keys better, \href{https://forms.gle/CZkbZmPbC9XALEE88}{please fill out the short survey here}.}

\textit{Note: This key is auto-generated and may contain issues and/or errors. The keys are reviewed after each exam to ensure grading is done accurately. If there are issues (like duplicate options), they are noted in the offline gradebook. The keys are a work-in-progress to give students as many resources to improve as possible.}

\rule{\textwidth}{0.4pt}

11. Using an interval or intervals, describe all the $x$-values within or including a distance of the given values.
\[ \text{ More than } 8 \text{ units from the number } -7. \] 
The solution is $ (-\infty, -15) \cup (1, \infty) $ 

\begin{enumerate}[label=\Alph*.] 
\item $ [-15, 1] $ 

 This describes the values no more than 8 from -7 
\item $ (-15, 1) $ 

 This describes the values less than 8 from -7 
\item $ (-\infty, -15) \cup (1, \infty) $ 

 This describes the values more than 8 from -7 
\item $ (-\infty, -15] \cup [1, \infty) $ 

 This describes the values no less than 8 from -7 
\item $ \text{None of the above} $ 

 You likely thought the values in the interval were not correct. 
\end{enumerate} 
 
General Comments: When thinking about this language, it helps to draw a number line and try points.

-----------------------------------------------

12. Solve the linear inequality below. Then, choose the constant and interval combination that describes the solution set.
\[ -4x + 5 < -3x + 10 \] 
The solution is $ (-5.0, \infty) $ 

\begin{enumerate}[label=\Alph*.] 
\item $ (-\infty, a), \text{ where } a \in [-6, -2] $ 

  $(-\infty, -5.0)$, which corresponds to switching the direction of the interval. You likely did this if you did not flip the inequality when dividing by a negative! 
\item $ (a, \infty), \text{ where } a \in [-14, -4] $ 

 * $(-5.0, \infty)$, which is the correct option. 
\item $ (-\infty, a), \text{ where } a \in [4, 7] $ 

  $(-\infty, 5.0)$, which corresponds to switching the direction of the interval AND negating the endpoint. You likely did this if you did not flip the inequality when dividing by a negative as well as not moving values over to a side properly. 
\item $ (a, \infty), \text{ where } a \in [4, 12] $ 

  $(5.0, \infty)$, which corresponds to negating the endpoint of the solution. 
\item $ \text{None of the above}. $ 

 You may have chosen this if you thought the inequality did not match the ends of the intervals. 
\end{enumerate} 
 
General Comments: Remember that less/greater than or equal to includes the endpoint, while less/greater do not. Also, remember that you need to flip the inequality when you multiply or divide by a negative.

-----------------------------------------------

13. Solve the linear inequality below. Then, choose the constant and interval combination that describes the solution set.
\[ -5 + 7 x > 8 x \text{ or } -4 + 4 x < 5 x \] 
The solution is $ (-\infty, -5.0) \text{ or } (-4.0, \infty) $ 

\begin{enumerate}[label=\Alph*.] 
\item $ (-\infty, a] \cup [b, \infty), \text{ where } a \in [-6, -4] \text{ and } b \in [-5, -3] $ 

 Corresponds to including the endpoints (when they should be excluded). 
\item $ (-\infty, a) \cup (b, \infty), \text{ where } a \in [-14, 0] \text{ and } b \in [-6, 1] $ 

  * Correct option. 
\item $ (-\infty, a] \cup [b, \infty), \text{ where } a \in [2, 8] \text{ and } b \in [3, 6] $ 

 Corresponds to including the endpoints AND negating. 
\item $ (-\infty, a) \cup (b, \infty), \text{ where } a \in [0, 5] \text{ and } b \in [-1, 7] $ 

 Corresponds to inverting the inequality and negating the solution. 
\item $ (-\infty, \infty) $ 

 Corresponds to the variable canceling, which does not happen in this instance. 
\end{enumerate} 
 
General Comments: When multiplying or dividing by a negative, flip the sign.

-----------------------------------------------

14. Solve the linear inequality below. Then, choose the constant and interval combination that describes the solution set.
\[ \frac{6}{4} + \frac{3}{8} x < \frac{10}{3} x - \frac{8}{2} \] 
The solution is $ (1.859, \infty) $ 

\begin{enumerate}[label=\Alph*.] 
\item $ (a, \infty), \text{ where } a \in [1, 3] $ 

 * $(1.859, \infty)$, which is the correct option. 
\item $ (-\infty, a), \text{ where } a \in [-3, 0] $ 

  $(-\infty, -1.859)$, which corresponds to switching the direction of the interval AND negating the endpoint. You likely did this if you did not flip the inequality when dividing by a negative as well as not moving values over to a side properly. 
\item $ (a, \infty), \text{ where } a \in [-3, 1] $ 

  $(-1.859, \infty)$, which corresponds to negating the endpoint of the solution. 
\item $ (-\infty, a), \text{ where } a \in [1, 4] $ 

  $(-\infty, 1.859)$, which corresponds to switching the direction of the interval. You likely did this if you did not flip the inequality when dividing by a negative! 
\item $ \text{None of the above}. $ 

 You may have chosen this if you thought the inequality did not match the ends of the intervals. 
\end{enumerate} 
 
General Comments: Remember that less/greater than or equal to includes the endpoint, while less/greater do not. Also, remember that you need to flip the inequality when you multiply or divide by a negative.

-----------------------------------------------

15. Solve the linear inequality below. Then, choose the constant and interval combination that describes the solution set.
\[ -9 + 7 x < \frac{58 x - 4}{8} \leq 3 + 5 x \] 
The solution is $ \text{None of the above.} $ 

\begin{enumerate}[label=\Alph*.] 
\item $ (a, b], \text{ where } a \in [31, 36] \text{ and } b \in [-3, 0] $ 

 $(34.00, -1.56]$, which is the correct interval but negatives of the actual endpoints. 
\item $ [a, b), \text{ where } a \in [29, 36] \text{ and } b \in [-6, 0] $ 

 $[34.00, -1.56)$, which corresponds to flipping the inequality and getting negatives of the actual endpoints. 
\item $ (-\infty, a) \cup [b, \infty), \text{ where } a \in [33, 35] \text{ and } b \in [-4, 0] $ 

 $(-\infty, 34.00) \cup [-1.56, \infty)$, which corresponds to displaying the and-inequality as an or-inequality and getting negatives of the actual endpoints. 
\item $ (-\infty, a] \cup (b, \infty), \text{ where } a \in [31, 37] \text{ and } b \in [-3, -1] $ 

 $(-\infty, 34.00] \cup (-1.56, \infty)$, which corresponds to displaying the and-inequality as an or-inequality AND flipping the inequality AND getting negatives of the actual endpoints. 
\item $ \text{None of the above.} $ 

 * This is correct as the answer should be $(-34.00, 1.56]$. 
\end{enumerate} 
 
To solve, you will need to break up the compound inequality into two inequalities. Be sure to keep track of the inequality! It may be best to draw a number line and graph your solution.

-----------------------------------------------


\end{document}

