\documentclass{extbook}[14pt]
\usepackage{multicol, enumerate, enumitem, hyperref, color, soul, setspace, parskip, fancyhdr, amssymb, amsthm, amsmath, bbm, latexsym, units, mathtools}
\everymath{\displaystyle}
\usepackage[headsep=0.5cm,headheight=0cm, left=1 in,right= 1 in,top= 1 in,bottom= 1 in]{geometry}
\usepackage{dashrule}  % Package to use the command below to create lines between items
\newcommand{\litem}[1]{\item #1

\rule{\textwidth}{0.4pt}}
\pagestyle{fancy}
\lhead{}
\chead{Answer Key for Progress Quiz 2 Version B}
\rhead{}
\lfoot{7862-5421}
\cfoot{}
\rfoot{Spring 2021}
\begin{document}
\textbf{This key should allow you to understand why you choose the option you did (beyond just getting a question right or wrong). \href{https://xronos.clas.ufl.edu/mac1105spring2020/courseDescriptionAndMisc/Exams/LearningFromResults}{More instructions on how to use this key can be found here}.}

\textbf{If you have a suggestion to make the keys better, \href{https://forms.gle/CZkbZmPbC9XALEE88}{please fill out the short survey here}.}

\textit{Note: This key is auto-generated and may contain issues and/or errors. The keys are reviewed after each exam to ensure grading is done accurately. If there are issues (like duplicate options), they are noted in the offline gradebook. The keys are a work-in-progress to give students as many resources to improve as possible.}

\rule{\textwidth}{0.4pt}

\begin{enumerate}\litem{
Solve the linear inequality below. Then, choose the constant and interval combination that describes the solution set.
\[ 8 - 4 x < \frac{25 x - 6}{5} \leq 4 + 4 x \]

The solution is \( \text{None of the above.} \), which is option E.\begin{enumerate}[label=\Alph*.]
\item \( (-\infty, a] \cup (b, \infty), \text{ where } a \in [-2.02, 0.98] \text{ and } b \in [-8.2, -1.2] \)

$(-\infty, -1.02] \cup (-5.20, \infty)$, which corresponds to displaying the and-inequality as an or-inequality AND flipping the inequality AND getting negatives of the actual endpoints.
\item \( (a, b], \text{ where } a \in [-3.1, 0.1] \text{ and } b \in [-9.2, 0.8] \)

$(-1.02, -5.20]$, which is the correct interval but negatives of the actual endpoints.
\item \( [a, b), \text{ where } a \in [-1.4, 0.7] \text{ and } b \in [-8.2, -4.2] \)

$[-1.02, -5.20)$, which corresponds to flipping the inequality and getting negatives of the actual endpoints.
\item \( (-\infty, a) \cup [b, \infty), \text{ where } a \in [-3.02, -0.02] \text{ and } b \in [-7.2, 0.8] \)

$(-\infty, -1.02) \cup [-5.20, \infty)$, which corresponds to displaying the and-inequality as an or-inequality and getting negatives of the actual endpoints.
\item \( \text{None of the above.} \)

* This is correct as the answer should be $(1.02, 5.20]$.
\end{enumerate}

\textbf{General Comment:} To solve, you will need to break up the compound inequality into two inequalities. Be sure to keep track of the inequality! It may be best to draw a number line and graph your solution.
}
\litem{
Solve the linear inequality below. Then, choose the constant and interval combination that describes the solution set.
\[ -9x + 6 \leq 5x -10 \]

The solution is \( [1.143, \infty) \), which is option B.\begin{enumerate}[label=\Alph*.]
\item \( [a, \infty), \text{ where } a \in [-1.4, -0.1] \)

 $[-1.143, \infty)$, which corresponds to negating the endpoint of the solution.
\item \( [a, \infty), \text{ where } a \in [0.8, 1.7] \)

* $[1.143, \infty)$, which is the correct option.
\item \( (-\infty, a], \text{ where } a \in [-1.6, 0.1] \)

 $(-\infty, -1.143]$, which corresponds to switching the direction of the interval AND negating the endpoint. You likely did this if you did not flip the inequality when dividing by a negative as well as not moving values over to a side properly.
\item \( (-\infty, a], \text{ where } a \in [-0.5, 2.2] \)

 $(-\infty, 1.143]$, which corresponds to switching the direction of the interval. You likely did this if you did not flip the inequality when dividing by a negative!
\item \( \text{None of the above}. \)

You may have chosen this if you thought the inequality did not match the ends of the intervals.
\end{enumerate}

\textbf{General Comment:} Remember that less/greater than or equal to includes the endpoint, while less/greater do not. Also, remember that you need to flip the inequality when you multiply or divide by a negative.
}
\litem{
Solve the linear inequality below. Then, choose the constant and interval combination that describes the solution set.
\[ \frac{7}{7} + \frac{9}{5} x < \frac{10}{3} x - \frac{7}{8} \]

The solution is \( (1.223, \infty) \), which is option B.\begin{enumerate}[label=\Alph*.]
\item \( (-\infty, a), \text{ where } a \in [-3.22, 0.78] \)

 $(-\infty, -1.223)$, which corresponds to switching the direction of the interval AND negating the endpoint. You likely did this if you did not flip the inequality when dividing by a negative as well as not moving values over to a side properly.
\item \( (a, \infty), \text{ where } a \in [0.22, 2.22] \)

* $(1.223, \infty)$, which is the correct option.
\item \( (-\infty, a), \text{ where } a \in [0.22, 2.22] \)

 $(-\infty, 1.223)$, which corresponds to switching the direction of the interval. You likely did this if you did not flip the inequality when dividing by a negative!
\item \( (a, \infty), \text{ where } a \in [-4.22, 0.78] \)

 $(-1.223, \infty)$, which corresponds to negating the endpoint of the solution.
\item \( \text{None of the above}. \)

You may have chosen this if you thought the inequality did not match the ends of the intervals.
\end{enumerate}

\textbf{General Comment:} Remember that less/greater than or equal to includes the endpoint, while less/greater do not. Also, remember that you need to flip the inequality when you multiply or divide by a negative.
}
\litem{
Using an interval or intervals, describe all the $x$-values within or including a distance of the given values.
\[ \text{ More than } 2 \text{ units from the number } -3. \]

The solution is \( (-\infty, -5) \cup (-1, \infty) \), which is option D.\begin{enumerate}[label=\Alph*.]
\item \( (-5, -1) \)

This describes the values less than 2 from -3
\item \( (-\infty, -5] \cup [-1, \infty) \)

This describes the values no less than 2 from -3
\item \( [-5, -1] \)

This describes the values no more than 2 from -3
\item \( (-\infty, -5) \cup (-1, \infty) \)

This describes the values more than 2 from -3
\item \( \text{None of the above} \)

You likely thought the values in the interval were not correct.
\end{enumerate}

\textbf{General Comment:} When thinking about this language, it helps to draw a number line and try points.
}
\litem{
Solve the linear inequality below. Then, choose the constant and interval combination that describes the solution set.
\[ -4 + 3 x > 4 x \text{ or } 8 + 8 x < 9 x \]

The solution is \( (-\infty, -4.0) \text{ or } (8.0, \infty) \), which is option D.\begin{enumerate}[label=\Alph*.]
\item \( (-\infty, a) \cup (b, \infty), \text{ where } a \in [-8.5, -6.2] \text{ and } b \in [0, 6] \)

Corresponds to inverting the inequality and negating the solution.
\item \( (-\infty, a] \cup [b, \infty), \text{ where } a \in [-8, -7] \text{ and } b \in [-1, 5] \)

Corresponds to including the endpoints AND negating.
\item \( (-\infty, a] \cup [b, \infty), \text{ where } a \in [-7, -2] \text{ and } b \in [8, 16] \)

Corresponds to including the endpoints (when they should be excluded).
\item \( (-\infty, a) \cup (b, \infty), \text{ where } a \in [-4.3, -1.7] \text{ and } b \in [8, 9] \)

 * Correct option.
\item \( (-\infty, \infty) \)

Corresponds to the variable canceling, which does not happen in this instance.
\end{enumerate}

\textbf{General Comment:} When multiplying or dividing by a negative, flip the sign.
}
\litem{
Solve the linear inequality below. Then, choose the constant and interval combination that describes the solution set.
\[ -7x -8 > 9x -7 \]

The solution is \( (-\infty, -0.062) \), which is option D.\begin{enumerate}[label=\Alph*.]
\item \( (a, \infty), \text{ where } a \in [-0.2, -0.06] \)

 $(-0.062, \infty)$, which corresponds to switching the direction of the interval. You likely did this if you did not flip the inequality when dividing by a negative!
\item \( (a, \infty), \text{ where } a \in [-0.03, 0.2] \)

 $(0.062, \infty)$, which corresponds to switching the direction of the interval AND negating the endpoint. You likely did this if you did not flip the inequality when dividing by a negative as well as not moving values over to a side properly.
\item \( (-\infty, a), \text{ where } a \in [-0.01, 0.08] \)

 $(-\infty, 0.062)$, which corresponds to negating the endpoint of the solution.
\item \( (-\infty, a), \text{ where } a \in [-0.17, -0.01] \)

* $(-\infty, -0.062)$, which is the correct option.
\item \( \text{None of the above}. \)

You may have chosen this if you thought the inequality did not match the ends of the intervals.
\end{enumerate}

\textbf{General Comment:} Remember that less/greater than or equal to includes the endpoint, while less/greater do not. Also, remember that you need to flip the inequality when you multiply or divide by a negative.
}
\litem{
Solve the linear inequality below. Then, choose the constant and interval combination that describes the solution set.
\[ \frac{-6}{8} + \frac{5}{4} x \geq \frac{8}{5} x + \frac{10}{3} \]

The solution is \( (-\infty, -11.667] \), which is option A.\begin{enumerate}[label=\Alph*.]
\item \( (-\infty, a], \text{ where } a \in [-11.67, -9.67] \)

* $(-\infty, -11.667]$, which is the correct option.
\item \( [a, \infty), \text{ where } a \in [-11.67, -9.67] \)

 $[-11.667, \infty)$, which corresponds to switching the direction of the interval. You likely did this if you did not flip the inequality when dividing by a negative!
\item \( (-\infty, a], \text{ where } a \in [10.67, 15.67] \)

 $(-\infty, 11.667]$, which corresponds to negating the endpoint of the solution.
\item \( [a, \infty), \text{ where } a \in [10.67, 16.67] \)

 $[11.667, \infty)$, which corresponds to switching the direction of the interval AND negating the endpoint. You likely did this if you did not flip the inequality when dividing by a negative as well as not moving values over to a side properly.
\item \( \text{None of the above}. \)

You may have chosen this if you thought the inequality did not match the ends of the intervals.
\end{enumerate}

\textbf{General Comment:} Remember that less/greater than or equal to includes the endpoint, while less/greater do not. Also, remember that you need to flip the inequality when you multiply or divide by a negative.
}
\litem{
Using an interval or intervals, describe all the $x$-values within or including a distance of the given values.
\[ \text{ No more than } 2 \text{ units from the number } 4. \]

The solution is \( \text{None of the above} \), which is option E.\begin{enumerate}[label=\Alph*.]
\item \( (-2, 6) \)

This describes the values less than 4 from 2
\item \( (-\infty, -2) \cup (6, \infty) \)

This describes the values more than 4 from 2
\item \( (-\infty, -2] \cup [6, \infty) \)

This describes the values no less than 4 from 2
\item \( [-2, 6] \)

This describes the values no more than 4 from 2
\item \( \text{None of the above} \)

Options A-D described the values [more/less than] 4 units from 2, which is the reverse of what the question asked.
\end{enumerate}

\textbf{General Comment:} When thinking about this language, it helps to draw a number line and try points.
}
\litem{
Solve the linear inequality below. Then, choose the constant and interval combination that describes the solution set.
\[ 7 - 8 x < \frac{-41 x - 6}{9} \leq 4 - 5 x \]

The solution is \( (2.23, 10.50] \), which is option C.\begin{enumerate}[label=\Alph*.]
\item \( (-\infty, a) \cup [b, \infty), \text{ where } a \in [0.23, 4.23] \text{ and } b \in [10.5, 15.5] \)

$(-\infty, 2.23) \cup [10.50, \infty)$, which corresponds to displaying the and-inequality as an or-inequality.
\item \( [a, b), \text{ where } a \in [1.23, 5.23] \text{ and } b \in [8.5, 14.5] \)

$[2.23, 10.50)$, which corresponds to flipping the inequality.
\item \( (a, b], \text{ where } a \in [-0.77, 4.23] \text{ and } b \in [5.5, 13.5] \)

* $(2.23, 10.50]$, which is the correct option.
\item \( (-\infty, a] \cup (b, \infty), \text{ where } a \in [-0.77, 4.23] \text{ and } b \in [8.5, 11.5] \)

$(-\infty, 2.23] \cup (10.50, \infty)$, which corresponds to displaying the and-inequality as an or-inequality AND flipping the inequality.
\item \( \text{None of the above.} \)


\end{enumerate}

\textbf{General Comment:} To solve, you will need to break up the compound inequality into two inequalities. Be sure to keep track of the inequality! It may be best to draw a number line and graph your solution.
}
\litem{
Solve the linear inequality below. Then, choose the constant and interval combination that describes the solution set.
\[ -9 + 9 x > 10 x \text{ or } 7 + 5 x < 8 x \]

The solution is \( (-\infty, -9.0) \text{ or } (2.333, \infty) \), which is option C.\begin{enumerate}[label=\Alph*.]
\item \( (-\infty, a) \cup (b, \infty), \text{ where } a \in [-4.33, -1.33] \text{ and } b \in [8, 10] \)

Corresponds to inverting the inequality and negating the solution.
\item \( (-\infty, a] \cup [b, \infty), \text{ where } a \in [-9, -8] \text{ and } b \in [-1.67, 5.33] \)

Corresponds to including the endpoints (when they should be excluded).
\item \( (-\infty, a) \cup (b, \infty), \text{ where } a \in [-14, -5] \text{ and } b \in [2.33, 5.33] \)

 * Correct option.
\item \( (-\infty, a] \cup [b, \infty), \text{ where } a \in [-3.33, -0.33] \text{ and } b \in [7, 10] \)

Corresponds to including the endpoints AND negating.
\item \( (-\infty, \infty) \)

Corresponds to the variable canceling, which does not happen in this instance.
\end{enumerate}

\textbf{General Comment:} When multiplying or dividing by a negative, flip the sign.
}
\end{enumerate}

\end{document}