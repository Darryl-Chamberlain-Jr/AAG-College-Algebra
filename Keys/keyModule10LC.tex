\documentclass{extbook}[14pt]
\usepackage{multicol, enumerate, enumitem, hyperref, color, soul, setspace, parskip, fancyhdr, amssymb, amsthm, amsmath, latexsym, units, mathtools}
\everymath{\displaystyle}
\usepackage[headsep=0.5cm,headheight=0cm, left=1 in,right= 1 in,top= 1 in,bottom= 1 in]{geometry}
\usepackage{dashrule}  % Package to use the command below to create lines between items
\newcommand{\litem}[1]{\item #1

\rule{\textwidth}{0.4pt}}
\pagestyle{fancy}
\lhead{}
\chead{Answer Key for Progress Quiz 7 Version C}
\rhead{}
\lfoot{4173-5738}
\cfoot{}
\rfoot{Spring 2021}
\begin{document}
\textbf{This key should allow you to understand why you choose the option you did (beyond just getting a question right or wrong). \href{https://xronos.clas.ufl.edu/mac1105spring2020/courseDescriptionAndMisc/Exams/LearningFromResults}{More instructions on how to use this key can be found here}.}

\textbf{If you have a suggestion to make the keys better, \href{https://forms.gle/CZkbZmPbC9XALEE88}{please fill out the short survey here}.}

\textit{Note: This key is auto-generated and may contain issues and/or errors. The keys are reviewed after each exam to ensure grading is done accurately. If there are issues (like duplicate options), they are noted in the offline gradebook. The keys are a work-in-progress to give students as many resources to improve as possible.}

\rule{\textwidth}{0.4pt}

\begin{enumerate}\litem{
What are the \textit{possible Integer} roots of the polynomial below?
\[ f(x) = 6x^{4} +6 x^{3} +5 x^{2} +4 x + 3 \]The solution is \( \pm 1,\pm 3 \), which is option C.\begin{enumerate}[label=\Alph*.]
\item \( \pm 1,\pm 2,\pm 3,\pm 6 \)

 Distractor 1: Corresponds to the plus or minus factors of a1 only.
\item \( \text{ All combinations of: }\frac{\pm 1,\pm 2,\pm 3,\pm 6}{\pm 1,\pm 3} \)

 Distractor 3: Corresponds to the plus or minus of the inverse quotient (an/a0) of the factors. 
\item \( \pm 1,\pm 3 \)

* This is the solution \textbf{since we asked for the possible Integer roots}!
\item \( \text{ All combinations of: }\frac{\pm 1,\pm 3}{\pm 1,\pm 2,\pm 3,\pm 6} \)

This would have been the solution \textbf{if asked for the possible Rational roots}!
\item \( \text{There is no formula or theorem that tells us all possible Integer roots.} \)

 Distractor 4: Corresponds to not recognizing Integers as a subset of Rationals.
\end{enumerate}

\textbf{General Comment:} We have a way to find the possible Rational roots. The possible Integer roots are the Integers in this list.
}
\litem{
Perform the division below. Then, find the intervals that correspond to the quotient in the form $ax^2+bx+c$ and remainder $r$.
\[ \frac{8x^{3} -28 x^{2} + 38}{x -3} \]The solution is \( 8x^{2} -4 x -12 + \frac{2}{x -3} \), which is option B.\begin{enumerate}[label=\Alph*.]
\item \( a \in [7, 16], b \in [-18, -9], c \in [-28, -22], \text{ and } r \in [-11, -5]. \)

 You multipled by the synthetic number and subtracted rather than adding during synthetic division.
\item \( a \in [7, 16], b \in [-6, 1], c \in [-18, -11], \text{ and } r \in [-1, 3]. \)

* This is the solution!
\item \( a \in [17, 25], b \in [43, 46], c \in [131, 145], \text{ and } r \in [433, 440]. \)

 You multipled by the synthetic number rather than bringing the first factor down.
\item \( a \in [17, 25], b \in [-101, -97], c \in [298, 306], \text{ and } r \in [-862, -861]. \)

 You divided by the opposite of the factor AND multipled the first factor rather than just bringing it down.
\item \( a \in [7, 16], b \in [-52, -48], c \in [155, 161], \text{ and } r \in [-432, -428]. \)

 You divided by the opposite of the factor.
\end{enumerate}

\textbf{General Comment:} Be sure to synthetically divide by the zero of the denominator! Also, make sure to include 0 placeholders for missing terms.
}
\litem{
What are the \textit{possible Rational} roots of the polynomial below?
\[ f(x) = 4x^{2} +3 x + 2 \]The solution is \( \text{ All combinations of: }\frac{\pm 1,\pm 2}{\pm 1,\pm 2,\pm 4} \), which is option D.\begin{enumerate}[label=\Alph*.]
\item \( \pm 1,\pm 2,\pm 4 \)

 Distractor 1: Corresponds to the plus or minus factors of a1 only.
\item \( \text{ All combinations of: }\frac{\pm 1,\pm 2,\pm 4}{\pm 1,\pm 2} \)

 Distractor 3: Corresponds to the plus or minus of the inverse quotient (an/a0) of the factors. 
\item \( \pm 1,\pm 2 \)

This would have been the solution \textbf{if asked for the possible Integer roots}!
\item \( \text{ All combinations of: }\frac{\pm 1,\pm 2}{\pm 1,\pm 2,\pm 4} \)

* This is the solution \textbf{since we asked for the possible Rational roots}!
\item \( \text{ There is no formula or theorem that tells us all possible Rational roots.} \)

 Distractor 4: Corresponds to not recalling the theorem for rational roots of a polynomial.
\end{enumerate}

\textbf{General Comment:} We have a way to find the possible Rational roots. The possible Integer roots are the Integers in this list.
}
\litem{
Factor the polynomial below completely. Then, choose the intervals the zeros of the polynomial belong to, where $z_1 \leq z_2 \leq z_3$. \textit{To make the problem easier, all zeros are between -5 and 5.}
\[ f(x) = 20x^{3} -63 x^{2} +52 x -12 \]The solution is \( [0.4, 0.75, 2] \), which is option B.\begin{enumerate}[label=\Alph*.]
\item \( z_1 \in [-2.21, -1.53], \text{   }  z_2 \in [-2.59, -1.38], \text{   and   } z_3 \in [-0.28, -0.08] \)

 Distractor 4: Corresponds to moving factors from one rational to another.
\item \( z_1 \in [-0.02, 0.94], \text{   }  z_2 \in [0.37, 0.81], \text{   and   } z_3 \in [1.78, 2.03] \)

* This is the solution!
\item \( z_1 \in [-2.69, -2.28], \text{   }  z_2 \in [-2.59, -1.38], \text{   and   } z_3 \in [-2.11, -1.3] \)

 Distractor 3: Corresponds to negatives of all zeros AND inversing rational roots.
\item \( z_1 \in [1.28, 1.44], \text{   }  z_2 \in [1.95, 2.12], \text{   and   } z_3 \in [2.18, 2.87] \)

 Distractor 2: Corresponds to inversing rational roots.
\item \( z_1 \in [-2.21, -1.53], \text{   }  z_2 \in [-1.14, -0.36], \text{   and   } z_3 \in [-0.49, -0.34] \)

 Distractor 1: Corresponds to negatives of all zeros.
\end{enumerate}

\textbf{General Comment:} Remember to try the middle-most integers first as these normally are the zeros. Also, once you get it to a quadratic, you can use your other factoring techniques to finish factoring.
}
\litem{
Factor the polynomial below completely, knowing that $x+2$ is a factor. Then, choose the intervals the zeros of the polynomial belong to, where $z_1 \leq z_2 \leq z_3 \leq z_4$. \textit{To make the problem easier, all zeros are between -5 and 5.}
\[ f(x) = 12x^{4} +73 x^{3} +28 x^{2} -215 x -150 \]The solution is \( [-5, -2, -0.75, 1.6666666666666667] \), which is option D.\begin{enumerate}[label=\Alph*.]
\item \( z_1 \in [-0.6, 5.4], \text{   }  z_2 \in [0.88, 1.81], z_3 \in [0.9, 2.12], \text{   and   } z_4 \in [4.5, 5.1] \)

 Distractor 3: Corresponds to negatives of all zeros AND inversing rational roots.
\item \( z_1 \in [-5, -4], \text{   }  z_2 \in [-2.09, -1.67], z_3 \in [-1.76, -1.15], \text{   and   } z_4 \in [-0.2, 0.7] \)

 Distractor 2: Corresponds to inversing rational roots.
\item \( z_1 \in [-5, -4], \text{   }  z_2 \in [0.07, 0.54], z_3 \in [0.9, 2.12], \text{   and   } z_4 \in [4.5, 5.1] \)

 Distractor 4: Corresponds to moving factors from one rational to another.
\item \( z_1 \in [-5, -4], \text{   }  z_2 \in [-2.09, -1.67], z_3 \in [-0.8, -0.48], \text{   and   } z_4 \in [1.1, 2.2] \)

* This is the solution!
\item \( z_1 \in [-4.67, -0.67], \text{   }  z_2 \in [0.72, 0.85], z_3 \in [0.9, 2.12], \text{   and   } z_4 \in [4.5, 5.1] \)

 Distractor 1: Corresponds to negatives of all zeros.
\end{enumerate}

\textbf{General Comment:} Remember to try the middle-most integers first as these normally are the zeros. Also, once you get it to a quadratic, you can use your other factoring techniques to finish factoring.
}
\litem{
Factor the polynomial below completely. Then, choose the intervals the zeros of the polynomial belong to, where $z_1 \leq z_2 \leq z_3$. \textit{To make the problem easier, all zeros are between -5 and 5.}
\[ f(x) = 15x^{3} +89 x^{2} +62 x -40 \]The solution is \( [-5, -1.3333333333333333, 0.4] \), which is option C.\begin{enumerate}[label=\Alph*.]
\item \( z_1 \in [-0.5, 0.5], \text{   }  z_2 \in [1.23, 2.03], \text{   and   } z_3 \in [4, 7] \)

 Distractor 1: Corresponds to negatives of all zeros.
\item \( z_1 \in [-5.5, -4.3], \text{   }  z_2 \in [-1.3, -0.15], \text{   and   } z_3 \in [2.5, 4.5] \)

 Distractor 2: Corresponds to inversing rational roots.
\item \( z_1 \in [-5.5, -4.3], \text{   }  z_2 \in [-1.74, -1.27], \text{   and   } z_3 \in [0.4, 1.4] \)

* This is the solution!
\item \( z_1 \in [-2.3, -1.5], \text{   }  z_2 \in [-0.13, 0.41], \text{   and   } z_3 \in [4, 7] \)

 Distractor 4: Corresponds to moving factors from one rational to another.
\item \( z_1 \in [-2.9, -2.2], \text{   }  z_2 \in [0.54, 0.95], \text{   and   } z_3 \in [4, 7] \)

 Distractor 3: Corresponds to negatives of all zeros AND inversing rational roots.
\end{enumerate}

\textbf{General Comment:} Remember to try the middle-most integers first as these normally are the zeros. Also, once you get it to a quadratic, you can use your other factoring techniques to finish factoring.
}
\litem{
Factor the polynomial below completely, knowing that $x-4$ is a factor. Then, choose the intervals the zeros of the polynomial belong to, where $z_1 \leq z_2 \leq z_3 \leq z_4$. \textit{To make the problem easier, all zeros are between -5 and 5.}
\[ f(x) = 12x^{4} +19 x^{3} -245 x^{2} -152 x + 240 \]The solution is \( [-5, -1.3333333333333333, 0.75, 4] \), which is option E.\begin{enumerate}[label=\Alph*.]
\item \( z_1 \in [-4.2, -2.8], \text{   }  z_2 \in [-1.76, -1.3], z_3 \in [0.69, 1.06], \text{   and   } z_4 \in [4.9, 7.3] \)

 Distractor 3: Corresponds to negatives of all zeros AND inversing rational roots.
\item \( z_1 \in [-5.7, -4.6], \text{   }  z_2 \in [-1.22, -0.75], z_3 \in [1.31, 1.8], \text{   and   } z_4 \in [3.8, 4.2] \)

 Distractor 2: Corresponds to inversing rational roots.
\item \( z_1 \in [-4.2, -2.8], \text{   }  z_2 \in [-0.4, 0.11], z_3 \in [3.97, 4.23], \text{   and   } z_4 \in [4.9, 7.3] \)

 Distractor 4: Corresponds to moving factors from one rational to another.
\item \( z_1 \in [-4.2, -2.8], \text{   }  z_2 \in [-1.22, -0.75], z_3 \in [1.31, 1.8], \text{   and   } z_4 \in [4.9, 7.3] \)

 Distractor 1: Corresponds to negatives of all zeros.
\item \( z_1 \in [-5.7, -4.6], \text{   }  z_2 \in [-1.76, -1.3], z_3 \in [0.69, 1.06], \text{   and   } z_4 \in [3.8, 4.2] \)

* This is the solution!
\end{enumerate}

\textbf{General Comment:} Remember to try the middle-most integers first as these normally are the zeros. Also, once you get it to a quadratic, you can use your other factoring techniques to finish factoring.
}
\litem{
Perform the division below. Then, find the intervals that correspond to the quotient in the form $ax^2+bx+c$ and remainder $r$.
\[ \frac{25x^{3} -50 x^{2} -9 x + 14}{x -2} \]The solution is \( 25x^{2} -9 + \frac{-4}{x -2} \), which is option D.\begin{enumerate}[label=\Alph*.]
\item \( a \in [50, 57], \text{   } b \in [-154, -142], \text{   } c \in [286, 295], \text{   and   } r \in [-572, -566]. \)

 You divided by the opposite of the factor AND multiplied the first factor rather than just bringing it down.
\item \( a \in [23, 29], \text{   } b \in [-100, -98], \text{   } c \in [189, 193], \text{   and   } r \in [-368, -363]. \)

 You divided by the opposite of the factor.
\item \( a \in [23, 29], \text{   } b \in [-29, -24], \text{   } c \in [-35, -31], \text{   and   } r \in [-22, -12]. \)

 You multiplied by the synthetic number and subtracted rather than adding during synthetic division.
\item \( a \in [23, 29], \text{   } b \in [-4, 8], \text{   } c \in [-9, -7], \text{   and   } r \in [-7, 1]. \)

* This is the solution!
\item \( a \in [50, 57], \text{   } b \in [49, 53], \text{   } c \in [87, 95], \text{   and   } r \in [194, 200]. \)

 You multiplied by the synthetic number rather than bringing the first factor down.
\end{enumerate}

\textbf{General Comment:} Be sure to synthetically divide by the zero of the denominator!
}
\litem{
Perform the division below. Then, find the intervals that correspond to the quotient in the form $ax^2+bx+c$ and remainder $r$.
\[ \frac{20x^{3} +62 x^{2} -21}{x + 3} \]The solution is \( 20x^{2} +2 x -6 + \frac{-3}{x + 3} \), which is option D.\begin{enumerate}[label=\Alph*.]
\item \( a \in [-63, -58], b \in [-124, -112], c \in [-357, -349], \text{ and } r \in [-1084, -1081]. \)

 You divided by the opposite of the factor AND multipled the first factor rather than just bringing it down.
\item \( a \in [20, 22], b \in [-19, -16], c \in [71, 76], \text{ and } r \in [-312, -308]. \)

 You multipled by the synthetic number and subtracted rather than adding during synthetic division.
\item \( a \in [-63, -58], b \in [237, 246], c \in [-727, -720], \text{ and } r \in [2157, 2164]. \)

 You multipled by the synthetic number rather than bringing the first factor down.
\item \( a \in [20, 22], b \in [0, 5], c \in [-11, 3], \text{ and } r \in [-8, 0]. \)

* This is the solution!
\item \( a \in [20, 22], b \in [120, 125], c \in [362, 367], \text{ and } r \in [1075, 1079]. \)

 You divided by the opposite of the factor.
\end{enumerate}

\textbf{General Comment:} Be sure to synthetically divide by the zero of the denominator! Also, make sure to include 0 placeholders for missing terms.
}
\litem{
Perform the division below. Then, find the intervals that correspond to the quotient in the form $ax^2+bx+c$ and remainder $r$.
\[ \frac{8x^{3} -18 x^{2} -50 x -26}{x -4} \]The solution is \( 8x^{2} +14 x + 6 + \frac{-2}{x -4} \), which is option E.\begin{enumerate}[label=\Alph*.]
\item \( a \in [31, 35], \text{   } b \in [-149, -136], \text{   } c \in [534, 536], \text{   and   } r \in [-2163, -2159]. \)

 You divided by the opposite of the factor AND multiplied the first factor rather than just bringing it down.
\item \( a \in [6, 15], \text{   } b \in [-51, -49], \text{   } c \in [150, 156], \text{   and   } r \in [-630, -621]. \)

 You divided by the opposite of the factor.
\item \( a \in [31, 35], \text{   } b \in [110, 114], \text{   } c \in [386, 393], \text{   and   } r \in [1530, 1538]. \)

 You multiplied by the synthetic number rather than bringing the first factor down.
\item \( a \in [6, 15], \text{   } b \in [6, 12], \text{   } c \in [-34, -24], \text{   and   } r \in [-126, -117]. \)

 You multiplied by the synthetic number and subtracted rather than adding during synthetic division.
\item \( a \in [6, 15], \text{   } b \in [13, 15], \text{   } c \in [4, 12], \text{   and   } r \in [-3, 5]. \)

* This is the solution!
\end{enumerate}

\textbf{General Comment:} Be sure to synthetically divide by the zero of the denominator!
}
\end{enumerate}

\end{document}