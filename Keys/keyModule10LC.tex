\documentclass{extbook}[14pt]
\usepackage{multicol, enumerate, enumitem, hyperref, color, soul, setspace, parskip, fancyhdr, amssymb, amsthm, amsmath, bbm, latexsym, units, mathtools}
\everymath{\displaystyle}
\usepackage[headsep=0.5cm,headheight=0cm, left=1 in,right= 1 in,top= 1 in,bottom= 1 in]{geometry}
\pagestyle{fancy}
\lhead{}
\chead{Answer Key for Module\,10L\,-\,Synthetic\,Division Version C}
\rhead{}
\lfoot{Summer\,C\,2020}
\cfoot{}
\rfoot{}
\begin{document}
\textbf{This key should allow you to understand why you choose the option you did (beyond just getting a question right or wrong). \href{https://xronos.clas.ufl.edu/mac1105spring2020/courseDescriptionAndMisc/Exams/LearningFromResults}{More instructions on how to use this key can be found here}.}

\textbf{If you have a suggestion to make the keys better, \href{https://forms.gle/CZkbZmPbC9XALEE88}{please fill out the short survey here}.}

\textit{Note: This key is auto-generated and may contain issues and/or errors. The keys are reviewed after each exam to ensure grading is done accurately. If there are issues (like duplicate options), they are noted in the offline gradebook. The keys are a work-in-progress to give students as many resources to improve as possible.}

\rule{\textwidth}{0.4pt}

1. Perform the division below. Then, find the intervals that correspond to the quotient in the form $ax^2+bx+c$ and remainder $r$.
\[ \frac{6x^{3} -44 x^{2} +78 x -42}{x -5} \] 
The solution is $ 6x^{2} -14 x + 8 + \frac{-2}{x -5} $ 

\begin{enumerate}[label=\Alph*.] 
\item $ a \in [29, 31], \text{   } b \in [-200, -180], \text{   } c \in [1046, 1053], \text{   and   } r \in [-5284, -5281]. $ 

  You divided by the opposite of the factor AND multiplied the first factor rather than just bringing it down. 
\item $ a \in [1, 9], \text{   } b \in [-22, -18], \text{   } c \in [-6, -1], \text{   and   } r \in [-52, -48]. $ 

  You multiplied by the synthetic number and subtracted rather than adding during synthetic division. 
\item $ a \in [29, 31], \text{   } b \in [105, 114], \text{   } c \in [605, 612], \text{   and   } r \in [2996, 3000]. $ 

  You multiplied by the synthetic number rather than bringing the first factor down. 
\item $ a \in [1, 9], \text{   } b \in [-15, -11], \text{   } c \in [7, 10], \text{   and   } r \in [-5, 1]. $ 

 * This is the solution! 
\item $ a \in [1, 9], \text{   } b \in [-78, -69], \text{   } c \in [446, 453], \text{   and   } r \in [-2284, -2280]. $ 

  You divided by the opposite of the factor. 
\end{enumerate} 
 
\textbf{General Comment:} General Comments: Be sure to synthetically divide by the zero of the denominator! 

-----------------------------------------------

2. Factor the polynomial below completely. Then, choose the intervals the zeros of the polynomial belong to, where $z_1 \leq z_2 \leq z_3$. \textit{To make the problem easier, all zeros are between -5 and 5.}
\[ f(x) = 9x^{3} -39 x^{2} -38 x + 40 \] 
The solution is $ [-1.3333333333333333, 0.6666666666666666, 5] $ 

\begin{enumerate}[label=\Alph*.] 
\item $ z_1 \in [-5.71, -4.64], \text{   }  z_2 \in [-1.73, -1.3], \text{   and   } z_3 \in [-0.12, 0.88] $ 

  Distractor 3: Corresponds to negatives of all zeros AND inversing rational roots. 
\item $ z_1 \in [-5.71, -4.64], \text{   }  z_2 \in [-0.95, -0.49], \text{   and   } z_3 \in [1.32, 1.78] $ 

  Distractor 1: Corresponds to negatives of all zeros. 
\item $ z_1 \in [-0.77, -0.7], \text{   }  z_2 \in [1.35, 1.77], \text{   and   } z_3 \in [4.58, 5.7] $ 

  Distractor 2: Corresponds to inversing rational roots. 
\item $ z_1 \in [-1.84, -1.03], \text{   }  z_2 \in [0.29, 1.21], \text{   and   } z_3 \in [4.58, 5.7] $ 

 * This is the solution! 
\item $ z_1 \in [-5.71, -4.64], \text{   }  z_2 \in [-0.25, -0.11], \text{   and   } z_3 \in [3.93, 4.41] $ 

  Distractor 4: Corresponds to moving factors from one rational to another. 
\end{enumerate} 
 
\textbf{General Comment:} General Comments: Remember to try the middle-most integers first as these normally are the zeros. Also, once you get it to a quadratic, you can use your other factoring techniques to finish factoring. 

-----------------------------------------------

3. What are the \textit{possible Integer} roots of the polynomial below?
\[ f(x) = 5x^{2} +7 x + 7 \] 
The solution is $ \pm 1,\pm 7 $ 

\begin{enumerate}[label=\Alph*.] 
\item $ \text{ All combinations of: }\frac{\pm 1,\pm 7}{\pm 1,\pm 5} $ 

 This would have been the solution \textbf{if asked for the possible Rational roots}! 
\item $ \pm 1,\pm 7 $ 

 * This is the solution \textbf{since we asked for the possible Integer roots}! 
\item $ \text{ All combinations of: }\frac{\pm 1,\pm 5}{\pm 1,\pm 7} $ 

  Distractor 3: Corresponds to the plus or minus of the inverse quotient (an/a0) of the factors.  
\item $ \pm 1,\pm 5 $ 

  Distractor 1: Corresponds to the plus or minus factors of a1 only. 
\item $ \text{There is no formula or theorem that tells us all possible Integer roots.} $ 

  Distractor 4: Corresponds to not recognizing Integers as a subset of Rationals. 
\end{enumerate} 
 
\textbf{General Comment:} General Comments: We have a way to find the possible Rational roots. The possible Integer roots are the Integers in this list. 

-----------------------------------------------

4. Factor the polynomial below completely, knowing that $x-4$ is a factor. Then, choose the intervals the zeros of the polynomial belong to, where $z_1 \leq z_2 \leq z_3 \leq z_4$. \textit{To make the problem easier, all zeros are between -5 and 5.}
\[ f(x) = 20x^{4} -127 x^{3} +134 x^{2} +261 x -180 \] 
The solution is $ [-1.25, 0.6, 3, 4] $ 

\begin{enumerate}[label=\Alph*.] 
\item $ z_1 \in [-4.12, -3.66], \text{   }  z_2 \in [-3.42, -2.89], z_3 \in [-0.77, -0.16], \text{   and   } z_4 \in [0.87, 2.6] $ 

  Distractor 1: Corresponds to negatives of all zeros. 
\item $ z_1 \in [-1.79, -1.07], \text{   }  z_2 \in [0.06, 1.02], z_3 \in [2.86, 3.08], \text{   and   } z_4 \in [3.18, 4.97] $ 

 * This is the solution! 
\item $ z_1 \in [-0.87, -0.59], \text{   }  z_2 \in [1.61, 1.77], z_3 \in [2.86, 3.08], \text{   and   } z_4 \in [3.18, 4.97] $ 

  Distractor 2: Corresponds to inversing rational roots. 
\item $ z_1 \in [-4.12, -3.66], \text{   }  z_2 \in [-3.42, -2.89], z_3 \in [-0.35, 0.12], \text{   and   } z_4 \in [4.84, 5.17] $ 

  Distractor 4: Corresponds to moving factors from one rational to another. 
\item $ z_1 \in [-4.12, -3.66], \text{   }  z_2 \in [-3.42, -2.89], z_3 \in [-1.8, -1.43], \text{   and   } z_4 \in [0.32, 1.06] $ 

  Distractor 3: Corresponds to negatives of all zeros AND inversing rational roots. 
\end{enumerate} 
 
\textbf{General Comment:} General Comments: Remember to try the middle-most integers first as these normally are the zeros. Also, once you get it to a quadratic, you can use your other factoring techniques to finish factoring. 

-----------------------------------------------

0. Perform the division below. Then, find the intervals that correspond to the quotient in the form $ax^2+bx+c$ and remainder $r$.
\[ \frac{8x^{3} +28 x^{2} -32}{x + 3} \] 
The solution is $ 8x^{2} +4 x -12 + \frac{4}{x + 3} $ 

\begin{enumerate}[label=\Alph*.] 
\item $ a \in [-28, -22], b \in [94, 105], c \in [-301, -299], \text{ and } r \in [860, 871]. $ 

  You multipled by the synthetic number rather than bringing the first factor down. 
\item $ a \in [-28, -22], b \in [-50, -41], c \in [-138, -130], \text{ and } r \in [-431, -422]. $ 

  You divided by the opposite of the factor AND multipled the first factor rather than just bringing it down. 
\item $ a \in [6, 10], b \in [0, 5], c \in [-17, -9], \text{ and } r \in [2, 14]. $ 

 * This is the solution! 
\item $ a \in [6, 10], b \in [-9, 0], c \in [15, 18], \text{ and } r \in [-101, -93]. $ 

  You multipled by the synthetic number and subtracted rather than adding during synthetic division. 
\item $ a \in [6, 10], b \in [50, 54], c \in [153, 159], \text{ and } r \in [432, 441]. $ 

  You divided by the opposite of the factor. 
\end{enumerate} 
 
\textbf{General Comment:} General Comments: Be sure to synthetically divide by the zero of the denominator! Also, make sure to include 0 placeholders for missing terms. 

-----------------------------------------------


\end{document}

