\documentclass{extbook}[14pt]
\usepackage{multicol, enumerate, enumitem, hyperref, color, soul, setspace, parskip, fancyhdr, amssymb, amsthm, amsmath, bbm, latexsym, units, mathtools}
\everymath{\displaystyle}
\usepackage[headsep=0.5cm,headheight=0cm, left=1 in,right= 1 in,top= 1 in,bottom= 1 in]{geometry}
\usepackage{dashrule}  % Package to use the command below to create lines between items
\newcommand{\litem}[1]{\item #1

\rule{\textwidth}{0.4pt}}
\pagestyle{fancy}
\lhead{}
\chead{Answer Key for Progress Quiz 2 Version C}
\rhead{}
\lfoot{7862-5421}
\cfoot{}
\rfoot{Spring 2021}
\begin{document}
\textbf{This key should allow you to understand why you choose the option you did (beyond just getting a question right or wrong). \href{https://xronos.clas.ufl.edu/mac1105spring2020/courseDescriptionAndMisc/Exams/LearningFromResults}{More instructions on how to use this key can be found here}.}

\textbf{If you have a suggestion to make the keys better, \href{https://forms.gle/CZkbZmPbC9XALEE88}{please fill out the short survey here}.}

\textit{Note: This key is auto-generated and may contain issues and/or errors. The keys are reviewed after each exam to ensure grading is done accurately. If there are issues (like duplicate options), they are noted in the offline gradebook. The keys are a work-in-progress to give students as many resources to improve as possible.}

\rule{\textwidth}{0.4pt}

\begin{enumerate}\litem{
Solve the linear inequality below. Then, choose the constant and interval combination that describes the solution set.
\[ 4 - 9 x < \frac{-30 x - 9}{7} \leq 6 - 5 x \]

The solution is \( \text{None of the above.} \), which is option E.\begin{enumerate}[label=\Alph*.]
\item \( (a, b], \text{ where } a \in [-4.12, 0.88] \text{ and } b \in [-12.2, -3.2] \)

$(-1.12, -10.20]$, which is the correct interval but negatives of the actual endpoints.
\item \( [a, b), \text{ where } a \in [-1.71, -0.6] \text{ and } b \in [-13.2, -9.2] \)

$[-1.12, -10.20)$, which corresponds to flipping the inequality and getting negatives of the actual endpoints.
\item \( (-\infty, a] \cup (b, \infty), \text{ where } a \in [-2.12, -0.12] \text{ and } b \in [-17.2, -9.2] \)

$(-\infty, -1.12] \cup (-10.20, \infty)$, which corresponds to displaying the and-inequality as an or-inequality AND flipping the inequality AND getting negatives of the actual endpoints.
\item \( (-\infty, a) \cup [b, \infty), \text{ where } a \in [-4, -0.6] \text{ and } b \in [-11.2, -6.2] \)

$(-\infty, -1.12) \cup [-10.20, \infty)$, which corresponds to displaying the and-inequality as an or-inequality and getting negatives of the actual endpoints.
\item \( \text{None of the above.} \)

* This is correct as the answer should be $(1.12, 10.20]$.
\end{enumerate}

\textbf{General Comment:} To solve, you will need to break up the compound inequality into two inequalities. Be sure to keep track of the inequality! It may be best to draw a number line and graph your solution.
}
\litem{
Solve the linear inequality below. Then, choose the constant and interval combination that describes the solution set.
\[ 3x + 8 \geq 8x -5 \]

The solution is \( (-\infty, 2.6] \), which is option A.\begin{enumerate}[label=\Alph*.]
\item \( (-\infty, a], \text{ where } a \in [2.6, 10.6] \)

* $(-\infty, 2.6]$, which is the correct option.
\item \( [a, \infty), \text{ where } a \in [-8.6, -1.6] \)

 $[-2.6, \infty)$, which corresponds to switching the direction of the interval AND negating the endpoint. You likely did this if you did not flip the inequality when dividing by a negative as well as not moving values over to a side properly.
\item \( [a, \infty), \text{ where } a \in [1.6, 9.6] \)

 $[2.6, \infty)$, which corresponds to switching the direction of the interval. You likely did this if you did not flip the inequality when dividing by a negative!
\item \( (-\infty, a], \text{ where } a \in [-2.6, -0.6] \)

 $(-\infty, -2.6]$, which corresponds to negating the endpoint of the solution.
\item \( \text{None of the above}. \)

You may have chosen this if you thought the inequality did not match the ends of the intervals.
\end{enumerate}

\textbf{General Comment:} Remember that less/greater than or equal to includes the endpoint, while less/greater do not. Also, remember that you need to flip the inequality when you multiply or divide by a negative.
}
\litem{
Solve the linear inequality below. Then, choose the constant and interval combination that describes the solution set.
\[ \frac{6}{6} + \frac{4}{2} x \leq \frac{8}{7} x - \frac{4}{3} \]

The solution is \( (-\infty, -2.722] \), which is option B.\begin{enumerate}[label=\Alph*.]
\item \( (-\infty, a], \text{ where } a \in [-1.28, 4.72] \)

 $(-\infty, 2.722]$, which corresponds to negating the endpoint of the solution.
\item \( (-\infty, a], \text{ where } a \in [-3.72, 0.28] \)

* $(-\infty, -2.722]$, which is the correct option.
\item \( [a, \infty), \text{ where } a \in [2.72, 4.72] \)

 $[2.722, \infty)$, which corresponds to switching the direction of the interval AND negating the endpoint. You likely did this if you did not flip the inequality when dividing by a negative as well as not moving values over to a side properly.
\item \( [a, \infty), \text{ where } a \in [-3.72, 0.28] \)

 $[-2.722, \infty)$, which corresponds to switching the direction of the interval. You likely did this if you did not flip the inequality when dividing by a negative!
\item \( \text{None of the above}. \)

You may have chosen this if you thought the inequality did not match the ends of the intervals.
\end{enumerate}

\textbf{General Comment:} Remember that less/greater than or equal to includes the endpoint, while less/greater do not. Also, remember that you need to flip the inequality when you multiply or divide by a negative.
}
\litem{
Using an interval or intervals, describe all the $x$-values within or including a distance of the given values.
\[ \text{ Less than } 2 \text{ units from the number } 8. \]

The solution is \( (6, 10) \), which is option C.\begin{enumerate}[label=\Alph*.]
\item \( (-\infty, 6) \cup (10, \infty) \)

This describes the values more than 2 from 8
\item \( [6, 10] \)

This describes the values no more than 2 from 8
\item \( (6, 10) \)

This describes the values less than 2 from 8
\item \( (-\infty, 6] \cup [10, \infty) \)

This describes the values no less than 2 from 8
\item \( \text{None of the above} \)

You likely thought the values in the interval were not correct.
\end{enumerate}

\textbf{General Comment:} When thinking about this language, it helps to draw a number line and try points.
}
\litem{
Solve the linear inequality below. Then, choose the constant and interval combination that describes the solution set.
\[ -9 + 6 x > 7 x \text{ or } -6 + 3 x < 5 x \]

The solution is \( (-\infty, -9.0) \text{ or } (-3.0, \infty) \), which is option B.\begin{enumerate}[label=\Alph*.]
\item \( (-\infty, a] \cup [b, \infty), \text{ where } a \in [-12, -5] \text{ and } b \in [-3, 2] \)

Corresponds to including the endpoints (when they should be excluded).
\item \( (-\infty, a) \cup (b, \infty), \text{ where } a \in [-9, -4] \text{ and } b \in [-6, -1] \)

 * Correct option.
\item \( (-\infty, a) \cup (b, \infty), \text{ where } a \in [-1, 5] \text{ and } b \in [9, 12] \)

Corresponds to inverting the inequality and negating the solution.
\item \( (-\infty, a] \cup [b, \infty), \text{ where } a \in [3, 7] \text{ and } b \in [8, 14] \)

Corresponds to including the endpoints AND negating.
\item \( (-\infty, \infty) \)

Corresponds to the variable canceling, which does not happen in this instance.
\end{enumerate}

\textbf{General Comment:} When multiplying or dividing by a negative, flip the sign.
}
\litem{
Solve the linear inequality below. Then, choose the constant and interval combination that describes the solution set.
\[ -7x -9 > 4x + 5 \]

The solution is \( (-\infty, -1.273) \), which is option A.\begin{enumerate}[label=\Alph*.]
\item \( (-\infty, a), \text{ where } a \in [-4.27, -0.27] \)

* $(-\infty, -1.273)$, which is the correct option.
\item \( (-\infty, a), \text{ where } a \in [1.27, 5.27] \)

 $(-\infty, 1.273)$, which corresponds to negating the endpoint of the solution.
\item \( (a, \infty), \text{ where } a \in [-2.2, 0.5] \)

 $(-1.273, \infty)$, which corresponds to switching the direction of the interval. You likely did this if you did not flip the inequality when dividing by a negative!
\item \( (a, \infty), \text{ where } a \in [0.2, 2.8] \)

 $(1.273, \infty)$, which corresponds to switching the direction of the interval AND negating the endpoint. You likely did this if you did not flip the inequality when dividing by a negative as well as not moving values over to a side properly.
\item \( \text{None of the above}. \)

You may have chosen this if you thought the inequality did not match the ends of the intervals.
\end{enumerate}

\textbf{General Comment:} Remember that less/greater than or equal to includes the endpoint, while less/greater do not. Also, remember that you need to flip the inequality when you multiply or divide by a negative.
}
\litem{
Solve the linear inequality below. Then, choose the constant and interval combination that describes the solution set.
\[ \frac{-3}{8} - \frac{7}{6} x \leq \frac{-5}{5} x - \frac{10}{2} \]

The solution is \( [27.75, \infty) \), which is option D.\begin{enumerate}[label=\Alph*.]
\item \( [a, \infty), \text{ where } a \in [-27.75, -25.75] \)

 $[-27.75, \infty)$, which corresponds to negating the endpoint of the solution.
\item \( (-\infty, a], \text{ where } a \in [26.75, 28.75] \)

 $(-\infty, 27.75]$, which corresponds to switching the direction of the interval. You likely did this if you did not flip the inequality when dividing by a negative!
\item \( (-\infty, a], \text{ where } a \in [-31.75, -23.75] \)

 $(-\infty, -27.75]$, which corresponds to switching the direction of the interval AND negating the endpoint. You likely did this if you did not flip the inequality when dividing by a negative as well as not moving values over to a side properly.
\item \( [a, \infty), \text{ where } a \in [25.75, 28.75] \)

* $[27.75, \infty)$, which is the correct option.
\item \( \text{None of the above}. \)

You may have chosen this if you thought the inequality did not match the ends of the intervals.
\end{enumerate}

\textbf{General Comment:} Remember that less/greater than or equal to includes the endpoint, while less/greater do not. Also, remember that you need to flip the inequality when you multiply or divide by a negative.
}
\litem{
Using an interval or intervals, describe all the $x$-values within or including a distance of the given values.
\[ \text{ Less than } 4 \text{ units from the number } 3. \]

The solution is \( \text{None of the above} \), which is option E.\begin{enumerate}[label=\Alph*.]
\item \( [1, 7] \)

This describes the values no more than 3 from 4
\item \( (-\infty, 1) \cup (7, \infty) \)

This describes the values more than 3 from 4
\item \( (1, 7) \)

This describes the values less than 3 from 4
\item \( (-\infty, 1] \cup [7, \infty) \)

This describes the values no less than 3 from 4
\item \( \text{None of the above} \)

Options A-D described the values [more/less than] 3 units from 4, which is the reverse of what the question asked.
\end{enumerate}

\textbf{General Comment:} When thinking about this language, it helps to draw a number line and try points.
}
\litem{
Solve the linear inequality below. Then, choose the constant and interval combination that describes the solution set.
\[ -3 - 6 x \leq \frac{-7 x + 7}{4} < 6 - 3 x \]

The solution is \( [-1.12, 3.40) \), which is option B.\begin{enumerate}[label=\Alph*.]
\item \( (a, b], \text{ where } a \in [-2.6, -0.5] \text{ and } b \in [-0.6, 4.4] \)

$(-1.12, 3.40]$, which corresponds to flipping the inequality.
\item \( [a, b), \text{ where } a \in [-5.12, -0.12] \text{ and } b \in [0.4, 5.4] \)

$[-1.12, 3.40)$, which is the correct option.
\item \( (-\infty, a] \cup (b, \infty), \text{ where } a \in [-2.4, -0.3] \text{ and } b \in [1.4, 5.4] \)

$(-\infty, -1.12] \cup (3.40, \infty)$, which corresponds to displaying the and-inequality as an or-inequality.
\item \( (-\infty, a) \cup [b, \infty), \text{ where } a \in [-1.9, -0.1] \text{ and } b \in [-2.6, 4.4] \)

$(-\infty, -1.12) \cup [3.40, \infty)$, which corresponds to displaying the and-inequality as an or-inequality AND flipping the inequality.
\item \( \text{None of the above.} \)


\end{enumerate}

\textbf{General Comment:} To solve, you will need to break up the compound inequality into two inequalities. Be sure to keep track of the inequality! It may be best to draw a number line and graph your solution.
}
\litem{
Solve the linear inequality below. Then, choose the constant and interval combination that describes the solution set.
\[ 9 - 3 x > 6 x \text{ or } 4 + 4 x < 5 x \]

The solution is \( (-\infty, 1.0) \text{ or } (4.0, \infty) \), which is option A.\begin{enumerate}[label=\Alph*.]
\item \( (-\infty, a) \cup (b, \infty), \text{ where } a \in [1, 3] \text{ and } b \in [4, 6] \)

 * Correct option.
\item \( (-\infty, a) \cup (b, \infty), \text{ where } a \in [-5, 0] \text{ and } b \in [-4, 3] \)

Corresponds to inverting the inequality and negating the solution.
\item \( (-\infty, a] \cup [b, \infty), \text{ where } a \in [-6, -1] \text{ and } b \in [-1, 0] \)

Corresponds to including the endpoints AND negating.
\item \( (-\infty, a] \cup [b, \infty), \text{ where } a \in [1, 4] \text{ and } b \in [4, 6] \)

Corresponds to including the endpoints (when they should be excluded).
\item \( (-\infty, \infty) \)

Corresponds to the variable canceling, which does not happen in this instance.
\end{enumerate}

\textbf{General Comment:} When multiplying or dividing by a negative, flip the sign.
}
\end{enumerate}

\end{document}