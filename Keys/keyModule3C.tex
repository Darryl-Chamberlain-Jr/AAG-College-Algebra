\documentclass{extbook}[14pt]
\usepackage{multicol, enumerate, enumitem, hyperref, color, soul, setspace, parskip, fancyhdr, amssymb, amsthm, amsmath, bbm, latexsym, units, mathtools}
\everymath{\displaystyle}
\usepackage[headsep=0.5cm,headheight=0cm, left=1 in,right= 1 in,top= 1 in,bottom= 1 in]{geometry}
\usepackage{dashrule}  % Package to use the command below to create lines between items
\newcommand{\litem}[1]{\item #1

\rule{\textwidth}{0.4pt}}
\pagestyle{fancy}
\lhead{}
\chead{Answer Key for Progress Quiz 4 Version C}
\rhead{}
\lfoot{6286-1986}
\cfoot{}
\rfoot{Fall 2020}
\begin{document}
\textbf{This key should allow you to understand why you choose the option you did (beyond just getting a question right or wrong). \href{https://xronos.clas.ufl.edu/mac1105spring2020/courseDescriptionAndMisc/Exams/LearningFromResults}{More instructions on how to use this key can be found here}.}

\textbf{If you have a suggestion to make the keys better, \href{https://forms.gle/CZkbZmPbC9XALEE88}{please fill out the short survey here}.}

\textit{Note: This key is auto-generated and may contain issues and/or errors. The keys are reviewed after each exam to ensure grading is done accurately. If there are issues (like duplicate options), they are noted in the offline gradebook. The keys are a work-in-progress to give students as many resources to improve as possible.}

\rule{\textwidth}{0.4pt}

\begin{enumerate}\litem{
Solve the linear inequality below. Then, choose the constant and interval combination that describes the solution set.
\[ -8 + 7 x > 8 x \text{ or } -6 + 7 x < 9 x \]
The solution is \( (-\infty, -8.0) \text{ or } (-3.0, \infty) \), which is option D.\begin{enumerate}[label=\Alph*.]
\item \( (-\infty, a] \cup [b, \infty), \text{ where } a \in [-9, -7] \text{ and } b \in [-3, 0] \)

Corresponds to including the endpoints (when they should be excluded).
\item \( (-\infty, a) \cup (b, \infty), \text{ where } a \in [3, 4] \text{ and } b \in [7, 10] \)

Corresponds to inverting the inequality and negating the solution.
\item \( (-\infty, a] \cup [b, \infty), \text{ where } a \in [2, 8] \text{ and } b \in [8, 11] \)

Corresponds to including the endpoints AND negating.
\item \( (-\infty, a) \cup (b, \infty), \text{ where } a \in [-10, -7] \text{ and } b \in [-6, -2] \)

 * Correct option.
\item \( (-\infty, \infty) \)

Corresponds to the variable canceling, which does not happen in this instance.
\end{enumerate}

\textbf{General Comment:} When multiplying or dividing by a negative, flip the sign.
}
\litem{
Using an interval or intervals, describe all the $x$-values within or including a distance of the given values.
\[ \text{ No more than } 10 \text{ units from the number } 8. \]
The solution is \( \text{None of the above} \), which is option E.\begin{enumerate}[label=\Alph*.]
\item \( (-\infty, 2] \cup [18, \infty) \)

This describes the values no less than 8 from 10
\item \( [2, 18] \)

This describes the values no more than 8 from 10
\item \( (2, 18) \)

This describes the values less than 8 from 10
\item \( (-\infty, 2) \cup (18, \infty) \)

This describes the values more than 8 from 10
\item \( \text{None of the above} \)

Options A-D described the values [more/less than] 8 units from 10, which is the reverse of what the question asked.
\end{enumerate}

\textbf{General Comment:} When thinking about this language, it helps to draw a number line and try points.
}
\litem{
Solve the linear inequality below. Then, choose the constant and interval combination that describes the solution set.
\[ -3x + 4 \leq 6x + 9 \]
The solution is \( [-0.556, \infty) \), which is option A.\begin{enumerate}[label=\Alph*.]
\item \( [a, \infty), \text{ where } a \in [-1.69, -0.39] \)

* $[-0.556, \infty)$, which is the correct option.
\item \( (-\infty, a], \text{ where } a \in [-2.2, -0.2] \)

 $(-\infty, -0.556]$, which corresponds to switching the direction of the interval. You likely did this if you did not flip the inequality when dividing by a negative!
\item \( (-\infty, a], \text{ where } a \in [0.1, 1.9] \)

 $(-\infty, 0.556]$, which corresponds to switching the direction of the interval AND negating the endpoint. You likely did this if you did not flip the inequality when dividing by a negative as well as not moving values over to a side properly.
\item \( [a, \infty), \text{ where } a \in [0.35, 0.95] \)

 $[0.556, \infty)$, which corresponds to negating the endpoint of the solution.
\item \( \text{None of the above}. \)

You may have chosen this if you thought the inequality did not match the ends of the intervals.
\end{enumerate}

\textbf{General Comment:} Remember that less/greater than or equal to includes the endpoint, while less/greater do not. Also, remember that you need to flip the inequality when you multiply or divide by a negative.
}
\litem{
Using an interval or intervals, describe all the $x$-values within or including a distance of the given values.
\[ \text{ No more than } 6 \text{ units from the number } -6. \]
The solution is \( [-12, 0] \), which is option B.\begin{enumerate}[label=\Alph*.]
\item \( (-\infty, -12) \cup (0, \infty) \)

This describes the values more than 6 from -6
\item \( [-12, 0] \)

This describes the values no more than 6 from -6
\item \( (-\infty, -12] \cup [0, \infty) \)

This describes the values no less than 6 from -6
\item \( (-12, 0) \)

This describes the values less than 6 from -6
\item \( \text{None of the above} \)

You likely thought the values in the interval were not correct.
\end{enumerate}

\textbf{General Comment:} When thinking about this language, it helps to draw a number line and try points.
}
\litem{
Solve the linear inequality below. Then, choose the constant and interval combination that describes the solution set.
\[ \frac{3}{6} - \frac{6}{4} x \leq \frac{-4}{2} x - \frac{3}{3} \]
The solution is \( (-\infty, -3.0] \), which is option A.\begin{enumerate}[label=\Alph*.]
\item \( (-\infty, a], \text{ where } a \in [-4, 0] \)

* $(-\infty, -3.0]$, which is the correct option.
\item \( (-\infty, a], \text{ where } a \in [2, 5] \)

 $(-\infty, 3.0]$, which corresponds to negating the endpoint of the solution.
\item \( [a, \infty), \text{ where } a \in [3, 5] \)

 $[3.0, \infty)$, which corresponds to switching the direction of the interval AND negating the endpoint. You likely did this if you did not flip the inequality when dividing by a negative as well as not moving values over to a side properly.
\item \( [a, \infty), \text{ where } a \in [-3, 0] \)

 $[-3.0, \infty)$, which corresponds to switching the direction of the interval. You likely did this if you did not flip the inequality when dividing by a negative!
\item \( \text{None of the above}. \)

You may have chosen this if you thought the inequality did not match the ends of the intervals.
\end{enumerate}

\textbf{General Comment:} Remember that less/greater than or equal to includes the endpoint, while less/greater do not. Also, remember that you need to flip the inequality when you multiply or divide by a negative.
}
\litem{
Solve the linear inequality below. Then, choose the constant and interval combination that describes the solution set.
\[ -5 + 8 x > 9 x \text{ or } 3 + 9 x < 12 x \]
The solution is \( (-\infty, -5.0) \text{ or } (1.0, \infty) \), which is option A.\begin{enumerate}[label=\Alph*.]
\item \( (-\infty, a) \cup (b, \infty), \text{ where } a \in [-10, -4] \text{ and } b \in [1, 3] \)

 * Correct option.
\item \( (-\infty, a) \cup (b, \infty), \text{ where } a \in [-1, 4] \text{ and } b \in [5, 6] \)

Corresponds to inverting the inequality and negating the solution.
\item \( (-\infty, a] \cup [b, \infty), \text{ where } a \in [-6, -4] \text{ and } b \in [-3, 3] \)

Corresponds to including the endpoints (when they should be excluded).
\item \( (-\infty, a] \cup [b, \infty), \text{ where } a \in [-2, 0] \text{ and } b \in [5, 6] \)

Corresponds to including the endpoints AND negating.
\item \( (-\infty, \infty) \)

Corresponds to the variable canceling, which does not happen in this instance.
\end{enumerate}

\textbf{General Comment:} When multiplying or dividing by a negative, flip the sign.
}
\litem{
Solve the linear inequality below. Then, choose the constant and interval combination that describes the solution set.
\[ \frac{-9}{7} - \frac{5}{8} x < \frac{4}{6} x + \frac{5}{3} \]
The solution is \( (-2.286, \infty) \), which is option B.\begin{enumerate}[label=\Alph*.]
\item \( (a, \infty), \text{ where } a \in [-0.71, 3.29] \)

 $(2.286, \infty)$, which corresponds to negating the endpoint of the solution.
\item \( (a, \infty), \text{ where } a \in [-5.29, -1.29] \)

* $(-2.286, \infty)$, which is the correct option.
\item \( (-\infty, a), \text{ where } a \in [0.29, 5.29] \)

 $(-\infty, 2.286)$, which corresponds to switching the direction of the interval AND negating the endpoint. You likely did this if you did not flip the inequality when dividing by a negative as well as not moving values over to a side properly.
\item \( (-\infty, a), \text{ where } a \in [-3.29, -0.29] \)

 $(-\infty, -2.286)$, which corresponds to switching the direction of the interval. You likely did this if you did not flip the inequality when dividing by a negative!
\item \( \text{None of the above}. \)

You may have chosen this if you thought the inequality did not match the ends of the intervals.
\end{enumerate}

\textbf{General Comment:} Remember that less/greater than or equal to includes the endpoint, while less/greater do not. Also, remember that you need to flip the inequality when you multiply or divide by a negative.
}
\litem{
Solve the linear inequality below. Then, choose the constant and interval combination that describes the solution set.
\[ -3x -3 \leq 9x + 8 \]
The solution is \( [-0.917, \infty) \), which is option C.\begin{enumerate}[label=\Alph*.]
\item \( (-\infty, a], \text{ where } a \in [0.1, 2.3] \)

 $(-\infty, 0.917]$, which corresponds to switching the direction of the interval AND negating the endpoint. You likely did this if you did not flip the inequality when dividing by a negative as well as not moving values over to a side properly.
\item \( [a, \infty), \text{ where } a \in [-0.08, 8.92] \)

 $[0.917, \infty)$, which corresponds to negating the endpoint of the solution.
\item \( [a, \infty), \text{ where } a \in [-5.92, 0.08] \)

* $[-0.917, \infty)$, which is the correct option.
\item \( (-\infty, a], \text{ where } a \in [-2, 0.8] \)

 $(-\infty, -0.917]$, which corresponds to switching the direction of the interval. You likely did this if you did not flip the inequality when dividing by a negative!
\item \( \text{None of the above}. \)

You may have chosen this if you thought the inequality did not match the ends of the intervals.
\end{enumerate}

\textbf{General Comment:} Remember that less/greater than or equal to includes the endpoint, while less/greater do not. Also, remember that you need to flip the inequality when you multiply or divide by a negative.
}
\litem{
Solve the linear inequality below. Then, choose the constant and interval combination that describes the solution set.
\[ -7 + 6 x \leq \frac{56 x + 6}{9} < -4 + 6 x \]
The solution is \( [-34.50, -21.00) \), which is option C.\begin{enumerate}[label=\Alph*.]
\item \( (-\infty, a] \cup (b, \infty), \text{ where } a \in [-36.5, -32.5] \text{ and } b \in [-25, -20] \)

$(-\infty, -34.50] \cup (-21.00, \infty)$, which corresponds to displaying the and-inequality as an or-inequality.
\item \( (a, b], \text{ where } a \in [-34.5, -30.5] \text{ and } b \in [-25, -20] \)

$(-34.50, -21.00]$, which corresponds to flipping the inequality.
\item \( [a, b), \text{ where } a \in [-38.5, -33.5] \text{ and } b \in [-21, -18] \)

$[-34.50, -21.00)$, which is the correct option.
\item \( (-\infty, a) \cup [b, \infty), \text{ where } a \in [-37.5, -33.5] \text{ and } b \in [-22, -17] \)

$(-\infty, -34.50) \cup [-21.00, \infty)$, which corresponds to displaying the and-inequality as an or-inequality AND flipping the inequality.
\item \( \text{None of the above.} \)


\end{enumerate}

\textbf{General Comment:} To solve, you will need to break up the compound inequality into two inequalities. Be sure to keep track of the inequality! It may be best to draw a number line and graph your solution.
}
\litem{
Solve the linear inequality below. Then, choose the constant and interval combination that describes the solution set.
\[ -8 - 7 x \leq \frac{-19 x + 8}{3} < 8 - 9 x \]
The solution is \( [-16.00, 2.00) \), which is option B.\begin{enumerate}[label=\Alph*.]
\item \( (-\infty, a) \cup [b, \infty), \text{ where } a \in [-16, -12] \text{ and } b \in [-1, 7] \)

$(-\infty, -16.00) \cup [2.00, \infty)$, which corresponds to displaying the and-inequality as an or-inequality AND flipping the inequality.
\item \( [a, b), \text{ where } a \in [-16, -15] \text{ and } b \in [1, 3] \)

$[-16.00, 2.00)$, which is the correct option.
\item \( (-\infty, a] \cup (b, \infty), \text{ where } a \in [-21, -14] \text{ and } b \in [0, 3] \)

$(-\infty, -16.00] \cup (2.00, \infty)$, which corresponds to displaying the and-inequality as an or-inequality.
\item \( (a, b], \text{ where } a \in [-16, -15] \text{ and } b \in [1.3, 4.5] \)

$(-16.00, 2.00]$, which corresponds to flipping the inequality.
\item \( \text{None of the above.} \)


\end{enumerate}

\textbf{General Comment:} To solve, you will need to break up the compound inequality into two inequalities. Be sure to keep track of the inequality! It may be best to draw a number line and graph your solution.
}
\end{enumerate}

\end{document}