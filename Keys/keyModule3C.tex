\documentclass{extbook}[14pt]
\usepackage{multicol, enumerate, enumitem, hyperref, color, soul, setspace, parskip, fancyhdr, amssymb, amsthm, amsmath, bbm, latexsym, units, mathtools}
\everymath{\displaystyle}
\usepackage[headsep=0.5cm,headheight=0cm, left=1 in,right= 1 in,top= 1 in,bottom= 1 in]{geometry}
\usepackage{dashrule}  % Package to use the command below to create lines between items
\newcommand{\litem}[1]{\item #1

\rule{\textwidth}{0.4pt}}
\pagestyle{fancy}
\lhead{}
\chead{Answer Key for Progress Quiz 5 Version C}
\rhead{}
\lfoot{9912-2038}
\cfoot{}
\rfoot{Spring 2021}
\begin{document}
\textbf{This key should allow you to understand why you choose the option you did (beyond just getting a question right or wrong). \href{https://xronos.clas.ufl.edu/mac1105spring2020/courseDescriptionAndMisc/Exams/LearningFromResults}{More instructions on how to use this key can be found here}.}

\textbf{If you have a suggestion to make the keys better, \href{https://forms.gle/CZkbZmPbC9XALEE88}{please fill out the short survey here}.}

\textit{Note: This key is auto-generated and may contain issues and/or errors. The keys are reviewed after each exam to ensure grading is done accurately. If there are issues (like duplicate options), they are noted in the offline gradebook. The keys are a work-in-progress to give students as many resources to improve as possible.}

\rule{\textwidth}{0.4pt}

\begin{enumerate}\litem{
Solve the linear inequality below. Then, choose the constant and interval combination that describes the solution set.
\[ -8 + 4 x > 6 x \text{ or } 3 + 9 x < 12 x \]The solution is \( (-\infty, -4.0) \text{ or } (1.0, \infty) \), which is option D.\begin{enumerate}[label=\Alph*.]
\item \( (-\infty, a] \cup [b, \infty), \text{ where } a \in [-3, 2] \text{ and } b \in [4, 5] \)

Corresponds to including the endpoints AND negating.
\item \( (-\infty, a) \cup (b, \infty), \text{ where } a \in [-2.6, 2.1] \text{ and } b \in [4, 7] \)

Corresponds to inverting the inequality and negating the solution.
\item \( (-\infty, a] \cup [b, \infty), \text{ where } a \in [-4, -3] \text{ and } b \in [1, 2] \)

Corresponds to including the endpoints (when they should be excluded).
\item \( (-\infty, a) \cup (b, \infty), \text{ where } a \in [-5.1, -1.2] \text{ and } b \in [1, 3] \)

 * Correct option.
\item \( (-\infty, \infty) \)

Corresponds to the variable canceling, which does not happen in this instance.
\end{enumerate}

\textbf{General Comment:} When multiplying or dividing by a negative, flip the sign.
}
\litem{
Solve the linear inequality below. Then, choose the constant and interval combination that describes the solution set.
\[ \frac{-9}{5} + \frac{5}{4} x > \frac{8}{7} x - \frac{8}{3} \]The solution is \( (-8.089, \infty) \), which is option A.\begin{enumerate}[label=\Alph*.]
\item \( (a, \infty), \text{ where } a \in [-10.09, -4.09] \)

* $(-8.089, \infty)$, which is the correct option.
\item \( (-\infty, a), \text{ where } a \in [-9.09, -5.09] \)

 $(-\infty, -8.089)$, which corresponds to switching the direction of the interval. You likely did this if you did not flip the inequality when dividing by a negative!
\item \( (-\infty, a), \text{ where } a \in [4.09, 9.09] \)

 $(-\infty, 8.089)$, which corresponds to switching the direction of the interval AND negating the endpoint. You likely did this if you did not flip the inequality when dividing by a negative as well as not moving values over to a side properly.
\item \( (a, \infty), \text{ where } a \in [6.09, 11.09] \)

 $(8.089, \infty)$, which corresponds to negating the endpoint of the solution.
\item \( \text{None of the above}. \)

You may have chosen this if you thought the inequality did not match the ends of the intervals.
\end{enumerate}

\textbf{General Comment:} Remember that less/greater than or equal to includes the endpoint, while less/greater do not. Also, remember that you need to flip the inequality when you multiply or divide by a negative.
}
\litem{
Solve the linear inequality below. Then, choose the constant and interval combination that describes the solution set.
\[ -4 + 4 x < \frac{59 x + 8}{7} \leq 5 + 8 x \]The solution is \( (-1.16, 9.00] \), which is option D.\begin{enumerate}[label=\Alph*.]
\item \( [a, b), \text{ where } a \in [-3.5, 0.3] \text{ and } b \in [5, 11] \)

$[-1.16, 9.00)$, which corresponds to flipping the inequality.
\item \( (-\infty, a) \cup [b, \infty), \text{ where } a \in [-2.6, -0.7] \text{ and } b \in [5, 12] \)

$(-\infty, -1.16) \cup [9.00, \infty)$, which corresponds to displaying the and-inequality as an or-inequality.
\item \( (-\infty, a] \cup (b, \infty), \text{ where } a \in [-1.7, -0.4] \text{ and } b \in [9, 11] \)

$(-\infty, -1.16] \cup (9.00, \infty)$, which corresponds to displaying the and-inequality as an or-inequality AND flipping the inequality.
\item \( (a, b], \text{ where } a \in [-3.16, 0.84] \text{ and } b \in [6, 12] \)

* $(-1.16, 9.00]$, which is the correct option.
\item \( \text{None of the above.} \)


\end{enumerate}

\textbf{General Comment:} To solve, you will need to break up the compound inequality into two inequalities. Be sure to keep track of the inequality! It may be best to draw a number line and graph your solution.
}
\litem{
Solve the linear inequality below. Then, choose the constant and interval combination that describes the solution set.
\[ -6x -3 > 9x -10 \]The solution is \( (-\infty, 0.467) \), which is option A.\begin{enumerate}[label=\Alph*.]
\item \( (-\infty, a), \text{ where } a \in [0.25, 0.66] \)

* $(-\infty, 0.467)$, which is the correct option.
\item \( (-\infty, a), \text{ where } a \in [-0.47, -0.21] \)

 $(-\infty, -0.467)$, which corresponds to negating the endpoint of the solution.
\item \( (a, \infty), \text{ where } a \in [-0.39, 0.59] \)

 $(0.467, \infty)$, which corresponds to switching the direction of the interval. You likely did this if you did not flip the inequality when dividing by a negative!
\item \( (a, \infty), \text{ where } a \in [-1.06, 0.39] \)

 $(-0.467, \infty)$, which corresponds to switching the direction of the interval AND negating the endpoint. You likely did this if you did not flip the inequality when dividing by a negative as well as not moving values over to a side properly.
\item \( \text{None of the above}. \)

You may have chosen this if you thought the inequality did not match the ends of the intervals.
\end{enumerate}

\textbf{General Comment:} Remember that less/greater than or equal to includes the endpoint, while less/greater do not. Also, remember that you need to flip the inequality when you multiply or divide by a negative.
}
\litem{
Using an interval or intervals, describe all the $x$-values within or including a distance of the given values.
\[ \text{ No more than } 3 \text{ units from the number } 8. \]The solution is \( \text{None of the above} \), which is option E.\begin{enumerate}[label=\Alph*.]
\item \( (-\infty, -5] \cup [11, \infty) \)

This describes the values no less than 8 from 3
\item \( (-\infty, -5) \cup (11, \infty) \)

This describes the values more than 8 from 3
\item \( [-5, 11] \)

This describes the values no more than 8 from 3
\item \( (-5, 11) \)

This describes the values less than 8 from 3
\item \( \text{None of the above} \)

Options A-D described the values [more/less than] 8 units from 3, which is the reverse of what the question asked.
\end{enumerate}

\textbf{General Comment:} When thinking about this language, it helps to draw a number line and try points.
}
\litem{
Solve the linear inequality below. Then, choose the constant and interval combination that describes the solution set.
\[ -3x + 10 \leq 3x -9 \]The solution is \( [3.167, \infty) \), which is option A.\begin{enumerate}[label=\Alph*.]
\item \( [a, \infty), \text{ where } a \in [1.17, 6.17] \)

* $[3.167, \infty)$, which is the correct option.
\item \( (-\infty, a], \text{ where } a \in [-0.83, 6.17] \)

 $(-\infty, 3.167]$, which corresponds to switching the direction of the interval. You likely did this if you did not flip the inequality when dividing by a negative!
\item \( (-\infty, a], \text{ where } a \in [-6.17, -1.17] \)

 $(-\infty, -3.167]$, which corresponds to switching the direction of the interval AND negating the endpoint. You likely did this if you did not flip the inequality when dividing by a negative as well as not moving values over to a side properly.
\item \( [a, \infty), \text{ where } a \in [-7.17, 0.83] \)

 $[-3.167, \infty)$, which corresponds to negating the endpoint of the solution.
\item \( \text{None of the above}. \)

You may have chosen this if you thought the inequality did not match the ends of the intervals.
\end{enumerate}

\textbf{General Comment:} Remember that less/greater than or equal to includes the endpoint, while less/greater do not. Also, remember that you need to flip the inequality when you multiply or divide by a negative.
}
\litem{
Solve the linear inequality below. Then, choose the constant and interval combination that describes the solution set.
\[ -3 + 4 x \leq \frac{66 x + 3}{9} < 4 + 7 x \]The solution is \( [-1.00, 11.00) \), which is option C.\begin{enumerate}[label=\Alph*.]
\item \( (-\infty, a] \cup (b, \infty), \text{ where } a \in [-4.2, 0.1] \text{ and } b \in [10, 13] \)

$(-\infty, -1.00] \cup (11.00, \infty)$, which corresponds to displaying the and-inequality as an or-inequality.
\item \( (-\infty, a) \cup [b, \infty), \text{ where } a \in [-3, 0] \text{ and } b \in [8, 13] \)

$(-\infty, -1.00) \cup [11.00, \infty)$, which corresponds to displaying the and-inequality as an or-inequality AND flipping the inequality.
\item \( [a, b), \text{ where } a \in [-2, 0] \text{ and } b \in [6, 13] \)

$[-1.00, 11.00)$, which is the correct option.
\item \( (a, b], \text{ where } a \in [-1.6, -0.5] \text{ and } b \in [10, 12] \)

$(-1.00, 11.00]$, which corresponds to flipping the inequality.
\item \( \text{None of the above.} \)


\end{enumerate}

\textbf{General Comment:} To solve, you will need to break up the compound inequality into two inequalities. Be sure to keep track of the inequality! It may be best to draw a number line and graph your solution.
}
\litem{
Solve the linear inequality below. Then, choose the constant and interval combination that describes the solution set.
\[ -5 + 4 x > 7 x \text{ or } 4 + 5 x < 6 x \]The solution is \( (-\infty, -1.667) \text{ or } (4.0, \infty) \), which is option C.\begin{enumerate}[label=\Alph*.]
\item \( (-\infty, a] \cup [b, \infty), \text{ where } a \in [-4.9, -2.9] \text{ and } b \in [-3.33, 2.67] \)

Corresponds to including the endpoints AND negating.
\item \( (-\infty, a] \cup [b, \infty), \text{ where } a \in [-1.9, -1.2] \text{ and } b \in [3, 14] \)

Corresponds to including the endpoints (when they should be excluded).
\item \( (-\infty, a) \cup (b, \infty), \text{ where } a \in [-2, -0.8] \text{ and } b \in [3.1, 7.3] \)

 * Correct option.
\item \( (-\infty, a) \cup (b, \infty), \text{ where } a \in [-4.8, -2.9] \text{ and } b \in [1, 3.3] \)

Corresponds to inverting the inequality and negating the solution.
\item \( (-\infty, \infty) \)

Corresponds to the variable canceling, which does not happen in this instance.
\end{enumerate}

\textbf{General Comment:} When multiplying or dividing by a negative, flip the sign.
}
\litem{
Solve the linear inequality below. Then, choose the constant and interval combination that describes the solution set.
\[ \frac{8}{4} - \frac{8}{9} x > \frac{-5}{3} x - \frac{7}{5} \]The solution is \( (-4.371, \infty) \), which is option C.\begin{enumerate}[label=\Alph*.]
\item \( (-\infty, a), \text{ where } a \in [4.37, 7.37] \)

 $(-\infty, 4.371)$, which corresponds to switching the direction of the interval AND negating the endpoint. You likely did this if you did not flip the inequality when dividing by a negative as well as not moving values over to a side properly.
\item \( (a, \infty), \text{ where } a \in [3.37, 8.37] \)

 $(4.371, \infty)$, which corresponds to negating the endpoint of the solution.
\item \( (a, \infty), \text{ where } a \in [-5.37, -1.37] \)

* $(-4.371, \infty)$, which is the correct option.
\item \( (-\infty, a), \text{ where } a \in [-6.37, -2.37] \)

 $(-\infty, -4.371)$, which corresponds to switching the direction of the interval. You likely did this if you did not flip the inequality when dividing by a negative!
\item \( \text{None of the above}. \)

You may have chosen this if you thought the inequality did not match the ends of the intervals.
\end{enumerate}

\textbf{General Comment:} Remember that less/greater than or equal to includes the endpoint, while less/greater do not. Also, remember that you need to flip the inequality when you multiply or divide by a negative.
}
\litem{
Using an interval or intervals, describe all the $x$-values within or including a distance of the given values.
\[ \text{ No more than } 8 \text{ units from the number } -2. \]The solution is \( [-10, 6] \), which is option A.\begin{enumerate}[label=\Alph*.]
\item \( [-10, 6] \)

This describes the values no more than 8 from -2
\item \( (-\infty, -10) \cup (6, \infty) \)

This describes the values more than 8 from -2
\item \( (-\infty, -10] \cup [6, \infty) \)

This describes the values no less than 8 from -2
\item \( (-10, 6) \)

This describes the values less than 8 from -2
\item \( \text{None of the above} \)

You likely thought the values in the interval were not correct.
\end{enumerate}

\textbf{General Comment:} When thinking about this language, it helps to draw a number line and try points.
}
\end{enumerate}

\end{document}