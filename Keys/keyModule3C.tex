\documentclass{extbook}[14pt]
\usepackage{multicol, enumerate, enumitem, hyperref, color, soul, setspace, parskip, fancyhdr, amssymb, amsthm, amsmath, bbm, latexsym, units, mathtools}
\everymath{\displaystyle}
\usepackage[headsep=0.5cm,headheight=0cm, left=1 in,right= 1 in,top= 1 in,bottom= 1 in]{geometry}
\usepackage{dashrule}  % Package to use the command below to create lines between items
\newcommand{\litem}[1]{\item #1

\rule{\textwidth}{0.4pt}}
\pagestyle{fancy}
\lhead{}
\chead{Answer Key for Progress Quiz 9 Version C}
\rhead{}
\lfoot{8590-6105}
\cfoot{}
\rfoot{Fall 2020}
\begin{document}
\textbf{This key should allow you to understand why you choose the option you did (beyond just getting a question right or wrong). \href{https://xronos.clas.ufl.edu/mac1105spring2020/courseDescriptionAndMisc/Exams/LearningFromResults}{More instructions on how to use this key can be found here}.}

\textbf{If you have a suggestion to make the keys better, \href{https://forms.gle/CZkbZmPbC9XALEE88}{please fill out the short survey here}.}

\textit{Note: This key is auto-generated and may contain issues and/or errors. The keys are reviewed after each exam to ensure grading is done accurately. If there are issues (like duplicate options), they are noted in the offline gradebook. The keys are a work-in-progress to give students as many resources to improve as possible.}

\rule{\textwidth}{0.4pt}

\begin{enumerate}\litem{
Solve the linear inequality below. Then, choose the constant and interval combination that describes the solution set.
\[ -6x + 7 > -3x -5 \]

The solution is \( (-\infty, 4.0) \), which is option A.\begin{enumerate}[label=\Alph*.]
\item \( (-\infty, a), \text{ where } a \in [0, 6] \)

* $(-\infty, 4.0)$, which is the correct option.
\item \( (-\infty, a), \text{ where } a \in [-5, -3] \)

 $(-\infty, -4.0)$, which corresponds to negating the endpoint of the solution.
\item \( (a, \infty), \text{ where } a \in [-9, -2] \)

 $(-4.0, \infty)$, which corresponds to switching the direction of the interval AND negating the endpoint. You likely did this if you did not flip the inequality when dividing by a negative as well as not moving values over to a side properly.
\item \( (a, \infty), \text{ where } a \in [1, 11] \)

 $(4.0, \infty)$, which corresponds to switching the direction of the interval. You likely did this if you did not flip the inequality when dividing by a negative!
\item \( \text{None of the above}. \)

You may have chosen this if you thought the inequality did not match the ends of the intervals.
\end{enumerate}

\textbf{General Comment:} Remember that less/greater than or equal to includes the endpoint, while less/greater do not. Also, remember that you need to flip the inequality when you multiply or divide by a negative.
}
\litem{
Solve the linear inequality below. Then, choose the constant and interval combination that describes the solution set.
\[ -4x + 10 > 10x + 3 \]

The solution is \( (-\infty, 0.5) \), which is option A.\begin{enumerate}[label=\Alph*.]
\item \( (-\infty, a), \text{ where } a \in [0.3, 2.1] \)

* $(-\infty, 0.5)$, which is the correct option.
\item \( (-\infty, a), \text{ where } a \in [-1.7, 0.3] \)

 $(-\infty, -0.5)$, which corresponds to negating the endpoint of the solution.
\item \( (a, \infty), \text{ where } a \in [-0.02, 0.95] \)

 $(0.5, \infty)$, which corresponds to switching the direction of the interval. You likely did this if you did not flip the inequality when dividing by a negative!
\item \( (a, \infty), \text{ where } a \in [-0.93, -0.12] \)

 $(-0.5, \infty)$, which corresponds to switching the direction of the interval AND negating the endpoint. You likely did this if you did not flip the inequality when dividing by a negative as well as not moving values over to a side properly.
\item \( \text{None of the above}. \)

You may have chosen this if you thought the inequality did not match the ends of the intervals.
\end{enumerate}

\textbf{General Comment:} Remember that less/greater than or equal to includes the endpoint, while less/greater do not. Also, remember that you need to flip the inequality when you multiply or divide by a negative.
}
\litem{
Solve the linear inequality below. Then, choose the constant and interval combination that describes the solution set.
\[ -3 + 3 x > 5 x \text{ or } 6 + 5 x < 7 x \]

The solution is \( (-\infty, -1.5) \text{ or } (3.0, \infty) \), which is option A.\begin{enumerate}[label=\Alph*.]
\item \( (-\infty, a) \cup (b, \infty), \text{ where } a \in [-2.5, 4.5] \text{ and } b \in [3, 5] \)

 * Correct option.
\item \( (-\infty, a] \cup [b, \infty), \text{ where } a \in [-1.5, -0.5] \text{ and } b \in [3, 4] \)

Corresponds to including the endpoints (when they should be excluded).
\item \( (-\infty, a] \cup [b, \infty), \text{ where } a \in [-6, -2] \text{ and } b \in [-6.5, 2.5] \)

Corresponds to including the endpoints AND negating.
\item \( (-\infty, a) \cup (b, \infty), \text{ where } a \in [-6, -2] \text{ and } b \in [1.5, 2.5] \)

Corresponds to inverting the inequality and negating the solution.
\item \( (-\infty, \infty) \)

Corresponds to the variable canceling, which does not happen in this instance.
\end{enumerate}

\textbf{General Comment:} When multiplying or dividing by a negative, flip the sign.
}
\litem{
Solve the linear inequality below. Then, choose the constant and interval combination that describes the solution set.
\[ -3 + 6 x \leq \frac{65 x - 5}{9} < 4 + 7 x \]

The solution is \( [-2.00, 20.50) \), which is option B.\begin{enumerate}[label=\Alph*.]
\item \( (a, b], \text{ where } a \in [-2, -1] \text{ and } b \in [16.5, 22.5] \)

$(-2.00, 20.50]$, which corresponds to flipping the inequality.
\item \( [a, b), \text{ where } a \in [-5, -1] \text{ and } b \in [20.5, 23.5] \)

$[-2.00, 20.50)$, which is the correct option.
\item \( (-\infty, a) \cup [b, \infty), \text{ where } a \in [-5, 1] \text{ and } b \in [19.5, 21.5] \)

$(-\infty, -2.00) \cup [20.50, \infty)$, which corresponds to displaying the and-inequality as an or-inequality AND flipping the inequality.
\item \( (-\infty, a] \cup (b, \infty), \text{ where } a \in [-3, 0] \text{ and } b \in [16.5, 25.5] \)

$(-\infty, -2.00] \cup (20.50, \infty)$, which corresponds to displaying the and-inequality as an or-inequality.
\item \( \text{None of the above.} \)


\end{enumerate}

\textbf{General Comment:} To solve, you will need to break up the compound inequality into two inequalities. Be sure to keep track of the inequality! It may be best to draw a number line and graph your solution.
}
\litem{
Using an interval or intervals, describe all the $x$-values within or including a distance of the given values.
\[ \text{ Less than } 3 \text{ units from the number } 6. \]

The solution is \( \text{None of the above} \), which is option E.\begin{enumerate}[label=\Alph*.]
\item \( (-\infty, -3] \cup [9, \infty) \)

This describes the values no less than 6 from 3
\item \( [-3, 9] \)

This describes the values no more than 6 from 3
\item \( (-3, 9) \)

This describes the values less than 6 from 3
\item \( (-\infty, -3) \cup (9, \infty) \)

This describes the values more than 6 from 3
\item \( \text{None of the above} \)

Options A-D described the values [more/less than] 6 units from 3, which is the reverse of what the question asked.
\end{enumerate}

\textbf{General Comment:} When thinking about this language, it helps to draw a number line and try points.
}
\litem{
Using an interval or intervals, describe all the $x$-values within or including a distance of the given values.
\[ \text{ No more than } 2 \text{ units from the number } -4. \]

The solution is \( [-6, -2] \), which is option B.\begin{enumerate}[label=\Alph*.]
\item \( (-\infty, -6) \cup (-2, \infty) \)

This describes the values more than 2 from -4
\item \( [-6, -2] \)

This describes the values no more than 2 from -4
\item \( (-6, -2) \)

This describes the values less than 2 from -4
\item \( (-\infty, -6] \cup [-2, \infty) \)

This describes the values no less than 2 from -4
\item \( \text{None of the above} \)

You likely thought the values in the interval were not correct.
\end{enumerate}

\textbf{General Comment:} When thinking about this language, it helps to draw a number line and try points.
}
\litem{
Solve the linear inequality below. Then, choose the constant and interval combination that describes the solution set.
\[ -4 + 7 x > 8 x \text{ or } 8 + 4 x < 5 x \]

The solution is \( (-\infty, -4.0) \text{ or } (8.0, \infty) \), which is option B.\begin{enumerate}[label=\Alph*.]
\item \( (-\infty, a) \cup (b, \infty), \text{ where } a \in [-10, -7] \text{ and } b \in [4, 6] \)

Corresponds to inverting the inequality and negating the solution.
\item \( (-\infty, a) \cup (b, \infty), \text{ where } a \in [-7, -3] \text{ and } b \in [8, 13] \)

 * Correct option.
\item \( (-\infty, a] \cup [b, \infty), \text{ where } a \in [-5, -1] \text{ and } b \in [7, 12] \)

Corresponds to including the endpoints (when they should be excluded).
\item \( (-\infty, a] \cup [b, \infty), \text{ where } a \in [-9, -5] \text{ and } b \in [3, 5] \)

Corresponds to including the endpoints AND negating.
\item \( (-\infty, \infty) \)

Corresponds to the variable canceling, which does not happen in this instance.
\end{enumerate}

\textbf{General Comment:} When multiplying or dividing by a negative, flip the sign.
}
\litem{
Solve the linear inequality below. Then, choose the constant and interval combination that describes the solution set.
\[ \frac{8}{5} - \frac{5}{8} x \leq \frac{-4}{6} x + \frac{3}{2} \]

The solution is \( (-\infty, -2.4] \), which is option D.\begin{enumerate}[label=\Alph*.]
\item \( (-\infty, a], \text{ where } a \in [2.4, 5.4] \)

 $(-\infty, 2.4]$, which corresponds to negating the endpoint of the solution.
\item \( [a, \infty), \text{ where } a \in [-4.4, 0.6] \)

 $[-2.4, \infty)$, which corresponds to switching the direction of the interval. You likely did this if you did not flip the inequality when dividing by a negative!
\item \( [a, \infty), \text{ where } a \in [0.4, 6.4] \)

 $[2.4, \infty)$, which corresponds to switching the direction of the interval AND negating the endpoint. You likely did this if you did not flip the inequality when dividing by a negative as well as not moving values over to a side properly.
\item \( (-\infty, a], \text{ where } a \in [-5.4, 0.6] \)

* $(-\infty, -2.4]$, which is the correct option.
\item \( \text{None of the above}. \)

You may have chosen this if you thought the inequality did not match the ends of the intervals.
\end{enumerate}

\textbf{General Comment:} Remember that less/greater than or equal to includes the endpoint, while less/greater do not. Also, remember that you need to flip the inequality when you multiply or divide by a negative.
}
\litem{
Solve the linear inequality below. Then, choose the constant and interval combination that describes the solution set.
\[ 4 - 5 x < \frac{-25 x - 5}{9} \leq 6 - 3 x \]

The solution is \( (2.05, 29.50] \), which is option C.\begin{enumerate}[label=\Alph*.]
\item \( (-\infty, a) \cup [b, \infty), \text{ where } a \in [0.05, 4.05] \text{ and } b \in [25.5, 34.5] \)

$(-\infty, 2.05) \cup [29.50, \infty)$, which corresponds to displaying the and-inequality as an or-inequality.
\item \( [a, b), \text{ where } a \in [2.05, 3.05] \text{ and } b \in [25.5, 33.5] \)

$[2.05, 29.50)$, which corresponds to flipping the inequality.
\item \( (a, b], \text{ where } a \in [0.05, 6.05] \text{ and } b \in [29.5, 30.5] \)

* $(2.05, 29.50]$, which is the correct option.
\item \( (-\infty, a] \cup (b, \infty), \text{ where } a \in [-0.95, 8.05] \text{ and } b \in [27.5, 32.5] \)

$(-\infty, 2.05] \cup (29.50, \infty)$, which corresponds to displaying the and-inequality as an or-inequality AND flipping the inequality.
\item \( \text{None of the above.} \)


\end{enumerate}

\textbf{General Comment:} To solve, you will need to break up the compound inequality into two inequalities. Be sure to keep track of the inequality! It may be best to draw a number line and graph your solution.
}
\litem{
Solve the linear inequality below. Then, choose the constant and interval combination that describes the solution set.
\[ \frac{8}{3} + \frac{6}{8} x < \frac{7}{2} x - \frac{3}{7} \]

The solution is \( (1.126, \infty) \), which is option B.\begin{enumerate}[label=\Alph*.]
\item \( (-\infty, a), \text{ where } a \in [0.13, 2.13] \)

 $(-\infty, 1.126)$, which corresponds to switching the direction of the interval. You likely did this if you did not flip the inequality when dividing by a negative!
\item \( (a, \infty), \text{ where } a \in [0.13, 2.13] \)

* $(1.126, \infty)$, which is the correct option.
\item \( (-\infty, a), \text{ where } a \in [-1.13, 0.87] \)

 $(-\infty, -1.126)$, which corresponds to switching the direction of the interval AND negating the endpoint. You likely did this if you did not flip the inequality when dividing by a negative as well as not moving values over to a side properly.
\item \( (a, \infty), \text{ where } a \in [-1.13, 0.87] \)

 $(-1.126, \infty)$, which corresponds to negating the endpoint of the solution.
\item \( \text{None of the above}. \)

You may have chosen this if you thought the inequality did not match the ends of the intervals.
\end{enumerate}

\textbf{General Comment:} Remember that less/greater than or equal to includes the endpoint, while less/greater do not. Also, remember that you need to flip the inequality when you multiply or divide by a negative.
}
\end{enumerate}

\end{document}