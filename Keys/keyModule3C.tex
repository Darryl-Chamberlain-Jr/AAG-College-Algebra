
\documentclass{article}[14pt]
\usepackage{multicol, enumerate, enumitem, hyperref, color, soul, setspace, parskip, fancyhdr, amssymb, amsthm, amsmath, bbm, latexsym, units, mathtools}
\everymath{\displaystyle}
\usepackage[headsep=0.5cm,headheight=0cm, left=1 in,right= 1 in,top= 1 in,bottom= 1 in]{geometry}
\pagestyle{fancy}
\lhead{}
\chead{Answer Key for Module\,3\,-\,Inequalities Version C}
\rhead{}
\lfoot{Summer\,C\,2020}
\cfoot{}
\rfoot{}
\begin{document}
\textbf{This key should allow you to understand why you choose the option you did (beyond just getting a question right or wrong). \href{https://xronos.clas.ufl.edu/mac1105spring2020/courseDescriptionAndMisc/Exams/LearningFromResults}{More instructions on how to use this key can be found here}.}

\textbf{If you have a suggestion to make the keys better, \href{https://forms.gle/CZkbZmPbC9XALEE88}{please fill out the short survey here}.}

\textit{Note: This key is auto-generated and may contain issues and/or errors. The keys are reviewed after each exam to ensure grading is done accurately. If there are issues (like duplicate options), they are noted in the offline gradebook. The keys are a work-in-progress to give students as many resources to improve as possible.}

\rule{\textwidth}{0.4pt}

11. Solve the linear inequality below. Then, choose the constant and interval combination that describes the solution set.
$$ 3x -3 > 4x + 4 $$ 
The solution is $ (-\infty, -7.0) $ 

\begin{enumerate}[label=\Alph*.] 
\item $ (a, \infty), \text{ where } a \in [-10, 1] $ 

  $(-7.0, \infty)$, which corresponds to switching the direction of the interval. You likely did this if you did not flip the inequality when dividing by a negative! 
\item $ (-\infty, a), \text{ where } a \in [6, 8] $ 

  $(-\infty, 7.0)$, which corresponds to negating the endpoint of the solution. 
\item $ (-\infty, a), \text{ where } a \in [-8, -1] $ 

 * $(-\infty, -7.0)$, which is the correct option. 
\item $ (a, \infty), \text{ where } a \in [6, 8] $ 

  $(7.0, \infty)$, which corresponds to switching the direction of the interval AND negating the endpoint. You likely did this if you did not flip the inequality when dividing by a negative as well as not moving values over to a side properly. 
\item $ \text{None of the above}. $ 

 You may have chosen this if you thought the inequality did not match the ends of the intervals. 
\end{enumerate} 
 
General Comments: Remember that less/greater than or equal to includes the endpoint, while less/greater do not. Also, remember that you need to flip the inequality when you multiply or divide by a negative.

-----------------------------------------------

12. Solve the linear inequality below. Then, choose the constant and interval combination that describes the solution set.
$$ -8 + 4 x > 6 x \text{ or } -3 + 8 x < 11 x $$ 
The solution is $ (-\infty, -4.0) \text{ or } (-1.0, \infty) $ 

\begin{enumerate}[label=\Alph*.] 
\item $ (-\infty, a] \cup [b, \infty), \text{ where } a \in [0, 6] \text{ and } b \in [2, 5] $ 

 Corresponds to including the endpoints AND negating. 
\item $ (-\infty, a] \cup [b, \infty), \text{ where } a \in [-8, -3] \text{ and } b \in [-5, 2] $ 

 Corresponds to including the endpoints (when they should be excluded). 
\item $ (-\infty, a) \cup (b, \infty), \text{ where } a \in [-9, -1] \text{ and } b \in [-3, 3] $ 

  * Correct option. 
\item $ (-\infty, a) \cup (b, \infty), \text{ where } a \in [-2, 2] \text{ and } b \in [0, 5] $ 

 Corresponds to inverting the inequality and negating the solution. 
\item $ (-\infty, \infty) $ 

 Corresponds to the variable canceling, which does not happen in this instance. 
\end{enumerate} 
 
General Comments: When multiplying or dividing by a negative, flip the sign.

-----------------------------------------------

13. Solve the linear inequality below. Then, choose the constant and interval combination that describes the solution set.
$$ \frac{9}{2} + \frac{4}{7} x < \frac{5}{4} x - \frac{10}{5} $$ 
The solution is $ (9.579, \infty) $ 

\begin{enumerate}[label=\Alph*.] 
\item $ (-\infty, a), \text{ where } a \in [-14, -8] $ 

  $(-\infty, -9.579)$, which corresponds to switching the direction of the interval AND negating the endpoint. You likely did this if you did not flip the inequality when dividing by a negative as well as not moving values over to a side properly. 
\item $ (-\infty, a), \text{ where } a \in [6, 11] $ 

  $(-\infty, 9.579)$, which corresponds to switching the direction of the interval. You likely did this if you did not flip the inequality when dividing by a negative! 
\item $ (a, \infty), \text{ where } a \in [7, 14] $ 

 * $(9.579, \infty)$, which is the correct option. 
\item $ (a, \infty), \text{ where } a \in [-13, -7] $ 

  $(-9.579, \infty)$, which corresponds to negating the endpoint of the solution. 
\item $ \text{None of the above}. $ 

 You may have chosen this if you thought the inequality did not match the ends of the intervals. 
\end{enumerate} 
 
General Comments: Remember that less/greater than or equal to includes the endpoint, while less/greater do not. Also, remember that you need to flip the inequality when you multiply or divide by a negative.

-----------------------------------------------

14. Solve the linear inequality below. Then, choose the constant and interval combination that describes the solution set.
$$ -3 + 8 x < \frac{76 x + 4}{9} \leq -5 + 7 x $$ 
The solution is $ (-7.75, -3.77] $ 

\begin{enumerate}[label=\Alph*.] 
\item $ [a, b), \text{ where } a \in [-12, -3] \text{ and } b \in [-7, 2] $ 

 $[-7.75, -3.77)$, which corresponds to flipping the inequality. 
\item $ (-\infty, a) \cup [b, \infty), \text{ where } a \in [-12, -6] \text{ and } b \in [-5, 0] $ 

 $(-\infty, -7.75) \cup [-3.77, \infty)$, which corresponds to displaying the and-inequality as an or-inequality. 
\item $ (a, b], \text{ where } a \in [-9, -6] \text{ and } b \in [-6, 0] $ 

 * $(-7.75, -3.77]$, which is the correct option. 
\item $ (-\infty, a] \cup (b, \infty), \text{ where } a \in [-12, -2] \text{ and } b \in [-10, -3] $ 

 $(-\infty, -7.75] \cup (-3.77, \infty)$, which corresponds to displaying the and-inequality as an or-inequality AND flipping the inequality. 
\item $ \text{None of the above.} $ 

  
\end{enumerate} 
 
To solve, you will need to break up the compound inequality into two inequalities. Be sure to keep track of the inequality! It may be best to draw a number line and graph your solution.

-----------------------------------------------

15. Using an interval or intervals, describe all the $x$-values within or including a distance of the given values.
$$ \text{ More than } 8 \text{ units from the number } 7. $$ 
The solution is $ \text{None of the above} $ 

\begin{enumerate}[label=\Alph*.] 
\item $ (1, 15) $ 

 This describes the values less than 7 from 8 
\item $ [1, 15] $ 

 This describes the values no more than 7 from 8 
\item $ (-\infty, 1) \cup (15, \infty) $ 

 This describes the values more than 7 from 8 
\item $ (-\infty, 1] \cup [15, \infty) $ 

 This describes the values no less than 7 from 8 
\item $ \text{None of the above} $ 

 Options A-D described the values [more/less than] 7 units from 8, which is the reverse of what the question asked. 
\end{enumerate} 
 
General Comments: When thinking about this language, it helps to draw a number line and try points.

-----------------------------------------------


\end{document}

