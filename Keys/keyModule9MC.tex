
\documentclass{article}[14pt]
\usepackage{multicol, enumerate, enumitem, hyperref, color, soul, setspace, parskip, fancyhdr, amssymb, amsthm, amsmath, bbm, latexsym, units, mathtools}
\everymath{\displaystyle}
\usepackage[headsep=0.5cm,headheight=0cm, left=1 in,right= 1 in,top= 1 in,bottom= 1 in]{geometry}
\pagestyle{fancy}
\lhead{}
\chead{Answer Key for Module\,9M\,-\,Modeling\,Linear\,Functions Version C}
\rhead{}
\lfoot{Summer\,C\,2020}
\cfoot{}
\rfoot{}
\begin{document}
\textbf{This key should allow you to understand why you choose the option you did (beyond just getting a question right or wrong). \href{https://xronos.clas.ufl.edu/mac1105spring2020/courseDescriptionAndMisc/Exams/LearningFromResults}{More instructions on how to use this key can be found here}.}

\textbf{If you have a suggestion to make the keys better, \href{https://forms.gle/CZkbZmPbC9XALEE88}{please fill out the short survey here}.}

\textit{Note: This key is auto-generated and may contain issues and/or errors. The keys are reviewed after each exam to ensure grading is done accurately. If there are issues (like duplicate options), they are noted in the offline gradebook. The keys are a work-in-progress to give students as many resources to improve as possible.}

\rule{\textwidth}{0.4pt}

41. What is the \textbf{best} way to describe the domain of the scenario below?
\begin{center} \textit{Bridges on highways often have expansion joints, which are small gaps in the roadway between one bridge section and the next. The gaps are put there so the bridge will have room to expand when the weather gets hot. Assume the gap width varies constantly with the temperature. Suppose a bridge has a gap of 1.3 cm when the temperature is 22 degrees C and that the gap narrows to 0.9 cm when the temperature warms to 30 degrees C.} \end{center} 
The solution is $ \text{There is no restricted domain in this scenario} $ 

\begin{enumerate}[label=\Alph*.] 
\item $ \text{There is no restricted domain in this scenario} $ 

 This means we have a domain of the Real numbers and we don't need to remove any values even in the real-world context. 
\item $ \text{Subset of the Natural numbers} $ 

 Recall that the Naturals are the counting numbers: 1, 2, 3, ... 
\item $ \text{Subset of the Integers} $ 

 Recall that the Integers are the positive and negative counting numbers: ..., -3, -2, -1, 0, 1, 2, 3, ...  
\item $ \text{Subset of the Rational numbers} $ 

 Recall that the Rationals are fractions with Integers in the numerator and denominator. 
\item $ \text{Proper subset of the Real numbers} $ 

 This means we have a domain of the Real numbers but need to throw out values based on the context. 
\end{enumerate} 
 
\textbf{General Comments:} We often have to remove values in the domain when working with real-world models.

-----------------------------------------------

42. For the information provided below, construct a linear model that describes the total distance of the path, $D$, in terms of the time spent on a particular path \textit{if we know that all parts of the path are equal length}.
\begin{center} \textit{A bicyclist is training for a race on a hilly path. Their bike keeps track of their speed at any time, but not the distance traveled. Their speed traveling up a hill is 5 mph, 10 mph when traveling down a hill, and 6 mph when traveling along a flat portion.} \end{center} 
The solution is $ \text{The model can be found with the information provided, but isn't options 1-3.} $ 

\begin{enumerate}[label=\Alph*.] 
\item $ 0.467 t $ 

 The coefficient here is calculated as if you were trying to model the time on the total path. 
\item $ 21 t $ 

 This would be correct if we knew the time spent on each path was equal. 
\item $ 300 t $ 

 The coefficient here is calculated by multiplying the speeds together rather than adding them. 
\item $ \text{The model can be found with the information provided, but isn't options 1-3.} $ 

 * This is the correct option. Since the paths are equal length and the bike can travel different speeds on each part, the time spent on each path is not equal! The model would be $5t_u + 10t_d +6t_f$, where $t_u$ is time traveling up, $t_d$ is time traveling down, and $t_f$ is time traveling on a flat portion. 
\item $ \text{The model cannot be found with the information provided.} $ 

 If you chose this option, please contact the coordinator to discuss why you think we cannot model the situation. 
\end{enumerate} 
 
\textbf{General Comments:} Be sure you pay attention to the variable we are writing the model in terms of. To create the model with a single variable, we have to know that variable is the same throughout each path!

-----------------------------------------------

43. For the information provided below, construct a linear model that describes her total budget, $B$, as a function of the number of months, $x$ she is at UF.
\begin{center} \textit{Aubrey is a college student going into her first year at UF. She will receive Bright Futures, which covers her tuition plus a \$400 educational expense each year. Before college, Aubrey saved up \$10000. She knows she will need to pay \$1100 in rent a month, \$80 for food a week, and \$64 in other weekly expenses.} \end{center} 
The solution is $ B(x) = 1676 - 10400 x $ 

\begin{enumerate}[label=\Alph*.] 
\item $ B(x) = 9156 x $ 

 This treats the educational expense and savings as something you get every month rather than a 1-time payment AND treats weekly expenses as month expenses rather than multiplying each weekly expense by 4. 
\item $ B(x) = 8724 x $ 

 This treats the educational expense and savings as something you get every month rather than a 1-time payment. 
\item $ B(x) = 1244 - 10400 x $ 

 This treats weekly expenses as month expenses rather than multiplying each weekly expense. 
\item $ B(x) = 1676 - 10400 x $ 

 * This is the correct option. 
\item $ \text{None of the above.} $ 

 You may have chosen this if you thought you were modeling total costs or income. 
\end{enumerate} 
 
\textbf{General Comments:} This is a Costs, Profit, Revenue question! The most common issues here are: (1) not converting the weekly costs to monthly costs, (2) treating the one-time values like savings and educational expense as happening per month, and (3) not checking that your model is for cost, profit [income], or revenue [budget].

-----------------------------------------------

44. A town has an initial population of 100000. The town's population for the next 10 years is provided below. Which type of function would be most appropriate to model the town's population?
Check for table in main PDF. 
The solution is $ \text{Direct variation} $ 

\begin{enumerate}[label=\Alph*.] 
\item $ \text{Exponential} $ 

 This suggests the fastest of growths that we know. 
\item $ \text{Logarithmic} $ 

 This suggests the slowest of growths that we know. 
\item $ \text{Linear} $ 

 This suggests a constant growth. You would be able to add or subtract the same amount year-to-year if this is the correct answer. 
\item $ \text{Direct variation} $ 

 This suggests a growth faster than constant but slower than exponential. 
\item $ \text{Indirect variation} $ 

 This suggests a growth slower than constant but faster than logarithmic. 
\end{enumerate} 
 
\textbf{General Comments:} We are trying to compare the growth rate of the population. Growth rates can be characterized from slowest to fastest as: logarithmic, indirect, linear, direct, exponential. The best way to approach this is to first compare it to linear (is it linear, faster than linear, or slower than linear)? If faster, is it as fast as exponential? If slower, is it as slow as logarithmic?

-----------------------------------------------

45. Using the situation below, construct a linear model that describes the cost of the coffee beans $C(h)$ in terms of the weight of the high-quality coffee beans $h$.
 \begin{center} \textit{Veronica needs to prepare 160 of blended coffee beans selling for \$3.36 per pound. She has a high-quality bean that sells for \$4.09 a pound and a low-quality bean that sells for \$2.27 a pound.} \end{center} 
The solution is $ C(h) = 1.82 h + 363.20 $ 

\begin{enumerate}[label=\Alph*.] 
\item $ C(h) = 4.09 h $ 

 This models the cost of the high-quality bean only, not the blended beans. 
\item $ C(h) = 3.18 h $ 

 This assumes that exactly half of the high- and low- quality beans are mixed to create the blended coffee beans. 
\item $ C(h) = -1.82 h + 654.40 $ 

 This would be correct if the question asked you to construct the cost model in terms of the weight of the low-quality bean. 
\item $ C(h) = 1.82 h + 363.20 $ 

 * This is the correct option since the questions asked you to construct the cost model in terms of the weight of the high-quality bean. 
\item $ \text{None of the above.} $ 

 If you chose this option, please talk to the coordinator to discuss why. 
\end{enumerate} 
 
\textbf{General Comments:} This is exactly like the chemistry mixture question from the homework! If you are having trouble with this problem, be sure to review the video for building linear models.

-----------------------------------------------


\end{document}

