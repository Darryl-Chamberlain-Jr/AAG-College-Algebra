\documentclass{extbook}[14pt]
\usepackage{multicol, enumerate, enumitem, hyperref, color, soul, setspace, parskip, fancyhdr, amssymb, amsthm, amsmath, bbm, latexsym, units, mathtools}
\everymath{\displaystyle}
\usepackage[headsep=0.5cm,headheight=0cm, left=1 in,right= 1 in,top= 1 in,bottom= 1 in]{geometry}
\usepackage{dashrule}  % Package to use the command below to create lines between items
\newcommand{\litem}[1]{\item #1

\rule{\textwidth}{0.4pt}}
\pagestyle{fancy}
\lhead{}
\chead{Answer Key for Progress Quiz 4 Version A}
\rhead{}
\lfoot{6286-1986}
\cfoot{}
\rfoot{Fall 2020}
\begin{document}
\textbf{This key should allow you to understand why you choose the option you did (beyond just getting a question right or wrong). \href{https://xronos.clas.ufl.edu/mac1105spring2020/courseDescriptionAndMisc/Exams/LearningFromResults}{More instructions on how to use this key can be found here}.}

\textbf{If you have a suggestion to make the keys better, \href{https://forms.gle/CZkbZmPbC9XALEE88}{please fill out the short survey here}.}

\textit{Note: This key is auto-generated and may contain issues and/or errors. The keys are reviewed after each exam to ensure grading is done accurately. If there are issues (like duplicate options), they are noted in the offline gradebook. The keys are a work-in-progress to give students as many resources to improve as possible.}

\rule{\textwidth}{0.4pt}

\begin{enumerate}\litem{
Solve the equation for $x$ and choose the interval that contains the solution (if it exists).
\[ 4^{5x-5} = 9^{4x+4} \]
The solution is \( x = -8.464 \), which is option D.\begin{enumerate}[label=\Alph*.]
\item \( x \in [14.72, 16.72] \)

$x = 15.720$, which corresponds to distributing the $\ln(base)$ to the second term of the exponent only.
\item \( x \in [-5.85, -1.85] \)

$x = -4.845$, which corresponds to distributing the $\ln(base)$ to the first term of the exponent only.
\item \( x \in [9, 12] \)

$x = 9.000$, which corresponds to solving the numerators as equal while ignoring the bases are different.
\item \( x \in [-11.46, -6.46] \)

* $x = -8.464$, which is the correct option.
\item \( \text{There is no Real solution to the equation.} \)

This corresponds to believing there is no solution since the bases are not powers of each other.
\end{enumerate}

\textbf{General Comment:} \textbf{General Comments:} This question was written so that the bases could not be written the same. You will need to take the log of both sides.
}
\litem{
Solve the equation for $x$ and choose the interval that contains the solution (if it exists).
\[ 2^{2x-4} = \left(\frac{1}{25}\right)^{5x+5} \]
The solution is \( x = -0.762 \), which is option C.\begin{enumerate}[label=\Alph*.]
\item \( x \in [-0.2, 1.3] \)

$x = 0.515$, which corresponds to distributing the $\ln(base)$ to the first term of the exponent only.
\item \( x \in [4.2, 4.5] \)

$x = 4.441$, which corresponds to distributing the $\ln(base)$ to the second term of the exponent only.
\item \( x \in [-0.8, -0.7] \)

* $x = -0.762$, which is the correct option.
\item \( x \in [-3.8, -2.6] \)

$x = -3.000$, which corresponds to solving the numerators as equal while ignoring the bases are different.
\item \( \text{There is no Real solution to the equation.} \)

This corresponds to believing there is no solution since the bases are not powers of each other.
\end{enumerate}

\textbf{General Comment:} \textbf{General Comments:} This question was written so that the bases could not be written the same. You will need to take the log of both sides.
}
\litem{
Which of the following intervals describes the Domain of the function below?
\[ f(x) = e^{x+7}+8 \]
The solution is \( (-\infty, \infty) \), which is option E.\begin{enumerate}[label=\Alph*.]
\item \( [a, \infty), a \in [-10, -3] \)

$[-8, \infty)$, which corresponds to using the negative vertical shift AND flipping the Range interval AND including the endpoint.
\item \( (-\infty, a), a \in [7, 11] \)

$(-\infty, 8)$, which corresponds to using the correct vertical shift *if we wanted the Range*.
\item \( (a, \infty), a \in [-10, -3] \)

$(-8, \infty)$, which corresponds to using the negative vertical shift AND flipping the Range interval.
\item \( (-\infty, a], a \in [7, 11] \)

$(-\infty, 8]$, which corresponds to using the correct vertical shift *if we wanted the Range* AND including the endpoint.
\item \( (-\infty, \infty) \)

* This is the correct option.
\end{enumerate}

\textbf{General Comment:} \textbf{General Comments}: Domain of a basic exponential function is $(-\infty, \infty)$ while the Range is $(0, \infty)$. We can shift these intervals [and even flip when $a<0$!] to find the new Domain/Range.
}
\litem{
Which of the following intervals describes the Domain of the function below?
\[ f(x) = -e^{x-6}+8 \]
The solution is \( (-\infty, \infty) \), which is option E.\begin{enumerate}[label=\Alph*.]
\item \( (-\infty, a], a \in [2, 14] \)

$(-\infty, 8]$, which corresponds to using the correct vertical shift *if we wanted the Range* AND including the endpoint.
\item \( [a, \infty), a \in [-10, -6] \)

$[-8, \infty)$, which corresponds to using the negative vertical shift AND flipping the Range interval AND including the endpoint.
\item \( (a, \infty), a \in [-10, -6] \)

$(-8, \infty)$, which corresponds to using the negative vertical shift AND flipping the Range interval.
\item \( (-\infty, a), a \in [2, 14] \)

$(-\infty, 8)$, which corresponds to using the correct vertical shift *if we wanted the Range*.
\item \( (-\infty, \infty) \)

* This is the correct option.
\end{enumerate}

\textbf{General Comment:} \textbf{General Comments}: Domain of a basic exponential function is $(-\infty, \infty)$ while the Range is $(0, \infty)$. We can shift these intervals [and even flip when $a<0$!] to find the new Domain/Range.
}
\litem{
Which of the following intervals describes the Domain of the function below?
\[ f(x) = \log_2{(x+8)}-1 \]
The solution is \( (-8, \infty) \), which is option C.\begin{enumerate}[label=\Alph*.]
\item \( [a, \infty), a \in [-1.6, -0.88] \)

$[-1, \infty)$, which corresponds to using the vertical shift when shifting the Domain AND including the endpoint.
\item \( (-\infty, a), a \in [6.97, 8.07] \)

$(-\infty, 8)$, which corresponds to flipping the Domain. Remember: the general for is $a*\log(x-h)+k$, \textbf{where $a$ does not affect the domain}.
\item \( (a, \infty), a \in [-8.55, -7.8] \)

* $(-8, \infty)$, which is the correct option.
\item \( (-\infty, a], a \in [-0.11, 2.96] \)

$(-\infty, 1]$, which corresponds to using the negative vertical shift AND including the endpoint AND flipping the domain.
\item \( (-\infty, \infty) \)

This corresponds to thinking of the range of the log function (or the domain of the exponential function).
\end{enumerate}

\textbf{General Comment:} \textbf{General Comments}: The domain of a basic logarithmic function is $(0, \infty)$ and the Range is $(-\infty, \infty)$. We can use shifts when finding the Domain, but the Range will always be all Real numbers.
}
\litem{
 Solve the equation for $x$ and choose the interval that contains $x$ (if it exists).
\[  15 = \sqrt[4]{\frac{20}{e^{7x}}} \]
The solution is \( x = -1.119 \), which is option A.\begin{enumerate}[label=\Alph*.]
\item \( x \in [-1.36, -1.09] \)

* $x = -1.119$, which is the correct option.
\item \( x \in [-0.74, 0.13] \)

$x = -0.346$, which corresponds to treating any root as a square root.
\item \( x \in [-9.79, -8.72] \)

$x = -8.999$, which corresponds to thinking you don't need to take the natural log of both sides before reducing, as if the equation already had a natural log on the right side.
\item \( \text{There is no Real solution to the equation.} \)

This corresponds to believing you cannot solve the equation.
\item \( \text{None of the above.} \)

This corresponds to making an unexpected error.
\end{enumerate}

\textbf{General Comment:} \textbf{General Comments}: After using the properties of logarithmic functions to break up the right-hand side, use $\ln(e) = 1$ to reduce the question to a linear function to solve. You can put $\ln(20)$ into a calculator if you are having trouble.
}
\litem{
Which of the following intervals describes the Domain of the function below?
\[ f(x) = \log_2{(x-1)}+6 \]
The solution is \( (1, \infty) \), which is option A.\begin{enumerate}[label=\Alph*.]
\item \( (a, \infty), a \in [0.7, 2.2] \)

* $(1, \infty)$, which is the correct option.
\item \( [a, \infty), a \in [4.2, 6.8] \)

$[6, \infty)$, which corresponds to using the vertical shift when shifting the Domain AND including the endpoint.
\item \( (-\infty, a], a \in [-6.6, -3.2] \)

$(-\infty, -6]$, which corresponds to using the negative vertical shift AND including the endpoint AND flipping the domain.
\item \( (-\infty, a), a \in [-1.7, 0.8] \)

$(-\infty, -1)$, which corresponds to flipping the Domain. Remember: the general for is $a*\log(x-h)+k$, \textbf{where $a$ does not affect the domain}.
\item \( (-\infty, \infty) \)

This corresponds to thinking of the range of the log function (or the domain of the exponential function).
\end{enumerate}

\textbf{General Comment:} \textbf{General Comments}: The domain of a basic logarithmic function is $(0, \infty)$ and the Range is $(-\infty, \infty)$. We can use shifts when finding the Domain, but the Range will always be all Real numbers.
}
\litem{
Solve the equation for $x$ and choose the interval that contains the solution (if it exists).
\[ \log_{2}{(-3x+6)}+6 = 2 \]
The solution is \( x = 1.979 \), which is option C.\begin{enumerate}[label=\Alph*.]
\item \( x \in [-7.81, -7.27] \)

$x = -7.333$, which corresponds to reversing the base and exponent when converting and reversing the value with $x$.
\item \( x \in [-5.08, -2.43] \)

$x = -3.333$, which corresponds to reversing the base and exponent when converting.
\item \( x \in [1.67, 4.29] \)

* $x = 1.979$, which is the correct option.
\item \( x \in [-0.45, 1.08] \)

$x = 0.667$, which corresponds to ignoring the vertical shift when converting to exponential form.
\item \( \text{There is no Real solution to the equation.} \)

Corresponds to believing a negative coefficient within the log equation means there is no Real solution.
\end{enumerate}

\textbf{General Comment:} \textbf{General Comments:} First, get the equation in the form $\log_b{(cx+d)} = a$. Then, convert to $b^a = cx+d$ and solve.
}
\litem{
Solve the equation for $x$ and choose the interval that contains the solution (if it exists).
\[ \log_{3}{(-2x+5)}+5 = 3 \]
The solution is \( x = 2.444 \), which is option B.\begin{enumerate}[label=\Alph*.]
\item \( x \in [0.68, 1.87] \)

$x = 1.500$, which corresponds to reversing the base and exponent when converting and reversing the value with $x$.
\item \( x \in [1.66, 2.67] \)

* $x = 2.444$, which is the correct option.
\item \( x \in [6.26, 7.44] \)

$x = 6.500$, which corresponds to reversing the base and exponent when converting.
\item \( x \in [-12.48, -10.73] \)

$x = -11.000$, which corresponds to ignoring the vertical shift when converting to exponential form.
\item \( \text{There is no Real solution to the equation.} \)

Corresponds to believing a negative coefficient within the log equation means there is no Real solution.
\end{enumerate}

\textbf{General Comment:} \textbf{General Comments:} First, get the equation in the form $\log_b{(cx+d)} = a$. Then, convert to $b^a = cx+d$ and solve.
}
\litem{
 Solve the equation for $x$ and choose the interval that contains $x$ (if it exists).
\[  6 = \sqrt[6]{\frac{8}{e^{8x}}} \]
The solution is \( x = -1.084 \), which is option C.\begin{enumerate}[label=\Alph*.]
\item \( x \in [-0.63, -0.06] \)

$x = -0.188$, which corresponds to treating any root as a square root.
\item \( x \in [-4.9, -4.63] \)

$x = -4.760$, which corresponds to thinking you don't need to take the natural log of both sides before reducing, as if the equation already had a natural log on the right side.
\item \( x \in [-1.38, -0.72] \)

* $x = -1.084$, which is the correct option.
\item \( \text{There is no Real solution to the equation.} \)

This corresponds to believing you cannot solve the equation.
\item \( \text{None of the above.} \)

This corresponds to making an unexpected error.
\end{enumerate}

\textbf{General Comment:} \textbf{General Comments}: After using the properties of logarithmic functions to break up the right-hand side, use $\ln(e) = 1$ to reduce the question to a linear function to solve. You can put $\ln(8)$ into a calculator if you are having trouble.
}
\end{enumerate}

\end{document}