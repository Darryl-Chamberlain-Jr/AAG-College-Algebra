\documentclass{extbook}[14pt]
\usepackage{multicol, enumerate, enumitem, hyperref, color, soul, setspace, parskip, fancyhdr, amssymb, amsthm, amsmath, bbm, latexsym, units, mathtools}
\everymath{\displaystyle}
\usepackage[headsep=0.5cm,headheight=0cm, left=1 in,right= 1 in,top= 1 in,bottom= 1 in]{geometry}
\usepackage{dashrule}  % Package to use the command below to create lines between items
\newcommand{\litem}[1]{\item #1

\rule{\textwidth}{0.4pt}}
\pagestyle{fancy}
\lhead{}
\chead{Answer Key for Progress Quiz 9 Version A}
\rhead{}
\lfoot{8590-6105}
\cfoot{}
\rfoot{Fall 2020}
\begin{document}
\textbf{This key should allow you to understand why you choose the option you did (beyond just getting a question right or wrong). \href{https://xronos.clas.ufl.edu/mac1105spring2020/courseDescriptionAndMisc/Exams/LearningFromResults}{More instructions on how to use this key can be found here}.}

\textbf{If you have a suggestion to make the keys better, \href{https://forms.gle/CZkbZmPbC9XALEE88}{please fill out the short survey here}.}

\textit{Note: This key is auto-generated and may contain issues and/or errors. The keys are reviewed after each exam to ensure grading is done accurately. If there are issues (like duplicate options), they are noted in the offline gradebook. The keys are a work-in-progress to give students as many resources to improve as possible.}

\rule{\textwidth}{0.4pt}

\begin{enumerate}\litem{
 Solve the equation for $x$ and choose the interval that contains $x$ (if it exists).
\[  24 = \sqrt[6]{\frac{22}{e^{7x}}} \]

The solution is \( x = -2.282, \text{ which does not fit in any of the interval options.} \), which is option E.\begin{enumerate}[label=\Alph*.]
\item \( x \in [-21.32, -20.97] \)

$x = -21.013$, which corresponds to thinking you don't need to take the natural log of both sides before reducing, as if the right side already has a natural log.
\item \( x \in [-0.81, -0.44] \)

$x = -0.466$, which corresponds to treating any root as a square root.
\item \( x \in [2.07, 2.75] \)

$x = 2.282$, which is the negative of the correct solution.
\item \( \text{There is no Real solution to the equation.} \)

This corresponds to believing you cannot solve the equation.
\item \( \text{None of the above.} \)

* $x = -2.282$ is the correct solution and does not fit in any of the other intervals.
\end{enumerate}

\textbf{General Comment:} \textbf{General Comments}: After using the properties of logarithmic functions to break up the right-hand side, use $\ln(e) = 1$ to reduce the question to a linear function to solve. You can put $\ln(22)$ into a calculator if you are having trouble.
}
\litem{
Solve the equation for $x$ and choose the interval that contains the solution (if it exists).
\[ 4^{5x-4} = \left(\frac{1}{25}\right)^{-3x+5} \]

The solution is \( x = 3.871 \), which is option A.\begin{enumerate}[label=\Alph*.]
\item \( x \in [2.7, 6.2] \)

* $x = 3.871$, which is the correct option.
\item \( x \in [-3, -1] \)

$x = -1.319$, which corresponds to distributing the $\ln(base)$ to the second term of the exponent only.
\item \( x \in [-0.6, 1.9] \)

$x = 1.125$, which corresponds to solving the numerators as equal while ignoring the bases are different.
\item \( x \in [-4.5, -1.6] \)

$x = -3.303$, which corresponds to distributing the $\ln(base)$ to the first term of the exponent only.
\item \( \text{There is no Real solution to the equation.} \)

This corresponds to believing there is no solution since the bases are not powers of each other.
\end{enumerate}

\textbf{General Comment:} \textbf{General Comments:} This question was written so that the bases could not be written the same. You will need to take the log of both sides.
}
\litem{
Solve the equation for $x$ and choose the interval that contains the solution (if it exists).
\[ 2^{-5x-4} = 25^{-4x+5} \]

The solution is \( x = 2.005 \), which is option A.\begin{enumerate}[label=\Alph*.]
\item \( x \in [1.8, 3] \)

* $x = 2.005$, which is the correct option.
\item \( x \in [-21.1, -17.6] \)

$x = -18.867$, which corresponds to distributing the $\ln(base)$ to the second term of the exponent only.
\item \( x \in [-9.8, -7.2] \)

$x = -9.000$, which corresponds to solving the numerators as equal while ignoring the bases are different.
\item \( x \in [0, 1.2] \)

$x = 0.956$, which corresponds to distributing the $\ln(base)$ to the first term of the exponent only.
\item \( \text{There is no Real solution to the equation.} \)

This corresponds to believing there is no solution since the bases are not powers of each other.
\end{enumerate}

\textbf{General Comment:} \textbf{General Comments:} This question was written so that the bases could not be written the same. You will need to take the log of both sides.
}
\litem{
Solve the equation for $x$ and choose the interval that contains the solution (if it exists).
\[ \log_{4}{(2x+8)}+6 = 3 \]

The solution is \( x = -3.992 \), which is option A.\begin{enumerate}[label=\Alph*.]
\item \( x \in [-8.99, -0.99] \)

* $x = -3.992$, which is the correct option.
\item \( x \in [35.5, 38.5] \)

$x = 36.500$, which corresponds to reversing the base and exponent when converting.
\item \( x \in [23, 30] \)

$x = 28.000$, which corresponds to ignoring the vertical shift when converting to exponential form.
\item \( x \in [39.5, 50.5] \)

$x = 44.500$, which corresponds to reversing the base and exponent when converting and reversing the value with $x$.
\item \( \text{There is no Real solution to the equation.} \)

Corresponds to believing a negative coefficient within the log equation means there is no Real solution.
\end{enumerate}

\textbf{General Comment:} \textbf{General Comments:} First, get the equation in the form $\log_b{(cx+d)} = a$. Then, convert to $b^a = cx+d$ and solve.
}
\litem{
Which of the following intervals describes the Range of the function below?
\[ f(x) = e^{x+2}+1 \]

The solution is \( (1, \infty) \), which is option B.\begin{enumerate}[label=\Alph*.]
\item \( (-\infty, a], a \in [-1.1, -0.6] \)

$(-\infty, -1]$, which corresponds to using the negative vertical shift AND flipping the Range interval AND including the endpoint.
\item \( (a, \infty), a \in [0.3, 2.4] \)

* $(1, \infty)$, which is the correct option.
\item \( (-\infty, a), a \in [-1.1, -0.6] \)

$(-\infty, -1)$, which corresponds to using the negative vertical shift AND flipping the Range interval.
\item \( [a, \infty), a \in [0.3, 2.4] \)

$[1, \infty)$, which corresponds to including the endpoint.
\item \( (-\infty, \infty) \)

This corresponds to confusing range of an exponential function with the domain of an exponential function.
\end{enumerate}

\textbf{General Comment:} \textbf{General Comments}: Domain of a basic exponential function is $(-\infty, \infty)$ while the Range is $(0, \infty)$. We can shift these intervals [and even flip when $a<0$!] to find the new Domain/Range.
}
\litem{
 Solve the equation for $x$ and choose the interval that contains $x$ (if it exists).
\[  14 = \sqrt[4]{\frac{10}{e^{5x}}} \]

The solution is \( x = -1.651 \), which is option B.\begin{enumerate}[label=\Alph*.]
\item \( x \in [-12.9, -11.3] \)

$x = -11.661$, which corresponds to thinking you don't need to take the natural log of both sides before reducing, as if the equation already had a natural log on the right side.
\item \( x \in [-1.9, -1.5] \)

* $x = -1.651$, which is the correct option.
\item \( x \in [-1.2, -0.2] \)

$x = -0.595$, which corresponds to treating any root as a square root.
\item \( \text{There is no Real solution to the equation.} \)

This corresponds to believing you cannot solve the equation.
\item \( \text{None of the above.} \)

This corresponds to making an unexpected error.
\end{enumerate}

\textbf{General Comment:} \textbf{General Comments}: After using the properties of logarithmic functions to break up the right-hand side, use $\ln(e) = 1$ to reduce the question to a linear function to solve. You can put $\ln(10)$ into a calculator if you are having trouble.
}
\litem{
Solve the equation for $x$ and choose the interval that contains the solution (if it exists).
\[ \log_{4}{(-3x+8)}+5 = 2 \]

The solution is \( x = 2.661 \), which is option B.\begin{enumerate}[label=\Alph*.]
\item \( x \in [-3.67, -0.67] \)

$x = -2.667$, which corresponds to ignoring the vertical shift when converting to exponential form.
\item \( x \in [1.66, 5.66] \)

* $x = 2.661$, which is the correct option.
\item \( x \in [-25.33, -21.33] \)

$x = -24.333$, which corresponds to reversing the base and exponent when converting.
\item \( x \in [-32.67, -24.67] \)

$x = -29.667$, which corresponds to reversing the base and exponent when converting and reversing the value with $x$.
\item \( \text{There is no Real solution to the equation.} \)

Corresponds to believing a negative coefficient within the log equation means there is no Real solution.
\end{enumerate}

\textbf{General Comment:} \textbf{General Comments:} First, get the equation in the form $\log_b{(cx+d)} = a$. Then, convert to $b^a = cx+d$ and solve.
}
\litem{
Which of the following intervals describes the Domain of the function below?
\[ f(x) = -\log_2{(x+5)}-4 \]

The solution is \( (-5, \infty) \), which is option C.\begin{enumerate}[label=\Alph*.]
\item \( (-\infty, a), a \in [4.53, 5.56] \)

$(-\infty, 5)$, which corresponds to flipping the Domain. Remember: the general for is $a*\log(x-h)+k$, \textbf{where $a$ does not affect the domain}.
\item \( (-\infty, a], a \in [3.58, 4.89] \)

$(-\infty, 4]$, which corresponds to using the negative vertical shift AND including the endpoint AND flipping the domain.
\item \( (a, \infty), a \in [-5.04, -4.64] \)

* $(-5, \infty)$, which is the correct option.
\item \( [a, \infty), a \in [-4.5, -3.84] \)

$[-4, \infty)$, which corresponds to using the vertical shift when shifting the Domain AND including the endpoint.
\item \( (-\infty, \infty) \)

This corresponds to thinking of the range of the log function (or the domain of the exponential function).
\end{enumerate}

\textbf{General Comment:} \textbf{General Comments}: The domain of a basic logarithmic function is $(0, \infty)$ and the Range is $(-\infty, \infty)$. We can use shifts when finding the Domain, but the Range will always be all Real numbers.
}
\litem{
Which of the following intervals describes the Domain of the function below?
\[ f(x) = e^{x-8}-6 \]

The solution is \( (-\infty, \infty) \), which is option E.\begin{enumerate}[label=\Alph*.]
\item \( (-\infty, a], a \in [-9, -5] \)

$(-\infty, -6]$, which corresponds to using the correct vertical shift *if we wanted the Range* AND including the endpoint.
\item \( (a, \infty), a \in [0, 13] \)

$(6, \infty)$, which corresponds to using the negative vertical shift AND flipping the Range interval.
\item \( [a, \infty), a \in [0, 13] \)

$[6, \infty)$, which corresponds to using the negative vertical shift AND flipping the Range interval AND including the endpoint.
\item \( (-\infty, a), a \in [-9, -5] \)

$(-\infty, -6)$, which corresponds to using the correct vertical shift *if we wanted the Range*.
\item \( (-\infty, \infty) \)

* This is the correct option.
\end{enumerate}

\textbf{General Comment:} \textbf{General Comments}: Domain of a basic exponential function is $(-\infty, \infty)$ while the Range is $(0, \infty)$. We can shift these intervals [and even flip when $a<0$!] to find the new Domain/Range.
}
\litem{
Which of the following intervals describes the Range of the function below?
\[ f(x) = -\log_2{(x+5)}-3 \]

The solution is \( (\infty, \infty) \), which is option E.\begin{enumerate}[label=\Alph*.]
\item \( [a, \infty), a \in [-7.6, -4.4] \)

$[-3, \infty)$, which corresponds to using the flipped Domain AND including the endpoint.
\item \( (-\infty, a), a \in [2.7, 3.3] \)

$(-\infty, 3)$, which corresponds to using the using the negative of vertical shift on $(0, \infty)$.
\item \( [a, \infty), a \in [4.7, 7.4] \)

$[5, \infty)$, which corresponds to using the negative of the horizontal shift AND including the endpoint.
\item \( (-\infty, a), a \in [-4.9, -0.8] \)

$(-\infty, -3)$, which corresponds to using the vertical shift while the Range is $(-\infty, \infty)$.
\item \( (-\infty, \infty) \)

*This is the correct option.
\end{enumerate}

\textbf{General Comment:} \textbf{General Comments}: The domain of a basic logarithmic function is $(0, \infty)$ and the Range is $(-\infty, \infty)$. We can use shifts when finding the Domain, but the Range will always be all Real numbers.
}
\end{enumerate}

\end{document}