\documentclass{extbook}[14pt]
\usepackage{multicol, enumerate, enumitem, hyperref, color, soul, setspace, parskip, fancyhdr, amssymb, amsthm, amsmath, bbm, latexsym, units, mathtools}
\everymath{\displaystyle}
\usepackage[headsep=0.5cm,headheight=0cm, left=1 in,right= 1 in,top= 1 in,bottom= 1 in]{geometry}
\usepackage{dashrule}  % Package to use the command below to create lines between items
\newcommand{\litem}[1]{\item #1

\rule{\textwidth}{0.4pt}}
\pagestyle{fancy}
\lhead{}
\chead{Answer Key for Progress Quiz 8 Version A}
\rhead{}
\lfoot{4553-3922}
\cfoot{}
\rfoot{Fall 2020}
\begin{document}
\textbf{This key should allow you to understand why you choose the option you did (beyond just getting a question right or wrong). \href{https://xronos.clas.ufl.edu/mac1105spring2020/courseDescriptionAndMisc/Exams/LearningFromResults}{More instructions on how to use this key can be found here}.}

\textbf{If you have a suggestion to make the keys better, \href{https://forms.gle/CZkbZmPbC9XALEE88}{please fill out the short survey here}.}

\textit{Note: This key is auto-generated and may contain issues and/or errors. The keys are reviewed after each exam to ensure grading is done accurately. If there are issues (like duplicate options), they are noted in the offline gradebook. The keys are a work-in-progress to give students as many resources to improve as possible.}

\rule{\textwidth}{0.4pt}

\begin{enumerate}\litem{
Which of the following intervals describes the Range of the function below?
\[ f(x) = -\log_2{(x+3)}-7 \]

The solution is \( (\infty, \infty) \), which is option E.\begin{enumerate}[label=\Alph*.]
\item \( [a, \infty), a \in [-6.3, -1.6] \)

$[-7, \infty)$, which corresponds to using the flipped Domain AND including the endpoint.
\item \( (-\infty, a), a \in [4.4, 8.8] \)

$(-\infty, 7)$, which corresponds to using the using the negative of vertical shift on $(0, \infty)$.
\item \( [a, \infty), a \in [1.9, 3.5] \)

$[3, \infty)$, which corresponds to using the negative of the horizontal shift AND including the endpoint.
\item \( (-\infty, a), a \in [-7.3, -6.7] \)

$(-\infty, -7)$, which corresponds to using the vertical shift while the Range is $(-\infty, \infty)$.
\item \( (-\infty, \infty) \)

*This is the correct option.
\end{enumerate}

\textbf{General Comment:} \textbf{General Comments}: The domain of a basic logarithmic function is $(0, \infty)$ and the Range is $(-\infty, \infty)$. We can use shifts when finding the Domain, but the Range will always be all Real numbers.
}
\litem{
Solve the equation for $x$ and choose the interval that contains the solution (if it exists).
\[ 4^{-3x+5} = \left(\frac{1}{27}\right)^{3x+2} \]

The solution is \( x = -2.361 \), which is option D.\begin{enumerate}[label=\Alph*.]
\item \( x \in [-1.7, -0.3] \)

$x = -0.524$, which corresponds to distributing the $\ln(base)$ to the first term of the exponent only.
\item \( x \in [-0.1, 1.7] \)

$x = 0.500$, which corresponds to solving the numerators as equal while ignoring the bases are different.
\item \( x \in [0.7, 3.1] \)

$x = 2.254$, which corresponds to distributing the $\ln(base)$ to the second term of the exponent only.
\item \( x \in [-4, -0.8] \)

* $x = -2.361$, which is the correct option.
\item \( \text{There is no Real solution to the equation.} \)

This corresponds to believing there is no solution since the bases are not powers of each other.
\end{enumerate}

\textbf{General Comment:} \textbf{General Comments:} This question was written so that the bases could not be written the same. You will need to take the log of both sides.
}
\litem{
 Solve the equation for $x$ and choose the interval that contains $x$ (if it exists).
\[  8 = \ln{\sqrt[5]{\frac{18}{e^{9x}}}} \]

The solution is \( x = -4.123, \text{ which does not fit in any of the interval options.} \), which is option E.\begin{enumerate}[label=\Alph*.]
\item \( x \in [4.08, 4.14] \)

$x = 4.123$, which is the negative of the correct solution.
\item \( x \in [-1.47, -1.45] \)

$x = -1.457$, which corresponds to treating any root as a square root.
\item \( x \in [-1.51, -1.46] \)

$x = -1.476$, which corresponds to thinking you need to take the natural log of the left side before reducing.
\item \( \text{There is no Real solution to the equation.} \)

This corresponds to believing you cannot solve the equation.
\item \( \text{None of the above.} \)

*$x = -4.123$ is the correct solution and does not fit in any of the other intervals.
\end{enumerate}

\textbf{General Comment:} \textbf{General Comments}: After using the properties of logarithmic functions to break up the right-hand side, use $\ln(e) = 1$ to reduce the question to a linear function to solve. You can put $\ln(18)$ into a calculator if you are having trouble.
}
\litem{
Solve the equation for $x$ and choose the interval that contains the solution (if it exists).
\[ 4^{4x-4} = \left(\frac{1}{25}\right)^{-2x+3} \]

The solution is \( x = 4.606 \), which is option C.\begin{enumerate}[label=\Alph*.]
\item \( x \in [-8.84, -1.84] \)

$x = -7.842$, which corresponds to distributing the $\ln(base)$ to the first term of the exponent only.
\item \( x \in [-1.69, 0.31] \)

$x = -0.685$, which corresponds to distributing the $\ln(base)$ to the second term of the exponent only.
\item \( x \in [3.61, 6.61] \)

* $x = 4.606$, which is the correct option.
\item \( x \in [0.17, 2.17] \)

$x = 1.167$, which corresponds to solving the numerators as equal while ignoring the bases are different.
\item \( \text{There is no Real solution to the equation.} \)

This corresponds to believing there is no solution since the bases are not powers of each other.
\end{enumerate}

\textbf{General Comment:} \textbf{General Comments:} This question was written so that the bases could not be written the same. You will need to take the log of both sides.
}
\litem{
Which of the following intervals describes the Range of the function below?
\[ f(x) = -\log_2{(x-5)}+7 \]

The solution is \( (\infty, \infty) \), which is option E.\begin{enumerate}[label=\Alph*.]
\item \( [a, \infty), a \in [4.61, 6.47] \)

$[7, \infty)$, which corresponds to using the flipped Domain AND including the endpoint.
\item \( [a, \infty), a \in [-5.47, -4.32] \)

$[-5, \infty)$, which corresponds to using the negative of the horizontal shift AND including the endpoint.
\item \( (-\infty, a), a \in [5.78, 8.19] \)

$(-\infty, 7)$, which corresponds to using the vertical shift while the Range is $(-\infty, \infty)$.
\item \( (-\infty, a), a \in [-7.35, -6.43] \)

$(-\infty, -7)$, which corresponds to using the using the negative of vertical shift on $(0, \infty)$.
\item \( (-\infty, \infty) \)

*This is the correct option.
\end{enumerate}

\textbf{General Comment:} \textbf{General Comments}: The domain of a basic logarithmic function is $(0, \infty)$ and the Range is $(-\infty, \infty)$. We can use shifts when finding the Domain, but the Range will always be all Real numbers.
}
\litem{
Which of the following intervals describes the Range of the function below?
\[ f(x) = e^{x+1}-7 \]

The solution is \( (-7, \infty) \), which is option A.\begin{enumerate}[label=\Alph*.]
\item \( (a, \infty), a \in [-7, -3] \)

* $(-7, \infty)$, which is the correct option.
\item \( [a, \infty), a \in [-7, -3] \)

$[-7, \infty)$, which corresponds to including the endpoint.
\item \( (-\infty, a], a \in [5, 8] \)

$(-\infty, 7]$, which corresponds to using the negative vertical shift AND flipping the Range interval AND including the endpoint.
\item \( (-\infty, a), a \in [5, 8] \)

$(-\infty, 7)$, which corresponds to using the negative vertical shift AND flipping the Range interval.
\item \( (-\infty, \infty) \)

This corresponds to confusing range of an exponential function with the domain of an exponential function.
\end{enumerate}

\textbf{General Comment:} \textbf{General Comments}: Domain of a basic exponential function is $(-\infty, \infty)$ while the Range is $(0, \infty)$. We can shift these intervals [and even flip when $a<0$!] to find the new Domain/Range.
}
\litem{
 Solve the equation for $x$ and choose the interval that contains $x$ (if it exists).
\[  24 = \sqrt[5]{\frac{7}{e^{3x}}} \]

The solution is \( x = -4.648, \text{ which does not fit in any of the interval options.} \), which is option E.\begin{enumerate}[label=\Alph*.]
\item \( x \in [-41.65, -38.65] \)

$x = -40.649$, which corresponds to thinking you don't need to take the natural log of both sides before reducing, as if the right side already has a natural log.
\item \( x \in [3.65, 7.65] \)

$x = 4.648$, which is the negative of the correct solution.
\item \( x \in [-3.47, -0.47] \)

$x = -1.470$, which corresponds to treating any root as a square root.
\item \( \text{There is no Real solution to the equation.} \)

This corresponds to believing you cannot solve the equation.
\item \( \text{None of the above.} \)

* $x = -4.648$ is the correct solution and does not fit in any of the other intervals.
\end{enumerate}

\textbf{General Comment:} \textbf{General Comments}: After using the properties of logarithmic functions to break up the right-hand side, use $\ln(e) = 1$ to reduce the question to a linear function to solve. You can put $\ln(7)$ into a calculator if you are having trouble.
}
\litem{
Solve the equation for $x$ and choose the interval that contains the solution (if it exists).
\[ \log_{5}{(4x+5)}+6 = 2 \]

The solution is \( x = -1.250 \), which is option A.\begin{enumerate}[label=\Alph*.]
\item \( x \in [-2, 2.8] \)

* $x = -1.250$, which is the correct option.
\item \( x \in [1.9, 5.5] \)

$x = 5.000$, which corresponds to ignoring the vertical shift when converting to exponential form.
\item \( x \in [-256, -252.8] \)

$x = -254.750$, which corresponds to reversing the base and exponent when converting and reversing the value with $x$.
\item \( x \in [-259.4, -255.3] \)

$x = -257.250$, which corresponds to reversing the base and exponent when converting.
\item \( \text{There is no Real solution to the equation.} \)

Corresponds to believing a negative coefficient within the log equation means there is no Real solution.
\end{enumerate}

\textbf{General Comment:} \textbf{General Comments:} First, get the equation in the form $\log_b{(cx+d)} = a$. Then, convert to $b^a = cx+d$ and solve.
}
\litem{
Solve the equation for $x$ and choose the interval that contains the solution (if it exists).
\[ \log_{4}{(-3x+7)}+4 = 2 \]

The solution is \( x = 2.312 \), which is option B.\begin{enumerate}[label=\Alph*.]
\item \( x \in [-8.7, -4.8] \)

$x = -7.667$, which corresponds to reversing the base and exponent when converting and reversing the value with $x$.
\item \( x \in [2.1, 4.1] \)

* $x = 2.312$, which is the correct option.
\item \( x \in [-5.7, -0.6] \)

$x = -3.000$, which corresponds to ignoring the vertical shift when converting to exponential form.
\item \( x \in [-5.7, -0.6] \)

$x = -3.000$, which corresponds to reversing the base and exponent when converting.
\item \( \text{There is no Real solution to the equation.} \)

Corresponds to believing a negative coefficient within the log equation means there is no Real solution.
\end{enumerate}

\textbf{General Comment:} \textbf{General Comments:} First, get the equation in the form $\log_b{(cx+d)} = a$. Then, convert to $b^a = cx+d$ and solve.
}
\litem{
Which of the following intervals describes the Range of the function below?
\[ f(x) = -e^{x-1}-5 \]

The solution is \( (-\infty, -5) \), which is option D.\begin{enumerate}[label=\Alph*.]
\item \( (a, \infty), a \in [5, 8] \)

$(5, \infty)$, which corresponds to using the negative vertical shift AND flipping the Range interval.
\item \( [a, \infty), a \in [5, 8] \)

$[5, \infty)$, which corresponds to using the negative vertical shift AND flipping the Range interval AND including the endpoint.
\item \( (-\infty, a], a \in [-5, 0] \)

$(-\infty, -5]$, which corresponds to including the endpoint.
\item \( (-\infty, a), a \in [-5, 0] \)

* $(-\infty, -5)$, which is the correct option.
\item \( (-\infty, \infty) \)

This corresponds to confusing range of an exponential function with the domain of an exponential function.
\end{enumerate}

\textbf{General Comment:} \textbf{General Comments}: Domain of a basic exponential function is $(-\infty, \infty)$ while the Range is $(0, \infty)$. We can shift these intervals [and even flip when $a<0$!] to find the new Domain/Range.
}
\end{enumerate}

\end{document}