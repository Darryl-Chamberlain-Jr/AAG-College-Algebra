\documentclass{extbook}[14pt]
\usepackage{multicol, enumerate, enumitem, hyperref, color, soul, setspace, parskip, fancyhdr, amssymb, amsthm, amsmath, latexsym, units, mathtools}
\everymath{\displaystyle}
\usepackage[headsep=0.5cm,headheight=0cm, left=1 in,right= 1 in,top= 1 in,bottom= 1 in]{geometry}
\usepackage{dashrule}  % Package to use the command below to create lines between items
\newcommand{\litem}[1]{\item #1

\rule{\textwidth}{0.4pt}}
\pagestyle{fancy}
\lhead{}
\chead{Answer Key for Progress Quiz 6 Version A}
\rhead{}
\lfoot{9689-6866}
\cfoot{}
\rfoot{Spring 2021}
\begin{document}
\textbf{This key should allow you to understand why you choose the option you did (beyond just getting a question right or wrong). \href{https://xronos.clas.ufl.edu/mac1105spring2020/courseDescriptionAndMisc/Exams/LearningFromResults}{More instructions on how to use this key can be found here}.}

\textbf{If you have a suggestion to make the keys better, \href{https://forms.gle/CZkbZmPbC9XALEE88}{please fill out the short survey here}.}

\textit{Note: This key is auto-generated and may contain issues and/or errors. The keys are reviewed after each exam to ensure grading is done accurately. If there are issues (like duplicate options), they are noted in the offline gradebook. The keys are a work-in-progress to give students as many resources to improve as possible.}

\rule{\textwidth}{0.4pt}

\begin{enumerate}\litem{
Perform the division below. Then, find the intervals that correspond to the quotient in the form $ax^2+bx+c$ and remainder $r$.
\[ \frac{8x^{3} -42 x^{2} +67 x -28}{x -2} \]The solution is \( 8x^{2} -26 x + 15 + \frac{2}{x -2} \), which is option E.\begin{enumerate}[label=\Alph*.]
\item \( a \in [14, 17], \text{   } b \in [-16, -7], \text{   } c \in [45, 51], \text{   and   } r \in [60, 69]. \)

 You multiplied by the synthetic number rather than bringing the first factor down.
\item \( a \in [8, 9], \text{   } b \in [-61, -57], \text{   } c \in [183, 189], \text{   and   } r \in [-398, -389]. \)

 You divided by the opposite of the factor.
\item \( a \in [8, 9], \text{   } b \in [-36, -31], \text{   } c \in [33, 34], \text{   and   } r \in [5, 9]. \)

 You multiplied by the synthetic number and subtracted rather than adding during synthetic division.
\item \( a \in [14, 17], \text{   } b \in [-74, -71], \text{   } c \in [213, 218], \text{   and   } r \in [-458, -455]. \)

 You divided by the opposite of the factor AND multiplied the first factor rather than just bringing it down.
\item \( a \in [8, 9], \text{   } b \in [-27, -25], \text{   } c \in [14, 18], \text{   and   } r \in [0, 4]. \)

* This is the solution!
\end{enumerate}

\textbf{General Comment:} Be sure to synthetically divide by the zero of the denominator!
}
\litem{
Factor the polynomial below completely. Then, choose the intervals the zeros of the polynomial belong to, where $z_1 \leq z_2 \leq z_3$. \textit{To make the problem easier, all zeros are between -5 and 5.}
\[ f(x) = 4x^{3} +20 x^{2} -9 x -45 \]The solution is \( [-5, -1.5, 1.5] \), which is option B.\begin{enumerate}[label=\Alph*.]
\item \( z_1 \in [-5.01, -5], \text{   }  z_2 \in [-1.39, -0.33], \text{   and   } z_3 \in [-0.8, 1.3] \)

 Distractor 2: Corresponds to inversing rational roots.
\item \( z_1 \in [-5.01, -5], \text{   }  z_2 \in [-2.11, -1.35], \text{   and   } z_3 \in [1.1, 2.7] \)

* This is the solution!
\item \( z_1 \in [-0.72, -0.66], \text{   }  z_2 \in [0.44, 0.85], \text{   and   } z_3 \in [4.4, 5.9] \)

 Distractor 3: Corresponds to negatives of all zeros AND inversing rational roots.
\item \( z_1 \in [-1.53, -1.44], \text{   }  z_2 \in [0.69, 1.69], \text{   and   } z_3 \in [4.4, 5.9] \)

 Distractor 1: Corresponds to negatives of all zeros.
\item \( z_1 \in [-0.78, -0.72], \text{   }  z_2 \in [2.59, 3.12], \text{   and   } z_3 \in [4.4, 5.9] \)

 Distractor 4: Corresponds to moving factors from one rational to another.
\end{enumerate}

\textbf{General Comment:} Remember to try the middle-most integers first as these normally are the zeros. Also, once you get it to a quadratic, you can use your other factoring techniques to finish factoring.
}
\litem{
Factor the polynomial below completely, knowing that $x+4$ is a factor. Then, choose the intervals the zeros of the polynomial belong to, where $z_1 \leq z_2 \leq z_3 \leq z_4$. \textit{To make the problem easier, all zeros are between -5 and 5.}
\[ f(x) = 4x^{4} -67 x^{2} +33 x + 180 \]The solution is \( [-4, -1.5, 2.5, 3] \), which is option E.\begin{enumerate}[label=\Alph*.]
\item \( z_1 \in [-3.23, -2.53], \text{   }  z_2 \in [-0.58, -0.22], z_3 \in [0.6, 0.72], \text{   and   } z_4 \in [3.4, 4.9] \)

 Distractor 3: Corresponds to negatives of all zeros AND inversing rational roots.
\item \( z_1 \in [-5.13, -4.59], \text{   }  z_2 \in [-3.44, -2.59], z_3 \in [0.7, 0.93], \text{   and   } z_4 \in [3.4, 4.9] \)

 Distractor 4: Corresponds to moving factors from one rational to another.
\item \( z_1 \in [-3.23, -2.53], \text{   }  z_2 \in [-2.86, -2.4], z_3 \in [1.26, 1.66], \text{   and   } z_4 \in [3.4, 4.9] \)

 Distractor 1: Corresponds to negatives of all zeros.
\item \( z_1 \in [-4.27, -3.21], \text{   }  z_2 \in [-0.7, -0.64], z_3 \in [0.27, 0.45], \text{   and   } z_4 \in [1.7, 3.8] \)

 Distractor 2: Corresponds to inversing rational roots.
\item \( z_1 \in [-4.27, -3.21], \text{   }  z_2 \in [-1.93, -1.36], z_3 \in [2.41, 2.88], \text{   and   } z_4 \in [1.7, 3.8] \)

* This is the solution!
\end{enumerate}

\textbf{General Comment:} Remember to try the middle-most integers first as these normally are the zeros. Also, once you get it to a quadratic, you can use your other factoring techniques to finish factoring.
}
\litem{
Factor the polynomial below completely, knowing that $x+4$ is a factor. Then, choose the intervals the zeros of the polynomial belong to, where $z_1 \leq z_2 \leq z_3 \leq z_4$. \textit{To make the problem easier, all zeros are between -5 and 5.}
\[ f(x) = 25x^{4} -10 x^{3} -408 x^{2} +160 x + 128 \]The solution is \( [-4, -0.4, 0.8, 4] \), which is option E.\begin{enumerate}[label=\Alph*.]
\item \( z_1 \in [-6, -3], \text{   }  z_2 \in [-1.28, -1.25], z_3 \in [2.4, 2.53], \text{   and   } z_4 \in [1, 6] \)

 Distractor 3: Corresponds to negatives of all zeros AND inversing rational roots.
\item \( z_1 \in [-6, -3], \text{   }  z_2 \in [-0.82, -0.42], z_3 \in [0.2, 0.42], \text{   and   } z_4 \in [1, 6] \)

 Distractor 1: Corresponds to negatives of all zeros.
\item \( z_1 \in [-6, -3], \text{   }  z_2 \in [-2.72, -2.19], z_3 \in [1.02, 1.34], \text{   and   } z_4 \in [1, 6] \)

 Distractor 2: Corresponds to inversing rational roots.
\item \( z_1 \in [-6, -3], \text{   }  z_2 \in [-4.12, -3.72], z_3 \in [-0.23, 0.18], \text{   and   } z_4 \in [1, 6] \)

 Distractor 4: Corresponds to moving factors from one rational to another.
\item \( z_1 \in [-6, -3], \text{   }  z_2 \in [-0.62, -0.31], z_3 \in [0.72, 0.97], \text{   and   } z_4 \in [1, 6] \)

* This is the solution!
\end{enumerate}

\textbf{General Comment:} Remember to try the middle-most integers first as these normally are the zeros. Also, once you get it to a quadratic, you can use your other factoring techniques to finish factoring.
}
\litem{
What are the \textit{possible Rational} roots of the polynomial below?
\[ f(x) = 2x^{4} +4 x^{3} +5 x^{2} +6 x + 5 \]The solution is \( \text{ All combinations of: }\frac{\pm 1,\pm 5}{\pm 1,\pm 2} \), which is option B.\begin{enumerate}[label=\Alph*.]
\item \( \pm 1,\pm 2 \)

 Distractor 1: Corresponds to the plus or minus factors of a1 only.
\item \( \text{ All combinations of: }\frac{\pm 1,\pm 5}{\pm 1,\pm 2} \)

* This is the solution \textbf{since we asked for the possible Rational roots}!
\item \( \pm 1,\pm 5 \)

This would have been the solution \textbf{if asked for the possible Integer roots}!
\item \( \text{ All combinations of: }\frac{\pm 1,\pm 2}{\pm 1,\pm 5} \)

 Distractor 3: Corresponds to the plus or minus of the inverse quotient (an/a0) of the factors. 
\item \( \text{ There is no formula or theorem that tells us all possible Rational roots.} \)

 Distractor 4: Corresponds to not recalling the theorem for rational roots of a polynomial.
\end{enumerate}

\textbf{General Comment:} We have a way to find the possible Rational roots. The possible Integer roots are the Integers in this list.
}
\litem{
Factor the polynomial below completely. Then, choose the intervals the zeros of the polynomial belong to, where $z_1 \leq z_2 \leq z_3$. \textit{To make the problem easier, all zeros are between -5 and 5.}
\[ f(x) = 15x^{3} +44 x^{2} -9 x -18 \]The solution is \( [-3, -0.6, 0.6666666666666666] \), which is option D.\begin{enumerate}[label=\Alph*.]
\item \( z_1 \in [-0.14, 0.33], \text{   }  z_2 \in [2.9, 3.23], \text{   and   } z_3 \in [2.7, 3.3] \)

 Distractor 4: Corresponds to moving factors from one rational to another.
\item \( z_1 \in [-3.59, -2.56], \text{   }  z_2 \in [-1.81, -1.37], \text{   and   } z_3 \in [1.2, 1.9] \)

 Distractor 2: Corresponds to inversing rational roots.
\item \( z_1 \in [-1, -0.26], \text{   }  z_2 \in [0.27, 1.1], \text{   and   } z_3 \in [2.7, 3.3] \)

 Distractor 1: Corresponds to negatives of all zeros.
\item \( z_1 \in [-3.59, -2.56], \text{   }  z_2 \in [-1, -0.32], \text{   and   } z_3 \in [-0.1, 1.4] \)

* This is the solution!
\item \( z_1 \in [-1.52, -1.01], \text{   }  z_2 \in [1.24, 2.19], \text{   and   } z_3 \in [2.7, 3.3] \)

 Distractor 3: Corresponds to negatives of all zeros AND inversing rational roots.
\end{enumerate}

\textbf{General Comment:} Remember to try the middle-most integers first as these normally are the zeros. Also, once you get it to a quadratic, you can use your other factoring techniques to finish factoring.
}
\litem{
Perform the division below. Then, find the intervals that correspond to the quotient in the form $ax^2+bx+c$ and remainder $r$.
\[ \frac{20x^{3} -85 x^{2} +5 x + 55}{x -4} \]The solution is \( 20x^{2} -5 x -15 + \frac{-5}{x -4} \), which is option C.\begin{enumerate}[label=\Alph*.]
\item \( a \in [72, 84], \text{   } b \in [-406, -403], \text{   } c \in [1622, 1626], \text{   and   } r \in [-6445, -6443]. \)

 You divided by the opposite of the factor AND multiplied the first factor rather than just bringing it down.
\item \( a \in [72, 84], \text{   } b \in [235, 241], \text{   } c \in [941, 951], \text{   and   } r \in [3835, 3840]. \)

 You multiplied by the synthetic number rather than bringing the first factor down.
\item \( a \in [13, 25], \text{   } b \in [-5, 2], \text{   } c \in [-17, -10], \text{   and   } r \in [-5, 1]. \)

* This is the solution!
\item \( a \in [13, 25], \text{   } b \in [-171, -164], \text{   } c \in [663, 670], \text{   and   } r \in [-2609, -2603]. \)

 You divided by the opposite of the factor.
\item \( a \in [13, 25], \text{   } b \in [-30, -24], \text{   } c \in [-74, -68], \text{   and   } r \in [-160, -148]. \)

 You multiplied by the synthetic number and subtracted rather than adding during synthetic division.
\end{enumerate}

\textbf{General Comment:} Be sure to synthetically divide by the zero of the denominator!
}
\litem{
What are the \textit{possible Integer} roots of the polynomial below?
\[ f(x) = 5x^{3} +2 x^{2} +4 x + 4 \]The solution is \( \pm 1,\pm 2,\pm 4 \), which is option A.\begin{enumerate}[label=\Alph*.]
\item \( \pm 1,\pm 2,\pm 4 \)

* This is the solution \textbf{since we asked for the possible Integer roots}!
\item \( \pm 1,\pm 5 \)

 Distractor 1: Corresponds to the plus or minus factors of a1 only.
\item \( \text{ All combinations of: }\frac{\pm 1,\pm 2,\pm 4}{\pm 1,\pm 5} \)

This would have been the solution \textbf{if asked for the possible Rational roots}!
\item \( \text{ All combinations of: }\frac{\pm 1,\pm 5}{\pm 1,\pm 2,\pm 4} \)

 Distractor 3: Corresponds to the plus or minus of the inverse quotient (an/a0) of the factors. 
\item \( \text{There is no formula or theorem that tells us all possible Integer roots.} \)

 Distractor 4: Corresponds to not recognizing Integers as a subset of Rationals.
\end{enumerate}

\textbf{General Comment:} We have a way to find the possible Rational roots. The possible Integer roots are the Integers in this list.
}
\litem{
Perform the division below. Then, find the intervals that correspond to the quotient in the form $ax^2+bx+c$ and remainder $r$.
\[ \frac{8x^{3} -42 x -18}{x + 2} \]The solution is \( 8x^{2} -16 x -10 + \frac{2}{x + 2} \), which is option C.\begin{enumerate}[label=\Alph*.]
\item \( a \in [-16, -14], b \in [29, 33], c \in [-107, -103], \text{ and } r \in [194, 199]. \)

 You multipled by the synthetic number rather than bringing the first factor down.
\item \( a \in [-16, -14], b \in [-36, -25], c \in [-107, -103], \text{ and } r \in [-231, -226]. \)

 You divided by the opposite of the factor AND multipled the first factor rather than just bringing it down.
\item \( a \in [8, 10], b \in [-21, -8], c \in [-13, -7], \text{ and } r \in [-4, 7]. \)

* This is the solution!
\item \( a \in [8, 10], b \in [15, 17], c \in [-13, -7], \text{ and } r \in [-42, -32]. \)

 You divided by the opposite of the factor.
\item \( a \in [8, 10], b \in [-25, -22], c \in [27, 32], \text{ and } r \in [-117, -105]. \)

 You multipled by the synthetic number and subtracted rather than adding during synthetic division.
\end{enumerate}

\textbf{General Comment:} Be sure to synthetically divide by the zero of the denominator! Also, make sure to include 0 placeholders for missing terms.
}
\litem{
Perform the division below. Then, find the intervals that correspond to the quotient in the form $ax^2+bx+c$ and remainder $r$.
\[ \frac{4x^{3} -28 x + 26}{x + 3} \]The solution is \( 4x^{2} -12 x + 8 + \frac{2}{x + 3} \), which is option D.\begin{enumerate}[label=\Alph*.]
\item \( a \in [-19, -11], b \in [-38.6, -34.1], c \in [-142, -128], \text{ and } r \in [-382, -378]. \)

 You divided by the opposite of the factor AND multipled the first factor rather than just bringing it down.
\item \( a \in [4, 10], b \in [-16.8, -13.7], c \in [34, 42], \text{ and } r \in [-123, -112]. \)

 You multipled by the synthetic number and subtracted rather than adding during synthetic division.
\item \( a \in [4, 10], b \in [9.1, 13.3], c \in [7, 9], \text{ and } r \in [46, 51]. \)

 You divided by the opposite of the factor.
\item \( a \in [4, 10], b \in [-13.5, -11.5], c \in [7, 9], \text{ and } r \in [2, 4]. \)

* This is the solution!
\item \( a \in [-19, -11], b \in [34.9, 36.3], c \in [-142, -128], \text{ and } r \in [434, 435]. \)

 You multipled by the synthetic number rather than bringing the first factor down.
\end{enumerate}

\textbf{General Comment:} Be sure to synthetically divide by the zero of the denominator! Also, make sure to include 0 placeholders for missing terms.
}
\end{enumerate}

\end{document}