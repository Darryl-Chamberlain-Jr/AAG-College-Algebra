\documentclass{extbook}[14pt]
\usepackage{multicol, enumerate, enumitem, hyperref, color, soul, setspace, parskip, fancyhdr, amssymb, amsthm, amsmath, latexsym, units, mathtools}
\everymath{\displaystyle}
\usepackage[headsep=0.5cm,headheight=0cm, left=1 in,right= 1 in,top= 1 in,bottom= 1 in]{geometry}
\usepackage{dashrule}  % Package to use the command below to create lines between items
\newcommand{\litem}[1]{\item #1

\rule{\textwidth}{0.4pt}}
\pagestyle{fancy}
\lhead{}
\chead{Answer Key for Progress Quiz 7 Version A}
\rhead{}
\lfoot{4173-5738}
\cfoot{}
\rfoot{Spring 2021}
\begin{document}
\textbf{This key should allow you to understand why you choose the option you did (beyond just getting a question right or wrong). \href{https://xronos.clas.ufl.edu/mac1105spring2020/courseDescriptionAndMisc/Exams/LearningFromResults}{More instructions on how to use this key can be found here}.}

\textbf{If you have a suggestion to make the keys better, \href{https://forms.gle/CZkbZmPbC9XALEE88}{please fill out the short survey here}.}

\textit{Note: This key is auto-generated and may contain issues and/or errors. The keys are reviewed after each exam to ensure grading is done accurately. If there are issues (like duplicate options), they are noted in the offline gradebook. The keys are a work-in-progress to give students as many resources to improve as possible.}

\rule{\textwidth}{0.4pt}

\begin{enumerate}\litem{
What are the \textit{possible Integer} roots of the polynomial below?
\[ f(x) = 3x^{3} +4 x^{2} +7 x + 5 \]The solution is \( \pm 1,\pm 5 \), which is option A.\begin{enumerate}[label=\Alph*.]
\item \( \pm 1,\pm 5 \)

* This is the solution \textbf{since we asked for the possible Integer roots}!
\item \( \pm 1,\pm 3 \)

 Distractor 1: Corresponds to the plus or minus factors of a1 only.
\item \( \text{ All combinations of: }\frac{\pm 1,\pm 3}{\pm 1,\pm 5} \)

 Distractor 3: Corresponds to the plus or minus of the inverse quotient (an/a0) of the factors. 
\item \( \text{ All combinations of: }\frac{\pm 1,\pm 5}{\pm 1,\pm 3} \)

This would have been the solution \textbf{if asked for the possible Rational roots}!
\item \( \text{There is no formula or theorem that tells us all possible Integer roots.} \)

 Distractor 4: Corresponds to not recognizing Integers as a subset of Rationals.
\end{enumerate}

\textbf{General Comment:} We have a way to find the possible Rational roots. The possible Integer roots are the Integers in this list.
}
\litem{
Perform the division below. Then, find the intervals that correspond to the quotient in the form $ax^2+bx+c$ and remainder $r$.
\[ \frac{16x^{3} -48 x -27}{x -2} \]The solution is \( 16x^{2} +32 x + 16 + \frac{5}{x -2} \), which is option E.\begin{enumerate}[label=\Alph*.]
\item \( a \in [30, 35], b \in [-72, -60], c \in [76, 81], \text{ and } r \in [-189, -180]. \)

 You divided by the opposite of the factor AND multipled the first factor rather than just bringing it down.
\item \( a \in [8, 23], b \in [-35, -31], c \in [11, 21], \text{ and } r \in [-61, -55]. \)

 You divided by the opposite of the factor.
\item \( a \in [8, 23], b \in [13, 20], c \in [-38, -31], \text{ and } r \in [-61, -55]. \)

 You multipled by the synthetic number and subtracted rather than adding during synthetic division.
\item \( a \in [30, 35], b \in [62, 69], c \in [76, 81], \text{ and } r \in [132, 139]. \)

 You multipled by the synthetic number rather than bringing the first factor down.
\item \( a \in [8, 23], b \in [30, 37], c \in [11, 21], \text{ and } r \in [4, 10]. \)

* This is the solution!
\end{enumerate}

\textbf{General Comment:} Be sure to synthetically divide by the zero of the denominator! Also, make sure to include 0 placeholders for missing terms.
}
\litem{
What are the \textit{possible Rational} roots of the polynomial below?
\[ f(x) = 3x^{2} +2 x + 6 \]The solution is \( \text{ All combinations of: }\frac{\pm 1,\pm 2,\pm 3,\pm 6}{\pm 1,\pm 3} \), which is option D.\begin{enumerate}[label=\Alph*.]
\item \( \text{ All combinations of: }\frac{\pm 1,\pm 3}{\pm 1,\pm 2,\pm 3,\pm 6} \)

 Distractor 3: Corresponds to the plus or minus of the inverse quotient (an/a0) of the factors. 
\item \( \pm 1,\pm 2,\pm 3,\pm 6 \)

This would have been the solution \textbf{if asked for the possible Integer roots}!
\item \( \pm 1,\pm 3 \)

 Distractor 1: Corresponds to the plus or minus factors of a1 only.
\item \( \text{ All combinations of: }\frac{\pm 1,\pm 2,\pm 3,\pm 6}{\pm 1,\pm 3} \)

* This is the solution \textbf{since we asked for the possible Rational roots}!
\item \( \text{ There is no formula or theorem that tells us all possible Rational roots.} \)

 Distractor 4: Corresponds to not recalling the theorem for rational roots of a polynomial.
\end{enumerate}

\textbf{General Comment:} We have a way to find the possible Rational roots. The possible Integer roots are the Integers in this list.
}
\litem{
Factor the polynomial below completely. Then, choose the intervals the zeros of the polynomial belong to, where $z_1 \leq z_2 \leq z_3$. \textit{To make the problem easier, all zeros are between -5 and 5.}
\[ f(x) = 10x^{3} -39 x^{2} +18 x + 27 \]The solution is \( [-0.6, 1.5, 3] \), which is option E.\begin{enumerate}[label=\Alph*.]
\item \( z_1 \in [-5, -2], \text{   }  z_2 \in [-0.73, -0.38], \text{   and   } z_3 \in [1.5, 1.9] \)

 Distractor 3: Corresponds to negatives of all zeros AND inversing rational roots.
\item \( z_1 \in [-2.67, -0.67], \text{   }  z_2 \in [0.59, 1.01], \text{   and   } z_3 \in [2.8, 3.6] \)

 Distractor 2: Corresponds to inversing rational roots.
\item \( z_1 \in [-5, -2], \text{   }  z_2 \in [-0.6, -0.03], \text{   and   } z_3 \in [2.8, 3.6] \)

 Distractor 4: Corresponds to moving factors from one rational to another.
\item \( z_1 \in [-5, -2], \text{   }  z_2 \in [-1.87, -0.88], \text{   and   } z_3 \in [-0.4, 0.8] \)

 Distractor 1: Corresponds to negatives of all zeros.
\item \( z_1 \in [-0.6, 3.4], \text{   }  z_2 \in [1.3, 2], \text{   and   } z_3 \in [2.8, 3.6] \)

* This is the solution!
\end{enumerate}

\textbf{General Comment:} Remember to try the middle-most integers first as these normally are the zeros. Also, once you get it to a quadratic, you can use your other factoring techniques to finish factoring.
}
\litem{
Factor the polynomial below completely, knowing that $x+2$ is a factor. Then, choose the intervals the zeros of the polynomial belong to, where $z_1 \leq z_2 \leq z_3 \leq z_4$. \textit{To make the problem easier, all zeros are between -5 and 5.}
\[ f(x) = 20x^{4} -13 x^{3} -95 x^{2} +52 x + 60 \]The solution is \( [-2, -0.6, 1.25, 2] \), which is option C.\begin{enumerate}[label=\Alph*.]
\item \( z_1 \in [-3, 1], \text{   }  z_2 \in [-1.92, -1.65], z_3 \in [0.77, 0.97], \text{   and   } z_4 \in [1.36, 2.38] \)

 Distractor 2: Corresponds to inversing rational roots.
\item \( z_1 \in [-3, 1], \text{   }  z_2 \in [-0.88, -0.66], z_3 \in [1.61, 1.84], \text{   and   } z_4 \in [1.36, 2.38] \)

 Distractor 3: Corresponds to negatives of all zeros AND inversing rational roots.
\item \( z_1 \in [-3, 1], \text{   }  z_2 \in [-0.76, -0.57], z_3 \in [1.23, 1.3], \text{   and   } z_4 \in [1.36, 2.38] \)

* This is the solution!
\item \( z_1 \in [-3, 1], \text{   }  z_2 \in [-0.51, 0.03], z_3 \in [1.9, 2.15], \text{   and   } z_4 \in [2.48, 3.24] \)

 Distractor 4: Corresponds to moving factors from one rational to another.
\item \( z_1 \in [-3, 1], \text{   }  z_2 \in [-1.52, -1.08], z_3 \in [0.55, 0.78], \text{   and   } z_4 \in [1.36, 2.38] \)

 Distractor 1: Corresponds to negatives of all zeros.
\end{enumerate}

\textbf{General Comment:} Remember to try the middle-most integers first as these normally are the zeros. Also, once you get it to a quadratic, you can use your other factoring techniques to finish factoring.
}
\litem{
Factor the polynomial below completely. Then, choose the intervals the zeros of the polynomial belong to, where $z_1 \leq z_2 \leq z_3$. \textit{To make the problem easier, all zeros are between -5 and 5.}
\[ f(x) = 6x^{3} -1 x^{2} -20 x + 12 \]The solution is \( [-2, 0.6666666666666666, 1.5] \), which is option A.\begin{enumerate}[label=\Alph*.]
\item \( z_1 \in [-2.57, -1.94], \text{   }  z_2 \in [0.56, 0.71], \text{   and   } z_3 \in [0.6, 1.7] \)

* This is the solution!
\item \( z_1 \in [-1.52, -1.02], \text{   }  z_2 \in [-0.93, -0.52], \text{   and   } z_3 \in [1.9, 2.4] \)

 Distractor 1: Corresponds to negatives of all zeros.
\item \( z_1 \in [-3.42, -2.64], \text{   }  z_2 \in [-0.49, -0.3], \text{   and   } z_3 \in [1.9, 2.4] \)

 Distractor 4: Corresponds to moving factors from one rational to another.
\item \( z_1 \in [-1.52, -1.02], \text{   }  z_2 \in [-0.93, -0.52], \text{   and   } z_3 \in [1.9, 2.4] \)

 Distractor 3: Corresponds to negatives of all zeros AND inversing rational roots.
\item \( z_1 \in [-2.57, -1.94], \text{   }  z_2 \in [0.56, 0.71], \text{   and   } z_3 \in [0.6, 1.7] \)

 Distractor 2: Corresponds to inversing rational roots.
\end{enumerate}

\textbf{General Comment:} Remember to try the middle-most integers first as these normally are the zeros. Also, once you get it to a quadratic, you can use your other factoring techniques to finish factoring.
}
\litem{
Factor the polynomial below completely, knowing that $x+3$ is a factor. Then, choose the intervals the zeros of the polynomial belong to, where $z_1 \leq z_2 \leq z_3 \leq z_4$. \textit{To make the problem easier, all zeros are between -5 and 5.}
\[ f(x) = 15x^{4} +91 x^{3} +5 x^{2} -339 x + 180 \]The solution is \( [-5, -3, 0.6, 1.3333333333333333] \), which is option E.\begin{enumerate}[label=\Alph*.]
\item \( z_1 \in [-5.54, -4.92], \text{   }  z_2 \in [-3.08, -2.8], z_3 \in [0.68, 1.11], \text{   and   } z_4 \in [1.64, 2.04] \)

 Distractor 2: Corresponds to inversing rational roots.
\item \( z_1 \in [-2.18, -1.46], \text{   }  z_2 \in [-1.31, -0.71], z_3 \in [2.89, 3.23], \text{   and   } z_4 \in [4.76, 5.58] \)

 Distractor 3: Corresponds to negatives of all zeros AND inversing rational roots.
\item \( z_1 \in [-1.53, -0.97], \text{   }  z_2 \in [-0.73, -0.5], z_3 \in [2.89, 3.23], \text{   and   } z_4 \in [4.76, 5.58] \)

 Distractor 1: Corresponds to negatives of all zeros.
\item \( z_1 \in [-4.14, -3.6], \text{   }  z_2 \in [-0.46, -0.15], z_3 \in [2.89, 3.23], \text{   and   } z_4 \in [4.76, 5.58] \)

 Distractor 4: Corresponds to moving factors from one rational to another.
\item \( z_1 \in [-5.54, -4.92], \text{   }  z_2 \in [-3.08, -2.8], z_3 \in [0.43, 0.7], \text{   and   } z_4 \in [1.28, 1.53] \)

* This is the solution!
\end{enumerate}

\textbf{General Comment:} Remember to try the middle-most integers first as these normally are the zeros. Also, once you get it to a quadratic, you can use your other factoring techniques to finish factoring.
}
\litem{
Perform the division below. Then, find the intervals that correspond to the quotient in the form $ax^2+bx+c$ and remainder $r$.
\[ \frac{20x^{3} +20 x^{2} -100 x + 63}{x + 3} \]The solution is \( 20x^{2} -40 x + 20 + \frac{3}{x + 3} \), which is option C.\begin{enumerate}[label=\Alph*.]
\item \( a \in [-63, -57], \text{   } b \in [-163, -156], \text{   } c \in [-581, -579], \text{   and   } r \in [-1678, -1673]. \)

 You divided by the opposite of the factor AND multiplied the first factor rather than just bringing it down.
\item \( a \in [20, 25], \text{   } b \in [-63, -59], \text{   } c \in [139, 145], \text{   and   } r \in [-504, -495]. \)

 You multiplied by the synthetic number and subtracted rather than adding during synthetic division.
\item \( a \in [20, 25], \text{   } b \in [-43, -35], \text{   } c \in [19, 27], \text{   and   } r \in [-1, 7]. \)

* This is the solution!
\item \( a \in [-63, -57], \text{   } b \in [199, 206], \text{   } c \in [-701, -696], \text{   and   } r \in [2162, 2167]. \)

 You multiplied by the synthetic number rather than bringing the first factor down.
\item \( a \in [20, 25], \text{   } b \in [76, 82], \text{   } c \in [139, 145], \text{   and   } r \in [478, 484]. \)

 You divided by the opposite of the factor.
\end{enumerate}

\textbf{General Comment:} Be sure to synthetically divide by the zero of the denominator!
}
\litem{
Perform the division below. Then, find the intervals that correspond to the quotient in the form $ax^2+bx+c$ and remainder $r$.
\[ \frac{25x^{3} +105 x^{2} -83}{x + 4} \]The solution is \( 25x^{2} +5 x -20 + \frac{-3}{x + 4} \), which is option D.\begin{enumerate}[label=\Alph*.]
\item \( a \in [-101, -97], b \in [-296, -291], c \in [-1183, -1175], \text{ and } r \in [-4809, -4802]. \)

 You divided by the opposite of the factor AND multipled the first factor rather than just bringing it down.
\item \( a \in [20, 26], b \in [-23, -18], c \in [91, 105], \text{ and } r \in [-583, -581]. \)

 You multipled by the synthetic number and subtracted rather than adding during synthetic division.
\item \( a \in [20, 26], b \in [201, 211], c \in [818, 824], \text{ and } r \in [3193, 3205]. \)

 You divided by the opposite of the factor.
\item \( a \in [20, 26], b \in [-1, 8], c \in [-20, -15], \text{ and } r \in [-10, -1]. \)

* This is the solution!
\item \( a \in [-101, -97], b \in [500, 508], c \in [-2025, -2019], \text{ and } r \in [7995, 8006]. \)

 You multipled by the synthetic number rather than bringing the first factor down.
\end{enumerate}

\textbf{General Comment:} Be sure to synthetically divide by the zero of the denominator! Also, make sure to include 0 placeholders for missing terms.
}
\litem{
Perform the division below. Then, find the intervals that correspond to the quotient in the form $ax^2+bx+c$ and remainder $r$.
\[ \frac{8x^{3} +22 x^{2} -80 x + 47}{x + 5} \]The solution is \( 8x^{2} -18 x + 10 + \frac{-3}{x + 5} \), which is option D.\begin{enumerate}[label=\Alph*.]
\item \( a \in [3, 13], \text{   } b \in [61, 63], \text{   } c \in [229, 232], \text{   and   } r \in [1191, 1201]. \)

 You divided by the opposite of the factor.
\item \( a \in [3, 13], \text{   } b \in [-31, -23], \text{   } c \in [74, 80], \text{   and   } r \in [-410, -405]. \)

 You multiplied by the synthetic number and subtracted rather than adding during synthetic division.
\item \( a \in [-41, -34], \text{   } b \in [216, 223], \text{   } c \in [-1191, -1187], \text{   and   } r \in [5991, 5998]. \)

 You multiplied by the synthetic number rather than bringing the first factor down.
\item \( a \in [3, 13], \text{   } b \in [-23, -16], \text{   } c \in [10, 13], \text{   and   } r \in [-5, 0]. \)

* This is the solution!
\item \( a \in [-41, -34], \text{   } b \in [-182, -177], \text{   } c \in [-971, -963], \text{   and   } r \in [-4807, -4797]. \)

 You divided by the opposite of the factor AND multiplied the first factor rather than just bringing it down.
\end{enumerate}

\textbf{General Comment:} Be sure to synthetically divide by the zero of the denominator!
}
\end{enumerate}

\end{document}