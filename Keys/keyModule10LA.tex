\documentclass{extbook}[14pt]
\usepackage{multicol, enumerate, enumitem, hyperref, color, soul, setspace, parskip, fancyhdr, amssymb, amsthm, amsmath, bbm, latexsym, units, mathtools}
\everymath{\displaystyle}
\usepackage[headsep=0.5cm,headheight=0cm, left=1 in,right= 1 in,top= 1 in,bottom= 1 in]{geometry}
\pagestyle{fancy}
\lhead{}
\chead{Answer Key for Module\,10L\,-\,Synthetic\,Division Version A}
\rhead{}
\lfoot{Summer\,C\,2020}
\cfoot{}
\rfoot{}
\begin{document}
\textbf{This key should allow you to understand why you choose the option you did (beyond just getting a question right or wrong). \href{https://xronos.clas.ufl.edu/mac1105spring2020/courseDescriptionAndMisc/Exams/LearningFromResults}{More instructions on how to use this key can be found here}.}

\textbf{If you have a suggestion to make the keys better, \href{https://forms.gle/CZkbZmPbC9XALEE88}{please fill out the short survey here}.}

\textit{Note: This key is auto-generated and may contain issues and/or errors. The keys are reviewed after each exam to ensure grading is done accurately. If there are issues (like duplicate options), they are noted in the offline gradebook. The keys are a work-in-progress to give students as many resources to improve as possible.}

\rule{\textwidth}{0.4pt}

1. Perform the division below. Then, find the intervals that correspond to the quotient in the form $ax^2+bx+c$ and remainder $r$.
\[ \frac{9x^{3} -28 x -13}{x -2} \] 
The solution is $ 9x^{2} +18 x + 8 + \frac{3}{x -2} $ 

\begin{enumerate}[label=\Alph*.] 
\item $ a \in [14, 20], b \in [29, 40], c \in [36, 45], \text{ and } r \in [71, 77]. $ 

  You multipled by the synthetic number rather than bringing the first factor down. 
\item $ a \in [2, 10], b \in [6, 13], c \in [-22, -14], \text{ and } r \in [-37, -30]. $ 

  You multipled by the synthetic number and subtracted rather than adding during synthetic division. 
\item $ a \in [14, 20], b \in [-42, -35], c \in [36, 45], \text{ and } r \in [-104, -95]. $ 

  You divided by the opposite of the factor AND multipled the first factor rather than just bringing it down. 
\item $ a \in [2, 10], b \in [15, 20], c \in [6, 10], \text{ and } r \in [2, 5]. $ 

 * This is the solution! 
\item $ a \in [2, 10], b \in [-22, -16], c \in [6, 10], \text{ and } r \in [-31, -23]. $ 

  You divided by the opposite of the factor. 
\end{enumerate} 
 
\textbf{General Comment:} General Comments: Be sure to synthetically divide by the zero of the denominator! Also, make sure to include 0 placeholders for missing terms. 

-----------------------------------------------

2. Perform the division below. Then, find the intervals that correspond to the quotient in the form $ax^2+bx+c$ and remainder $r$.
\[ \frac{6x^{3} +9 x^{2} -51 x + 34}{x + 4} \] 
The solution is $ 6x^{2} -15 x + 9 + \frac{-2}{x + 4} $ 

\begin{enumerate}[label=\Alph*.] 
\item $ a \in [-2, 9], \text{   } b \in [-24, -18], \text{   } c \in [48, 56], \text{   and   } r \in [-238, -229]. $ 

  You multiplied by the synthetic number and subtracted rather than adding during synthetic division. 
\item $ a \in [-2, 9], \text{   } b \in [-17, -13], \text{   } c \in [3, 12], \text{   and   } r \in [-5, 3]. $ 

 * This is the solution! 
\item $ a \in [-30, -21], \text{   } b \in [-90, -86], \text{   } c \in [-402, -396], \text{   and   } r \in [-1563, -1558]. $ 

  You divided by the opposite of the factor AND multiplied the first factor rather than just bringing it down. 
\item $ a \in [-2, 9], \text{   } b \in [28, 36], \text{   } c \in [76, 85], \text{   and   } r \in [354, 359]. $ 

  You divided by the opposite of the factor. 
\item $ a \in [-30, -21], \text{   } b \in [104, 113], \text{   } c \in [-472, -465], \text{   and   } r \in [1913, 1924]. $ 

  You multiplied by the synthetic number rather than bringing the first factor down. 
\end{enumerate} 
 
\textbf{General Comment:} General Comments: Be sure to synthetically divide by the zero of the denominator! 

-----------------------------------------------

3. What are the \textit{possible Integer} roots of the polynomial below?
\[ f(x) = 2x^{3} +3 x^{2} +7 x + 3 \] 
The solution is $ \pm 1,\pm 3 $ 

\begin{enumerate}[label=\Alph*.] 
\item $ \pm 1,\pm 3 $ 

 * This is the solution \textbf{since we asked for the possible Integer roots}! 
\item $ \text{ All combinations of: }\frac{\pm 1,\pm 3}{\pm 1,\pm 2} $ 

 This would have been the solution \textbf{if asked for the possible Rational roots}! 
\item $ \text{ All combinations of: }\frac{\pm 1,\pm 2}{\pm 1,\pm 3} $ 

  Distractor 3: Corresponds to the plus or minus of the inverse quotient (an/a0) of the factors.  
\item $ \pm 1,\pm 2 $ 

  Distractor 1: Corresponds to the plus or minus factors of a1 only. 
\item $ \text{There is no formula or theorem that tells us all possible Integer roots.} $ 

  Distractor 4: Corresponds to not recognizing Integers as a subset of Rationals. 
\end{enumerate} 
 
\textbf{General Comment:} General Comments: We have a way to find the possible Rational roots. The possible Integer roots are the Integers in this list. 

-----------------------------------------------

4. Factor the polynomial below completely, knowing that $x+5$ is a factor. Then, choose the intervals the zeros of the polynomial belong to, where $z_1 \leq z_2 \leq z_3 \leq z_4$. \textit{To make the problem easier, all zeros are between -5 and 5.}
\[ f(x) = 6x^{4} +55 x^{3} +117 x^{2} -88 x -240 \] 
The solution is $ [-5, -4, -1.5, 1.3333333333333333] $ 

\begin{enumerate}[label=\Alph*.] 
\item $ z_1 \in [-5.54, -4.65], \text{   }  z_2 \in [-4.26, -3.9], z_3 \in [-2.14, -0.83], \text{   and   } z_4 \in [1.27, 1.9] $ 

 * This is the solution! 
\item $ z_1 \in [-1.37, -1.22], \text{   }  z_2 \in [1.32, 1.6], z_3 \in [3.23, 4.14], \text{   and   } z_4 \in [4.12, 5.15] $ 

  Distractor 1: Corresponds to negatives of all zeros. 
\item $ z_1 \in [-1.05, -0.46], \text{   }  z_2 \in [0.66, 0.73], z_3 \in [3.23, 4.14], \text{   and   } z_4 \in [4.12, 5.15] $ 

  Distractor 3: Corresponds to negatives of all zeros AND inversing rational roots. 
\item $ z_1 \in [-4.96, -3.52], \text{   }  z_2 \in [0.44, 0.51], z_3 \in [3.23, 4.14], \text{   and   } z_4 \in [4.12, 5.15] $ 

  Distractor 4: Corresponds to moving factors from one rational to another. 
\item $ z_1 \in [-5.54, -4.65], \text{   }  z_2 \in [-4.26, -3.9], z_3 \in [-1.49, -0.16], \text{   and   } z_4 \in [-0.12, 1.25] $ 

  Distractor 2: Corresponds to inversing rational roots. 
\end{enumerate} 
 
\textbf{General Comment:} General Comments: Remember to try the middle-most integers first as these normally are the zeros. Also, once you get it to a quadratic, you can use your other factoring techniques to finish factoring. 

-----------------------------------------------

0. Factor the polynomial below completely. Then, choose the intervals the zeros of the polynomial belong to, where $z_1 \leq z_2 \leq z_3$. \textit{To make the problem easier, all zeros are between -5 and 5.}
\[ f(x) = 4x^{3} -4 x^{2} -33 x + 45 \] 
The solution is $ [-3, 1.5, 2.5] $ 

\begin{enumerate}[label=\Alph*.] 
\item $ z_1 \in [-2.7, -2.1], \text{   }  z_2 \in [-2.08, -0.87], \text{   and   } z_3 \in [2.7, 3.47] $ 

  Distractor 1: Corresponds to negatives of all zeros. 
\item $ z_1 \in [-3.8, -2.9], \text{   }  z_2 \in [0.15, 0.55], \text{   and   } z_3 \in [0.49, 0.74] $ 

  Distractor 2: Corresponds to inversing rational roots. 
\item $ z_1 \in [-1.8, 0.2], \text{   }  z_2 \in [-0.49, -0.08], \text{   and   } z_3 \in [2.7, 3.47] $ 

  Distractor 3: Corresponds to negatives of all zeros AND inversing rational roots. 
\item $ z_1 \in [-3.8, -2.9], \text{   }  z_2 \in [1.25, 1.89], \text{   and   } z_3 \in [1.76, 2.8] $ 

 * This is the solution! 
\item $ z_1 \in [-5.1, -4.8], \text{   }  z_2 \in [-1.25, -0.55], \text{   and   } z_3 \in [2.7, 3.47] $ 

  Distractor 4: Corresponds to moving factors from one rational to another. 
\end{enumerate} 
 
\textbf{General Comment:} General Comments: Remember to try the middle-most integers first as these normally are the zeros. Also, once you get it to a quadratic, you can use your other factoring techniques to finish factoring. 

-----------------------------------------------


\end{document}

