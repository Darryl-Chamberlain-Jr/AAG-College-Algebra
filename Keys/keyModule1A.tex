\documentclass{extbook}[14pt]
\usepackage{multicol, enumerate, enumitem, hyperref, color, soul, setspace, parskip, fancyhdr, amssymb, amsthm, amsmath, bbm, latexsym, units, mathtools}
\everymath{\displaystyle}
\usepackage[headsep=0.5cm,headheight=0cm, left=1 in,right= 1 in,top= 1 in,bottom= 1 in]{geometry}
\usepackage{dashrule}  % Package to use the command below to create lines between items
\newcommand{\litem}[1]{\item #1

\rule{\textwidth}{0.4pt}}
\pagestyle{fancy}
\lhead{}
\chead{Answer Key for Progress Quiz 4 Version A}
\rhead{}
\lfoot{8448-1521}
\cfoot{}
\rfoot{Fall 2020}
\begin{document}
\textbf{This key should allow you to understand why you choose the option you did (beyond just getting a question right or wrong). \href{https://xronos.clas.ufl.edu/mac1105spring2020/courseDescriptionAndMisc/Exams/LearningFromResults}{More instructions on how to use this key can be found here}.}

\textbf{If you have a suggestion to make the keys better, \href{https://forms.gle/CZkbZmPbC9XALEE88}{please fill out the short survey here}.}

\textit{Note: This key is auto-generated and may contain issues and/or errors. The keys are reviewed after each exam to ensure grading is done accurately. If there are issues (like duplicate options), they are noted in the offline gradebook. The keys are a work-in-progress to give students as many resources to improve as possible.}

\rule{\textwidth}{0.4pt}

\begin{enumerate}\litem{
Choose the \textbf{smallest} set of Real numbers that the number below belongs to.
\[ -\sqrt{\frac{186624}{324}} \]
The solution is \( \text{Integer} \), which is option D.\begin{enumerate}[label=\Alph*.]
\item \( \text{Whole} \)

These are the counting numbers with 0 (0, 1, 2, 3, ...)
\item \( \text{Irrational} \)

These cannot be written as a fraction of Integers.
\item \( \text{Not a Real number} \)

These are Nonreal Complex numbers \textbf{OR} things that are not numbers (e.g., dividing by 0).
\item \( \text{Integer} \)

* This is the correct option!
\item \( \text{Rational} \)

These are numbers that can be written as fraction of Integers (e.g., -2/3)
\end{enumerate}

\textbf{General Comment:} First, you \textbf{NEED} to simplify the expression. This question simplifies to $-432$. 
 
 Be sure you look at the simplified fraction and not just the decimal expansion. Numbers such as 13, 17, and 19 provide \textbf{long but repeating/terminating decimal expansions!} 
 
 The only ways to *not* be a Real number are: dividing by 0 or taking the square root of a negative number. 
 
 Irrational numbers are more than just square root of 3: adding or subtracting values from square root of 3 is also irrational.
}
\litem{
Simplify the expression below into the form $a+bi$. Then, choose the intervals that $a$ and $b$ belong to.
\[ (2 + 10 i)(-6 + 3 i) \]
The solution is \( -42 - 54 i \), which is option E.\begin{enumerate}[label=\Alph*.]
\item \( a \in [17, 22] \text{ and } b \in [65, 68] \)

 $18 + 66 i$, which corresponds to adding a minus sign in the first term.
\item \( a \in [-12, -7] \text{ and } b \in [28, 31] \)

 $-12 + 30 i$, which corresponds to just multiplying the real terms to get the real part of the solution and the coefficients in the complex terms to get the complex part.
\item \( a \in [17, 22] \text{ and } b \in [-67, -65] \)

 $18 - 66 i$, which corresponds to adding a minus sign in the second term.
\item \( a \in [-43, -41] \text{ and } b \in [53, 60] \)

 $-42 + 54 i$, which corresponds to adding a minus sign in both terms.
\item \( a \in [-43, -41] \text{ and } b \in [-54, -51] \)

* $-42 - 54 i$, which is the correct option.
\end{enumerate}

\textbf{General Comment:} You can treat $i$ as a variable and distribute. Just remember that $i^2=-1$, so you can continue to reduce after you distribute.
}
\litem{
Simplify the expression below and choose the interval the simplification is contained within.
\[ 3 - 12 \div 6 * 7 - (1 * 8) \]
The solution is \( -19.000 \), which is option A.\begin{enumerate}[label=\Alph*.]
\item \( [-22, -18] \)

* -19.000, which is the correct option.
\item \( [-8.29, 0.71] \)

 -5.286, which corresponds to an Order of Operations error: not reading left-to-right for multiplication/division.
\item \( [-97, -94] \)

 -96.000, which corresponds to not distributing a negative correctly.
\item \( [10.71, 12.71] \)

 10.714, which corresponds to not distributing addition and subtraction correctly.
\item \( \text{None of the above} \)

 You may have gotten this by making an unanticipated error. If you got a value that is not any of the others, please let the coordinator know so they can help you figure out what happened.
\end{enumerate}

\textbf{General Comment:} While you may remember (or were taught) PEMDAS is done in order, it is actually done as P/E/MD/AS. When we are at MD or AS, we read left to right.
}
\litem{
Choose the \textbf{smallest} set of Real numbers that the number below belongs to.
\[ -\sqrt{\frac{-2431}{13}} \]
The solution is \( \text{Not a Real number} \), which is option E.\begin{enumerate}[label=\Alph*.]
\item \( \text{Rational} \)

These are numbers that can be written as fraction of Integers (e.g., -2/3)
\item \( \text{Irrational} \)

These cannot be written as a fraction of Integers.
\item \( \text{Whole} \)

These are the counting numbers with 0 (0, 1, 2, 3, ...)
\item \( \text{Integer} \)

These are the negative and positive counting numbers (..., -3, -2, -1, 0, 1, 2, 3, ...)
\item \( \text{Not a Real number} \)

* This is the correct option!
\end{enumerate}

\textbf{General Comment:} First, you \textbf{NEED} to simplify the expression. This question simplifies to $-\sqrt{187} i$. 
 
 Be sure you look at the simplified fraction and not just the decimal expansion. Numbers such as 13, 17, and 19 provide \textbf{long but repeating/terminating decimal expansions!} 
 
 The only ways to *not* be a Real number are: dividing by 0 or taking the square root of a negative number. 
 
 Irrational numbers are more than just square root of 3: adding or subtracting values from square root of 3 is also irrational.
}
\litem{
Simplify the expression below into the form $a+bi$. Then, choose the intervals that $a$ and $b$ belong to.
\[ (-9 + 3 i)(-7 - 10 i) \]
The solution is \( 93 + 69 i \), which is option A.\begin{enumerate}[label=\Alph*.]
\item \( a \in [92, 98] \text{ and } b \in [67, 71] \)

* $93 + 69 i$, which is the correct option.
\item \( a \in [28, 39] \text{ and } b \in [-116, -108] \)

 $33 - 111 i$, which corresponds to adding a minus sign in the second term.
\item \( a \in [28, 39] \text{ and } b \in [108, 116] \)

 $33 + 111 i$, which corresponds to adding a minus sign in the first term.
\item \( a \in [55, 65] \text{ and } b \in [-31, -28] \)

 $63 - 30 i$, which corresponds to just multiplying the real terms to get the real part of the solution and the coefficients in the complex terms to get the complex part.
\item \( a \in [92, 98] \text{ and } b \in [-73, -65] \)

 $93 - 69 i$, which corresponds to adding a minus sign in both terms.
\end{enumerate}

\textbf{General Comment:} You can treat $i$ as a variable and distribute. Just remember that $i^2=-1$, so you can continue to reduce after you distribute.
}
\litem{
Simplify the expression below into the form $a+bi$. Then, choose the intervals that $a$ and $b$ belong to.
\[ \frac{-54 - 77 i}{-1 + 8 i} \]
The solution is \( -8.65  + 7.83 i \), which is option B.\begin{enumerate}[label=\Alph*.]
\item \( a \in [-9.5, -7] \text{ and } b \in [508, 510] \)

 $-8.65  + 509.00 i$, which corresponds to forgetting to multiply the conjugate by the numerator.
\item \( a \in [-9.5, -7] \text{ and } b \in [6.5, 8.5] \)

* $-8.65  + 7.83 i$, which is the correct option.
\item \( a \in [-563.5, -561] \text{ and } b \in [6.5, 8.5] \)

 $-562.00  + 7.83 i$, which corresponds to forgetting to multiply the conjugate by the numerator and using a plus instead of a minus in the denominator.
\item \( a \in [52.5, 54.5] \text{ and } b \in [-10, -9] \)

 $54.00  - 9.62 i$, which corresponds to just dividing the first term by the first term and the second by the second.
\item \( a \in [9.5, 11] \text{ and } b \in [-6, -4.5] \)

 $10.31  - 5.46 i$, which corresponds to forgetting to multiply the conjugate by the numerator and not computing the conjugate correctly.
\end{enumerate}

\textbf{General Comment:} Multiply the numerator and denominator by the *conjugate* of the denominator, then simplify. For example, if we have $2+3i$, the conjugate is $2-3i$.
}
\litem{
Choose the \textbf{smallest} set of Complex numbers that the number below belongs to.
\[ \frac{14}{-5}+\sqrt{-25}i \]
The solution is \( \text{Rational} \), which is option D.\begin{enumerate}[label=\Alph*.]
\item \( \text{Irrational} \)

These cannot be written as a fraction of Integers. Remember: $\pi$ is not an Integer!
\item \( \text{Not a Complex Number} \)

This is not a number. The only non-Complex number we know is dividing by 0 as this is not a number!
\item \( \text{Pure Imaginary} \)

This is a Complex number $(a+bi)$ that \textbf{only} has an imaginary part like $2i$.
\item \( \text{Rational} \)

* This is the correct option!
\item \( \text{Nonreal Complex} \)

This is a Complex number $(a+bi)$ that is not Real (has $i$ as part of the number).
\end{enumerate}

\textbf{General Comment:} Be sure to simplify $i^2 = -1$. This may remove the imaginary portion for your number. If you are having trouble, you may want to look at the \textit{Subgroups of the Real Numbers} section.
}
\litem{
Simplify the expression below and choose the interval the simplification is contained within.
\[ 15 - 10 \div 1 * 3 - (13 * 6) \]
The solution is \( -93.000 \), which is option C.\begin{enumerate}[label=\Alph*.]
\item \( [-68.33, -61.33] \)

 -66.333, which corresponds to an Order of Operations error: not reading left-to-right for multiplication/division.
\item \( [-168, -163] \)

 -168.000, which corresponds to not distributing a negative correctly.
\item \( [-94, -89] \)

* -93.000, which is the correct option.
\item \( [86.67, 90.67] \)

 89.667, which corresponds to not distributing addition and subtraction correctly.
\item \( \text{None of the above} \)

 You may have gotten this by making an unanticipated error. If you got a value that is not any of the others, please let the coordinator know so they can help you figure out what happened.
\end{enumerate}

\textbf{General Comment:} While you may remember (or were taught) PEMDAS is done in order, it is actually done as P/E/MD/AS. When we are at MD or AS, we read left to right.
}
\litem{
Simplify the expression below into the form $a+bi$. Then, choose the intervals that $a$ and $b$ belong to.
\[ \frac{-9 + 55 i}{-7 - 3 i} \]
The solution is \( -1.76  - 7.10 i \), which is option C.\begin{enumerate}[label=\Alph*.]
\item \( a \in [2.5, 5.5] \text{ and } b \in [-7, -6] \)

 $3.93  - 6.17 i$, which corresponds to forgetting to multiply the conjugate by the numerator and not computing the conjugate correctly.
\item \( a \in [-103, -101] \text{ and } b \in [-7.5, -6.5] \)

 $-102.00  - 7.10 i$, which corresponds to forgetting to multiply the conjugate by the numerator and using a plus instead of a minus in the denominator.
\item \( a \in [-2, 0] \text{ and } b \in [-7.5, -6.5] \)

* $-1.76  - 7.10 i$, which is the correct option.
\item \( a \in [1, 2] \text{ and } b \in [-18.5, -17.5] \)

 $1.29  - 18.33 i$, which corresponds to just dividing the first term by the first term and the second by the second.
\item \( a \in [-2, 0] \text{ and } b \in [-413, -411.5] \)

 $-1.76  - 412.00 i$, which corresponds to forgetting to multiply the conjugate by the numerator.
\end{enumerate}

\textbf{General Comment:} Multiply the numerator and denominator by the *conjugate* of the denominator, then simplify. For example, if we have $2+3i$, the conjugate is $2-3i$.
}
\litem{
Choose the \textbf{smallest} set of Complex numbers that the number below belongs to.
\[ \sqrt{\frac{-952}{0}} i+\sqrt{110}i \]
The solution is \( \text{Not a Complex Number} \), which is option E.\begin{enumerate}[label=\Alph*.]
\item \( \text{Irrational} \)

These cannot be written as a fraction of Integers. Remember: $\pi$ is not an Integer!
\item \( \text{Pure Imaginary} \)

This is a Complex number $(a+bi)$ that \textbf{only} has an imaginary part like $2i$.
\item \( \text{Rational} \)

These are numbers that can be written as fraction of Integers (e.g., -2/3 + 5)
\item \( \text{Nonreal Complex} \)

This is a Complex number $(a+bi)$ that is not Real (has $i$ as part of the number).
\item \( \text{Not a Complex Number} \)

* This is the correct option!
\end{enumerate}

\textbf{General Comment:} Be sure to simplify $i^2 = -1$. This may remove the imaginary portion for your number. If you are having trouble, you may want to look at the \textit{Subgroups of the Real Numbers} section.
}
\end{enumerate}

\end{document}