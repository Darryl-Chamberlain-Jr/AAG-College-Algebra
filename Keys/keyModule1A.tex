\documentclass{extbook}[14pt]
\usepackage{multicol, enumerate, enumitem, hyperref, color, soul, setspace, parskip, fancyhdr, amssymb, amsthm, amsmath, bbm, latexsym, units, mathtools}
\everymath{\displaystyle}
\usepackage[headsep=0.5cm,headheight=0cm, left=1 in,right= 1 in,top= 1 in,bottom= 1 in]{geometry}
\usepackage{dashrule}  % Package to use the command below to create lines between items
\newcommand{\litem}[1]{\item #1

\rule{\textwidth}{0.4pt}}
\pagestyle{fancy}
\lhead{}
\chead{Answer Key for Progress Quiz 1 Version A}
\rhead{}
\lfoot{3114-1073}
\cfoot{}
\rfoot{Fall 2020}
\begin{document}
\textbf{This key should allow you to understand why you choose the option you did (beyond just getting a question right or wrong). \href{https://xronos.clas.ufl.edu/mac1105spring2020/courseDescriptionAndMisc/Exams/LearningFromResults}{More instructions on how to use this key can be found here}.}

\textbf{If you have a suggestion to make the keys better, \href{https://forms.gle/CZkbZmPbC9XALEE88}{please fill out the short survey here}.}

\textit{Note: This key is auto-generated and may contain issues and/or errors. The keys are reviewed after each exam to ensure grading is done accurately. If there are issues (like duplicate options), they are noted in the offline gradebook. The keys are a work-in-progress to give students as many resources to improve as possible.}

\rule{\textwidth}{0.4pt}

\begin{enumerate}\litem{
Simplify the expression below into the form $a+bi$. Then, choose the intervals that $a$ and $b$ belong to.
\[ (2 - 4 i)(-5 + 6 i) \]
The solution is \( 14 + 32 i \), which is option B.\begin{enumerate}[label=\Alph*.]
\item \( a \in [13, 17] \text{ and } b \in [-32, -31] \)

 $14 - 32 i$, which corresponds to adding a minus sign in both terms.
\item \( a \in [13, 17] \text{ and } b \in [32, 35] \)

* $14 + 32 i$, which is the correct option.
\item \( a \in [-34, -32] \text{ and } b \in [-11, -7] \)

 $-34 - 8 i$, which corresponds to adding a minus sign in the first term.
\item \( a \in [-12, -7] \text{ and } b \in [-27, -21] \)

 $-10 - 24 i$, which corresponds to just multiplying the real terms to get the real part of the solution and the coefficients in the complex terms to get the complex part.
\item \( a \in [-34, -32] \text{ and } b \in [8, 9] \)

 $-34 + 8 i$, which corresponds to adding a minus sign in the second term.
\end{enumerate}

\textbf{General Comment:} You can treat $i$ as a variable and distribute. Just remember that $i^2=-1$, so you can continue to reduce after you distribute.
}
\litem{
Simplify the expression below into the form $a+bi$. Then, choose the intervals that $a$ and $b$ belong to.
\[ \frac{45 + 88 i}{-1 - 2 i} \]
The solution is \( -44.20  + 0.40 i \), which is option A.\begin{enumerate}[label=\Alph*.]
\item \( a \in [-44.5, -44] \text{ and } b \in [-0.5, 1] \)

* $-44.20  + 0.40 i$, which is the correct option.
\item \( a \in [26, 26.5] \text{ and } b \in [-36, -35] \)

 $26.20  - 35.60 i$, which corresponds to forgetting to multiply the conjugate by the numerator and not computing the conjugate correctly.
\item \( a \in [-44.5, -44] \text{ and } b \in [1.5, 2.5] \)

 $-44.20  + 2.00 i$, which corresponds to forgetting to multiply the conjugate by the numerator.
\item \( a \in [-222, -220.5] \text{ and } b \in [-0.5, 1] \)

 $-221.00  + 0.40 i$, which corresponds to forgetting to multiply the conjugate by the numerator and using a plus instead of a minus in the denominator.
\item \( a \in [-45.5, -44.5] \text{ and } b \in [-45.5, -43.5] \)

 $-45.00  - 44.00 i$, which corresponds to just dividing the first term by the first term and the second by the second.
\end{enumerate}

\textbf{General Comment:} Multiply the numerator and denominator by the *conjugate* of the denominator, then simplify. For example, if we have $2+3i$, the conjugate is $2-3i$.
}
\litem{
Choose the \textbf{smallest} set of Complex numbers that the number below belongs to.
\[ \sqrt{\frac{361}{0}}+\sqrt{45} i \]
The solution is \( \text{Not a Complex Number} \), which is option A.\begin{enumerate}[label=\Alph*.]
\item \( \text{Not a Complex Number} \)

* This is the correct option!
\item \( \text{Nonreal Complex} \)

This is a Complex number $(a+bi)$ that is not Real (has $i$ as part of the number).
\item \( \text{Pure Imaginary} \)

This is a Complex number $(a+bi)$ that \textbf{only} has an imaginary part like $2i$.
\item \( \text{Rational} \)

These are numbers that can be written as fraction of Integers (e.g., -2/3 + 5)
\item \( \text{Irrational} \)

These cannot be written as a fraction of Integers. Remember: $\pi$ is not an Integer!
\end{enumerate}

\textbf{General Comment:} Be sure to simplify $i^2 = -1$. This may remove the imaginary portion for your number. If you are having trouble, you may want to look at the \textit{Subgroups of the Real Numbers} section.
}
\litem{
Simplify the expression below and choose the interval the simplification is contained within.
\[ 9 - 1 \div 12 * 6 - (11 * 10) \]
The solution is \( -101.500 \), which is option A.\begin{enumerate}[label=\Alph*.]
\item \( [-101.87, -101.23] \)

* -101.500, which is the correct option.
\item \( [-101.49, -101] \)

 -101.014, which corresponds to an Order of Operations error: not reading left-to-right for multiplication/division.
\item \( [118.87, 119.35] \)

 118.986, which corresponds to not distributing addition and subtraction correctly.
\item \( [-25.17, -24.88] \)

 -25.000, which corresponds to not distributing a negative correctly.
\item \( \text{None of the above} \)

 You may have gotten this by making an unanticipated error. If you got a value that is not any of the others, please let the coordinator know so they can help you figure out what happened.
\end{enumerate}

\textbf{General Comment:} While you may remember (or were taught) PEMDAS is done in order, it is actually done as P/E/MD/AS. When we are at MD or AS, we read left to right.
}
\litem{
Choose the \textbf{smallest} set of Complex numbers that the number below belongs to.
\[ \sqrt{\frac{-1078}{14}} i+\sqrt{156}i \]
The solution is \( \text{Nonreal Complex} \), which is option E.\begin{enumerate}[label=\Alph*.]
\item \( \text{Pure Imaginary} \)

This is a Complex number $(a+bi)$ that \textbf{only} has an imaginary part like $2i$.
\item \( \text{Rational} \)

These are numbers that can be written as fraction of Integers (e.g., -2/3 + 5)
\item \( \text{Irrational} \)

These cannot be written as a fraction of Integers. Remember: $\pi$ is not an Integer!
\item \( \text{Not a Complex Number} \)

This is not a number. The only non-Complex number we know is dividing by 0 as this is not a number!
\item \( \text{Nonreal Complex} \)

* This is the correct option!
\end{enumerate}

\textbf{General Comment:} Be sure to simplify $i^2 = -1$. This may remove the imaginary portion for your number. If you are having trouble, you may want to look at the \textit{Subgroups of the Real Numbers} section.
}
\litem{
Simplify the expression below into the form $a+bi$. Then, choose the intervals that $a$ and $b$ belong to.
\[ \frac{18 - 55 i}{-7 - 8 i} \]
The solution is \( 2.78  + 4.68 i \), which is option B.\begin{enumerate}[label=\Alph*.]
\item \( a \in [-3, -1.5] \text{ and } b \in [6.5, 8] \)

 $-2.57  + 6.88 i$, which corresponds to just dividing the first term by the first term and the second by the second.
\item \( a \in [1.5, 3.5] \text{ and } b \in [4, 5.5] \)

* $2.78  + 4.68 i$, which is the correct option.
\item \( a \in [312.5, 314.5] \text{ and } b \in [4, 5.5] \)

 $314.00  + 4.68 i$, which corresponds to forgetting to multiply the conjugate by the numerator and using a plus instead of a minus in the denominator.
\item \( a \in [1.5, 3.5] \text{ and } b \in [528, 529.5] \)

 $2.78  + 529.00 i$, which corresponds to forgetting to multiply the conjugate by the numerator.
\item \( a \in [-5.5, -4.5] \text{ and } b \in [1.5, 3.5] \)

 $-5.01  + 2.13 i$, which corresponds to forgetting to multiply the conjugate by the numerator and not computing the conjugate correctly.
\end{enumerate}

\textbf{General Comment:} Multiply the numerator and denominator by the *conjugate* of the denominator, then simplify. For example, if we have $2+3i$, the conjugate is $2-3i$.
}
\litem{
Simplify the expression below and choose the interval the simplification is contained within.
\[ 9 - 15 \div 14 * 8 - (16 * 4) \]
The solution is \( -63.571 \), which is option A.\begin{enumerate}[label=\Alph*.]
\item \( [-64.05, -63.43] \)

* -63.571, which is the correct option.
\item \( [-62.63, -61.8] \)

 -62.286, which corresponds to not distributing a negative correctly.
\item \( [-56.03, -54.58] \)

 -55.134, which corresponds to an Order of Operations error: not reading left-to-right for multiplication/division.
\item \( [72.84, 74.4] \)

 72.866, which corresponds to not distributing addition and subtraction correctly.
\item \( \text{None of the above} \)

 You may have gotten this by making an unanticipated error. If you got a value that is not any of the others, please let the coordinator know so they can help you figure out what happened.
\end{enumerate}

\textbf{General Comment:} While you may remember (or were taught) PEMDAS is done in order, it is actually done as P/E/MD/AS. When we are at MD or AS, we read left to right.
}
\litem{
Choose the \textbf{smallest} set of Real numbers that the number below belongs to.
\[ -\sqrt{\frac{490}{7}} \]
The solution is \( \text{Irrational} \), which is option E.\begin{enumerate}[label=\Alph*.]
\item \( \text{Integer} \)

These are the negative and positive counting numbers (..., -3, -2, -1, 0, 1, 2, 3, ...)
\item \( \text{Not a Real number} \)

These are Nonreal Complex numbers \textbf{OR} things that are not numbers (e.g., dividing by 0).
\item \( \text{Whole} \)

These are the counting numbers with 0 (0, 1, 2, 3, ...)
\item \( \text{Rational} \)

These are numbers that can be written as fraction of Integers (e.g., -2/3)
\item \( \text{Irrational} \)

* This is the correct option!
\end{enumerate}

\textbf{General Comment:} First, you \textbf{NEED} to simplify the expression. This question simplifies to $-\sqrt{70}$. 
 
 Be sure you look at the simplified fraction and not just the decimal expansion. Numbers such as 13, 17, and 19 provide \textbf{long but repeating/terminating decimal expansions!} 
 
 The only ways to *not* be a Real number are: dividing by 0 or taking the square root of a negative number. 
 
 Irrational numbers are more than just square root of 3: adding or subtracting values from square root of 3 is also irrational.
}
\litem{
Simplify the expression below into the form $a+bi$. Then, choose the intervals that $a$ and $b$ belong to.
\[ (-4 + 10 i)(2 + 8 i) \]
The solution is \( -88 - 12 i \), which is option E.\begin{enumerate}[label=\Alph*.]
\item \( a \in [70, 78] \text{ and } b \in [-56, -48] \)

 $72 - 52 i$, which corresponds to adding a minus sign in the first term.
\item \( a \in [70, 78] \text{ and } b \in [46, 55] \)

 $72 + 52 i$, which corresponds to adding a minus sign in the second term.
\item \( a \in [-88, -83] \text{ and } b \in [6, 15] \)

 $-88 + 12 i$, which corresponds to adding a minus sign in both terms.
\item \( a \in [-12, -7] \text{ and } b \in [80, 83] \)

 $-8 + 80 i$, which corresponds to just multiplying the real terms to get the real part of the solution and the coefficients in the complex terms to get the complex part.
\item \( a \in [-88, -83] \text{ and } b \in [-15, -7] \)

* $-88 - 12 i$, which is the correct option.
\end{enumerate}

\textbf{General Comment:} You can treat $i$ as a variable and distribute. Just remember that $i^2=-1$, so you can continue to reduce after you distribute.
}
\litem{
Choose the \textbf{smallest} set of Real numbers that the number below belongs to.
\[ \sqrt{\frac{896}{8}} \]
The solution is \( \text{Irrational} \), which is option A.\begin{enumerate}[label=\Alph*.]
\item \( \text{Irrational} \)

* This is the correct option!
\item \( \text{Integer} \)

These are the negative and positive counting numbers (..., -3, -2, -1, 0, 1, 2, 3, ...)
\item \( \text{Rational} \)

These are numbers that can be written as fraction of Integers (e.g., -2/3)
\item \( \text{Whole} \)

These are the counting numbers with 0 (0, 1, 2, 3, ...)
\item \( \text{Not a Real number} \)

These are Nonreal Complex numbers \textbf{OR} things that are not numbers (e.g., dividing by 0).
\end{enumerate}

\textbf{General Comment:} First, you \textbf{NEED} to simplify the expression. This question simplifies to $\sqrt{112}$. 
 
 Be sure you look at the simplified fraction and not just the decimal expansion. Numbers such as 13, 17, and 19 provide \textbf{long but repeating/terminating decimal expansions!} 
 
 The only ways to *not* be a Real number are: dividing by 0 or taking the square root of a negative number. 
 
 Irrational numbers are more than just square root of 3: adding or subtracting values from square root of 3 is also irrational.
}
\end{enumerate}

\end{document}