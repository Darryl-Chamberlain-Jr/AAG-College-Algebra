\documentclass{extbook}[14pt]
\usepackage{multicol, enumerate, enumitem, hyperref, color, soul, setspace, parskip, fancyhdr, amssymb, amsthm, amsmath, bbm, latexsym, units, mathtools}
\everymath{\displaystyle}
\usepackage[headsep=0.5cm,headheight=0cm, left=1 in,right= 1 in,top= 1 in,bottom= 1 in]{geometry}
\pagestyle{fancy}
\lhead{}
\chead{Answer Key for Module\,1\,-\,Real\,and\,Complex\,Numbers Version A}
\rhead{}
\lfoot{Summer\,C\,2020}
\cfoot{}
\rfoot{}
\begin{document}
\textbf{This key should allow you to understand why you choose the option you did (beyond just getting a question right or wrong). \href{https://xronos.clas.ufl.edu/mac1105spring2020/courseDescriptionAndMisc/Exams/LearningFromResults}{More instructions on how to use this key can be found here}.}

\textbf{If you have a suggestion to make the keys better, \href{https://forms.gle/CZkbZmPbC9XALEE88}{please fill out the short survey here}.}

\textit{Note: This key is auto-generated and may contain issues and/or errors. The keys are reviewed after each exam to ensure grading is done accurately. If there are issues (like duplicate options), they are noted in the offline gradebook. The keys are a work-in-progress to give students as many resources to improve as possible.}

\rule{\textwidth}{0.4pt}

1. Simplify the expression below into the form $a+bi$. Then, choose the intervals that $a$ and $b$ belong to.
\[ (-9  - 7 i)(8  + 10 i) \] 
The solution is $ -2  - 146 i $ 

\begin{enumerate}[label=\Alph*.] 
\item $ a \in [-147, -138] \text{ and } b \in [-35, -33] $ 

  $-142  - 34 i$, which corresponds to adding a minus sign in the first term. 
\item $ a \in [-147, -138] \text{ and } b \in [32, 41] $ 

  $-142  + 34 i$, which corresponds to adding a minus sign in the second term. 
\item $ a \in [-4, 6] \text{ and } b \in [-149, -144] $ 

 * $-2  - 146 i$, which is the correct option. 
\item $ a \in [-4, 6] \text{ and } b \in [143, 150] $ 

  $-2  + 146 i$, which corresponds to adding a minus sign in both terms. 
\item $ a \in [-74, -67] \text{ and } b \in [-71, -66] $ 

  $-72  - 70 i$, which corresponds to just multiplying the real terms to get the real part of the solution and the coefficients in the complex terms to get the complex part. 
\end{enumerate} 
 
General Comments: You can treat $i$ as a variable and distribute. Just remember that $i^2=-1$, so you can continue to reduce after you distribute.

-----------------------------------------------

2. Simplify the expression below into the form $a+bi$. Then, choose the intervals that $a$ and $b$ belong to.
\[ \frac{27  - 22 i}{-6  + 5 i} \] 
The solution is $ -4.46  - 0.05 i $ 

\begin{enumerate}[label=\Alph*.] 
\item $ a \in [-4.54, -4.48] \text{ and } b \in [-5.92, -4.35] $ 

  $-4.50  - 4.40 i$, which corresponds to just dividing the first term by the first term and the second by the second. 
\item $ a \in [-272.01, -271.98] \text{ and } b \in [-0.78, 0.77] $ 

  $-272.00  - 0.05 i$, which corresponds to forgetting to multiply the conjugate by the numerator and using a plus instead of a minus in the denominator. 
\item $ a \in [-4.47, -4.44] \text{ and } b \in [-3.03, -2.73] $ 

  $-4.46  - 3.00 i$, which corresponds to forgetting to multiply the conjugate by the numerator. 
\item $ a \in [-4.47, -4.44] \text{ and } b \in [-0.78, 0.77] $ 

 * $-4.46  - 0.05 i$, which is the correct option. 
\item $ a \in [-0.86, -0.83] \text{ and } b \in [2.8, 4.87] $ 

  $-0.85  + 4.38 i$, which corresponds to forgetting to multiply the conjugate by the numerator and not computing the conjugate correctly. 
\end{enumerate} 
 
General Comment: Multiply the numerator and denominator by the *conjugate* of the denominator, then simplify. For example, if we have $2+3i$, the conjugate is $2-3i$.

-----------------------------------------------

3. Choose the \textbf{smallest} set of Real numbers that the number below belongs to.
\[ -\sqrt{\frac{20449}{121}} \] 
The solution is $ \text{\text{Integer}} $ 

\begin{enumerate}[label=\Alph*.] 
\item $ \text{Not a Real number} $ 

 These are Nonreal Complex numbers OR things that are not numbers (dividing by 0). 
\item $ \text{Irrational} $ 

 These cannot be written as a fraction of Integers. 
\item $ \text{Whole} $ 

 These are the counting numbers with 0 (0, 1, 2, 3, ...) 
\item $ \text{Integer} $ 

 These are the negative and positive counting numbers (..., -3, -2, -1, 0, 1, 2, 3, ...) 
\item $ \text{Rational} $ 

 These are numbers that can be written as fraction of Integers (e.g., -2/3) 
\end{enumerate} 
 
General Comments: First, you \textbf{NEED} to simplify the expression. This question simplifies to $-143$. 
 
 Be sure you look at the simplified fraction and not just the decimal expansion. Numbers such as 13, 17, and 19 provide \textbf{long but repeating/terminating decimal expansions!} 
 
 The only ways to *not* be a Real number are: dividing by 0 or taking the square root of a negative number. Irrational numbers are more than just square root of 3: adding or subtracting values from square root of 3 is also irrational.

-----------------------------------------------

4. Choose the \textbf{smallest} set of Complex numbers that the number below belongs to.
\[ \sqrt{\frac{880}{8}}+4i^2 \] 
The solution is $ \text{Irrational} $ 

\begin{enumerate}[label=\Alph*.] 
\item $ \text{Rational} $ 

 These are numbers that can be written as fraction of Integers (e.g., -2/3 + 5) 
\item $ \text{Irrational} $ 

 These cannot be written as a fraction of Integers. Remember: $\pi$ is not an Integer! 
\item $ \text{Nonreal Complex} $ 

 This is a Complex number $(a+bi)$ that is not Real (has $i$ as part of the number). 
\item $ \text{Not a Complex Number} $ 

 This is not a number. The only non-Complex number we know is dividing by 0 as this is not a number! 
\item $ \text{Pure Imaginary} $ 

 This is a Complex number $(a+bi)$ that \textbf{only} has an imaginary part like $2i$. 
\end{enumerate} 
 
General Comments: Be sure to simplify $i^2 = -1$. This may remove the imaginary portion for your number. If you are having trouble, you may want to look at the \textit{Subgroups of the Real Numbers} section.

-----------------------------------------------

5. Simplify the expression below and choose the interval the simplification is contained within.
\[ 3 - 15^2 + 18 \div 10 * 19 \div 14 \] 
The solution is $ -219.557 $ 

\begin{enumerate}[label=\Alph*.] 
\item $ [229.74, 231.65] $ 

  230.443000, which corresponds to an Order of Operations error: multiplying by negative before squaring. For example: $(-3)^2 \neq -3^2$ 
\item $ [227.53, 228.31] $ 

  228.007000, which corresponds to two Order of Operations errors. 
\item $ [-220.44, -219.19] $ 

 * -219.557000, this is the correct option 
\item $ [-222.27, -220.7] $ 

  -221.993000, which corresponds to an Order of Operations error: not reading left-to-right for multiplication/division. 
\item $ \text{None of the above} $ 

  You may have gotten this by making an unanticipated error. If you got a value that is not any of the others, please let the coordinator know so they can help you figure out what happened. 
\end{enumerate} 
 
General Comments: While you may remember (or were taught) PEMDAS is done in order, it is actually done as P/E/MD/AS. When we are at MD or AS, we read left to right.

-----------------------------------------------


\end{document}

