\documentclass{extbook}[14pt]
\usepackage{multicol, enumerate, enumitem, hyperref, color, soul, setspace, parskip, fancyhdr, amssymb, amsthm, amsmath, bbm, latexsym, units, mathtools}
\everymath{\displaystyle}
\usepackage[headsep=0.5cm,headheight=0cm, left=1 in,right= 1 in,top= 1 in,bottom= 1 in]{geometry}
\usepackage{dashrule}  % Package to use the command below to create lines between items
\newcommand{\litem}[1]{\item #1

\rule{\textwidth}{0.4pt}}
\pagestyle{fancy}
\lhead{}
\chead{Answer Key for Progress Quiz 9 Version A}
\rhead{}
\lfoot{8590-6105}
\cfoot{}
\rfoot{Fall 2020}
\begin{document}
\textbf{This key should allow you to understand why you choose the option you did (beyond just getting a question right or wrong). \href{https://xronos.clas.ufl.edu/mac1105spring2020/courseDescriptionAndMisc/Exams/LearningFromResults}{More instructions on how to use this key can be found here}.}

\textbf{If you have a suggestion to make the keys better, \href{https://forms.gle/CZkbZmPbC9XALEE88}{please fill out the short survey here}.}

\textit{Note: This key is auto-generated and may contain issues and/or errors. The keys are reviewed after each exam to ensure grading is done accurately. If there are issues (like duplicate options), they are noted in the offline gradebook. The keys are a work-in-progress to give students as many resources to improve as possible.}

\rule{\textwidth}{0.4pt}

\begin{enumerate}\litem{
Choose the \textbf{smallest} set of Real numbers that the number below belongs to.
\[ -\sqrt{\frac{81}{625}} \]

The solution is \( \text{Rational} \), which is option D.\begin{enumerate}[label=\Alph*.]
\item \( \text{Whole} \)

These are the counting numbers with 0 (0, 1, 2, 3, ...)
\item \( \text{Integer} \)

These are the negative and positive counting numbers (..., -3, -2, -1, 0, 1, 2, 3, ...)
\item \( \text{Irrational} \)

These cannot be written as a fraction of Integers.
\item \( \text{Rational} \)

* This is the correct option!
\item \( \text{Not a Real number} \)

These are Nonreal Complex numbers \textbf{OR} things that are not numbers (e.g., dividing by 0).
\end{enumerate}

\textbf{General Comment:} First, you \textbf{NEED} to simplify the expression. This question simplifies to $-\frac{9}{25}$. 
 
 Be sure you look at the simplified fraction and not just the decimal expansion. Numbers such as 13, 17, and 19 provide \textbf{long but repeating/terminating decimal expansions!} 
 
 The only ways to *not* be a Real number are: dividing by 0 or taking the square root of a negative number. 
 
 Irrational numbers are more than just square root of 3: adding or subtracting values from square root of 3 is also irrational.
}
\litem{
Simplify the expression below into the form $a+bi$. Then, choose the intervals that $a$ and $b$ belong to.
\[ (3 - 8 i)(6 + 4 i) \]

The solution is \( 50 - 36 i \), which is option E.\begin{enumerate}[label=\Alph*.]
\item \( a \in [-14, -9] \text{ and } b \in [57, 62] \)

 $-14 + 60 i$, which corresponds to adding a minus sign in the first term.
\item \( a \in [16, 20] \text{ and } b \in [-34, -29] \)

 $18 - 32 i$, which corresponds to just multiplying the real terms to get the real part of the solution and the coefficients in the complex terms to get the complex part.
\item \( a \in [-14, -9] \text{ and } b \in [-64, -53] \)

 $-14 - 60 i$, which corresponds to adding a minus sign in the second term.
\item \( a \in [48, 58] \text{ and } b \in [28, 37] \)

 $50 + 36 i$, which corresponds to adding a minus sign in both terms.
\item \( a \in [48, 58] \text{ and } b \in [-40, -33] \)

* $50 - 36 i$, which is the correct option.
\end{enumerate}

\textbf{General Comment:} You can treat $i$ as a variable and distribute. Just remember that $i^2=-1$, so you can continue to reduce after you distribute.
}
\litem{
Choose the \textbf{smallest} set of Complex numbers that the number below belongs to.
\[ \sqrt{\frac{1190}{10}}+\sqrt{143} i \]

The solution is \( \text{Nonreal Complex} \), which is option A.\begin{enumerate}[label=\Alph*.]
\item \( \text{Nonreal Complex} \)

* This is the correct option!
\item \( \text{Irrational} \)

These cannot be written as a fraction of Integers. Remember: $\pi$ is not an Integer!
\item \( \text{Not a Complex Number} \)

This is not a number. The only non-Complex number we know is dividing by 0 as this is not a number!
\item \( \text{Rational} \)

These are numbers that can be written as fraction of Integers (e.g., -2/3 + 5)
\item \( \text{Pure Imaginary} \)

This is a Complex number $(a+bi)$ that \textbf{only} has an imaginary part like $2i$.
\end{enumerate}

\textbf{General Comment:} Be sure to simplify $i^2 = -1$. This may remove the imaginary portion for your number. If you are having trouble, you may want to look at the \textit{Subgroups of the Real Numbers} section.
}
\litem{
Simplify the expression below into the form $a+bi$. Then, choose the intervals that $a$ and $b$ belong to.
\[ \frac{9 - 44 i}{-2 + 3 i} \]

The solution is \( -11.54  + 4.69 i \), which is option E.\begin{enumerate}[label=\Alph*.]
\item \( a \in [-151, -148] \text{ and } b \in [3.5, 5] \)

 $-150.00  + 4.69 i$, which corresponds to forgetting to multiply the conjugate by the numerator and using a plus instead of a minus in the denominator.
\item \( a \in [8, 9.5] \text{ and } b \in [8, 10.5] \)

 $8.77  + 8.85 i$, which corresponds to forgetting to multiply the conjugate by the numerator and not computing the conjugate correctly.
\item \( a \in [-12.5, -10] \text{ and } b \in [60.5, 61.5] \)

 $-11.54  + 61.00 i$, which corresponds to forgetting to multiply the conjugate by the numerator.
\item \( a \in [-5.5, -4] \text{ and } b \in [-15.5, -13.5] \)

 $-4.50  - 14.67 i$, which corresponds to just dividing the first term by the first term and the second by the second.
\item \( a \in [-12.5, -10] \text{ and } b \in [3.5, 5] \)

* $-11.54  + 4.69 i$, which is the correct option.
\end{enumerate}

\textbf{General Comment:} Multiply the numerator and denominator by the *conjugate* of the denominator, then simplify. For example, if we have $2+3i$, the conjugate is $2-3i$.
}
\litem{
Simplify the expression below and choose the interval the simplification is contained within.
\[ 6 - 12^2 + 8 \div 20 * 7 \div 18 \]

The solution is \( -137.844 \), which is option C.\begin{enumerate}[label=\Alph*.]
\item \( [-138.19, -137.89] \)

 -137.997, which corresponds to an Order of Operations error: not reading left-to-right for multiplication/division.
\item \( [150.06, 150.37] \)

 150.156, which corresponds to an Order of Operations error: multiplying by negative before squaring. For example: $(-3)^2 \neq -3^2$
\item \( [-137.85, -137.52] \)

* -137.844, this is the correct option
\item \( [149.71, 150.09] \)

 150.003, which corresponds to two Order of Operations errors.
\item \( \text{None of the above} \)

 You may have gotten this by making an unanticipated error. If you got a value that is not any of the others, please let the coordinator know so they can help you figure out what happened.
\end{enumerate}

\textbf{General Comment:} While you may remember (or were taught) PEMDAS is done in order, it is actually done as P/E/MD/AS. When we are at MD or AS, we read left to right.
}
\litem{
Simplify the expression below into the form $a+bi$. Then, choose the intervals that $a$ and $b$ belong to.
\[ (-2 + 8 i)(-5 + 6 i) \]

The solution is \( -38 - 52 i \), which is option C.\begin{enumerate}[label=\Alph*.]
\item \( a \in [57, 61] \text{ and } b \in [-31, -22] \)

 $58 - 28 i$, which corresponds to adding a minus sign in the second term.
\item \( a \in [9, 12] \text{ and } b \in [45, 51] \)

 $10 + 48 i$, which corresponds to just multiplying the real terms to get the real part of the solution and the coefficients in the complex terms to get the complex part.
\item \( a \in [-41, -37] \text{ and } b \in [-56, -46] \)

* $-38 - 52 i$, which is the correct option.
\item \( a \in [57, 61] \text{ and } b \in [27, 29] \)

 $58 + 28 i$, which corresponds to adding a minus sign in the first term.
\item \( a \in [-41, -37] \text{ and } b \in [50, 53] \)

 $-38 + 52 i$, which corresponds to adding a minus sign in both terms.
\end{enumerate}

\textbf{General Comment:} You can treat $i$ as a variable and distribute. Just remember that $i^2=-1$, so you can continue to reduce after you distribute.
}
\litem{
Choose the \textbf{smallest} set of Complex numbers that the number below belongs to.
\[ \frac{\sqrt{143}}{8}+4i^2 \]

The solution is \( \text{Irrational} \), which is option A.\begin{enumerate}[label=\Alph*.]
\item \( \text{Irrational} \)

* This is the correct option!
\item \( \text{Not a Complex Number} \)

This is not a number. The only non-Complex number we know is dividing by 0 as this is not a number!
\item \( \text{Nonreal Complex} \)

This is a Complex number $(a+bi)$ that is not Real (has $i$ as part of the number).
\item \( \text{Rational} \)

These are numbers that can be written as fraction of Integers (e.g., -2/3 + 5)
\item \( \text{Pure Imaginary} \)

This is a Complex number $(a+bi)$ that \textbf{only} has an imaginary part like $2i$.
\end{enumerate}

\textbf{General Comment:} Be sure to simplify $i^2 = -1$. This may remove the imaginary portion for your number. If you are having trouble, you may want to look at the \textit{Subgroups of the Real Numbers} section.
}
\litem{
Choose the \textbf{smallest} set of Real numbers that the number below belongs to.
\[ \sqrt{\frac{1040}{13}} \]

The solution is \( \text{Irrational} \), which is option E.\begin{enumerate}[label=\Alph*.]
\item \( \text{Integer} \)

These are the negative and positive counting numbers (..., -3, -2, -1, 0, 1, 2, 3, ...)
\item \( \text{Rational} \)

These are numbers that can be written as fraction of Integers (e.g., -2/3)
\item \( \text{Not a Real number} \)

These are Nonreal Complex numbers \textbf{OR} things that are not numbers (e.g., dividing by 0).
\item \( \text{Whole} \)

These are the counting numbers with 0 (0, 1, 2, 3, ...)
\item \( \text{Irrational} \)

* This is the correct option!
\end{enumerate}

\textbf{General Comment:} First, you \textbf{NEED} to simplify the expression. This question simplifies to $\sqrt{80}$. 
 
 Be sure you look at the simplified fraction and not just the decimal expansion. Numbers such as 13, 17, and 19 provide \textbf{long but repeating/terminating decimal expansions!} 
 
 The only ways to *not* be a Real number are: dividing by 0 or taking the square root of a negative number. 
 
 Irrational numbers are more than just square root of 3: adding or subtracting values from square root of 3 is also irrational.
}
\litem{
Simplify the expression below and choose the interval the simplification is contained within.
\[ 17 - 7^2 + 8 \div 2 * 3 \div 13 \]

The solution is \( -31.077 \), which is option A.\begin{enumerate}[label=\Alph*.]
\item \( [-31.36, -30.93] \)

* -31.077, this is the correct option
\item \( [-32.85, -31.71] \)

 -31.897, which corresponds to an Order of Operations error: not reading left-to-right for multiplication/division.
\item \( [65.48, 66.41] \)

 66.103, which corresponds to two Order of Operations errors.
\item \( [66.81, 67.15] \)

 66.923, which corresponds to an Order of Operations error: multiplying by negative before squaring. For example: $(-3)^2 \neq -3^2$
\item \( \text{None of the above} \)

 You may have gotten this by making an unanticipated error. If you got a value that is not any of the others, please let the coordinator know so they can help you figure out what happened.
\end{enumerate}

\textbf{General Comment:} While you may remember (or were taught) PEMDAS is done in order, it is actually done as P/E/MD/AS. When we are at MD or AS, we read left to right.
}
\litem{
Simplify the expression below into the form $a+bi$. Then, choose the intervals that $a$ and $b$ belong to.
\[ \frac{-45 + 66 i}{3 + 2 i} \]

The solution is \( -0.23  + 22.15 i \), which is option B.\begin{enumerate}[label=\Alph*.]
\item \( a \in [-3.5, -2.5] \text{ and } b \in [20.5, 23.5] \)

 $-3.00  + 22.15 i$, which corresponds to forgetting to multiply the conjugate by the numerator and using a plus instead of a minus in the denominator.
\item \( a \in [-1, 0.5] \text{ and } b \in [20.5, 23.5] \)

* $-0.23  + 22.15 i$, which is the correct option.
\item \( a \in [-16, -14.5] \text{ and } b \in [32.5, 33.5] \)

 $-15.00  + 33.00 i$, which corresponds to just dividing the first term by the first term and the second by the second.
\item \( a \in [-21.5, -19.5] \text{ and } b \in [7.5, 9.5] \)

 $-20.54  + 8.31 i$, which corresponds to forgetting to multiply the conjugate by the numerator and not computing the conjugate correctly.
\item \( a \in [-1, 0.5] \text{ and } b \in [287.5, 289] \)

 $-0.23  + 288.00 i$, which corresponds to forgetting to multiply the conjugate by the numerator.
\end{enumerate}

\textbf{General Comment:} Multiply the numerator and denominator by the *conjugate* of the denominator, then simplify. For example, if we have $2+3i$, the conjugate is $2-3i$.
}
\end{enumerate}

\end{document}