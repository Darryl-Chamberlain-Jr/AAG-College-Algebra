\documentclass{extbook}[14pt]
\usepackage{multicol, enumerate, enumitem, hyperref, color, soul, setspace, parskip, fancyhdr, amssymb, amsthm, amsmath, bbm, latexsym, units, mathtools}
\everymath{\displaystyle}
\usepackage[headsep=0.5cm,headheight=0cm, left=1 in,right= 1 in,top= 1 in,bottom= 1 in]{geometry}
\usepackage{dashrule}  % Package to use the command below to create lines between items
\newcommand{\litem}[1]{\item #1

\rule{\textwidth}{0.4pt}}
\pagestyle{fancy}
\lhead{}
\chead{Answer Key for Progress Quiz 9 Version B}
\rhead{}
\lfoot{8590-6105}
\cfoot{}
\rfoot{Fall 2020}
\begin{document}
\textbf{This key should allow you to understand why you choose the option you did (beyond just getting a question right or wrong). \href{https://xronos.clas.ufl.edu/mac1105spring2020/courseDescriptionAndMisc/Exams/LearningFromResults}{More instructions on how to use this key can be found here}.}

\textbf{If you have a suggestion to make the keys better, \href{https://forms.gle/CZkbZmPbC9XALEE88}{please fill out the short survey here}.}

\textit{Note: This key is auto-generated and may contain issues and/or errors. The keys are reviewed after each exam to ensure grading is done accurately. If there are issues (like duplicate options), they are noted in the offline gradebook. The keys are a work-in-progress to give students as many resources to improve as possible.}

\rule{\textwidth}{0.4pt}

\begin{enumerate}\litem{
 Solve the equation for $x$ and choose the interval that contains $x$ (if it exists).
\[  6 = \sqrt[4]{\frac{27}{e^{8x}}} \]

The solution is \( x = -0.484 \), which is option A.\begin{enumerate}[label=\Alph*.]
\item \( x \in [-0.95, -0.34] \)

* $x = -0.484$, which is the correct option.
\item \( x \in [-0.22, 0] \)

$x = -0.036$, which corresponds to treating any root as a square root.
\item \( x \in [-4.76, -2.44] \)

$x = -3.412$, which corresponds to thinking you don't need to take the natural log of both sides before reducing, as if the equation already had a natural log on the right side.
\item \( \text{There is no Real solution to the equation.} \)

This corresponds to believing you cannot solve the equation.
\item \( \text{None of the above.} \)

This corresponds to making an unexpected error.
\end{enumerate}

\textbf{General Comment:} \textbf{General Comments}: After using the properties of logarithmic functions to break up the right-hand side, use $\ln(e) = 1$ to reduce the question to a linear function to solve. You can put $\ln(27)$ into a calculator if you are having trouble.
}
\litem{
Solve the equation for $x$ and choose the interval that contains the solution (if it exists).
\[ 3^{3x+5} = \left(\frac{1}{16}\right)^{4x+4} \]

The solution is \( x = -1.153 \), which is option D.\begin{enumerate}[label=\Alph*.]
\item \( x \in [16.3, 17.2] \)

$x = 16.583$, which corresponds to distributing the $\ln(base)$ to the second term of the exponent only.
\item \( x \in [0.8, 1.71] \)

$x = 1.000$, which corresponds to solving the numerators as equal while ignoring the bases are different.
\item \( x \in [-0.85, 0.45] \)

$x = -0.070$, which corresponds to distributing the $\ln(base)$ to the first term of the exponent only.
\item \( x \in [-1.92, -0.79] \)

* $x = -1.153$, which is the correct option.
\item \( \text{There is no Real solution to the equation.} \)

This corresponds to believing there is no solution since the bases are not powers of each other.
\end{enumerate}

\textbf{General Comment:} \textbf{General Comments:} This question was written so that the bases could not be written the same. You will need to take the log of both sides.
}
\litem{
Solve the equation for $x$ and choose the interval that contains the solution (if it exists).
\[ 2^{-2x-3} = \left(\frac{1}{9}\right)^{-5x+5} \]

The solution is \( x = 0.720 \), which is option B.\begin{enumerate}[label=\Alph*.]
\item \( x \in [-3.2, -1.7] \)

$x = -2.969$, which corresponds to distributing the $\ln(base)$ to the second term of the exponent only.
\item \( x \in [0.3, 1.2] \)

* $x = 0.720$, which is the correct option.
\item \( x \in [1.6, 4.1] \)

$x = 2.667$, which corresponds to solving the numerators as equal while ignoring the bases are different.
\item \( x \in [-1.9, 0.3] \)

$x = -0.647$, which corresponds to distributing the $\ln(base)$ to the first term of the exponent only.
\item \( \text{There is no Real solution to the equation.} \)

This corresponds to believing there is no solution since the bases are not powers of each other.
\end{enumerate}

\textbf{General Comment:} \textbf{General Comments:} This question was written so that the bases could not be written the same. You will need to take the log of both sides.
}
\litem{
Solve the equation for $x$ and choose the interval that contains the solution (if it exists).
\[ \log_{5}{(-3x+8)}+6 = 3 \]

The solution is \( x = 2.664 \), which is option A.\begin{enumerate}[label=\Alph*.]
\item \( x \in [-2.34, 8.66] \)

* $x = 2.664$, which is the correct option.
\item \( x \in [78.33, 82.33] \)

$x = 78.333$, which corresponds to reversing the base and exponent when converting and reversing the value with $x$.
\item \( x \in [81.67, 84.67] \)

$x = 83.667$, which corresponds to reversing the base and exponent when converting.
\item \( x \in [-43, -35] \)

$x = -39.000$, which corresponds to ignoring the vertical shift when converting to exponential form.
\item \( \text{There is no Real solution to the equation.} \)

Corresponds to believing a negative coefficient within the log equation means there is no Real solution.
\end{enumerate}

\textbf{General Comment:} \textbf{General Comments:} First, get the equation in the form $\log_b{(cx+d)} = a$. Then, convert to $b^a = cx+d$ and solve.
}
\litem{
Which of the following intervals describes the Domain of the function below?
\[ f(x) = e^{x+7}+2 \]

The solution is \( (-\infty, \infty) \), which is option E.\begin{enumerate}[label=\Alph*.]
\item \( (-\infty, a], a \in [0.6, 5.1] \)

$(-\infty, 2]$, which corresponds to using the correct vertical shift *if we wanted the Range* AND including the endpoint.
\item \( (a, \infty), a \in [-3, -1.2] \)

$(-2, \infty)$, which corresponds to using the negative vertical shift AND flipping the Range interval.
\item \( [a, \infty), a \in [-3, -1.2] \)

$[-2, \infty)$, which corresponds to using the negative vertical shift AND flipping the Range interval AND including the endpoint.
\item \( (-\infty, a), a \in [0.6, 5.1] \)

$(-\infty, 2)$, which corresponds to using the correct vertical shift *if we wanted the Range*.
\item \( (-\infty, \infty) \)

* This is the correct option.
\end{enumerate}

\textbf{General Comment:} \textbf{General Comments}: Domain of a basic exponential function is $(-\infty, \infty)$ while the Range is $(0, \infty)$. We can shift these intervals [and even flip when $a<0$!] to find the new Domain/Range.
}
\litem{
 Solve the equation for $x$ and choose the interval that contains $x$ (if it exists).
\[  17 = \ln{\sqrt[4]{\frac{23}{e^{7x}}}} \]

The solution is \( x = -9.266, \text{ which does not fit in any of the interval options.} \), which is option E.\begin{enumerate}[label=\Alph*.]
\item \( x \in [-5.41, -3.41] \)

$x = -4.409$, which corresponds to treating any root as a square root.
\item \( x \in [-2.07, -1.07] \)

$x = -2.067$, which corresponds to thinking you need to take the natural log of the left side before reducing.
\item \( x \in [8.27, 12.27] \)

$x = 9.266$, which is the negative of the correct solution.
\item \( \text{There is no Real solution to the equation.} \)

This corresponds to believing you cannot solve the equation.
\item \( \text{None of the above.} \)

*$x = -9.266$ is the correct solution and does not fit in any of the other intervals.
\end{enumerate}

\textbf{General Comment:} \textbf{General Comments}: After using the properties of logarithmic functions to break up the right-hand side, use $\ln(e) = 1$ to reduce the question to a linear function to solve. You can put $\ln(23)$ into a calculator if you are having trouble.
}
\litem{
Solve the equation for $x$ and choose the interval that contains the solution (if it exists).
\[ \log_{4}{(4x+6)}+5 = 2 \]

The solution is \( x = -1.496 \), which is option A.\begin{enumerate}[label=\Alph*.]
\item \( x \in [-3.6, 1] \)

* $x = -1.496$, which is the correct option.
\item \( x \in [2.4, 5.3] \)

$x = 2.500$, which corresponds to ignoring the vertical shift when converting to exponential form.
\item \( x \in [19.3, 24.7] \)

$x = 21.750$, which corresponds to reversing the base and exponent when converting and reversing the value with $x$.
\item \( x \in [15.8, 20.2] \)

$x = 18.750$, which corresponds to reversing the base and exponent when converting.
\item \( \text{There is no Real solution to the equation.} \)

Corresponds to believing a negative coefficient within the log equation means there is no Real solution.
\end{enumerate}

\textbf{General Comment:} \textbf{General Comments:} First, get the equation in the form $\log_b{(cx+d)} = a$. Then, convert to $b^a = cx+d$ and solve.
}
\litem{
Which of the following intervals describes the Domain of the function below?
\[ f(x) = -\log_2{(x-9)}-1 \]

The solution is \( (9, \infty) \), which is option D.\begin{enumerate}[label=\Alph*.]
\item \( (-\infty, a), a \in [-9.1, -7.5] \)

$(-\infty, -9)$, which corresponds to flipping the Domain. Remember: the general for is $a*\log(x-h)+k$, \textbf{where $a$ does not affect the domain}.
\item \( (-\infty, a], a \in [-0.9, 2.6] \)

$(-\infty, 1]$, which corresponds to using the negative vertical shift AND including the endpoint AND flipping the domain.
\item \( [a, \infty), a \in [-2.5, 0.5] \)

$[-1, \infty)$, which corresponds to using the vertical shift when shifting the Domain AND including the endpoint.
\item \( (a, \infty), a \in [7, 12.8] \)

* $(9, \infty)$, which is the correct option.
\item \( (-\infty, \infty) \)

This corresponds to thinking of the range of the log function (or the domain of the exponential function).
\end{enumerate}

\textbf{General Comment:} \textbf{General Comments}: The domain of a basic logarithmic function is $(0, \infty)$ and the Range is $(-\infty, \infty)$. We can use shifts when finding the Domain, but the Range will always be all Real numbers.
}
\litem{
Which of the following intervals describes the Range of the function below?
\[ f(x) = -e^{x-2}-3 \]

The solution is \( (-\infty, -3) \), which is option A.\begin{enumerate}[label=\Alph*.]
\item \( (-\infty, a), a \in [-5, 0] \)

* $(-\infty, -3)$, which is the correct option.
\item \( (-\infty, a], a \in [-5, 0] \)

$(-\infty, -3]$, which corresponds to including the endpoint.
\item \( [a, \infty), a \in [3, 4] \)

$[3, \infty)$, which corresponds to using the negative vertical shift AND flipping the Range interval AND including the endpoint.
\item \( (a, \infty), a \in [3, 4] \)

$(3, \infty)$, which corresponds to using the negative vertical shift AND flipping the Range interval.
\item \( (-\infty, \infty) \)

This corresponds to confusing range of an exponential function with the domain of an exponential function.
\end{enumerate}

\textbf{General Comment:} \textbf{General Comments}: Domain of a basic exponential function is $(-\infty, \infty)$ while the Range is $(0, \infty)$. We can shift these intervals [and even flip when $a<0$!] to find the new Domain/Range.
}
\litem{
Which of the following intervals describes the Domain of the function below?
\[ f(x) = -\log_2{(x+2)}-8 \]

The solution is \( (-2, \infty) \), which is option A.\begin{enumerate}[label=\Alph*.]
\item \( (a, \infty), a \in [-2.5, -0.4] \)

* $(-2, \infty)$, which is the correct option.
\item \( (-\infty, a), a \in [-0.3, 3.4] \)

$(-\infty, 2)$, which corresponds to flipping the Domain. Remember: the general for is $a*\log(x-h)+k$, \textbf{where $a$ does not affect the domain}.
\item \( [a, \infty), a \in [-8.6, -6.8] \)

$[-8, \infty)$, which corresponds to using the vertical shift when shifting the Domain AND including the endpoint.
\item \( (-\infty, a], a \in [5.7, 10.3] \)

$(-\infty, 8]$, which corresponds to using the negative vertical shift AND including the endpoint AND flipping the domain.
\item \( (-\infty, \infty) \)

This corresponds to thinking of the range of the log function (or the domain of the exponential function).
\end{enumerate}

\textbf{General Comment:} \textbf{General Comments}: The domain of a basic logarithmic function is $(0, \infty)$ and the Range is $(-\infty, \infty)$. We can use shifts when finding the Domain, but the Range will always be all Real numbers.
}
\end{enumerate}

\end{document}