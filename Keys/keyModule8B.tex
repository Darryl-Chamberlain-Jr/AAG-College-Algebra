\documentclass{extbook}[14pt]
\usepackage{multicol, enumerate, enumitem, hyperref, color, soul, setspace, parskip, fancyhdr, amssymb, amsthm, amsmath, bbm, latexsym, units, mathtools}
\everymath{\displaystyle}
\usepackage[headsep=0.5cm,headheight=0cm, left=1 in,right= 1 in,top= 1 in,bottom= 1 in]{geometry}
\usepackage{dashrule}  % Package to use the command below to create lines between items
\newcommand{\litem}[1]{\item #1

\rule{\textwidth}{0.4pt}}
\pagestyle{fancy}
\lhead{}
\chead{Answer Key for Progress Quiz 5 Version B}
\rhead{}
\lfoot{9912-2038}
\cfoot{}
\rfoot{Spring 2021}
\begin{document}
\textbf{This key should allow you to understand why you choose the option you did (beyond just getting a question right or wrong). \href{https://xronos.clas.ufl.edu/mac1105spring2020/courseDescriptionAndMisc/Exams/LearningFromResults}{More instructions on how to use this key can be found here}.}

\textbf{If you have a suggestion to make the keys better, \href{https://forms.gle/CZkbZmPbC9XALEE88}{please fill out the short survey here}.}

\textit{Note: This key is auto-generated and may contain issues and/or errors. The keys are reviewed after each exam to ensure grading is done accurately. If there are issues (like duplicate options), they are noted in the offline gradebook. The keys are a work-in-progress to give students as many resources to improve as possible.}

\rule{\textwidth}{0.4pt}

\begin{enumerate}\litem{
Which of the following intervals describes the Domain of the function below?
\[ f(x) = \log_2{(x+1)}-5 \]The solution is \( (-1, \infty) \), which is option A.\begin{enumerate}[label=\Alph*.]
\item \( (a, \infty), a \in [-2.6, 0.7] \)

* $(-1, \infty)$, which is the correct option.
\item \( (-\infty, a), a \in [0.7, 2.5] \)

$(-\infty, 1)$, which corresponds to flipping the Domain. Remember: the general for is $a*\log(x-h)+k$, \textbf{where $a$ does not affect the domain}.
\item \( (-\infty, a], a \in [2.9, 8.2] \)

$(-\infty, 5]$, which corresponds to using the negative vertical shift AND including the endpoint AND flipping the domain.
\item \( [a, \infty), a \in [-7.8, -3.6] \)

$[-5, \infty)$, which corresponds to using the vertical shift when shifting the Domain AND including the endpoint.
\item \( (-\infty, \infty) \)

This corresponds to thinking of the range of the log function (or the domain of the exponential function).
\end{enumerate}

\textbf{General Comment:} \textbf{General Comments}: The domain of a basic logarithmic function is $(0, \infty)$ and the Range is $(-\infty, \infty)$. We can use shifts when finding the Domain, but the Range will always be all Real numbers.
}
\litem{
Solve the equation for $x$ and choose the interval that contains the solution (if it exists).
\[ \log_{5}{(-2x+5)}+6 = 3 \]The solution is \( x = 2.496 \), which is option B.\begin{enumerate}[label=\Alph*.]
\item \( x \in [-61, -58] \)

$x = -60.000$, which corresponds to ignoring the vertical shift when converting to exponential form.
\item \( x \in [1.5, 3.5] \)

* $x = 2.496$, which is the correct option.
\item \( x \in [123, 134] \)

$x = 124.000$, which corresponds to reversing the base and exponent when converting.
\item \( x \in [116, 121] \)

$x = 119.000$, which corresponds to reversing the base and exponent when converting and reversing the value with $x$.
\item \( \text{There is no Real solution to the equation.} \)

Corresponds to believing a negative coefficient within the log equation means there is no Real solution.
\end{enumerate}

\textbf{General Comment:} \textbf{General Comments:} First, get the equation in the form $\log_b{(cx+d)} = a$. Then, convert to $b^a = cx+d$ and solve.
}
\litem{
Solve the equation for $x$ and choose the interval that contains the solution (if it exists).
\[ \log_{4}{(-3x+7)}+5 = 2 \]The solution is \( x = 2.328 \), which is option B.\begin{enumerate}[label=\Alph*.]
\item \( x \in [-4.2, -1.6] \)

$x = -3.000$, which corresponds to ignoring the vertical shift when converting to exponential form.
\item \( x \in [1.2, 5.3] \)

* $x = 2.328$, which is the correct option.
\item \( x \in [-26.3, -22.1] \)

$x = -24.667$, which corresponds to reversing the base and exponent when converting.
\item \( x \in [-31.8, -28.5] \)

$x = -29.333$, which corresponds to reversing the base and exponent when converting and reversing the value with $x$.
\item \( \text{There is no Real solution to the equation.} \)

Corresponds to believing a negative coefficient within the log equation means there is no Real solution.
\end{enumerate}

\textbf{General Comment:} \textbf{General Comments:} First, get the equation in the form $\log_b{(cx+d)} = a$. Then, convert to $b^a = cx+d$ and solve.
}
\litem{
Which of the following intervals describes the Range of the function below?
\[ f(x) = -e^{x+9}+7 \]The solution is \( (-\infty, 7) \), which is option D.\begin{enumerate}[label=\Alph*.]
\item \( (a, \infty), a \in [-7, -5] \)

$(-7, \infty)$, which corresponds to using the negative vertical shift AND flipping the Range interval.
\item \( (-\infty, a], a \in [4, 9] \)

$(-\infty, 7]$, which corresponds to including the endpoint.
\item \( [a, \infty), a \in [-7, -5] \)

$[-7, \infty)$, which corresponds to using the negative vertical shift AND flipping the Range interval AND including the endpoint.
\item \( (-\infty, a), a \in [4, 9] \)

* $(-\infty, 7)$, which is the correct option.
\item \( (-\infty, \infty) \)

This corresponds to confusing range of an exponential function with the domain of an exponential function.
\end{enumerate}

\textbf{General Comment:} \textbf{General Comments}: Domain of a basic exponential function is $(-\infty, \infty)$ while the Range is $(0, \infty)$. We can shift these intervals [and even flip when $a<0$!] to find the new Domain/Range.
}
\litem{
Which of the following intervals describes the Range of the function below?
\[ f(x) = e^{x-1}+9 \]The solution is \( (9, \infty) \), which is option C.\begin{enumerate}[label=\Alph*.]
\item \( [a, \infty), a \in [9, 10] \)

$[9, \infty)$, which corresponds to including the endpoint.
\item \( (-\infty, a), a \in [-9, -8] \)

$(-\infty, -9)$, which corresponds to using the negative vertical shift AND flipping the Range interval.
\item \( (a, \infty), a \in [9, 10] \)

* $(9, \infty)$, which is the correct option.
\item \( (-\infty, a], a \in [-9, -8] \)

$(-\infty, -9]$, which corresponds to using the negative vertical shift AND flipping the Range interval AND including the endpoint.
\item \( (-\infty, \infty) \)

This corresponds to confusing range of an exponential function with the domain of an exponential function.
\end{enumerate}

\textbf{General Comment:} \textbf{General Comments}: Domain of a basic exponential function is $(-\infty, \infty)$ while the Range is $(0, \infty)$. We can shift these intervals [and even flip when $a<0$!] to find the new Domain/Range.
}
\litem{
Solve the equation for $x$ and choose the interval that contains the solution (if it exists).
\[ 4^{-4x-5} = \left(\frac{1}{49}\right)^{2x+4} \]The solution is \( x = -3.858 \), which is option A.\begin{enumerate}[label=\Alph*.]
\item \( x \in [-3.86, -2.86] \)

* $x = -3.858$, which is the correct option.
\item \( x \in [-3.5, 0.5] \)

$x = -1.500$, which corresponds to solving the numerators as equal while ignoring the bases are different.
\item \( x \in [2.02, 6.02] \)

$x = 4.021$, which corresponds to distributing the $\ln(base)$ to the first term of the exponent only.
\item \( x \in [0.44, 3.44] \)

$x = 1.439$, which corresponds to distributing the $\ln(base)$ to the second term of the exponent only.
\item \( \text{There is no Real solution to the equation.} \)

This corresponds to believing there is no solution since the bases are not powers of each other.
\end{enumerate}

\textbf{General Comment:} \textbf{General Comments:} This question was written so that the bases could not be written the same. You will need to take the log of both sides.
}
\litem{
 Solve the equation for $x$ and choose the interval that contains $x$ (if it exists).
\[  18 = \ln{\sqrt[6]{\frac{19}{e^{7x}}}} \]The solution is \( x = -15.008 \), which is option C.\begin{enumerate}[label=\Alph*.]
\item \( x \in [-6.72, -3.72] \)

$x = -4.722$, which corresponds to treating any root as a square root.
\item \( x \in [-3.9, 1.1] \)

$x = -2.898$, which corresponds to thinking you need to take the natural log of on the left before reducing.
\item \( x \in [-16.01, -13.01] \)

* $x = -15.008$, which is the correct option.
\item \( \text{There is no Real solution to the equation.} \)

This corresponds to believing you cannot solve the equation.
\item \( \text{None of the above.} \)

This corresponds to making an unexpected error.
\end{enumerate}

\textbf{General Comment:} \textbf{General Comments}: After using the properties of logarithmic functions to break up the right-hand side, use $\ln(e) = 1$ to reduce the question to a linear function to solve. You can put $\ln(19)$ into a calculator if you are having trouble.
}
\litem{
Which of the following intervals describes the Range of the function below?
\[ f(x) = -\log_2{(x+2)}-7 \]The solution is \( (\infty, \infty) \), which is option E.\begin{enumerate}[label=\Alph*.]
\item \( (-\infty, a), a \in [5, 12] \)

$(-\infty, 7)$, which corresponds to using the using the negative of vertical shift on $(0, \infty)$.
\item \( [a, \infty), a \in [-6, 1] \)

$[-7, \infty)$, which corresponds to using the flipped Domain AND including the endpoint.
\item \( [a, \infty), a \in [2, 4] \)

$[2, \infty)$, which corresponds to using the negative of the horizontal shift AND including the endpoint.
\item \( (-\infty, a), a \in [-13, -3] \)

$(-\infty, -7)$, which corresponds to using the vertical shift while the Range is $(-\infty, \infty)$.
\item \( (-\infty, \infty) \)

*This is the correct option.
\end{enumerate}

\textbf{General Comment:} \textbf{General Comments}: The domain of a basic logarithmic function is $(0, \infty)$ and the Range is $(-\infty, \infty)$. We can use shifts when finding the Domain, but the Range will always be all Real numbers.
}
\litem{
 Solve the equation for $x$ and choose the interval that contains $x$ (if it exists).
\[  20 = \sqrt[7]{\frac{17}{e^{6x}}} \]The solution is \( x = -3.023 \), which is option C.\begin{enumerate}[label=\Alph*.]
\item \( x \in [-23.81, -22.81] \)

$x = -23.806$, which corresponds to thinking you don't need to take the natural log of both sides before reducing, as if the equation already had a natural log on the right side.
\item \( x \in [-2.53, 0.47] \)

$x = -0.526$, which corresponds to treating any root as a square root.
\item \( x \in [-8.02, -2.02] \)

* $x = -3.023$, which is the correct option.
\item \( \text{There is no Real solution to the equation.} \)

This corresponds to believing you cannot solve the equation.
\item \( \text{None of the above.} \)

This corresponds to making an unexpected error.
\end{enumerate}

\textbf{General Comment:} \textbf{General Comments}: After using the properties of logarithmic functions to break up the right-hand side, use $\ln(e) = 1$ to reduce the question to a linear function to solve. You can put $\ln(17)$ into a calculator if you are having trouble.
}
\litem{
Solve the equation for $x$ and choose the interval that contains the solution (if it exists).
\[ 2^{5x-2} = \left(\frac{1}{9}\right)^{2x+3} \]The solution is \( x = -0.662 \), which is option D.\begin{enumerate}[label=\Alph*.]
\item \( x \in [0.67, 4.67] \)

$x = 1.667$, which corresponds to solving the numerators as equal while ignoring the bases are different.
\item \( x \in [-1.74, -0.74] \)

$x = -1.735$, which corresponds to distributing the $\ln(base)$ to the second term of the exponent only.
\item \( x \in [-0.36, 1.64] \)

$x = 0.636$, which corresponds to distributing the $\ln(base)$ to the first term of the exponent only.
\item \( x \in [-1.66, 0.34] \)

* $x = -0.662$, which is the correct option.
\item \( \text{There is no Real solution to the equation.} \)

This corresponds to believing there is no solution since the bases are not powers of each other.
\end{enumerate}

\textbf{General Comment:} \textbf{General Comments:} This question was written so that the bases could not be written the same. You will need to take the log of both sides.
}
\end{enumerate}

\end{document}