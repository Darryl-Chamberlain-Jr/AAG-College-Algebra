
\documentclass{article}[14pt]
\usepackage{multicol, enumerate, enumitem, hyperref, color, soul, setspace, parskip, fancyhdr, amssymb, amsthm, amsmath, bbm, latexsym, units, mathtools}
\everymath{\displaystyle}
\usepackage[headsep=0.5cm,headheight=0cm, left=1 in,right= 1 in,top= 1 in,bottom= 1 in]{geometry}
\pagestyle{fancy}
\lhead{}
\chead{Answer Key for Module\,8\,-\,Logarithmic\,and\,Exponential\,Functions Version B}
\rhead{}
\lfoot{Summer\,C\,2020}
\cfoot{}
\rfoot{}
\begin{document}
\textbf{This key should allow you to understand why you choose the option you did (beyond just getting a question right or wrong). \href{https://xronos.clas.ufl.edu/mac1105spring2020/courseDescriptionAndMisc/Exams/LearningFromResults}{More instructions on how to use this key can be found here}.}

\textbf{If you have a suggestion to make the keys better, \href{https://forms.gle/CZkbZmPbC9XALEE88}{please fill out the short survey here}.}

\textit{Note: This key is auto-generated and may contain issues and/or errors. The keys are reviewed after each exam to ensure grading is done accurately. If there are issues (like duplicate options), they are noted in the offline gradebook. The keys are a work-in-progress to give students as many resources to improve as possible.}

\rule{\textwidth}{0.4pt}

36. Solve the equation for $x$ and choose the interval that contains the solution (if it exists).
$$ 3^{2x+2} = 64^{3x-5} $$ 
The solution is $ x = 2.237 $ 

\begin{enumerate}[label=\Alph*.] 
\item $ x \in [21.6, 24.2] $ 

 $x = 22.992$, which corresponds to distributing the $\ln(base)$ to the second term of the exponent only. 
\item $ x \in [0.7, 3] $ 

 * $x = 2.237$, which is the correct option. 
\item $ x \in [-0.1, 1] $ 

 $x = 0.681$, which corresponds to distributing the $\ln(base)$ to the first term of the exponent only. 
\item $ x \in [6.2, 7.7] $ 

 $x = 7.000$, which corresponds to solving the numerators as equal while ignoring the bases are different. 
\item $ \text{There is no Real solution to the equation.} $ 

 This corresponds to believing there is no solution since the bases are not powers of each other. 
\end{enumerate} 
 
\textbf{General Comments:} This question was written so that the bases could not be written the same. You will need to take the log of both sides.

-----------------------------------------------

37. Which of the following intervals describes the Domain of the function below?
$$ f(x) = \log_2{(x-9)}-1 $$ 
The solution is $ (9, \infty) $ 

\begin{enumerate}[label=\Alph*.] 
\item $ [a, \infty), a \in [-1.9, 0.6] $ 

 $[-1, \infty)$, which corresponds to using the vertical shift when shifting the Domain AND including the endpoint. 
\item $ (-\infty, a), a \in [-13, -7.6] $ 

 $(-\infty, -9)$, which corresponds to flipping the Domain. Remember: the general for is $a*\log(x-h)+k$, \textbf{where $a$ does not affect the domain}. 
\item $ (-\infty, a], a \in [0.3, 2.6] $ 

 $(-\infty, 1]$, which corresponds to using the negative vertical shift AND including the endpoint AND flipping the domain. 
\item $ (a, \infty), a \in [8.9, 11] $ 

 * $(9, \infty)$, which is the correct option. 
\item $ (-\infty, \infty) $ 

 This corresponds to thinking of the range of the log function (or the domain of the exponential function). 
\end{enumerate} 
 
\textbf{General Comments}: The domain of a basic logarithmic function is $(0, \infty)$ and the Range is $(-\infty, \infty)$. We can use shifts when finding the Domain, but the Range will always be all Real numbers.

-----------------------------------------------

38. Which of the following intervals describes the Domain of the function below?
$$ f(x) = -e^{x-2}+3 $$ 
The solution is $ (-\infty, \infty) $ 

\begin{enumerate}[label=\Alph*.] 
\item $ (-\infty, a], a \in [1, 11] $ 

 $(-\infty, 3]$, which corresponds to using the correct vertical shift *if we wanted the Range* AND including the endpoint. 
\item $ [a, \infty), a \in [-4, -2] $ 

 $[-3, \infty)$, which corresponds to using the negative vertical shift AND flipping the Range interval AND including the endpoint. 
\item $ (a, \infty), a \in [-4, -2] $ 

 $(-3, \infty)$, which corresponds to using the negative vertical shift AND flipping the Range interval. 
\item $ (-\infty, a), a \in [1, 11] $ 

 $(-\infty, 3)$, which corresponds to using the correct vertical shift *if we wanted the Range*. 
\item $ (-\infty, \infty) $ 

 * This is the correct option. 
\end{enumerate} 
 
\textbf{General Comments}: Domain of a basic exponential function is $(-\infty, \infty)$ while the Range is $(0, \infty)$. We can shift these intervals [and even flip when $a<0$!] to find the new Domain/Range.

-----------------------------------------------

39. Solve the equation for $x$ and choose the interval that contains the solution (if it exists).
$$ \log_{2}{(-3x+5)}+5 = 2 $$ 
The solution is $ x = 1.625 $ 

\begin{enumerate}[label=\Alph*.] 
\item $ x \in [-2.48, 0.32] $ 

 $x = -1.333$, which corresponds to reversing the base and exponent when converting. 
\item $ x \in [1.54, 2.32] $ 

 * $x = 1.625$, which is the correct option. 
\item $ x \in [-5.73, -3.56] $ 

 $x = -4.667$, which corresponds to reversing the base and exponent when converting and reversing the value with $x$. 
\item $ x \in [0.14, 0.38] $ 

 $x = 0.333$, which corresponds to ignoring the vertical shift when converting to exponential form. 
\item $ \text{There is no Real solution to the equation.} $ 

 Corresponds to believing a negative coefficient within the log equation means there is no Real solution. 
\end{enumerate} 
 
\textbf{General Comments:} First, get the equation in the form $\log_b{(cx+d)} = a$. Then, convert to $b^a = cx+d$ and solve.

-----------------------------------------------

40.  Solve the equation for $x$ and choose the interval that contains $x$ (if it exists).
$$  24 = \sqrt[3]{\frac{12}{e^{4x}}} $$ 
The solution is $ x = -1.762, \text{ which does not fit in any of the interval options.} $ 

\begin{enumerate}[label=\Alph*.] 
\item $ x \in [-1.57, -0.64] $ 

 $x = -0.968$, which corresponds to treating any root as a square root. 
\item $ x \in [0.77, 2.07] $ 

 $x = 1.762$, which is the negative of the correct solution. 
\item $ x \in [-19.42, -18.32] $ 

 $x = -18.621$, which corresponds to thinking you don't need to take the natural log of both sides before reducing, as if the right side already has a natural log. 
\item $ \text{There is no Real solution to the equation.} $ 

 This corresponds to believing you cannot solve the equation. 
\item $ \text{None of the above.} $ 

 * $x = -1.762$ is the correct solution and does not fit in any of the other intervals. 
\end{enumerate} 
 
\textbf{General Comments}: After using the properties of logarithmic functions to break up the right-hand side, use $\ln(e) = 1$ to reduce the question to a linear function to solve. You can put $\ln(12)$ into a calculator if you are having trouble.

-----------------------------------------------


\end{document}

