\documentclass{extbook}[14pt]
\usepackage{multicol, enumerate, enumitem, hyperref, color, soul, setspace, parskip, fancyhdr, amssymb, amsthm, amsmath, bbm, latexsym, units, mathtools}
\everymath{\displaystyle}
\usepackage[headsep=0.5cm,headheight=0cm, left=1 in,right= 1 in,top= 1 in,bottom= 1 in]{geometry}
\usepackage{dashrule}  % Package to use the command below to create lines between items
\newcommand{\litem}[1]{\item #1

\rule{\textwidth}{0.4pt}}
\pagestyle{fancy}
\lhead{}
\chead{Answer Key for Progress Quiz 4 Version B}
\rhead{}
\lfoot{6286-1986}
\cfoot{}
\rfoot{Fall 2020}
\begin{document}
\textbf{This key should allow you to understand why you choose the option you did (beyond just getting a question right or wrong). \href{https://xronos.clas.ufl.edu/mac1105spring2020/courseDescriptionAndMisc/Exams/LearningFromResults}{More instructions on how to use this key can be found here}.}

\textbf{If you have a suggestion to make the keys better, \href{https://forms.gle/CZkbZmPbC9XALEE88}{please fill out the short survey here}.}

\textit{Note: This key is auto-generated and may contain issues and/or errors. The keys are reviewed after each exam to ensure grading is done accurately. If there are issues (like duplicate options), they are noted in the offline gradebook. The keys are a work-in-progress to give students as many resources to improve as possible.}

\rule{\textwidth}{0.4pt}

\begin{enumerate}\litem{
Solve the equation for $x$ and choose the interval that contains the solution (if it exists).
\[ 3^{-2x-3} = \left(\frac{1}{25}\right)^{-5x+5} \]
The solution is \( x = 0.700 \), which is option A.\begin{enumerate}[label=\Alph*.]
\item \( x \in [0.1, 1.5] \)

* $x = 0.700$, which is the correct option.
\item \( x \in [-0.9, 0.5] \)

$x = -0.437$, which corresponds to distributing the $\ln(base)$ to the first term of the exponent only.
\item \( x \in [-5.8, -4.2] \)

$x = -4.266$, which corresponds to distributing the $\ln(base)$ to the second term of the exponent only.
\item \( x \in [2.2, 3.4] \)

$x = 2.667$, which corresponds to solving the numerators as equal while ignoring the bases are different.
\item \( \text{There is no Real solution to the equation.} \)

This corresponds to believing there is no solution since the bases are not powers of each other.
\end{enumerate}

\textbf{General Comment:} \textbf{General Comments:} This question was written so that the bases could not be written the same. You will need to take the log of both sides.
}
\litem{
Solve the equation for $x$ and choose the interval that contains the solution (if it exists).
\[ 5^{5x+3} = 16^{3x-3} \]
The solution is \( x = 48.585 \), which is option D.\begin{enumerate}[label=\Alph*.]
\item \( x \in [-9.57, -5.57] \)

$x = -6.573$, which corresponds to distributing the $\ln(base)$ to the second term of the exponent only.
\item \( x \in [21.17, 24.17] \)

$x = 22.175$, which corresponds to distributing the $\ln(base)$ to the first term of the exponent only.
\item \( x \in [-3, 1] \)

$x = -3.000$, which corresponds to solving the numerators as equal while ignoring the bases are different.
\item \( x \in [46.59, 49.59] \)

* $x = 48.585$, which is the correct option.
\item \( \text{There is no Real solution to the equation.} \)

This corresponds to believing there is no solution since the bases are not powers of each other.
\end{enumerate}

\textbf{General Comment:} \textbf{General Comments:} This question was written so that the bases could not be written the same. You will need to take the log of both sides.
}
\litem{
Which of the following intervals describes the Domain of the function below?
\[ f(x) = e^{x-5}+3 \]
The solution is \( (-\infty, \infty) \), which is option E.\begin{enumerate}[label=\Alph*.]
\item \( (-\infty, a], a \in [0, 7] \)

$(-\infty, 3]$, which corresponds to using the correct vertical shift *if we wanted the Range* AND including the endpoint.
\item \( (-\infty, a), a \in [0, 7] \)

$(-\infty, 3)$, which corresponds to using the correct vertical shift *if we wanted the Range*.
\item \( (a, \infty), a \in [-7, 2] \)

$(-3, \infty)$, which corresponds to using the negative vertical shift AND flipping the Range interval.
\item \( [a, \infty), a \in [-7, 2] \)

$[-3, \infty)$, which corresponds to using the negative vertical shift AND flipping the Range interval AND including the endpoint.
\item \( (-\infty, \infty) \)

* This is the correct option.
\end{enumerate}

\textbf{General Comment:} \textbf{General Comments}: Domain of a basic exponential function is $(-\infty, \infty)$ while the Range is $(0, \infty)$. We can shift these intervals [and even flip when $a<0$!] to find the new Domain/Range.
}
\litem{
Which of the following intervals describes the Range of the function below?
\[ f(x) = -e^{x+7}+3 \]
The solution is \( (-\infty, 3) \), which is option A.\begin{enumerate}[label=\Alph*.]
\item \( (-\infty, a), a \in [2, 7] \)

* $(-\infty, 3)$, which is the correct option.
\item \( (a, \infty), a \in [-3, -1] \)

$(-3, \infty)$, which corresponds to using the negative vertical shift AND flipping the Range interval.
\item \( (-\infty, a], a \in [2, 7] \)

$(-\infty, 3]$, which corresponds to including the endpoint.
\item \( [a, \infty), a \in [-3, -1] \)

$[-3, \infty)$, which corresponds to using the negative vertical shift AND flipping the Range interval AND including the endpoint.
\item \( (-\infty, \infty) \)

This corresponds to confusing range of an exponential function with the domain of an exponential function.
\end{enumerate}

\textbf{General Comment:} \textbf{General Comments}: Domain of a basic exponential function is $(-\infty, \infty)$ while the Range is $(0, \infty)$. We can shift these intervals [and even flip when $a<0$!] to find the new Domain/Range.
}
\litem{
Which of the following intervals describes the Domain of the function below?
\[ f(x) = \log_2{(x-1)}-2 \]
The solution is \( (1, \infty) \), which is option D.\begin{enumerate}[label=\Alph*.]
\item \( (-\infty, a], a \in [1.26, 2.29] \)

$(-\infty, 2]$, which corresponds to using the negative vertical shift AND including the endpoint AND flipping the domain.
\item \( [a, \infty), a \in [-2.12, -1.53] \)

$[-2, \infty)$, which corresponds to using the vertical shift when shifting the Domain AND including the endpoint.
\item \( (-\infty, a), a \in [-1.11, -0.31] \)

$(-\infty, -1)$, which corresponds to flipping the Domain. Remember: the general for is $a*\log(x-h)+k$, \textbf{where $a$ does not affect the domain}.
\item \( (a, \infty), a \in [0.79, 1.14] \)

* $(1, \infty)$, which is the correct option.
\item \( (-\infty, \infty) \)

This corresponds to thinking of the range of the log function (or the domain of the exponential function).
\end{enumerate}

\textbf{General Comment:} \textbf{General Comments}: The domain of a basic logarithmic function is $(0, \infty)$ and the Range is $(-\infty, \infty)$. We can use shifts when finding the Domain, but the Range will always be all Real numbers.
}
\litem{
 Solve the equation for $x$ and choose the interval that contains $x$ (if it exists).
\[  10 = \sqrt[5]{\frac{15}{e^{9x}}} \]
The solution is \( x = -0.978 \), which is option B.\begin{enumerate}[label=\Alph*.]
\item \( x \in [-0.4, 0.52] \)

$x = -0.211$, which corresponds to treating any root as a square root.
\item \( x \in [-1.55, -0.88] \)

* $x = -0.978$, which is the correct option.
\item \( x \in [-6.19, -5.68] \)

$x = -5.856$, which corresponds to thinking you don't need to take the natural log of both sides before reducing, as if the equation already had a natural log on the right side.
\item \( \text{There is no Real solution to the equation.} \)

This corresponds to believing you cannot solve the equation.
\item \( \text{None of the above.} \)

This corresponds to making an unexpected error.
\end{enumerate}

\textbf{General Comment:} \textbf{General Comments}: After using the properties of logarithmic functions to break up the right-hand side, use $\ln(e) = 1$ to reduce the question to a linear function to solve. You can put $\ln(15)$ into a calculator if you are having trouble.
}
\litem{
Which of the following intervals describes the Domain of the function below?
\[ f(x) = \log_2{(x-8)}-7 \]
The solution is \( (8, \infty) \), which is option B.\begin{enumerate}[label=\Alph*.]
\item \( (-\infty, a), a \in [-8.82, -7.32] \)

$(-\infty, -8)$, which corresponds to flipping the Domain. Remember: the general for is $a*\log(x-h)+k$, \textbf{where $a$ does not affect the domain}.
\item \( (a, \infty), a \in [7.39, 8.26] \)

* $(8, \infty)$, which is the correct option.
\item \( (-\infty, a], a \in [6.78, 7.05] \)

$(-\infty, 7]$, which corresponds to using the negative vertical shift AND including the endpoint AND flipping the domain.
\item \( [a, \infty), a \in [-7.64, -5.02] \)

$[-7, \infty)$, which corresponds to using the vertical shift when shifting the Domain AND including the endpoint.
\item \( (-\infty, \infty) \)

This corresponds to thinking of the range of the log function (or the domain of the exponential function).
\end{enumerate}

\textbf{General Comment:} \textbf{General Comments}: The domain of a basic logarithmic function is $(0, \infty)$ and the Range is $(-\infty, \infty)$. We can use shifts when finding the Domain, but the Range will always be all Real numbers.
}
\litem{
Solve the equation for $x$ and choose the interval that contains the solution (if it exists).
\[ \log_{5}{(4x+7)}+4 = 2 \]
The solution is \( x = -1.740 \), which is option C.\begin{enumerate}[label=\Alph*.]
\item \( x \in [-7.3, -3.2] \)

$x = -6.250$, which corresponds to reversing the base and exponent when converting and reversing the value with $x$.
\item \( x \in [1.8, 5.3] \)

$x = 4.500$, which corresponds to ignoring the vertical shift when converting to exponential form.
\item \( x \in [-2.7, 1.1] \)

* $x = -1.740$, which is the correct option.
\item \( x \in [-11.8, -8.7] \)

$x = -9.750$, which corresponds to reversing the base and exponent when converting.
\item \( \text{There is no Real solution to the equation.} \)

Corresponds to believing a negative coefficient within the log equation means there is no Real solution.
\end{enumerate}

\textbf{General Comment:} \textbf{General Comments:} First, get the equation in the form $\log_b{(cx+d)} = a$. Then, convert to $b^a = cx+d$ and solve.
}
\litem{
Solve the equation for $x$ and choose the interval that contains the solution (if it exists).
\[ \log_{5}{(-4x+6)}+5 = 3 \]
The solution is \( x = 1.490 \), which is option A.\begin{enumerate}[label=\Alph*.]
\item \( x \in [0.4, 4.6] \)

* $x = 1.490$, which is the correct option.
\item \( x \in [-31.4, -28.2] \)

$x = -29.750$, which corresponds to ignoring the vertical shift when converting to exponential form.
\item \( x \in [7, 12.5] \)

$x = 9.500$, which corresponds to reversing the base and exponent when converting.
\item \( x \in [4.8, 8.1] \)

$x = 6.500$, which corresponds to reversing the base and exponent when converting and reversing the value with $x$.
\item \( \text{There is no Real solution to the equation.} \)

Corresponds to believing a negative coefficient within the log equation means there is no Real solution.
\end{enumerate}

\textbf{General Comment:} \textbf{General Comments:} First, get the equation in the form $\log_b{(cx+d)} = a$. Then, convert to $b^a = cx+d$ and solve.
}
\litem{
 Solve the equation for $x$ and choose the interval that contains $x$ (if it exists).
\[  15 = \ln{\sqrt[7]{\frac{16}{e^{5x}}}} \]
The solution is \( x = -20.445, \text{ which does not fit in any of the interval options.} \), which is option E.\begin{enumerate}[label=\Alph*.]
\item \( x \in [-4.9, -4.1] \)

$x = -4.346$, which corresponds to thinking you need to take the natural log of the left side before reducing.
\item \( x \in [-6.4, -4.8] \)

$x = -5.445$, which corresponds to treating any root as a square root.
\item \( x \in [19.8, 21] \)

$x = 20.445$, which is the negative of the correct solution.
\item \( \text{There is no Real solution to the equation.} \)

This corresponds to believing you cannot solve the equation.
\item \( \text{None of the above.} \)

*$x = -20.445$ is the correct solution and does not fit in any of the other intervals.
\end{enumerate}

\textbf{General Comment:} \textbf{General Comments}: After using the properties of logarithmic functions to break up the right-hand side, use $\ln(e) = 1$ to reduce the question to a linear function to solve. You can put $\ln(16)$ into a calculator if you are having trouble.
}
\end{enumerate}

\end{document}