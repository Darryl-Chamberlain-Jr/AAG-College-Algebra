\documentclass{extbook}[14pt]
\usepackage{multicol, enumerate, enumitem, hyperref, color, soul, setspace, parskip, fancyhdr, amssymb, amsthm, amsmath, bbm, latexsym, units, mathtools}
\everymath{\displaystyle}
\usepackage[headsep=0.5cm,headheight=0cm, left=1 in,right= 1 in,top= 1 in,bottom= 1 in]{geometry}
\usepackage{dashrule}  % Package to use the command below to create lines between items
\newcommand{\litem}[1]{\item #1

\rule{\textwidth}{0.4pt}}
\pagestyle{fancy}
\lhead{}
\chead{Answer Key for Progress Quiz 9 Version B}
\rhead{}
\lfoot{8590-6105}
\cfoot{}
\rfoot{Fall 2020}
\begin{document}
\textbf{This key should allow you to understand why you choose the option you did (beyond just getting a question right or wrong). \href{https://xronos.clas.ufl.edu/mac1105spring2020/courseDescriptionAndMisc/Exams/LearningFromResults}{More instructions on how to use this key can be found here}.}

\textbf{If you have a suggestion to make the keys better, \href{https://forms.gle/CZkbZmPbC9XALEE88}{please fill out the short survey here}.}

\textit{Note: This key is auto-generated and may contain issues and/or errors. The keys are reviewed after each exam to ensure grading is done accurately. If there are issues (like duplicate options), they are noted in the offline gradebook. The keys are a work-in-progress to give students as many resources to improve as possible.}

\rule{\textwidth}{0.4pt}

\begin{enumerate}\litem{
Choose the \textbf{smallest} set of Real numbers that the number below belongs to.
\[ -\sqrt{\frac{1848}{12}} \]

The solution is \( \text{Irrational} \), which is option A.\begin{enumerate}[label=\Alph*.]
\item \( \text{Irrational} \)

* This is the correct option!
\item \( \text{Integer} \)

These are the negative and positive counting numbers (..., -3, -2, -1, 0, 1, 2, 3, ...)
\item \( \text{Not a Real number} \)

These are Nonreal Complex numbers \textbf{OR} things that are not numbers (e.g., dividing by 0).
\item \( \text{Whole} \)

These are the counting numbers with 0 (0, 1, 2, 3, ...)
\item \( \text{Rational} \)

These are numbers that can be written as fraction of Integers (e.g., -2/3)
\end{enumerate}

\textbf{General Comment:} First, you \textbf{NEED} to simplify the expression. This question simplifies to $-\sqrt{154}$. 
 
 Be sure you look at the simplified fraction and not just the decimal expansion. Numbers such as 13, 17, and 19 provide \textbf{long but repeating/terminating decimal expansions!} 
 
 The only ways to *not* be a Real number are: dividing by 0 or taking the square root of a negative number. 
 
 Irrational numbers are more than just square root of 3: adding or subtracting values from square root of 3 is also irrational.
}
\litem{
Simplify the expression below into the form $a+bi$. Then, choose the intervals that $a$ and $b$ belong to.
\[ (-8 - 9 i)(-10 - 7 i) \]

The solution is \( 17 + 146 i \), which is option A.\begin{enumerate}[label=\Alph*.]
\item \( a \in [10, 22] \text{ and } b \in [143, 147] \)

* $17 + 146 i$, which is the correct option.
\item \( a \in [10, 22] \text{ and } b \in [-149, -142] \)

 $17 - 146 i$, which corresponds to adding a minus sign in both terms.
\item \( a \in [143, 146] \text{ and } b \in [-34, -33] \)

 $143 - 34 i$, which corresponds to adding a minus sign in the first term.
\item \( a \in [79, 81] \text{ and } b \in [61, 66] \)

 $80 + 63 i$, which corresponds to just multiplying the real terms to get the real part of the solution and the coefficients in the complex terms to get the complex part.
\item \( a \in [143, 146] \text{ and } b \in [26, 36] \)

 $143 + 34 i$, which corresponds to adding a minus sign in the second term.
\end{enumerate}

\textbf{General Comment:} You can treat $i$ as a variable and distribute. Just remember that $i^2=-1$, so you can continue to reduce after you distribute.
}
\litem{
Choose the \textbf{smallest} set of Complex numbers that the number below belongs to.
\[ \sqrt{\frac{720}{9}}+9i^2 \]

The solution is \( \text{Irrational} \), which is option B.\begin{enumerate}[label=\Alph*.]
\item \( \text{Rational} \)

These are numbers that can be written as fraction of Integers (e.g., -2/3 + 5)
\item \( \text{Irrational} \)

* This is the correct option!
\item \( \text{Nonreal Complex} \)

This is a Complex number $(a+bi)$ that is not Real (has $i$ as part of the number).
\item \( \text{Not a Complex Number} \)

This is not a number. The only non-Complex number we know is dividing by 0 as this is not a number!
\item \( \text{Pure Imaginary} \)

This is a Complex number $(a+bi)$ that \textbf{only} has an imaginary part like $2i$.
\end{enumerate}

\textbf{General Comment:} Be sure to simplify $i^2 = -1$. This may remove the imaginary portion for your number. If you are having trouble, you may want to look at the \textit{Subgroups of the Real Numbers} section.
}
\litem{
Simplify the expression below into the form $a+bi$. Then, choose the intervals that $a$ and $b$ belong to.
\[ \frac{27 + 44 i}{8 - 7 i} \]

The solution is \( -0.81  + 4.79 i \), which is option E.\begin{enumerate}[label=\Alph*.]
\item \( a \in [-92.5, -91.5] \text{ and } b \in [3.5, 5] \)

 $-92.00  + 4.79 i$, which corresponds to forgetting to multiply the conjugate by the numerator and using a plus instead of a minus in the denominator.
\item \( a \in [4, 5.5] \text{ and } b \in [0.5, 2] \)

 $4.64  + 1.44 i$, which corresponds to forgetting to multiply the conjugate by the numerator and not computing the conjugate correctly.
\item \( a \in [-1.5, 0.5] \text{ and } b \in [540.5, 541.5] \)

 $-0.81  + 541.00 i$, which corresponds to forgetting to multiply the conjugate by the numerator.
\item \( a \in [2.5, 4] \text{ and } b \in [-7, -6] \)

 $3.38  - 6.29 i$, which corresponds to just dividing the first term by the first term and the second by the second.
\item \( a \in [-1.5, 0.5] \text{ and } b \in [3.5, 5] \)

* $-0.81  + 4.79 i$, which is the correct option.
\end{enumerate}

\textbf{General Comment:} Multiply the numerator and denominator by the *conjugate* of the denominator, then simplify. For example, if we have $2+3i$, the conjugate is $2-3i$.
}
\litem{
Simplify the expression below and choose the interval the simplification is contained within.
\[ 13 - 12^2 + 20 \div 17 * 19 \div 8 \]

The solution is \( -128.206 \), which is option C.\begin{enumerate}[label=\Alph*.]
\item \( [153.3, 158.4] \)

 157.008, which corresponds to two Order of Operations errors.
\item \( [158.3, 161.4] \)

 159.794, which corresponds to an Order of Operations error: multiplying by negative before squaring. For example: $(-3)^2 \neq -3^2$
\item \( [-129.3, -128.1] \)

* -128.206, this is the correct option
\item \( [-133, -130.7] \)

 -130.992, which corresponds to an Order of Operations error: not reading left-to-right for multiplication/division.
\item \( \text{None of the above} \)

 You may have gotten this by making an unanticipated error. If you got a value that is not any of the others, please let the coordinator know so they can help you figure out what happened.
\end{enumerate}

\textbf{General Comment:} While you may remember (or were taught) PEMDAS is done in order, it is actually done as P/E/MD/AS. When we are at MD or AS, we read left to right.
}
\litem{
Simplify the expression below into the form $a+bi$. Then, choose the intervals that $a$ and $b$ belong to.
\[ (10 + 8 i)(-3 - 9 i) \]

The solution is \( 42 - 114 i \), which is option E.\begin{enumerate}[label=\Alph*.]
\item \( a \in [-105, -100] \text{ and } b \in [-66, -63] \)

 $-102 - 66 i$, which corresponds to adding a minus sign in the first term.
\item \( a \in [-105, -100] \text{ and } b \in [63, 70] \)

 $-102 + 66 i$, which corresponds to adding a minus sign in the second term.
\item \( a \in [37, 44] \text{ and } b \in [114, 117] \)

 $42 + 114 i$, which corresponds to adding a minus sign in both terms.
\item \( a \in [-30, -26] \text{ and } b \in [-73, -67] \)

 $-30 - 72 i$, which corresponds to just multiplying the real terms to get the real part of the solution and the coefficients in the complex terms to get the complex part.
\item \( a \in [37, 44] \text{ and } b \in [-119, -112] \)

* $42 - 114 i$, which is the correct option.
\end{enumerate}

\textbf{General Comment:} You can treat $i$ as a variable and distribute. Just remember that $i^2=-1$, so you can continue to reduce after you distribute.
}
\litem{
Choose the \textbf{smallest} set of Complex numbers that the number below belongs to.
\[ \sqrt{\frac{850}{10}}+9i^2 \]

The solution is \( \text{Irrational} \), which is option C.\begin{enumerate}[label=\Alph*.]
\item \( \text{Nonreal Complex} \)

This is a Complex number $(a+bi)$ that is not Real (has $i$ as part of the number).
\item \( \text{Not a Complex Number} \)

This is not a number. The only non-Complex number we know is dividing by 0 as this is not a number!
\item \( \text{Irrational} \)

* This is the correct option!
\item \( \text{Rational} \)

These are numbers that can be written as fraction of Integers (e.g., -2/3 + 5)
\item \( \text{Pure Imaginary} \)

This is a Complex number $(a+bi)$ that \textbf{only} has an imaginary part like $2i$.
\end{enumerate}

\textbf{General Comment:} Be sure to simplify $i^2 = -1$. This may remove the imaginary portion for your number. If you are having trouble, you may want to look at the \textit{Subgroups of the Real Numbers} section.
}
\litem{
Choose the \textbf{smallest} set of Real numbers that the number below belongs to.
\[ -\sqrt{\frac{585}{5}} \]

The solution is \( \text{Irrational} \), which is option D.\begin{enumerate}[label=\Alph*.]
\item \( \text{Rational} \)

These are numbers that can be written as fraction of Integers (e.g., -2/3)
\item \( \text{Not a Real number} \)

These are Nonreal Complex numbers \textbf{OR} things that are not numbers (e.g., dividing by 0).
\item \( \text{Whole} \)

These are the counting numbers with 0 (0, 1, 2, 3, ...)
\item \( \text{Irrational} \)

* This is the correct option!
\item \( \text{Integer} \)

These are the negative and positive counting numbers (..., -3, -2, -1, 0, 1, 2, 3, ...)
\end{enumerate}

\textbf{General Comment:} First, you \textbf{NEED} to simplify the expression. This question simplifies to $-\sqrt{117}$. 
 
 Be sure you look at the simplified fraction and not just the decimal expansion. Numbers such as 13, 17, and 19 provide \textbf{long but repeating/terminating decimal expansions!} 
 
 The only ways to *not* be a Real number are: dividing by 0 or taking the square root of a negative number. 
 
 Irrational numbers are more than just square root of 3: adding or subtracting values from square root of 3 is also irrational.
}
\litem{
Simplify the expression below and choose the interval the simplification is contained within.
\[ 6 - 12 \div 4 * 19 - (9 * 7) \]

The solution is \( -114.000 \), which is option B.\begin{enumerate}[label=\Alph*.]
\item \( [67.84, 73.84] \)

 68.842, which corresponds to not distributing addition and subtraction correctly.
\item \( [-117, -111] \)

* -114.000, which is the correct option.
\item \( [-421, -417] \)

 -420.000, which corresponds to not distributing a negative correctly.
\item \( [-59.16, -52.16] \)

 -57.158, which corresponds to an Order of Operations error: not reading left-to-right for multiplication/division.
\item \( \text{None of the above} \)

 You may have gotten this by making an unanticipated error. If you got a value that is not any of the others, please let the coordinator know so they can help you figure out what happened.
\end{enumerate}

\textbf{General Comment:} While you may remember (or were taught) PEMDAS is done in order, it is actually done as P/E/MD/AS. When we are at MD or AS, we read left to right.
}
\litem{
Simplify the expression below into the form $a+bi$. Then, choose the intervals that $a$ and $b$ belong to.
\[ \frac{-72 + 77 i}{2 - 3 i} \]

The solution is \( -28.85  - 4.77 i \), which is option C.\begin{enumerate}[label=\Alph*.]
\item \( a \in [-38, -35.5] \text{ and } b \in [-27, -24.5] \)

 $-36.00  - 25.67 i$, which corresponds to just dividing the first term by the first term and the second by the second.
\item \( a \in [-30, -27] \text{ and } b \in [-62.5, -60.5] \)

 $-28.85  - 62.00 i$, which corresponds to forgetting to multiply the conjugate by the numerator.
\item \( a \in [-30, -27] \text{ and } b \in [-5, -3] \)

* $-28.85  - 4.77 i$, which is the correct option.
\item \( a \in [6.5, 7] \text{ and } b \in [28, 29] \)

 $6.69  + 28.46 i$, which corresponds to forgetting to multiply the conjugate by the numerator and not computing the conjugate correctly.
\item \( a \in [-376, -374] \text{ and } b \in [-5, -3] \)

 $-375.00  - 4.77 i$, which corresponds to forgetting to multiply the conjugate by the numerator and using a plus instead of a minus in the denominator.
\end{enumerate}

\textbf{General Comment:} Multiply the numerator and denominator by the *conjugate* of the denominator, then simplify. For example, if we have $2+3i$, the conjugate is $2-3i$.
}
\end{enumerate}

\end{document}