\documentclass{extbook}[14pt]
\usepackage{multicol, enumerate, enumitem, hyperref, color, soul, setspace, parskip, fancyhdr, amssymb, amsthm, amsmath, latexsym, units, mathtools}
\everymath{\displaystyle}
\usepackage[headsep=0.5cm,headheight=0cm, left=1 in,right= 1 in,top= 1 in,bottom= 1 in]{geometry}
\usepackage{dashrule}  % Package to use the command below to create lines between items
\newcommand{\litem}[1]{\item #1

\rule{\textwidth}{0.4pt}}
\pagestyle{fancy}
\lhead{}
\chead{Answer Key for Module1 Version B}
\rhead{}
\lfoot{8000-1344}
\cfoot{}
\rfoot{testing}
\begin{document}
\textbf{This key should allow you to understand why you choose the option you did (beyond just getting a question right or wrong). \href{https://xronos.clas.ufl.edu/mac1105spring2020/courseDescriptionAndMisc/Exams/LearningFromResults}{More instructions on how to use this key can be found here}.}

\textbf{If you have a suggestion to make the keys better, \href{https://forms.gle/CZkbZmPbC9XALEE88}{please fill out the short survey here}.}

\textit{Note: This key is auto-generated and may contain issues and/or errors. The keys are reviewed after each exam to ensure grading is done accurately. If there are issues (like duplicate options), they are noted in the offline gradebook. The keys are a work-in-progress to give students as many resources to improve as possible.}

\rule{\textwidth}{0.4pt}

\begin{enumerate}\litem{
Choose the \textbf{smallest} set of Real numbers that the number below belongs to.
\[ \sqrt{\frac{3969}{81}} \]The solution is \( \text{Whole} \), which is option C.\begin{enumerate}[label=\Alph*.]
\item \( \text{Irrational} \)

These cannot be written as a fraction of Integers.
\item \( \text{Integer} \)

These are the negative and positive counting numbers (..., -3, -2, -1, 0, 1, 2, 3, ...)
\item \( \text{Whole} \)

* This is the correct option!
\item \( \text{Rational} \)

These are numbers that can be written as fraction of Integers (e.g., -2/3)
\item \( \text{Not a Real number} \)

These are Nonreal Complex numbers \textbf{OR} things that are not numbers (e.g., dividing by 0).
\end{enumerate}

\textbf{General Comment:} First, you \textbf{NEED} to simplify the expression. This question simplifies to $63$. 
 
 Be sure you look at the simplified fraction and not just the decimal expansion. Numbers such as 13, 17, and 19 provide \textbf{long but repeating/terminating decimal expansions!} 
 
 The only ways to *not* be a Real number are: dividing by 0 or taking the square root of a negative number. 
 
 Irrational numbers are more than just square root of 3: adding or subtracting values from square root of 3 is also irrational.
}
\litem{
Simplify the expression below into the form $a+bi$. Then, choose the intervals that $a$ and $b$ belong to.
\[ (2 - 7 i)(5 - 9 i) \]The solution is \( -53 - 53 i \), which is option A.\begin{enumerate}[label=\Alph*.]
\item \( a \in [-56, -48] \text{ and } b \in [-57, -52] \)

* $-53 - 53 i$, which is the correct option.
\item \( a \in [4, 17] \text{ and } b \in [58, 66] \)

 $10 + 63 i$, which corresponds to just multiplying the real terms to get the real part of the solution and the coefficients in the complex terms to get the complex part.
\item \( a \in [69, 75] \text{ and } b \in [-20, -12] \)

 $73 - 17 i$, which corresponds to adding a minus sign in the second term.
\item \( a \in [-56, -48] \text{ and } b \in [49, 57] \)

 $-53 + 53 i$, which corresponds to adding a minus sign in both terms.
\item \( a \in [69, 75] \text{ and } b \in [16, 18] \)

 $73 + 17 i$, which corresponds to adding a minus sign in the first term.
\end{enumerate}

\textbf{General Comment:} You can treat $i$ as a variable and distribute. Just remember that $i^2=-1$, so you can continue to reduce after you distribute.
}
\litem{
Simplify the expression below into the form $a+bi$. Then, choose the intervals that $a$ and $b$ belong to.
\[ \frac{54 + 44 i}{-7 - 5 i} \]The solution is \( -8.08  - 0.51 i \), which is option C.\begin{enumerate}[label=\Alph*.]
\item \( a \in [-8.2, -7.75] \text{ and } b \in [-38.5, -37] \)

 $-8.08  - 38.00 i$, which corresponds to forgetting to multiply the conjugate by the numerator.
\item \( a \in [-598.35, -597.5] \text{ and } b \in [-1.5, 0] \)

 $-598.00  - 0.51 i$, which corresponds to forgetting to multiply the conjugate by the numerator and using a plus instead of a minus in the denominator.
\item \( a \in [-8.2, -7.75] \text{ and } b \in [-1.5, 0] \)

* $-8.08  - 0.51 i$, which is the correct option.
\item \( a \in [-2.55, -1.6] \text{ and } b \in [-8.5, -7] \)

 $-2.14  - 7.81 i$, which corresponds to forgetting to multiply the conjugate by the numerator and not computing the conjugate correctly.
\item \( a \in [-7.8, -7.65] \text{ and } b \in [-10, -8.5] \)

 $-7.71  - 8.80 i$, which corresponds to just dividing the first term by the first term and the second by the second.
\end{enumerate}

\textbf{General Comment:} Multiply the numerator and denominator by the *conjugate* of the denominator, then simplify. For example, if we have $2+3i$, the conjugate is $2-3i$.
}
\litem{
Simplify the expression below and choose the interval the simplification is contained within.
\[ 17 - 6^2 + 11 \div 10 * 16 \div 3 \]The solution is \( -13.133 \), which is option B.\begin{enumerate}[label=\Alph*.]
\item \( [-19.98, -17.98] \)

 -18.977, which corresponds to an Order of Operations error: not reading left-to-right for multiplication/division.
\item \( [-17.13, -12.13] \)

* -13.133, this is the correct option
\item \( [53.02, 56.02] \)

 53.023, which corresponds to two Order of Operations errors.
\item \( [55.87, 60.87] \)

 58.867, which corresponds to an Order of Operations error: multiplying by negative before squaring. For example: $(-3)^2 \neq -3^2$
\item \( \text{None of the above} \)

 You may have gotten this by making an unanticipated error. If you got a value that is not any of the others, please let the coordinator know so they can help you figure out what happened.
\end{enumerate}

\textbf{General Comment:} While you may remember (or were taught) PEMDAS is done in order, it is actually done as P/E/MD/AS. When we are at MD or AS, we read left to right.
}
\litem{
Choose the \textbf{smallest} set of Complex numbers that the number below belongs to.
\[ \frac{5}{-9}+36i^2 \]The solution is \( \text{Rational} \), which is option A.\begin{enumerate}[label=\Alph*.]
\item \( \text{Rational} \)

* This is the correct option!
\item \( \text{Not a Complex Number} \)

This is not a number. The only non-Complex number we know is dividing by 0 as this is not a number!
\item \( \text{Pure Imaginary} \)

This is a Complex number $(a+bi)$ that \textbf{only} has an imaginary part like $2i$.
\item \( \text{Nonreal Complex} \)

This is a Complex number $(a+bi)$ that is not Real (has $i$ as part of the number).
\item \( \text{Irrational} \)

These cannot be written as a fraction of Integers. Remember: $\pi$ is not an Integer!
\end{enumerate}

\textbf{General Comment:} Be sure to simplify $i^2 = -1$. This may remove the imaginary portion for your number. If you are having trouble, you may want to look at the \textit{Subgroups of the Real Numbers} section.
}
\litem{
Simplify the expression below into the form $a+bi$. Then, choose the intervals that $a$ and $b$ belong to.
\[ \frac{9 - 44 i}{2 + 7 i} \]The solution is \( -5.47  - 2.85 i \), which is option A.\begin{enumerate}[label=\Alph*.]
\item \( a \in [-5.5, -4.5] \text{ and } b \in [-3.5, -1] \)

* $-5.47  - 2.85 i$, which is the correct option.
\item \( a \in [-290.5, -289.5] \text{ and } b \in [-3.5, -1] \)

 $-290.00  - 2.85 i$, which corresponds to forgetting to multiply the conjugate by the numerator and using a plus instead of a minus in the denominator.
\item \( a \in [5.5, 7] \text{ and } b \in [-1, 1] \)

 $6.15  - 0.47 i$, which corresponds to forgetting to multiply the conjugate by the numerator and not computing the conjugate correctly.
\item \( a \in [3.5, 5.5] \text{ and } b \in [-7.5, -6] \)

 $4.50  - 6.29 i$, which corresponds to just dividing the first term by the first term and the second by the second.
\item \( a \in [-5.5, -4.5] \text{ and } b \in [-151.5, -149.5] \)

 $-5.47  - 151.00 i$, which corresponds to forgetting to multiply the conjugate by the numerator.
\end{enumerate}

\textbf{General Comment:} Multiply the numerator and denominator by the *conjugate* of the denominator, then simplify. For example, if we have $2+3i$, the conjugate is $2-3i$.
}
\litem{
Choose the \textbf{smallest} set of Real numbers that the number below belongs to.
\[ -\sqrt{\frac{1176}{14}} \]The solution is \( \text{Irrational} \), which is option E.\begin{enumerate}[label=\Alph*.]
\item \( \text{Not a Real number} \)

These are Nonreal Complex numbers \textbf{OR} things that are not numbers (e.g., dividing by 0).
\item \( \text{Whole} \)

These are the counting numbers with 0 (0, 1, 2, 3, ...)
\item \( \text{Rational} \)

These are numbers that can be written as fraction of Integers (e.g., -2/3)
\item \( \text{Integer} \)

These are the negative and positive counting numbers (..., -3, -2, -1, 0, 1, 2, 3, ...)
\item \( \text{Irrational} \)

* This is the correct option!
\end{enumerate}

\textbf{General Comment:} First, you \textbf{NEED} to simplify the expression. This question simplifies to $-\sqrt{84}$. 
 
 Be sure you look at the simplified fraction and not just the decimal expansion. Numbers such as 13, 17, and 19 provide \textbf{long but repeating/terminating decimal expansions!} 
 
 The only ways to *not* be a Real number are: dividing by 0 or taking the square root of a negative number. 
 
 Irrational numbers are more than just square root of 3: adding or subtracting values from square root of 3 is also irrational.
}
\litem{
Simplify the expression below into the form $a+bi$. Then, choose the intervals that $a$ and $b$ belong to.
\[ (-7 + 5 i)(-4 + 2 i) \]The solution is \( 18 - 34 i \), which is option A.\begin{enumerate}[label=\Alph*.]
\item \( a \in [16, 20] \text{ and } b \in [-37, -30] \)

* $18 - 34 i$, which is the correct option.
\item \( a \in [37, 42] \text{ and } b \in [6, 8] \)

 $38 + 6 i$, which corresponds to adding a minus sign in the first term.
\item \( a \in [37, 42] \text{ and } b \in [-6, 1] \)

 $38 - 6 i$, which corresponds to adding a minus sign in the second term.
\item \( a \in [16, 20] \text{ and } b \in [29, 37] \)

 $18 + 34 i$, which corresponds to adding a minus sign in both terms.
\item \( a \in [26, 33] \text{ and } b \in [8, 16] \)

 $28 + 10 i$, which corresponds to just multiplying the real terms to get the real part of the solution and the coefficients in the complex terms to get the complex part.
\end{enumerate}

\textbf{General Comment:} You can treat $i$ as a variable and distribute. Just remember that $i^2=-1$, so you can continue to reduce after you distribute.
}
\litem{
Choose the \textbf{smallest} set of Complex numbers that the number below belongs to.
\[ \sqrt{\frac{1078}{0}}+\sqrt{182} i \]The solution is \( \text{Not a Complex Number} \), which is option A.\begin{enumerate}[label=\Alph*.]
\item \( \text{Not a Complex Number} \)

* This is the correct option!
\item \( \text{Nonreal Complex} \)

This is a Complex number $(a+bi)$ that is not Real (has $i$ as part of the number).
\item \( \text{Rational} \)

These are numbers that can be written as fraction of Integers (e.g., -2/3 + 5)
\item \( \text{Pure Imaginary} \)

This is a Complex number $(a+bi)$ that \textbf{only} has an imaginary part like $2i$.
\item \( \text{Irrational} \)

These cannot be written as a fraction of Integers. Remember: $\pi$ is not an Integer!
\end{enumerate}

\textbf{General Comment:} Be sure to simplify $i^2 = -1$. This may remove the imaginary portion for your number. If you are having trouble, you may want to look at the \textit{Subgroups of the Real Numbers} section.
}
\litem{
Simplify the expression below and choose the interval the simplification is contained within.
\[ 18 - 1 \div 8 * 20 - (15 * 6) \]The solution is \( -74.500 \), which is option C.\begin{enumerate}[label=\Alph*.]
\item \( [3, 8] \)

 3.000, which corresponds to not distributing a negative correctly.
\item \( [-72.01, -64.01] \)

 -72.006, which corresponds to an Order of Operations error: not reading left-to-right for multiplication/division.
\item \( [-76.5, -72.5] \)

* -74.500, which is the correct option.
\item \( [105.99, 111.99] \)

 107.994, which corresponds to not distributing addition and subtraction correctly.
\item \( \text{None of the above} \)

 You may have gotten this by making an unanticipated error. If you got a value that is not any of the others, please let the coordinator know so they can help you figure out what happened.
\end{enumerate}

\textbf{General Comment:} While you may remember (or were taught) PEMDAS is done in order, it is actually done as P/E/MD/AS. When we are at MD or AS, we read left to right.
}
\end{enumerate}

\end{document}