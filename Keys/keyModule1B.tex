\documentclass{extbook}[14pt]
\usepackage{multicol, enumerate, enumitem, hyperref, color, soul, setspace, parskip, fancyhdr, amssymb, amsthm, amsmath, bbm, latexsym, units, mathtools}
\everymath{\displaystyle}
\usepackage[headsep=0.5cm,headheight=0cm, left=1 in,right= 1 in,top= 1 in,bottom= 1 in]{geometry}
\usepackage{dashrule}  % Package to use the command below to create lines between items
\newcommand{\litem}[1]{\item #1

\rule{\textwidth}{0.4pt}}
\pagestyle{fancy}
\lhead{}
\chead{Answer Key for Progress Quiz 5 Version B}
\rhead{}
\lfoot{9912-2038}
\cfoot{}
\rfoot{Spring 2021}
\begin{document}
\textbf{This key should allow you to understand why you choose the option you did (beyond just getting a question right or wrong). \href{https://xronos.clas.ufl.edu/mac1105spring2020/courseDescriptionAndMisc/Exams/LearningFromResults}{More instructions on how to use this key can be found here}.}

\textbf{If you have a suggestion to make the keys better, \href{https://forms.gle/CZkbZmPbC9XALEE88}{please fill out the short survey here}.}

\textit{Note: This key is auto-generated and may contain issues and/or errors. The keys are reviewed after each exam to ensure grading is done accurately. If there are issues (like duplicate options), they are noted in the offline gradebook. The keys are a work-in-progress to give students as many resources to improve as possible.}

\rule{\textwidth}{0.4pt}

\begin{enumerate}\litem{
Choose the \textbf{smallest} set of Complex numbers that the number below belongs to.
\[ -\sqrt{\frac{529}{81}} + 9i^2 \]The solution is \( \text{Rational} \), which is option C.\begin{enumerate}[label=\Alph*.]
\item \( \text{Nonreal Complex} \)

This is a Complex number $(a+bi)$ that is not Real (has $i$ as part of the number).
\item \( \text{Irrational} \)

These cannot be written as a fraction of Integers. Remember: $\pi$ is not an Integer!
\item \( \text{Rational} \)

* This is the correct option!
\item \( \text{Pure Imaginary} \)

This is a Complex number $(a+bi)$ that \textbf{only} has an imaginary part like $2i$.
\item \( \text{Not a Complex Number} \)

This is not a number. The only non-Complex number we know is dividing by 0 as this is not a number!
\end{enumerate}

\textbf{General Comment:} Be sure to simplify $i^2 = -1$. This may remove the imaginary portion for your number. If you are having trouble, you may want to look at the \textit{Subgroups of the Real Numbers} section.
}
\litem{
Simplify the expression below into the form $a+bi$. Then, choose the intervals that $a$ and $b$ belong to.
\[ (4 - 7 i)(-2 - 10 i) \]The solution is \( -78 - 26 i \), which is option C.\begin{enumerate}[label=\Alph*.]
\item \( a \in [62, 65] \text{ and } b \in [-54, -50] \)

 $62 - 54 i$, which corresponds to adding a minus sign in the first term.
\item \( a \in [-11, -3] \text{ and } b \in [66, 74] \)

 $-8 + 70 i$, which corresponds to just multiplying the real terms to get the real part of the solution and the coefficients in the complex terms to get the complex part.
\item \( a \in [-78, -76] \text{ and } b \in [-29, -24] \)

* $-78 - 26 i$, which is the correct option.
\item \( a \in [-78, -76] \text{ and } b \in [25, 31] \)

 $-78 + 26 i$, which corresponds to adding a minus sign in both terms.
\item \( a \in [62, 65] \text{ and } b \in [51, 61] \)

 $62 + 54 i$, which corresponds to adding a minus sign in the second term.
\end{enumerate}

\textbf{General Comment:} You can treat $i$ as a variable and distribute. Just remember that $i^2=-1$, so you can continue to reduce after you distribute.
}
\litem{
Simplify the expression below into the form $a+bi$. Then, choose the intervals that $a$ and $b$ belong to.
\[ \frac{36 + 77 i}{-6 - 8 i} \]The solution is \( -8.32  - 1.74 i \), which is option D.\begin{enumerate}[label=\Alph*.]
\item \( a \in [3, 6] \text{ and } b \in [-8, -7] \)

 $4.00  - 7.50 i$, which corresponds to forgetting to multiply the conjugate by the numerator and not computing the conjugate correctly.
\item \( a \in [-833.5, -831] \text{ and } b \in [-2.5, -1] \)

 $-832.00  - 1.74 i$, which corresponds to forgetting to multiply the conjugate by the numerator and using a plus instead of a minus in the denominator.
\item \( a \in [-7, -5.5] \text{ and } b \in [-11, -9] \)

 $-6.00  - 9.62 i$, which corresponds to just dividing the first term by the first term and the second by the second.
\item \( a \in [-10, -7.5] \text{ and } b \in [-2.5, -1] \)

* $-8.32  - 1.74 i$, which is the correct option.
\item \( a \in [-10, -7.5] \text{ and } b \in [-175.5, -173] \)

 $-8.32  - 174.00 i$, which corresponds to forgetting to multiply the conjugate by the numerator.
\end{enumerate}

\textbf{General Comment:} Multiply the numerator and denominator by the *conjugate* of the denominator, then simplify. For example, if we have $2+3i$, the conjugate is $2-3i$.
}
\litem{
Simplify the expression below into the form $a+bi$. Then, choose the intervals that $a$ and $b$ belong to.
\[ \frac{-27 + 44 i}{2 - 8 i} \]The solution is \( -5.97  - 1.88 i \), which is option B.\begin{enumerate}[label=\Alph*.]
\item \( a \in [-6.5, -4] \text{ and } b \in [-128.5, -127.5] \)

 $-5.97  - 128.00 i$, which corresponds to forgetting to multiply the conjugate by the numerator.
\item \( a \in [-6.5, -4] \text{ and } b \in [-2.5, -1.5] \)

* $-5.97  - 1.88 i$, which is the correct option.
\item \( a \in [-15, -12.5] \text{ and } b \in [-6, -5] \)

 $-13.50  - 5.50 i$, which corresponds to just dividing the first term by the first term and the second by the second.
\item \( a \in [-408, -405.5] \text{ and } b \in [-2.5, -1.5] \)

 $-406.00  - 1.88 i$, which corresponds to forgetting to multiply the conjugate by the numerator and using a plus instead of a minus in the denominator.
\item \( a \in [4, 4.5] \text{ and } b \in [4, 5.5] \)

 $4.38  + 4.47 i$, which corresponds to forgetting to multiply the conjugate by the numerator and not computing the conjugate correctly.
\end{enumerate}

\textbf{General Comment:} Multiply the numerator and denominator by the *conjugate* of the denominator, then simplify. For example, if we have $2+3i$, the conjugate is $2-3i$.
}
\litem{
Simplify the expression below and choose the interval the simplification is contained within.
\[ 15 - 10 \div 17 * 5 - (9 * 11) \]The solution is \( -86.941 \), which is option B.\begin{enumerate}[label=\Alph*.]
\item \( [111.3, 115.6] \)

 113.882, which corresponds to not distributing addition and subtraction correctly.
\item \( [-87.9, -84.7] \)

* -86.941, which is the correct option.
\item \( [-84.7, -82.5] \)

 -84.118, which corresponds to an Order of Operations error: not reading left-to-right for multiplication/division.
\item \( [31.8, 35.2] \)

 33.647, which corresponds to not distributing a negative correctly.
\item \( \text{None of the above} \)

 You may have gotten this by making an unanticipated error. If you got a value that is not any of the others, please let the coordinator know so they can help you figure out what happened.
\end{enumerate}

\textbf{General Comment:} While you may remember (or were taught) PEMDAS is done in order, it is actually done as P/E/MD/AS. When we are at MD or AS, we read left to right.
}
\litem{
Choose the \textbf{smallest} set of Real numbers that the number below belongs to.
\[ -\sqrt{\frac{-2652}{12}} \]The solution is \( \text{Not a Real number} \), which is option C.\begin{enumerate}[label=\Alph*.]
\item \( \text{Whole} \)

These are the counting numbers with 0 (0, 1, 2, 3, ...)
\item \( \text{Irrational} \)

These cannot be written as a fraction of Integers.
\item \( \text{Not a Real number} \)

* This is the correct option!
\item \( \text{Integer} \)

These are the negative and positive counting numbers (..., -3, -2, -1, 0, 1, 2, 3, ...)
\item \( \text{Rational} \)

These are numbers that can be written as fraction of Integers (e.g., -2/3)
\end{enumerate}

\textbf{General Comment:} First, you \textbf{NEED} to simplify the expression. This question simplifies to $-\sqrt{221} i$. 
 
 Be sure you look at the simplified fraction and not just the decimal expansion. Numbers such as 13, 17, and 19 provide \textbf{long but repeating/terminating decimal expansions!} 
 
 The only ways to *not* be a Real number are: dividing by 0 or taking the square root of a negative number. 
 
 Irrational numbers are more than just square root of 3: adding or subtracting values from square root of 3 is also irrational.
}
\litem{
Choose the \textbf{smallest} set of Real numbers that the number below belongs to.
\[ -\sqrt{\frac{81}{169}} \]The solution is \( \text{Rational} \), which is option E.\begin{enumerate}[label=\Alph*.]
\item \( \text{Integer} \)

These are the negative and positive counting numbers (..., -3, -2, -1, 0, 1, 2, 3, ...)
\item \( \text{Not a Real number} \)

These are Nonreal Complex numbers \textbf{OR} things that are not numbers (e.g., dividing by 0).
\item \( \text{Irrational} \)

These cannot be written as a fraction of Integers.
\item \( \text{Whole} \)

These are the counting numbers with 0 (0, 1, 2, 3, ...)
\item \( \text{Rational} \)

* This is the correct option!
\end{enumerate}

\textbf{General Comment:} First, you \textbf{NEED} to simplify the expression. This question simplifies to $-\frac{9}{13}$. 
 
 Be sure you look at the simplified fraction and not just the decimal expansion. Numbers such as 13, 17, and 19 provide \textbf{long but repeating/terminating decimal expansions!} 
 
 The only ways to *not* be a Real number are: dividing by 0 or taking the square root of a negative number. 
 
 Irrational numbers are more than just square root of 3: adding or subtracting values from square root of 3 is also irrational.
}
\litem{
Choose the \textbf{smallest} set of Complex numbers that the number below belongs to.
\[ -\sqrt{\frac{1690}{13}}+4i^2 \]The solution is \( \text{Irrational} \), which is option A.\begin{enumerate}[label=\Alph*.]
\item \( \text{Irrational} \)

* This is the correct option!
\item \( \text{Not a Complex Number} \)

This is not a number. The only non-Complex number we know is dividing by 0 as this is not a number!
\item \( \text{Rational} \)

These are numbers that can be written as fraction of Integers (e.g., -2/3 + 5)
\item \( \text{Pure Imaginary} \)

This is a Complex number $(a+bi)$ that \textbf{only} has an imaginary part like $2i$.
\item \( \text{Nonreal Complex} \)

This is a Complex number $(a+bi)$ that is not Real (has $i$ as part of the number).
\end{enumerate}

\textbf{General Comment:} Be sure to simplify $i^2 = -1$. This may remove the imaginary portion for your number. If you are having trouble, you may want to look at the \textit{Subgroups of the Real Numbers} section.
}
\litem{
Simplify the expression below into the form $a+bi$. Then, choose the intervals that $a$ and $b$ belong to.
\[ (10 + 3 i)(-4 + 8 i) \]The solution is \( -64 + 68 i \), which is option C.\begin{enumerate}[label=\Alph*.]
\item \( a \in [-67, -63] \text{ and } b \in [-69, -65] \)

 $-64 - 68 i$, which corresponds to adding a minus sign in both terms.
\item \( a \in [-18, -15] \text{ and } b \in [90, 98] \)

 $-16 + 92 i$, which corresponds to adding a minus sign in the first term.
\item \( a \in [-67, -63] \text{ and } b \in [67, 75] \)

* $-64 + 68 i$, which is the correct option.
\item \( a \in [-18, -15] \text{ and } b \in [-97, -91] \)

 $-16 - 92 i$, which corresponds to adding a minus sign in the second term.
\item \( a \in [-45, -34] \text{ and } b \in [19, 27] \)

 $-40 + 24 i$, which corresponds to just multiplying the real terms to get the real part of the solution and the coefficients in the complex terms to get the complex part.
\end{enumerate}

\textbf{General Comment:} You can treat $i$ as a variable and distribute. Just remember that $i^2=-1$, so you can continue to reduce after you distribute.
}
\litem{
Simplify the expression below and choose the interval the simplification is contained within.
\[ 14 - 1 \div 3 * 16 - (2 * 15) \]The solution is \( -21.333 \), which is option B.\begin{enumerate}[label=\Alph*.]
\item \( [-18.7, -12.3] \)

 -16.021, which corresponds to an Order of Operations error: not reading left-to-right for multiplication/division.
\item \( [-22.9, -17.7] \)

* -21.333, which is the correct option.
\item \( [43.8, 44.5] \)

 43.979, which corresponds to not distributing addition and subtraction correctly.
\item \( [98.9, 101.4] \)

 100.000, which corresponds to not distributing a negative correctly.
\item \( \text{None of the above} \)

 You may have gotten this by making an unanticipated error. If you got a value that is not any of the others, please let the coordinator know so they can help you figure out what happened.
\end{enumerate}

\textbf{General Comment:} While you may remember (or were taught) PEMDAS is done in order, it is actually done as P/E/MD/AS. When we are at MD or AS, we read left to right.
}
\end{enumerate}

\end{document}