\documentclass{extbook}[14pt]
\usepackage{multicol, enumerate, enumitem, hyperref, color, soul, setspace, parskip, fancyhdr, amssymb, amsthm, amsmath, bbm, latexsym, units, mathtools}
\everymath{\displaystyle}
\usepackage[headsep=0.5cm,headheight=0cm, left=1 in,right= 1 in,top= 1 in,bottom= 1 in]{geometry}
\usepackage{dashrule}  % Package to use the command below to create lines between items
\newcommand{\litem}[1]{\item #1

\rule{\textwidth}{0.4pt}}
\pagestyle{fancy}
\lhead{}
\chead{Answer Key for Progress Quiz 4 Version C}
\rhead{}
\lfoot{6286-1986}
\cfoot{}
\rfoot{Fall 2020}
\begin{document}
\textbf{This key should allow you to understand why you choose the option you did (beyond just getting a question right or wrong). \href{https://xronos.clas.ufl.edu/mac1105spring2020/courseDescriptionAndMisc/Exams/LearningFromResults}{More instructions on how to use this key can be found here}.}

\textbf{If you have a suggestion to make the keys better, \href{https://forms.gle/CZkbZmPbC9XALEE88}{please fill out the short survey here}.}

\textit{Note: This key is auto-generated and may contain issues and/or errors. The keys are reviewed after each exam to ensure grading is done accurately. If there are issues (like duplicate options), they are noted in the offline gradebook. The keys are a work-in-progress to give students as many resources to improve as possible.}

\rule{\textwidth}{0.4pt}

\begin{enumerate}\litem{
Simplify the expression below into the form $a+bi$. Then, choose the intervals that $a$ and $b$ belong to.
\[ (-7 + 3 i)(-4 - 10 i) \]
The solution is \( 58 + 58 i \), which is option E.\begin{enumerate}[label=\Alph*.]
\item \( a \in [24, 31] \text{ and } b \in [-31, -27] \)

 $28 - 30 i$, which corresponds to just multiplying the real terms to get the real part of the solution and the coefficients in the complex terms to get the complex part.
\item \( a \in [-3, 3] \text{ and } b \in [76, 84] \)

 $-2 + 82 i$, which corresponds to adding a minus sign in the first term.
\item \( a \in [57, 61] \text{ and } b \in [-65, -57] \)

 $58 - 58 i$, which corresponds to adding a minus sign in both terms.
\item \( a \in [-3, 3] \text{ and } b \in [-85, -75] \)

 $-2 - 82 i$, which corresponds to adding a minus sign in the second term.
\item \( a \in [57, 61] \text{ and } b \in [58, 62] \)

* $58 + 58 i$, which is the correct option.
\end{enumerate}

\textbf{General Comment:} You can treat $i$ as a variable and distribute. Just remember that $i^2=-1$, so you can continue to reduce after you distribute.
}
\litem{
Choose the \textbf{smallest} set of Complex numbers that the number below belongs to.
\[ \sqrt{\frac{-660}{0}}+\sqrt{143} \]
The solution is \( \text{Not a Complex Number} \), which is option B.\begin{enumerate}[label=\Alph*.]
\item \( \text{Pure Imaginary} \)

This is a Complex number $(a+bi)$ that \textbf{only} has an imaginary part like $2i$.
\item \( \text{Not a Complex Number} \)

* This is the correct option!
\item \( \text{Irrational} \)

These cannot be written as a fraction of Integers. Remember: $\pi$ is not an Integer!
\item \( \text{Nonreal Complex} \)

This is a Complex number $(a+bi)$ that is not Real (has $i$ as part of the number).
\item \( \text{Rational} \)

These are numbers that can be written as fraction of Integers (e.g., -2/3 + 5)
\end{enumerate}

\textbf{General Comment:} Be sure to simplify $i^2 = -1$. This may remove the imaginary portion for your number. If you are having trouble, you may want to look at the \textit{Subgroups of the Real Numbers} section.
}
\litem{
Simplify the expression below into the form $a+bi$. Then, choose the intervals that $a$ and $b$ belong to.
\[ \frac{54 - 77 i}{-2 + 5 i} \]
The solution is \( -17.00  - 4.00 i \), which is option D.\begin{enumerate}[label=\Alph*.]
\item \( a \in [-28.5, -26.5] \text{ and } b \in [-16.5, -14.5] \)

 $-27.00  - 15.40 i$, which corresponds to just dividing the first term by the first term and the second by the second.
\item \( a \in [9, 10] \text{ and } b \in [14, 16] \)

 $9.55  + 14.62 i$, which corresponds to forgetting to multiply the conjugate by the numerator and not computing the conjugate correctly.
\item \( a \in [-493.5, -492] \text{ and } b \in [-4.5, -3] \)

 $-493.00  - 4.00 i$, which corresponds to forgetting to multiply the conjugate by the numerator and using a plus instead of a minus in the denominator.
\item \( a \in [-17.5, -16] \text{ and } b \in [-4.5, -3] \)

* $-17.00  - 4.00 i$, which is the correct option.
\item \( a \in [-17.5, -16] \text{ and } b \in [-116.5, -115] \)

 $-17.00  - 116.00 i$, which corresponds to forgetting to multiply the conjugate by the numerator.
\end{enumerate}

\textbf{General Comment:} Multiply the numerator and denominator by the *conjugate* of the denominator, then simplify. For example, if we have $2+3i$, the conjugate is $2-3i$.
}
\litem{
Choose the \textbf{smallest} set of Real numbers that the number below belongs to.
\[ -\sqrt{\frac{300}{5}} \]
The solution is \( \text{Irrational} \), which is option C.\begin{enumerate}[label=\Alph*.]
\item \( \text{Not a Real number} \)

These are Nonreal Complex numbers \textbf{OR} things that are not numbers (e.g., dividing by 0).
\item \( \text{Whole} \)

These are the counting numbers with 0 (0, 1, 2, 3, ...)
\item \( \text{Irrational} \)

* This is the correct option!
\item \( \text{Integer} \)

These are the negative and positive counting numbers (..., -3, -2, -1, 0, 1, 2, 3, ...)
\item \( \text{Rational} \)

These are numbers that can be written as fraction of Integers (e.g., -2/3)
\end{enumerate}

\textbf{General Comment:} First, you \textbf{NEED} to simplify the expression. This question simplifies to $-\sqrt{60}$. 
 
 Be sure you look at the simplified fraction and not just the decimal expansion. Numbers such as 13, 17, and 19 provide \textbf{long but repeating/terminating decimal expansions!} 
 
 The only ways to *not* be a Real number are: dividing by 0 or taking the square root of a negative number. 
 
 Irrational numbers are more than just square root of 3: adding or subtracting values from square root of 3 is also irrational.
}
\litem{
Simplify the expression below into the form $a+bi$. Then, choose the intervals that $a$ and $b$ belong to.
\[ \frac{54 + 22 i}{-4 - 3 i} \]
The solution is \( -11.28  + 2.96 i \), which is option E.\begin{enumerate}[label=\Alph*.]
\item \( a \in [-7, -5.5] \text{ and } b \in [-10.5, -8.5] \)

 $-6.00  - 10.00 i$, which corresponds to forgetting to multiply the conjugate by the numerator and not computing the conjugate correctly.
\item \( a \in [-11.5, -10.5] \text{ and } b \in [73.5, 75] \)

 $-11.28  + 74.00 i$, which corresponds to forgetting to multiply the conjugate by the numerator.
\item \( a \in [-14.5, -13] \text{ and } b \in [-8, -6.5] \)

 $-13.50  - 7.33 i$, which corresponds to just dividing the first term by the first term and the second by the second.
\item \( a \in [-282.5, -280.5] \text{ and } b \in [2.5, 4] \)

 $-282.00  + 2.96 i$, which corresponds to forgetting to multiply the conjugate by the numerator and using a plus instead of a minus in the denominator.
\item \( a \in [-11.5, -10.5] \text{ and } b \in [2.5, 4] \)

* $-11.28  + 2.96 i$, which is the correct option.
\end{enumerate}

\textbf{General Comment:} Multiply the numerator and denominator by the *conjugate* of the denominator, then simplify. For example, if we have $2+3i$, the conjugate is $2-3i$.
}
\litem{
Simplify the expression below and choose the interval the simplification is contained within.
\[ 13 - 19 \div 5 * 4 - (2 * 9) \]
The solution is \( -20.200 \), which is option A.\begin{enumerate}[label=\Alph*.]
\item \( [-20.2, -18.2] \)

* -20.200, which is the correct option.
\item \( [-7.95, -3.95] \)

 -5.950, which corresponds to an Order of Operations error: not reading left-to-right for multiplication/division.
\item \( [29.05, 34.05] \)

 30.050, which corresponds to not distributing addition and subtraction correctly.
\item \( [-42.8, -34.8] \)

 -37.800, which corresponds to not distributing a negative correctly.
\item \( \text{None of the above} \)

 You may have gotten this by making an unanticipated error. If you got a value that is not any of the others, please let the coordinator know so they can help you figure out what happened.
\end{enumerate}

\textbf{General Comment:} While you may remember (or were taught) PEMDAS is done in order, it is actually done as P/E/MD/AS. When we are at MD or AS, we read left to right.
}
\litem{
Choose the \textbf{smallest} set of Real numbers that the number below belongs to.
\[ \sqrt{\frac{46656}{81}} \]
The solution is \( \text{Whole} \), which is option D.\begin{enumerate}[label=\Alph*.]
\item \( \text{Integer} \)

These are the negative and positive counting numbers (..., -3, -2, -1, 0, 1, 2, 3, ...)
\item \( \text{Rational} \)

These are numbers that can be written as fraction of Integers (e.g., -2/3)
\item \( \text{Not a Real number} \)

These are Nonreal Complex numbers \textbf{OR} things that are not numbers (e.g., dividing by 0).
\item \( \text{Whole} \)

* This is the correct option!
\item \( \text{Irrational} \)

These cannot be written as a fraction of Integers.
\end{enumerate}

\textbf{General Comment:} First, you \textbf{NEED} to simplify the expression. This question simplifies to $216$. 
 
 Be sure you look at the simplified fraction and not just the decimal expansion. Numbers such as 13, 17, and 19 provide \textbf{long but repeating/terminating decimal expansions!} 
 
 The only ways to *not* be a Real number are: dividing by 0 or taking the square root of a negative number. 
 
 Irrational numbers are more than just square root of 3: adding or subtracting values from square root of 3 is also irrational.
}
\litem{
Simplify the expression below and choose the interval the simplification is contained within.
\[ 13 - 19^2 + 18 \div 12 * 11 \div 3 \]
The solution is \( -342.500 \), which is option B.\begin{enumerate}[label=\Alph*.]
\item \( [-350.95, -342.95] \)

 -347.955, which corresponds to an Order of Operations error: not reading left-to-right for multiplication/division.
\item \( [-347.5, -336.5] \)

* -342.500, this is the correct option
\item \( [376.5, 382.5] \)

 379.500, which corresponds to an Order of Operations error: multiplying by negative before squaring. For example: $(-3)^2 \neq -3^2$
\item \( [372.05, 375.05] \)

 374.045, which corresponds to two Order of Operations errors.
\item \( \text{None of the above} \)

 You may have gotten this by making an unanticipated error. If you got a value that is not any of the others, please let the coordinator know so they can help you figure out what happened.
\end{enumerate}

\textbf{General Comment:} While you may remember (or were taught) PEMDAS is done in order, it is actually done as P/E/MD/AS. When we are at MD or AS, we read left to right.
}
\litem{
Choose the \textbf{smallest} set of Complex numbers that the number below belongs to.
\[ \sqrt{\frac{-880}{0}} i+\sqrt{130}i \]
The solution is \( \text{Not a Complex Number} \), which is option A.\begin{enumerate}[label=\Alph*.]
\item \( \text{Not a Complex Number} \)

* This is the correct option!
\item \( \text{Rational} \)

These are numbers that can be written as fraction of Integers (e.g., -2/3 + 5)
\item \( \text{Nonreal Complex} \)

This is a Complex number $(a+bi)$ that is not Real (has $i$ as part of the number).
\item \( \text{Pure Imaginary} \)

This is a Complex number $(a+bi)$ that \textbf{only} has an imaginary part like $2i$.
\item \( \text{Irrational} \)

These cannot be written as a fraction of Integers. Remember: $\pi$ is not an Integer!
\end{enumerate}

\textbf{General Comment:} Be sure to simplify $i^2 = -1$. This may remove the imaginary portion for your number. If you are having trouble, you may want to look at the \textit{Subgroups of the Real Numbers} section.
}
\litem{
Simplify the expression below into the form $a+bi$. Then, choose the intervals that $a$ and $b$ belong to.
\[ (10 - 5 i)(-6 - 7 i) \]
The solution is \( -95 - 40 i \), which is option D.\begin{enumerate}[label=\Alph*.]
\item \( a \in [-97, -90] \text{ and } b \in [38, 46] \)

 $-95 + 40 i$, which corresponds to adding a minus sign in both terms.
\item \( a \in [-60, -56] \text{ and } b \in [35, 36] \)

 $-60 + 35 i$, which corresponds to just multiplying the real terms to get the real part of the solution and the coefficients in the complex terms to get the complex part.
\item \( a \in [-29, -24] \text{ and } b \in [-101, -97] \)

 $-25 - 100 i$, which corresponds to adding a minus sign in the first term.
\item \( a \in [-97, -90] \text{ and } b \in [-42, -34] \)

* $-95 - 40 i$, which is the correct option.
\item \( a \in [-29, -24] \text{ and } b \in [97, 101] \)

 $-25 + 100 i$, which corresponds to adding a minus sign in the second term.
\end{enumerate}

\textbf{General Comment:} You can treat $i$ as a variable and distribute. Just remember that $i^2=-1$, so you can continue to reduce after you distribute.
}
\end{enumerate}

\end{document}