\documentclass{extbook}[14pt]
\usepackage{multicol, enumerate, enumitem, hyperref, color, soul, setspace, parskip, fancyhdr, amssymb, amsthm, amsmath, bbm, latexsym, units, mathtools}
\everymath{\displaystyle}
\usepackage[headsep=0.5cm,headheight=0cm, left=1 in,right= 1 in,top= 1 in,bottom= 1 in]{geometry}
\usepackage{dashrule}  % Package to use the command below to create lines between items
\newcommand{\litem}[1]{\item #1

\rule{\textwidth}{0.4pt}}
\pagestyle{fancy}
\lhead{}
\chead{Answer Key for Progress Quiz 4 Version C}
\rhead{}
\lfoot{8448-1521}
\cfoot{}
\rfoot{Fall 2020}
\begin{document}
\textbf{This key should allow you to understand why you choose the option you did (beyond just getting a question right or wrong). \href{https://xronos.clas.ufl.edu/mac1105spring2020/courseDescriptionAndMisc/Exams/LearningFromResults}{More instructions on how to use this key can be found here}.}

\textbf{If you have a suggestion to make the keys better, \href{https://forms.gle/CZkbZmPbC9XALEE88}{please fill out the short survey here}.}

\textit{Note: This key is auto-generated and may contain issues and/or errors. The keys are reviewed after each exam to ensure grading is done accurately. If there are issues (like duplicate options), they are noted in the offline gradebook. The keys are a work-in-progress to give students as many resources to improve as possible.}

\rule{\textwidth}{0.4pt}

\begin{enumerate}\litem{
Simplify the expression below and choose the interval the simplification is contained within.
\[ 14 - 2^2 + 18 \div 19 * 10 \div 16 \]
The solution is \( 10.592 \), which is option D.\begin{enumerate}[label=\Alph*.]
\item \( [16.2, 18.38] \)

 18.006, which corresponds to two Order of Operations errors.
\item \( [18.18, 19.28] \)

 18.592, which corresponds to an Order of Operations error: multiplying by negative before squaring. For example: $(-3)^2 \neq -3^2$
\item \( [9.91, 10.54] \)

 10.006, which corresponds to an Order of Operations error: not reading left-to-right for multiplication/division.
\item \( [10.22, 10.67] \)

* 10.592, this is the correct option
\item \( \text{None of the above} \)

 You may have gotten this by making an unanticipated error. If you got a value that is not any of the others, please let the coordinator know so they can help you figure out what happened.
\end{enumerate}

\textbf{General Comment:} While you may remember (or were taught) PEMDAS is done in order, it is actually done as P/E/MD/AS. When we are at MD or AS, we read left to right.
}
\litem{
Choose the \textbf{smallest} set of Complex numbers that the number below belongs to.
\[ \sqrt{\frac{-2475}{0}} i+\sqrt{55}i \]
The solution is \( \text{Not a Complex Number} \), which is option A.\begin{enumerate}[label=\Alph*.]
\item \( \text{Not a Complex Number} \)

* This is the correct option!
\item \( \text{Irrational} \)

These cannot be written as a fraction of Integers. Remember: $\pi$ is not an Integer!
\item \( \text{Rational} \)

These are numbers that can be written as fraction of Integers (e.g., -2/3 + 5)
\item \( \text{Nonreal Complex} \)

This is a Complex number $(a+bi)$ that is not Real (has $i$ as part of the number).
\item \( \text{Pure Imaginary} \)

This is a Complex number $(a+bi)$ that \textbf{only} has an imaginary part like $2i$.
\end{enumerate}

\textbf{General Comment:} Be sure to simplify $i^2 = -1$. This may remove the imaginary portion for your number. If you are having trouble, you may want to look at the \textit{Subgroups of the Real Numbers} section.
}
\litem{
Simplify the expression below into the form $a+bi$. Then, choose the intervals that $a$ and $b$ belong to.
\[ \frac{54 - 33 i}{4 - 7 i} \]
The solution is \( 6.88  + 3.78 i \), which is option E.\begin{enumerate}[label=\Alph*.]
\item \( a \in [6.5, 8.5] \text{ and } b \in [245, 247] \)

 $6.88  + 246.00 i$, which corresponds to forgetting to multiply the conjugate by the numerator.
\item \( a \in [13, 14] \text{ and } b \in [4, 5.5] \)

 $13.50  + 4.71 i$, which corresponds to just dividing the first term by the first term and the second by the second.
\item \( a \in [445.5, 448.5] \text{ and } b \in [3, 4] \)

 $447.00  + 3.78 i$, which corresponds to forgetting to multiply the conjugate by the numerator and using a plus instead of a minus in the denominator.
\item \( a \in [-1.5, 0] \text{ and } b \in [-9, -7] \)

 $-0.23  - 7.85 i$, which corresponds to forgetting to multiply the conjugate by the numerator and not computing the conjugate correctly.
\item \( a \in [6.5, 8.5] \text{ and } b \in [3, 4] \)

* $6.88  + 3.78 i$, which is the correct option.
\end{enumerate}

\textbf{General Comment:} Multiply the numerator and denominator by the *conjugate* of the denominator, then simplify. For example, if we have $2+3i$, the conjugate is $2-3i$.
}
\litem{
Choose the \textbf{smallest} set of Complex numbers that the number below belongs to.
\[ \sqrt{\frac{-756}{9}} i+\sqrt{187}i \]
The solution is \( \text{Nonreal Complex} \), which is option E.\begin{enumerate}[label=\Alph*.]
\item \( \text{Not a Complex Number} \)

This is not a number. The only non-Complex number we know is dividing by 0 as this is not a number!
\item \( \text{Pure Imaginary} \)

This is a Complex number $(a+bi)$ that \textbf{only} has an imaginary part like $2i$.
\item \( \text{Rational} \)

These are numbers that can be written as fraction of Integers (e.g., -2/3 + 5)
\item \( \text{Irrational} \)

These cannot be written as a fraction of Integers. Remember: $\pi$ is not an Integer!
\item \( \text{Nonreal Complex} \)

* This is the correct option!
\end{enumerate}

\textbf{General Comment:} Be sure to simplify $i^2 = -1$. This may remove the imaginary portion for your number. If you are having trouble, you may want to look at the \textit{Subgroups of the Real Numbers} section.
}
\litem{
Choose the \textbf{smallest} set of Real numbers that the number below belongs to.
\[ \sqrt{\frac{660}{12}} \]
The solution is \( \text{Irrational} \), which is option A.\begin{enumerate}[label=\Alph*.]
\item \( \text{Irrational} \)

* This is the correct option!
\item \( \text{Not a Real number} \)

These are Nonreal Complex numbers \textbf{OR} things that are not numbers (e.g., dividing by 0).
\item \( \text{Rational} \)

These are numbers that can be written as fraction of Integers (e.g., -2/3)
\item \( \text{Whole} \)

These are the counting numbers with 0 (0, 1, 2, 3, ...)
\item \( \text{Integer} \)

These are the negative and positive counting numbers (..., -3, -2, -1, 0, 1, 2, 3, ...)
\end{enumerate}

\textbf{General Comment:} First, you \textbf{NEED} to simplify the expression. This question simplifies to $\sqrt{55}$. 
 
 Be sure you look at the simplified fraction and not just the decimal expansion. Numbers such as 13, 17, and 19 provide \textbf{long but repeating/terminating decimal expansions!} 
 
 The only ways to *not* be a Real number are: dividing by 0 or taking the square root of a negative number. 
 
 Irrational numbers are more than just square root of 3: adding or subtracting values from square root of 3 is also irrational.
}
\litem{
Simplify the expression below into the form $a+bi$. Then, choose the intervals that $a$ and $b$ belong to.
\[ (6 + 4 i)(-8 - 2 i) \]
The solution is \( -40 - 44 i \), which is option B.\begin{enumerate}[label=\Alph*.]
\item \( a \in [-65, -54] \text{ and } b \in [19, 25] \)

 $-56 + 20 i$, which corresponds to adding a minus sign in the first term.
\item \( a \in [-43, -33] \text{ and } b \in [-48, -40] \)

* $-40 - 44 i$, which is the correct option.
\item \( a \in [-53, -47] \text{ and } b \in [-12, -6] \)

 $-48 - 8 i$, which corresponds to just multiplying the real terms to get the real part of the solution and the coefficients in the complex terms to get the complex part.
\item \( a \in [-43, -33] \text{ and } b \in [37, 46] \)

 $-40 + 44 i$, which corresponds to adding a minus sign in both terms.
\item \( a \in [-65, -54] \text{ and } b \in [-20, -16] \)

 $-56 - 20 i$, which corresponds to adding a minus sign in the second term.
\end{enumerate}

\textbf{General Comment:} You can treat $i$ as a variable and distribute. Just remember that $i^2=-1$, so you can continue to reduce after you distribute.
}
\litem{
Simplify the expression below and choose the interval the simplification is contained within.
\[ 2 - 3 \div 4 * 9 - (5 * 18) \]
The solution is \( -94.750 \), which is option B.\begin{enumerate}[label=\Alph*.]
\item \( [89.92, 94.92] \)

 91.917, which corresponds to not distributing addition and subtraction correctly.
\item \( [-97.75, -91.75] \)

* -94.750, which is the correct option.
\item \( [-90.08, -86.08] \)

 -88.083, which corresponds to an Order of Operations error: not reading left-to-right for multiplication/division.
\item \( [-175.5, -170.5] \)

 -175.500, which corresponds to not distributing a negative correctly.
\item \( \text{None of the above} \)

 You may have gotten this by making an unanticipated error. If you got a value that is not any of the others, please let the coordinator know so they can help you figure out what happened.
\end{enumerate}

\textbf{General Comment:} While you may remember (or were taught) PEMDAS is done in order, it is actually done as P/E/MD/AS. When we are at MD or AS, we read left to right.
}
\litem{
Choose the \textbf{smallest} set of Real numbers that the number below belongs to.
\[ \sqrt{\frac{-882}{7}} \]
The solution is \( \text{Not a Real number} \), which is option E.\begin{enumerate}[label=\Alph*.]
\item \( \text{Integer} \)

These are the negative and positive counting numbers (..., -3, -2, -1, 0, 1, 2, 3, ...)
\item \( \text{Whole} \)

These are the counting numbers with 0 (0, 1, 2, 3, ...)
\item \( \text{Rational} \)

These are numbers that can be written as fraction of Integers (e.g., -2/3)
\item \( \text{Irrational} \)

These cannot be written as a fraction of Integers.
\item \( \text{Not a Real number} \)

* This is the correct option!
\end{enumerate}

\textbf{General Comment:} First, you \textbf{NEED} to simplify the expression. This question simplifies to $\sqrt{126} i$. 
 
 Be sure you look at the simplified fraction and not just the decimal expansion. Numbers such as 13, 17, and 19 provide \textbf{long but repeating/terminating decimal expansions!} 
 
 The only ways to *not* be a Real number are: dividing by 0 or taking the square root of a negative number. 
 
 Irrational numbers are more than just square root of 3: adding or subtracting values from square root of 3 is also irrational.
}
\litem{
Simplify the expression below into the form $a+bi$. Then, choose the intervals that $a$ and $b$ belong to.
\[ \frac{-45 - 77 i}{-3 - 8 i} \]
The solution is \( 10.29  - 1.77 i \), which is option B.\begin{enumerate}[label=\Alph*.]
\item \( a \in [10, 11] \text{ and } b \in [-129.5, -128] \)

 $10.29  - 129.00 i$, which corresponds to forgetting to multiply the conjugate by the numerator.
\item \( a \in [10, 11] \text{ and } b \in [-3, 0] \)

* $10.29  - 1.77 i$, which is the correct option.
\item \( a \in [-7.5, -5.5] \text{ and } b \in [7.5, 9] \)

 $-6.59  + 8.10 i$, which corresponds to forgetting to multiply the conjugate by the numerator and not computing the conjugate correctly.
\item \( a \in [13.5, 17.5] \text{ and } b \in [8.5, 11] \)

 $15.00  + 9.62 i$, which corresponds to just dividing the first term by the first term and the second by the second.
\item \( a \in [749, 751.5] \text{ and } b \in [-3, 0] \)

 $751.00  - 1.77 i$, which corresponds to forgetting to multiply the conjugate by the numerator and using a plus instead of a minus in the denominator.
\end{enumerate}

\textbf{General Comment:} Multiply the numerator and denominator by the *conjugate* of the denominator, then simplify. For example, if we have $2+3i$, the conjugate is $2-3i$.
}
\litem{
Simplify the expression below into the form $a+bi$. Then, choose the intervals that $a$ and $b$ belong to.
\[ (5 + 4 i)(9 + 10 i) \]
The solution is \( 5 + 86 i \), which is option A.\begin{enumerate}[label=\Alph*.]
\item \( a \in [-5, 10] \text{ and } b \in [81, 91] \)

* $5 + 86 i$, which is the correct option.
\item \( a \in [80, 86] \text{ and } b \in [-16, -10] \)

 $85 - 14 i$, which corresponds to adding a minus sign in the second term.
\item \( a \in [41, 50] \text{ and } b \in [39, 43] \)

 $45 + 40 i$, which corresponds to just multiplying the real terms to get the real part of the solution and the coefficients in the complex terms to get the complex part.
\item \( a \in [80, 86] \text{ and } b \in [14, 15] \)

 $85 + 14 i$, which corresponds to adding a minus sign in the first term.
\item \( a \in [-5, 10] \text{ and } b \in [-92, -82] \)

 $5 - 86 i$, which corresponds to adding a minus sign in both terms.
\end{enumerate}

\textbf{General Comment:} You can treat $i$ as a variable and distribute. Just remember that $i^2=-1$, so you can continue to reduce after you distribute.
}
\end{enumerate}

\end{document}