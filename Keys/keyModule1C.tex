\documentclass{extbook}[14pt]
\usepackage{multicol, enumerate, enumitem, hyperref, color, soul, setspace, parskip, fancyhdr, amssymb, amsthm, amsmath, bbm, latexsym, units, mathtools}
\everymath{\displaystyle}
\usepackage[headsep=0.5cm,headheight=0cm, left=1 in,right= 1 in,top= 1 in,bottom= 1 in]{geometry}
\usepackage{dashrule}  % Package to use the command below to create lines between items
\newcommand{\litem}[1]{\item #1

\rule{\textwidth}{0.4pt}}
\pagestyle{fancy}
\lhead{}
\chead{Answer Key for Progress Quiz 1 Version C}
\rhead{}
\lfoot{3114-1073}
\cfoot{}
\rfoot{Fall 2020}
\begin{document}
\textbf{This key should allow you to understand why you choose the option you did (beyond just getting a question right or wrong). \href{https://xronos.clas.ufl.edu/mac1105spring2020/courseDescriptionAndMisc/Exams/LearningFromResults}{More instructions on how to use this key can be found here}.}

\textbf{If you have a suggestion to make the keys better, \href{https://forms.gle/CZkbZmPbC9XALEE88}{please fill out the short survey here}.}

\textit{Note: This key is auto-generated and may contain issues and/or errors. The keys are reviewed after each exam to ensure grading is done accurately. If there are issues (like duplicate options), they are noted in the offline gradebook. The keys are a work-in-progress to give students as many resources to improve as possible.}

\rule{\textwidth}{0.4pt}

\begin{enumerate}\litem{
Simplify the expression below and choose the interval the simplification is contained within.
\[ 12 - 3^2 + 10 \div 8 * 14 \div 18 \]
The solution is \( 3.972 \), which is option D.\begin{enumerate}[label=\Alph*.]
\item \( [19.97, 21.49] \)

 21.005, which corresponds to two Order of Operations errors.
\item \( [2.65, 3.16] \)

 3.005, which corresponds to an Order of Operations error: not reading left-to-right for multiplication/division.
\item \( [21.43, 22.28] \)

 21.972, which corresponds to an Order of Operations error: multiplying by negative before squaring. For example: $(-3)^2 \neq -3^2$
\item \( [3.82, 5.19] \)

* 3.972, this is the correct option
\item \( \text{None of the above} \)

 You may have gotten this by making an unanticipated error. If you got a value that is not any of the others, please let the coordinator know so they can help you figure out what happened.
\end{enumerate}

\textbf{General Comment:} While you may remember (or were taught) PEMDAS is done in order, it is actually done as P/E/MD/AS. When we are at MD or AS, we read left to right.
}
\litem{
Choose the \textbf{smallest} set of Real numbers that the number below belongs to.
\[ \sqrt{\frac{193600}{400}} \]
The solution is \( \text{Whole} \), which is option A.\begin{enumerate}[label=\Alph*.]
\item \( \text{Whole} \)

* This is the correct option!
\item \( \text{Not a Real number} \)

These are Nonreal Complex numbers \textbf{OR} things that are not numbers (e.g., dividing by 0).
\item \( \text{Integer} \)

These are the negative and positive counting numbers (..., -3, -2, -1, 0, 1, 2, 3, ...)
\item \( \text{Irrational} \)

These cannot be written as a fraction of Integers.
\item \( \text{Rational} \)

These are numbers that can be written as fraction of Integers (e.g., -2/3)
\end{enumerate}

\textbf{General Comment:} First, you \textbf{NEED} to simplify the expression. This question simplifies to $440$. 
 
 Be sure you look at the simplified fraction and not just the decimal expansion. Numbers such as 13, 17, and 19 provide \textbf{long but repeating/terminating decimal expansions!} 
 
 The only ways to *not* be a Real number are: dividing by 0 or taking the square root of a negative number. 
 
 Irrational numbers are more than just square root of 3: adding or subtracting values from square root of 3 is also irrational.
}
\litem{
Simplify the expression below into the form $a+bi$. Then, choose the intervals that $a$ and $b$ belong to.
\[ (-6 - 7 i)(8 + 9 i) \]
The solution is \( 15 - 110 i \), which is option A.\begin{enumerate}[label=\Alph*.]
\item \( a \in [13, 20] \text{ and } b \in [-113, -107] \)

* $15 - 110 i$, which is the correct option.
\item \( a \in [-50, -43] \text{ and } b \in [-65, -57] \)

 $-48 - 63 i$, which corresponds to just multiplying the real terms to get the real part of the solution and the coefficients in the complex terms to get the complex part.
\item \( a \in [13, 20] \text{ and } b \in [106, 113] \)

 $15 + 110 i$, which corresponds to adding a minus sign in both terms.
\item \( a \in [-117, -108] \text{ and } b \in [-2, 0] \)

 $-111 - 2 i$, which corresponds to adding a minus sign in the second term.
\item \( a \in [-117, -108] \text{ and } b \in [1, 11] \)

 $-111 + 2 i$, which corresponds to adding a minus sign in the first term.
\end{enumerate}

\textbf{General Comment:} You can treat $i$ as a variable and distribute. Just remember that $i^2=-1$, so you can continue to reduce after you distribute.
}
\litem{
Choose the \textbf{smallest} set of Complex numbers that the number below belongs to.
\[ -\sqrt{\frac{1872}{9}}+8i^2 \]
The solution is \( \text{Irrational} \), which is option A.\begin{enumerate}[label=\Alph*.]
\item \( \text{Irrational} \)

* This is the correct option!
\item \( \text{Nonreal Complex} \)

This is a Complex number $(a+bi)$ that is not Real (has $i$ as part of the number).
\item \( \text{Pure Imaginary} \)

This is a Complex number $(a+bi)$ that \textbf{only} has an imaginary part like $2i$.
\item \( \text{Rational} \)

These are numbers that can be written as fraction of Integers (e.g., -2/3 + 5)
\item \( \text{Not a Complex Number} \)

This is not a number. The only non-Complex number we know is dividing by 0 as this is not a number!
\end{enumerate}

\textbf{General Comment:} Be sure to simplify $i^2 = -1$. This may remove the imaginary portion for your number. If you are having trouble, you may want to look at the \textit{Subgroups of the Real Numbers} section.
}
\litem{
Choose the \textbf{smallest} set of Real numbers that the number below belongs to.
\[ \sqrt{\frac{43264}{256}} \]
The solution is \( \text{Whole} \), which is option B.\begin{enumerate}[label=\Alph*.]
\item \( \text{Not a Real number} \)

These are Nonreal Complex numbers \textbf{OR} things that are not numbers (e.g., dividing by 0).
\item \( \text{Whole} \)

* This is the correct option!
\item \( \text{Integer} \)

These are the negative and positive counting numbers (..., -3, -2, -1, 0, 1, 2, 3, ...)
\item \( \text{Rational} \)

These are numbers that can be written as fraction of Integers (e.g., -2/3)
\item \( \text{Irrational} \)

These cannot be written as a fraction of Integers.
\end{enumerate}

\textbf{General Comment:} First, you \textbf{NEED} to simplify the expression. This question simplifies to $208$. 
 
 Be sure you look at the simplified fraction and not just the decimal expansion. Numbers such as 13, 17, and 19 provide \textbf{long but repeating/terminating decimal expansions!} 
 
 The only ways to *not* be a Real number are: dividing by 0 or taking the square root of a negative number. 
 
 Irrational numbers are more than just square root of 3: adding or subtracting values from square root of 3 is also irrational.
}
\litem{
Simplify the expression below into the form $a+bi$. Then, choose the intervals that $a$ and $b$ belong to.
\[ \frac{9 + 66 i}{4 + 2 i} \]
The solution is \( 8.40  + 12.30 i \), which is option B.\begin{enumerate}[label=\Alph*.]
\item \( a \in [-6, -4.5] \text{ and } b \in [13, 14.5] \)

 $-4.80  + 14.10 i$, which corresponds to forgetting to multiply the conjugate by the numerator and not computing the conjugate correctly.
\item \( a \in [8, 10.5] \text{ and } b \in [12, 13] \)

* $8.40  + 12.30 i$, which is the correct option.
\item \( a \in [167.5, 168.5] \text{ and } b \in [12, 13] \)

 $168.00  + 12.30 i$, which corresponds to forgetting to multiply the conjugate by the numerator and using a plus instead of a minus in the denominator.
\item \( a \in [8, 10.5] \text{ and } b \in [245.5, 247] \)

 $8.40  + 246.00 i$, which corresponds to forgetting to multiply the conjugate by the numerator.
\item \( a \in [1, 2.5] \text{ and } b \in [32, 33.5] \)

 $2.25  + 33.00 i$, which corresponds to just dividing the first term by the first term and the second by the second.
\end{enumerate}

\textbf{General Comment:} Multiply the numerator and denominator by the *conjugate* of the denominator, then simplify. For example, if we have $2+3i$, the conjugate is $2-3i$.
}
\litem{
Simplify the expression below into the form $a+bi$. Then, choose the intervals that $a$ and $b$ belong to.
\[ \frac{-63 - 88 i}{-3 + 5 i} \]
The solution is \( -7.38  + 17.03 i \), which is option A.\begin{enumerate}[label=\Alph*.]
\item \( a \in [-8, -6] \text{ and } b \in [16.5, 17.5] \)

* $-7.38  + 17.03 i$, which is the correct option.
\item \( a \in [16.5, 19] \text{ and } b \in [-2, 0.5] \)

 $18.50  - 1.50 i$, which corresponds to forgetting to multiply the conjugate by the numerator and not computing the conjugate correctly.
\item \( a \in [-251.5, -250] \text{ and } b \in [16.5, 17.5] \)

 $-251.00  + 17.03 i$, which corresponds to forgetting to multiply the conjugate by the numerator and using a plus instead of a minus in the denominator.
\item \( a \in [20, 21.5] \text{ and } b \in [-18.5, -17] \)

 $21.00  - 17.60 i$, which corresponds to just dividing the first term by the first term and the second by the second.
\item \( a \in [-8, -6] \text{ and } b \in [578.5, 579.5] \)

 $-7.38  + 579.00 i$, which corresponds to forgetting to multiply the conjugate by the numerator.
\end{enumerate}

\textbf{General Comment:} Multiply the numerator and denominator by the *conjugate* of the denominator, then simplify. For example, if we have $2+3i$, the conjugate is $2-3i$.
}
\litem{
Simplify the expression below into the form $a+bi$. Then, choose the intervals that $a$ and $b$ belong to.
\[ (6 - 5 i)(-2 + 4 i) \]
The solution is \( 8 + 34 i \), which is option A.\begin{enumerate}[label=\Alph*.]
\item \( a \in [7, 13] \text{ and } b \in [32, 37] \)

* $8 + 34 i$, which is the correct option.
\item \( a \in [-33, -29] \text{ and } b \in [8, 16] \)

 $-32 + 14 i$, which corresponds to adding a minus sign in the first term.
\item \( a \in [-33, -29] \text{ and } b \in [-17, -13] \)

 $-32 - 14 i$, which corresponds to adding a minus sign in the second term.
\item \( a \in [7, 13] \text{ and } b \in [-38, -32] \)

 $8 - 34 i$, which corresponds to adding a minus sign in both terms.
\item \( a \in [-14, -7] \text{ and } b \in [-26, -17] \)

 $-12 - 20 i$, which corresponds to just multiplying the real terms to get the real part of the solution and the coefficients in the complex terms to get the complex part.
\end{enumerate}

\textbf{General Comment:} You can treat $i$ as a variable and distribute. Just remember that $i^2=-1$, so you can continue to reduce after you distribute.
}
\litem{
Choose the \textbf{smallest} set of Complex numbers that the number below belongs to.
\[ -\sqrt{\frac{1404}{12}}+2i^2 \]
The solution is \( \text{Irrational} \), which is option D.\begin{enumerate}[label=\Alph*.]
\item \( \text{Rational} \)

These are numbers that can be written as fraction of Integers (e.g., -2/3 + 5)
\item \( \text{Pure Imaginary} \)

This is a Complex number $(a+bi)$ that \textbf{only} has an imaginary part like $2i$.
\item \( \text{Nonreal Complex} \)

This is a Complex number $(a+bi)$ that is not Real (has $i$ as part of the number).
\item \( \text{Irrational} \)

* This is the correct option!
\item \( \text{Not a Complex Number} \)

This is not a number. The only non-Complex number we know is dividing by 0 as this is not a number!
\end{enumerate}

\textbf{General Comment:} Be sure to simplify $i^2 = -1$. This may remove the imaginary portion for your number. If you are having trouble, you may want to look at the \textit{Subgroups of the Real Numbers} section.
}
\litem{
Simplify the expression below and choose the interval the simplification is contained within.
\[ 20 - 15 \div 3 * 14 - (2 * 11) \]
The solution is \( -72.000 \), which is option B.\begin{enumerate}[label=\Alph*.]
\item \( [-576, -570] \)

 -572.000, which corresponds to not distributing a negative correctly.
\item \( [-76, -68] \)

* -72.000, which is the correct option.
\item \( [-4.36, 2.64] \)

 -2.357, which corresponds to an Order of Operations error: not reading left-to-right for multiplication/division.
\item \( [40.64, 42.64] \)

 41.643, which corresponds to not distributing addition and subtraction correctly.
\item \( \text{None of the above} \)

 You may have gotten this by making an unanticipated error. If you got a value that is not any of the others, please let the coordinator know so they can help you figure out what happened.
\end{enumerate}

\textbf{General Comment:} While you may remember (or were taught) PEMDAS is done in order, it is actually done as P/E/MD/AS. When we are at MD or AS, we read left to right.
}
\end{enumerate}

\end{document}