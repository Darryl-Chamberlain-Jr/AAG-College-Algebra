\documentclass{extbook}[14pt]
\usepackage{multicol, enumerate, enumitem, hyperref, color, soul, setspace, parskip, fancyhdr, amssymb, amsthm, amsmath, bbm, latexsym, units, mathtools}
\everymath{\displaystyle}
\usepackage[headsep=0.5cm,headheight=0cm, left=1 in,right= 1 in,top= 1 in,bottom= 1 in]{geometry}
\usepackage{dashrule}  % Package to use the command below to create lines between items
\newcommand{\litem}[1]{\item #1

\rule{\textwidth}{0.4pt}}
\pagestyle{fancy}
\lhead{}
\chead{Answer Key for Progress Quiz 5 Version C}
\rhead{}
\lfoot{9912-2038}
\cfoot{}
\rfoot{Spring 2021}
\begin{document}
\textbf{This key should allow you to understand why you choose the option you did (beyond just getting a question right or wrong). \href{https://xronos.clas.ufl.edu/mac1105spring2020/courseDescriptionAndMisc/Exams/LearningFromResults}{More instructions on how to use this key can be found here}.}

\textbf{If you have a suggestion to make the keys better, \href{https://forms.gle/CZkbZmPbC9XALEE88}{please fill out the short survey here}.}

\textit{Note: This key is auto-generated and may contain issues and/or errors. The keys are reviewed after each exam to ensure grading is done accurately. If there are issues (like duplicate options), they are noted in the offline gradebook. The keys are a work-in-progress to give students as many resources to improve as possible.}

\rule{\textwidth}{0.4pt}

\begin{enumerate}\litem{
Choose the \textbf{smallest} set of Complex numbers that the number below belongs to.
\[ \sqrt{\frac{0}{6}}+\sqrt{10}i \]The solution is \( \text{Pure Imaginary} \), which is option D.\begin{enumerate}[label=\Alph*.]
\item \( \text{Rational} \)

These are numbers that can be written as fraction of Integers (e.g., -2/3 + 5)
\item \( \text{Irrational} \)

These cannot be written as a fraction of Integers. Remember: $\pi$ is not an Integer!
\item \( \text{Not a Complex Number} \)

This is not a number. The only non-Complex number we know is dividing by 0 as this is not a number!
\item \( \text{Pure Imaginary} \)

* This is the correct option!
\item \( \text{Nonreal Complex} \)

This is a Complex number $(a+bi)$ that is not Real (has $i$ as part of the number).
\end{enumerate}

\textbf{General Comment:} Be sure to simplify $i^2 = -1$. This may remove the imaginary portion for your number. If you are having trouble, you may want to look at the \textit{Subgroups of the Real Numbers} section.
}
\litem{
Simplify the expression below into the form $a+bi$. Then, choose the intervals that $a$ and $b$ belong to.
\[ (6 - 8 i)(7 - 3 i) \]The solution is \( 18 - 74 i \), which is option D.\begin{enumerate}[label=\Alph*.]
\item \( a \in [62, 72] \text{ and } b \in [35, 42] \)

 $66 + 38 i$, which corresponds to adding a minus sign in the first term.
\item \( a \in [13, 23] \text{ and } b \in [69, 76] \)

 $18 + 74 i$, which corresponds to adding a minus sign in both terms.
\item \( a \in [62, 72] \text{ and } b \in [-38, -34] \)

 $66 - 38 i$, which corresponds to adding a minus sign in the second term.
\item \( a \in [13, 23] \text{ and } b \in [-75, -71] \)

* $18 - 74 i$, which is the correct option.
\item \( a \in [42, 45] \text{ and } b \in [23, 26] \)

 $42 + 24 i$, which corresponds to just multiplying the real terms to get the real part of the solution and the coefficients in the complex terms to get the complex part.
\end{enumerate}

\textbf{General Comment:} You can treat $i$ as a variable and distribute. Just remember that $i^2=-1$, so you can continue to reduce after you distribute.
}
\litem{
Simplify the expression below into the form $a+bi$. Then, choose the intervals that $a$ and $b$ belong to.
\[ \frac{36 + 77 i}{-8 + 2 i} \]The solution is \( -1.97  - 10.12 i \), which is option A.\begin{enumerate}[label=\Alph*.]
\item \( a \in [-2.5, -1] \text{ and } b \in [-10.5, -9] \)

* $-1.97  - 10.12 i$, which is the correct option.
\item \( a \in [-5, -4] \text{ and } b \in [38, 39] \)

 $-4.50  + 38.50 i$, which corresponds to just dividing the first term by the first term and the second by the second.
\item \( a \in [-2.5, -1] \text{ and } b \in [-689.5, -687.5] \)

 $-1.97  - 688.00 i$, which corresponds to forgetting to multiply the conjugate by the numerator.
\item \( a \in [-7.5, -5] \text{ and } b \in [-8.5, -7.5] \)

 $-6.50  - 8.00 i$, which corresponds to forgetting to multiply the conjugate by the numerator and not computing the conjugate correctly.
\item \( a \in [-135, -133.5] \text{ and } b \in [-10.5, -9] \)

 $-134.00  - 10.12 i$, which corresponds to forgetting to multiply the conjugate by the numerator and using a plus instead of a minus in the denominator.
\end{enumerate}

\textbf{General Comment:} Multiply the numerator and denominator by the *conjugate* of the denominator, then simplify. For example, if we have $2+3i$, the conjugate is $2-3i$.
}
\litem{
Simplify the expression below into the form $a+bi$. Then, choose the intervals that $a$ and $b$ belong to.
\[ \frac{36 + 33 i}{2 + 8 i} \]The solution is \( 4.94  - 3.26 i \), which is option E.\begin{enumerate}[label=\Alph*.]
\item \( a \in [4.5, 6] \text{ and } b \in [-222.4, -221.9] \)

 $4.94  - 222.00 i$, which corresponds to forgetting to multiply the conjugate by the numerator.
\item \( a \in [17, 18.5] \text{ and } b \in [3.8, 5.1] \)

 $18.00  + 4.12 i$, which corresponds to just dividing the first term by the first term and the second by the second.
\item \( a \in [-4, -1.5] \text{ and } b \in [4.65, 5.45] \)

 $-2.82  + 5.21 i$, which corresponds to forgetting to multiply the conjugate by the numerator and not computing the conjugate correctly.
\item \( a \in [335, 336.5] \text{ and } b \in [-4, -3.15] \)

 $336.00  - 3.26 i$, which corresponds to forgetting to multiply the conjugate by the numerator and using a plus instead of a minus in the denominator.
\item \( a \in [4.5, 6] \text{ and } b \in [-4, -3.15] \)

* $4.94  - 3.26 i$, which is the correct option.
\end{enumerate}

\textbf{General Comment:} Multiply the numerator and denominator by the *conjugate* of the denominator, then simplify. For example, if we have $2+3i$, the conjugate is $2-3i$.
}
\litem{
Simplify the expression below and choose the interval the simplification is contained within.
\[ 20 - 15 \div 17 * 18 - (12 * 13) \]The solution is \( -151.882 \), which is option B.\begin{enumerate}[label=\Alph*.]
\item \( [-109.47, -95.47] \)

 -102.471, which corresponds to not distributing a negative correctly.
\item \( [-154.88, -149.88] \)

* -151.882, which is the correct option.
\item \( [-137.05, -127.05] \)

 -136.049, which corresponds to an Order of Operations error: not reading left-to-right for multiplication/division.
\item \( [171.95, 180.95] \)

 175.951, which corresponds to not distributing addition and subtraction correctly.
\item \( \text{None of the above} \)

 You may have gotten this by making an unanticipated error. If you got a value that is not any of the others, please let the coordinator know so they can help you figure out what happened.
\end{enumerate}

\textbf{General Comment:} While you may remember (or were taught) PEMDAS is done in order, it is actually done as P/E/MD/AS. When we are at MD or AS, we read left to right.
}
\litem{
Choose the \textbf{smallest} set of Real numbers that the number below belongs to.
\[ \sqrt{\frac{144}{529}} \]The solution is \( \text{Rational} \), which is option D.\begin{enumerate}[label=\Alph*.]
\item \( \text{Not a Real number} \)

These are Nonreal Complex numbers \textbf{OR} things that are not numbers (e.g., dividing by 0).
\item \( \text{Irrational} \)

These cannot be written as a fraction of Integers.
\item \( \text{Integer} \)

These are the negative and positive counting numbers (..., -3, -2, -1, 0, 1, 2, 3, ...)
\item \( \text{Rational} \)

* This is the correct option!
\item \( \text{Whole} \)

These are the counting numbers with 0 (0, 1, 2, 3, ...)
\end{enumerate}

\textbf{General Comment:} First, you \textbf{NEED} to simplify the expression. This question simplifies to $\frac{12}{23}$. 
 
 Be sure you look at the simplified fraction and not just the decimal expansion. Numbers such as 13, 17, and 19 provide \textbf{long but repeating/terminating decimal expansions!} 
 
 The only ways to *not* be a Real number are: dividing by 0 or taking the square root of a negative number. 
 
 Irrational numbers are more than just square root of 3: adding or subtracting values from square root of 3 is also irrational.
}
\litem{
Choose the \textbf{smallest} set of Real numbers that the number below belongs to.
\[ -\sqrt{\frac{57600}{400}} \]The solution is \( \text{Integer} \), which is option E.\begin{enumerate}[label=\Alph*.]
\item \( \text{Not a Real number} \)

These are Nonreal Complex numbers \textbf{OR} things that are not numbers (e.g., dividing by 0).
\item \( \text{Whole} \)

These are the counting numbers with 0 (0, 1, 2, 3, ...)
\item \( \text{Rational} \)

These are numbers that can be written as fraction of Integers (e.g., -2/3)
\item \( \text{Irrational} \)

These cannot be written as a fraction of Integers.
\item \( \text{Integer} \)

* This is the correct option!
\end{enumerate}

\textbf{General Comment:} First, you \textbf{NEED} to simplify the expression. This question simplifies to $-240$. 
 
 Be sure you look at the simplified fraction and not just the decimal expansion. Numbers such as 13, 17, and 19 provide \textbf{long but repeating/terminating decimal expansions!} 
 
 The only ways to *not* be a Real number are: dividing by 0 or taking the square root of a negative number. 
 
 Irrational numbers are more than just square root of 3: adding or subtracting values from square root of 3 is also irrational.
}
\litem{
Choose the \textbf{smallest} set of Complex numbers that the number below belongs to.
\[ \sqrt{\frac{0}{289}}+\sqrt{8}i \]The solution is \( \text{Pure Imaginary} \), which is option A.\begin{enumerate}[label=\Alph*.]
\item \( \text{Pure Imaginary} \)

* This is the correct option!
\item \( \text{Irrational} \)

These cannot be written as a fraction of Integers. Remember: $\pi$ is not an Integer!
\item \( \text{Not a Complex Number} \)

This is not a number. The only non-Complex number we know is dividing by 0 as this is not a number!
\item \( \text{Nonreal Complex} \)

This is a Complex number $(a+bi)$ that is not Real (has $i$ as part of the number).
\item \( \text{Rational} \)

These are numbers that can be written as fraction of Integers (e.g., -2/3 + 5)
\end{enumerate}

\textbf{General Comment:} Be sure to simplify $i^2 = -1$. This may remove the imaginary portion for your number. If you are having trouble, you may want to look at the \textit{Subgroups of the Real Numbers} section.
}
\litem{
Simplify the expression below into the form $a+bi$. Then, choose the intervals that $a$ and $b$ belong to.
\[ (-3 - 6 i)(-7 - 5 i) \]The solution is \( -9 + 57 i \), which is option B.\begin{enumerate}[label=\Alph*.]
\item \( a \in [-9, -1] \text{ and } b \in [-64, -52] \)

 $-9 - 57 i$, which corresponds to adding a minus sign in both terms.
\item \( a \in [-9, -1] \text{ and } b \in [54, 60] \)

* $-9 + 57 i$, which is the correct option.
\item \( a \in [46, 52] \text{ and } b \in [22, 28] \)

 $51 + 27 i$, which corresponds to adding a minus sign in the second term.
\item \( a \in [46, 52] \text{ and } b \in [-27, -23] \)

 $51 - 27 i$, which corresponds to adding a minus sign in the first term.
\item \( a \in [17, 28] \text{ and } b \in [28, 31] \)

 $21 + 30 i$, which corresponds to just multiplying the real terms to get the real part of the solution and the coefficients in the complex terms to get the complex part.
\end{enumerate}

\textbf{General Comment:} You can treat $i$ as a variable and distribute. Just remember that $i^2=-1$, so you can continue to reduce after you distribute.
}
\litem{
Simplify the expression below and choose the interval the simplification is contained within.
\[ 5 - 10^2 + 4 \div 20 * 14 \div 8 \]The solution is \( -94.650 \), which is option A.\begin{enumerate}[label=\Alph*.]
\item \( [-94.89, -94.41] \)

* -94.650, this is the correct option
\item \( [105.16, 105.51] \)

 105.350, which corresponds to an Order of Operations error: multiplying by negative before squaring. For example: $(-3)^2 \neq -3^2$
\item \( [-95.32, -94.66] \)

 -94.998, which corresponds to an Order of Operations error: not reading left-to-right for multiplication/division.
\item \( [104.7, 105.1] \)

 105.002, which corresponds to two Order of Operations errors.
\item \( \text{None of the above} \)

 You may have gotten this by making an unanticipated error. If you got a value that is not any of the others, please let the coordinator know so they can help you figure out what happened.
\end{enumerate}

\textbf{General Comment:} While you may remember (or were taught) PEMDAS is done in order, it is actually done as P/E/MD/AS. When we are at MD or AS, we read left to right.
}
\end{enumerate}

\end{document}