\documentclass{extbook}[14pt]
\usepackage{multicol, enumerate, enumitem, hyperref, color, soul, setspace, parskip, fancyhdr, amssymb, amsthm, amsmath, latexsym, units, mathtools}
\everymath{\displaystyle}
\usepackage[headsep=0.5cm,headheight=0cm, left=1 in,right= 1 in,top= 1 in,bottom= 1 in]{geometry}
\usepackage{dashrule}  % Package to use the command below to create lines between items
\newcommand{\litem}[1]{\item #1

\rule{\textwidth}{0.4pt}}
\pagestyle{fancy}
\lhead{}
\chead{Answer Key for Module1 Version C}
\rhead{}
\lfoot{6227-9062}
\cfoot{}
\rfoot{testing}
\begin{document}
\textbf{This key should allow you to understand why you choose the option you did (beyond just getting a question right or wrong). \href{https://xronos.clas.ufl.edu/mac1105spring2020/courseDescriptionAndMisc/Exams/LearningFromResults}{More instructions on how to use this key can be found here}.}

\textbf{If you have a suggestion to make the keys better, \href{https://forms.gle/CZkbZmPbC9XALEE88}{please fill out the short survey here}.}

\textit{Note: This key is auto-generated and may contain issues and/or errors. The keys are reviewed after each exam to ensure grading is done accurately. If there are issues (like duplicate options), they are noted in the offline gradebook. The keys are a work-in-progress to give students as many resources to improve as possible.}

\rule{\textwidth}{0.4pt}

\begin{enumerate}\litem{
Choose the \textbf{smallest} set of Complex numbers that the number below belongs to.
\[ \sqrt{\frac{-1404}{0}}+\sqrt{126} \]The solution is \( \text{Not a Complex Number} \), which is option E.\begin{enumerate}[label=\Alph*.]
\item \( \text{Rational} \)

These are numbers that can be written as fraction of Integers (e.g., -2/3 + 5)
\item \( \text{Irrational} \)

These cannot be written as a fraction of Integers. Remember: $\pi$ is not an Integer!
\item \( \text{Pure Imaginary} \)

This is a Complex number $(a+bi)$ that \textbf{only} has an imaginary part like $2i$.
\item \( \text{Nonreal Complex} \)

This is a Complex number $(a+bi)$ that is not Real (has $i$ as part of the number).
\item \( \text{Not a Complex Number} \)

* This is the correct option!
\end{enumerate}

\textbf{General Comment:} Be sure to simplify $i^2 = -1$. This may remove the imaginary portion for your number. If you are having trouble, you may want to look at the \textit{Subgroups of the Real Numbers} section.
}
\litem{
Simplify the expression below into the form $a+bi$. Then, choose the intervals that $a$ and $b$ belong to.
\[ \frac{63 - 33 i}{-5 - i} \]The solution is \( -10.85  + 8.77 i \), which is option C.\begin{enumerate}[label=\Alph*.]
\item \( a \in [-13.45, -13.3] \text{ and } b \in [3.5, 5] \)

 $-13.38  + 3.92 i$, which corresponds to forgetting to multiply the conjugate by the numerator and not computing the conjugate correctly.
\item \( a \in [-13.15, -11.9] \text{ and } b \in [32.5, 33.5] \)

 $-12.60  + 33.00 i$, which corresponds to just dividing the first term by the first term and the second by the second.
\item \( a \in [-11.45, -10.5] \text{ and } b \in [8.5, 10.5] \)

* $-10.85  + 8.77 i$, which is the correct option.
\item \( a \in [-11.45, -10.5] \text{ and } b \in [227.5, 229.5] \)

 $-10.85  + 228.00 i$, which corresponds to forgetting to multiply the conjugate by the numerator.
\item \( a \in [-282.1, -281.3] \text{ and } b \in [8.5, 10.5] \)

 $-282.00  + 8.77 i$, which corresponds to forgetting to multiply the conjugate by the numerator and using a plus instead of a minus in the denominator.
\end{enumerate}

\textbf{General Comment:} Multiply the numerator and denominator by the *conjugate* of the denominator, then simplify. For example, if we have $2+3i$, the conjugate is $2-3i$.
}
\litem{
Simplify the expression below into the form $a+bi$. Then, choose the intervals that $a$ and $b$ belong to.
\[ (-5 - 2 i)(-6 - 8 i) \]The solution is \( 14 + 52 i \), which is option C.\begin{enumerate}[label=\Alph*.]
\item \( a \in [44, 52] \text{ and } b \in [-32, -21] \)

 $46 - 28 i$, which corresponds to adding a minus sign in the second term.
\item \( a \in [44, 52] \text{ and } b \in [28, 30] \)

 $46 + 28 i$, which corresponds to adding a minus sign in the first term.
\item \( a \in [13, 20] \text{ and } b \in [48, 57] \)

* $14 + 52 i$, which is the correct option.
\item \( a \in [13, 20] \text{ and } b \in [-52, -50] \)

 $14 - 52 i$, which corresponds to adding a minus sign in both terms.
\item \( a \in [27, 33] \text{ and } b \in [16, 18] \)

 $30 + 16 i$, which corresponds to just multiplying the real terms to get the real part of the solution and the coefficients in the complex terms to get the complex part.
\end{enumerate}

\textbf{General Comment:} You can treat $i$ as a variable and distribute. Just remember that $i^2=-1$, so you can continue to reduce after you distribute.
}
\litem{
Simplify the expression below into the form $a+bi$. Then, choose the intervals that $a$ and $b$ belong to.
\[ \frac{9 + 55 i}{6 + 2 i} \]The solution is \( 4.10  + 7.80 i \), which is option B.\begin{enumerate}[label=\Alph*.]
\item \( a \in [163.5, 164.5] \text{ and } b \in [6.5, 8] \)

 $164.00  + 7.80 i$, which corresponds to forgetting to multiply the conjugate by the numerator and using a plus instead of a minus in the denominator.
\item \( a \in [3.5, 4.5] \text{ and } b \in [6.5, 8] \)

* $4.10  + 7.80 i$, which is the correct option.
\item \( a \in [1, 2.5] \text{ and } b \in [26, 28.5] \)

 $1.50  + 27.50 i$, which corresponds to just dividing the first term by the first term and the second by the second.
\item \( a \in [-2.5, 0] \text{ and } b \in [8.5, 9] \)

 $-1.40  + 8.70 i$, which corresponds to forgetting to multiply the conjugate by the numerator and not computing the conjugate correctly.
\item \( a \in [3.5, 4.5] \text{ and } b \in [311.5, 312.5] \)

 $4.10  + 312.00 i$, which corresponds to forgetting to multiply the conjugate by the numerator.
\end{enumerate}

\textbf{General Comment:} Multiply the numerator and denominator by the *conjugate* of the denominator, then simplify. For example, if we have $2+3i$, the conjugate is $2-3i$.
}
\litem{
Simplify the expression below and choose the interval the simplification is contained within.
\[ 18 - 20 \div 1 * 17 - (3 * 4) \]The solution is \( -334.000 \), which is option B.\begin{enumerate}[label=\Alph*.]
\item \( [26.82, 31.82] \)

 28.824, which corresponds to not distributing addition and subtraction correctly.
\item \( [-339, -333] \)

* -334.000, which is the correct option.
\item \( [-1302, -1298] \)

 -1300.000, which corresponds to not distributing a negative correctly.
\item \( [2.82, 5.82] \)

 4.824, which corresponds to an Order of Operations error: not reading left-to-right for multiplication/division.
\item \( \text{None of the above} \)

 You may have gotten this by making an unanticipated error. If you got a value that is not any of the others, please let the coordinator know so they can help you figure out what happened.
\end{enumerate}

\textbf{General Comment:} While you may remember (or were taught) PEMDAS is done in order, it is actually done as P/E/MD/AS. When we are at MD or AS, we read left to right.
}
\litem{
Choose the \textbf{smallest} set of Real numbers that the number below belongs to.
\[ \sqrt{\frac{540}{12}} \]The solution is \( \text{Irrational} \), which is option B.\begin{enumerate}[label=\Alph*.]
\item \( \text{Whole} \)

These are the counting numbers with 0 (0, 1, 2, 3, ...)
\item \( \text{Irrational} \)

* This is the correct option!
\item \( \text{Not a Real number} \)

These are Nonreal Complex numbers \textbf{OR} things that are not numbers (e.g., dividing by 0).
\item \( \text{Rational} \)

These are numbers that can be written as fraction of Integers (e.g., -2/3)
\item \( \text{Integer} \)

These are the negative and positive counting numbers (..., -3, -2, -1, 0, 1, 2, 3, ...)
\end{enumerate}

\textbf{General Comment:} First, you \textbf{NEED} to simplify the expression. This question simplifies to $\sqrt{45}$. 
 
 Be sure you look at the simplified fraction and not just the decimal expansion. Numbers such as 13, 17, and 19 provide \textbf{long but repeating/terminating decimal expansions!} 
 
 The only ways to *not* be a Real number are: dividing by 0 or taking the square root of a negative number. 
 
 Irrational numbers are more than just square root of 3: adding or subtracting values from square root of 3 is also irrational.
}
\litem{
Choose the \textbf{smallest} set of Complex numbers that the number below belongs to.
\[ \sqrt{\frac{-693}{7}}+\sqrt{0}i \]The solution is \( \text{Pure Imaginary} \), which is option A.\begin{enumerate}[label=\Alph*.]
\item \( \text{Pure Imaginary} \)

* This is the correct option!
\item \( \text{Irrational} \)

These cannot be written as a fraction of Integers. Remember: $\pi$ is not an Integer!
\item \( \text{Not a Complex Number} \)

This is not a number. The only non-Complex number we know is dividing by 0 as this is not a number!
\item \( \text{Rational} \)

These are numbers that can be written as fraction of Integers (e.g., -2/3 + 5)
\item \( \text{Nonreal Complex} \)

This is a Complex number $(a+bi)$ that is not Real (has $i$ as part of the number).
\end{enumerate}

\textbf{General Comment:} Be sure to simplify $i^2 = -1$. This may remove the imaginary portion for your number. If you are having trouble, you may want to look at the \textit{Subgroups of the Real Numbers} section.
}
\litem{
Simplify the expression below into the form $a+bi$. Then, choose the intervals that $a$ and $b$ belong to.
\[ (-7 - 3 i)(4 - 2 i) \]The solution is \( -34 + 2 i \), which is option D.\begin{enumerate}[label=\Alph*.]
\item \( a \in [-32, -26] \text{ and } b \in [4, 8.1] \)

 $-28 + 6 i$, which corresponds to just multiplying the real terms to get the real part of the solution and the coefficients in the complex terms to get the complex part.
\item \( a \in [-36, -29] \text{ and } b \in [-3.7, -1.2] \)

 $-34 - 2 i$, which corresponds to adding a minus sign in both terms.
\item \( a \in [-27, -20] \text{ and } b \in [-26.7, -24.5] \)

 $-22 - 26 i$, which corresponds to adding a minus sign in the second term.
\item \( a \in [-36, -29] \text{ and } b \in [-1.3, 3.1] \)

* $-34 + 2 i$, which is the correct option.
\item \( a \in [-27, -20] \text{ and } b \in [23.8, 27] \)

 $-22 + 26 i$, which corresponds to adding a minus sign in the first term.
\end{enumerate}

\textbf{General Comment:} You can treat $i$ as a variable and distribute. Just remember that $i^2=-1$, so you can continue to reduce after you distribute.
}
\litem{
Choose the \textbf{smallest} set of Real numbers that the number below belongs to.
\[ \sqrt{\frac{196}{169}} \]The solution is \( \text{Rational} \), which is option D.\begin{enumerate}[label=\Alph*.]
\item \( \text{Irrational} \)

These cannot be written as a fraction of Integers.
\item \( \text{Integer} \)

These are the negative and positive counting numbers (..., -3, -2, -1, 0, 1, 2, 3, ...)
\item \( \text{Whole} \)

These are the counting numbers with 0 (0, 1, 2, 3, ...)
\item \( \text{Rational} \)

* This is the correct option!
\item \( \text{Not a Real number} \)

These are Nonreal Complex numbers \textbf{OR} things that are not numbers (e.g., dividing by 0).
\end{enumerate}

\textbf{General Comment:} First, you \textbf{NEED} to simplify the expression. This question simplifies to $\frac{14}{13}$. 
 
 Be sure you look at the simplified fraction and not just the decimal expansion. Numbers such as 13, 17, and 19 provide \textbf{long but repeating/terminating decimal expansions!} 
 
 The only ways to *not* be a Real number are: dividing by 0 or taking the square root of a negative number. 
 
 Irrational numbers are more than just square root of 3: adding or subtracting values from square root of 3 is also irrational.
}
\litem{
Simplify the expression below and choose the interval the simplification is contained within.
\[ 1 - 17 \div 4 * 16 - (10 * 12) \]The solution is \( -187.000 \), which is option D.\begin{enumerate}[label=\Alph*.]
\item \( [-121.27, -118.27] \)

 -119.266, which corresponds to an Order of Operations error: not reading left-to-right for multiplication/division.
\item \( [114.73, 123.73] \)

 120.734, which corresponds to not distributing addition and subtraction correctly.
\item \( [-926, -920] \)

 -924.000, which corresponds to not distributing a negative correctly.
\item \( [-188, -182] \)

* -187.000, which is the correct option.
\item \( \text{None of the above} \)

 You may have gotten this by making an unanticipated error. If you got a value that is not any of the others, please let the coordinator know so they can help you figure out what happened.
\end{enumerate}

\textbf{General Comment:} While you may remember (or were taught) PEMDAS is done in order, it is actually done as P/E/MD/AS. When we are at MD or AS, we read left to right.
}
\end{enumerate}

\end{document}