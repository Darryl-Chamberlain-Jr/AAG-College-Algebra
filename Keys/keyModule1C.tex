\documentclass{extbook}[14pt]
\usepackage{multicol, enumerate, enumitem, hyperref, color, soul, setspace, parskip, fancyhdr, amssymb, amsthm, amsmath, bbm, latexsym, units, mathtools}
\everymath{\displaystyle}
\usepackage[headsep=0.5cm,headheight=0cm, left=1 in,right= 1 in,top= 1 in,bottom= 1 in]{geometry}
\usepackage{dashrule}  % Package to use the command below to create lines between items
\newcommand{\litem}[1]{\item #1

\rule{\textwidth}{0.4pt}}
\pagestyle{fancy}
\lhead{}
\chead{Answer Key for Progress Quiz 2 Version C}
\rhead{}
\lfoot{7862-5421}
\cfoot{}
\rfoot{Spring 2021}
\begin{document}
\textbf{This key should allow you to understand why you choose the option you did (beyond just getting a question right or wrong). \href{https://xronos.clas.ufl.edu/mac1105spring2020/courseDescriptionAndMisc/Exams/LearningFromResults}{More instructions on how to use this key can be found here}.}

\textbf{If you have a suggestion to make the keys better, \href{https://forms.gle/CZkbZmPbC9XALEE88}{please fill out the short survey here}.}

\textit{Note: This key is auto-generated and may contain issues and/or errors. The keys are reviewed after each exam to ensure grading is done accurately. If there are issues (like duplicate options), they are noted in the offline gradebook. The keys are a work-in-progress to give students as many resources to improve as possible.}

\rule{\textwidth}{0.4pt}

\begin{enumerate}\litem{
Choose the \textbf{smallest} set of Real numbers that the number below belongs to.
\[ -\sqrt{\frac{63504}{196}} \]

The solution is \( \text{Integer} \), which is option D.\begin{enumerate}[label=\Alph*.]
\item \( \text{Irrational} \)

These cannot be written as a fraction of Integers.
\item \( \text{Not a Real number} \)

These are Nonreal Complex numbers \textbf{OR} things that are not numbers (e.g., dividing by 0).
\item \( \text{Whole} \)

These are the counting numbers with 0 (0, 1, 2, 3, ...)
\item \( \text{Integer} \)

* This is the correct option!
\item \( \text{Rational} \)

These are numbers that can be written as fraction of Integers (e.g., -2/3)
\end{enumerate}

\textbf{General Comment:} First, you \textbf{NEED} to simplify the expression. This question simplifies to $-252$. 
 
 Be sure you look at the simplified fraction and not just the decimal expansion. Numbers such as 13, 17, and 19 provide \textbf{long but repeating/terminating decimal expansions!} 
 
 The only ways to *not* be a Real number are: dividing by 0 or taking the square root of a negative number. 
 
 Irrational numbers are more than just square root of 3: adding or subtracting values from square root of 3 is also irrational.
}
\litem{
Simplify the expression below and choose the interval the simplification is contained within.
\[ 11 - 4 \div 5 * 6 - (16 * 20) \]

The solution is \( -313.800 \), which is option D.\begin{enumerate}[label=\Alph*.]
\item \( [327.87, 331.87] \)

 330.867, which corresponds to not distributing addition and subtraction correctly.
\item \( [-312.13, -302.13] \)

 -309.133, which corresponds to an Order of Operations error: not reading left-to-right for multiplication/division.
\item \( [-196, -194] \)

 -196.000, which corresponds to not distributing a negative correctly.
\item \( [-317.8, -312.8] \)

* -313.800, which is the correct option.
\item \( \text{None of the above} \)

 You may have gotten this by making an unanticipated error. If you got a value that is not any of the others, please let the coordinator know so they can help you figure out what happened.
\end{enumerate}

\textbf{General Comment:} While you may remember (or were taught) PEMDAS is done in order, it is actually done as P/E/MD/AS. When we are at MD or AS, we read left to right.
}
\litem{
Choose the \textbf{smallest} set of Real numbers that the number below belongs to.
\[ -\sqrt{\frac{1547}{7}} \]

The solution is \( \text{Irrational} \), which is option A.\begin{enumerate}[label=\Alph*.]
\item \( \text{Irrational} \)

* This is the correct option!
\item \( \text{Not a Real number} \)

These are Nonreal Complex numbers \textbf{OR} things that are not numbers (e.g., dividing by 0).
\item \( \text{Integer} \)

These are the negative and positive counting numbers (..., -3, -2, -1, 0, 1, 2, 3, ...)
\item \( \text{Whole} \)

These are the counting numbers with 0 (0, 1, 2, 3, ...)
\item \( \text{Rational} \)

These are numbers that can be written as fraction of Integers (e.g., -2/3)
\end{enumerate}

\textbf{General Comment:} First, you \textbf{NEED} to simplify the expression. This question simplifies to $-\sqrt{221}$. 
 
 Be sure you look at the simplified fraction and not just the decimal expansion. Numbers such as 13, 17, and 19 provide \textbf{long but repeating/terminating decimal expansions!} 
 
 The only ways to *not* be a Real number are: dividing by 0 or taking the square root of a negative number. 
 
 Irrational numbers are more than just square root of 3: adding or subtracting values from square root of 3 is also irrational.
}
\litem{
Simplify the expression below into the form $a+bi$. Then, choose the intervals that $a$ and $b$ belong to.
\[ \frac{-18 + 11 i}{4 + 5 i} \]

The solution is \( -0.41  + 3.27 i \), which is option B.\begin{enumerate}[label=\Alph*.]
\item \( a \in [-17.5, -14.5] \text{ and } b \in [2.5, 4] \)

 $-17.00  + 3.27 i$, which corresponds to forgetting to multiply the conjugate by the numerator and using a plus instead of a minus in the denominator.
\item \( a \in [-1.5, 0] \text{ and } b \in [2.5, 4] \)

* $-0.41  + 3.27 i$, which is the correct option.
\item \( a \in [-1.5, 0] \text{ and } b \in [133.5, 134.5] \)

 $-0.41  + 134.00 i$, which corresponds to forgetting to multiply the conjugate by the numerator.
\item \( a \in [-5.5, -4] \text{ and } b \in [1.5, 3] \)

 $-4.50  + 2.20 i$, which corresponds to just dividing the first term by the first term and the second by the second.
\item \( a \in [-3.5, -2.5] \text{ and } b \in [-2, -0.5] \)

 $-3.10  - 1.12 i$, which corresponds to forgetting to multiply the conjugate by the numerator and not computing the conjugate correctly.
\end{enumerate}

\textbf{General Comment:} Multiply the numerator and denominator by the *conjugate* of the denominator, then simplify. For example, if we have $2+3i$, the conjugate is $2-3i$.
}
\litem{
Simplify the expression below into the form $a+bi$. Then, choose the intervals that $a$ and $b$ belong to.
\[ \frac{9 - 66 i}{2 + 3 i} \]

The solution is \( -13.85  - 12.23 i \), which is option D.\begin{enumerate}[label=\Alph*.]
\item \( a \in [-181, -179.5] \text{ and } b \in [-13, -12] \)

 $-180.00  - 12.23 i$, which corresponds to forgetting to multiply the conjugate by the numerator and using a plus instead of a minus in the denominator.
\item \( a \in [16, 17.5] \text{ and } b \in [-8.5, -7.5] \)

 $16.62  - 8.08 i$, which corresponds to forgetting to multiply the conjugate by the numerator and not computing the conjugate correctly.
\item \( a \in [3.5, 5.5] \text{ and } b \in [-22.5, -21] \)

 $4.50  - 22.00 i$, which corresponds to just dividing the first term by the first term and the second by the second.
\item \( a \in [-15, -13.5] \text{ and } b \in [-13, -12] \)

* $-13.85  - 12.23 i$, which is the correct option.
\item \( a \in [-15, -13.5] \text{ and } b \in [-159.5, -158.5] \)

 $-13.85  - 159.00 i$, which corresponds to forgetting to multiply the conjugate by the numerator.
\end{enumerate}

\textbf{General Comment:} Multiply the numerator and denominator by the *conjugate* of the denominator, then simplify. For example, if we have $2+3i$, the conjugate is $2-3i$.
}
\litem{
Simplify the expression below and choose the interval the simplification is contained within.
\[ 4 - 20 \div 5 * 2 - (12 * 14) \]

The solution is \( -172.000 \), which is option C.\begin{enumerate}[label=\Alph*.]
\item \( [-226.2, -222.1] \)

 -224.000, which corresponds to not distributing a negative correctly.
\item \( [-167, -165.4] \)

 -166.000, which corresponds to an Order of Operations error: not reading left-to-right for multiplication/division.
\item \( [-173.7, -171.1] \)

* -172.000, which is the correct option.
\item \( [169.6, 170.5] \)

 170.000, which corresponds to not distributing addition and subtraction correctly.
\item \( \text{None of the above} \)

 You may have gotten this by making an unanticipated error. If you got a value that is not any of the others, please let the coordinator know so they can help you figure out what happened.
\end{enumerate}

\textbf{General Comment:} While you may remember (or were taught) PEMDAS is done in order, it is actually done as P/E/MD/AS. When we are at MD or AS, we read left to right.
}
\litem{
Choose the \textbf{smallest} set of Complex numbers that the number below belongs to.
\[ \frac{18}{-14}+\sqrt{-64}i \]

The solution is \( \text{Rational} \), which is option A.\begin{enumerate}[label=\Alph*.]
\item \( \text{Rational} \)

* This is the correct option!
\item \( \text{Not a Complex Number} \)

This is not a number. The only non-Complex number we know is dividing by 0 as this is not a number!
\item \( \text{Irrational} \)

These cannot be written as a fraction of Integers. Remember: $\pi$ is not an Integer!
\item \( \text{Nonreal Complex} \)

This is a Complex number $(a+bi)$ that is not Real (has $i$ as part of the number).
\item \( \text{Pure Imaginary} \)

This is a Complex number $(a+bi)$ that \textbf{only} has an imaginary part like $2i$.
\end{enumerate}

\textbf{General Comment:} Be sure to simplify $i^2 = -1$. This may remove the imaginary portion for your number. If you are having trouble, you may want to look at the \textit{Subgroups of the Real Numbers} section.
}
\litem{
Simplify the expression below into the form $a+bi$. Then, choose the intervals that $a$ and $b$ belong to.
\[ (6 + 5 i)(-7 + 4 i) \]

The solution is \( -62 - 11 i \), which is option E.\begin{enumerate}[label=\Alph*.]
\item \( a \in [-25, -19] \text{ and } b \in [-66, -56] \)

 $-22 - 59 i$, which corresponds to adding a minus sign in the second term.
\item \( a \in [-25, -19] \text{ and } b \in [58, 62] \)

 $-22 + 59 i$, which corresponds to adding a minus sign in the first term.
\item \( a \in [-42, -41] \text{ and } b \in [16, 25] \)

 $-42 + 20 i$, which corresponds to just multiplying the real terms to get the real part of the solution and the coefficients in the complex terms to get the complex part.
\item \( a \in [-63, -56] \text{ and } b \in [10, 16] \)

 $-62 + 11 i$, which corresponds to adding a minus sign in both terms.
\item \( a \in [-63, -56] \text{ and } b \in [-15, -7] \)

* $-62 - 11 i$, which is the correct option.
\end{enumerate}

\textbf{General Comment:} You can treat $i$ as a variable and distribute. Just remember that $i^2=-1$, so you can continue to reduce after you distribute.
}
\litem{
Choose the \textbf{smallest} set of Complex numbers that the number below belongs to.
\[ \sqrt{\frac{-2145}{0}} i+\sqrt{182}i \]

The solution is \( \text{Not a Complex Number} \), which is option A.\begin{enumerate}[label=\Alph*.]
\item \( \text{Not a Complex Number} \)

* This is the correct option!
\item \( \text{Nonreal Complex} \)

This is a Complex number $(a+bi)$ that is not Real (has $i$ as part of the number).
\item \( \text{Pure Imaginary} \)

This is a Complex number $(a+bi)$ that \textbf{only} has an imaginary part like $2i$.
\item \( \text{Rational} \)

These are numbers that can be written as fraction of Integers (e.g., -2/3 + 5)
\item \( \text{Irrational} \)

These cannot be written as a fraction of Integers. Remember: $\pi$ is not an Integer!
\end{enumerate}

\textbf{General Comment:} Be sure to simplify $i^2 = -1$. This may remove the imaginary portion for your number. If you are having trouble, you may want to look at the \textit{Subgroups of the Real Numbers} section.
}
\litem{
Simplify the expression below into the form $a+bi$. Then, choose the intervals that $a$ and $b$ belong to.
\[ (7 - 9 i)(3 - 5 i) \]

The solution is \( -24 - 62 i \), which is option A.\begin{enumerate}[label=\Alph*.]
\item \( a \in [-27, -17] \text{ and } b \in [-65, -61] \)

* $-24 - 62 i$, which is the correct option.
\item \( a \in [62, 67] \text{ and } b \in [8, 9] \)

 $66 + 8 i$, which corresponds to adding a minus sign in the second term.
\item \( a \in [62, 67] \text{ and } b \in [-13, -7] \)

 $66 - 8 i$, which corresponds to adding a minus sign in the first term.
\item \( a \in [21, 22] \text{ and } b \in [40, 52] \)

 $21 + 45 i$, which corresponds to just multiplying the real terms to get the real part of the solution and the coefficients in the complex terms to get the complex part.
\item \( a \in [-27, -17] \text{ and } b \in [59, 64] \)

 $-24 + 62 i$, which corresponds to adding a minus sign in both terms.
\end{enumerate}

\textbf{General Comment:} You can treat $i$ as a variable and distribute. Just remember that $i^2=-1$, so you can continue to reduce after you distribute.
}
\end{enumerate}

\end{document}