\documentclass{extbook}[14pt]
\usepackage{multicol, enumerate, enumitem, hyperref, color, soul, setspace, parskip, fancyhdr, amssymb, amsthm, amsmath, bbm, latexsym, units, mathtools}
\everymath{\displaystyle}
\usepackage[headsep=0.5cm,headheight=0cm, left=1 in,right= 1 in,top= 1 in,bottom= 1 in]{geometry}
\usepackage{dashrule}  % Package to use the command below to create lines between items
\newcommand{\litem}[1]{\item #1

\rule{\textwidth}{0.4pt}}
\pagestyle{fancy}
\lhead{}
\chead{Answer Key for Progress Quiz 9 Version C}
\rhead{}
\lfoot{8590-6105}
\cfoot{}
\rfoot{Fall 2020}
\begin{document}
\textbf{This key should allow you to understand why you choose the option you did (beyond just getting a question right or wrong). \href{https://xronos.clas.ufl.edu/mac1105spring2020/courseDescriptionAndMisc/Exams/LearningFromResults}{More instructions on how to use this key can be found here}.}

\textbf{If you have a suggestion to make the keys better, \href{https://forms.gle/CZkbZmPbC9XALEE88}{please fill out the short survey here}.}

\textit{Note: This key is auto-generated and may contain issues and/or errors. The keys are reviewed after each exam to ensure grading is done accurately. If there are issues (like duplicate options), they are noted in the offline gradebook. The keys are a work-in-progress to give students as many resources to improve as possible.}

\rule{\textwidth}{0.4pt}

\begin{enumerate}\litem{
Choose the \textbf{smallest} set of Real numbers that the number below belongs to.
\[ \sqrt{\frac{17424}{121}} \]

The solution is \( \text{Whole} \), which is option A.\begin{enumerate}[label=\Alph*.]
\item \( \text{Whole} \)

* This is the correct option!
\item \( \text{Not a Real number} \)

These are Nonreal Complex numbers \textbf{OR} things that are not numbers (e.g., dividing by 0).
\item \( \text{Integer} \)

These are the negative and positive counting numbers (..., -3, -2, -1, 0, 1, 2, 3, ...)
\item \( \text{Rational} \)

These are numbers that can be written as fraction of Integers (e.g., -2/3)
\item \( \text{Irrational} \)

These cannot be written as a fraction of Integers.
\end{enumerate}

\textbf{General Comment:} First, you \textbf{NEED} to simplify the expression. This question simplifies to $132$. 
 
 Be sure you look at the simplified fraction and not just the decimal expansion. Numbers such as 13, 17, and 19 provide \textbf{long but repeating/terminating decimal expansions!} 
 
 The only ways to *not* be a Real number are: dividing by 0 or taking the square root of a negative number. 
 
 Irrational numbers are more than just square root of 3: adding or subtracting values from square root of 3 is also irrational.
}
\litem{
Simplify the expression below into the form $a+bi$. Then, choose the intervals that $a$ and $b$ belong to.
\[ (-6 - 2 i)(4 - 7 i) \]

The solution is \( -38 + 34 i \), which is option E.\begin{enumerate}[label=\Alph*.]
\item \( a \in [-42, -33] \text{ and } b \in [-35, -32] \)

 $-38 - 34 i$, which corresponds to adding a minus sign in both terms.
\item \( a \in [-30, -18] \text{ and } b \in [12, 25] \)

 $-24 + 14 i$, which corresponds to just multiplying the real terms to get the real part of the solution and the coefficients in the complex terms to get the complex part.
\item \( a \in [-10, -8] \text{ and } b \in [47, 53] \)

 $-10 + 50 i$, which corresponds to adding a minus sign in the first term.
\item \( a \in [-10, -8] \text{ and } b \in [-51, -42] \)

 $-10 - 50 i$, which corresponds to adding a minus sign in the second term.
\item \( a \in [-42, -33] \text{ and } b \in [33, 35] \)

* $-38 + 34 i$, which is the correct option.
\end{enumerate}

\textbf{General Comment:} You can treat $i$ as a variable and distribute. Just remember that $i^2=-1$, so you can continue to reduce after you distribute.
}
\litem{
Choose the \textbf{smallest} set of Complex numbers that the number below belongs to.
\[ \sqrt{\frac{-1001}{7}}+\sqrt{182} \]

The solution is \( \text{Nonreal Complex} \), which is option A.\begin{enumerate}[label=\Alph*.]
\item \( \text{Nonreal Complex} \)

* This is the correct option!
\item \( \text{Pure Imaginary} \)

This is a Complex number $(a+bi)$ that \textbf{only} has an imaginary part like $2i$.
\item \( \text{Not a Complex Number} \)

This is not a number. The only non-Complex number we know is dividing by 0 as this is not a number!
\item \( \text{Irrational} \)

These cannot be written as a fraction of Integers. Remember: $\pi$ is not an Integer!
\item \( \text{Rational} \)

These are numbers that can be written as fraction of Integers (e.g., -2/3 + 5)
\end{enumerate}

\textbf{General Comment:} Be sure to simplify $i^2 = -1$. This may remove the imaginary portion for your number. If you are having trouble, you may want to look at the \textit{Subgroups of the Real Numbers} section.
}
\litem{
Simplify the expression below into the form $a+bi$. Then, choose the intervals that $a$ and $b$ belong to.
\[ \frac{-45 - 33 i}{2 - i} \]

The solution is \( -11.40  - 22.20 i \), which is option C.\begin{enumerate}[label=\Alph*.]
\item \( a \in [-26, -24] \text{ and } b \in [-5, -3] \)

 $-24.60  - 4.20 i$, which corresponds to forgetting to multiply the conjugate by the numerator and not computing the conjugate correctly.
\item \( a \in [-57.5, -56] \text{ and } b \in [-23, -22] \)

 $-57.00  - 22.20 i$, which corresponds to forgetting to multiply the conjugate by the numerator and using a plus instead of a minus in the denominator.
\item \( a \in [-11.5, -11] \text{ and } b \in [-23, -22] \)

* $-11.40  - 22.20 i$, which is the correct option.
\item \( a \in [-23, -21.5] \text{ and } b \in [32, 34] \)

 $-22.50  + 33.00 i$, which corresponds to just dividing the first term by the first term and the second by the second.
\item \( a \in [-11.5, -11] \text{ and } b \in [-111.5, -109.5] \)

 $-11.40  - 111.00 i$, which corresponds to forgetting to multiply the conjugate by the numerator.
\end{enumerate}

\textbf{General Comment:} Multiply the numerator and denominator by the *conjugate* of the denominator, then simplify. For example, if we have $2+3i$, the conjugate is $2-3i$.
}
\litem{
Simplify the expression below and choose the interval the simplification is contained within.
\[ 1 - 13 \div 4 * 11 - (20 * 16) \]

The solution is \( -354.750 \), which is option B.\begin{enumerate}[label=\Alph*.]
\item \( [-876, -874] \)

 -876.000, which corresponds to not distributing a negative correctly.
\item \( [-355.75, -349.75] \)

* -354.750, which is the correct option.
\item \( [318.7, 322.7] \)

 320.705, which corresponds to not distributing addition and subtraction correctly.
\item \( [-321.3, -311.3] \)

 -319.295, which corresponds to an Order of Operations error: not reading left-to-right for multiplication/division.
\item \( \text{None of the above} \)

 You may have gotten this by making an unanticipated error. If you got a value that is not any of the others, please let the coordinator know so they can help you figure out what happened.
\end{enumerate}

\textbf{General Comment:} While you may remember (or were taught) PEMDAS is done in order, it is actually done as P/E/MD/AS. When we are at MD or AS, we read left to right.
}
\litem{
Simplify the expression below into the form $a+bi$. Then, choose the intervals that $a$ and $b$ belong to.
\[ (2 + 9 i)(-6 - 7 i) \]

The solution is \( 51 - 68 i \), which is option E.\begin{enumerate}[label=\Alph*.]
\item \( a \in [45, 53] \text{ and } b \in [68, 69] \)

 $51 + 68 i$, which corresponds to adding a minus sign in both terms.
\item \( a \in [-76, -72] \text{ and } b \in [-41, -32] \)

 $-75 - 40 i$, which corresponds to adding a minus sign in the second term.
\item \( a \in [-76, -72] \text{ and } b \in [39, 42] \)

 $-75 + 40 i$, which corresponds to adding a minus sign in the first term.
\item \( a \in [-15, -7] \text{ and } b \in [-64, -62] \)

 $-12 - 63 i$, which corresponds to just multiplying the real terms to get the real part of the solution and the coefficients in the complex terms to get the complex part.
\item \( a \in [45, 53] \text{ and } b \in [-68, -64] \)

* $51 - 68 i$, which is the correct option.
\end{enumerate}

\textbf{General Comment:} You can treat $i$ as a variable and distribute. Just remember that $i^2=-1$, so you can continue to reduce after you distribute.
}
\litem{
Choose the \textbf{smallest} set of Complex numbers that the number below belongs to.
\[ \frac{\sqrt{65}}{7}+6i^2 \]

The solution is \( \text{Irrational} \), which is option B.\begin{enumerate}[label=\Alph*.]
\item \( \text{Nonreal Complex} \)

This is a Complex number $(a+bi)$ that is not Real (has $i$ as part of the number).
\item \( \text{Irrational} \)

* This is the correct option!
\item \( \text{Rational} \)

These are numbers that can be written as fraction of Integers (e.g., -2/3 + 5)
\item \( \text{Not a Complex Number} \)

This is not a number. The only non-Complex number we know is dividing by 0 as this is not a number!
\item \( \text{Pure Imaginary} \)

This is a Complex number $(a+bi)$ that \textbf{only} has an imaginary part like $2i$.
\end{enumerate}

\textbf{General Comment:} Be sure to simplify $i^2 = -1$. This may remove the imaginary portion for your number. If you are having trouble, you may want to look at the \textit{Subgroups of the Real Numbers} section.
}
\litem{
Choose the \textbf{smallest} set of Real numbers that the number below belongs to.
\[ \sqrt{\frac{116964}{361}} \]

The solution is \( \text{Whole} \), which is option C.\begin{enumerate}[label=\Alph*.]
\item \( \text{Rational} \)

These are numbers that can be written as fraction of Integers (e.g., -2/3)
\item \( \text{Integer} \)

These are the negative and positive counting numbers (..., -3, -2, -1, 0, 1, 2, 3, ...)
\item \( \text{Whole} \)

* This is the correct option!
\item \( \text{Not a Real number} \)

These are Nonreal Complex numbers \textbf{OR} things that are not numbers (e.g., dividing by 0).
\item \( \text{Irrational} \)

These cannot be written as a fraction of Integers.
\end{enumerate}

\textbf{General Comment:} First, you \textbf{NEED} to simplify the expression. This question simplifies to $342$. 
 
 Be sure you look at the simplified fraction and not just the decimal expansion. Numbers such as 13, 17, and 19 provide \textbf{long but repeating/terminating decimal expansions!} 
 
 The only ways to *not* be a Real number are: dividing by 0 or taking the square root of a negative number. 
 
 Irrational numbers are more than just square root of 3: adding or subtracting values from square root of 3 is also irrational.
}
\litem{
Simplify the expression below and choose the interval the simplification is contained within.
\[ 5 - 14^2 + 9 \div 8 * 10 \div 3 \]

The solution is \( -187.250 \), which is option A.\begin{enumerate}[label=\Alph*.]
\item \( [-188.2, -185.5] \)

* -187.250, this is the correct option
\item \( [200.9, 202.7] \)

 201.037, which corresponds to two Order of Operations errors.
\item \( [204.2, 206.8] \)

 204.750, which corresponds to an Order of Operations error: multiplying by negative before squaring. For example: $(-3)^2 \neq -3^2$
\item \( [-192.3, -187.6] \)

 -190.963, which corresponds to an Order of Operations error: not reading left-to-right for multiplication/division.
\item \( \text{None of the above} \)

 You may have gotten this by making an unanticipated error. If you got a value that is not any of the others, please let the coordinator know so they can help you figure out what happened.
\end{enumerate}

\textbf{General Comment:} While you may remember (or were taught) PEMDAS is done in order, it is actually done as P/E/MD/AS. When we are at MD or AS, we read left to right.
}
\litem{
Simplify the expression below into the form $a+bi$. Then, choose the intervals that $a$ and $b$ belong to.
\[ \frac{-63 + 11 i}{-2 - 4 i} \]

The solution is \( 4.10  - 13.70 i \), which is option D.\begin{enumerate}[label=\Alph*.]
\item \( a \in [3.5, 5] \text{ and } b \in [-274.5, -273.5] \)

 $4.10  - 274.00 i$, which corresponds to forgetting to multiply the conjugate by the numerator.
\item \( a \in [8, 10.5] \text{ and } b \in [11, 12.5] \)

 $8.50  + 11.50 i$, which corresponds to forgetting to multiply the conjugate by the numerator and not computing the conjugate correctly.
\item \( a \in [81, 83] \text{ and } b \in [-14, -13] \)

 $82.00  - 13.70 i$, which corresponds to forgetting to multiply the conjugate by the numerator and using a plus instead of a minus in the denominator.
\item \( a \in [3.5, 5] \text{ and } b \in [-14, -13] \)

* $4.10  - 13.70 i$, which is the correct option.
\item \( a \in [30.5, 32.5] \text{ and } b \in [-4.5, -2.5] \)

 $31.50  - 2.75 i$, which corresponds to just dividing the first term by the first term and the second by the second.
\end{enumerate}

\textbf{General Comment:} Multiply the numerator and denominator by the *conjugate* of the denominator, then simplify. For example, if we have $2+3i$, the conjugate is $2-3i$.
}
\end{enumerate}

\end{document}