\documentclass{extbook}[14pt]
\usepackage{multicol, enumerate, enumitem, hyperref, color, soul, setspace, parskip, fancyhdr, amssymb, amsthm, amsmath, bbm, latexsym, units, mathtools}
\everymath{\displaystyle}
\usepackage[headsep=0.5cm,headheight=0cm, left=1 in,right= 1 in,top= 1 in,bottom= 1 in]{geometry}
\pagestyle{fancy}
\lhead{}
\chead{Answer Key for Module\,1\,-\,Real\,and\,Complex\,Numbers Version C}
\rhead{}
\lfoot{Summer\,C\,2020}
\cfoot{}
\rfoot{}
\begin{document}
\textbf{This key should allow you to understand why you choose the option you did (beyond just getting a question right or wrong). \href{https://xronos.clas.ufl.edu/mac1105spring2020/courseDescriptionAndMisc/Exams/LearningFromResults}{More instructions on how to use this key can be found here}.}

\textbf{If you have a suggestion to make the keys better, \href{https://forms.gle/CZkbZmPbC9XALEE88}{please fill out the short survey here}.}

\textit{Note: This key is auto-generated and may contain issues and/or errors. The keys are reviewed after each exam to ensure grading is done accurately. If there are issues (like duplicate options), they are noted in the offline gradebook. The keys are a work-in-progress to give students as many resources to improve as possible.}

\rule{\textwidth}{0.4pt}

1. Choose the \textbf{smallest} set of Real numbers that the number below belongs to.
\[ \sqrt{\frac{441}{256}} \] 
The solution is $ \text{Rational} $ 

\begin{enumerate}[label=\Alph*.] 
\item $ \text{Irrational} $ 

 These cannot be written as a fraction of Integers. 
\item $ \text{Not a Real number} $ 

 These are Nonreal Complex numbers \textbf{OR} things that are not numbers (e.g., dividing by 0). 
\item $ \text{Integer} $ 

 These are the negative and positive counting numbers (..., -3, -2, -1, 0, 1, 2, 3, ...) 
\item $ \text{Rational} $ 

 * This is the correct option! 
\item $ \text{Whole} $ 

 These are the counting numbers with 0 (0, 1, 2, 3, ...) 
\end{enumerate} 
 
\textbf{General Comment:} First, you \textbf{NEED} to simplify the expression. This question simplifies to $\frac{21}{16}$. 
 
 Be sure you look at the simplified fraction and not just the decimal expansion. Numbers such as 13, 17, and 19 provide \textbf{long but repeating/terminating decimal expansions!} 
 
 The only ways to *not* be a Real number are: dividing by 0 or taking the square root of a negative number. 
 
 Irrational numbers are more than just square root of 3: adding or subtracting values from square root of 3 is also irrational. 

-----------------------------------------------

2. Simplify the expression below into the form $a+bi$. Then, choose the intervals that $a$ and $b$ belong to.
\[ \frac{-54 + 33 i}{8 - i} \] 
The solution is $ -7.15  + 3.23 i $ 

\begin{enumerate}[label=\Alph*.] 
\item $ a \in [-465.35, -464.5] \text{ and } b \in [2.5, 4.2] $ 

  $-465.00  + 3.23 i$, which corresponds to forgetting to multiply the conjugate by the numerator and using a plus instead of a minus in the denominator. 
\item $ a \in [-6.21, -5.3] \text{ and } b \in [4.4, 8.7] $ 

  $-6.14  + 4.89 i$, which corresponds to forgetting to multiply the conjugate by the numerator and not computing the conjugate correctly. 
\item $ a \in [-7.36, -6.96] \text{ and } b \in [2.5, 4.2] $ 

 * $-7.15  + 3.23 i$, which is the correct option. 
\item $ a \in [-7.36, -6.96] \text{ and } b \in [209.6, 211.9] $ 

  $-7.15  + 210.00 i$, which corresponds to forgetting to multiply the conjugate by the numerator. 
\item $ a \in [-7.07, -6.29] \text{ and } b \in [-34.7, -31.8] $ 

  $-6.75  - 33.00 i$, which corresponds to just dividing the first term by the first term and the second by the second. 
\end{enumerate} 
 
\textbf{General Comment:} Multiply the numerator and denominator by the *conjugate* of the denominator, then simplify. For example, if we have $2+3i$, the conjugate is $2-3i$. 

-----------------------------------------------

3. Choose the \textbf{smallest} set of Complex numbers that the number below belongs to.
\[ \frac{0}{2 \pi}+\sqrt{8}i \] 
The solution is $ \text{Pure Imaginary} $ 

\begin{enumerate}[label=\Alph*.] 
\item $ \text{Nonreal Complex} $ 

 This is a Complex number $(a+bi)$ that is not Real (has $i$ as part of the number). 
\item $ \text{Not a Complex Number} $ 

 This is not a number. The only non-Complex number we know is dividing by 0 as this is not a number! 
\item $ \text{Rational} $ 

 These are numbers that can be written as fraction of Integers (e.g., -2/3 + 5) 
\item $ \text{Irrational} $ 

 These cannot be written as a fraction of Integers. Remember: $\pi$ is not an Integer! 
\item $ \text{Pure Imaginary} $ 

 * This is the correct option! 
\end{enumerate} 
 
\textbf{General Comment:} Be sure to simplify $i^2 = -1$. This may remove the imaginary portion for your number. If you are having trouble, you may want to look at the \textit{Subgroups of the Real Numbers} section. 

-----------------------------------------------

4. Simplify the expression below into the form $a+bi$. Then, choose the intervals that $a$ and $b$ belong to.
\[ (4 - 8 i)(2 + 6 i) \] 
The solution is $ 56 + 8 i $ 

\begin{enumerate}[label=\Alph*.] 
\item $ a \in [52, 62] \text{ and } b \in [-12, -4] $ 

  $56 - 8 i$, which corresponds to adding a minus sign in both terms. 
\item $ a \in [-42, -35] \text{ and } b \in [-42, -36] $ 

  $-40 - 40 i$, which corresponds to adding a minus sign in the second term. 
\item $ a \in [7, 10] \text{ and } b \in [-57, -47] $ 

  $8 - 48 i$, which corresponds to just multiplying the real terms to get the real part of the solution and the coefficients in the complex terms to get the complex part. 
\item $ a \in [52, 62] \text{ and } b \in [6, 11] $ 

 * $56 + 8 i$, which is the correct option. 
\item $ a \in [-42, -35] \text{ and } b \in [39, 42] $ 

  $-40 + 40 i$, which corresponds to adding a minus sign in the first term. 
\end{enumerate} 
 
\textbf{General Comment:} You can treat $i$ as a variable and distribute. Just remember that $i^2=-1$, so you can continue to reduce after you distribute. 

-----------------------------------------------

0. Simplify the expression below and choose the interval the simplification is contained within.
\[ 10 - 15 \div 12 * 3 - (5 * 13) \] 
The solution is $ -58.750 $ 

\begin{enumerate}[label=\Alph*.] 
\item $ [16.2, 17.5] $ 

  16.250, which corresponds to not distributing a negative correctly. 
\item $ [-57.1, -52.9] $ 

  -55.417, which corresponds to an Order of Operations error: not reading left-to-right for multiplication/division. 
\item $ [-60.3, -57.3] $ 

 * -58.750, which is the correct option. 
\item $ [74.4, 74.8] $ 

  74.583, which corresponds to not distributing addition and subtraction correctly. 
\item $ \text{None of the above} $ 

  You may have gotten this by making an unanticipated error. If you got a value that is not any of the others, please let the coordinator know so they can help you figure out what happened. 
\end{enumerate} 
 
\textbf{General Comment:} While you may remember (or were taught) PEMDAS is done in order, it is actually done as P/E/MD/AS. When we are at MD or AS, we read left to right. 

-----------------------------------------------


\end{document}

