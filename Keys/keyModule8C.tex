\documentclass{extbook}[14pt]
\usepackage{multicol, enumerate, enumitem, hyperref, color, soul, setspace, parskip, fancyhdr, amssymb, amsthm, amsmath, latexsym, units, mathtools}
\everymath{\displaystyle}
\usepackage[headsep=0.5cm,headheight=0cm, left=1 in,right= 1 in,top= 1 in,bottom= 1 in]{geometry}
\usepackage{dashrule}  % Package to use the command below to create lines between items
\newcommand{\litem}[1]{\item #1

\rule{\textwidth}{0.4pt}}
\pagestyle{fancy}
\lhead{}
\chead{Answer Key for Progress Quiz 6 Version C}
\rhead{}
\lfoot{9689-6866}
\cfoot{}
\rfoot{Spring 2021}
\begin{document}
\textbf{This key should allow you to understand why you choose the option you did (beyond just getting a question right or wrong). \href{https://xronos.clas.ufl.edu/mac1105spring2020/courseDescriptionAndMisc/Exams/LearningFromResults}{More instructions on how to use this key can be found here}.}

\textbf{If you have a suggestion to make the keys better, \href{https://forms.gle/CZkbZmPbC9XALEE88}{please fill out the short survey here}.}

\textit{Note: This key is auto-generated and may contain issues and/or errors. The keys are reviewed after each exam to ensure grading is done accurately. If there are issues (like duplicate options), they are noted in the offline gradebook. The keys are a work-in-progress to give students as many resources to improve as possible.}

\rule{\textwidth}{0.4pt}

\begin{enumerate}\litem{
Solve the equation for $x$ and choose the interval that contains the solution (if it exists).
\[ \log_{5}{(3x+5)}+4 = 3 \]The solution is \( x = -1.600 \), which is option A.\begin{enumerate}[label=\Alph*.]
\item \( x \in [-1.64, -1.31] \)

* $x = -1.600$, which is the correct option.
\item \( x \in [-2.71, -1.68] \)

$x = -2.000$, which corresponds to reversing the base and exponent when converting.
\item \( x \in [39.52, 40.03] \)

$x = 40.000$, which corresponds to ignoring the vertical shift when converting to exponential form.
\item \( x \in [0.47, 1.36] \)

$x = 1.333$, which corresponds to reversing the base and exponent when converting and reversing the value with $x$.
\item \( \text{There is no Real solution to the equation.} \)

Corresponds to believing a negative coefficient within the log equation means there is no Real solution.
\end{enumerate}

\textbf{General Comment:} \textbf{General Comments:} First, get the equation in the form $\log_b{(cx+d)} = a$. Then, convert to $b^a = cx+d$ and solve.
}
\litem{
Which of the following intervals describes the Domain of the function below?
\[ f(x) = -\log_2{(x-4)}+1 \]The solution is \( (4, \infty) \), which is option B.\begin{enumerate}[label=\Alph*.]
\item \( (-\infty, a), a \in [-4.36, -2.94] \)

$(-\infty, -4)$, which corresponds to flipping the Domain. Remember: the general for is $a*\log(x-h)+k$, \textbf{where $a$ does not affect the domain}.
\item \( (a, \infty), a \in [2.71, 4.5] \)

* $(4, \infty)$, which is the correct option.
\item \( (-\infty, a], a \in [-1.4, 0.74] \)

$(-\infty, -1]$, which corresponds to using the negative vertical shift AND including the endpoint AND flipping the domain.
\item \( [a, \infty), a \in [0.3, 1.95] \)

$[1, \infty)$, which corresponds to using the vertical shift when shifting the Domain AND including the endpoint.
\item \( (-\infty, \infty) \)

This corresponds to thinking of the range of the log function (or the domain of the exponential function).
\end{enumerate}

\textbf{General Comment:} \textbf{General Comments}: The domain of a basic logarithmic function is $(0, \infty)$ and the Range is $(-\infty, \infty)$. We can use shifts when finding the Domain, but the Range will always be all Real numbers.
}
\litem{
 Solve the equation for $x$ and choose the interval that contains $x$ (if it exists).
\[  10 = \sqrt[3]{\frac{10}{e^{7x}}} \]The solution is \( x = -0.658 \), which is option B.\begin{enumerate}[label=\Alph*.]
\item \( x \in [-4.81, -4.53] \)

$x = -4.615$, which corresponds to thinking you don't need to take the natural log of both sides before reducing, as if the equation already had a natural log on the right side.
\item \( x \in [-0.67, -0.55] \)

* $x = -0.658$, which is the correct option.
\item \( x \in [-0.34, -0.15] \)

$x = -0.329$, which corresponds to treating any root as a square root.
\item \( \text{There is no Real solution to the equation.} \)

This corresponds to believing you cannot solve the equation.
\item \( \text{None of the above.} \)

This corresponds to making an unexpected error.
\end{enumerate}

\textbf{General Comment:} \textbf{General Comments}: After using the properties of logarithmic functions to break up the right-hand side, use $\ln(e) = 1$ to reduce the question to a linear function to solve. You can put $\ln(10)$ into a calculator if you are having trouble.
}
\litem{
 Solve the equation for $x$ and choose the interval that contains $x$ (if it exists).
\[  6 = \ln{\sqrt[5]{\frac{10}{e^{3x}}}} \]The solution is \( x = -9.232, \text{ which does not fit in any of the interval options.} \), which is option E.\begin{enumerate}[label=\Alph*.]
\item \( x \in [8, 11.2] \)

$x = 9.232$, which is the negative of the correct solution.
\item \( x \in [-5.6, -3.6] \)

$x = -3.754$, which corresponds to thinking you need to take the natural log of the left side before reducing.
\item \( x \in [-3.6, -3] \)

$x = -3.232$, which corresponds to treating any root as a square root.
\item \( \text{There is no Real solution to the equation.} \)

This corresponds to believing you cannot solve the equation.
\item \( \text{None of the above.} \)

*$x = -9.232$ is the correct solution and does not fit in any of the other intervals.
\end{enumerate}

\textbf{General Comment:} \textbf{General Comments}: After using the properties of logarithmic functions to break up the right-hand side, use $\ln(e) = 1$ to reduce the question to a linear function to solve. You can put $\ln(10)$ into a calculator if you are having trouble.
}
\litem{
Which of the following intervals describes the Range of the function below?
\[ f(x) = \log_2{(x-2)}-4 \]The solution is \( (\infty, \infty) \), which is option E.\begin{enumerate}[label=\Alph*.]
\item \( [a, \infty), a \in [1.2, 2.18] \)

$[-4, \infty)$, which corresponds to using the flipped Domain AND including the endpoint.
\item \( (-\infty, a), a \in [2.95, 4.4] \)

$(-\infty, 4)$, which corresponds to using the using the negative of vertical shift on $(0, \infty)$.
\item \( (-\infty, a), a \in [-5.21, -3.64] \)

$(-\infty, -4)$, which corresponds to using the vertical shift while the Range is $(-\infty, \infty)$.
\item \( [a, \infty), a \in [-3.17, -0.87] \)

$[-2, \infty)$, which corresponds to using the negative of the horizontal shift AND including the endpoint.
\item \( (-\infty, \infty) \)

*This is the correct option.
\end{enumerate}

\textbf{General Comment:} \textbf{General Comments}: The domain of a basic logarithmic function is $(0, \infty)$ and the Range is $(-\infty, \infty)$. We can use shifts when finding the Domain, but the Range will always be all Real numbers.
}
\litem{
Which of the following intervals describes the Domain of the function below?
\[ f(x) = -e^{x-8}-7 \]The solution is \( (-\infty, \infty) \), which is option E.\begin{enumerate}[label=\Alph*.]
\item \( [a, \infty), a \in [7, 12] \)

$[7, \infty)$, which corresponds to using the negative vertical shift AND flipping the Range interval AND including the endpoint.
\item \( (-\infty, a), a \in [-15, -6] \)

$(-\infty, -7)$, which corresponds to using the correct vertical shift *if we wanted the Range*.
\item \( (-\infty, a], a \in [-15, -6] \)

$(-\infty, -7]$, which corresponds to using the correct vertical shift *if we wanted the Range* AND including the endpoint.
\item \( (a, \infty), a \in [7, 12] \)

$(7, \infty)$, which corresponds to using the negative vertical shift AND flipping the Range interval.
\item \( (-\infty, \infty) \)

* This is the correct option.
\end{enumerate}

\textbf{General Comment:} \textbf{General Comments}: Domain of a basic exponential function is $(-\infty, \infty)$ while the Range is $(0, \infty)$. We can shift these intervals [and even flip when $a<0$!] to find the new Domain/Range.
}
\litem{
Solve the equation for $x$ and choose the interval that contains the solution (if it exists).
\[ 3^{4x-4} = 49^{3x+3} \]The solution is \( x = -2.207 \), which is option D.\begin{enumerate}[label=\Alph*.]
\item \( x \in [4.8, 7.8] \)

$x = 7.000$, which corresponds to solving the numerators as equal while ignoring the bases are different.
\item \( x \in [-1.3, 0.1] \)

$x = -0.961$, which corresponds to distributing the $\ln(base)$ to the first term of the exponent only.
\item \( x \in [14.1, 17.1] \)

$x = 16.070$, which corresponds to distributing the $\ln(base)$ to the second term of the exponent only.
\item \( x \in [-3.7, -1.4] \)

* $x = -2.207$, which is the correct option.
\item \( \text{There is no Real solution to the equation.} \)

This corresponds to believing there is no solution since the bases are not powers of each other.
\end{enumerate}

\textbf{General Comment:} \textbf{General Comments:} This question was written so that the bases could not be written the same. You will need to take the log of both sides.
}
\litem{
Solve the equation for $x$ and choose the interval that contains the solution (if it exists).
\[ 4^{5x-2} = \left(\frac{1}{27}\right)^{-2x-5} \]The solution is \( x = 56.657 \), which is option D.\begin{enumerate}[label=\Alph*.]
\item \( x \in [-1.43, 1.57] \)

$x = -0.429$, which corresponds to solving the numerators as equal while ignoring the bases are different.
\item \( x \in [-9.83, -6.83] \)

$x = -8.829$, which corresponds to distributing the $\ln(base)$ to the first term of the exponent only.
\item \( x \in [-0.25, 4.75] \)

$x = 2.750$, which corresponds to distributing the $\ln(base)$ to the second term of the exponent only.
\item \( x \in [51.66, 58.66] \)

* $x = 56.657$, which is the correct option.
\item \( \text{There is no Real solution to the equation.} \)

This corresponds to believing there is no solution since the bases are not powers of each other.
\end{enumerate}

\textbf{General Comment:} \textbf{General Comments:} This question was written so that the bases could not be written the same. You will need to take the log of both sides.
}
\litem{
Solve the equation for $x$ and choose the interval that contains the solution (if it exists).
\[ \log_{5}{(2x+7)}+5 = 3 \]The solution is \( x = -3.480 \), which is option C.\begin{enumerate}[label=\Alph*.]
\item \( x \in [-18.5, -8.5] \)

$x = -12.500$, which corresponds to reversing the base and exponent when converting and reversing the value with $x$.
\item \( x \in [58, 63] \)

$x = 59.000$, which corresponds to ignoring the vertical shift when converting to exponential form.
\item \( x \in [-5.48, -0.48] \)

* $x = -3.480$, which is the correct option.
\item \( x \in [-21.5, -18.5] \)

$x = -19.500$, which corresponds to reversing the base and exponent when converting.
\item \( \text{There is no Real solution to the equation.} \)

Corresponds to believing a negative coefficient within the log equation means there is no Real solution.
\end{enumerate}

\textbf{General Comment:} \textbf{General Comments:} First, get the equation in the form $\log_b{(cx+d)} = a$. Then, convert to $b^a = cx+d$ and solve.
}
\litem{
Which of the following intervals describes the Domain of the function below?
\[ f(x) = e^{x+5}+5 \]The solution is \( (-\infty, \infty) \), which is option E.\begin{enumerate}[label=\Alph*.]
\item \( (a, \infty), a \in [-11, -4] \)

$(-5, \infty)$, which corresponds to using the negative vertical shift AND flipping the Range interval.
\item \( [a, \infty), a \in [-11, -4] \)

$[-5, \infty)$, which corresponds to using the negative vertical shift AND flipping the Range interval AND including the endpoint.
\item \( (-\infty, a), a \in [-1, 6] \)

$(-\infty, 5)$, which corresponds to using the correct vertical shift *if we wanted the Range*.
\item \( (-\infty, a], a \in [-1, 6] \)

$(-\infty, 5]$, which corresponds to using the correct vertical shift *if we wanted the Range* AND including the endpoint.
\item \( (-\infty, \infty) \)

* This is the correct option.
\end{enumerate}

\textbf{General Comment:} \textbf{General Comments}: Domain of a basic exponential function is $(-\infty, \infty)$ while the Range is $(0, \infty)$. We can shift these intervals [and even flip when $a<0$!] to find the new Domain/Range.
}
\end{enumerate}

\end{document}