\documentclass{extbook}[14pt]
\usepackage{multicol, enumerate, enumitem, hyperref, color, soul, setspace, parskip, fancyhdr, amssymb, amsthm, amsmath, bbm, latexsym, units, mathtools}
\everymath{\displaystyle}
\usepackage[headsep=0.5cm,headheight=0cm, left=1 in,right= 1 in,top= 1 in,bottom= 1 in]{geometry}
\pagestyle{fancy}
\lhead{}
\chead{Answer Key for Module\,8\,-\,Logarithmic\,and\,Exponential\,Functions Version C}
\rhead{}
\lfoot{Summer\,C\,2020}
\cfoot{}
\rfoot{}
\begin{document}
\textbf{This key should allow you to understand why you choose the option you did (beyond just getting a question right or wrong). \href{https://xronos.clas.ufl.edu/mac1105spring2020/courseDescriptionAndMisc/Exams/LearningFromResults}{More instructions on how to use this key can be found here}.}

\textbf{If you have a suggestion to make the keys better, \href{https://forms.gle/CZkbZmPbC9XALEE88}{please fill out the short survey here}.}

\textit{Note: This key is auto-generated and may contain issues and/or errors. The keys are reviewed after each exam to ensure grading is done accurately. If there are issues (like duplicate options), they are noted in the offline gradebook. The keys are a work-in-progress to give students as many resources to improve as possible.}

\rule{\textwidth}{0.4pt}

36. Which of the following intervals describes the Range of the function below?
\[ f(x) = -\log_2{(x-4)}+1 \] 
The solution is $ (\infty, \infty) $ 

\begin{enumerate}[label=\Alph*.] 
\item $ [a, \infty), a \in [2.23, 4.48] $ 

 $[1, \infty)$, which corresponds to using the flipped Domain AND including the endpoint. 
\item $ (-\infty, a), a \in [-0.47, 1.92] $ 

 $(-\infty, 1)$, which corresponds to using the vertical shift while the Range is $(-\infty, \infty)$. 
\item $ (-\infty, a), a \in [-2.57, -0.76] $ 

 $(-\infty, -1)$, which corresponds to using the using the negative of vertical shift on $(0, \infty)$. 
\item $ [a, \infty), a \in [-4.12, -2.16] $ 

 $[-4, \infty)$, which corresponds to using the negative of the horizontal shift AND including the endpoint. 
\item $ (-\infty, \infty) $ 

 *This is the correct option. 
\end{enumerate} 
 
\textbf{General Comments}: The domain of a basic logarithmic function is $(0, \infty)$ and the Range is $(-\infty, \infty)$. We can use shifts when finding the Domain, but the Range will always be all Real numbers.

-----------------------------------------------

37.  Solve the equation for $x$ and choose the interval that contains $x$ (if it exists).
\[  22 = \ln{\sqrt[6]{\frac{14}{e^{5x}}}} \] 
The solution is $ x = -25.872 $ 

\begin{enumerate}[label=\Alph*.] 
\item $ x \in [-29, -23] $ 

 * $x = -25.872$, which is the correct option. 
\item $ x \in [-5, -2] $ 

 $x = -4.237$, which corresponds to thinking you need to take the natural log of on the left before reducing. 
\item $ x \in [-9, -6] $ 

 $x = -8.272$, which corresponds to treating any root as a square root. 
\item $ \text{There is no Real solution to the equation.} $ 

 This corresponds to believing you cannot solve the equation. 
\item $ \text{None of the above.} $ 

 This corresponds to making an unexpected error. 
\end{enumerate} 
 
\textbf{General Comments}: After using the properties of logarithmic functions to break up the right-hand side, use $\ln(e) = 1$ to reduce the question to a linear function to solve. You can put $\ln(14)$ into a calculator if you are having trouble.

-----------------------------------------------

38. Solve the equation for $x$ and choose the interval that contains the solution (if it exists).
\[ \log_{3}{(-4x+8)}+5 = 2 \] 
The solution is $ x = 1.991 $ 

\begin{enumerate}[label=\Alph*.] 
\item $ x \in [-2.8, 0.3] $ 

 $x = -0.250$, which corresponds to ignoring the vertical shift when converting to exponential form. 
\item $ x \in [1, 3.8] $ 

 * $x = 1.991$, which is the correct option. 
\item $ x \in [6.7, 9.2] $ 

 $x = 8.750$, which corresponds to reversing the base and exponent when converting. 
\item $ x \in [4.2, 6.4] $ 

 $x = 4.750$, which corresponds to reversing the base and exponent when converting and reversing the value with $x$. 
\item $ \text{There is no Real solution to the equation.} $ 

 Corresponds to believing a negative coefficient within the log equation means there is no Real solution. 
\end{enumerate} 
 
\textbf{General Comments:} First, get the equation in the form $\log_b{(cx+d)} = a$. Then, convert to $b^a = cx+d$ and solve.

-----------------------------------------------

39. Which of the following intervals describes the Domain of the function below?
\[ f(x) = e^{x+7}+6 \] 
The solution is $ (-\infty, \infty) $ 

\begin{enumerate}[label=\Alph*.] 
\item $ (-\infty, a], a \in [4, 7] $ 

 $(-\infty, 6]$, which corresponds to using the correct vertical shift *if we wanted the Range* AND including the endpoint. 
\item $ [a, \infty), a \in [-11, -4] $ 

 $[-6, \infty)$, which corresponds to using the negative vertical shift AND flipping the Range interval AND including the endpoint. 
\item $ (a, \infty), a \in [-11, -4] $ 

 $(-6, \infty)$, which corresponds to using the negative vertical shift AND flipping the Range interval. 
\item $ (-\infty, a), a \in [4, 7] $ 

 $(-\infty, 6)$, which corresponds to using the correct vertical shift *if we wanted the Range*. 
\item $ (-\infty, \infty) $ 

 * This is the correct option. 
\end{enumerate} 
 
\textbf{General Comments}: Domain of a basic exponential function is $(-\infty, \infty)$ while the Range is $(0, \infty)$. We can shift these intervals [and even flip when $a<0$!] to find the new Domain/Range.

-----------------------------------------------

40. Solve the equation for $x$ and choose the interval that contains the solution (if it exists).
\[ 5^{3x+4} = 16^{2x-3} \] 
The solution is $ x = 20.583 $ 

\begin{enumerate}[label=\Alph*.] 
\item $ x \in [9, 12] $ 

 $x = 9.765$, which corresponds to distributing the $\ln(base)$ to the first term of the exponent only. 
\item $ x \in [20, 25] $ 

 * $x = 20.583$, which is the correct option. 
\item $ x \in [-9, -6] $ 

 $x = -7.000$, which corresponds to solving the numerators as equal while ignoring the bases are different. 
\item $ x \in [-17, -12] $ 

 $x = -14.756$, which corresponds to distributing the $\ln(base)$ to the second term of the exponent only. 
\item $ \text{There is no Real solution to the equation.} $ 

 This corresponds to believing there is no solution since the bases are not powers of each other. 
\end{enumerate} 
 
\textbf{General Comments:} This question was written so that the bases could not be written the same. You will need to take the log of both sides.

-----------------------------------------------


\end{document}

