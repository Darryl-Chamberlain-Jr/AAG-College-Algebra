\documentclass{extbook}[14pt]
\usepackage{multicol, enumerate, enumitem, hyperref, color, soul, setspace, parskip, fancyhdr, amssymb, amsthm, amsmath, bbm, latexsym, units, mathtools}
\everymath{\displaystyle}
\usepackage[headsep=0.5cm,headheight=0cm, left=1 in,right= 1 in,top= 1 in,bottom= 1 in]{geometry}
\usepackage{dashrule}  % Package to use the command below to create lines between items
\newcommand{\litem}[1]{\item #1

\rule{\textwidth}{0.4pt}}
\pagestyle{fancy}
\lhead{}
\chead{Answer Key for Progress Quiz 9 Version C}
\rhead{}
\lfoot{8590-6105}
\cfoot{}
\rfoot{Fall 2020}
\begin{document}
\textbf{This key should allow you to understand why you choose the option you did (beyond just getting a question right or wrong). \href{https://xronos.clas.ufl.edu/mac1105spring2020/courseDescriptionAndMisc/Exams/LearningFromResults}{More instructions on how to use this key can be found here}.}

\textbf{If you have a suggestion to make the keys better, \href{https://forms.gle/CZkbZmPbC9XALEE88}{please fill out the short survey here}.}

\textit{Note: This key is auto-generated and may contain issues and/or errors. The keys are reviewed after each exam to ensure grading is done accurately. If there are issues (like duplicate options), they are noted in the offline gradebook. The keys are a work-in-progress to give students as many resources to improve as possible.}

\rule{\textwidth}{0.4pt}

\begin{enumerate}\litem{
 Solve the equation for $x$ and choose the interval that contains $x$ (if it exists).
\[  24 = \ln{\sqrt[3]{\frac{7}{e^{3x}}}} \]

The solution is \( x = -23.351 \), which is option A.\begin{enumerate}[label=\Alph*.]
\item \( x \in [-25.35, -22.35] \)

* $x = -23.351$, which is the correct option.
\item \( x \in [-6.83, -1.83] \)

$x = -3.827$, which corresponds to thinking you need to take the natural log of on the left before reducing.
\item \( x \in [-17.35, -12.35] \)

$x = -15.351$, which corresponds to treating any root as a square root.
\item \( \text{There is no Real solution to the equation.} \)

This corresponds to believing you cannot solve the equation.
\item \( \text{None of the above.} \)

This corresponds to making an unexpected error.
\end{enumerate}

\textbf{General Comment:} \textbf{General Comments}: After using the properties of logarithmic functions to break up the right-hand side, use $\ln(e) = 1$ to reduce the question to a linear function to solve. You can put $\ln(7)$ into a calculator if you are having trouble.
}
\litem{
Solve the equation for $x$ and choose the interval that contains the solution (if it exists).
\[ 2^{5x-3} = \left(\frac{1}{27}\right)^{4x+4} \]

The solution is \( x = -0.667 \), which is option D.\begin{enumerate}[label=\Alph*.]
\item \( x \in [0.2, 2.5] \)

$x = 0.420$, which corresponds to distributing the $\ln(base)$ to the first term of the exponent only.
\item \( x \in [6.2, 7.2] \)

$x = 7.000$, which corresponds to solving the numerators as equal while ignoring the bases are different.
\item \( x \in [-12.8, -9.2] \)

$x = -11.104$, which corresponds to distributing the $\ln(base)$ to the second term of the exponent only.
\item \( x \in [-2.3, -0.1] \)

* $x = -0.667$, which is the correct option.
\item \( \text{There is no Real solution to the equation.} \)

This corresponds to believing there is no solution since the bases are not powers of each other.
\end{enumerate}

\textbf{General Comment:} \textbf{General Comments:} This question was written so that the bases could not be written the same. You will need to take the log of both sides.
}
\litem{
Solve the equation for $x$ and choose the interval that contains the solution (if it exists).
\[ 3^{3x+2} = 343^{4x+5} \]

The solution is \( x = -1.346 \), which is option A.\begin{enumerate}[label=\Alph*.]
\item \( x \in [-2.1, -0.3] \)

* $x = -1.346$, which is the correct option.
\item \( x \in [-27.1, -26.5] \)

$x = -26.991$, which corresponds to distributing the $\ln(base)$ to the second term of the exponent only.
\item \( x \in [-0.2, 0.3] \)

$x = -0.150$, which corresponds to distributing the $\ln(base)$ to the first term of the exponent only.
\item \( x \in [-3.2, -1.4] \)

$x = -3.000$, which corresponds to solving the numerators as equal while ignoring the bases are different.
\item \( \text{There is no Real solution to the equation.} \)

This corresponds to believing there is no solution since the bases are not powers of each other.
\end{enumerate}

\textbf{General Comment:} \textbf{General Comments:} This question was written so that the bases could not be written the same. You will need to take the log of both sides.
}
\litem{
Solve the equation for $x$ and choose the interval that contains the solution (if it exists).
\[ \log_{2}{(-3x+5)}+6 = 3 \]

The solution is \( x = 1.625 \), which is option C.\begin{enumerate}[label=\Alph*.]
\item \( x \in [-1.18, -0.7] \)

$x = -1.000$, which corresponds to ignoring the vertical shift when converting to exponential form.
\item \( x \in [-5.01, -4.39] \)

$x = -4.667$, which corresponds to reversing the base and exponent when converting and reversing the value with $x$.
\item \( x \in [1.31, 1.64] \)

* $x = 1.625$, which is the correct option.
\item \( x \in [-2.72, -1.24] \)

$x = -1.333$, which corresponds to reversing the base and exponent when converting.
\item \( \text{There is no Real solution to the equation.} \)

Corresponds to believing a negative coefficient within the log equation means there is no Real solution.
\end{enumerate}

\textbf{General Comment:} \textbf{General Comments:} First, get the equation in the form $\log_b{(cx+d)} = a$. Then, convert to $b^a = cx+d$ and solve.
}
\litem{
Which of the following intervals describes the Range of the function below?
\[ f(x) = -e^{x+7}-9 \]

The solution is \( (-\infty, -9) \), which is option D.\begin{enumerate}[label=\Alph*.]
\item \( [a, \infty), a \in [7, 17] \)

$[9, \infty)$, which corresponds to using the negative vertical shift AND flipping the Range interval AND including the endpoint.
\item \( (-\infty, a], a \in [-9, -2] \)

$(-\infty, -9]$, which corresponds to including the endpoint.
\item \( (a, \infty), a \in [7, 17] \)

$(9, \infty)$, which corresponds to using the negative vertical shift AND flipping the Range interval.
\item \( (-\infty, a), a \in [-9, -2] \)

* $(-\infty, -9)$, which is the correct option.
\item \( (-\infty, \infty) \)

This corresponds to confusing range of an exponential function with the domain of an exponential function.
\end{enumerate}

\textbf{General Comment:} \textbf{General Comments}: Domain of a basic exponential function is $(-\infty, \infty)$ while the Range is $(0, \infty)$. We can shift these intervals [and even flip when $a<0$!] to find the new Domain/Range.
}
\litem{
 Solve the equation for $x$ and choose the interval that contains $x$ (if it exists).
\[  24 = \ln{\sqrt[4]{\frac{11}{e^{7x}}}} \]

The solution is \( x = -13.372, \text{ which does not fit in any of the interval options.} \), which is option E.\begin{enumerate}[label=\Alph*.]
\item \( x \in [-8.51, -3.51] \)

$x = -6.515$, which corresponds to treating any root as a square root.
\item \( x \in [13.37, 14.37] \)

$x = 13.372$, which is the negative of the correct solution.
\item \( x \in [-2.16, -1.16] \)

$x = -2.159$, which corresponds to thinking you need to take the natural log of the left side before reducing.
\item \( \text{There is no Real solution to the equation.} \)

This corresponds to believing you cannot solve the equation.
\item \( \text{None of the above.} \)

*$x = -13.372$ is the correct solution and does not fit in any of the other intervals.
\end{enumerate}

\textbf{General Comment:} \textbf{General Comments}: After using the properties of logarithmic functions to break up the right-hand side, use $\ln(e) = 1$ to reduce the question to a linear function to solve. You can put $\ln(11)$ into a calculator if you are having trouble.
}
\litem{
Solve the equation for $x$ and choose the interval that contains the solution (if it exists).
\[ \log_{2}{(3x+7)}+6 = 3 \]

The solution is \( x = -2.292 \), which is option A.\begin{enumerate}[label=\Alph*.]
\item \( x \in [-2.62, -1.77] \)

* $x = -2.292$, which is the correct option.
\item \( x \in [0.56, 0.8] \)

$x = 0.667$, which corresponds to reversing the base and exponent when converting.
\item \( x \in [5.1, 5.71] \)

$x = 5.333$, which corresponds to reversing the base and exponent when converting and reversing the value with $x$.
\item \( x \in [-0.01, 0.64] \)

$x = 0.333$, which corresponds to ignoring the vertical shift when converting to exponential form.
\item \( \text{There is no Real solution to the equation.} \)

Corresponds to believing a negative coefficient within the log equation means there is no Real solution.
\end{enumerate}

\textbf{General Comment:} \textbf{General Comments:} First, get the equation in the form $\log_b{(cx+d)} = a$. Then, convert to $b^a = cx+d$ and solve.
}
\litem{
Which of the following intervals describes the Range of the function below?
\[ f(x) = -\log_2{(x+9)}-3 \]

The solution is \( (\infty, \infty) \), which is option E.\begin{enumerate}[label=\Alph*.]
\item \( (-\infty, a), a \in [3, 5] \)

$(-\infty, 3)$, which corresponds to using the using the negative of vertical shift on $(0, \infty)$.
\item \( (-\infty, a), a \in [-3, -1] \)

$(-\infty, -3)$, which corresponds to using the vertical shift while the Range is $(-\infty, \infty)$.
\item \( [a, \infty), a \in [-9, -7] \)

$[-3, \infty)$, which corresponds to using the flipped Domain AND including the endpoint.
\item \( [a, \infty), a \in [5, 13] \)

$[9, \infty)$, which corresponds to using the negative of the horizontal shift AND including the endpoint.
\item \( (-\infty, \infty) \)

*This is the correct option.
\end{enumerate}

\textbf{General Comment:} \textbf{General Comments}: The domain of a basic logarithmic function is $(0, \infty)$ and the Range is $(-\infty, \infty)$. We can use shifts when finding the Domain, but the Range will always be all Real numbers.
}
\litem{
Which of the following intervals describes the Domain of the function below?
\[ f(x) = e^{x+3}+4 \]

The solution is \( (-\infty, \infty) \), which is option E.\begin{enumerate}[label=\Alph*.]
\item \( (-\infty, a], a \in [4, 6] \)

$(-\infty, 4]$, which corresponds to using the correct vertical shift *if we wanted the Range* AND including the endpoint.
\item \( (a, \infty), a \in [-8, 1] \)

$(-4, \infty)$, which corresponds to using the negative vertical shift AND flipping the Range interval.
\item \( (-\infty, a), a \in [4, 6] \)

$(-\infty, 4)$, which corresponds to using the correct vertical shift *if we wanted the Range*.
\item \( [a, \infty), a \in [-8, 1] \)

$[-4, \infty)$, which corresponds to using the negative vertical shift AND flipping the Range interval AND including the endpoint.
\item \( (-\infty, \infty) \)

* This is the correct option.
\end{enumerate}

\textbf{General Comment:} \textbf{General Comments}: Domain of a basic exponential function is $(-\infty, \infty)$ while the Range is $(0, \infty)$. We can shift these intervals [and even flip when $a<0$!] to find the new Domain/Range.
}
\litem{
Which of the following intervals describes the Range of the function below?
\[ f(x) = -\log_2{(x+6)}-8 \]

The solution is \( (\infty, \infty) \), which is option E.\begin{enumerate}[label=\Alph*.]
\item \( (-\infty, a), a \in [-9.1, -7.1] \)

$(-\infty, -8)$, which corresponds to using the vertical shift while the Range is $(-\infty, \infty)$.
\item \( [a, \infty), a \in [-7.2, -5.8] \)

$[-8, \infty)$, which corresponds to using the flipped Domain AND including the endpoint.
\item \( [a, \infty), a \in [5.6, 7] \)

$[6, \infty)$, which corresponds to using the negative of the horizontal shift AND including the endpoint.
\item \( (-\infty, a), a \in [6.1, 9] \)

$(-\infty, 8)$, which corresponds to using the using the negative of vertical shift on $(0, \infty)$.
\item \( (-\infty, \infty) \)

*This is the correct option.
\end{enumerate}

\textbf{General Comment:} \textbf{General Comments}: The domain of a basic logarithmic function is $(0, \infty)$ and the Range is $(-\infty, \infty)$. We can use shifts when finding the Domain, but the Range will always be all Real numbers.
}
\end{enumerate}

\end{document}