\documentclass{extbook}[14pt]
\usepackage{multicol, enumerate, enumitem, hyperref, color, soul, setspace, parskip, fancyhdr, amssymb, amsthm, amsmath, bbm, latexsym, units, mathtools}
\everymath{\displaystyle}
\usepackage[headsep=0.5cm,headheight=0cm, left=1 in,right= 1 in,top= 1 in,bottom= 1 in]{geometry}
\usepackage{dashrule}  % Package to use the command below to create lines between items
\newcommand{\litem}[1]{\item #1

\rule{\textwidth}{0.4pt}}
\pagestyle{fancy}
\lhead{}
\chead{Answer Key for Progress Quiz 5 Version C}
\rhead{}
\lfoot{9912-2038}
\cfoot{}
\rfoot{Spring 2021}
\begin{document}
\textbf{This key should allow you to understand why you choose the option you did (beyond just getting a question right or wrong). \href{https://xronos.clas.ufl.edu/mac1105spring2020/courseDescriptionAndMisc/Exams/LearningFromResults}{More instructions on how to use this key can be found here}.}

\textbf{If you have a suggestion to make the keys better, \href{https://forms.gle/CZkbZmPbC9XALEE88}{please fill out the short survey here}.}

\textit{Note: This key is auto-generated and may contain issues and/or errors. The keys are reviewed after each exam to ensure grading is done accurately. If there are issues (like duplicate options), they are noted in the offline gradebook. The keys are a work-in-progress to give students as many resources to improve as possible.}

\rule{\textwidth}{0.4pt}

\begin{enumerate}\litem{
Which of the following intervals describes the Domain of the function below?
\[ f(x) = -\log_2{(x+6)}+3 \]The solution is \( (-6, \infty) \), which is option B.\begin{enumerate}[label=\Alph*.]
\item \( (-\infty, a), a \in [3.7, 7.2] \)

$(-\infty, 6)$, which corresponds to flipping the Domain. Remember: the general for is $a*\log(x-h)+k$, \textbf{where $a$ does not affect the domain}.
\item \( (a, \infty), a \in [-7.3, -5.7] \)

* $(-6, \infty)$, which is the correct option.
\item \( [a, \infty), a \in [2.1, 4.9] \)

$[3, \infty)$, which corresponds to using the vertical shift when shifting the Domain AND including the endpoint.
\item \( (-\infty, a], a \in [-3.6, -2.8] \)

$(-\infty, -3]$, which corresponds to using the negative vertical shift AND including the endpoint AND flipping the domain.
\item \( (-\infty, \infty) \)

This corresponds to thinking of the range of the log function (or the domain of the exponential function).
\end{enumerate}

\textbf{General Comment:} \textbf{General Comments}: The domain of a basic logarithmic function is $(0, \infty)$ and the Range is $(-\infty, \infty)$. We can use shifts when finding the Domain, but the Range will always be all Real numbers.
}
\litem{
Solve the equation for $x$ and choose the interval that contains the solution (if it exists).
\[ \log_{4}{(-4x+8)}+4 = 2 \]The solution is \( x = 1.984 \), which is option C.\begin{enumerate}[label=\Alph*.]
\item \( x \in [-4, 0] \)

$x = -2.000$, which corresponds to ignoring the vertical shift when converting to exponential form.
\item \( x \in [-10, -4] \)

$x = -6.000$, which corresponds to reversing the base and exponent when converting and reversing the value with $x$.
\item \( x \in [-0.02, 7.98] \)

* $x = 1.984$, which is the correct option.
\item \( x \in [-4, 0] \)

$x = -2.000$, which corresponds to reversing the base and exponent when converting.
\item \( \text{There is no Real solution to the equation.} \)

Corresponds to believing a negative coefficient within the log equation means there is no Real solution.
\end{enumerate}

\textbf{General Comment:} \textbf{General Comments:} First, get the equation in the form $\log_b{(cx+d)} = a$. Then, convert to $b^a = cx+d$ and solve.
}
\litem{
Solve the equation for $x$ and choose the interval that contains the solution (if it exists).
\[ \log_{4}{(-3x+6)}+5 = 2 \]The solution is \( x = 1.995 \), which is option A.\begin{enumerate}[label=\Alph*.]
\item \( x \in [1.99, 7.99] \)

* $x = 1.995$, which is the correct option.
\item \( x \in [-5.33, -2.33] \)

$x = -3.333$, which corresponds to ignoring the vertical shift when converting to exponential form.
\item \( x \in [-27, -22] \)

$x = -25.000$, which corresponds to reversing the base and exponent when converting.
\item \( x \in [-30, -26] \)

$x = -29.000$, which corresponds to reversing the base and exponent when converting and reversing the value with $x$.
\item \( \text{There is no Real solution to the equation.} \)

Corresponds to believing a negative coefficient within the log equation means there is no Real solution.
\end{enumerate}

\textbf{General Comment:} \textbf{General Comments:} First, get the equation in the form $\log_b{(cx+d)} = a$. Then, convert to $b^a = cx+d$ and solve.
}
\litem{
Which of the following intervals describes the Domain of the function below?
\[ f(x) = e^{x+5}+6 \]The solution is \( (-\infty, \infty) \), which is option E.\begin{enumerate}[label=\Alph*.]
\item \( (-\infty, a), a \in [4, 10] \)

$(-\infty, 6)$, which corresponds to using the correct vertical shift *if we wanted the Range*.
\item \( [a, \infty), a \in [-11, -5] \)

$[-6, \infty)$, which corresponds to using the negative vertical shift AND flipping the Range interval AND including the endpoint.
\item \( (-\infty, a], a \in [4, 10] \)

$(-\infty, 6]$, which corresponds to using the correct vertical shift *if we wanted the Range* AND including the endpoint.
\item \( (a, \infty), a \in [-11, -5] \)

$(-6, \infty)$, which corresponds to using the negative vertical shift AND flipping the Range interval.
\item \( (-\infty, \infty) \)

* This is the correct option.
\end{enumerate}

\textbf{General Comment:} \textbf{General Comments}: Domain of a basic exponential function is $(-\infty, \infty)$ while the Range is $(0, \infty)$. We can shift these intervals [and even flip when $a<0$!] to find the new Domain/Range.
}
\litem{
Which of the following intervals describes the Domain of the function below?
\[ f(x) = e^{x-5}-7 \]The solution is \( (-\infty, \infty) \), which is option E.\begin{enumerate}[label=\Alph*.]
\item \( (a, \infty), a \in [6, 10] \)

$(7, \infty)$, which corresponds to using the negative vertical shift AND flipping the Range interval.
\item \( [a, \infty), a \in [6, 10] \)

$[7, \infty)$, which corresponds to using the negative vertical shift AND flipping the Range interval AND including the endpoint.
\item \( (-\infty, a], a \in [-7, -3] \)

$(-\infty, -7]$, which corresponds to using the correct vertical shift *if we wanted the Range* AND including the endpoint.
\item \( (-\infty, a), a \in [-7, -3] \)

$(-\infty, -7)$, which corresponds to using the correct vertical shift *if we wanted the Range*.
\item \( (-\infty, \infty) \)

* This is the correct option.
\end{enumerate}

\textbf{General Comment:} \textbf{General Comments}: Domain of a basic exponential function is $(-\infty, \infty)$ while the Range is $(0, \infty)$. We can shift these intervals [and even flip when $a<0$!] to find the new Domain/Range.
}
\litem{
Solve the equation for $x$ and choose the interval that contains the solution (if it exists).
\[ 2^{5x+4} = 9^{2x-4} \]The solution is \( x = 12.449 \), which is option C.\begin{enumerate}[label=\Alph*.]
\item \( x \in [-3.67, 0.33] \)

$x = -2.667$, which corresponds to solving the numerators as equal while ignoring the bases are different.
\item \( x \in [-5.85, -2.85] \)

$x = -3.854$, which corresponds to distributing the $\ln(base)$ to the second term of the exponent only.
\item \( x \in [10.45, 15.45] \)

* $x = 12.449$, which is the correct option.
\item \( x \in [7.61, 9.61] \)

$x = 8.614$, which corresponds to distributing the $\ln(base)$ to the first term of the exponent only.
\item \( \text{There is no Real solution to the equation.} \)

This corresponds to believing there is no solution since the bases are not powers of each other.
\end{enumerate}

\textbf{General Comment:} \textbf{General Comments:} This question was written so that the bases could not be written the same. You will need to take the log of both sides.
}
\litem{
 Solve the equation for $x$ and choose the interval that contains $x$ (if it exists).
\[  14 = \sqrt[7]{\frac{22}{e^{4x}}} \]The solution is \( x = -3.846 \), which is option A.\begin{enumerate}[label=\Alph*.]
\item \( x \in [-6.85, -1.85] \)

* $x = -3.846$, which is the correct option.
\item \( x \in [-3.55, 3.45] \)

$x = -0.547$, which corresponds to treating any root as a square root.
\item \( x \in [-27.27, -23.27] \)

$x = -25.273$, which corresponds to thinking you don't need to take the natural log of both sides before reducing, as if the equation already had a natural log on the right side.
\item \( \text{There is no Real solution to the equation.} \)

This corresponds to believing you cannot solve the equation.
\item \( \text{None of the above.} \)

This corresponds to making an unexpected error.
\end{enumerate}

\textbf{General Comment:} \textbf{General Comments}: After using the properties of logarithmic functions to break up the right-hand side, use $\ln(e) = 1$ to reduce the question to a linear function to solve. You can put $\ln(22)$ into a calculator if you are having trouble.
}
\litem{
Which of the following intervals describes the Domain of the function below?
\[ f(x) = -\log_2{(x+7)}+7 \]The solution is \( (-7, \infty) \), which is option B.\begin{enumerate}[label=\Alph*.]
\item \( (-\infty, a), a \in [5, 8] \)

$(-\infty, 7)$, which corresponds to flipping the Domain. Remember: the general for is $a*\log(x-h)+k$, \textbf{where $a$ does not affect the domain}.
\item \( (a, \infty), a \in [-7, -5] \)

* $(-7, \infty)$, which is the correct option.
\item \( (-\infty, a], a \in [-7, -5] \)

$(-\infty, -7]$, which corresponds to using the negative vertical shift AND including the endpoint AND flipping the domain.
\item \( [a, \infty), a \in [5, 8] \)

$[7, \infty)$, which corresponds to using the vertical shift when shifting the Domain AND including the endpoint.
\item \( (-\infty, \infty) \)

This corresponds to thinking of the range of the log function (or the domain of the exponential function).
\end{enumerate}

\textbf{General Comment:} \textbf{General Comments}: The domain of a basic logarithmic function is $(0, \infty)$ and the Range is $(-\infty, \infty)$. We can use shifts when finding the Domain, but the Range will always be all Real numbers.
}
\litem{
 Solve the equation for $x$ and choose the interval that contains $x$ (if it exists).
\[  20 = \sqrt[4]{\frac{12}{e^{7x}}} \]The solution is \( x = -1.357 \), which is option C.\begin{enumerate}[label=\Alph*.]
\item \( x \in [-13.3, -11.7] \)

$x = -11.784$, which corresponds to thinking you don't need to take the natural log of both sides before reducing, as if the equation already had a natural log on the right side.
\item \( x \in [-0.7, 0.8] \)

$x = -0.501$, which corresponds to treating any root as a square root.
\item \( x \in [-2, -0.6] \)

* $x = -1.357$, which is the correct option.
\item \( \text{There is no Real solution to the equation.} \)

This corresponds to believing you cannot solve the equation.
\item \( \text{None of the above.} \)

This corresponds to making an unexpected error.
\end{enumerate}

\textbf{General Comment:} \textbf{General Comments}: After using the properties of logarithmic functions to break up the right-hand side, use $\ln(e) = 1$ to reduce the question to a linear function to solve. You can put $\ln(12)$ into a calculator if you are having trouble.
}
\litem{
Solve the equation for $x$ and choose the interval that contains the solution (if it exists).
\[ 4^{-2x-4} = \left(\frac{1}{125}\right)^{-4x+5} \]The solution is \( x = 0.842 \), which is option D.\begin{enumerate}[label=\Alph*.]
\item \( x \in [-1.4, 0.7] \)

$x = -0.408$, which corresponds to distributing the $\ln(base)$ to the first term of the exponent only.
\item \( x \in [-11, -8.8] \)

$x = -9.298$, which corresponds to distributing the $\ln(base)$ to the second term of the exponent only.
\item \( x \in [3.4, 6.4] \)

$x = 4.500$, which corresponds to solving the numerators as equal while ignoring the bases are different.
\item \( x \in [0, 2.3] \)

* $x = 0.842$, which is the correct option.
\item \( \text{There is no Real solution to the equation.} \)

This corresponds to believing there is no solution since the bases are not powers of each other.
\end{enumerate}

\textbf{General Comment:} \textbf{General Comments:} This question was written so that the bases could not be written the same. You will need to take the log of both sides.
}
\end{enumerate}

\end{document}