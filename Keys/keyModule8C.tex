\documentclass{extbook}[14pt]
\usepackage{multicol, enumerate, enumitem, hyperref, color, soul, setspace, parskip, fancyhdr, amssymb, amsthm, amsmath, bbm, latexsym, units, mathtools}
\everymath{\displaystyle}
\usepackage[headsep=0.5cm,headheight=0cm, left=1 in,right= 1 in,top= 1 in,bottom= 1 in]{geometry}
\usepackage{dashrule}  % Package to use the command below to create lines between items
\newcommand{\litem}[1]{\item #1

\rule{\textwidth}{0.4pt}}
\pagestyle{fancy}
\lhead{}
\chead{Answer Key for Progress Quiz 4 Version C}
\rhead{}
\lfoot{6286-1986}
\cfoot{}
\rfoot{Fall 2020}
\begin{document}
\textbf{This key should allow you to understand why you choose the option you did (beyond just getting a question right or wrong). \href{https://xronos.clas.ufl.edu/mac1105spring2020/courseDescriptionAndMisc/Exams/LearningFromResults}{More instructions on how to use this key can be found here}.}

\textbf{If you have a suggestion to make the keys better, \href{https://forms.gle/CZkbZmPbC9XALEE88}{please fill out the short survey here}.}

\textit{Note: This key is auto-generated and may contain issues and/or errors. The keys are reviewed after each exam to ensure grading is done accurately. If there are issues (like duplicate options), they are noted in the offline gradebook. The keys are a work-in-progress to give students as many resources to improve as possible.}

\rule{\textwidth}{0.4pt}

\begin{enumerate}\litem{
Solve the equation for $x$ and choose the interval that contains the solution (if it exists).
\[ 4^{-4x-5} = 49^{-2x+2} \]
The solution is \( x = 6.574 \), which is option A.\begin{enumerate}[label=\Alph*.]
\item \( x \in [4.57, 7.57] \)

* $x = 6.574$, which is the correct option.
\item \( x \in [-10.36, -4.36] \)

$x = -7.358$, which corresponds to distributing the $\ln(base)$ to the second term of the exponent only.
\item \( x \in [-4.5, -2.5] \)

$x = -3.500$, which corresponds to solving the numerators as equal while ignoring the bases are different.
\item \( x \in [2.13, 6.13] \)

$x = 3.127$, which corresponds to distributing the $\ln(base)$ to the first term of the exponent only.
\item \( \text{There is no Real solution to the equation.} \)

This corresponds to believing there is no solution since the bases are not powers of each other.
\end{enumerate}

\textbf{General Comment:} \textbf{General Comments:} This question was written so that the bases could not be written the same. You will need to take the log of both sides.
}
\litem{
Solve the equation for $x$ and choose the interval that contains the solution (if it exists).
\[ 2^{-2x-5} = \left(\frac{1}{125}\right)^{-3x-4} \]
The solution is \( x = -1.435 \), which is option A.\begin{enumerate}[label=\Alph*.]
\item \( x \in [-2.2, -0.4] \)

* $x = -1.435$, which is the correct option.
\item \( x \in [22.2, 23.4] \)

$x = 22.779$, which corresponds to distributing the $\ln(base)$ to the second term of the exponent only.
\item \( x \in [-0.1, 0.3] \)

$x = -0.063$, which corresponds to distributing the $\ln(base)$ to the first term of the exponent only.
\item \( x \in [0.7, 2.4] \)

$x = 1.000$, which corresponds to solving the numerators as equal while ignoring the bases are different.
\item \( \text{There is no Real solution to the equation.} \)

This corresponds to believing there is no solution since the bases are not powers of each other.
\end{enumerate}

\textbf{General Comment:} \textbf{General Comments:} This question was written so that the bases could not be written the same. You will need to take the log of both sides.
}
\litem{
Which of the following intervals describes the Domain of the function below?
\[ f(x) = e^{x-1}-9 \]
The solution is \( (-\infty, \infty) \), which is option E.\begin{enumerate}[label=\Alph*.]
\item \( (-\infty, a], a \in [-12, -6] \)

$(-\infty, -9]$, which corresponds to using the correct vertical shift *if we wanted the Range* AND including the endpoint.
\item \( (-\infty, a), a \in [-12, -6] \)

$(-\infty, -9)$, which corresponds to using the correct vertical shift *if we wanted the Range*.
\item \( [a, \infty), a \in [6, 13] \)

$[9, \infty)$, which corresponds to using the negative vertical shift AND flipping the Range interval AND including the endpoint.
\item \( (a, \infty), a \in [6, 13] \)

$(9, \infty)$, which corresponds to using the negative vertical shift AND flipping the Range interval.
\item \( (-\infty, \infty) \)

* This is the correct option.
\end{enumerate}

\textbf{General Comment:} \textbf{General Comments}: Domain of a basic exponential function is $(-\infty, \infty)$ while the Range is $(0, \infty)$. We can shift these intervals [and even flip when $a<0$!] to find the new Domain/Range.
}
\litem{
Which of the following intervals describes the Domain of the function below?
\[ f(x) = -e^{x-6}+9 \]
The solution is \( (-\infty, \infty) \), which is option E.\begin{enumerate}[label=\Alph*.]
\item \( (a, \infty), a \in [-13, -7] \)

$(-9, \infty)$, which corresponds to using the negative vertical shift AND flipping the Range interval.
\item \( [a, \infty), a \in [-13, -7] \)

$[-9, \infty)$, which corresponds to using the negative vertical shift AND flipping the Range interval AND including the endpoint.
\item \( (-\infty, a], a \in [9, 13] \)

$(-\infty, 9]$, which corresponds to using the correct vertical shift *if we wanted the Range* AND including the endpoint.
\item \( (-\infty, a), a \in [9, 13] \)

$(-\infty, 9)$, which corresponds to using the correct vertical shift *if we wanted the Range*.
\item \( (-\infty, \infty) \)

* This is the correct option.
\end{enumerate}

\textbf{General Comment:} \textbf{General Comments}: Domain of a basic exponential function is $(-\infty, \infty)$ while the Range is $(0, \infty)$. We can shift these intervals [and even flip when $a<0$!] to find the new Domain/Range.
}
\litem{
Which of the following intervals describes the Range of the function below?
\[ f(x) = -\log_2{(x+4)}-7 \]
The solution is \( (\infty, \infty) \), which is option E.\begin{enumerate}[label=\Alph*.]
\item \( [a, \infty), a \in [2.8, 6.3] \)

$[4, \infty)$, which corresponds to using the negative of the horizontal shift AND including the endpoint.
\item \( (-\infty, a), a \in [4.6, 7.9] \)

$(-\infty, 7)$, which corresponds to using the using the negative of vertical shift on $(0, \infty)$.
\item \( (-\infty, a), a \in [-7.3, -5.6] \)

$(-\infty, -7)$, which corresponds to using the vertical shift while the Range is $(-\infty, \infty)$.
\item \( [a, \infty), a \in [-6, -2.9] \)

$[-7, \infty)$, which corresponds to using the flipped Domain AND including the endpoint.
\item \( (-\infty, \infty) \)

*This is the correct option.
\end{enumerate}

\textbf{General Comment:} \textbf{General Comments}: The domain of a basic logarithmic function is $(0, \infty)$ and the Range is $(-\infty, \infty)$. We can use shifts when finding the Domain, but the Range will always be all Real numbers.
}
\litem{
 Solve the equation for $x$ and choose the interval that contains $x$ (if it exists).
\[  15 = \sqrt[3]{\frac{25}{e^{6x}}} \]
The solution is \( x = -0.818, \text{ which does not fit in any of the interval options.} \), which is option E.\begin{enumerate}[label=\Alph*.]
\item \( x \in [-8.89, -6.42] \)

$x = -8.036$, which corresponds to thinking you don't need to take the natural log of both sides before reducing, as if the right side already has a natural log.
\item \( x \in [0.64, 1.47] \)

$x = 0.818$, which is the negative of the correct solution.
\item \( x \in [-0.8, 0.38] \)

$x = -0.366$, which corresponds to treating any root as a square root.
\item \( \text{There is no Real solution to the equation.} \)

This corresponds to believing you cannot solve the equation.
\item \( \text{None of the above.} \)

* $x = -0.818$ is the correct solution and does not fit in any of the other intervals.
\end{enumerate}

\textbf{General Comment:} \textbf{General Comments}: After using the properties of logarithmic functions to break up the right-hand side, use $\ln(e) = 1$ to reduce the question to a linear function to solve. You can put $\ln(25)$ into a calculator if you are having trouble.
}
\litem{
Which of the following intervals describes the Range of the function below?
\[ f(x) = -\log_2{(x+7)}+3 \]
The solution is \( (\infty, \infty) \), which is option E.\begin{enumerate}[label=\Alph*.]
\item \( [a, \infty), a \in [5, 9] \)

$[7, \infty)$, which corresponds to using the negative of the horizontal shift AND including the endpoint.
\item \( (-\infty, a), a \in [-6, 2] \)

$(-\infty, -3)$, which corresponds to using the using the negative of vertical shift on $(0, \infty)$.
\item \( (-\infty, a), a \in [1, 6] \)

$(-\infty, 3)$, which corresponds to using the vertical shift while the Range is $(-\infty, \infty)$.
\item \( [a, \infty), a \in [-10, -5] \)

$[3, \infty)$, which corresponds to using the flipped Domain AND including the endpoint.
\item \( (-\infty, \infty) \)

*This is the correct option.
\end{enumerate}

\textbf{General Comment:} \textbf{General Comments}: The domain of a basic logarithmic function is $(0, \infty)$ and the Range is $(-\infty, \infty)$. We can use shifts when finding the Domain, but the Range will always be all Real numbers.
}
\litem{
Solve the equation for $x$ and choose the interval that contains the solution (if it exists).
\[ \log_{4}{(3x+8)}+5 = 3 \]
The solution is \( x = -2.646 \), which is option C.\begin{enumerate}[label=\Alph*.]
\item \( x \in [17.67, 21.67] \)

$x = 18.667$, which corresponds to ignoring the vertical shift when converting to exponential form.
\item \( x \in [7, 11] \)

$x = 8.000$, which corresponds to reversing the base and exponent when converting and reversing the value with $x$.
\item \( x \in [-2.65, -1.65] \)

* $x = -2.646$, which is the correct option.
\item \( x \in [-2.33, 6.67] \)

$x = 2.667$, which corresponds to reversing the base and exponent when converting.
\item \( \text{There is no Real solution to the equation.} \)

Corresponds to believing a negative coefficient within the log equation means there is no Real solution.
\end{enumerate}

\textbf{General Comment:} \textbf{General Comments:} First, get the equation in the form $\log_b{(cx+d)} = a$. Then, convert to $b^a = cx+d$ and solve.
}
\litem{
Solve the equation for $x$ and choose the interval that contains the solution (if it exists).
\[ \log_{5}{(-3x+5)}+4 = 2 \]
The solution is \( x = 1.653 \), which is option D.\begin{enumerate}[label=\Alph*.]
\item \( x \in [-10.67, -2.67] \)

$x = -6.667$, which corresponds to ignoring the vertical shift when converting to exponential form.
\item \( x \in [3, 11] \)

$x = 9.000$, which corresponds to reversing the base and exponent when converting and reversing the value with $x$.
\item \( x \in [10.33, 14.33] \)

$x = 12.333$, which corresponds to reversing the base and exponent when converting.
\item \( x \in [-1.35, 3.65] \)

* $x = 1.653$, which is the correct option.
\item \( \text{There is no Real solution to the equation.} \)

Corresponds to believing a negative coefficient within the log equation means there is no Real solution.
\end{enumerate}

\textbf{General Comment:} \textbf{General Comments:} First, get the equation in the form $\log_b{(cx+d)} = a$. Then, convert to $b^a = cx+d$ and solve.
}
\litem{
 Solve the equation for $x$ and choose the interval that contains $x$ (if it exists).
\[  6 = \ln{\sqrt[5]{\frac{13}{e^{8x}}}} \]
The solution is \( x = -3.429 \), which is option C.\begin{enumerate}[label=\Alph*.]
\item \( x \in [-1.35, -1.02] \)

$x = -1.179$, which corresponds to treating any root as a square root.
\item \( x \in [-1.54, -1.35] \)

$x = -1.440$, which corresponds to thinking you need to take the natural log of on the left before reducing.
\item \( x \in [-3.52, -3.29] \)

* $x = -3.429$, which is the correct option.
\item \( \text{There is no Real solution to the equation.} \)

This corresponds to believing you cannot solve the equation.
\item \( \text{None of the above.} \)

This corresponds to making an unexpected error.
\end{enumerate}

\textbf{General Comment:} \textbf{General Comments}: After using the properties of logarithmic functions to break up the right-hand side, use $\ln(e) = 1$ to reduce the question to a linear function to solve. You can put $\ln(13)$ into a calculator if you are having trouble.
}
\end{enumerate}

\end{document}