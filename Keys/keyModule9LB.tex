\documentclass{extbook}[14pt]
\usepackage{multicol, enumerate, enumitem, hyperref, color, soul, setspace, parskip, fancyhdr, amssymb, amsthm, amsmath, bbm, latexsym, units, mathtools}
\everymath{\displaystyle}
\usepackage[headsep=0.5cm,headheight=0cm, left=1 in,right= 1 in,top= 1 in,bottom= 1 in]{geometry}
\usepackage{dashrule}  % Package to use the command below to create lines between items
\newcommand{\litem}[1]{\item #1

\rule{\textwidth}{0.4pt}}
\pagestyle{fancy}
\lhead{}
\chead{Answer Key for Progress Quiz 5 Version B}
\rhead{}
\lfoot{9912-2038}
\cfoot{}
\rfoot{Spring 2021}
\begin{document}
\textbf{This key should allow you to understand why you choose the option you did (beyond just getting a question right or wrong). \href{https://xronos.clas.ufl.edu/mac1105spring2020/courseDescriptionAndMisc/Exams/LearningFromResults}{More instructions on how to use this key can be found here}.}

\textbf{If you have a suggestion to make the keys better, \href{https://forms.gle/CZkbZmPbC9XALEE88}{please fill out the short survey here}.}

\textit{Note: This key is auto-generated and may contain issues and/or errors. The keys are reviewed after each exam to ensure grading is done accurately. If there are issues (like duplicate options), they are noted in the offline gradebook. The keys are a work-in-progress to give students as many resources to improve as possible.}

\rule{\textwidth}{0.4pt}

\begin{enumerate}\litem{
Find the inverse of the function below (if it exists). Then, evaluate the inverse at $x = -15$ and choose the interval the $f^{-1}(-15)$ belongs to.
\[ f(x) = \sqrt[3]{5 x - 2} \]The solution is \( -674.6 \), which is option A.\begin{enumerate}[label=\Alph*.]
\item \( f^{-1}(-15) \in [-674.9, -674.14] \)

* This is the correct solution.
\item \( f^{-1}(-15) \in [-675.91, -675.39] \)

 Distractor 1: This corresponds to 
\item \( f^{-1}(-15) \in [675.02, 675.54] \)

 This solution corresponds to distractor 3.
\item \( f^{-1}(-15) \in [674.57, 675.24] \)

 This solution corresponds to distractor 2.
\item \( \text{ The function is not invertible for all Real numbers. } \)

 This solution corresponds to distractor 4.
\end{enumerate}

\textbf{General Comment:} Be sure you check that the function is 1-1 before trying to find the inverse!
}
\litem{
Choose the interval below that $f$ composed with $g$ at $x=1$ is in.
\[ f(x) = -2x^{3} +3 x^{2} +x -1 \text{ and } g(x) = 2x^{3} -1 x^{2} -2 x \]The solution is \( 3.0 \), which is option A.\begin{enumerate}[label=\Alph*.]
\item \( (f \circ g)(1) \in [1.72, 3.84] \)

* This is the correct solution
\item \( (f \circ g)(1) \in [8.47, 9.37] \)

 Distractor 2: Corresponds to being slightly off from the solution.
\item \( (f \circ g)(1) \in [6.95, 8.28] \)

 Distractor 3: Corresponds to being slightly off from the solution.
\item \( (f \circ g)(1) \in [-2.32, -0.32] \)

 Distractor 1: Corresponds to reversing the composition.
\item \( \text{It is not possible to compose the two functions.} \)


\end{enumerate}

\textbf{General Comment:} $f$ composed with $g$ at $x$ means $f(g(x))$. The order matters!
}
\litem{
Find the inverse of the function below. Then, evaluate the inverse at $x = 7$ and choose the interval that $f^{-1}(7)$ belongs to.
\[ f(x) = e^{x-5}+4 \]The solution is \( f^{-1}(7) = 6.099 \), which is option B.\begin{enumerate}[label=\Alph*.]
\item \( f^{-1}(7) \in [-3.92, -3.77] \)

 This solution corresponds to distractor 1.
\item \( f^{-1}(7) \in [6.08, 6.18] \)

 This is the solution.
\item \( f^{-1}(7) \in [4.61, 4.72] \)

 This solution corresponds to distractor 4.
\item \( f^{-1}(7) \in [6.44, 6.5] \)

 This solution corresponds to distractor 3.
\item \( f^{-1}(7) \in [6.34, 6.45] \)

 This solution corresponds to distractor 2.
\end{enumerate}

\textbf{General Comment:} Natural log and exponential functions always have an inverse. Once you switch the $x$ and $y$, use the conversion $ e^y = x \leftrightarrow y=\ln(x)$.
}
\litem{
Determine whether the function below is 1-1.
\[ f(x) = -18 x^2 + 132 x - 224 \]The solution is \( \text{no} \), which is option A.\begin{enumerate}[label=\Alph*.]
\item \( \text{No, because there is a $y$-value that goes to 2 different $x$-values.} \)

* This is the solution.
\item \( \text{Yes, the function is 1-1.} \)

Corresponds to believing the function passes the Horizontal Line test.
\item \( \text{No, because the range of the function is not $(-\infty, \infty)$.} \)

Corresponds to believing 1-1 means the range is all Real numbers.
\item \( \text{No, because there is an $x$-value that goes to 2 different $y$-values.} \)

Corresponds to the Vertical Line test, which checks if an expression is a function.
\item \( \text{No, because the domain of the function is not $(-\infty, \infty)$.} \)

Corresponds to believing 1-1 means the domain is all Real numbers.
\end{enumerate}

\textbf{General Comment:} There are only two valid options: The function is 1-1 OR No because there is a $y$-value that goes to 2 different $x$-values.
}
\litem{
Determine whether the function below is 1-1.
\[ f(x) = \sqrt{4 x - 20} \]The solution is \( \text{yes} \), which is option D.\begin{enumerate}[label=\Alph*.]
\item \( \text{No, because the range of the function is not $(-\infty, \infty)$.} \)

Corresponds to believing 1-1 means the range is all Real numbers.
\item \( \text{No, because there is an $x$-value that goes to 2 different $y$-values.} \)

Corresponds to the Vertical Line test, which checks if an expression is a function.
\item \( \text{No, because there is a $y$-value that goes to 2 different $x$-values.} \)

Corresponds to the Horizontal Line test, which this function passes.
\item \( \text{Yes, the function is 1-1.} \)

* This is the solution.
\item \( \text{No, because the domain of the function is not $(-\infty, \infty)$.} \)

Corresponds to believing 1-1 means the domain is all Real numbers.
\end{enumerate}

\textbf{General Comment:} There are only two valid options: The function is 1-1 OR No because there is a $y$-value that goes to 2 different $x$-values.
}
\litem{
Choose the interval below that $f$ composed with $g$ at $x=1$ is in.
\[ f(x) = 2x^{3} +4 x^{2} -2 x \text{ and } g(x) = -x^{3} +3 x^{2} -2 x + 1 \]The solution is \( 4.0 \), which is option C.\begin{enumerate}[label=\Alph*.]
\item \( (f \circ g)(1) \in [-34.1, -31.8] \)

 Distractor 3: Corresponds to being slightly off from the solution.
\item \( (f \circ g)(1) \in [-23.9, -21.6] \)

 Distractor 1: Corresponds to reversing the composition.
\item \( (f \circ g)(1) \in [1.9, 7.5] \)

* This is the correct solution
\item \( (f \circ g)(1) \in [8.4, 10.6] \)

 Distractor 2: Corresponds to being slightly off from the solution.
\item \( \text{It is not possible to compose the two functions.} \)


\end{enumerate}

\textbf{General Comment:} $f$ composed with $g$ at $x$ means $f(g(x))$. The order matters!
}
\litem{
Find the inverse of the function below (if it exists). Then, evaluate the inverse at $x = 10$ and choose the interval the $f^{-1}(10)$ belongs to.
\[ f(x) = \sqrt[3]{4 x + 3} \]The solution is \( 249.25 \), which is option A.\begin{enumerate}[label=\Alph*.]
\item \( f^{-1}(10) \in [248.46, 249.97] \)

* This is the correct solution.
\item \( f^{-1}(10) \in [-249.7, -248.81] \)

 This solution corresponds to distractor 2.
\item \( f^{-1}(10) \in [-251.41, -249.48] \)

 This solution corresponds to distractor 3.
\item \( f^{-1}(10) \in [249.4, 252.77] \)

 Distractor 1: This corresponds to 
\item \( \text{ The function is not invertible for all Real numbers. } \)

 This solution corresponds to distractor 4.
\end{enumerate}

\textbf{General Comment:} Be sure you check that the function is 1-1 before trying to find the inverse!
}
\litem{
Find the inverse of the function below. Then, evaluate the inverse at $x = 6$ and choose the interval that $f^{-1}(6)$ belongs to.
\[ f(x) = e^{x+4}+2 \]The solution is \( f^{-1}(6) = -2.614 \), which is option A.\begin{enumerate}[label=\Alph*.]
\item \( f^{-1}(6) \in [-3.24, -2.5] \)

 This is the solution.
\item \( f^{-1}(6) \in [2.44, 2.76] \)

 This solution corresponds to distractor 3.
\item \( f^{-1}(6) \in [5.09, 5.42] \)

 This solution corresponds to distractor 1.
\item \( f^{-1}(6) \in [3.8, 4.29] \)

 This solution corresponds to distractor 2.
\item \( f^{-1}(6) \in [4.12, 4.68] \)

 This solution corresponds to distractor 4.
\end{enumerate}

\textbf{General Comment:} Natural log and exponential functions always have an inverse. Once you switch the $x$ and $y$, use the conversion $ e^y = x \leftrightarrow y=\ln(x)$.
}
\litem{
Subtract the following functions, then choose the domain of the resulting function from the list below.
\[ f(x) = \sqrt{-5x-13}  \text{ and } g(x) = 4x + 6 \]The solution is \( \text{ The domain is all Real numbers less than or equal to} x = -2.6. \), which is option C.\begin{enumerate}[label=\Alph*.]
\item \( \text{ The domain is all Real numbers except } x = a, \text{ where } a \in [0.17, 7.17] \)


\item \( \text{ The domain is all Real numbers greater than or equal to } x = a, \text{ where } a \in [-7.67, 0.33] \)


\item \( \text{ The domain is all Real numbers less than or equal to } x = a, \text{ where } a \in [-3.6, -0.6] \)


\item \( \text{ The domain is all Real numbers except } x = a \text{ and } x = b, \text{ where } a \in [4.33, 10.33] \text{ and } b \in [3.2, 10.2] \)


\item \( \text{ The domain is all Real numbers. } \)


\end{enumerate}

\textbf{General Comment:} The new domain is the intersection of the previous domains.
}
\litem{
Subtract the following functions, then choose the domain of the resulting function from the list below.
\[ f(x) = \sqrt{-5x-15}  \text{ and } g(x) = 5x^{3} +4 x^{2} +x + 2 \]The solution is \( \text{ The domain is all Real numbers less than or equal to} x = -3.0. \), which is option C.\begin{enumerate}[label=\Alph*.]
\item \( \text{ The domain is all Real numbers except } x = a, \text{ where } a \in [-9.25, -5.25] \)


\item \( \text{ The domain is all Real numbers greater than or equal to } x = a, \text{ where } a \in [-5.5, -1.5] \)


\item \( \text{ The domain is all Real numbers less than or equal to } x = a, \text{ where } a \in [-5, 1] \)


\item \( \text{ The domain is all Real numbers except } x = a \text{ and } x = b, \text{ where } a \in [1.2, 10.2] \text{ and } b \in [6.33, 8.33] \)


\item \( \text{ The domain is all Real numbers. } \)


\end{enumerate}

\textbf{General Comment:} The new domain is the intersection of the previous domains.
}
\end{enumerate}

\end{document}