\documentclass{extbook}[14pt]
\usepackage{multicol, enumerate, enumitem, hyperref, color, soul, setspace, parskip, fancyhdr, amssymb, amsthm, amsmath, latexsym, units, mathtools}
\everymath{\displaystyle}
\usepackage[headsep=0.5cm,headheight=0cm, left=1 in,right= 1 in,top= 1 in,bottom= 1 in]{geometry}
\usepackage{dashrule}  % Package to use the command below to create lines between items
\newcommand{\litem}[1]{\item #1

\rule{\textwidth}{0.4pt}}
\pagestyle{fancy}
\lhead{}
\chead{Answer Key for Module9L Version B}
\rhead{}
\lfoot{3538-2368}
\cfoot{}
\rfoot{test}
\begin{document}
\textbf{This key should allow you to understand why you choose the option you did (beyond just getting a question right or wrong). \href{https://xronos.clas.ufl.edu/mac1105spring2020/courseDescriptionAndMisc/Exams/LearningFromResults}{More instructions on how to use this key can be found here}.}

\textbf{If you have a suggestion to make the keys better, \href{https://forms.gle/CZkbZmPbC9XALEE88}{please fill out the short survey here}.}

\textit{Note: This key is auto-generated and may contain issues and/or errors. The keys are reviewed after each exam to ensure grading is done accurately. If there are issues (like duplicate options), they are noted in the offline gradebook. The keys are a work-in-progress to give students as many resources to improve as possible.}

\rule{\textwidth}{0.4pt}

\begin{enumerate}\litem{
Evaluate $f$ composed with $g$ at $x=-1$.
\[ f(x) = 2x^{3} +2 x^{2} +4 x -1 \text{ and } g(x) = x^{3} + x^{2} -x -3 \]The solution is \( -17.0 \).\begin{enumerate}[label=\Alph*.]
\textbf{Plausible alternative answers include:} Distractor 1: Corresponds to reversing the composition.
* This is the correct solution
 Distractor 3: Corresponds to being slightly off from the solution.
 Distractor 2: Corresponds to being slightly off from the solution.

\end{enumerate}

\textbf{General Comment:} $f$ composed with $g$ at $x$ means $f(g(x))$. The order matters!
}
\litem{
Find the inverse of the function below (if it exists). If the inverse exists, evaluate the inverse at $x = -15.0$
\[ f(x) = 5 x^2 - 2 \]The solution is \( \text{ The function is not invertible for all Real numbers. } \).\begin{enumerate}[label=\Alph*.]
\textbf{Plausible alternative answers include:} Distractor 1: This corresponds to trying to find the inverse even though the function is not 1-1. 
 Distractor 2: This corresponds to finding the (nonexistent) inverse and not subtracting by the vertical shift.
 Distractor 3: This corresponds to finding the (nonexistent) inverse and dividing by a negative.
 Distractor 4: This corresponds to both distractors 2 and 3.
* This is the correct option.
\end{enumerate}

\textbf{General Comment:} Be sure you check that the function is 1-1 before trying to find the inverse!
}
\litem{
Find the inverse of the function below (if it exists). If the inverse exists, evaluate the inverse at $x = 7$.
\[ f(x) = e^{x-3}+5 \]The solution is \( f^{-1}(7) = 3.693 \).\begin{enumerate}[label=\Alph*.]
\textbf{Plausible alternative answers include:} This solution corresponds to distractor 3.
 This solution corresponds to distractor 4.
 This is the solution.
 This solution corresponds to distractor 2.
 This solution corresponds to distractor 1.
\end{enumerate}

\textbf{General Comment:} Natural log and exponential functions always have an inverse. Once you switch the $x$ and $y$, use the conversion $ e^y = x \leftrightarrow y=\ln(x)$.
}
\litem{
Determine whether the function below is 1-1. Provide reasoning for your response.
\[ f(x) = 36 x^2 - 192 x + 256 \]The solution is \( \text{no} \).\begin{enumerate}[label=\Alph*.]
\textbf{Plausible alternative answers include:}Corresponds to believing 1-1 means the domain is all Real numbers.
Corresponds to believing 1-1 means the range is all Real numbers.
Corresponds to believing the function passes the Horizontal Line test.
Corresponds to the Vertical Line test, which checks if an expression is a function.
* This is the solution.
\end{enumerate}

\textbf{General Comment:} There are only two valid options: The function is 1-1 OR No because there is a $y$-value that goes to 2 different $x$-values.
}
\litem{
Multiply the following functions and write the domain of the resulting function.
\[ f(x) = 6x + 7 \text{ and } g(x) = \sqrt{-6x+22}  \]The solution is \( \text{ The domain is all Real numbers less than or equal to} x = 3.67. \).\begin{enumerate}[label=\Alph*.]
\textbf{Plausible alternative answers include:}




\end{enumerate}

\textbf{General Comment:} The new domain is the intersection of the previous domains.
}
\litem{
Find the inverse of the function below (if it exists). If the inverse exists, evaluate the inverse at $x = 12.0$
\[ f(x) = 4 x^2 - 5 \]The solution is \( \text{ The function is not invertible for all Real numbers. } \).\begin{enumerate}[label=\Alph*.]
\textbf{Plausible alternative answers include:} Distractor 1: This corresponds to trying to find the inverse even though the function is not 1-1. 
 Distractor 2: This corresponds to finding the (nonexistent) inverse and not subtracting by the vertical shift.
 Distractor 3: This corresponds to finding the (nonexistent) inverse and dividing by a negative.
 Distractor 4: This corresponds to both distractors 2 and 3.
* This is the correct option.
\end{enumerate}

\textbf{General Comment:} Be sure you check that the function is 1-1 before trying to find the inverse!
}
\litem{
Find the inverse of the function below (if it exists). If the inverse exists, evaluate the inverse at $x = 8$.
\[ f(x) = \ln{(x+3)}-4 \]The solution is \( f^{-1}(8) = 162751.791 \).\begin{enumerate}[label=\Alph*.]
\textbf{Plausible alternative answers include:} This solution corresponds to distractor 1.
 This solution corresponds to distractor 3.
 This solution corresponds to distractor 4.
 This solution corresponds to distractor 2.
 This is the solution.
\end{enumerate}

\textbf{General Comment:} Natural log and exponential functions always have an inverse. Once you switch the $x$ and $y$, use the conversion $ e^y = x \leftrightarrow y=\ln(x)$.
}
\litem{
Add the following functions and write the domain of the resulting function.
\[ f(x) = 7x^{3} +7 x + 1 \text{ and } g(x) = \sqrt{5x+26}  \]The solution is \( \text{ The domain is all Real numbers greater than or equal to} x = -5.2. \).\begin{enumerate}[label=\Alph*.]
\textbf{Plausible alternative answers include:}




\end{enumerate}

\textbf{General Comment:} The new domain is the intersection of the previous domains.
}
\litem{
Determine whether the function below is 1-1. Provide reasoning for your response.
\[ f(x) = (6 x + 31)^3 \]The solution is \( \text{yes} \).\begin{enumerate}[label=\Alph*.]
\textbf{Plausible alternative answers include:}Corresponds to the Vertical Line test, which checks if an expression is a function.
Corresponds to believing 1-1 means the domain is all Real numbers.
Corresponds to the Horizontal Line test, which this function passes.
* This is the solution.
Corresponds to believing 1-1 means the range is all Real numbers.
\end{enumerate}

\textbf{General Comment:} There are only two valid options: The function is 1-1 OR No because there is a $y$-value that goes to 2 different $x$-values.
}
\litem{
Evaluate $f$ composed with $g$ at $x=1$.
\[ f(x) = -x^{3} -1 x^{2} +3 x -4 \text{ and } g(x) = -2x^{3} +4 x^{2} -3 x -3 \]The solution is \( 32.0 \).\begin{enumerate}[label=\Alph*.]
\textbf{Plausible alternative answers include:}* This is the correct solution
 Distractor 1: Corresponds to reversing the composition.
 Distractor 3: Corresponds to being slightly off from the solution.
 Distractor 2: Corresponds to being slightly off from the solution.

\end{enumerate}

\textbf{General Comment:} $f$ composed with $g$ at $x$ means $f(g(x))$. The order matters!
}
\end{enumerate}

\end{document}