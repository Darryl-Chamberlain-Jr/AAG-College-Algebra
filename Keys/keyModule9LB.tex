\documentclass{extbook}[14pt]
\usepackage{multicol, enumerate, enumitem, hyperref, color, soul, setspace, parskip, fancyhdr, amssymb, amsthm, amsmath, bbm, latexsym, units, mathtools}
\everymath{\displaystyle}
\usepackage[headsep=0.5cm,headheight=0cm, left=1 in,right= 1 in,top= 1 in,bottom= 1 in]{geometry}
\pagestyle{fancy}
\lhead{}
\chead{Answer Key for Module\,9L\,-\,Operations\,on\,Functions Version B}
\rhead{}
\lfoot{Summer\,C\,2020}
\cfoot{}
\rfoot{}
\begin{document}
\textbf{This key should allow you to understand why you choose the option you did (beyond just getting a question right or wrong). \href{https://xronos.clas.ufl.edu/mac1105spring2020/courseDescriptionAndMisc/Exams/LearningFromResults}{More instructions on how to use this key can be found here}.}

\textbf{If you have a suggestion to make the keys better, \href{https://forms.gle/CZkbZmPbC9XALEE88}{please fill out the short survey here}.}

\textit{Note: This key is auto-generated and may contain issues and/or errors. The keys are reviewed after each exam to ensure grading is done accurately. If there are issues (like duplicate options), they are noted in the offline gradebook. The keys are a work-in-progress to give students as many resources to improve as possible.}

\rule{\textwidth}{0.4pt}

1. Choose the interval below that $f$ composed with $g$ at $x=1$ is in.
\[ f(x) = 3x^{3} -2 x^{2} +2 x \text{ and } g(x) = 4x^{3} -1 x^{2} -2 x \] 
The solution is $ 3.0 $ 

\begin{enumerate}[label=\Alph*.] 
\item $ (f \circ g)(1) \in [84, 88] $ 

  Distractor 3: Corresponds to being slightly off from the solution. 
\item $ (f \circ g)(1) \in [89, 98] $ 

  Distractor 1: Corresponds to reversing the composition. 
\item $ (f \circ g)(1) \in [7, 16] $ 

  Distractor 2: Corresponds to being slightly off from the solution. 
\item $ (f \circ g)(1) \in [-4, 6] $ 

 * This is the correct solution 
\item $ \text{It is not possible to compose the two functions.} $ 

  
\end{enumerate} 
 
\textbf{General Comment:} General Comments: $f$ composed with $g$ at $x$ means $f(g(x))$. The order matters! 

-----------------------------------------------

2. Find the inverse of the function below. Then, evaluate the inverse at $x = 8$ and choose the interval that $f^{-1}(8)$ belongs to.
\[ f(x) = e^{x-3}-3 \] 
The solution is $ f^{-1}(8) = 5.398 $ 

\begin{enumerate}[label=\Alph*.] 
\item $ f^{-1}(8) \in [-2.81, -0.67] $ 

  This solution corresponds to distractor 2. 
\item $ f^{-1}(8) \in [4.56, 6.64] $ 

  This is the solution. 
\item $ f^{-1}(8) \in [-1.34, -0.52] $ 

  This solution corresponds to distractor 1. 
\item $ f^{-1}(8) \in [-2.81, -0.67] $ 

  This solution corresponds to distractor 4. 
\item $ f^{-1}(8) \in [-1.34, -0.52] $ 

  This solution corresponds to distractor 3. 
\end{enumerate} 
 
\textbf{General Comment:} Natural log and exponential functions always have an inverse. Once you switch the $x$ and $y$, use the conversion $ e^y = x \leftrightarrow y=\ln(x)$. 

-----------------------------------------------

3. Multiply the following functions, then choose the domain of the resulting function from the list below.
\[ f(x) = \sqrt{-5x+17}  \text{ and } g(x) = 8x^{2} +9 x + 3 \] 
The solution is $ \text{ The domain is all Real numbers less than or equal to} x = 3.4. $ 

\begin{enumerate}[label=\Alph*.] 
\item $ \text{ The domain is all Real numbers greater than or equal to } x = a, \text{ where } a \in [-9, 3] $ 

  
\item $ \text{ The domain is all Real numbers except } x = a, \text{ where } a \in [1, 7] $ 

  
\item $ \text{ The domain is all Real numbers less than or equal to } x = a, \text{ where } a \in [0, 7] $ 

  
\item $ \text{ The domain is all Real numbers except } x = a \text{ and } x = b, \text{ where } a \in [-6, -1] \text{ and } b \in [2, 6] $ 

  
\item $ \text{ The domain is all Real numbers. } $ 

  
\end{enumerate} 
 
\textbf{General Comment:} General Comments: The new domain is the intersection of the previous domains. 

-----------------------------------------------

4. Determine whether the function below is 1-1.
\[ f(x) = 18 x^2 + 105 x - 375 \] 
The solution is $ \text{no} $ 

\begin{enumerate}[label=\Alph*.] 
\item $ \text{Yes, the function is 1-1.} $ 

 Corresponds to believing the function passes the Horizontal Line test. 
\item $ \text{No, because the domain of the function is not $(-\infty, \infty)$.} $ 

 Corresponds to believing 1-1 means the domain is all Real numbers. 
\item $ \text{No, because the range of the function is not $(-\infty, \infty)$.} $ 

 Corresponds to believing 1-1 means the range is all Real numbers. 
\item $ \text{No, because there is an $x$-value that goes to 2 different $y$-values.} $ 

 Corresponds to the Vertical Line test, which checks if an expression is a function. 
\item $ \text{No, because there is a $y$-value that goes to 2 different $x$-values.} $ 

 * This is the solution. 
\end{enumerate} 
 
\textbf{General Comment:} \textbf{General Comments:} There are only two valid options: The function is 1-1 OR No because there is a $y$-value that goes to 2 different $x$-values. 

-----------------------------------------------

0. Find the inverse of the function below (if it exists). Then, evaluate the inverse at $x = 15$ and choose the interval the $f^{-1}(15)$ belongs to.
\[ f(x) = \sqrt[3]{5 x + 2} \] 
The solution is $ 674.6 $ 

\begin{enumerate}[label=\Alph*.] 
\item $ f^{-1}(15) \in [674.16, 674.66] $ 

 * This is the correct solution. 
\item $ f^{-1}(15) \in [-675.16, -674.53] $ 

  This solution corresponds to distractor 2. 
\item $ f^{-1}(15) \in [-675.99, -675.21] $ 

  This solution corresponds to distractor 3. 
\item $ f^{-1}(15) \in [675.36, 675.4] $ 

  Distractor 1: This corresponds to  
\item $ \text{ The function is not invertible for all Real numbers. } $ 

  This solution corresponds to distractor 4. 
\end{enumerate} 
 
\textbf{General Comment:} General Comments: Be sure you check that the function is 1-1 before trying to find the inverse! 

-----------------------------------------------


\end{document}

