\documentclass[14pt]{extbook}
\usepackage{multicol, enumerate, enumitem, hyperref, color, soul, setspace, parskip, fancyhdr} %General Packages
\usepackage{amssymb, amsthm, amsmath, latexsym, units, mathtools} %Math Packages
\everymath{\displaystyle} %All math in Display Style
% Packages with additional options
\usepackage[headsep=0.5cm,headheight=12pt, left=1 in,right= 1 in,top= 1 in,bottom= 1 in]{geometry}
\usepackage[usenames,dvipsnames]{xcolor}
\usepackage{dashrule}  % Package to use the command below to create lines between items
\newcommand{\litem}[1]{\item#1\hspace*{-1cm}\rule{\textwidth}{0.4pt}}
\pagestyle{fancy}
\lhead{Progress Quiz 7}
\chead{}
\rhead{Version B}
\lfoot{4173-5738}
\cfoot{}
\rfoot{Spring 2021}
\begin{document}

\begin{enumerate}
\litem{
Find the inverse of the function below (if it exists). Then, evaluate the inverse at $x = 12$ and choose the interval that $f^{-1}(12)$ belongs to.\[ f(x) = 3 x^2 - 5 \]\begin{enumerate}[label=\Alph*.]
\item \( f^{-1}(12) \in [1.03, 1.59] \)
\item \( f^{-1}(12) \in [6.21, 6.82] \)
\item \( f^{-1}(12) \in [4.65, 6.05] \)
\item \( f^{-1}(12) \in [2.31, 2.5] \)
\item \( \text{ The function is not invertible for all Real numbers. } \)

\end{enumerate} }
\litem{
Determine whether the function below is 1-1.\[ f(x) = \sqrt{-3 x + 13} \]\begin{enumerate}[label=\Alph*.]
\item \( \text{No, because the domain of the function is not $(-\infty, \infty)$.} \)
\item \( \text{No, because there is a $y$-value that goes to 2 different $x$-values.} \)
\item \( \text{Yes, the function is 1-1.} \)
\item \( \text{No, because the range of the function is not $(-\infty, \infty)$.} \)
\item \( \text{No, because there is an $x$-value that goes to 2 different $y$-values.} \)

\end{enumerate} }
\litem{
Choose the interval below that $f$ composed with $g$ at $x=1$ is in.\[ f(x) = 4x^{3} -1 x^{2} -x \text{ and } g(x) = 3x^{3} -4 x^{2} +x \]\begin{enumerate}[label=\Alph*.]
\item \( (f \circ g)(1) \in [5, 11] \)
\item \( (f \circ g)(1) \in [15, 20] \)
\item \( (f \circ g)(1) \in [-4, 2] \)
\item \( (f \circ g)(1) \in [-7, -3] \)
\item \( \text{It is not possible to compose the two functions.} \)

\end{enumerate} }
\litem{
Add the following functions, then choose the domain of the resulting function from the list below.\[ f(x) = \frac{5}{3x+20} \text{ and } g(x) = \frac{3}{3x+20} \]\begin{enumerate}[label=\Alph*.]
\item \( \text{ The domain is all Real numbers greater than or equal to } x = a, \text{ where } a \in [-8, -5] \)
\item \( \text{ The domain is all Real numbers except } x = a, \text{ where } a \in [-4.2, -1.2] \)
\item \( \text{ The domain is all Real numbers less than or equal to } x = a, \text{ where } a \in [-6, 0] \)
\item \( \text{ The domain is all Real numbers except } x = a \text{ and } x = b, \text{ where } a \in [-7.67, -3.67] \text{ and } b \in [-8.67, -0.67] \)
\item \( \text{ The domain is all Real numbers. } \)

\end{enumerate} }
\litem{
Find the inverse of the function below. Then, evaluate the inverse at $x = 10$ and choose the interval that $f^{-1}(10)$ belongs to.\[ f(x) = \ln{(x+5)}-4 \]\begin{enumerate}[label=\Alph*.]
\item \( f^{-1}(10) \in [392.43, 400.43] \)
\item \( f^{-1}(10) \in [140.41, 148.41] \)
\item \( f^{-1}(10) \in [1202609.28, 1202610.28] \)
\item \( f^{-1}(10) \in [1202599.28, 1202603.28] \)
\item \( f^{-1}(10) \in [3269011.37, 3269020.37] \)

\end{enumerate} }
\litem{
Subtract the following functions, then choose the domain of the resulting function from the list below.\[ f(x) = \frac{4}{5x+33} \text{ and } g(x) = \frac{3}{5x+28} \]\begin{enumerate}[label=\Alph*.]
\item \( \text{ The domain is all Real numbers less than or equal to } x = a, \text{ where } a \in [-5.67, -3.67] \)
\item \( \text{ The domain is all Real numbers greater than or equal to } x = a, \text{ where } a \in [1.67, 7.67] \)
\item \( \text{ The domain is all Real numbers except } x = a, \text{ where } a \in [-5.25, 0.75] \)
\item \( \text{ The domain is all Real numbers except } x = a \text{ and } x = b, \text{ where } a \in [-7.6, 1.4] \text{ and } b \in [-5.6, -1.6] \)
\item \( \text{ The domain is all Real numbers. } \)

\end{enumerate} }
\litem{
Determine whether the function below is 1-1.\[ f(x) = \sqrt{-6 x + 20} \]\begin{enumerate}[label=\Alph*.]
\item \( \text{No, because the range of the function is not $(-\infty, \infty)$.} \)
\item \( \text{No, because there is an $x$-value that goes to 2 different $y$-values.} \)
\item \( \text{Yes, the function is 1-1.} \)
\item \( \text{No, because the domain of the function is not $(-\infty, \infty)$.} \)
\item \( \text{No, because there is a $y$-value that goes to 2 different $x$-values.} \)

\end{enumerate} }
\litem{
Find the inverse of the function below (if it exists). Then, evaluate the inverse at $x = 12$ and choose the interval that $f^{-1}(12)$ belongs to.\[ f(x) = 3 x^2 - 4 \]\begin{enumerate}[label=\Alph*.]
\item \( f^{-1}(12) \in [1.27, 2.18] \)
\item \( f^{-1}(12) \in [4.75, 6.2] \)
\item \( f^{-1}(12) \in [3.01, 3.47] \)
\item \( f^{-1}(12) \in [1.97, 2.74] \)
\item \( \text{ The function is not invertible for all Real numbers. } \)

\end{enumerate} }
\litem{
Find the inverse of the function below. Then, evaluate the inverse at $x = 10$ and choose the interval that $f^{-1}(10)$ belongs to.\[ f(x) = e^{x+2}-5 \]\begin{enumerate}[label=\Alph*.]
\item \( f^{-1}(10) \in [-3.78, -3.1] \)
\item \( f^{-1}(10) \in [0.67, 0.82] \)
\item \( f^{-1}(10) \in [4.55, 5.02] \)
\item \( f^{-1}(10) \in [-3.07, -2.86] \)
\item \( f^{-1}(10) \in [-2.54, -2.23] \)

\end{enumerate} }
\litem{
Choose the interval below that $f$ composed with $g$ at $x=1$ is in.\[ f(x) = -2x^{3} +3 x^{2} -x + 1 \text{ and } g(x) = -x^{3} -2 x^{2} +4 x -2 \]\begin{enumerate}[label=\Alph*.]
\item \( (f \circ g)(1) \in [-2.75, -1.81] \)
\item \( (f \circ g)(1) \in [6.96, 7.32] \)
\item \( (f \circ g)(1) \in [-8.81, -7.52] \)
\item \( (f \circ g)(1) \in [-1.87, 0.65] \)
\item \( \text{It is not possible to compose the two functions.} \)

\end{enumerate} }
\end{enumerate}

\end{document}