\documentclass[14pt]{extbook}
\usepackage{multicol, enumerate, enumitem, hyperref, color, soul, setspace, parskip, fancyhdr} %General Packages
\usepackage{amssymb, amsthm, amsmath, latexsym, units, mathtools} %Math Packages
\everymath{\displaystyle} %All math in Display Style
% Packages with additional options
\usepackage[headsep=0.5cm,headheight=12pt, left=1 in,right= 1 in,top= 1 in,bottom= 1 in]{geometry}
\usepackage[usenames,dvipsnames]{xcolor}
\usepackage{dashrule}  % Package to use the command below to create lines between items
\newcommand{\litem}[1]{\item#1\hspace*{-1cm}\rule{\textwidth}{0.4pt}}
\pagestyle{fancy}
\lhead{Progress Quiz 7}
\chead{}
\rhead{Version A}
\lfoot{4173-5738}
\cfoot{}
\rfoot{Spring 2021}
\begin{document}

\begin{enumerate}
\litem{
Find the inverse of the function below (if it exists). Then, evaluate the inverse at $x = 11$ and choose the interval that $f^{-1}(11)$ belongs to.\[ f(x) = 2 x^2 + 4 \]\begin{enumerate}[label=\Alph*.]
\item \( f^{-1}(11) \in [1.65, 1.89] \)
\item \( f^{-1}(11) \in [4.83, 5.13] \)
\item \( f^{-1}(11) \in [2.82, 3.03] \)
\item \( f^{-1}(11) \in [2.59, 2.81] \)
\item \( \text{ The function is not invertible for all Real numbers. } \)

\end{enumerate} }
\litem{
Determine whether the function below is 1-1.\[ f(x) = 36 x^2 + 420 x + 1225 \]\begin{enumerate}[label=\Alph*.]
\item \( \text{Yes, the function is 1-1.} \)
\item \( \text{No, because the domain of the function is not $(-\infty, \infty)$.} \)
\item \( \text{No, because there is an $x$-value that goes to 2 different $y$-values.} \)
\item \( \text{No, because there is a $y$-value that goes to 2 different $x$-values.} \)
\item \( \text{No, because the range of the function is not $(-\infty, \infty)$.} \)

\end{enumerate} }
\litem{
Choose the interval below that $f$ composed with $g$ at $x=-1$ is in.\[ f(x) = 3x^{3} -3 x^{2} -2 x + 3 \text{ and } g(x) = -x^{3} -1 x^{2} -3 x \]\begin{enumerate}[label=\Alph*.]
\item \( (f \circ g)(-1) \in [-11, -1] \)
\item \( (f \circ g)(-1) \in [2, 4] \)
\item \( (f \circ g)(-1) \in [42, 45] \)
\item \( (f \circ g)(-1) \in [48, 58] \)
\item \( \text{It is not possible to compose the two functions.} \)

\end{enumerate} }
\litem{
Multiply the following functions, then choose the domain of the resulting function from the list below.\[ f(x) = \frac{1}{6x+29} \text{ and } g(x) = \frac{4}{5x-27} \]\begin{enumerate}[label=\Alph*.]
\item \( \text{ The domain is all Real numbers less than or equal to } x = a, \text{ where } a \in [-5.6, 3.4] \)
\item \( \text{ The domain is all Real numbers greater than or equal to } x = a, \text{ where } a \in [-8.67, -1.67] \)
\item \( \text{ The domain is all Real numbers except } x = a, \text{ where } a \in [-8.25, 0.75] \)
\item \( \text{ The domain is all Real numbers except } x = a \text{ and } x = b, \text{ where } a \in [-6.83, -3.83] \text{ and } b \in [3.4, 8.4] \)
\item \( \text{ The domain is all Real numbers. } \)

\end{enumerate} }
\litem{
Find the inverse of the function below. Then, evaluate the inverse at $x = 7$ and choose the interval that $f^{-1}(7)$ belongs to.\[ f(x) = e^{x+3}+2 \]\begin{enumerate}[label=\Alph*.]
\item \( f^{-1}(7) \in [4.38, 4.67] \)
\item \( f^{-1}(7) \in [-1.58, -1.34] \)
\item \( f^{-1}(7) \in [4, 4.2] \)
\item \( f^{-1}(7) \in [4.21, 4.54] \)
\item \( f^{-1}(7) \in [3.37, 3.41] \)

\end{enumerate} }
\litem{
Add the following functions, then choose the domain of the resulting function from the list below.\[ f(x) = \frac{1}{4x+25} \text{ and } g(x) = \frac{3}{3x-20} \]\begin{enumerate}[label=\Alph*.]
\item \( \text{ The domain is all Real numbers except } x = a, \text{ where } a \in [-5.17, -2.17] \)
\item \( \text{ The domain is all Real numbers greater than or equal to } x = a, \text{ where } a \in [-15.2, -4.2] \)
\item \( \text{ The domain is all Real numbers less than or equal to } x = a, \text{ where } a \in [0.33, 3.33] \)
\item \( \text{ The domain is all Real numbers except } x = a \text{ and } x = b, \text{ where } a \in [-9.25, -2.25] \text{ and } b \in [3.67, 16.67] \)
\item \( \text{ The domain is all Real numbers. } \)

\end{enumerate} }
\litem{
Determine whether the function below is 1-1.\[ f(x) = (5 x - 31)^3 \]\begin{enumerate}[label=\Alph*.]
\item \( \text{No, because there is an $x$-value that goes to 2 different $y$-values.} \)
\item \( \text{No, because the domain of the function is not $(-\infty, \infty)$.} \)
\item \( \text{Yes, the function is 1-1.} \)
\item \( \text{No, because there is a $y$-value that goes to 2 different $x$-values.} \)
\item \( \text{No, because the range of the function is not $(-\infty, \infty)$.} \)

\end{enumerate} }
\litem{
Find the inverse of the function below (if it exists). Then, evaluate the inverse at $x = 11$ and choose the interval the $f^{-1}(11)$ belongs to.\[ f(x) = \sqrt[3]{4 x + 2} \]\begin{enumerate}[label=\Alph*.]
\item \( f^{-1}(11) \in [332.1, 332.6] \)
\item \( f^{-1}(11) \in [332.7, 334.5] \)
\item \( f^{-1}(11) \in [-333, -332] \)
\item \( f^{-1}(11) \in [-335.1, -332.4] \)
\item \( \text{ The function is not invertible for all Real numbers. } \)

\end{enumerate} }
\litem{
Find the inverse of the function below. Then, evaluate the inverse at $x = 9$ and choose the interval that $f^{-1}(9)$ belongs to.\[ f(x) = \ln{(x+4)}-2 \]\begin{enumerate}[label=\Alph*.]
\item \( f^{-1}(9) \in [1089.63, 1096.63] \)
\item \( f^{-1}(9) \in [59877.14, 59887.14] \)
\item \( f^{-1}(9) \in [442404.39, 442424.39] \)
\item \( f^{-1}(9) \in [59870.14, 59872.14] \)
\item \( f^{-1}(9) \in [144.41, 150.41] \)

\end{enumerate} }
\litem{
Choose the interval below that $f$ composed with $g$ at $x=-1$ is in.\[ f(x) = 2x^{3} +3 x^{2} -x \text{ and } g(x) = 4x^{3} +3 x^{2} +x + 4 \]\begin{enumerate}[label=\Alph*.]
\item \( (f \circ g)(-1) \in [25, 29] \)
\item \( (f \circ g)(-1) \in [42, 43] \)
\item \( (f \circ g)(-1) \in [46, 51] \)
\item \( (f \circ g)(-1) \in [32, 38] \)
\item \( \text{It is not possible to compose the two functions.} \)

\end{enumerate} }
\end{enumerate}

\end{document}