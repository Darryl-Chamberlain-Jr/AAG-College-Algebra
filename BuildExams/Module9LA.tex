\documentclass[14pt]{extbook}
\usepackage{multicol, enumerate, enumitem, hyperref, color, soul, setspace, parskip, fancyhdr} %General Packages
\usepackage{amssymb, amsthm, amsmath, latexsym, units, mathtools} %Math Packages
\everymath{\displaystyle} %All math in Display Style
% Packages with additional options
\usepackage[headsep=0.5cm,headheight=12pt, left=1 in,right= 1 in,top= 1 in,bottom= 1 in]{geometry}
\usepackage[usenames,dvipsnames]{xcolor}
\usepackage{dashrule}  % Package to use the command below to create lines between items
\newcommand{\litem}[1]{\item#1\hspace*{-1cm}\rule{\textwidth}{0.4pt}}
\pagestyle{fancy}
\lhead{Progress Quiz 6}
\chead{}
\rhead{Version A}
\lfoot{9689-6866}
\cfoot{}
\rfoot{Spring 2021}
\begin{document}

\begin{enumerate}
\litem{
Choose the interval below that $f$ composed with $g$ at $x=-1$ is in.\[ f(x) = 2x^{3} +4 x^{2} +3 x + 1 \text{ and } g(x) = 2x^{3} -1 x^{2} -4 x -3 \]\begin{enumerate}[label=\Alph*.]
\item \( (f \circ g)(-1) \in [-13, -7] \)
\item \( (f \circ g)(-1) \in [-3, 1] \)
\item \( (f \circ g)(-1) \in [-6, -4] \)
\item \( (f \circ g)(-1) \in [-1, 4] \)
\item \( \text{It is not possible to compose the two functions.} \)

\end{enumerate} }
\litem{
Choose the interval below that $f$ composed with $g$ at $x=1$ is in.\[ f(x) = 3x^{3} -1 x^{2} +3 x -4 \text{ and } g(x) = -2x^{3} -3 x^{2} +4 x \]\begin{enumerate}[label=\Alph*.]
\item \( (f \circ g)(1) \in [-12, -9] \)
\item \( (f \circ g)(1) \in [-23, -15] \)
\item \( (f \circ g)(1) \in [3, 8] \)
\item \( (f \circ g)(1) \in [-2, 3] \)
\item \( \text{It is not possible to compose the two functions.} \)

\end{enumerate} }
\litem{
Subtract the following functions, then choose the domain of the resulting function from the list below.\[ f(x) = 4x^{4} +7 x^{3} +2 x^{2} + 7 \text{ and } g(x) = 3x^{4} +9 x^{3} + x^{2} +4 x + 5 \]\begin{enumerate}[label=\Alph*.]
\item \( \text{ The domain is all Real numbers except } x = a, \text{ where } a \in [-7.8, 1.2] \)
\item \( \text{ The domain is all Real numbers less than or equal to } x = a, \text{ where } a \in [-2.2, 4.8] \)
\item \( \text{ The domain is all Real numbers greater than or equal to } x = a, \text{ where } a \in [-9.67, 4.33] \)
\item \( \text{ The domain is all Real numbers except } x = a \text{ and } x = b, \text{ where } a \in [-8.4, -1.4] \text{ and } b \in [3.33, 9.33] \)
\item \( \text{ The domain is all Real numbers. } \)

\end{enumerate} }
\litem{
Determine whether the function below is 1-1.\[ f(x) = (3 x + 14)^3 \]\begin{enumerate}[label=\Alph*.]
\item \( \text{No, because the range of the function is not $(-\infty, \infty)$.} \)
\item \( \text{Yes, the function is 1-1.} \)
\item \( \text{No, because the domain of the function is not $(-\infty, \infty)$.} \)
\item \( \text{No, because there is an $x$-value that goes to 2 different $y$-values.} \)
\item \( \text{No, because there is a $y$-value that goes to 2 different $x$-values.} \)

\end{enumerate} }
\litem{
Find the inverse of the function below (if it exists). Then, evaluate the inverse at $x = 10$ and choose the interval that $f^{-1}(10)$ belongs to.\[ f(x) = 4 x^2 + 5 \]\begin{enumerate}[label=\Alph*.]
\item \( f^{-1}(10) \in [4.08, 4.2] \)
\item \( f^{-1}(10) \in [1.76, 1.96] \)
\item \( f^{-1}(10) \in [2, 2.2] \)
\item \( f^{-1}(10) \in [1.1, 1.13] \)
\item \( \text{ The function is not invertible for all Real numbers. } \)

\end{enumerate} }
\litem{
Find the inverse of the function below (if it exists). Then, evaluate the inverse at $x = -15$ and choose the interval the $f^{-1}(-15)$ belongs to.\[ f(x) = \sqrt[3]{4 x + 5} \]\begin{enumerate}[label=\Alph*.]
\item \( f^{-1}(-15) \in [-843.34, -841.99] \)
\item \( f^{-1}(-15) \in [-845.3, -844.73] \)
\item \( f^{-1}(-15) \in [843.94, 846.01] \)
\item \( f^{-1}(-15) \in [842.15, 843.44] \)
\item \( \text{ The function is not invertible for all Real numbers. } \)

\end{enumerate} }
\litem{
Determine whether the function below is 1-1.\[ f(x) = (6 x - 19)^3 \]\begin{enumerate}[label=\Alph*.]
\item \( \text{No, because the domain of the function is not $(-\infty, \infty)$.} \)
\item \( \text{No, because there is a $y$-value that goes to 2 different $x$-values.} \)
\item \( \text{No, because the range of the function is not $(-\infty, \infty)$.} \)
\item \( \text{Yes, the function is 1-1.} \)
\item \( \text{No, because there is an $x$-value that goes to 2 different $y$-values.} \)

\end{enumerate} }
\litem{
Find the inverse of the function below. Then, evaluate the inverse at $x = 7$ and choose the interval that $f^{-1}(7)$ belongs to.\[ f(x) = \ln{(x-2)}-4 \]\begin{enumerate}[label=\Alph*.]
\item \( f^{-1}(7) \in [8097.3, 8103.4] \)
\item \( f^{-1}(7) \in [59874.5, 59878.1] \)
\item \( f^{-1}(7) \in [21.5, 26.5] \)
\item \( f^{-1}(7) \in [59871.2, 59872.5] \)
\item \( f^{-1}(7) \in [141.8, 145.4] \)

\end{enumerate} }
\litem{
Add the following functions, then choose the domain of the resulting function from the list below.\[ f(x) = \sqrt{6x-42}  \text{ and } g(x) = 5x + 4 \]\begin{enumerate}[label=\Alph*.]
\item \( \text{ The domain is all Real numbers less than or equal to } x = a, \text{ where } a \in [0.8, 5.8] \)
\item \( \text{ The domain is all Real numbers except } x = a, \text{ where } a \in [3.33, 13.33] \)
\item \( \text{ The domain is all Real numbers greater than or equal to } x = a, \text{ where } a \in [1, 9] \)
\item \( \text{ The domain is all Real numbers except } x = a \text{ and } x = b, \text{ where } a \in [6.25, 9.25] \text{ and } b \in [6.67, 9.67] \)
\item \( \text{ The domain is all Real numbers. } \)

\end{enumerate} }
\litem{
Find the inverse of the function below. Then, evaluate the inverse at $x = 7$ and choose the interval that $f^{-1}(7)$ belongs to.\[ f(x) = e^{x-3}-3 \]\begin{enumerate}[label=\Alph*.]
\item \( f^{-1}(7) \in [-1.75, -1.29] \)
\item \( f^{-1}(7) \in [-1.04, -0.1] \)
\item \( f^{-1}(7) \in [-1.75, -1.29] \)
\item \( f^{-1}(7) \in [5.24, 6.53] \)
\item \( f^{-1}(7) \in [-1.04, -0.1] \)

\end{enumerate} }
\end{enumerate}

\end{document}