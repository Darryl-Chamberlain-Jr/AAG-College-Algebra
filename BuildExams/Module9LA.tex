\documentclass[14pt]{extbook}
\usepackage{multicol, enumerate, enumitem, hyperref, color, soul, setspace, parskip, fancyhdr} %General Packages
\usepackage{amssymb, amsthm, amsmath, bbm, latexsym, units, mathtools} %Math Packages
\everymath{\displaystyle} %All math in Display Style
% Packages with additional options
\usepackage[headsep=0.5cm,headheight=12pt, left=1 in,right= 1 in,top= 1 in,bottom= 1 in]{geometry}
\usepackage[usenames,dvipsnames]{xcolor}
\usepackage{dashrule}  % Package to use the command below to create lines between items
\newcommand{\litem}[1]{\item#1\hspace*{-1cm}\rule{\textwidth}{0.4pt}}
\pagestyle{fancy}
\lhead{Progress Quiz 9}
\chead{}
\rhead{Version A}
\lfoot{8590-6105}
\cfoot{}
\rfoot{Fall 2020}
\begin{document}

\begin{enumerate}
\litem{
Subtract the following functions, then choose the domain of the resulting function from the list below.\[ f(x) = 6x + 3 \text{ and } g(x) = \frac{4}{4x+17} \]\begin{enumerate}[label=\Alph*.]
\item \( \text{ The domain is all Real numbers except } x = a, \text{ where } a \in [-4.25, 0.75] \)
\item \( \text{ The domain is all Real numbers less than or equal to } x = a, \text{ where } a \in [-8.5, 5.5] \)
\item \( \text{ The domain is all Real numbers greater than or equal to } x = a, \text{ where } a \in [-8.75, -4.75] \)
\item \( \text{ The domain is all Real numbers except } x = a \text{ and } x = b, \text{ where } a \in [2.75, 8.75] \text{ and } b \in [-1.75, 7.25] \)
\item \( \text{ The domain is all Real numbers. } \)

\end{enumerate} }
\litem{
Find the inverse of the function below (if it exists). Then, evaluate the inverse at $x = -12$ and choose the interval the $f^{-1}(-12)$ belongs to.\[ f(x) = \sqrt[3]{4 x + 3} \]\begin{enumerate}[label=\Alph*.]
\item \( f^{-1}(-12) \in [-433.9, -431.9] \)
\item \( f^{-1}(-12) \in [-431.6, -428.2] \)
\item \( f^{-1}(-12) \in [431.8, 433.7] \)
\item \( f^{-1}(-12) \in [429, 432.7] \)
\item \( \text{ The function is not invertible for all Real numbers. } \)

\end{enumerate} }
\litem{
Determine whether the function below is 1-1.\[ f(x) = \sqrt{-4 x - 15} \]\begin{enumerate}[label=\Alph*.]
\item \( \text{No, because there is an $x$-value that goes to 2 different $y$-values.} \)
\item \( \text{No, because there is a $y$-value that goes to 2 different $x$-values.} \)
\item \( \text{No, because the range of the function is not $(-\infty, \infty)$.} \)
\item \( \text{Yes, the function is 1-1.} \)
\item \( \text{No, because the domain of the function is not $(-\infty, \infty)$.} \)

\end{enumerate} }
\litem{
Find the inverse of the function below (if it exists). Then, evaluate the inverse at $x = 11$ and choose the interval the $f^{-1}(11)$ belongs to.\[ f(x) = \sqrt[3]{5 x + 3} \]\begin{enumerate}[label=\Alph*.]
\item \( f^{-1}(11) \in [-265.87, -265.38] \)
\item \( f^{-1}(11) \in [265.17, 265.77] \)
\item \( f^{-1}(11) \in [-267.63, -266.79] \)
\item \( f^{-1}(11) \in [266.38, 267.55] \)
\item \( \text{ The function is not invertible for all Real numbers. } \)

\end{enumerate} }
\litem{
Find the inverse of the function below. Then, evaluate the inverse at $x = 5$ and choose the interval that $f^{-1}(5)$ belongs to.\[ f(x) = \ln{(x+2)}-3 \]\begin{enumerate}[label=\Alph*.]
\item \( f^{-1}(5) \in [2975.96, 2980.96] \)
\item \( f^{-1}(5) \in [4.39, 9.39] \)
\item \( f^{-1}(5) \in [14.09, 20.09] \)
\item \( f^{-1}(5) \in [1088.63, 1100.63] \)
\item \( f^{-1}(5) \in [2979.96, 2983.96] \)

\end{enumerate} }
\litem{
Choose the interval below that $f$ composed with $g$ at $x=-1$ is in.\[ f(x) = 2x^{3} +4 x^{2} +x \text{ and } g(x) = -x^{3} +2 x^{2} +3 x \]\begin{enumerate}[label=\Alph*.]
\item \( (f \circ g)(-1) \in [8.77, 9.31] \)
\item \( (f \circ g)(-1) \in [9.79, 11.27] \)
\item \( (f \circ g)(-1) \in [2.48, 5.86] \)
\item \( (f \circ g)(-1) \in [-0.31, 1.82] \)
\item \( \text{It is not possible to compose the two functions.} \)

\end{enumerate} }
\litem{
Choose the interval below that $f$ composed with $g$ at $x=-1$ is in.\[ f(x) = 3x^{3} +3 x^{2} +x \text{ and } g(x) = -3x^{3} -2 x^{2} +2 x \]\begin{enumerate}[label=\Alph*.]
\item \( (f \circ g)(-1) \in [1.2, 7.3] \)
\item \( (f \circ g)(-1) \in [-1.9, 0.2] \)
\item \( (f \circ g)(-1) \in [7.5, 9.7] \)
\item \( (f \circ g)(-1) \in [-1.9, 0.2] \)
\item \( \text{It is not possible to compose the two functions.} \)

\end{enumerate} }
\litem{
Find the inverse of the function below. Then, evaluate the inverse at $x = 9$ and choose the interval that $f^{-1}(9)$ belongs to.\[ f(x) = \ln{(x-2)}+5 \]\begin{enumerate}[label=\Alph*.]
\item \( f^{-1}(9) \in [1202606.28, 1202610.28] \)
\item \( f^{-1}(9) \in [59877.14, 59882.14] \)
\item \( f^{-1}(9) \in [1101.63, 1103.63] \)
\item \( f^{-1}(9) \in [49.6, 54.6] \)
\item \( f^{-1}(9) \in [54.6, 59.6] \)

\end{enumerate} }
\litem{
Multiply the following functions, then choose the domain of the resulting function from the list below.\[ f(x) = \frac{4}{3x+17} \text{ and } g(x) = \frac{4}{5x-28} \]\begin{enumerate}[label=\Alph*.]
\item \( \text{ The domain is all Real numbers except } x = a, \text{ where } a \in [3.67, 10.67] \)
\item \( \text{ The domain is all Real numbers greater than or equal to } x = a, \text{ where } a \in [0.33, 7.33] \)
\item \( \text{ The domain is all Real numbers less than or equal to } x = a, \text{ where } a \in [-5.33, -2.33] \)
\item \( \text{ The domain is all Real numbers except } x = a \text{ and } x = b, \text{ where } a \in [-12.67, -3.67] \text{ and } b \in [4.6, 13.6] \)
\item \( \text{ The domain is all Real numbers. } \)

\end{enumerate} }
\litem{
Determine whether the function below is 1-1.\[ f(x) = \sqrt{3 x - 20} \]\begin{enumerate}[label=\Alph*.]
\item \( \text{No, because the range of the function is not $(-\infty, \infty)$.} \)
\item \( \text{No, because there is a $y$-value that goes to 2 different $x$-values.} \)
\item \( \text{Yes, the function is 1-1.} \)
\item \( \text{No, because the domain of the function is not $(-\infty, \infty)$.} \)
\item \( \text{No, because there is an $x$-value that goes to 2 different $y$-values.} \)

\end{enumerate} }
\end{enumerate}

\end{document}