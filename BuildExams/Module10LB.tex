\documentclass[14pt]{extbook}
\usepackage{multicol, enumerate, enumitem, hyperref, color, soul, setspace, parskip, fancyhdr} %General Packages
\usepackage{amssymb, amsthm, amsmath, latexsym, units, mathtools} %Math Packages
\everymath{\displaystyle} %All math in Display Style
% Packages with additional options
\usepackage[headsep=0.5cm,headheight=12pt, left=1 in,right= 1 in,top= 1 in,bottom= 1 in]{geometry}
\usepackage[usenames,dvipsnames]{xcolor}
\usepackage{dashrule}  % Package to use the command below to create lines between items
\newcommand{\litem}[1]{\item#1\hspace*{-1cm}\rule{\textwidth}{0.4pt}}
\pagestyle{fancy}
\lhead{Progress Quiz 6}
\chead{}
\rhead{Version B}
\lfoot{9689-6866}
\cfoot{}
\rfoot{Spring 2021}
\begin{document}

\begin{enumerate}
\litem{
Perform the division below. Then, find the intervals that correspond to the quotient in the form $ax^2+bx+c$ and remainder $r$.\[ \frac{9x^{3} -72 x^{2} +155 x -104}{x -5} \]\begin{enumerate}[label=\Alph*.]
\item \( a \in [40, 48], \text{   } b \in [150, 154], \text{   } c \in [918, 926], \text{   and   } r \in [4488, 4500]. \)
\item \( a \in [7, 12], \text{   } b \in [-30, -19], \text{   } c \in [17, 23], \text{   and   } r \in [-6, -1]. \)
\item \( a \in [7, 12], \text{   } b \in [-117, -112], \text{   } c \in [735, 743], \text{   and   } r \in [-3804, -3797]. \)
\item \( a \in [7, 12], \text{   } b \in [-37, -33], \text{   } c \in [8, 16], \text{   and   } r \in [-64, -57]. \)
\item \( a \in [40, 48], \text{   } b \in [-299, -294], \text{   } c \in [1635, 1650], \text{   and   } r \in [-8308, -8300]. \)

\end{enumerate} }
\litem{
Factor the polynomial below completely. Then, choose the intervals the zeros of the polynomial belong to, where $z_1 \leq z_2 \leq z_3$. \textit{To make the problem easier, all zeros are between -5 and 5.}\[ f(x) = 9x^{3} -39 x^{2} -38 x + 40 \]\begin{enumerate}[label=\Alph*.]
\item \( z_1 \in [-5.47, -4.95], \text{   }  z_2 \in [-1.65, -0.97], \text{   and   } z_3 \in [0.61, 1.1] \)
\item \( z_1 \in [-1.17, -0.63], \text{   }  z_2 \in [1.41, 1.74], \text{   and   } z_3 \in [4.72, 5.11] \)
\item \( z_1 \in [-1.67, -0.77], \text{   }  z_2 \in [0.66, 1.21], \text{   and   } z_3 \in [4.72, 5.11] \)
\item \( z_1 \in [-5.47, -4.95], \text{   }  z_2 \in [-2.01, -1.92], \text{   and   } z_3 \in [0.16, 0.7] \)
\item \( z_1 \in [-5.47, -4.95], \text{   }  z_2 \in [-0.89, -0.18], \text{   and   } z_3 \in [1.32, 1.46] \)

\end{enumerate} }
\litem{
Factor the polynomial below completely, knowing that $x-4$ is a factor. Then, choose the intervals the zeros of the polynomial belong to, where $z_1 \leq z_2 \leq z_3 \leq z_4$. \textit{To make the problem easier, all zeros are between -5 and 5.}\[ f(x) = 25x^{4} -150 x^{3} +191 x^{2} +54 x -72 \]\begin{enumerate}[label=\Alph*.]
\item \( z_1 \in [-2.34, -0.86], \text{   }  z_2 \in [0.67, 4.67], z_3 \in [1.66, 2.34], \text{   and   } z_4 \in [3.84, 4.71] \)
\item \( z_1 \in [-4.09, -3.29], \text{   }  z_2 \in [-4, 0], z_3 \in [-1.03, -0.55], \text{   and   } z_4 \in [0.3, 1.01] \)
\item \( z_1 \in [-4.09, -3.29], \text{   }  z_2 \in [-4, 0], z_3 \in [-0.23, -0.09], \text{   and   } z_4 \in [2.81, 3.08] \)
\item \( z_1 \in [-0.88, -0.58], \text{   }  z_2 \in [0.6, 1.6], z_3 \in [1.66, 2.34], \text{   and   } z_4 \in [3.84, 4.71] \)
\item \( z_1 \in [-4.09, -3.29], \text{   }  z_2 \in [-4, 0], z_3 \in [-1.68, -1.55], \text{   and   } z_4 \in [1.61, 2.09] \)

\end{enumerate} }
\litem{
Factor the polynomial below completely, knowing that $x+4$ is a factor. Then, choose the intervals the zeros of the polynomial belong to, where $z_1 \leq z_2 \leq z_3 \leq z_4$. \textit{To make the problem easier, all zeros are between -5 and 5.}\[ f(x) = 20x^{4} +63 x^{3} -108 x^{2} -172 x -48 \]\begin{enumerate}[label=\Alph*.]
\item \( z_1 \in [-5.3, -3.9], \text{   }  z_2 \in [-0.9, -0.58], z_3 \in [-0.54, -0.22], \text{   and   } z_4 \in [1.4, 2.8] \)
\item \( z_1 \in [-2.1, -1.8], \text{   }  z_2 \in [1.27, 1.54], z_3 \in [2.28, 2.52], \text{   and   } z_4 \in [3.9, 4.6] \)
\item \( z_1 \in [-5.3, -3.9], \text{   }  z_2 \in [-2.97, -2.47], z_3 \in [-1.51, -1.26], \text{   and   } z_4 \in [1.4, 2.8] \)
\item \( z_1 \in [-2.1, -1.8], \text{   }  z_2 \in [-0.44, 0.14], z_3 \in [2.54, 3.41], \text{   and   } z_4 \in [3.9, 4.6] \)
\item \( z_1 \in [-2.1, -1.8], \text{   }  z_2 \in [0.31, 0.94], z_3 \in [0, 0.8], \text{   and   } z_4 \in [3.9, 4.6] \)

\end{enumerate} }
\litem{
What are the \textit{possible Rational} roots of the polynomial below?\[ f(x) = 4x^{2} +6 x + 6 \]\begin{enumerate}[label=\Alph*.]
\item \( \pm 1,\pm 2,\pm 4 \)
\item \( \text{ All combinations of: }\frac{\pm 1,\pm 2,\pm 4}{\pm 1,\pm 2,\pm 3,\pm 6} \)
\item \( \pm 1,\pm 2,\pm 3,\pm 6 \)
\item \( \text{ All combinations of: }\frac{\pm 1,\pm 2,\pm 3,\pm 6}{\pm 1,\pm 2,\pm 4} \)
\item \( \text{ There is no formula or theorem that tells us all possible Rational roots.} \)

\end{enumerate} }
\litem{
Factor the polynomial below completely. Then, choose the intervals the zeros of the polynomial belong to, where $z_1 \leq z_2 \leq z_3$. \textit{To make the problem easier, all zeros are between -5 and 5.}\[ f(x) = 16x^{3} +32 x^{2} -25 x -50 \]\begin{enumerate}[label=\Alph*.]
\item \( z_1 \in [-2.09, -1.86], \text{   }  z_2 \in [-1.06, -0.54], \text{   and   } z_3 \in [0.67, 1.19] \)
\item \( z_1 \in [-2.09, -1.86], \text{   }  z_2 \in [-1.28, -1.11], \text{   and   } z_3 \in [1.23, 1.9] \)
\item \( z_1 \in [-1.27, -0.82], \text{   }  z_2 \in [1.14, 1.32], \text{   and   } z_3 \in [1.73, 3.34] \)
\item \( z_1 \in [-5.26, -4.5], \text{   }  z_2 \in [0.09, 0.34], \text{   and   } z_3 \in [1.73, 3.34] \)
\item \( z_1 \in [-0.87, 0.04], \text{   }  z_2 \in [0.5, 0.94], \text{   and   } z_3 \in [1.73, 3.34] \)

\end{enumerate} }
\litem{
Perform the division below. Then, find the intervals that correspond to the quotient in the form $ax^2+bx+c$ and remainder $r$.\[ \frac{8x^{3} +20 x^{2} -56 x -27}{x + 4} \]\begin{enumerate}[label=\Alph*.]
\item \( a \in [5, 14], \text{   } b \in [43, 58], \text{   } c \in [152, 156], \text{   and   } r \in [579, 585]. \)
\item \( a \in [-35, -28], \text{   } b \in [148, 153], \text{   } c \in [-652, -645], \text{   and   } r \in [2558, 2573]. \)
\item \( a \in [5, 14], \text{   } b \in [-18, -7], \text{   } c \in [-12, 0], \text{   and   } r \in [0, 7]. \)
\item \( a \in [5, 14], \text{   } b \in [-24, -18], \text{   } c \in [43, 45], \text{   and   } r \in [-249, -246]. \)
\item \( a \in [-35, -28], \text{   } b \in [-109, -105], \text{   } c \in [-495, -485], \text{   and   } r \in [-1984, -1978]. \)

\end{enumerate} }
\litem{
What are the \textit{possible Rational} roots of the polynomial below?\[ f(x) = 7x^{3} +2 x^{2} +2 x + 6 \]\begin{enumerate}[label=\Alph*.]
\item \( \pm 1,\pm 2,\pm 3,\pm 6 \)
\item \( \text{ All combinations of: }\frac{\pm 1,\pm 7}{\pm 1,\pm 2,\pm 3,\pm 6} \)
\item \( \text{ All combinations of: }\frac{\pm 1,\pm 2,\pm 3,\pm 6}{\pm 1,\pm 7} \)
\item \( \pm 1,\pm 7 \)
\item \( \text{ There is no formula or theorem that tells us all possible Rational roots.} \)

\end{enumerate} }
\litem{
Perform the division below. Then, find the intervals that correspond to the quotient in the form $ax^2+bx+c$ and remainder $r$.\[ \frac{4x^{3} +12 x^{2} -20}{x + 2} \]\begin{enumerate}[label=\Alph*.]
\item \( a \in [-10, -7], b \in [-5.6, -3.6], c \in [-15, -4], \text{ and } r \in [-36, -31]. \)
\item \( a \in [3, 13], b \in [-1.5, 0.5], c \in [-2, 3], \text{ and } r \in [-20, -16]. \)
\item \( a \in [3, 13], b \in [18.9, 21.4], c \in [39, 43], \text{ and } r \in [57, 61]. \)
\item \( a \in [-10, -7], b \in [27.9, 29.1], c \in [-60, -55], \text{ and } r \in [90, 97]. \)
\item \( a \in [3, 13], b \in [3.1, 4.1], c \in [-15, -4], \text{ and } r \in [-6, -1]. \)

\end{enumerate} }
\litem{
Perform the division below. Then, find the intervals that correspond to the quotient in the form $ax^2+bx+c$ and remainder $r$.\[ \frac{20x^{3} +63 x^{2} -31}{x + 3} \]\begin{enumerate}[label=\Alph*.]
\item \( a \in [18, 23], b \in [2, 9], c \in [-9, -8], \text{ and } r \in [-5, -2]. \)
\item \( a \in [-61, -54], b \in [239, 251], c \in [-733, -721], \text{ and } r \in [2154, 2161]. \)
\item \( a \in [-61, -54], b \in [-118, -115], c \in [-351, -349], \text{ and } r \in [-1084, -1082]. \)
\item \( a \in [18, 23], b \in [122, 127], c \in [366, 371], \text{ and } r \in [1073, 1078]. \)
\item \( a \in [18, 23], b \in [-22, -16], c \in [68, 70], \text{ and } r \in [-306, -300]. \)

\end{enumerate} }
\end{enumerate}

\end{document}