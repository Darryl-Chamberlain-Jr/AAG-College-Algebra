\documentclass[14pt]{extbook}
\usepackage{multicol, enumerate, enumitem, hyperref, color, soul, setspace, parskip, fancyhdr} %General Packages
\usepackage{amssymb, amsthm, amsmath, latexsym, units, mathtools} %Math Packages
\everymath{\displaystyle} %All math in Display Style
% Packages with additional options
\usepackage[headsep=0.5cm,headheight=12pt, left=1 in,right= 1 in,top= 1 in,bottom= 1 in]{geometry}
\usepackage[usenames,dvipsnames]{xcolor}
\usepackage{dashrule}  % Package to use the command below to create lines between items
\newcommand{\litem}[1]{\item#1\hspace*{-1cm}\rule{\textwidth}{0.4pt}}
\pagestyle{fancy}
\lhead{Progress Quiz 7}
\chead{}
\rhead{Version B}
\lfoot{4173-5738}
\cfoot{}
\rfoot{Spring 2021}
\begin{document}

\begin{enumerate}
\litem{
What are the \textit{possible Integer} roots of the polynomial below?\[ f(x) = 3x^{3} +4 x^{2} +7 x + 4 \]\begin{enumerate}[label=\Alph*.]
\item \( \text{ All combinations of: }\frac{\pm 1,\pm 3}{\pm 1,\pm 2,\pm 4} \)
\item \( \text{ All combinations of: }\frac{\pm 1,\pm 2,\pm 4}{\pm 1,\pm 3} \)
\item \( \pm 1,\pm 2,\pm 4 \)
\item \( \pm 1,\pm 3 \)
\item \( \text{There is no formula or theorem that tells us all possible Integer roots.} \)

\end{enumerate} }
\litem{
Perform the division below. Then, find the intervals that correspond to the quotient in the form $ax^2+bx+c$ and remainder $r$.\[ \frac{15x^{3} -45 x -26}{x -2} \]\begin{enumerate}[label=\Alph*.]
\item \( a \in [30, 34], b \in [-63, -55], c \in [73, 80], \text{ and } r \in [-181, -171]. \)
\item \( a \in [15, 18], b \in [-34, -25], c \in [13, 20], \text{ and } r \in [-59, -52]. \)
\item \( a \in [30, 34], b \in [59, 61], c \in [73, 80], \text{ and } r \in [122, 126]. \)
\item \( a \in [15, 18], b \in [15, 18], c \in [-34, -25], \text{ and } r \in [-59, -52]. \)
\item \( a \in [15, 18], b \in [27, 34], c \in [13, 20], \text{ and } r \in [3, 8]. \)

\end{enumerate} }
\litem{
What are the \textit{possible Integer} roots of the polynomial below?\[ f(x) = 7x^{3} +5 x^{2} +2 x + 3 \]\begin{enumerate}[label=\Alph*.]
\item \( \text{ All combinations of: }\frac{\pm 1,\pm 7}{\pm 1,\pm 3} \)
\item \( \pm 1,\pm 7 \)
\item \( \text{ All combinations of: }\frac{\pm 1,\pm 3}{\pm 1,\pm 7} \)
\item \( \pm 1,\pm 3 \)
\item \( \text{There is no formula or theorem that tells us all possible Integer roots.} \)

\end{enumerate} }
\litem{
Factor the polynomial below completely. Then, choose the intervals the zeros of the polynomial belong to, where $z_1 \leq z_2 \leq z_3$. \textit{To make the problem easier, all zeros are between -5 and 5.}\[ f(x) = 8x^{3} -34 x^{2} -7 x + 60 \]\begin{enumerate}[label=\Alph*.]
\item \( z_1 \in [-1.74, -1.13], \text{   }  z_2 \in [1.4, 1.5], \text{   and   } z_3 \in [3, 4.2] \)
\item \( z_1 \in [-4.37, -3.36], \text{   }  z_2 \in [-1.56, -1.22], \text{   and   } z_3 \in [0.9, 1.8] \)
\item \( z_1 \in [-0.93, -0.13], \text{   }  z_2 \in [0.56, 0.69], \text{   and   } z_3 \in [3, 4.2] \)
\item \( z_1 \in [-4.37, -3.36], \text{   }  z_2 \in [-0.42, -0.3], \text{   and   } z_3 \in [4.7, 5.5] \)
\item \( z_1 \in [-4.37, -3.36], \text{   }  z_2 \in [-0.72, -0.66], \text{   and   } z_3 \in [0.1, 1] \)

\end{enumerate} }
\litem{
Factor the polynomial below completely, knowing that $x+4$ is a factor. Then, choose the intervals the zeros of the polynomial belong to, where $z_1 \leq z_2 \leq z_3 \leq z_4$. \textit{To make the problem easier, all zeros are between -5 and 5.}\[ f(x) = 12x^{4} +67 x^{3} +41 x^{2} -190 x -200 \]\begin{enumerate}[label=\Alph*.]
\item \( z_1 \in [-4.22, -3.73], \text{   }  z_2 \in [-2.07, -1.91], z_3 \in [-1.72, -0.91], \text{   and   } z_4 \in [1.16, 2.51] \)
\item \( z_1 \in [-0.74, -0.45], \text{   }  z_2 \in [0.71, 1.07], z_3 \in [1.61, 2.07], \text{   and   } z_4 \in [3.96, 4.1] \)
\item \( z_1 \in [-5.82, -4.92], \text{   }  z_2 \in [0.04, 0.65], z_3 \in [1.61, 2.07], \text{   and   } z_4 \in [3.96, 4.1] \)
\item \( z_1 \in [-1.69, -0.82], \text{   }  z_2 \in [1.14, 1.41], z_3 \in [1.61, 2.07], \text{   and   } z_4 \in [3.96, 4.1] \)
\item \( z_1 \in [-4.22, -3.73], \text{   }  z_2 \in [-2.07, -1.91], z_3 \in [-0.81, -0.68], \text{   and   } z_4 \in [0.26, 0.74] \)

\end{enumerate} }
\litem{
Factor the polynomial below completely. Then, choose the intervals the zeros of the polynomial belong to, where $z_1 \leq z_2 \leq z_3$. \textit{To make the problem easier, all zeros are between -5 and 5.}\[ f(x) = 6x^{3} -1 x^{2} -75 x + 100 \]\begin{enumerate}[label=\Alph*.]
\item \( z_1 \in [-4.3, -3.8], \text{   }  z_2 \in [1.3, 2.11], \text{   and   } z_3 \in [2.3, 3] \)
\item \( z_1 \in [-3.6, -2.4], \text{   }  z_2 \in [-1.76, -1.57], \text{   and   } z_3 \in [3.8, 5] \)
\item \( z_1 \in [-4.3, -3.8], \text{   }  z_2 \in [0.07, 0.46], \text{   and   } z_3 \in [-0.7, 0.7] \)
\item \( z_1 \in [-5.6, -4.6], \text{   }  z_2 \in [-1.14, -0.66], \text{   and   } z_3 \in [3.8, 5] \)
\item \( z_1 \in [-1.8, 0.9], \text{   }  z_2 \in [-0.51, 0.22], \text{   and   } z_3 \in [3.8, 5] \)

\end{enumerate} }
\litem{
Factor the polynomial below completely, knowing that $x-2$ is a factor. Then, choose the intervals the zeros of the polynomial belong to, where $z_1 \leq z_2 \leq z_3 \leq z_4$. \textit{To make the problem easier, all zeros are between -5 and 5.}\[ f(x) = 4x^{4} +4 x^{3} -55 x^{2} +2 x + 120 \]\begin{enumerate}[label=\Alph*.]
\item \( z_1 \in [-4.48, -3.24], \text{   }  z_2 \in [-0.91, -0.56], z_3 \in [0.15, 0.41], \text{   and   } z_4 \in [1.93, 2.35] \)
\item \( z_1 \in [-2.27, -1.92], \text{   }  z_2 \in [-1.37, -1.04], z_3 \in [2.75, 3.28], \text{   and   } z_4 \in [3.83, 4.13] \)
\item \( z_1 \in [-2.27, -1.92], \text{   }  z_2 \in [-0.46, -0.35], z_3 \in [0.62, 0.83], \text{   and   } z_4 \in [3.83, 4.13] \)
\item \( z_1 \in [-4.48, -3.24], \text{   }  z_2 \in [-1.55, -1.47], z_3 \in [1.97, 2.59], \text{   and   } z_4 \in [2.48, 2.97] \)
\item \( z_1 \in [-3.43, -2.27], \text{   }  z_2 \in [-2.05, -1.67], z_3 \in [0.87, 1.55], \text{   and   } z_4 \in [3.83, 4.13] \)

\end{enumerate} }
\litem{
Perform the division below. Then, find the intervals that correspond to the quotient in the form $ax^2+bx+c$ and remainder $r$.\[ \frac{8x^{3} +4 x^{2} -48 x + 40}{x + 3} \]\begin{enumerate}[label=\Alph*.]
\item \( a \in [-26, -23], \text{   } b \in [75, 82], \text{   } c \in [-283, -275], \text{   and   } r \in [865, 869]. \)
\item \( a \in [7, 9], \text{   } b \in [21, 32], \text{   } c \in [31, 41], \text{   and   } r \in [142, 150]. \)
\item \( a \in [-26, -23], \text{   } b \in [-70, -64], \text{   } c \in [-253, -250], \text{   and   } r \in [-719, -715]. \)
\item \( a \in [7, 9], \text{   } b \in [-28, -26], \text{   } c \in [59, 66], \text{   and   } r \in [-222, -210]. \)
\item \( a \in [7, 9], \text{   } b \in [-23, -18], \text{   } c \in [11, 15], \text{   and   } r \in [1, 10]. \)

\end{enumerate} }
\litem{
Perform the division below. Then, find the intervals that correspond to the quotient in the form $ax^2+bx+c$ and remainder $r$.\[ \frac{20x^{3} -60 x + 38}{x + 2} \]\begin{enumerate}[label=\Alph*.]
\item \( a \in [-41, -37], b \in [77, 84], c \in [-223, -218], \text{ and } r \in [477, 481]. \)
\item \( a \in [-41, -37], b \in [-87, -76], c \in [-223, -218], \text{ and } r \in [-404, -399]. \)
\item \( a \in [20, 25], b \in [-65, -59], c \in [117, 124], \text{ and } r \in [-327, -320]. \)
\item \( a \in [20, 25], b \in [-44, -36], c \in [19, 23], \text{ and } r \in [-5, 1]. \)
\item \( a \in [20, 25], b \in [34, 43], c \in [19, 23], \text{ and } r \in [71, 84]. \)

\end{enumerate} }
\litem{
Perform the division below. Then, find the intervals that correspond to the quotient in the form $ax^2+bx+c$ and remainder $r$.\[ \frac{8x^{3} +6 x^{2} -89 x + 63}{x + 4} \]\begin{enumerate}[label=\Alph*.]
\item \( a \in [7, 12], \text{   } b \in [-39, -27], \text{   } c \in [78, 83], \text{   and   } r \in [-343, -339]. \)
\item \( a \in [-33, -25], \text{   } b \in [125, 135], \text{   } c \in [-628, -621], \text{   and   } r \in [2560, 2564]. \)
\item \( a \in [7, 12], \text{   } b \in [35, 46], \text{   } c \in [63, 69], \text{   and   } r \in [311, 316]. \)
\item \( a \in [7, 12], \text{   } b \in [-28, -24], \text{   } c \in [14, 17], \text{   and   } r \in [-1, 4]. \)
\item \( a \in [-33, -25], \text{   } b \in [-125, -118], \text{   } c \in [-579, -573], \text{   and   } r \in [-2247, -2242]. \)

\end{enumerate} }
\end{enumerate}

\end{document}